\section{Formule di Stokes}\label{sezione:FormuleStokes}

Durante tutta la sezione assumeremo sempre che $\Omega\subseteq\R^n$ sia un aperto.

\begin{theorem}\label{thm:PtRegEquiv}
	Per un punto $x\in \partial \Omega$ le seguenti $3$ condizioni sono equivalenti:
	\begin{enumerate}
		\item esistono $U$ intorno di $x$ e $f\in C^1(U,\R)$ tale che si abbia
			\begin{itemize}
				\item $U\cap \partial \Omega=f^{-1}(0)$,
				\item $U\cap \Omega=f^{-1}(]-\infty,0[)$,
			\end{itemize}
			e tale che $Df(x)\neq 0$ se $f(x)=0$;\label{PRE:i}
		\item esistono $U$ intorno di $x$ e $g$ diffeomorfismo da $\R^n$ in $U$ tale che\label{PRE:ii}
			\begin{itemize}
				\item $U\cap \partial \Omega = g(\R^{n-1}\times\{0\})$,
				\item $U\cap \Omega = g(\R^{n-1}\times]-\infty,0[)$;
			\end{itemize}
		\item esistono, a meno di riordinare le coordinate, un intorno $U=V\times I$ di $x=(y,t)$ tale che $V\in\R^{n-1}$ è intorno di $y$
			e $I\in\R$ è intorno di $t$ e una funzione $\psi\in C^1(V,I)$ tale che
			\begin{itemize}
				\item $U\cap \partial \Omega = \{(z,s):\psi(z)=s\}$,
				\item $U\cap \Omega = \{(z,s):\psi(z)>s\}$.
			\end{itemize}\label{PRE:iii}
	\end{enumerate}
\end{theorem}

\begin{proof}
	Dimostriamo la catena in ordine inverso:
	\begin{description}
		\item [\ImplicationProof{PRE:i}{PRE:iii}] Dato che $Df(x)\neq 0$, esiste una derivata parziale non nulla, supponiamo senza perdita di
			generalità che sia l'ultima e che sia positiva (se negativa il ragionamento è lo stesso).
			Allora le ipotesi del teorema della funzione implicita del Dini sono verificate, per cui otteniamo
			che esiste un intorno $V\times I\subseteq U$ di $x$ per cui esiste $\psi\in C^1(V,I)$ tale che $f(v,\psi(v))=0$ e $\psi(v)$
			è l'unico punto $t'\in I$ tale che $f(v,t')=0$.
			Inoltre, per la continuità di $Df(x)$, esiste un intorno $V'\subseteq V$ di $v$ tale che $Df(v')_n>\frac{1}{2}Df(x)_n$ per tutti
			i $v'\in V'$. Ora, per la differenziabilità di $f$ in $V'$ si ha che
			\[
				f(v',\psi(v')+\delta)=f(v', \psi(v'))+Df(v')_n\delta+\smallO(\delta)=Df(v')\delta+\smallO(\delta),
			\]
			che, per $\delta>0$ e abbastanza piccolo, dà $f(v',\psi(v')+\delta)\geq \frac{1}{4}Df(x)_n\delta>0$. Infine, non può esistere un $\delta>0$ tale che $\psi(v')+\delta\in I$ e
			$f(v',\psi(v')+\delta)<0$, altrimenti per la continuità di $f$ si avrebbe che esiste un punto $\psi(v')+\delta'\neq \psi(v')$
			che sia zero per la funzione $f(v',\cdot)$, contraddicendo il teorema della funzione implicita del Dini.
			Per $\delta<0$ si ha lo stesso risultato. Questo ci dice che $f^{-1}(0)\cap (V'\times I)=\{(v',s):s=\psi(v')\}$ e
			$f^{-1}(]-\infty,0[)\cap (V'\times I)=\{(v',s):s<\psi(v')\}$, che conclude per le ipotesi sulle controimmagini di $f$.
		\item [\ImplicationProof{PRE:iii}{PRE:ii}] Se abbiamo $V\times I$ intorno di $x$ e $\psi$, detto $J=]-\varepsilon,\varepsilon[$
			possiamo ottenere un diffeomorfismo $g:V\times J \rightarrow V\times J$ ponendo $g(z,s)=(z,\psi(z)-s)$. Allora,
			$g(V\times \{0\})=$graf$(\psi)=U\cap \partial \Omega$ e $g(V\times ]-\varepsilon,0[)=\{(z,s):\psi(z)<s\}=U\cap \Omega$.
			Per ottenere un diffeomorfismo di tutto $\R^n$ basta comporre $g$ con opportuni diffeomorfismi tra un intorno rettangolare
			di $x$ e $\R^n$.
		\item [\ImplicationProof{PRE:ii}{PRE:i}] Per la prima condizione basta considerare$f=g^{-1}_n$ l'ultima componente di $g^{-1}$.
	\end{description}
\end{proof}


\begin{definition}[punto regolare]
	Diremo che un punto $x\in \partial \Omega$ è un punto regolare del bordo se verifica una delle $3$ condizioni di cui
	al \cref{thm:PtRegEquiv}.
\end{definition}

D'ora in poi assumeremo che $\Omega$ sia anche limitato.

\begin{definition}[aperto regolare]
	Diremo che $\Omega$ è un aperto regolare se tutti i punti del suo bordo sono regolari.
\end{definition}


\begin{theorem}[Partizione dell'unità]\label{thm:PartizioneUnita}
	Dato un ricoprimento di $\R^n$ con aperti $(\Omega_i)_{i\in I}$, esiste una famiglia di funzioni $f_i:\R^n\to\R$ infinitamente derivabili e non negative tali che:
	\begin{itemize}
		\item la chiusura dell'insieme in cui $f_i$ non si annulla è contenuta in $\Omega_i$ per ogni $i\in I$,
		\item per ogni $x\in\R^n$ vale che $\sum_{i\in I} f_i(x)=1$\footnote{La serie la definiamo come il $\sup$ delle somme finite}.
	\end{itemize}
\end{theorem}
\begin{proof}
	TODO
\end{proof}

