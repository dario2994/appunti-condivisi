\section{Formule di Stokes}\label{sezione:FormuleStokes}

Iniziamo con lo studio degli insiemi regolari che ci interessano per il teorema di Stokes.
A questo proposito, durante tutta la sezione assumeremo sempre che $\Omega\subseteq\R^n$ sia un aperto limitato.

Per quanto riguarda la notazione, data una funzione $f:\R^n\to \R$ indicheremo inoltre con $\partial_i f=\frac{\partial f}{\partial x_i}$ la derivata parziale di $f$ rispetto all'$i$-esima componente e similmente indicheremo $\partial_{ij}f=\frac{\partial^2 f}{\partial x_j\partial x_i}$. 

\begin{theorem}\label{thm:PtRegEquiv}
	Per un punto $\bar x\in \partial \Omega$ le seguenti tre condizioni sono equivalenti:
	\begin{enumerate}
		\item esistono $U$ intorno di $\bar x$ e $f\in C^1(U,\R)$ tale che si abbia
			\begin{itemize}
				\item $U\cap \partial \Omega=f^{-1}(0)$,
				\item $U\cap \Omega=f^{-1}(\oo{-\infty}{0})$,
			\end{itemize}
			e tale che $\grad f(x)\neq 0$ se $f(x)=0$;\label{PRE:i}
		\item esistono $U$ intorno di $\bar x$ e $g$ diffeomorfismo $C^1$ da $\R^n$ in $U$ tale che\label{PRE:ii}
			\begin{itemize}
				\item $U\cap \partial \Omega = g(\R^{n-1}\times\{0\})$,
				\item $U\cap \Omega = g(\R^{n-1}\times\oo{-\infty}{0})$;
			\end{itemize}
		\item esistono, a meno di riordinare le coordinate, un intorno $U=V\times I$ di $\bar x=(\bar v,\bar t)$ tale che $V\in\R^{n-1}$ è intorno di $y$
			e $I\in\R$ è intorno di $t$ e una funzione $\psi\in C^1(V,I)$ tale che
			\begin{itemize}
				\item $U\cap \partial \Omega = \{(v,t)\in V\times I: t=\psi(v)\}$,
				\item $U\cap \Omega = \{(v,t)\in V\times I:t<\psi(v)\}$.
			\end{itemize}\label{PRE:iii}
	\end{enumerate}
\end{theorem}

\begin{proof}
	Dimostriamo la catena in ordine inverso:
	\begin{description}
		\item [\ImplicationProof{PRE:i}{PRE:iii}] Dato che $\grad f(\bar x)\neq 0$, esiste una derivata parziale non nulla, supponiamo senza perdita
			di generalità che sia l'ultima e che sia positiva (se negativa il ragionamento è lo stesso).
			Allora le ipotesi del teorema della funzione implicita del Dini sono verificate, otteniamo quindi
			che esiste un intorno $V\times I\subseteq U$ di $\bar x$, con $V\in\R^{n-1}$ e $I\in\R$, per cui esiste $\psi\in C^1(V,I)$ tale che, per ogni $v\in V$, $f(v,\psi(v))=0$ e $\psi(v)$
			è l'unico punto $t'\in I$ tale che $f(v,t')=0$.
			Inoltre, per la continuità di $\grad f(x)$, esiste un intorno $V'\subseteq V$ di $\bar v$, con $\bar x=(\bar v, \psi(\bar v))$, tale che $\partial_n f(v',\psi(v'))>\frac{1}{2}
			\partial_n f(\bar x)$ per tutti i $v'\in V'$. 
			
			Ora, per la differenziabilità di $f$ in $(v',\psi(v'))$ si ha, per $\delta\in \R$,
			\[
				f(v',\psi(v')+\delta)=f(v', \psi(v'))+\partial_n f(v',\psi(v'))\delta+\smallO(\delta)=\partial_n f(v',\psi(v'))\delta+\smallO(\delta)\virgola
			\]
			che, per $\delta>0$ e abbastanza piccolo, dà
			\begin{equation*}
				f(v',\psi(v')+\delta)\geq \frac{1}{2}\partial_n f(v',\psi(v'))\delta\geq \frac{1}{4}\partial_n f(\bar x)\delta>0\punto
			\end{equation*}
			
			Infine, non può esistere un $\delta>0$ tale che $\psi(v')+\delta\in I$ e
			$f(v',\psi(v')+\delta)<0$, altrimenti per la continuità di $f$ si avrebbe che esiste un punto $\psi(v')+\delta'\neq \psi(v')$
			che sia zero per la funzione $f(v',\cdot)$, contraddicendo il teorema della funzione implicita del Dini.
			Per $\delta<0$ si ha lo stesso risultato. Questo ci dice che 
			\begin{align*}
				f^{-1}(0)\cap (V'\times I)&=\{(v',t)\in V'\times I:t=\psi(v')\} \virgola\\
				f^{-1}(\oo{-\infty}{0})\cap (V'\times I)&=\{(v',t)\in V'\times I:t<\psi(v')\}\virgola
			\end{align*}
			che conclude per le ipotesi sulle controimmagini di $f$.
		\item [\ImplicationProof{PRE:iii}{PRE:ii}] Se abbiamo $V\times I$ intorno di $\bar x$ e $\psi\in C^1(V,I)$ come nelle ipotesi, detto $J=\oo{-\varepsilon}{\varepsilon}$
			possiamo ottenere un diffeomorfismo $g:V\times I \rightarrow V\times J$ ponendo $g(v,t)=(v,\psi(v)+t)$. 
			Di conseguenza otteniamo
			\begin{align*}
				g(V\times \{0\})&=\graf(\psi)=U\cap \partial \Omega \virgola\\
				g(V\times \oo{-\varepsilon}{0})&=\{(v,\psi(v)+t)\in V\times I:t<0\}=\\
												&=\{(v,s)\in V\times I:s<\psi(v)\}=U\cap \Omega\punto
			\end{align*}
			Per ottenere un diffeomorfismo di tutto $\R^n$ basta comporre $g$ con opportuni diffeomorfismi tra un intorno rettangolare
			di $\bar x$ e $\R^n$.
		\item [\ImplicationProof{PRE:ii}{PRE:i}] È sufficiente scegliere $f=(g^{-1})_n$; infatti, $g^{-1}$ va da $U$ in $\R^n$ e, essendo un diffeomorfismo, ha differenziale invertibile in ogni punto.
			In particolare, ogni riga della matrice del differenziale non è nulla (quindi neanche $\nabla f$) e le condizioni sulla $g$ si traducono immediatamente in quelle chieste alla $f$:
			\[
				U\cap \partial \Omega = g(\R^{n-1}\times\{0\}) \Rightarrow g^{-1}(U\cap \partial \Omega) = \R^{n-1}\times\{0\}\virgola
			\]
			cioè $(g^{-1})_n(\R^{n-1}\times\{0\}) = \{0\}$. E lo stesso per $U\cap\Omega$.
	\end{description}
\end{proof}
\begin{lemma}\label{lemma:BuonaDefinizioneNormale}
	Fissato $\bar x\in\partial\Omega$, se $(U_1,f_1)$ e $(U_2,f_2)$ sono due coppie di aperto e funzione che rispettano le richieste enunciate in \ref{PRE:i} allora i vettori $\grad f_1(\bar x)$ e $\grad f_2(\bar x)$ sono paralleli ed hanno lo stesso verso.
\end{lemma}
\begin{proof}
	Possiamo assumere $U_1=U_2=U$ visto che se così non fosse basterebbe sostituirli con la loro intersezione. Fissiamo un generico vettore $v\in\R^n$.
	
	Per ogni reale positivo $\lambda>0$ sufficientemente piccolo risulta $x+\lambda v\in U$ e perciò $f_1(x+\lambda v)$ e $f_2(x+\lambda v)$ hanno lo stesso segno, poiché tale segno dipende unicamente dall'appartenenza o meno di $x+\lambda v$ ad $\Omega$. 
	
	Sempre scegliendo $\lambda$ tale che $x+\lambda v\in U$, per definizione di gradiente abbiamo
	\begin{equation*}
		f_1(\bar x+\lambda v)=f_1(\bar x)+\grad f_1(\bar x)\cdot \lambda v+\smallO(\lambda)=\lambda \grad f_1(\bar x)\cdot v+\smallO(\lambda)
	\end{equation*}
	e allora, supponendo $\lambda$ sufficientemente piccolo e $\grad f_1(\bar x)\cdot v\not =0$, il segno di $f_1(\bar x+\lambda v)$ è lo stesso di $\grad f_1(\bar x)\cdot v$.
	Ovviamente un risultato analogo vale anche per $f_2$.
	
	Unendo quanto detto, si ha che $\grad f_1(\bar x)\cdot v$ ha lo stesso segno di $\grad f_2(\bar x)\cdot v$, se sono entrambi non nulli, per ogni $v\in\R^n$. 
	Se per assurdo i gradienti non fossero paralleli, ponendo $v=\grad f_1(\bar x)+\lambda h$, con $h$ ortogonale a $\grad f_1(\bar x)$ ma \emph{non} a $\grad f_2(\bar x)$, la condizione sui prodotti scalari non può essere verificata per ogni $\lambda\in\R$; quindi i gradienti sono paralleli. Per quanto riguardo il verso, basta scegliere $v=\grad f_1(\bar x)$ per ottenere che hanno lo stesso verso.
\end{proof}


\begin{definition}[Punto regolare]
	Diremo che un punto $x\in \partial \Omega$ è un punto regolare del bordo se verifica una delle tre condizioni di cui
	al \cref{thm:PtRegEquiv}.
\end{definition}

Abbiamo quindi definito il tipo di regolarità che serve ai nostri scopi. Ma questa, ancora, è una nozione locale. Diamo allora la definizione globale.

\begin{definition}[Aperto regolare]
	Diremo che $\Omega$ è un aperto regolare se tutti i punti del suo bordo sono regolari.
\end{definition}

Questi insiemi saranno di cruciale importanza perché i principali risultati di questa sezione riguardano tutti insiemi di questo tipo.

Facciamo ora una piccola parentesi per introdurre uno strumento molto utile per estendere funzioni definite solo localmente a funzioni in tutto lo spazio. Tale teorema sarà utilizzato in più punti della sezione.

\begin{theorem}[Partizione dell'unità]\label{thm:PartizioneUnita}
	Dato un ricoprimento di $\R^n$ con aperti $(\Omega_i)_{i\in I}$, esiste una famiglia di funzioni $f_i:\R^n\to\Rpiu$ infinitamente derivabili
	e non negative tali che:
	\begin{itemize}
		\item La chiusura dell'insieme in cui $f_i$ non si annulla è contenuta in $\Omega_i$ per ogni $i\in I$;
		\item Per ogni $x\in\R^n$, solo per un numero finito di $i\in I$ la quantità $f_i(x)$ non si annulla;
		\item Per ogni $x\in\R^n$, vale $\sum_{i\in I} f_i(x)=1$.
	\end{itemize}
\end{theorem}
\begin{proof}
	Non proponiamo una dimostrazione completa di questo teorema poiché risulterebbe eccessivamente lunga rispetto all'inerenza che ha con la teoria della misura. Il teorema qui enunciato è infatti dal carattere fortemente topologico ed è la sua applicazione ad avere grande valore in teoria della misura, piuttosto che la sua dimostrazione.
	
	In ogni caso, per completezza, proponiamo al lettore interessato una via dimostrativa guidata, da considerare come un esercizio.
	Ci saranno utili i seguenti due lemmi, topologicamente interessanti anche di per sè.
	\begin{enumerate}
		\item Dati due aperti $A,B\subseteq \R^n$ tali che $A\cup B=\R^n$, esiste un aperto $C$ tale che $\overline{C}\subseteq A$ e $C\cup B=\R^n$.
		\item Dato un chiuso $C\subseteq \R^n$, esiste una funzione $f:\R^n\to\R$ derivabile infinite volte che come supporto ha esattamente $C$.
	\end{enumerate}
	
	Il primo passo della dimostrazione della partizione dell'unità è notare che se si dimostra la tesi per un sottoricoprimento, anche ammettendo di poter ``rimpicciolire gli aperti'', la tesi generale segue facilmente. 
	Allora, ricordando la \cref{prop:BaseNumerabileLindelof}, si può assumere che $I$ sia numerabile e, a meno di rimpicciolire adeguatamente gli aperti $\Omega_i$, si può assumere che ogni compatto intersechi solo un numero finito di aperti del ricoprimento.
	
	A questo punto applicando i due lemmi enunciati, è facile trovare delle funzioni $(f_i)_{i\in I}$ che rispettano tutte le richieste tranne che, al posto di valere $\sum_{i\in I} f_i(x)=1$, abbiamo soltanto che vale $\sum_{i\in I} f_i(x)>0$.
	Per concludere basta dimostrare che $\sum_{i\in I}f_i$ è infinitamente derivabile e dividere ogni $f_i$ per tale funzione.
\end{proof}

Mostriamo ora, facendo uso di questo teorema, che è possibile dare una definizione di aperto regolare che coinvolga una sola funzione definita globalmente.

\begin{lemma}\label{lemma:EquivRegolare}
	Abbiamo che $\Omega$ è un aperto regolare se e solo se esiste una funzione $\phi\in C^1(\R^n,\R)$ tale che $\grad \phi(x)\neq 0$ se $\phi(x)=0$ e tale che
	\begin{itemize}
		\item $\Omega = \phi^{-1}(\oo{-\infty}{0})$,
		\item $\partial \Omega = \phi^{-1}(0)$.
	\end{itemize}
\end{lemma}

\begin{proof}
	L'implicazione del ``se'' è banale per la caratterizzazione del punto \ref{PRE:i}, dimostriamo quindi l'altra implicazione. 
	
	Per la regolarità dei punti
	del bordo di $\Omega$, consideriamo per ogni $x\in\partial \Omega$ un intorno $U_x$ tale che esista una funzione $\psi_x:U_x\rightarrow \R$
	come nel punto \ref{PRE:iii} del \cref{thm:PtRegEquiv}. Dato che $\Omega$ è limitato, $\partial\Omega$ è compatto e gli $U_x$ sono un ricoprimento aperto di $\partial \Omega$; possiamo
	quindi estrarre un sottoricoprimento finito $U_1, \dots, U_n$ di $\partial\Omega$, dove per comodità di notazione abbiamo sostituito $U_{x_i}$ con $U_i$ ed analogamente $f_{x_i}$ con $f_i$. 
	
	Chiamando $U_0 = \R^n \setminus \partial \Omega$,
	abbiamo perciò che gli aperti $U_0,\dots,U_n$ sono un ricoprimento di $\R^n$ e di conseguenza
	il \cref{thm:PartizioneUnita} ci fornisce delle funzioni $f_i$ per ogni $i\in\{0,\dots, n\}$ con tutte le proprietà descritte.
	Infine, definiamo $\psi_0:U_0\rightarrow \R$ tale che $\psi_0(x)=1$ se $x\notin \Omega$, $\psi_0(x)=-1$ altrimenti (per ognuno degli altri $U_i$ abbiamo già delle funzioni $\psi_i$). Sia ora 
	\[
		\phi(x)=\sum_{i=0}^n f_i(x) \psi_i(x)\virgola
	\]
	dove, con una leggera imprecisione formale, poniamo $f_i(x) \psi_i(x) = 0$ se $x\notin U_i$, in modo che $f_i(x) \psi_i(x)$ sia una funzione
	$C^1$ in tutto $\R^n$. Infatti, dove sono definite entrambe (cioè in $U_i$), $f_i(x)\psi_i(x)$ è $C^1$, perché $\psi_i$ è $C^1$ e $f_i$ è addirittura $C^{\infty}$; inoltre la chiusura dell'insieme in cui $f_i$ non si annulla è contenuta in $U_i$ e perciò non ci sono problemi nel passaggio all'esterno di $U_i$, dove abbiamo posto il prodotto uguale a 0.
	
	Ora abbiamo che $\phi\in C^1$ perché somma di funzioni $C^1$; $\phi(x)=0$ per ogni $x\in \partial\Omega$, dato che ognuna delle
	$\psi_i$ aveva questa proprietà se era definita in $x$;
	$\phi(x)<0$ per ogni $x \in \Omega$, perché $\phi$ è combinazione convessa di funzioni che, per ciascun $x\in\Omega$, sono non positive e
	delle quali almeno una strettamente negativa; similmente, $\phi(x)>0$ per ogni $x \notin \overline\Omega$.
	
	Rimane da mostrare che $\grad \phi(x)\neq 0$ per ogni $x$ tale che $\phi(x)=0$. 
	Per quanto già detto, se $\phi(x)=0$ allora $x\in\partial\Omega$ e perciò, scelto un tale $x$, in un suo intorno $f_0=0$. 
	Quindi, ricordando la definizione di $\phi$, è facile ottenere (ancora abusando lievemente di notazione)
	\begin{equation*}
		\grad\phi(x)=\sum_{i=1}^n\grad f_i(x)\psi_i(x)+\sum_{i=1}^n f_i(x)\grad\psi_i(x)=\sum_{i=1}^n f_i(x)\grad\psi_i(x)\virgola
	\end{equation*}
	cioè $\grad\phi(x)$ è una combinazione convessa dei $\grad\psi_i(x)$ (se definiti), che sono paralleli e concordi per il \cref{lemma:BuonaDefinizioneNormale}. 
	Perciò $\grad\phi(x)$ non si annulla poiché è una combinazione convessa di vettori paralleli e concordi, di cui almeno uno non nullo.
\end{proof}

La conseguenza più importante dell'esistenza di un'unica funzione $\phi$ che discrimina punti interni ed esterni ad $\Omega$ è certamente la seguente.

\begin{definition}[Campo normale esterno]
	Fissato $\Omega\subseteq\R^n$ aperto regolare, sia $\phi$ come nel \cref{lemma:EquivRegolare}. Chiamiamo campo normale esterno la funzione $\nu:\partial\Omega\rightarrow \R^n$ tale che
	\[
		\nu(x)=\frac{\grad \phi(x)}{\lVert\grad \phi(x)\rVert}\punto
	\]
\end{definition}
\begin{remark}
	Il \cref{lemma:BuonaDefinizioneNormale}, assicura che $\nu(x)$ è ben definita.
\end{remark}

\begin{remark}
	Si ha che per $t>0$ e $x\in\partial\Omega$, $x+t\nu(x)\notin \Omega$ per $t$ abbastanza piccoli e per $t<0$, $x+t\nu(x)\in \Omega$ per $t$
	abbastanza piccoli.
\end{remark}
\begin{proof}
	Stesso ragionamento fatto nella dimostrazione del \cref{lemma:BuonaDefinizioneNormale}.
\end{proof}

Forti di una solida definizione di campo normale uscente possiamo procedere con il formalizzare la definizione di integrale sulla superficie $\partial\Omega$ sfruttando anche i risultati della sezione precedente.

\begin{lemma}\label{lemma:GrapVar}
	Sia $V\in\R^{n-1}$ aperto e $f\in C^1(V,\R)$, allora il grafico di $f$ è una varietà differenziabile di dimensione $n-1$ immersa in $\R^n$.
\end{lemma}
\begin{proof}
	Sia $g:V\rightarrow \R^n$ data da $g(x)=(x,f(x))$. Ora è chiaro che $\graf(f)$ è l'immagine di $g$; inoltre $g$ è banalmente di classe $C^1$
	e, se vista come funzione da $V$ in $\graf(f)$, è anche un omeomorfismo. Infine $\Diff g(x)[h]=(h,\Diff f(x)[h])$, perciò in particolare il differenziale di $g$ è
	iniettivo in ogni punto. Quindi $g$ è una immersione iniettiva, cioè è una parametrizzazione $C^1$ di $\graf(f)$.
\end{proof}

\begin{remark}
	Dalla dimostrazione di questo lemma, dato che $\grad f(x)=\Diff f(x)$, segue immediatamente che $\Diff g(x)^T \Diff g(x)=
	(I \ \ \grad f(x)^T)
	\left(\begin{smallmatrix}
	I \\
	\grad f(x)
	\end{smallmatrix}\right)$, quindi per il \cref{lemma:DeterminanteMatriceQuasiIdentita} vale che $\det (\Diff g(x)^T \Diff g(x)) = 1+\lVert\grad f(x)\rVert^2$. 
\end{remark}

Il \cref{lemma:GrapVar} ci permette quindi di costruire un misura superficiale su $\graf(f)$ in accordo con la \cref{def:MisuraKDimensionale}:
\begin{equation}\label{eq:MisuraSuperficialeGrafico}
	\sigma(E)=\int_{g^{-1}(E)} \sqrt{1+\lVert\grad f(x)\rVert^2} \de x\punto
\end{equation}


Ora che sappiamo cos'è la normale uscente ad un aperto regolare e sappiamo come intendere l'integrale sulla sua superficie,
abbiamo ottenuto tutti gli strumenti che ci servono per formalizzare il principale risultato della sezione. Possiamo quindi enunciare il teorema di Stokes.

\begin{theorem}[Formula di Stokes]\label{thm:Stokes}
	Sia $\Omega$ un aperto regolare e siano $u\in C^1(\overline\Omega,\R)$ (ossia $u$ è la restrizione a $\overline\Omega$ di una funzione $C^1$ definita su un aperto
	$\Omega'\supseteq \overline\Omega$) e $\nu:\partial \Omega \rightarrow \R^n$ la normale uscente da $\Omega$, allora vale che
	\begin{equation}\label{eq:FormulaStokes}
		\int_{\Omega} \partial_j u(x) \de x = \int_{\partial \Omega} u(y)\nu(y)_j \de \sigma(y)\punto
	\end{equation}
\end{theorem}

\begin{proof}
	Per la dimostrazione procederà in due step:
	\begin{enumerate}
		\item per ogni $p\in \overline\Omega$ esiste un intorno aperto rettangolare $U_p=\prod_{i=1}^n \oo{p_i-\varepsilon_i}{p_i+\varepsilon_i}$
			tale che l'\cref{eq:FormulaStokes} vale per ogni funzione $u$ tale che $\supp(u)\subseteq U_p$;
		\item il compatto $\overline\Omega$ si può ricoprire con un numero finito di intorni come sopra, quindi si applica il \cref{thm:PartizioneUnita} per ottenere
			la tesi per ogni $u$.
	\end{enumerate}
	
	Dimostriamo quindi il primo punto, distinguendo i casi in cui $p\in \Omega$ e $p\in \partial \Omega$.
	Se $p\in \Omega$ esiste un intorno rettangolare $U_p=V\times \oo{a}{b}$ di $p$ (con $V\subseteq \R^{n-1}$) contenuto in $\Omega$, poiché quest'ultimo è un aperto. Se ora $\supp(u)\subseteq U_p$,
	allora $u(\partial \Omega)=0$; quindi abbiamo che l'integrale a destra è $0$. Per l'integrale a sinistra, invece, abbiamo che una funzione $u$
	con supporto contenuto in $V\times \oo{a}{b}$ è tale che $u(x)=0$ per ogni
	$x\in \partial (V\times \oo{a}{b})$. Allora, per il \cref{thm:Fubini} e il teorema fondamentale del calcolo integrale, vale che
	\begin{align*}
		\int_{\Omega} \partial_n u(x) \de x &= \int_{V\times \oo{a}{b}} \partial_n u(v,t) \de (v,t) = \\
		& = \int_V \int_a^b \frac{\partial}{\partial t} u(v,t) \de t \de v = \int_V (u(v,b) - u(v,a)) \de v = 0\punto
	\end{align*}
	Quindi abbiamo la tesi del primo punto quando $p\in\Omega$. Se invece $p\in \partial \Omega$, allora $p$ è un punto regolare, cioè esiste
	un intorno $U_p=V\times I$ di $p$ e una $\psi\in C^1(V,I)$ tale che
	\begin{itemize}
		\item $U_p\cap \partial \Omega = \{(v,t)\in V\times I:t=\psi(v)\}$,
		\item $U_p\cap \Omega = \{(v,t)\in V\times I:t<\psi(v)\}$.
	\end{itemize}
	Distinguiamo ora i casi $j=n$ e $j\neq n$.
	\begin{description}
		\item [$j=n$:] Dato che stiamo supponendo che $\supp(u)\subseteq U_p$, per il \cref{thm:Fubini} abbiamo che
			\begin{align*}
				\int_{\Omega} \partial_n u(x) \de x  &= \int_{\Omega\cap U_p} \partial_n u(v,t)\de (v,t) =
				\int_{U_p} \partial_n u(v,t) \chi_{\{t<\psi(v)\}} \de (v,t) = \\
				& = \int_V \int_I \partial_n u(v,t) \chi_{\{t<\psi(v)\}} \de t \de v =
				\int_V \int_{-\infty}^{\psi(v)} \frac{\partial}{\partial t} u(v,t) \de t \de v = \\
				& = \int_V (u(v,\psi(v)) - 0) \de v = \int_V u(v,\psi(v)) \de v\punto
			\end{align*}
			
			Ora abbiamo che la funzione $g:V\rightarrow \R^n$ data da $g(v)=(v,\psi(v))$ parametrizza l'ipersuperfice data dal grafico di
			$\psi$, quindi la misura superficiale è data dall' \cref{eq:MisuraSuperficialeGrafico}.
			Inoltre, detta $\nu:\partial \Omega \cap U_p \rightarrow \R^n$ la normale uscente da $\Omega$, sappiamo che
			\begin{equation*}
				\nu(y)=\frac{\grad f(y)}{\lVert \grad f(y) \rVert}\virgola
			\end{equation*}
			dove $f:V\times I\to\R$ è tale che $f(v,t)=t-\psi(v)$. Perciò utilizzando che $\grad f=(\grad \psi, 1)$, otteniamo facilmente che $\nu(v,\psi(v))_n=(1+\lVert\grad\psi(v)\rVert^2)^{-\frac{1}{2}}$.
			
			Quindi per definizione di integrale superficiale abbiamo che
			\begin{align*}
				\int_V u(v,\psi(v)) \de v &= \int_V u(v,\psi(v)) (1+\lVert\grad\psi(v)\rVert^2)^{-\frac{1}{2}} (1+\lVert\grad\psi(v)\rVert^2)^{\frac{1}{2}} \de v = \\
				& = \int_V u(v,\psi(v)) \nu(v,\psi(v))_n (1+\lVert\grad\psi(v)\rVert^2)^{\frac{1}{2}} \de v = \\
				& = \int_{\partial \Omega \cap U_p} u(x) \nu(x)_n \de \sigma(x) = \\
				& = \int_{\partial \Omega} u(x) \nu(x)_n \de \sigma(x)\virgola
			\end{align*}
			dove nell'ultimo passaggio si è usato che $\supp(u)\subseteq U_p$.
		\item [$j\neq n$:] Possiamo assumere (a meno di riordinare le coordinate) che $j=n-1$ e, a meno di considerare un intorno più piccolo,
			che $V=W \times J$, con $W\subseteq \R^{n-1}$ limitato e $J=\oo{a}{b}\subseteq \R$. Ora, per il \cref{thm:Fubini}, abbiamo che
			\[
				\int_{\Omega} \partial_{n-1} u(x) \de x = \int_W \int_J \int_I \frac{\partial}{\partial s} u(w,s,t) \de t \de s \de w=\int_W \int_J \int_{-\infty}^{\psi(w,s)} \frac{\partial}{\partial s} u(w,s,t) \de t \de s \de w\punto
			\]
			
			Dato che $u$ è una funzione $C^1$ definita su un compatto che contiene $W\times J\times I$, rispetta facilmente le ipotesi\footnote{L'unica ipotesi non ovvia è che la derivata parziale sia maggiorata da una funzione integrabile. Ma ciò è vero poiché la derivata parziale, essendo continua su un compatto, è maggiorata da una costante.} del \cref{thm:CommutazioneDerivataIntegrale}, quindi sfruttando che
			\begin{multline*}
				 \int_{-\infty}^{\psi(w,s+\de s)} u(w,s+\de s,t) \de t- \int_{-\infty}^{\psi(w,s)} u(w,s,t) \de t\\ =\int_{-\infty}^{\psi(w,s)} u(w,s+\de s,t)- u(w,s,t) \de t+  \int_{\psi(w,s)}^{\psi(w,s+\de s)} u(w,s+\de s,t) \de t\virgola
			\end{multline*}
			dividendo per $\de s$ e mandando al limite $\de s\to 0$, otteniamo che, per quanto detto sulla possibilità di commutare derivata e integrale, vale
			\begin{equation*}
				\frac{\partial}{\partial s} \int_{-\infty}^{\psi(w,s)} u(w,s,t) \de t =
				\int_{-\infty}^{\psi(w,s)} \frac\partial{\partial s} u(w,s,t) \de t + u(w,s,\psi(w,s))\frac\partial{\partial s} \psi (w,s)\punto
			\end{equation*}
			
			Per cui, integrando il termine a sinistra su $V=W\times J=W\times \oo{a}{b}$, per il teorema fondamentale del
			calcolo integrale, abbiamo 
			\begin{align*}
				&\int_W \int_a^b \frac\partial{\partial s} \int_{-\infty}^{\psi(w,s)} u(w,s,t) \de t \de s \de w = \\
				&\int_W \left(\int_{-\infty}^{\psi(w,b)} u(w,b,t) \de t - \int_{-\infty}^{\psi(w,a)} u(w,a,t) \de t\right) \de w
				= 0\virgola
			\end{align*}
			poiché i punti $(w,b,t)$ e $(w,a,t)$ sono sul bordo di $U_p$, dove la funzione $u$ è nulla per ipotesi. 
			
			Usando quindi le due relazioni appena trovate otteniamo che
			\[
				\int_W\int_J\int_I\frac\partial{\partial s} u(w,s,t) \de t \de s \de w =-\int_W \int_J u(w,s,\psi(w,s))\frac\partial{\partial s} \psi (w,s)\de s\de w\punto
			\]
			Ora, dato che $\nu(x)_{n-1}=\nu(v,\psi(v))_{n-1}=-\partial_{n-1}\psi(v)(1+\lVert\grad\psi(v)\rVert^2)^{-\frac{1}{2}}$, otteniamo
			\begin{align*}
				\int_{\Omega} \partial_{n-1} u(x) \de x & = -\int_W \int_J u(w,s,\psi(w,s))\frac\partial{\partial s} \psi (w,s) \de s \de w\\
				& = -\int_V u(v,\psi(v))\partial_{n-1} \psi(v) \de v\\
				& = -\int_V u(v,\psi(v))(-\nu(v,\psi(v))_{n-1})(1+\lVert\grad\psi(v)\rVert^2)^{\frac{1}{2}} \de v\\
				& = \int_{\partial \Omega \cap U} u(x)\nu(x)_{n-1} \de \sigma(x)\punto
			\end{align*}
	\end{description}
	Questo conclude il primo punto.
	
	Per la seconda parte procediamo in questo modo.
	Consideriamo, per ogni punto $p\in \overline\Omega$, un intorno $U_p$ della forma richiesta al punto precedente. È chiaro che $\{U_p\}_{p\in\overline\Omega}$ è un
	ricoprimento aperto del compatto $\overline\Omega$, quindi possiamo estrarne un sottoricoprimento finito $U_1,\dots,U_n$, a cui aggiungiamo anche
	$U_0=\R^n \setminus \overline\Omega$. Il \cref{thm:PartizioneUnita} ci fornisce delle funzioni $f_i\in C^{\infty}$, per $0\leq i \leq n$,
	tali che la loro somma è la funzione costante $1$.
	Sia ora $u\in C^1(\R^n,\R)$ una qualsiasi funzione. Otteniamo che
	\[
		u = \sum_{i=0}^n f_i\cdot u\punto
	\]
	Ma quindi ora abbiamo che $f_i \cdot u$ (che è una funzione $C^1$) soddisfa l'\cref{eq:FormulaStokes} per ogni $i$.
	Infatti, per $i=0$, la funzione $f_0\cdot u$ è nulla all'interno di $\overline\Omega$, quindi l'\cref{eq:FormulaStokes} è banalmente soddisfatta; per $i>0$, la funzione $f_i \cdot u$ ha supporto contenuto in $U_p$ per qualche $p\in \overline\Omega$ e, per quanto
	dimostrato nel punto precedente, abbiamo che per questo intorno $U_p$ ogni funzione $C^1$ con supporto al suo interno soddisfa l'\cref{eq:FormulaStokes}.
	
	Per linearità di tutte le espressioni in gioco è chiaro che le funzioni che soddisfano la tesi formano
	un sottospazio vettoriale di $C^1$. Quindi in particolare $u$, essendo combinazione lineare di funzioni appartenenti a questo spazio, soddisfa la formula di Stokes.
\end{proof}

Questo teorema generalizza molti risultati di calcolo vettoriale. Vediamone alcuni.

Un'immediata conseguenza del teorema appena dimostrato è la seguente:

\begin{corollary}(Teorema della divergenza)
	\label{cor:thDivergenza}
	Siano $\Omega$ un aperto regolare e $u:\overline\Omega \rightarrow \R^n$ una funzione $C^1$ (nello stesso senso di prima), allora si ha che
	\begin{equation}\label{eq:thDivergenza}
		\int_{\Omega} \div u (x) \de x = \int_{\partial \Omega} u(y)\cdot\nu(y) \de \sigma(y)\virgola
	\end{equation}
	dove l'operatore di divergenza è definito da $\div u(x)=\tr\Diff u(x)$.
\end{corollary}

\begin{proof}
	Utilizzando il \cref{thm:Stokes} sulla componente $j$-esima di $u$ che indichiamo con $u^j$ si ha che, per ogni $1\leq j\leq n$,
	\[
		\int_{\Omega} \partial_j u^j(x) \de x = \int_{\partial \Omega} u^j(y)\nu(y)_j \de \sigma(y)\virgola
	\]
	da cui sommando otteniamo, per linearità degli integrali:
	\[
		\int_{\Omega} \sum_{j=1}^n\partial_j u^j(x) \de x = \int_{\partial \Omega} \sum_{j=1}^n u^j(y)\nu(y)_j \de \sigma(y)\punto
	\]
	Ma ora, $\div u (x)= \tr\Diff u(x) = \sum_{j=1}^n \partial_j u^j(x)$ e $u(x)\cdot\nu(x)=\sum_{j=1}^n u^j(y)\nu(y)_j$,
	quindi si ha la tesi.
\end{proof}

Giocherellando ancora con queste equazioni si ottengono anche le seguenti formule dette di Green:

\begin{corollary}(Prima formula di Green)
	Sia $\Omega$ un aperto regolare e siano $u\in C^1(\overline\Omega, \R)$ e $v\in C^2(\overline\Omega, \R)$, allora si ha
	\[
		\int_{\Omega} \left( u \Delta v+ \grad u\cdot\grad v \right) \de x=
		\int_{\partial\Omega} u \frac{\partial v}{\partial \nu} \de \sigma(y) \virgola
	\]
	dove l'operatore $\Delta v = \div \grad v$ è il laplaciano e $\frac{\partial v}{\partial \nu}(x)=\grad v(x)[\nu]$.
\end{corollary}
\begin{proof}
	Dato che
	\begin{equation*}
		\partial_j (u(x)\cdot\partial_j v(x) ) = \partial_j u(x)\cdot \partial_j v(x) + u(x)\cdot\partial_{jj} v(x)\virgola
	\end{equation*}
	applicando il \cref{thm:Stokes} sulle funzioni $u(x)\cdot\partial_j v(x)$ otteniamo
	\[
		\int_{\Omega} \left(\partial_j u(x)\cdot \partial_j v(x) + u(x)\cdot\partial_{jj} v(x) \right)\de x = 
		\int_{\Omega} \partial_j (u(x)\cdot\partial_j v(x)) \de x =
		\int_{\partial \Omega} u(y)\cdot\partial_j v(y)\cdot\nu(y)_j \de \sigma(y)\punto
	\]
	
	Quindi sommando su $j$ otteniamo la tesi, visto che
	\begin{align*}
		&\sum_{j=1}^n \partial_j u(x) \cdot\partial_j v(x) = \sum_{j=1}^n \partial_j u(x) \cdot\partial_j v(x) = \grad u\cdot\grad v \virgola\\
		&\sum_{j=1}^n u(x)\cdot\partial_{jj} v(x) = u \cdot\Delta v\virgola\\
		&\sum_{j=1}^n u(y)\cdot\partial_j v(y)\cdot\nu(y)_j = u\cdot (\grad v\cdot \nu)=u \cdot\frac{\partial v}{\partial \nu}\punto
	\end{align*}
\end{proof}

\begin{corollary}(Seconda formula di Green)
	Sia $\Omega$ un aperto regolare e siano $u,v\in C^2(\overline\Omega, \R)$, allora si ha che
	\[
		\int_{\Omega}\left(u\Delta v-v\Delta u\right) \de x=
		\int_{\partial\Omega}\left(u\frac{\partial v}{\partial\nu}-v\frac{\partial u}{\partial\nu}\right) \de \sigma(y)\punto
	\]
\end{corollary}

\begin{proof}
	Immediata conseguenza della prima formula applicata a $u,v$ e a $v,u$; poi basta sottrarre membro a membro.
\end{proof}


