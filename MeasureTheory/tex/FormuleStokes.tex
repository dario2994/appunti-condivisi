\section{Formule di Stokes}\label{sezione:FormuleStokes}

Durante tutta la sezione assumeremo sempre che $\Omega\subseteq\R^n$ sia un aperto.

\begin{theorem}\label{thm:PtRegEquiv}
	Per un punto $x\in \partial \Omega$ le seguenti $3$ condizioni sono equivalenti:
	\begin{enumerate}
		\item esistono $U$ intorno di $x$ e $f\in C^1(U,\R)$ tale che si abbia
			\begin{itemize}
				\item $U\cap \partial \Omega=f^{-1}(0)$,
				\item $U\cap \Omega=f^{-1}(\oo{-\infty}{0})$,
			\end{itemize}
			e tale che $\Diff f(x)\neq 0$ se $f(x)=0$;\label{PRE:i}
		\item esistono $U$ intorno di $x$ e $g$ diffeomorfismo da $\R^n$ in $U$ tale che\label{PRE:ii}
			\begin{itemize}
				\item $U\cap \partial \Omega = g(\R^{n-1}\times\{0\})$,
				\item $U\cap \Omega = g(\R^{n-1}\times\oo{-\infty}{0})$;
			\end{itemize}
		\item esistono, a meno di riordinare le coordinate, un intorno $U=V\times I$ di $x=(y,t)$ tale che $V\in\R^{n-1}$ è intorno di $y$
			e $I\in\R$ è intorno di $t$ e una funzione $\psi\in C^1(V,I)$ tale che
			\begin{itemize}
				\item $U\cap \partial \Omega = \{(z,s):\psi(z)=s\}$,
				\item $U\cap \Omega = \{(z,s):\psi(z)>s\}$.
			\end{itemize}\label{PRE:iii}
	\end{enumerate}
\end{theorem}

\begin{proof}
	Dimostriamo la catena in ordine inverso:
	\begin{description}
		\item [\ImplicationProof{PRE:i}{PRE:iii}] Dato che $\Diff f(x)\neq 0$, esiste una derivata parziale non nulla, supponiamo senza perdita di
			generalità che sia l'ultima e che sia positiva (se negativa il ragionamento è lo stesso).
			Allora le ipotesi del teorema della funzione implicita del Dini sono verificate, per cui otteniamo
			che esiste un intorno $V\times I\subseteq U$ di $x$ per cui esiste $\psi\in C^1(V,I)$ tale che $f(v,\psi(v))=0$ e $\psi(v)$
			è l'unico punto $t'\in I$ tale che $f(v,t')=0$.
			Inoltre, per la continuità di $\Diff f(x)$, esiste un intorno $V'\subseteq V$ di $v$ tale che $\Diff f(v')_n>\frac{1}{2}\Diff f(x)_n$ per tutti
			i $v'\in V'$. Ora, per la differenziabilità di $f$ in $V'$ si ha che
			\[
				f(v',\psi(v')+\delta)=f(v', \psi(v'))+\Diff f(v')_n\delta+\smallO(\delta)=\Diff f(v')\delta+\smallO(\delta),
			\]
			che, per $\delta>0$ e abbastanza piccolo, dà $f(v',\psi(v')+\delta)\geq \frac{1}{4}\Diff f(x)_n\delta>0$. Infine, non può esistere un $\delta>0$ tale che $\psi(v')+\delta\in I$ e
			$f(v',\psi(v')+\delta)<0$, altrimenti per la continuità di $f$ si avrebbe che esiste un punto $\psi(v')+\delta'\neq \psi(v')$
			che sia zero per la funzione $f(v',\cdot)$, contraddicendo il teorema della funzione implicita del Dini.
			Per $\delta<0$ si ha lo stesso risultato. Questo ci dice che $f^{-1}(0)\cap (V'\times I)=\{(v',s):s=\psi(v')\}$ e
			$f^{-1}(\oo{-\infty}{0})\cap (V'\times I)=\{(v',s):s<\psi(v')\}$, che conclude per le ipotesi sulle controimmagini di $f$.
		\item [\ImplicationProof{PRE:iii}{PRE:ii}] Se abbiamo $V\times I$ intorno di $x$ e $\psi$, detto $J=\oo{-\varepsilon}{\varepsilon}$
			possiamo ottenere un diffeomorfismo $g:V\times J \rightarrow V\times J$ ponendo $g(z,s)=(z,\psi(z)-s)$. Allora,
			$g(V\times \{0\})=$graf$(\psi)=U\cap \partial \Omega$ e $g(V\times \oo{-\varepsilon}{0})=\{(z,s):\psi(z)<s\}=U\cap \Omega$.
			Per ottenere un diffeomorfismo di tutto $\R^n$ basta comporre $g$ con opportuni diffeomorfismi tra un intorno rettangolare
			di $x$ e $\R^n$.
		\item [\ImplicationProof{PRE:ii}{PRE:i}] Per la prima condizione basta considerare$f=g^{-1}_n$ l'ultima componente di $g^{-1}$.
	\end{description}
\end{proof}


\begin{definition}[punto regolare]
	Diremo che un punto $x\in \partial \Omega$ è un punto regolare del bordo se verifica una delle $3$ condizioni di cui
	al \cref{thm:PtRegEquiv}.
\end{definition}

D'ora in poi assumeremo che $\Omega$ sia anche limitato.

\begin{definition}[aperto regolare]
	Diremo che $\Omega$ è un aperto regolare se tutti i punti del suo bordo sono regolari.
\end{definition}


\begin{theorem}[Partizione dell'unità]\label{thm:PartizioneUnita}
	Dato un ricoprimento di $\R^n$ con aperti $(\Omega_i)_{i\in I}$, esiste una famiglia di funzioni $f_i:\R^n\to\Rpiu$ infinitamente derivabili
	e non negative tali che:
	\begin{itemize}
		\item la chiusura dell'insieme in cui $f_i$ non si annulla è contenuta in $\Omega_i$ per ogni $i\in I$,
		\item per ogni $x\in\R^n$ vale che $\sum_{i\in I} f_i(x)=1$\footnote{La serie la definiamo come il $\sup$ delle somme finite}.
	\end{itemize}
\end{theorem}
\begin{proof}
	TODO
\end{proof}

\begin{lemma}\label{lem:EquivRegolare}
	$\Omega$ è un aperto regolare se e solo se esiste una funzione $\phi\in C^1(\R^n,\R)$ tale che $\Diff f(x)\neq 0$ se $f(x)=0$ e tale che
	\begin{itemize}
		\item $\Omega = f^{-1}(\oo{-\infty}{0})$,
		\item $\partial \Omega = f^{-1}(0)$.
	\end{itemize}
\end{lemma}

\begin{proof}
	L'implicazione del ``se'' è banale. Per l'altra implicazione, per la regolarità dei punti del bordo di $\Omega$, consideriamo per ogni
	$x\in\partial \Omega$ un intorno $U_x$ tale che esista una funzione $\psi_x:U_x\rightarrow \R$ come nel \cref{thm:PtRegEquiv} e prendiamo
	anche $U_0 = \R^n \setminus \partial \Omega$. Dato che $\Omega$ è limitato, $\partial\Omega$ è compatto e gli $U_x$ sono un ricoprimento
	aperto; quindi consideriamo un sottoricoprimento finito $U_1, \dots, U_n$. Ora abbiamo che gli aperti $U_0,\dots,U_n$ sono un ricoprimento
	di $\R^n$; il \cref{thm:PartizioneUnita} fornisce quindi delle funzioni $f_x$ per ogni $x\in\{0,\dots, n\}$.
	Infine, definiamo $\psi_0:U_0\rightarrow \R$ tale che $\psi_0(x)=1$ se $x\notin \Omega$, $\psi_0(x)=-1$ altrimenti. Sia ora 
	\[
		\phi(x)=\sum_{i=0}^n f_i(x) \psi_i(x),
	\]
	con una leggera imprecisione formale, poniamo $f_i(x) \psi_i(x) = 0$ se $x\notin U_i$, in modo che $f_i(x) \psi_i(x)$ sia una funzione
	$C^1$ in tutto $\R^n$, infatti, in $U_i$, $\psi_i$ è $C^1$ e $f_i$ è addirittura $C^{\infty}$, altrove l'abbiamo posta uguale a zero.
	Ora abbiamo che chiaramente $\phi\in C^1$ perché somma di funzioni $C^1$; $\phi(x)=0$ per ogni $x\in \partial\Omega$, dato che ognuna delle
	$\psi_i$ aveva questa proprietà se era definita in $x$ e la $\phi$ è una combinazione convessa di $\psi_i$;
	$\phi(x)<0$ per ogni $x \in \Omega$, perché $\phi$ è combinazione convessa di funzioni non positive e, per ciascun $x \in \Omega$,
	almeno di una strettamente negativa; similmente, $\phi(x)>0$ per ogni $x \notin \bar{\Omega}$.
	
	Rimane da mostrare che $\Diff f(x)\neq 0$ se $f(x)=0$. Ma in ciascun punto $x$ del bordo, le funzioni $\psi_i$ non hanno differenziale nullo quando
	sono definite in $x$. Esisterà quindi una direzione $e_j$ per la quale esiste almeno un $i$ tale che $\Diff \psi_i(x)_j\neq 0$. Dato che le $\psi_i$
	sono differenziabili, si ha che
	\[
		\psi_i(x+\delta e_j)=\psi_i(x)+\Diff \psi_i(x)_j \delta + \smallO(\delta) = \Diff \psi_i(x)_j \delta + \smallO(\delta)
	\]
	per cui, dato che $\psi_i(x+\delta e_j)<0$ solo se $x+\delta e_j\in \Omega$, per $\delta$ uguali si deve avere che i $\Diff \psi_i(x)_j$ non possono
	essere discordi al variare di $i$, altrimenti si avrebbe che per degli indici $i,i'$ si ha $\psi_i(x+\delta e_j)=
	\Diff \psi_i(x)_j \delta + \smallO(\delta)>0$ e $\psi_i'(x+\delta e_j)=\Diff \psi_i'(x)_j \delta + \smallO(\delta)<0$ pur avendo in entrambi i casi
	che $x+\delta e_j$ sta in $\Omega$. Quindi, sempre per lo stesso $x\in\partial \Omega$,
	\[ 
		\Diff \phi(x)=\sum_{i=0}^n \Diff (f_i \cdot\psi_i)(x)=\sum_{i=0}^n \Diff f_i(x)\cdot \psi_i(x)+\sum_{i=0}^n f_i(x)\cdot \Diff  \psi_i(x)=
		\sum_{i=0}^n f_i(x)\cdot \Diff  \psi_i(x)
	\]
	poiché $\psi_i\equiv 0$ sul bordo; quindi $\Diff \phi(x)$ non è nullo perché la $j$-esima componente è combinazione convessa di $\Diff \psi_i(x)_j$.
\end{proof}

\begin{definition}
	Chiamiamo campo normale esterno la funzione $\nu:\partial\Omega\rightarrow \R^n$ tale che
	\[
		\nu(x)=\frac{\nabla \phi(x)}{||\nabla \phi(x)||}.
	\]
\end{definition}

\begin{remark}
	Il \cref{lem:EquivRegolare}, oltre che a fornire la funzione $\phi$ per l'aperto regolare $\Omega$, dice anche che $\nu(x)$ è ben definita
	su tutto $\partial \Omega$ e che è continua.
\end{remark}

\begin{remark}
	Si ha che per $t>0$ e $x\in\partial\Omega$, $x+t\nu(x)\notin \Omega$ per $t$ abbastanza piccoli e per $t<0$, $x+t\nu(x)\in \Omega$ per $t$
	abbastanza piccoli.
\end{remark}
\begin{proof}
	Stesso ragionamento fatto nella dimostrazione di \cref{lem:EquivRegolare}.
\end{proof}

\begin{remark}
	Per ogni $v$ tangente a $\partial\Omega$ in $x$ si ha che $<v,\nu(x)>=0$. Ma qui bisognerebbe spendere un paio di parole su cosa significhi
	essere tangente a $\partial\Omega$.
\end{remark}


Vogliamo ora formalizzare la definizione di integrale sulla superficie $\partial\Omega$ sfruttando anche i risultati della sezione precedente.

\begin{lemma}\label{lem:GrapVar}
	Sia $V\in\R^{n-1}$ aperto e $f\in C^1(V,\R)$, allora il grafico di $f$ è una varietà differenziabile di dimensione $n-1$ immersa in $\R^n$.
\end{lemma}
\begin{proof}
	Sia $g:V\rightarrow \R^n$ data da $g(x)=(x,f(x))$. Ora è chiaro che graf$(f)$ è l'immagine di $g$; inoltre $g$ è banalmente di classe $C^1$
	e, se vista come funzione da $V$ in graf$(f)$ è anche un omeomorfismo. Inoltre $\Diff g(x)[h]=(h,\Diff f(x)[h])$, in particolare il differenziale è
	iniettivo in ogni punto, quindi $g$ è una immersione iniettiva, cioè è una parametrizzazione $C^1$ di graf$(f)$.
\end{proof}

\begin{remark}
	Dalla dimostrazione di questo lemma, dato che $\nabla f(x)=\Diff f(x)$, segue immediatamente che $\Diff g(x)^T \Diff g(x)=
	(I \ \ \nabla f(x)^T)
	\left(\begin{smallmatrix}
	I \\
	\nabla f(x)
	\end{smallmatrix}\right)=
	I+\nabla f(x)^T\nabla f(x)$, quindi gli autovalori di questa matrice sono $1$, con molteplicità $n-1$, e $1+\nabla f(x)^T\nabla f(x)$, con 
	molteplicità $1$. Tutto ciò per dire che $\det (\Diff g(x)^T \Diff g(x)) = 1+||\nabla f(x)||^2$. 
\end{remark}

Il \cref{lem:GrapVar} ci permette quindi di costruire un misura superficiale su graf$(f)$ in accordo con la \cref{def:MisuraKDimensionale}:
\[
	\sigma(E)=\int_{g^{-1}(E)} \sqrt{1+||\nabla f(x)||^2} \de x.
\]


\begin{proposition}\label{prop:FormulaStokes}
	Sia $u\in C^1(\bar \Omega)$ (ossia $u$ è la restrizione a $\bar \Omega$ di una funzione $C^1$ definita su un aperto
	$\Omega'\supseteq \bar \Omega$), sia $\nu:\partial \Omega \rightarrow \R^n$ la normale uscente da $\Omega$, allora vale che
	\begin{equation}\label{equ:FormulaStokes}
		\int_{\Omega} \Diff u(x)_j \de x = \int_{\partial \Omega} u(x)\nu(x) \de \sigma(y)\punto
	\end{equation}
\end{proposition}

\begin{proof}
	Per la dimostrazione si procederà in 2 step:
	\begin{enumerate}
		\item per ogni $p\in \bar \Omega$ esiste un intorno aperto rettangolare $U_p=\prod_{i=1}^n \oo{p_i-\varepsilon_i}{p_i+\varepsilon_i}$
			tale che la \cref{equ:FormulaStokes} vale per ogni funzione $u$ tale che $supp(u)\subseteq U_p$;
		\item ricoprire $\bar \Omega$ con un numero finito di intorni come sopra per poi applicare il \cref{thm:PartizioneUnita} per ottenere
			la tesi per ogni $u$.
	\end{enumerate}

\end{proof}

