\section{Formule di Stokes}\label{sezione:FormuleStokes}

Durante tutta la sezione assumeremo sempre che $\Omega\subseteq\R^n$ sia un aperto limitato.

\begin{theorem}\label{thm:PtRegEquiv}
	Per un punto $x\in \partial \Omega$ le seguenti $3$ condizioni sono equivalenti:
	\begin{enumerate}
		\item esistono $U$ intorno di $x$ e $f\in C^1(U,\R)$ tale che si abbia
			\begin{itemize}
				\item $U\cap \partial \Omega=f^{-1}(0)$,
				\item $U\cap \Omega=f^{-1}(]-\infty,0[)$,
			\end{itemize}
			e tale che $Df(x)\neq 0$ se $f(x)=0$;\label{PRE:i}
		\item esistono $U$ intorno di $x$ e $f$ diffeomorfismo da $\R^n$ in $U$ tale che\label{PRE:ii}
			\begin{itemize}
				\item $U\cap \partial \Omega = f(\R^{n-1}\times\{0\})$,
				\item $U\cap \Omega = f(\R^{n-1}\times]-\infty,0[)$;
			\end{itemize}
		\item esistono, a meno di riordinare le coordinate, un intorno $U=V\times I$ di $x=(y,t)$ tale che $V\in\R^{n-1}$ è intorno di $y$
			e $I\in\R$ è intorno di $t$ e una funzione $\psi\in C^1(V,I)$ tale che
			\begin{itemize}
				\item $U\cap \partial \Omega = \{(z,s):\psi(z)=s\}$,
				\item $U\cap \Omega = \{(z,s):\psi(z)<s\}$.
			\end{itemize}\label{PRE:iii}
	\end{enumerate}
\end{theorem}

\begin{proof}
	Dimostriamo la catena in ordine inverso:
	\begin{description}
		\item [\ImplicationProof{PRE:i}{PRE:iii}] Dato che $Df(x)\neq 0$, esiste una derivata parziale non nulla, supponiamo senza perdita di
			generalità che sia l'ultima. Allora le ipotesi del teorema della funzione implicita del Dini sono applicabili, per cui
			otteniamo che il luogo degli zeri di $f$ si scrive come il grafico di una funzione $\psi$ che soddisfa esattamente le condizioni
			del punto \ref{PRE:iii}.
		\item [\ImplicationProof{PRE:iii}{PRE:ii}]
		\item [\ImplicationProof{PRE:ii}{PRE:i}]
	\end{description}

\end{proof}


\begin{definition}[punto regolare]
	Diremo che un punto $x\in \partial \Omega$ è un punto regolare del bordo se verifica una delle $3$ condizioni di cui
	al \cref{thm:PtRegEquiv}.
\end{definition}

\begin{definition}[aperto regolare]
	Diremo che $\Omega$ è un aperto regolare se tutti i punti del suo bordo sono regolari.
\end{definition}

