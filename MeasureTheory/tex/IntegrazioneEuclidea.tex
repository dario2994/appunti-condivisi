\section{L'integrazione negli spazi Euclidei}

\begin{proposition}\label{prop:MisuraProdottoEuclidea}
	La \sigalg{} prodotto completata $\overline{\M_n\otimes\M_m}$ dei misurabili di $\R^n$ e $\R^m$ coincide con la \sigalg{} dei misurabili in $\R^{n+m}$.
\end{proposition}
\begin{proof}
	Lavoriamo sulla \sigalg{} prodotto \emph{non completata} $\M_n\otimes\M_m$, per poi ottenere l'enunciato ricordando che il completamento dei Boreliani sono i misurabili.

	Ricordando la \cref{def:LebesgueSemiaperti}, poichè valgono ovviamente i contenimenti $S_n\subseteq M_n$ e $S_m\subseteq M_m$, è facile ricavarne che $S_{n+m}\subseteq M_n\otimes M_m$.
	Ma allora, applicando la \cref{prop:SigAlgUgualeBoreliani} e ricordando che $\M_n\otimes M_m$ è una \sigalg{}, otteniamo che i Boreliani sono contenuti in $\M_n\otimes\M_m$.
	
	Sia $E_1\times E_2\in \M_n\times\M_m$. Per il \cref{thm:LebesgueEquivalenzeMisurabilita}, esistono $B_1,B_2$ Boreliani e $N_1,N_2$ trascurabili, rispettivamente in $R^n$ e $R^m$, tali che $E_1=A_1\sqcup N_1$ e $E_2=A_2\sqcup N_2$.
	Perciò, grazie alla distributività del prodotto insiemistico rispetto all'unione disgiunta, otteniamo
	\begin{equation*}
		E_1\times E_2=(A_1\sqcup N_1)\times(A_2\sqcup N_2)=\left(A_1\times A_2\right)\sqcup\left(A_1\times N_2\sqcup A_2\times N_1 \sqcup N_1\times N_2\right) \punto
	\end{equation*}
	Però l'insieme $A_1\times N_2\sqcup A_2\times N_1 \sqcup N_1\times N_2$ è trascurabile per la \cref{prop:TrascurabilePerInsiemeTrascurabile} e l'insieme $A_1\times A_2$ è un Boreliano in quanto prodotto di Boreliani\footnote{Lasciamo al lettore la dimostrazione che il prodotto di Boreliani è un Boreliano.}, allora applicando ancora il \cref{thm:LebesgueEquivalenzeMisurabilita} si ottiene $E_1\times E_2\in\M_{n+m}$.
	Vista la generalità della scelta di $E_1\times E_2$, quanto appena mostrato implica che $\M_n\times\M_m\subseteq \M_{n+m}$.
	
	Quindi abbiamo dimostrato che $\M_n\otimes\M_m$ contiene i Boreliani ed è un sottoinsieme dei misurabili, perciò, ricordando la \cref{prop:CompletamentoBoreliani}, è ovvio concluderne che il suo completamento\footnote{È fondamentale notare che la misura indotta dal prodotto e la misura di Lebesgue coincidono in virtù della \cref{prop:UnicitaCaratheodory} e di conseguenza anche l'operatore di completamento di una \sigalg{} è coincidente.} $\overline{\M_n\otimes\M_m}$ coincide con l'insieme dei misurabili.
\end{proof}


\begin{proposition}\label{prop:IntegraleRiemannCoincide}
	Data una funzione $f:\cc ab\to \mathbb R$, se essa è integrabile secondo Riemann ed integrabile secondo Lebesgue allora i due integrali coincidono.
\end{proposition}
\begin{proof}
	Nel caso in cui la funzione sia integrabile secondo Riemann (e quindi anche limitata, dando senso agli $\inf,\sup$ che compariranno nelle formule), il valore dell'integrale secondo Riemann corrisponde all'estremo superiore delle somme inferiori alla Riemann\footnote{Indicheremo con $\int_a^b$ l'integrale di Riemann, e con $\int_{\cc ab}$ l'integrale di Lebesgue}
	\begin{equation*}
		\int_a^b f(x)\de x=\sup\left\{\sum_{i=0}^{k-1} (x_{i+1}-x_i)\left(\inf_{t\in\co{x_i}{x_{i+1}}}f(t)\right):\ a=x_0<x_1<\cdots<x_{k-1}<x_k=b\right\} \virgola
	\end{equation*}
	ed anche all'estremo inferiore delle somme superiori alla Riemann
	\begin{equation*}
		\int_a^b f(x)\de x=\inf\left\{\sum_{i=0}^{k-1} (x_{i+1}-x_i)\left(\sup_{t\in\co{x_i}{x_{i+1}}}f(t)\right):\ a=x_0<x_1<\cdots<x_{k-1}<x_k=b\right\} \punto
	\end{equation*}
	
	Data una partizione $a=x_0<x_1<\cdots<x_{k-1}<x_k=b$ dell'intervallo $\cc ab$, definiamo le funzioni
	\begin{align*}
		f^-_{x_0,x_1,\dots,x_k}(x)=\sum_{i=0}^{k-1}\chi_{\co{x_i}{x_{i+1}}}(x)\cdot\inf_{t\in\co{x_i}{x_{i+1}}}f(t) \virgola \\
		f^+_{x_0,x_1,\dots,x_k}(x)=\sum_{i=0}^{k-1}\chi_{\co{x_i}{x_{i+1}}}(x)\cdot\sup_{t\in\co{x_i}{x_{i+1}}}f(t) \punto
	\end{align*}

	Con le definizioni mostrate è facile accorgersi che l'integrale di Riemann allora è l'estremo superiore dell'integrale di Lebesgue delle funzioni semplici $f^-_{x_0,\dots,x_k}$ su tutte le partizioni e analogamente l'estremo inferiore dell'integrale di Lebesgue delle funzioni semplici $f^+_{x_0,\dots,x_k}$ ancora su tutte le partizioni.
	Allora, per la monotonia dell'integrale di Lebesgue è facile accorgersi che
	\begin{multline*}
		\int_{\cc ab}f(x)\de x\le \\
		\inf_{a=x_0<\dots<x_k=b}\left\{\int_{\cc ab}f^+_{x_0,\dots,x_k}(x)\de x\right\}
		=\int_a^b f(x)\de x= 
		\sup_{a=x_0<\dots<x_k=b}\left\{\int_{\cc ab}f^-_{x_0,\dots,x_k}(x)\de x\right\}\\
		 \le \int_{\cc ab} f(x)\de x
	\end{multline*}
	e ciò equivale ovviamente alla tesi.
\end{proof}


\begin{proposition}\label{prop:MisuraImmagineLineare}
	Fissata un'applicazione lineare $L:\R^n\to\R^n$, per ogni $E\in\M_n$ misurabile secondo Lebesgue risulta che $L(E)$ è misurabile e rispetta
	\begin{equation}
		m_n(L(E))=\lvert \det L\rvert\cdot m_n(E)\punto
	\end{equation}
\end{proposition}
\begin{proof}
	La funzione $L$ essendo lineare è in particolare Lipschitziana (con costante la propria norma) quindi, applicando la \cref{prop:LipschitzTengonoMisurabili}, otteniamo che per ogni $E$ misurabile, $L(E)$ è anch'esso misurabile.
	
	Consideriamo ora la funzione $\mu_L:\M_n\to\Rpiu$ definita come $\mu_L(E)=m_n(L(E))$. Per quanto appena mostrato questa è una buona definizione.
	Per la \cref{prop:BigettivaInduceMisura} la $\mu_L$ è una misura.
	Inoltre, ricordando che la misura di Lebesgue è invariante per traslazione come mostrato nella \cref{nota:LebesgueProprieta}, si ottiene
	\begin{equation*}
		\forall v\in\R^n:\ \mu_L(E+v)=m_n(L(E+v))=m_n(L(E)+Lv)=m_n(L(E))=\mu_L(E)
	\end{equation*}
	che equivale a dire che $\mu_L$ è invariante per traslazione.
	\newcommand{\linR}{\ensuremath{\mathcal L(\R^n,\R^n)}}
	Allora possiamo applicare il \cref{thm:LebesgueUnicaInvarianteTraslazione}\footnote{Il teorema ci assicura l'identità delle misure solo sui Boreliani, ma i misurabili sono il completamento dei Boreliani e perciò le due misure coincidono anche sui misurabili.} ottenendo che esiste una funzione $c:\linR\to\Rpiu$ con dominio le applicazioni lineari, tale che valga
	\begin{equation}\label{eq:DefQuasiDet}
		\forall L\in \linR,\ E\in\M_n:\ \mu_L(E)=c(L)m_n(E)\punto
	\end{equation}
	
	Mostriamo ora che $c(\cdot)$ è moltiplicativa e che coindice con il valore assoluto del determinante sulle applicazioni diagonali e ortogonali. Da questo, sfruttando una decomposizione nota delle applicazioni lineari di $\R^n$, seguirà che coincide con il valore assoluto del determinante su ogni applicazione lineare e questo è proprio quanto richiesto dalla tesi. 
	
	Fissate $A,B\in \linR$ applicazioni lineari ed $E\in\M_n$ un misurabili non trascurabile, applicando unicamente l'\cref{eq:DefQuasiDet} risulta
	\begin{align*}
		c(AB)m_n(E)&=\mu_{AB}(E)=m_n(AB(E))=m_n(A(B(E))\\
		&=\mu_A(B(E))=c(A)m_n(B(E))=c(A)\mu_B(E)=c(A)c(B)m_n(E)\virgola
	\end{align*}
	da cui si ricava $c(AB)=c(A)c(B)$ dividendo per $m_n(E)$. Perciò $c$ è moltiplicativa.
	
	Sia $D\in\linR$ un'applicazione diagonale, in particolare siano $(\lambda_i)_{1\le i\le n}$ i valori sulla diagonale.
	Allora è facile ricavare
	\begin{equation*}
		D\left(\co01\times\co01\times\dots\times\co01\right)=\co0{\lambda_1}\times\co0{\lambda_2}\times\cdots\times\co0{\lambda_n}\virgola
	\end{equation*}
	da cui, applicando ad entrambi i membri la misura di Lebesgue, si ottiene
	\begin{align*}
		c(D)&=c(D)m_n\left(\co01\times\dots\times\co01\right)=\mu_D\left(\co01\times\dots\times\co01\right)\\
		&=m_n\left(\co0{\lambda_1}\times\cdots\times\co0{\lambda_n}\right)=
		\lvert\lambda_1\rvert\cdot\lvert\lambda_2\rvert\cdots\lvert\lambda_n\rvert=\lvert\det(D)\rvert
	\end{align*}
	che equivale a dire che $c(\cdot)$ e $\lvert\det(\cdot)\rvert$ coincidono sulle matrici diagonali.
	
	Fissata un'applicazione $O\in\linR$ ortogonale, chiamando $P$ la palla unitaria di $\R^n$ è ovvio che $O(P)=P$.
	Da questo, applicando ad entrambi i membri la misura di Lebesgue, si ricava $\mu_O(P)=m_n(P)$ e, ricordando che $P$ non è trascurabile, ne discende $c(O)=1$.
	Ma $O$ è ortogonale, quindi $\det O=\pm 1$ e allora è dimostrato anche in questo caso $\lvert\det O\rvert =c(O)$.
	
	Infine, data un'applicazione lineare generica $L\in\linR$, sia $L=OS$ la sua decomposizione polare\footnote{La si ottiene notando che $LL^t$ è una matrice simmetrica definita positiva che perciò ammette una ``radice quadrata''.} dove $O$ è ortogonale e $S$ è simmetrica. Per il teorema spettrale esistono $P,D$ rispettivamente invertibile e simmetrica tali che $S=PDP^{-1}$.
	Ricordando le proprietà che rispetta la funzione $c$ e la moltiplicatività del determinante, ricaviamo
	\begin{align*}
		c(L)&=c(OPDP^{-1})=c(O)c(P)c(D)c(P^{-1})=c(O)c(D)c(P)c(P^{-1})\\
		&=c(O)c(D)c(PP^{-1})=\lvert\det O\rvert\cdot\lvert\det D\rvert=\lvert \det(OPDP^{-1})\rvert=\lvert\det L\rvert \virgola
	\end{align*}
	che è equivale a dire che $c(\cdot)$ e $\lvert\det(\cdot)\rvert$ coincidono come si voleva.
\end{proof}

\begin{definition}\label{def:AssolutamenteContinua}
	Date due misure $\mu:\A\to\Rpiu,\nu:\A\to\Rpiu$ sullo stesso spazio misurabile $(X,\A)$, diciamo che $\nu$ è assolutamente continua rispetto a $\mu$, indicandolo con $\nu\ll\mu$, se, per ogni $E\in\A$, se $\mu(E)=0$ allora $\nu(E)=0$, cioè se tutti gli insiemi trascurabili per $\mu$ sono trascurabili anche per $\nu$.
\end{definition}

\begin{theorem}\label{thm:RadonNikodym} [Radon-Nikodym]
	Dato uno spazio di misura $(X,\A,\mu)$ \sigfin[o], se la misura $\nu:\A\to\Rpiu$ è assolutamente continua rispetto a $\mu$, allora esiste una funzione $\rho:X\to\Rpiu$ misurabile rispetto a $\mu$, tale che
	\begin{equation*}
		\forall A\in\A:\ \nu(A)=\int_A \rho(x)\de\mu x\punto
	\end{equation*}
\end{theorem}
\begin{proof}
	%TODO
	Magari da lasciare al lettore.
\end{proof}

\begin{lemma}\label{lemma:LimiteDensita}%TODO Si possono sicuro allegerire le ipotesi.
	Fissato $\Omega,\Omega'\subseteq\R^n$ un aperto, sia $\varphi:\Omega\to\Omega'$ un diffeomorfismo con differenziale continuo.
	
	Allora, per ogni $A\subset\Omega$ Boreliano, l'insieme $\varphi(A)$ è misurabile\footnote{In realtà è anche Boreliano.} e perciò $m_n\circ\varphi$ è una misura sui Boreliani di $\Omega$ in virtù della \cref{prop:BigettivaInduceMisura}.
	Sia quindi $\rho:\Omega\to\Rpiu$ una funzione misurabile, assicurataci dal \cref{thm:RadonNikodym}, tale che
	\begin{equation*}
		\forall A\ \text{Boreliano},\ A\subseteq \Omega:\ \int_A \rho(x)\de m_n(x)=m_n(f(A)) \punto
	\end{equation*}
	Allora la serie di funzioni
	\begin{equation*}
		\rho_r(x)=\frac{m_n\left(\varphi\left(B_r(x)\cap \Omega\right)\right)}{m_n\left(B_r(x)\right)}\virgola
	\end{equation*}
	dove $B_r(x)$ è la palla aperta di raggio $r$ e centro $x$, converge in $\L$, con $r\to 0$, alla funzione $\rho$.
\end{lemma}
\begin{proof}
	Innanzitutto, ricordando che $\varphi^{-1}$ è continua e perciò misurabile, applichiamo la \cref{prop:CounterImgMis} otteniamo che $\varphi$ manda Boreliani in misurabili, come detto implicitamente nell'enunciato.
	
	Ora allarghiamo per comodità il dominio di $\rho$ a tutto $\R^n$ ponendo $\rho(\Omega^{\mathsf{c}})=\{0\}$.
	
	Calcoliamo la norma $\L$ della differenza tra $\rho$ e $\rho_r$ applicando la definizione di $\rho$:
	\newcommand{\C}{\ensuremath{m_n\left(B_r(x)\right)}}
	\begin{multline*}
		\LNorm{\rho-\rho_r}=\int_\Omega \rho(x)-\frac{m_n\left(\varphi\left(B_r(x)\cap \Omega\right)\right)}{\C} \de x\\
		= \C^{-1}\int_\Omega \left(\C\rho(x)-\int_{B_r(x)\cap\omega}\rho(t)\de t\right)\de x\\
		=\C^{-1}\int_\Omega \int_{B_r(x)}\rho(x)-\rho(t)\de t\de x
		=\C^{-1}\int_\Omega \int_{B_r(0)}\rho(x)-\rho(x+t)\de t\de x\\
		=\C^{-1}\int_{B_r(0)}\int_\Omega \rho(x)-\rho(x+t) \de x\de t
		\le \C^{-1}\int_{B_r(0)}\LNorm{\rho(\cdot+t)-\rho(\cdot)}\de t \\
		\le \sup_{t\in B_r(0)} \LNorm{\rho(\cdot+t)-\rho(\cdot)}
	\end{multline*}
	%TODO: Bisogna finire questa dimostrazione!
\end{proof}


\begin{lemma}\label{lemma:LimiteDeterminante}
	Fissato $\Omega\subseteq\R^n$ un aperto, sia $\varphi:\Omega\to\R^n$ una funzione differenziabile con continuità.
	Allora vale la seguente disuguaglianza:
	\begin{equation*}
		\forall x\in\Omega:\ \limsup_{r\to 0} \frac{ m_n\left(\varphi\left(B_r(x)\right)\right)} {m_n\left(B_r(x)\right)}\le \lvert\det D\varphi(x)\rvert \virgola 
	\end{equation*}
	dove $B_r(x)$ è la palla aperta di raggio $r$ e centro $x$ e $D\varphi$ è la matrice Jacobiana della funzione $\varphi$.
\end{lemma}
\begin{proof}
	Fissiamo $\bar x\in\Omega$ e chiamiamo $A=D \varphi(\bar x)$.
	Allora, per definizione di differenziale, risulta vero che
	\begin{equation*}
		\varphi(\bar x+h)=\varphi(\bar x)+Ah+\smallO(h)=\varphi(\bar x)+A(h+\smallO(h))\in \varphi(\bar x)+A\left(B_{|h|+\smallO(h)}(0)\right) \virgola
	\end{equation*}
	dove abbiamo implicitamente sfruttato che moltiplicare per la norma degli operatori di $A$ non cambia il fatto che una funzione sia $\smallO$-piccolo di un'altra.
	
	Da quanto appena detto discende facilmente che, per ogni $r>0$, vale
	\begin{equation}\label{eq:ContenimentoPallaLineare}
		\varphi(B_r(\bar x))\subseteq \varphi(\bar x)+A\left(B_{r+\smallO(r)}(0)\right).
	\end{equation}
	Essendo però $\varphi$ differenziabile con continuità, è anche localmente Lipschitziana e perciò, purchè $r$ sia sufficientemente piccolo, la funzione è Lipschitziana in $B_r(\bar x)$ e quindi, per la \cref{prop:LipschitzTengonoMisurabili}, l'insieme $\varphi(B_r(\bar x))$ è misurabile.
	Allora possiamo applicare la misura di Lebesgue ad entrambi i membri del contenimento mostrato nell'\cref{eq:ContenimentoPallaLineare} ottenendo
	\begin{equation}\label{eq:StimaImmaginePalla}
		m_n\left( \varphi(B_r(\bar x)) \right)\le m_n\left(\varphi(\bar x)+A\left(B_{r+\smallO(r)}(0)\right)\right)
		=\lvert \det A\rvert m_n\left(B_{r+\smallO(r)}(\bar x)\right)\virgola
	\end{equation}
	dove i passaggi sono giustificati dall'invarianza per traslazione della misura di Lebesgue e dalla \cref{prop:MisuraImmagineLineare}.
	Ma ora, grazie alla $n$-omogeneità della misura $m_n$, ricaviamo
	\begin{equation*} 
		m_n\left(B_{r+\smallO(r)}(\bar x)\right)=\left(1+\smallO(r)\right)^n m_n\left( B_r(\bar x)\right)
	\end{equation*}
	e perciò sostituendo questa nell'\cref{eq:StimaImmaginePalla} arriviamo a
	\begin{equation*}
		m_n\left( \varphi(B_r(\bar x)) \right)\le \left(1+\smallO(r)\right)^n m_n\left( B_r(\bar x)\right)\lvert \det A\rvert \virgola
	\end{equation*}
	che implica banalmente la tesi.
\end{proof}


\begin{theorem}\label{thm:CambioVariabile}
	Fissato $\Omega\subseteq\R^n$ un aperto, sia $\varphi:\Omega\to\R^n$ una funzione differenziabile con continuità, tale che in differenziale non si annulla mai.
	Allora per ogni funzione $f:\varphi(\Omega)\to\R$ integrabile vale
	\begin{equation*}
		\int_{\varphi(\Omega)} f(v)\de m_n(v) = \int_{\Omega} f(\varphi(u))\lvert \det D\varphi(u) \rvert \de m_n(y) \virgola
	\end{equation*}
	dove $D\varphi$ è la matrice Jacobiana del diffeomorfismo $\varphi$.
\end{theorem}
\begin{proof}
	%TODO
\end{proof}

