\section{Funzioni misurabili}

In questa sezione daremo la definizione di funzione misurabile e dimostreremo alcuni fatti basilari su di esse.

\begin{definition}[Funzione misurabile]
	Dato uno spazio di misura $(X,\A,\mu)$, una funzione $f:X\rightarrow \R$ si dice misurabile se
	$\forall A \subseteq \R$ aperto si ha $f^{-1}(A)\in \A$.
\end{definition}

\begin{proposition}
	\label{BasicMis}
	Dato uno spazio di misura $(X,\A,\mu)$, sia $f:X\rightarrow \R$ una funzione, sono equivalenti i seguenti fatti:
	\begin{enumerate}[label=(\arabic*),ref=(\arabic*)]
		\item $f$ è misurabile; \label{BM:mis}
		\item $\{f<a\}\in \A \quad \forall a\in \R$; \label{BM:sot}
		\item $\{f\leq a\}\in \A \quad \forall a\in \R$; \label{BM:soteq}
		\item $\{f>a\}\in \A \quad \forall a\in \R$; \label{BM:sov}
		\item $\{f\geq a\}\in \A \quad \forall a\in \R$;  \label{BM:soveq}
		\item $\{a<f<b\}\in \A \quad \forall a,b\in \overline{\R}$; \label{BM:int}
	\end{enumerate}
\end{proposition}
\begin{proof}
	$\text{\ref{BM:mis}}\implies\text{\ref{BM:sot}}$ per definizione;
	
	$\text{\ref{BM:sot}}\implies\text{\ref{BM:soteq}}$ sfruttando il fatto che $\A$ è \sigalg{} e che
	$(-\infty,a]=\cap_{n\in \N}(-\infty,a+\frac{1}{n})$;
	
	$\text{\ref{BM:soteq}}\implies\text{\ref{BM:sov}}$ passando al complementare;
	
	$\text{\ref{BM:sov}}\implies\text{\ref{BM:soveq}}$, $\text{\ref{BM:soveq}}\implies\text{\ref{BM:sot}}$ come i precedenti;
	
	$\text{\ref{BM:sot}}\ +\ \text{\ref{BM:sov}}\implies\text{\ref{BM:int}}$ per intersezione;
	
	$\text{\ref{BM:int}}\implies\text{\ref{BM:mis}}$ perché in $\R$ gli aperti sono unione numerabile di intervalli aperti (o semirette aperte).
\end{proof}

\begin{proposition}
	\label{CounterImgMis}
	Dato uno spazio di misura $(X,\A,\mu)$, e data $f:X\rightarrow \R$ una funzione misurabile, la famiglia di insiemi
	\[
		\mathcal{E} = \{ E\subseteq \R : f^{-1}(E)\in \A \}
	\]
	è una \sigalg{}.
\end{proposition}
\begin{proof}
	Unione numerabile: siano $\{E_n\}_{n\in \N}\subseteq \mathcal{E}$ e sia $E = \cup_{n\in \N}E_n$,
	allora $f^{-1}(E)=f^{-1}\left(\cup_{n\in \N}E_n\right) = \cup_{n\in \N}f^{-1}(E_n)\in \A \Rightarrow E \in \mathcal{E}$;
	
	Complementare: sia $E\in \mathcal{E}$, allora $f^{-1}(^{c}E)= ^{c}f^{-1}(E)\in \A \Rightarrow ^{c}E \in \mathcal{E}$.
\end{proof}

\begin{proposition}
	Dato uno spazio di misura $(X,\A,\mu)$, sia $\mathcal{M}=\{f:X\rightarrow \R : f\ \grave{e} \ misurabile\}$.
	Allora $\mathcal{M}$ è un'algebra nel senso che valgono le seguenti:
	\begin{enumerate}[label=(\arabic*),ref=(\arabic*)]
		\item $f,g\in \mathcal{M} \Rightarrow f+g\in \mathcal{M}$; \label{AlM:sum}
		\item $f\in \mathcal{M}, \lambda \in \R \Rightarrow \lambda f\in \mathcal{M}$. \label{AlM:sca}
		\item $f,g\in \mathcal{M} \Rightarrow fg\in \mathcal{M}$. \label{AlM:pro}
	\end{enumerate}
\end{proposition}
\begin{proof}
	Mostriamo per ogni punto che vale \ref{BM:sov} in \cref{BasicMis}.
	\ref{AlM:sum}: 
	\[
		\{f+g>a\}=\bigcup_{q\in \Q}[\{f>q\}\cap\{g>a-q\}]\in \A
	\]

	
	\ref{AlM:sca}:
	\[
		\{\lambda f>a\}=\left\{f>\frac{a}{\lambda}\right\}\in \A
	\]

	\ref{AlM:pro}: Per ogni funzione $h$ possiamo scrivere la parte positiva e la parte negativa: $h^+ = max\{h,0\}$, $h^- = max\{0,-h\}$,
	così $h^+,h^- \geq 0$ e $h = h^+ - h^-$. Ora scomponiamo $f=f^+ - f^-$, $g=g^+- g^-$, quindi $fg$ si scrive
	come una qualche combinazione di prodotti di funzioni non negative. Grazie ai punti \ref{AlM:sum} e \ref{AlM:sca} e a questa osservazione
	ci basta mostrare il caso in cui $f,g\geq0$:
	\[
		\{fg>a\}=\left\{\begin{array}{ll}
			\R &\qquad se\ a<0;\\
			\{f=0\}\cup\{g=0\} &\qquad se\ a=0;\\
			\bigcup_{q\in \Q^+}\left[\{f>q\}\cap\left\{g>\frac{a}{q}\right\}\right]\in \A &\qquad se\ a>0.
		\end{array}\right.
	\]
\end{proof}

\begin{remark}
	È facile vedere che le funzioni caratteristiche degli insiemi misurabili sono misurabili.
\end{remark}
\begin{proof}
	Basta osservare che se $A\in \A$ è misurabile, $\{ \chi_A > a\}$ può valere solo $\emptyset$, $A$, $\R$ (tutti e 3 misurabili) a seconda che
	$a\geq 1$, $a\geq 0$ oppure $a < 0$ rispettivamente.
\end{proof}
