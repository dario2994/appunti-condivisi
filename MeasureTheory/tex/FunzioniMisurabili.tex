\section{Funzioni misurabili}

In questa sezione daremo la definizione di funzione misurabile e dimostreremo alcuni fatti basilari su di esse.

\begin{definition}[Funzione misurabile]
	Dato uno spazio di misura $(X,\A,\mu)$, una funzione $f:X\rightarrow \R$ si dice misurabile se
	$\forall A \subseteq \R$ aperto si ha $f^{-1}(A)\in \A$.
\end{definition}

\begin{proposition}
	\label{BasicMis}
	Dato uno spazio di misura $(X,\A,\mu)$, sia $f:X\rightarrow \R$ una funzione, sono equivalenti i seguenti fatti:
	\begin{enumerate}
		\item $f$ è misurabile; \label{BM:mis}
		\item $\{f<a\}\in \A \quad \forall a\in \R$; \label{BM:sot}
		\item $\{f\leq a\}\in \A \quad \forall a\in \R$; \label{BM:soteq}
		\item $\{f>a\}\in \A \quad \forall a\in \R$; \label{BM:sov}
		\item $\{f\geq a\}\in \A \quad \forall a\in \R$;  \label{BM:soveq}
		\item $\{a<f<b\}\in \A \quad \forall a,b\in \overline{\R}$; \label{BM:int}
	\end{enumerate}
\end{proposition}
\begin{proof}
	\cref{BM:mis} $\Rightarrow$ \cref{BM:sot} per definizione;
	
	\cref{BM:sot} $\Rightarrow$ \cref{BM:soteq} sfruttando il fatto che $\A$ è \sigalg{} e che
	$(-\infty,a]=\cap_{n\in \N}(-\infty,a+\frac{1}{n})$;
	
	\cref{BM:soteq} $\Rightarrow$ \cref{BM:sov} passando al complementare;
	
	\cref{BM:sov} $\Rightarrow$ \cref{BM:soveq}, \cref{BM:soveq} $\Rightarrow$ \cref{BM:sot} come i precedenti;
	
	\cref{BM:sot} $+$ \cref{BM:sov} $\Rightarrow$ \cref{BM:int} per intersezione;
	
	\cref{BM:int} $\Rightarrow$ \cref{BM:mis} perché in $\R$ gli aperti sono unione numerabile di intervalli aperti (o semirette aperte).
\end{proof}

\begin{proposition}
	\label{CounterImgMis}
	Dato uno spazio di misura $(X,\A,\mu)$, e data $f:X\rightarrow \R$ una funzione misurabile, la famiglia di insiemi
	\[
		\mathcal{E} = \{ E\subseteq \R : f^{-1}(E)\in \A \}
	\]
	è una \sigalg{}.
\end{proposition}
\begin{proof}
	Unione numerabile: siano $\{E_n\}_{n\in \N}\subseteq \mathcal{E}$ e sia $E = \cup_{n\in \N}E_n$,
	allora $f^{-1}(E)=f^{-1}\left(\cup_{n\in \N}E_n\right) = \cup_{n\in \N}f^{-1}(E_n)\in \A \Rightarrow E \in \mathcal{E}$;
	
	Complementare: sia $E\in \mathcal{E}$, allora $f^{-1}(^{c}E)= ^{c}f^{-1}(E)\in \A \Rightarrow ^{c}E \in \mathcal{E}$.
\end{proof}
