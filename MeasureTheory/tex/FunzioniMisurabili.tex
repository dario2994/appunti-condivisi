\section{Funzioni misurabili}

In questa sezione daremo la definizione di funzione misurabile e dimostreremo alcuni fatti basilari su di esse.

\begin{definition}[Funzione misurabile]
	Dato uno spazio di misura $(X,\A,\mu)$, una funzione $f:X\rightarrow \R$ si dice misurabile se
	$\forall A \subseteq \R$ aperto si ha $f^{-1}(A)\in \A$.
\end{definition}

\begin{proposition}
	\label{BasicMis}
	Dato uno spazio di misura $(X,\A,\mu)$, sia $f:X\rightarrow \R$ una funzione, sono equivalenti i seguenti fatti:
	\begin{enumerate}[label=(\arabic*),ref=(\arabic*)]
		\item $f$ è misurabile; \label{BM:mis}
		\item $\{f<a\}\in \A \quad \forall a\in \R$; \label{BM:sot}
		\item $\{f\leq a\}\in \A \quad \forall a\in \R$; \label{BM:soteq}
		\item $\{f>a\}\in \A \quad \forall a\in \R$; \label{BM:sov}
		\item $\{f\geq a\}\in \A \quad \forall a\in \R$;  \label{BM:soveq}
		\item $\{a<f<b\}\in \A \quad \forall a,b\in \overline{\R}$; \label{BM:int}
	\end{enumerate}
\end{proposition}
\begin{proof}
	Mostriamo a catena tutte le implicazioni:
	\begin{description}
	\item[\ImplicationProof{BM:mis}{BM:sot}] per definizione;
	\item[\ImplicationProof{BM:sot}{BM:soteq}] sfruttando il fatto che $\A$ è \sigalg{} e che
	$(-\infty,a]=\cap_{n\in \N}(-\infty,a+\frac{1}{n})$;
	\item[\ImplicationProof{BM:soteq}{BM:sov}] passando al complementare;
	\item[\ImplicationProof{BM:sov}{BM:soveq}] analogamente a \ImplicationProof{BM:sot}{BM:soteq}; 
	\item[\ImplicationProof{BM:soveq}{BM:sot}] analogamente a \ImplicationProof{BM:soteq}{BM:sov};
	\item[$\text{\ref{BM:sot}}\ +\ \text{\ref{BM:sov}}\implies\text{\ref{BM:int}}$] per intersezione;
	\item[\ImplicationProof{BM:int}{BM:mis}] perché in $\R$ gli aperti sono unione numerabile di intervalli aperti (o semirette aperte).
	\end{description}
\end{proof}

\begin{proposition}
	\label{CounterImgMis}
	Dato uno spazio di misura $(X,\A,\mu)$, e data $f:X\rightarrow \R$ una funzione misurabile, la famiglia di insiemi
	\[
		\mathcal{E} = \{ E\subseteq \R : f^{-1}(E)\in \A \}
	\]
	è una \sigalg{}.
\end{proposition}
\begin{proof}
	Verifichiamo che $\mathcal E$ è stabile per unioni numerabili e passaggio al complementare.
	
	Fissati $\{E_n\}_{n\in \N}\subseteq \mathcal{E}$ sia $E = \cup_{n\in \N}E_n$, vale:
	\begin{equation*}
		f^{-1}(E)=f^{-1}\left(\cup_{n\in \N}E_n\right) = \cup_{n\in \N}f^{-1}(E_n)\in \A \implies E \in \mathcal{E}
	\end{equation*}
	e questo dimostra la stabilità per unione numerabile.
	
	Per quanto riguard il passaggio al complementare, fissato $E\in \mathcal{E}$, risulta:
	\begin{equation*}
		f^{-1}(E^\mathsf{c})= f^{-1}(E)^\mathsf{c} \in \A \implies E^\mathsf{c} \in \mathcal{E}
	\end{equation*}
\end{proof}

\begin{proposition}
	\label{AlgMis}
	Dato uno spazio di misura $(X,\A,\mu)$, sia $\mathcal{M}=\{f:X\rightarrow \R : f\ \grave{e} \ misurabile\}$.
	Allora $\mathcal{M}$ è un'algebra nel senso che valgono le seguenti:
	\begin{enumerate}[label=(\arabic*),ref=(\arabic*)]
		\item $f,g\in \mathcal{M} \Rightarrow f+g\in \mathcal{M}$; \label{AlM:sum}
		\item $f\in \mathcal{M}, \lambda \in \R \Rightarrow \lambda f\in \mathcal{M}$. \label{AlM:sca}
		\item $f,g\in \mathcal{M} \Rightarrow fg\in \mathcal{M}$. \label{AlM:pro}
	\end{enumerate}
\end{proposition}
\begin{proof}
	Mostriamo per ogni punto che vale la proposizione \ref{BM:sov} in \cref{BasicMis} (che come lì mostrato, equivale alla misurabilità).
	\begin{description}
	\item[\ref{AlM:sum}] 
	\[
		\{f+g>a\}=\bigcup_{q\in \Q}\left(\{f>q\}\cap\{g>a-q\}\right)\in \A
	\]
	\item[\ref{AlM:sca}]
	\[
		\{\lambda f>a\}=\left\{f>\frac{a}{\lambda}\right\}\in \A
	\]
	\item[\ref{AlM:pro}] Per ogni funzione $h$ possiamo scrivere la parte positiva $h^+ = \max\{h,0\}$ e la parte negativa $h^- = \max\{0,-h\}$,
	così $h^+,h^- \geq 0$ e $h = h^+ - h^-$. 
	
	Ora scomponiamo $f=f^+ - f^-$, $g=g^+- g^-$, quindi la funzione prodotto $fg$ si scrive come una qualche combinazione di prodotti di funzioni non negative. Grazie ai punti \ref{AlM:sum} e \ref{AlM:sca} e a questa osservazione ci basta mostrare il caso in cui $f,g\geq0$:
	\[
		\{fg>a\}=\left\{\begin{array}{ll}
			\R &\qquad se\ a<0;\\
			\{f=0\}\cup\{g=0\} &\qquad se\ a=0;\\
			\bigcup_{q\in \Q^+}\left(\{f>q\}\cap\left\{g>\frac{a}{q}\right\}\right)\in \A &\qquad se\ a>0.
		\end{array}\right.
	\]
	\end{description}
\end{proof}

\begin{remark}
	\label{CarMis}
	È facile vedere che le funzioni caratteristiche degli insiemi misurabili sono misurabili.
\end{remark}
\begin{proof}
	Basta osservare che se $A\in \A$ è misurabile, $\{ \chi_A > a\}$ può valere solo $\emptyset$, $A$, $\R$ (tutti e 3 misurabili) a seconda che
	$a\geq 1$, $a\geq 0$ oppure $a < 0$ rispettivamente.
\end{proof}

\begin{definition}
	Una funzione $f:X \rightarrow \R$ con dominio lo spazio di misura $(X,\A,\mu)$ si dice semplice se è combinazione lineare di
	funzioni caratteristiche di insiemi misurabili.
\end{definition}
\begin{remark}
	È immediato che le funzioni semplici sono misurabili.
\end{remark}
\begin{proof}
	Discende da \cref{CarMis} e \cref{AlgMis}.
\end{proof}


\begin{proposition}
	Sia $\{f_n\}_{n\in \N}$ una famiglia di funzioni misurabili definite dallo spazio di misura $(X,\A,\mu)$ a $\R$.
	Allora $F=\sup\{f_n:n\in \N\}$ è misurabile.
\end{proposition}
\begin{proof}
	Consideriamo il sovralivello della funzione $F$: $\{F>a\}=\bigcup_{n\in \N}\{f_n>a\}$, ma allora $\{F>a\}\in \A$ per le proprietà 
	di chiusura della \sigalg.
\end{proof}

\begin{remark}
	\label{LimMis}
	Da quest'ultima proposizione, segue immediatamente che anche $\inf$ di una famiglia numerabile di misurabili è misurabile,
	quindi anche $\limsup$ e $\liminf$ e come ultima conseguenza che il limite puntuale di di una successione di misurabili è misurabile.
\end{remark}
\begin{proof}
	Per l'$\inf$ basta notare che $\inf\{f_n\}=-\sup\{-f_n\}$;
	
	per definizione, $\limsup\{f_n\}$ e $\liminf\{f_n\}$ sono rispettivamente $\lim_n\{\sup\{f_n\}\}=\inf\{\sup\{f_n\}\}$ e
	$\lim_n\{\inf\{f_n\}\}=\sup\{\inf\{f_n\}\}$, quindi sono funzioni misurabili;
	
	infine se esiste il limite $\lim_n\{f_n\}$ allora $\liminf\{f_n\}=\limsup\{f_n\}=\lim_n\{f_n\}$, pertanto è misurabile.
\end{proof}

\begin{proposition}
	\label{LimSemMis}
	Sia $f:X \rightarrow \R$ una funzione con dominio lo spazio di misura $(X,\A,\mu)$.
	Allora $f$ è misurabile se e solo se esiste una successione di funzioni semplici $\phi_n$ che converge puntualmente a $f$.
\end{proposition}
\begin{proof}
	Il se è mostrato in \cref{LimMis}.
	
	Per il solo se facciamo vedere che la seguente successione converge puntualmente a $f$:
	\[
		\phi_n(x) =
		\left\{ \begin{array}{ll}
			n &\qquad se\ f(x)>n;\\
			\frac{k}{2^n} &\qquad se\ \frac{k}{2^n}<f(x)\leq \frac{k+1}{2^n} \qquad k=-n2^n,-n2^n+1,\dots,n2^n;\\%
%			\frac{k}{2^n} &\qquad se\ \frac{k-1}{2^n}<f(x)\leq \frac{k}{2^n} \qquad k=-n2^n,\dots,0;\\
			-n &\qquad se\ f(x)\leq-n;
		\end{array} \right.\ .
	\]
	Prima di tutto abbiamo 
	\[\phi_n=
		n\chi_{\{f>n\}}+
		\sum_{k=-n2^n}^{n2^n}\frac{k}{2^n}\chi_{\left\{ \frac{k}{2^n}<f\leq \frac{k+1}{2^n} \right\}}
		-n\chi_{\{f<-n\}},
	\]
	che mostra che le $\phi_n$ sono funzioni semplici.
	
	Per mostrare la convergenza puntuale abbiamo che $|f(x)-\phi_n(x)|\leq \frac{1}{2^n}$ definitivamente, cioè $\forall n\geq f(x)$,
	quindi $\phi_n(x)\rightarrow f(x)$, $\forall x\in X$.
\end{proof}

