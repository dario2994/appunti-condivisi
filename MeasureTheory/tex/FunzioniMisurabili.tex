\section{Funzioni misurabili}
In questa sezione daremo la definizione di funzione misurabile e dimostreremo alcuni fatti basilari su di esse.

La teoria delle funzioni misurabili, oltre ad essere strettamente necessaria per la successiva teoria dell'integrazione, ci permetterà di dimostrare che i misurabili secondo Lebesgue, introdotti nella sezione precedente, non coincidono con i Boreliani usando strumenti propri della teoria della misura.
Questo fatto lo dimostriamo però solo nel caso unidimensionale sia perché il risultato è già stato dimostrato, sia perché la dimostrazione si adatta facilmente in dimensione maggiore e sia perché troviamo importante rendere chiara l'idea piuttosto che celarla in una notazione troppo pesante.

\begin{definition}[Retta reale estesa]
	Indichiamo con $\Rbar$ l'insieme $\R\cup\{-\infty,+\infty\}$ munito della topologia che ha come base tutti gli aperti tipici di $\R$ e gli insiemi del tipo $\{-\infty\}\cup\oo{-\infty}a$ e $\{+\infty\}\cup\oo{a}{+\infty}$ dove $a$ è un numero reale. 
	Quest'ultima categoria di insiemi aperti verrà indicata tramite la notazione $\co{-\infty}a$ e $\oc a{+\infty}$.
\end{definition}


\begin{definition}[Funzione misurabile]
	Dato uno spazio di misura $(X,\A,\mu)$, una funzione $f:X\rightarrow \Rbar$ si dice misurabile se
	$\forall A \subseteq \Rbar$ aperto si ha $f^{-1}(A)\in \A$.
\end{definition}

\begin{proposition}\label{prop:BasicMis}
	Dato uno spazio di misura $(X,\A,\mu)$, sia $f:X\rightarrow \Rbar$ una funzione, sono equivalenti i seguenti fatti
	\footnote{Qui introduciamo la notazione per i \textit{sovralivelli} di una funzione che useremo in tutti gli appunti:
		data una funzione $f$ di codominio reale e un certo reale $k$ indichiamo con $\{f>k\}$ l'insieme
		$\{x:f(x)>k\}=f^{-1}((k,+\infty])$; stessa notazione verrà usata anche per il sottolivello.}:
	\begin{enumerate}[label=(\arabic*),ref=(\arabic*)]
		\item $f$ è misurabile; \label{BM:mis}
		\item $\{f<a\}\in \A \quad \forall a\in \Rbar$; \label{BM:sot}
		\item $\{f\le a\}\in \A \quad \forall a\in \Rbar$; \label{BM:soteq}
		\item $\{f>a\}\in \A \quad \forall a\in \Rbar$; \label{BM:sov}
		\item $\{f\ge a\}\in \A \quad \forall a\in \Rbar$;  \label{BM:soveq}
		\item $\{a<f<b\}\in \A \quad \forall a,b\in \Rbar$. \label{BM:int}
	\end{enumerate}
\end{proposition}
\begin{proof}
	Sfruttando le proprietà di $\A$ come \sigalg, mostriamo a catena tutte le implicazioni:
	\begin{description}
	\item[\ImplicationProof{BM:mis}{BM:sot}] per definizione di misurabile, $\{f<a\}=f^{-1}\left(\co{-\infty}{a}\right)\in \A$,
		perché $\co{-\infty}{a}$ è aperto;
	\item[\ImplicationProof{BM:sot}{BM:soteq}] poiché $\cc{-\infty}{a}=\bigcap_{n\in \N}\co{-\infty}{a+\frac{1}{n}}$, allora otteniamo
		\begin{equation*}
			\{f\le a\}=f^{-1}\left(\cc{-\infty}{a}\right)=\bigcap_{n\in \N}f^{-1}\left(\co{-\infty}{a+\frac{1}{n}}\right)\in \A	
		\end{equation*}

	\item[\ImplicationProof{BM:soteq}{BM:sov}] passando al complementare, $\{f>a\}^\mathsf{c}=\{f\le a\}\in \A$;
	\item[\ImplicationProof{BM:sov}{BM:soveq}] analogamente a \ImplicationProof{BM:sot}{BM:soteq}; 
	\item[\ImplicationProof{BM:soveq}{BM:sot}] analogamente a \ImplicationProof{BM:soteq}{BM:sov};
	\item[$\text{\ref{BM:sot}}\ +\ \text{\ref{BM:sov}}\implies\text{\ref{BM:int}}$] perché
		$\{a<f<b\}=\{a<f\}\cap\{f<b\}\in \A$;
	\item[\ImplicationProof{BM:int}{BM:mis}] dato che un aperto $A\subseteq\Rbar$ si può scrivere come
		$A=\bigcup_{n\in \N}A_n$ dove ciascun $A_n$ è un intervallo aperto di $\Rbar$ (o una semiretta aperta),
		allora $f^{-1}(A)=\bigcup_{n\in \N}f^{-1}(A_n)\in \A$, dove abbiamo sfruttato che per il punto \ref{BM:int} $f^{-1}(A_n)\in \A\ \ \forall n$.
	\end{description}
\end{proof}

\begin{proposition}\label{prop:CounterImgMis}
	Dato uno spazio di misura $(X,\A,\mu)$, e data $f:X\rightarrow \Rbar$ una funzione misurabile, la famiglia di insiemi
	\[
		\mathcal{E} = \{ E\subseteq \Rbar : f^{-1}(E)\in \A \}
	\]
	è una \sigalg{} ed inoltre contiene i Boreliani.
\end{proposition}
\begin{proof}
	Verifichiamo che $\mathcal E$ è stabile per unioni numerabili e passaggio al complementare.
	
	Fissati $\{E_n\}_{n\in \N}\subseteq \mathcal{E}$ sia $E = \cup_{n\in \N}E_n$, vale:
	\begin{equation*}
		f^{-1}(E)=f^{-1}\left(\cup_{n\in \N}E_n\right) = \cup_{n\in \N}f^{-1}(E_n)\in \A \implies E \in \mathcal{E}
	\end{equation*}
	dove l'appartenenza ad $\A$ si ha per le proprietà di \sigalg{}, e questo dimostra la stabilità per unione numerabile.
	
	Per quanto riguarda il passaggio al complementare, fissato $E\in \mathcal{E}$, risulta:
	\begin{equation*}
		f^{-1}(E^\mathsf{c})= f^{-1}(E)^\mathsf{c} \in \A \implies E^\mathsf{c} \in \mathcal{E}.
	\end{equation*}
	
	Inoltre $\mathcal E$ contiene gli aperti per definizione di funzione misurabile, da cui, per quanto appena dimostrato, contiene la \sigalg{} generata da questi, cioè i Boreliani.
\end{proof}

\begin{definition}\label{def:FpiuFmeno}
	Sia $f:X\to\Rbar$ una funzione misurabile su uno spazio di misura $(X,\A,\mu)$. Definiamo $f^+ = \max\{f,0\}$ e $f^- = \max\{-f,0\}$.
\end{definition}
\begin{remark}\label{nota:ProprietaFpiuFmeno}
	Data una funzione $f$ misurabile, vale che $f^+,f^-$ sono misurabili. Inoltre $f^+,f^-\ge 0$ e $f=f^+-f^-$. 
\end{remark}
\begin{proof}
	Abbiamo che per ogni $a\in\Rbar$ vale
	\begin{equation*}
		\{f^+>a\}=\left\{\begin{array}{ll}
			\{f>a\}\in\A &\text{se $a>0$}\\
			X\in\A &\text{se $a\le 0$}
	\end{array}\right.
	\end{equation*}
	Quindi, per la \ref{BM:sov} della \cref{prop:BasicMis}, $f^+$ è misurabile. Analogamente si dimostra che $f^-$ è misurabile.
\end{proof}


\begin{proposition}\label{prop:AlgMis}
	Dato uno spazio di misura $(X,\A,\mu)$, sia $\mathcal{M}$ l'insieme delle funzioni misurabili da $X$ in $\Rbar$.
	Allora $\mathcal{M}$ è un'algebra nel senso che, dove sono definite\footnote{Nel definire le operazioni algebriche su $\mathcal{M}$ adottiamo le seguenti convenzioni: la somma è definita se non accade che entrambe $f$ e $-g$ siano $\pm\infty$, per la moltiplicazione $0\cdot \infty = 0$.} ,
	valgono le seguenti:
	\begin{enumerate}[label=(\arabic*),ref=(\arabic*)]
		\item $f,g\in \mathcal{M} \Rightarrow f+g\in \mathcal{M}$; \label{AlM:sum}
		\item $f\in \mathcal{M}, \lambda \in \R \Rightarrow \lambda f\in \mathcal{M}$; \label{AlM:sca}
		\item $f,g\in \mathcal{M} \Rightarrow fg\in \mathcal{M}$. \label{AlM:pro}
	\end{enumerate}
\end{proposition}

\begin{proof}
	Mostriamo per ogni punto che vale la proposizione \ref{BM:sov} nella \cref{prop:BasicMis} (che come lì mostrato, equivale alla misurabilità),
	distinguendo vari casi di $a\in \Rbar$.
	\begin{description}
	\item[\ref{AlM:sum}]
	\[
		\{f+g>a\}=\left\{\begin{array}{ll}
			\{f\ge -\infty\}\cap\{g\ge -\infty\}\in \A &\qquad \text{se $a=-\infty$}\puntovirgola\\
			\bigcup_{q\in \Q}\left(\{f>q\}\cap\{g>a-q\}\right)\in \A &\qquad \text{se $a\in \R$}\puntovirgola\\
			\{f=+\infty\}\cup\{g=+\infty\}\in \A &\qquad \text{se $a=+\infty$}\punto
		\end{array}\right.
	\]
	\item[\ref{AlM:sca}]
	\[
		\{\lambda f>a\}=\left\{\begin{array}{ll}
			\left\{f<\frac{a}{\lambda}\right\}\in \A &\qquad \text{se $\lambda<0$}\puntovirgola \\
			X \in \A &\qquad \text{se $\lambda=0$ e $a< 0$}\puntovirgola\\
			\emptyset \in \A &\qquad \text{se $\lambda=0$ e $a\ge 0$}\puntovirgola\\
			\left\{f>\frac{a}{\lambda}\right\}\in \A &\qquad \text{se $\lambda>0$}\punto
		\end{array}\right.
	\]
	\item[\ref{AlM:pro}] Scomponiamo $f=f^+ - f^-$, $g=g^+- g^-$, quindi la funzione prodotto $fg$ si scrive come una qualche combinazione di prodotti di funzioni misurabili non negative (abbiamo già dimostrato nella \cref{nota:ProprietaFpiuFmeno} le proprietà di $f^+,f^-,g^+,g^-$ che ci servono). Grazie ai punti \ref{AlM:sum} e \ref{AlM:sca} e a questa osservazione ci basta mostrare il caso in cui $f,g\ge0$:
	\[
		\{fg>a\}=\left\{\begin{array}{ll}
			X\in \A &\enspace \text{se $a<0$}\puntovirgola\\
			\{f>0\}\cup\{g>0\}\in \A &\enspace \text{se $a=0$}\puntovirgola\\
			\bigcup_{q\in \Q^+}\left(\{q<f<+\infty\}\cap\left\{\frac{a}{q}<g<+\infty \right\} \right)\in \A &\enspace \text{se $0<a<+\infty$}\puntovirgola\\
			(\{f=+\infty\}\cap\{g>0\})\cup (\{f>0\}\cap\{g=+\infty\})\in \A &\enspace \text{se $a=+\infty$}\punto
		\end{array}\right.
	\]
	\end{description}
\end{proof}

\begin{remark}\label{nota:CarMis}
	È facile vedere che le funzioni caratteristiche degli insiemi misurabili sono misurabili.
\end{remark}
\begin{proof}
	Basta osservare che se $A\in \A$ è misurabile, $\{ \chi_A > a\}$ può essere solo $\emptyset$, $A$, $X$ (tutti e tre misurabili) a seconda che
	$a\ge 1$, $0\le a< 1$ oppure $a < 0$ rispettivamente.
\end{proof}

\begin{remark}\label{nota:ContinueMisurabili}
	Sia $X$ un insieme dotato sia di una topologia che di una misura su di esso, tali che in particolare la \sigalg{} dei misurabili contenga tutti gli aperti.
	Data una funzione $f:X\to\R$, se $f$ è continua è anche misurabile.
\end{remark}
\begin{proof}
	Basta notare che la controimmagine di un aperto è un aperto per continuità, ma gli aperti sono misurabili per ipotesi e di conseguenza la funzione è misurabile.
\end{proof}

\begin{remark}\label{nota:MonotoneMisurabili}
	Fissato $A\subseteq \R$ misurabile, ogni funzione $f:A\to\R$ monotona è misurabile, munendo $\R$ della misura di Lebesgue definita nella precedente sezione.
\end{remark}
\begin{proof}
	È sufficiente notare che la controimmagine di un intervallo\footnote{Definiamo, unicamente in questa dimostrazione, un intervallo come un generico sottoinsieme connesso di $\R$.} è a sua volta un intervallo intersecato $A$ poiché la funzione è monotona, perciò applicando la \cref{prop:BasicMis} ricaviamo che la funzione è misurabile visto che gli intervalli sono misurabili secondo Lebesgue (e lo è la loro interesezione con $A$, per la \cref{nota:RiduzioneMisura}).
\end{proof}

\begin{proposition}\label{prop:BorelianiNonMisurabili2}
	I Boreliani di $\R$ non coincidono con l'insieme $\M_1$ dei misurabili.
\end{proposition}
\begin{proof}
	Definiamo la funzione $f:\co{0}{1}\to \co{0}{1}$ in modo che $f(x)$ sia il numero che corrisponde alla lettura in base $3$ della scrittura in base $2$ di $x$.
	Poiché alcuni numeri hanno due scritture in base $2$, sceglieremo sempre quella che non ha una coda infinita di $1$.
	
	Qui di seguito un diagramma che mostra la definizione di $f$:
	\begin{equation*}
		x=\>\stackrel{\text{Scrittura in base $2$ di $x$}}{\overline{0.x_1x_2x_3\cdots}_2} \>  \longmapsto
		\> \stackrel{\text{Lettura in base $3$ di $x$ in base $2$}}{\overline{0.x_1x_2x_3\cdots}_3}\>=f(x)
	\end{equation*}

	La funzione $f$ appena definita è strettamente crescente, poiché lo è la funzione che associa ad un numero $x$ la sua lettura in qualche base (dove le sequenze di cifre sono ordinate lessicograficamente). Allora per la \cref{nota:MonotoneMisurabili} $f$ è misurabile.
	
	Inoltre l'immagine di $f$ è un insieme trascurabile, infatti questa coincide con l'insieme dei numeri tra $0$ e $1$ che si scrivono unicamente usando cifre $0,1$ in base $3$ ed è facile dimostrare che questo è trascurabile (esercizio per il lettore molto simile all'\cref{ex:CantorTrascurabile}).
	
	Per il \cref{thm:InsiemeVitali} esiste $A\subseteq \co{0}{1}$ che non è misurabile.
	Sia $B=f(A)$.
	
	Poiché $f$ è strettamente crescente è in particolare iniettiva e perciò $A=f^{-1}(B)$.
	Allora la \cref{prop:CounterImgMis} ci assicura che $B$ non appartiene ai Boreliani, altrimenti la sua controimmagine sarebbe misurabile. 
	Infine $B$ è sottoinsieme di $f\left(\co01\right)$, che è trascurabile, perciò per la completezza della misura di Lebesgue $B$ è misurabile.
	
	Allora, unendo quanto detto, abbiamo che $B$ è un misurabile non Boreliano come voluto.
\end{proof}



\begin{definition}
	Una funzione $\simp:X \rightarrow \Rbar$ con dominio lo spazio di misura $(X,\A,\mu)$ si dice semplice se è combinazione lineare di
	funzioni caratteristiche di insiemi misurabili.
\end{definition}
\begin{remark}
	È immediato che le funzioni semplici sono misurabili.
\end{remark}
\begin{proof}
	Discende dalla \cref{nota:CarMis} e dalla \cref{prop:AlgMis}.
\end{proof}


\begin{proposition}\label{prop:SupDiMisurabili}
	Sia $\{f_n\}_{n\in \N}$ una famiglia di funzioni misurabili definite dallo spazio di misura $(X,\A,\mu)$ a $\Rbar$.
	Allora $F:X\rightarrow \Rbar$ definita da $F(x)=\sup\{f_n(x):n\in \N\}$ è misurabile.
\end{proposition}
\begin{proof}
	Consideriamo il sovralivello della funzione $F$: $\{F>a\}=\bigcup_{n\in \N}\{f_n>a\}$, allora è evidente che $\{F>a\}\in \A$ per le proprietà 
	di chiusura della \sigalg.
\end{proof}

\begin{remark}\label{nota:LimMis}
	Quest'ultima proposizione ha alcune notevoli conseguenze immediate:
	\begin{enumerate}
		\item $\inf$ di una famiglia numerabile di misurabili è misurabile;\label{LM:inf}
		\item $\limsup$ e $\liminf$ di una famiglia numerabile di misurabili sono misurabili;\label{LM:lim_infsup}
		\item limite puntuale di funzioni misurabili è misurabile.\label{LM:lim}
	\end{enumerate}
\end{remark}
\begin{proof}
	\begin{description}
		\item[\ref{LM:inf}] Per l'$\inf$ basta notare che $\inf\{f_n\}=-\sup\{-f_n\}$, quindi è misurabile per la \cref{prop:SupDiMisurabili};
		\item[\ref{LM:lim_infsup}] per definizione, $\limsup\{f_n\}$ e $\liminf\{f_n\}$ sono rispettivamente
			$\lim_n\{\sup\{f_n\}\}=\inf\{\sup\{f_n\}\}$ e
			$\lim_n\{\inf\{f_n\}\}=\sup\{\inf\{f_n\}\}$, quindi sono funzioni misurabili per il punto precedente;
		\item[\ref{LM:lim}] infine se esiste il limite $\lim_n\{f_n\}$ allora
			$\liminf\{f_n\}=\limsup\{f_n\}=\lim_n\{f_n\}$, pertanto è misurabile per il punto precedente.
	\end{description}
\end{proof}

\begin{proposition}\label{prop:LimSemMis}
	Sia $f:X \rightarrow \Rbar$ una funzione con dominio lo spazio di misura $(X,\A,\mu)$.
	Allora $f$ è misurabile se e solo se esiste una successione di funzioni semplici $\simp_n$ che converge puntualmente a $f$.
\end{proposition}
\begin{proof}
	Il se è mostrato nella \cref{nota:LimMis}.
	
	Per il solo se facciamo vedere che la seguente successione converge puntualmente a $f$:
	\[
		\simp_n(x) =
		\left\{ \begin{array}{ll}
			n &\qquad se\ f(x)>n;\\
			\frac{k}{2^n} &\qquad se\ \frac{k}{2^n}<f(x)\le \frac{k+1}{2^n} \qquad k=-n2^n,-n2^n+1,\dots,n2^n;\\
			-n &\qquad se\ f(x)\le-n;
		\end{array} \right.\ .
	\]
	Prima di tutto abbiamo 
	\[\simp_n=
		n\chi_{\{f>n\}}+
		\sum_{k=-n2^n}^{n2^n}\frac{k}{2^n}\chi_{\left\{ \frac{k}{2^n}<f\le \frac{k+1}{2^n} \right\}}
		-n\chi_{\{f<-n\}},
	\]
	che mostra che le $\simp_n$ sono funzioni semplici.
	
	Per mostrare la convergenza puntuale distinguiamo $f(x)$ a seconda che sia un numero finito o meno:
	nel primo caso abbiamo che $|f(x)-\simp_n(x)|\le \frac{1}{2^n}$ definitivamente, cioè $\forall n\ge |f(x)|$,
	nel secondo caso $\simp_n(x)=\pm n\rightarrow \pm\infty = f(x)$;
	quindi $\simp_n(x)\rightarrow f(x)$, $\forall x\in X$.
\end{proof}

\begin{definition}
	Una funzione $f$ definita su $(X,\A,\mu)$ e a valori in $\Rbar$ si dice positiva se assume solo valori maggiori o uguali a 0.
\end{definition}


\begin{corollary}\label{cor:LimSemCrescMis}
	Sia $f:X\rightarrow \Rbar$ misurabile e positiva su $(X,\A,\mu)$ spazio di misura, allora esiste una successione crescente di funzioni semplici e positive $(\simp_n)$ che converge puntualmente a $f$.
\end{corollary}
\begin{proof}
	Costruendo le $\simp_n$ come nella \cref{prop:LimSemMis}, se $f$ è positiva otteniamo facilmente che le $\simp_n$ sono anche crescenti, da cui la tesi.
\end{proof}

\begin{theorem}\label{thm:ChiusuraMonotonaFunzioni}
	Se $\mathcal F$ è una famiglia di funzioni da $X$ a $\Rbar$, dove $(X,\A,\mu)$ è uno spazio di misura, tale che
	\begin{itemize}
	 \item $\mathcal F$ è uno spazio vettoriale;
	 \item $\mathcal F$ contiene le funzioni indicatrici di ogni insieme $A\in\A$;
	 \item se $(f_n)\subseteq \mathcal F$ è una successione di funzioni misurabili positive che converge crescentemente a $f$, allora $f\in \mathcal F$;
	\end{itemize}
	allora $\mathcal F$ contiene tutte le funzioni misurabili.
\end{theorem}
\begin{proof}
	Per prima cosa notiamo che $\mathcal F$ contiene le funzioni semplici: essendo $\mathcal F$ uno spazio vettoriale, contiene le combinazioni
	lineari delle funzioni caratteristiche, cioè le funzioni semplici.
	
	Notiamo allora che per il \cref{cor:LimSemCrescMis}, $\mathcal F$ contiene le funzioni misurabili positive. Allora, data $f$ misurabile,
	$f = f_+-f_-$ quindi è contenuta in $\mathcal F$, ancora per le proprietà di spazio vettoriale, poiché entrambe $f_+,f_-$ sono
	positive e misurabili per la \cref{nota:ProprietaFpiuFmeno}.
\end{proof}
