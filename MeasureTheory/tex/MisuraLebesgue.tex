\section{Misura di Lebesgue}
Ora applicheremo i risultati astratti ottenuti nelle due precedenti sezioni al caso più tangibile della retta reale.

Definiremo la misura di Lebesgue e, oltre a chiarire come mai questa sia la misura più naturale su $\R$, studieremo i misurabili secondo Lebesgue mostrando sia che non coincidono con la \sigalg{} dei boreliani (cioè la \sigalg{} generata dagli aperti) sia che non coincidono con le parti di $\R$.

\begin{definition}
	I Boreliani sono la \sigalg{} generata dai sottoinsiemi aperti della retta reale.
\end{definition}

\newcommand{\so}[2]{\ensuremath{[\,#1,\,#2\,)}}
\newcommand{\oo}[2]{\ensuremath{(\,#1,\,#2\,)}}
\newcommand{\cc}[2]{\ensuremath{[\,#1,\,#2\,]}}

\begin{theorem}\label{LebesguePremisura}
	Sia $\S$ l'insieme degli intervalli semiaperti a destra di $\R$, cioè i sottoinsiemi della retta reale della forma $[a,b)$ con $a,b\in\R$ \footnote{Se $a\ge b$ con la scrittura $[a,b)$ si intenderà l'insieme vuoto.}.
	
	Definendo la funzione $\mu:\S\to\Rbar$ in modo che $\mu\left([a,b)\right)=b-a$, la terna $(\R,\S,\mu)$ risulta essere uno spazio di misura elementare.
\end{theorem}
\begin{proof}
	L'insieme vuoto appartiene ovviamente a $\S$.
	Inoltre, fissati $a,b,c,d\in\R$ valgono:
	\begin{align*}
		\so ab\cap \so cd &= \so {\max(a,c)}{\min(b,d)}\\
		\so ab\setminus \so cd&= \so ac\sqcup \so db
	\end{align*}
	e questo implica che $\S$ è un \semiring{}.
	
	Ora resta da dimostrare che $\mu$ sia \sigadd{} su $\S$.
	Consideriamo $\so {a_n}{b_n}=I_n\in \S$ una successione numerabile di elementi di $\S$ la cui unione \emph{disgiunta} dia $\so ab=I\in \S$.
	
	Riconducendoci ad un numero finito di intervalli, e sfruttando facili ragionamenti combinatorici, otteniamo:
	\begin{equation*}
		\sum_{n\le k} \mu(I_n)\le \mu(I) \implies \sum_{n\in\N}\mu(I_n)\le \mu(I)
	\end{equation*}
	
	Per ottenere la disuguaglianza opposta sfrutteremo la compattezza degli intervalli limitati di $\mathbb R$, in particolare la possibilità di estrarre ricoprimenti finiti di aperti a partire da ricoprimenti numerabili.
	
	Fissiamo $\epsilon>0$ arbitrario.
	
	Definiamo gli intervalli aperti $I'_n=\oo{a_n-\frac\epsilon{2^n}}{b_n}$ e notiamo che $I_n\subseteq I'_n$.
	Vale perciò che la successione di aperti $(I'_n)_{n\in\N}$ è un ricoprimento del compatto $\cc a{b-\epsilon}$ e di conseguenza esiste un insieme di indici finito $J$ tale che $(I'_n)_{n\in J}$ è un ricoprimento finito di $\cc a{b-\epsilon}$.
	
	Allora, ancora per facili motivazioni combinatoriche, abbiamo:
	\begin{equation*}
		\sum_{n\in J} b_n-\left(a_n-\frac\epsilon{2^n}\right)\ge b-\epsilon-a\implies
		\sum_{n\in\N} b_n-\left(a_n-\frac\epsilon{2^n}\right)\ge b-a-\epsilon \implies
		\sum_{n\in\N} \mu(I_n)\ge \mu(I)-2\epsilon
	\end{equation*}
	e visto che questo vale per ogni $\epsilon>0$ ne ricaviamo:
	\begin{equation*}
		\sum_{n\in\N}\mu(I_n)\ge \mu(I)
	\end{equation*}
	che insieme alla disuguaglianza opposta già dimostrata dimostra la \sigadd[ità] della funzione $\mu$.
	
	Quindi visto che $\S$ è un \semiring{} e $\mu$ una premisura su $\S$ concludo che $(\R,\S,\mu)$ è uno spazio di misura elementare.
\end{proof}

\begin{remark}
	Il \cref{LebesguePremisura} vale anche, con le opportune modifiche alla definizioni, in $\R^n$ piuttosto che in $\R$. La dimostrazione procede in modo del tutto analogo, solo che le ragioni combinatoriche che rendevano banale tutto non appena il numero di intervalli era finito divengono più sfuggenti e necessitano di dimostrazione. Tratteremo però la possibilità di porre una misura su $\R^n$ nelle sezioni successive.
\end{remark}

\begin{remark}\label{LebesguePremisuraSigFin}
	Lo spazio di premisura $(\R,\S,\mu)$ come definito nel \cref{LebesguePremisura} è \sigfin[o].
\end{remark}
\begin{proof}
	Mostriamo esplicitamente la \sigfin[ezza]:
	\begin{equation*}
		\R=\bigsqcup_{n\in\mathbb Z} \so{n}{n+1}
	\end{equation*}
\end{proof}

\begin{remark}\label{SigAlgUgualeBoreliani}
	La \sigalg{} generata da $\S$, come definito nell'enunciato di \cref{LebesguePremisura}, coincide con i Boreliani.
\end{remark}
\begin{proof}
	Fissati $a<b$ in $\R$ vale:
	\begin{equation*}
		\oo{a}{b}=\bigcup_{n\in\N}\so{a+\frac1{2^n}}b
	\end{equation*}
	e perciò otteniamo che gli intervalli aperti appartengono a $\sigma A(\S)$.
	Ma in $\R$ gli aperti sono sempre unione numerabile di intervalli aperti e di conseguenza gli aperti appartengono a $\sigma A(\S)$.
\end{proof}



\begin{definition}\label{LebesgueMisura}
	Dato lo spazio di misura elementare $(\R,\S,\mu)$ definito nel \cref{LebesguePremisura}, sia $\M$, che verrà chiamato l'insieme dei misurabili secondo Lebesgue, la relativa \sigalg{} di Caratheodory e $m_1:\M\to\Rpiu$ \footnote{L'$1$ al pedice di $m_1$, che verrà a volte omesso, sottolinea il fatto che la misura è quella riferita ad $\R^1$ e non $\R^n$.}, che verrà chiamata misura di Lebesgue, la misura associata, la cui esistenza ci è assicurata dal \cref{EstensioneCaratheodory}.
\end{definition}

\begin{remark}
	I Boreliani sono un sottoinsieme dei misurabili secondo Lebesgue.
\end{remark}
\begin{proof}
	Il \cref{EstensioneCaratheodory} mi assicura che la \sigalg{} di Caratheodory definita nella \cref{LebesgueMisura} contiene la \sigalg{} generata da $\S$ e perciò, sfruttando \cref{SigAlgUgualeBoreliani}, otteniamo i Boreliani sono misurabili secondo Lebesgue.
\end{proof}

\begin{remark}
	La misura di Lebesgue è l'unica misura che estenda la misura definita sugli intervalli semiaperti nel \cref{LebesguePremisura} ai Boreliani.
\end{remark}
\begin{proof}
	È una banale conseguenza di \cref{UnicitaCaratheodory}, che si può applicare ricordando \cref{LebesguePremisuraSigFin} e notando che la \sigalg{} generata dagli aperti sono i Boreliani come mostrato in \cref{SigAlgUgualeBoreliani}.
\end{proof}


