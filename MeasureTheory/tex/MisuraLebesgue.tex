\section{Misura di Lebesgue}
Ora applicheremo i risultati astratti ottenuti nelle due precedenti sezioni al caso più tangibile della retta reale.

Definiremo la misura di Lebesgue e, oltre a chiarire come mai questa sia la misura più naturale su $\R$, studieremo i misurabili secondo Lebesgue mostrando sia che non coincidono con la \sigalg{} dei boreliani (cioè la \sigalg{} generata dagli aperti) sia che non coincidono con le parti di $\R$.

\begin{definition}
	I Boreliani sono la \sigalg{} generata dai sottoinsiemi aperti della retta reale.
\end{definition}

\begin{theorem}
	Sia $\S$ l'insieme degli intervalli semiaperti a destra di $\R$, cioè i sottoinsiemi della retta reale della forma $[a,b)$ con $a<b$. 
	Definiamo inoltre la funzione $\mu:\S\to\Rbar$ in modo che $f\left([a,b)\right)=b-a$.
	
	Allora $(\R,\S,m)$ è uno spazio di misura elementare.
\end{theorem}
\begin{proof}
	TODO
\end{proof}