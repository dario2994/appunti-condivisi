\section{Misura di Lebesgue}
Ora applicheremo i risultati astratti ottenuti nelle due precedenti sezioni al caso più tangibile della retta reale e dello spazio vettoriale $\R^n$.

Definiremo la misura di Lebesgue e, oltre a chiarire come mai questa sia la misura più naturale su $\R^n$, studieremo i misurabili secondo Lebesgue mostrando sia che non coincidono con la \sigalg{} dei Boreliani (cioè la \sigalg{} generata dagli aperti) sia che non coincidono con le parti di $\R^n$.
In particolare, la distinzione tra Boreliani e misurabili verrà fatta nella sezione seguente, poichè la dimostrazione risulta naturale solo dopo aver introdotto la teoria delle funzioni misurabili

Nella prima parte di questa sezione, quella riguardante degli aspetti fondamentalmente combinatorici degli $n$-cubi di $\R^n$ lasceremo due dimostrazioni al lettore.
Queste vengono omesse in quanto, oltre ad essere molto pesanti notazionalmente e molto facili intuitivamente, non aggiungono nulla alla comprensione che si mira ad avere della misura di Lebesgue.

\begin{definition}\label{def:Boreliani}
	I Boreliani di $\R^n$ sono la \sigalg{} generata dai sottoinsiemi aperti di $\R^n$.
\end{definition}

\begin{definition}\label{def:LebesgueSemiaperti}
	Indicheremo con $\S_n\subseteq\mathcal P(\R^n)$ l'insieme dei parallelepipedi $n$-dimensionali semiaperti a destra\footnote{Se $a\ge b$ con la scrittura $\co{a}{b}$ si intenderà l'insieme vuoto.}:
	\begin{equation*}
		\S_n=\left\{\co{a_1}{b_1}\times\cdots\times\co{a_n}{b_n}:\ (a_i)_{1\le i\le n},(b_i)_{1\le i\le n}\subseteq \R\right\}
	\end{equation*}
	
	Inoltre, dato $S=\co{a_1}{b_1}\times\cdots\times\co{a_n}{b_n}$, definiamo $S^-_i=a_i$ e $S^+_i=b_i$.
\end{definition}

\begin{proposition}\label{prop:SpaccareUnioneSemiaperti}
	Dati $A,B\in\S_n$ esiste una famiglia finita $(S_i)_{i\in I}\subseteq\S_n$ di insiemi disgiunti la cui unione disgiunta dà $A\cup B$ e tale che per ogni $i\in I$ vale una, ed una sola, delle seguenti:
	\begin{align*}
		S_i\subseteq &A\setminus B\\
		S_i\subseteq &B\setminus A\\
		S_i\subseteq &A\cap B
	\end{align*}
\end{proposition}
\begin{proof}
	Lasciata al lettore.
\end{proof}


\begin{lemma}\label{lem:SemianelloSemiAperti}
	La famiglia $\S_n$ è un \semiring{}.
\end{lemma}
\begin{proof}
	Ovviamente $\emptyset\in\S_n$.
	Inoltre dati $A,B\in\S_n$ si ricava facilmente che
	\begin{equation*}
		A\cap B=\co{\max(A^-_1,B^-_1)}{\min(A^+_1,B^+_1)}\times\cdots\times\co{\max(A^-_n,B^-_n)}{\min(A^+_n,B^+_n)}\in\S_n
	\end{equation*}

	Mentre per la differenza $A\setminus B$ applichiamo la \cref{prop:SpaccareUnioneSemiaperti} sugli insiemi $A,B$ per ottenere la famiglia $(S_i)_{i\in I}\subseteq \S_n$ come descritta dall'enunciato.
	Ora basta considerare la famiglia
	\begin{equation*}
		\mathcal F=\left\{S_i:\ i\in I\wedge S_i\subseteq A\setminus B\right\}
	\end{equation*}
	e notare che, viste le proprietà che ha la famiglia degli $(S_i)_{i\in I}$, l'unione disgiunta degli elementi della famiglia $\mathcal F$ è proprio $A\setminus B$.
	
	Ma visto che sia intersezione che differenza si scrivono come unione disgiunta di elementi di $\S_n$ abbiamo dimostrato che $\S_n$ è un \semiring{}.
\end{proof}

\begin{definition}\label{def:LebesgueElementare}
	Indicherò con $m_n:\S_n\to\Rpiu$ la funzione che associa ad ogni parallelepipedo il suo volume $n$-dimensionale:
	\begin{equation*}
		m_n(S)=\prod_{i=1}^n\max(0,S^+_i-S^-_i)
	\end{equation*}
\end{definition}
\begin{remark}\label{nota:LebesgueElementareProprieta}
	La funzione di insiemi $m_n$ è invariante per traslazione, cioè risulta che per ogni $S\in\S_n$ e $v\in\R^n$ vale\footnote{Definiamo lo shift di un insieme come $A+k=\{a+k:\ a\in A\}$.}
	\begin{equation*}
		m_n(S)=m_n(S+v)
	\end{equation*}
	ed è anche $n$-omogenea, cioè per ogni $\lambda>0$ e $S\in\S_n$ vale\footnote{Definisco la moltiplicazione per scalare come $\lambda S=\{\lambda s:\ s\in S\}$}
	\begin{equation*}
		m_n(\lambda S)=\lambda^n m_n(S)
	\end{equation*}
\end{remark}
\begin{proof}
	Entrambe le proprietà sono di facile verifica:
	\begin{multline*}
		m_n(S+v)=m_n\left(\co{S^-_1+v_1}{S^+_1+v_1}\times\cdots\times\co{S^-_n+v_n}{S^+_n+v_n}\right)\\
		=\prod_{i=1}^n\left((S^+_i+v_i)-(S^-_i+v_i)\right)=\prod_{i=1}^n\left(S^+_i-S^-_i\right)=m_n(S)
	\end{multline*}
	
	\begin{multline*}
		m_n(\lambda S)=m_n\left(\co{\lambda S^-_1}{\lambda S^+_1}\times\cdots\times\co{\lambda S^-_n}{\lambda S^+_n}\right)\\
		=\prod_{i=1}^n\left(\lambda S^+_i-\lambda S^-_i\right)=\lambda^n\prod_{i=1}^n\left(S^+_i-S^-_i\right)=\lambda^n m_n(S)
	\end{multline*}
\end{proof}


\begin{lemma}\label{lem:LebesgueElementareFinita}
	Dati $(S_i)_{1\le i\le k},S\subseteq \S_n$, valgono le seguenti disuguaglianze:
	\begin{itemize}
		\item Se gli $S_i$ sono disgiunti e $\bigsqcup_{i=1}^k S_i\subseteq S$ allora risulta
		\begin{equation*}
			\sum_{i=1}^k m_n(S_i)\le m_n(S)
		\end{equation*}
		\item Se vale il contenimento $S\subseteq\bigcup_{i=1}^k S_i$ allora risulta
		\begin{equation*}
			\sum_{i=1}^k m_n(S_i)\ge m_n(S)
		\end{equation*}
	\end{itemize}
\end{lemma}
\begin{proof}
	Lasciata al lettore.
\end{proof}

\begin{definition}\label{def:AllargamentoSemiaperti}
	Fissato $\lambda>0$ definiamo $F_\lambda:\S_n\to\S_n$ come l'operatore che associa ad $S\in\S_n$ l'insieme nullo se $S=\emptyset$ e altrimenti
	\begin{equation*}
		F_\lambda(S)=\co{S^-_1-\epsilon}{S^+_1}\times\cdots\times\co{S^-_n-\epsilon}{S^+_n}
	\end{equation*}
	dove $\epsilon>0$ è definito come
	\begin{equation*}
		\epsilon=\min\left(1,\frac{\lambda}{2^n\cdot m_n(S)}\right)\cdot\min_{1\le i\le n}\{S^+_i-S^-_o\}
	\end{equation*}
\end{definition}
\begin{remark}\label{nota:ParteInternaAllargamento}
	Segue banalmente dalla definizione che $S\in\S_n$ è un sottoinsieme della parte interna di $F_\lambda(S)$ per ogni $\lambda>0$.
\end{remark}

\begin{proposition}\label{prop:MisuraAllargamento}
	Fissati $\lambda>0$ e $S\in\S_n$ vale la seguente stima:
	\begin{equation*}
		m_n(S)\le m_n(F_\lambda(S))\le m_n(S)+\lambda
	\end{equation*}
\end{proposition}
\begin{proof}
	La prima disuguaglianza è ovvia.
	
	La seconda è ovvia se $S$ è vuoto, quindi assumiamo $S\not=\emptyset$.
	
	Ponendo $\epsilon$ come nella \cref{def:AllargamentoSemiaperti}, risulta vero
	\begin{multline*}
		\frac{m_n(F_\lambda(S))}{m_n(S)}=\prod_{i=1}^n\frac{S^+_i-S^-_i+\epsilon}{S^+_i-S^-_i}=
		\prod_{i=1}^n\left(1+\frac{\epsilon}{S^+_i-S^-_i}\right)\\\le
		\prod_{i=1}^n\left(1+\min\left(1,\frac{\lambda}{2^n\cdot m_n(S)}\right)\right)\le
		1+2^n\frac{\lambda}{2^n\cdot m_n(S)}=1+\frac{\lambda}{m_n(S)}
	\end{multline*}
	che implica la tesi moltiplicando ambo i membri per $m_n(S)$.
\end{proof}

\begin{theorem}\label{LebesguePremisura}
	La terna $(\R^n,\S_n,m_n)$ è uno spazio di misura elementare.
\end{theorem}
\begin{proof}
	Per quanto mostrato nel \cref{lem:SemianelloSemiAperti} la famiglia $\S_n$ è un \semiring{}.
	
	Resta da dimostrare solo che $m_n$ è \sigadd{} su $\S_n$.
	Perciò fisso $(S_i)_{i\in\N}\subseteq \S_n$ disgiunti tali che la loro unione disgiunta da $S\in\S_n$.
	
	Applicando il \cref{lem:LebesgueElementareFinita} abbiamo facilmente
	\begin{equation*}
		\forall k\in\N:\ \sqcup_{i=1}^k S_i\subseteq S\implies \forall k\in\N:\ \sum_{i=1}^k m_n(S_i)\le m_n(S) \implies \sum_{i\in\N}m_n(S_i)\le m_n(S)
	\end{equation*}

	
	Per ottenere la disuguaglianza opposta, concludendo quindi la dimostrazione, sfrutteremo la compattezza dei chiusi e limitati di $\R^n$, in particolare la possibilità di estrarre ricoprimenti finiti di aperti a partire da ricoprimenti numerabili.
	
	Fissiamo $\epsilon>0$ arbitrario.
	
	Grazie alla \cref{nota:ParteInternaAllargamento} abbiamo che le parti interne degli elementi della successione $\left(F_{\frac\epsilon{2^i}}(S_i)\right)_{i\in\N}$ sono un ricoprimento del compatto $S$.
	Perciò esiste un insieme finito di indici $I\subseteq \N$ tale che $\left(F_{\frac\epsilon{2^i}}(S_i)\right)_{i\in I}$ è un ricoprimento finito di $S$.
	
	Sfruttando ancora il \cref{lem:LebesgueElementareFinita} e applicando \cref{prop:MisuraAllargamento} otteniamo
	\begin{equation*}
		\sum_{i\in\N}m_N(S_i)\ge \sum_{i\in I}m_N(S_i)\ge \sum_{i\in I}\left(m_n\left(F_{\frac\epsilon{2^i}}(S_i)\right)-\frac\epsilon{2^i}\right)
		\ge -\epsilon+\sum_{i\in I}m_n\left(F_{\frac\epsilon{2^i}}(S_i)\right)\ge m_n(S)-\epsilon
	\end{equation*}
	e visto che questo vale per ogni $\epsilon>0$ ne ricaviamo:
	\begin{equation*}
		\sum_{i\in\N}m_N(S_i)\ge m_n(S)
	\end{equation*}
	che conclude la dimostrazione della \sigadd[ità] di $m_n$.
\end{proof}

\begin{proposition}\label{prop:LebesguePremisuraSigFin}
	Lo spazio di misura elementare $(\R^n,\S_n,m_n)$ è \sigfin[o].
\end{proposition}
\begin{proof}
	Sfruttando che $\mathbb Z^n$ è numerabile, mostriamo esplicitamente la \sigfin[ezza]:
	\begin{equation*}
		\R^n=\bigsqcup_{i_1,i_2,\dots,i_n\in\mathbb Z^n} \co{i_1}{i_1+1}\times\cdots \times\co{i_n}{i_n+1}
	\end{equation*}
	
\end{proof}

\begin{proposition}\label{SigAlgUgualeBoreliani}
	La \sigalg{} generata da $\S_n$ coincide con i Boreliani di $\R^n$.
\end{proposition}
\begin{proof}
	Fissati $(a_i)_{1\le i\le n},(b_i)_{1\le i\le n}\subseteq\R^n$ tali che per ogni $1\le i\le n$ valga $a_i<b_i$, risulta l'identità insiemistica
	\begin{equation*}
		\oo{a_1}{b_1}\times\cdots\times\oo{a_n}{b_n}=\bigcup_{i\in\N}\co{a_1+\frac1i}{b_1}\times\cdots\times\co{a_n+\frac1i}{b_n}
	\end{equation*}
	da cui otteniamo che gli $n$-parallelepipedi aperti appartengono a $\sigma A(\S_n)$.
	Ma in $\R^n$ gli aperti sono sempre unione numerabile di $n$-parallelepipedi aperti\footnote{Basta considerare l'unione di tutti gli $n$-cubi con centro razionale e lato razionale contenuti nell'aperto.} e di conseguenza gli aperti appartengono a $\sigma A(\S_n)$.
	
	Ma allora la \sigalg{} generata dagli aperti e quella generata da $\S_n$ e perciò si ha la tesi.
\end{proof}

\begin{definition}\label{LebesgueMisura}
	Dato lo spazio di misura elementare $(\R_n,\S_n,m_n)$, sia $\M_n$, che verrà chiamato l'insieme dei misurabili secondo Lebesgue, la relativa \sigalg{} di Caratheodory e $m_n:\M_n\to\Rpiu$ \footnote{Qui abusiamo leggermente di notazione, visto che con $m_n$ si indicava la premisura.}, che verrà chiamata misura di Lebesgue, la misura associata, la cui esistenza ci è assicurata dal \cref{EstensioneCaratheodory}. 
	Inoltre indicheremo con $m_n^*:\mathcal P(\R^n)\to\Rpiu$ la misura esterna associata a $m_n$, che quindi ridotta su $\M_n$ coincide con la misura di Lebesgue.
\end{definition}

\begin{remark}\label{nota:LebesgueCompletezza}
	La misura di Lebesgue su $\R^n$ è completa.
\end{remark}
\begin{proof}
	È un'ovvia conseguenza del \cref{EstensioneCaratheodory}.
\end{proof}

\begin{remark}\label{nota:LebesgueSigFin}
	La misura di Lebesgue è \sigfin{}.
\end{remark}
\begin{proof}
	È una banale conseguenza della \cref{prop:LebesguePremisuraSigFin}
\end{proof}


%TODO Definire da qualche parte quasi ogni
\begin{proposition}\label{prop:CompletamentoBoreliani}
	I Boreliani di $\R^n$ sono un sottoinsieme di $\M_n$ ed in particolare il loro completamento rispetto alla misura di Lebesgue è proprio $\M_n$.
\end{proposition}
\begin{proof}
	Il \cref{EstensioneCaratheodory} ci assicura che la \sigalg{} di Caratheodory definita nella \cref{LebesgueMisura} contiene la \sigalg{} generata da $\S_n$ e perciò, sfruttando \cref{SigAlgUgualeBoreliani}, otteniamo i Boreliani sono misurabili secondo Lebesgue come voluto.
	
	Per avere che il completamento dei Boreliani coincide con $\M_n$ basta applicare \cref{prop:CaratheodoryCompletamentoSigAlg} ricordando \cref{prop:LebesguePremisuraSigFin}.
\end{proof}

\begin{theorem}\label{thm:LebesgueEquivalenzeMisurabilita}
	Dato $A\subseteq \R^n$ sono equivalenti le seguenti proposizioni:
	\begin{enumerate}[label=(\arabic*),ref=(\arabic*)]
		\item $A$ è misurabile secondo Lebesgue.
		\item Esistono 
	\end{enumerate}
\end{theorem}


\begin{proposition}\label{CardBoreliani}
	I Boreliani hanno la cardinalità del continuo.
\end{proposition}
\begin{remark}
	I Boreliani si possono ottenere a partire dall'insieme degli aperti $\tau$ con una successione di insiemi, ottenendo uno dal precedente
	mediante aggiunta dei complementari e delle unioni numerabili. Tuttavia non basta una successione finita, né l'unione degli insiemi
	così ottenuti. Vediamo allora come formalizzare questa costruzione utilizzando un'indicizzazione sugli ordinali.
	
	Questa costruzione permetterà poi di ottenere facilmente la cardinalità dei Boreliani.
\end{remark}
\begin{proof}
	Consideriamo la seguente funzione, definita $f:ON \rightarrow \mathcal{P}(\mathcal{P}(\R))$:
	\[
	f(\alpha) = \left\{
		\begin{array}{ll}
			\tau & \quad se\ \alpha=0 \\
			\{ E\subseteq \R : \exists \{A_n\}_{n\in \N}\subseteq f(\beta), E = \bigcap_{n\in\N}A_n\}
			\cup\{E\subseteq\R : E^\mathsf{c}\in f(\beta)\} & \quad se\ \alpha=\beta+1\\
			\bigcup_{\beta < \alpha}f(\alpha) & \quad se\ \alpha\ \grave{e}\ un\ ordinale\ limite.
		\end{array}
	\right.
	\]

\end{proof}


\begin{proposition}\label{prop:LebesgueUnicaEstensione}
	La misura di Lebesgue è l'unica possibile estensione alla \sigalg{} dei Boreliani della premisura $m_n$ definita su $\S_n$.
\end{proposition}
\begin{proof}
	È una banale conseguenza di \cref{UnicitaCaratheodory}, che si può applicare ricordando la \cref{prop:LebesguePremisuraSigFin} e notando che la \sigalg{} generata da $\S_n$ sono i Boreliani come mostrato in \cref{SigAlgUgualeBoreliani}.
\end{proof}

\begin{remark}\label{nota:LebesgueProprieta}
	La misura di Lebesgue in $\R^n$ ha le seguenti proprietà:
	\begin{itemize}
		\item È invariante per traslazione.
		\item È $n$-omogenea.
	\end{itemize}
\end{remark}
\begin{proof}
	Ricordiamo innanzitutto che la misura di Lebesgue è stata costruita appplicando il teorema di Caratheodory e perciò nasce come la riduzione della misura esterna associata a $m_n$ (questa volta da vedere come ridotta su $\S_n$). 
	Basta quindi mostrare che la misura esterna rispetta le proprietà richieste.
	
	Ma è chiaro che le proprietà richieste, poichè gli operatori in gioco commutano con l'unione, vengono ereditate dalla misura esterna se la premisura le rispetta.
	
	Ma nel nostro caso la premisura le rispetta come dimostrato nella \cref{nota:LebesgueElementareProprieta} e questo chiude la dimostrazione.
\end{proof}

\begin{theorem}\label{thm:LebesgueUnicaInvarianteTraslazione}
	Se $\mu$ è una misura sui Boreliani di $\R^n$ invariante per traslazione tale che la misura del cubo unitario valga $1$, cioè
	\begin{equation*}
		\mu\left(\co01\times\co01\times\cdots\times\co01\right)=1
	\end{equation*}
	allora $\mu\equiv m_n$, in altre parole la misura di Lebesgue è l'unica misura con queste proprietà (l'invarianza per traslazione è dimostrata nella \cref{nota:LebesgueProprieta}).
\end{theorem}
\begin{proof}
	L'idea della dimostrazione è provare che $\mu$ e $m_n$ coincidono sugli elementi di $\S_n$ con estremi interi, poi con estremi razionali ed infine con estremi generici.
	Una volta ottenuto questo, basterà applicare \cref{prop:LebesgueUnicaEstensione} per avere la tesi.
	
	Chiamiamo $C$ il cubo unitario come definito nell'enunciato.
	
	Fissato $S\in\S_n$ tale che per ogni $1\le i\le n$ valga $S^-_i,S^+_i\in \mathbb Z$, sia $\mathbb Z(S)$ l'insieme dei punti con tutte le coordinate razionali appartenenti a $S$.
	
	È facile vedere che
	\begin{equation*}
		\bigsqcup_{v\in\mathbb Z(S)} C+v=S
	\end{equation*}
	da cui otteniamo, visto che $\mu,m_n$ sono misure invarianti per traslazione che coincidono su $C$, che $\mu,m_n$ coincidono anche su $S$.
	
	Ora invece scegliamo $S\in\S_n$ tale che per ogni $1\le i\le n$ valga $S^-_i,S^+_i\in \mathbb Q$. Posso assumere, vista l'invarianza per traslazione di tutto quanto, che per ogni $1\le i\le n$ valga $S^-_i=0$.
	
	Sia $m\in\mathbb Z$ tale che per ogni $1\le i\le n$ valga $mS^+_i\in \mathbb Z$, cioè ad esempio si può porre $m$ come il minimo comune multiplo di tutti i denominatori.
	
	Allora, analogamente a quanto già fatto nel caso intero, vale la seguente identità:
	\begin{equation*}
		\co0{mS^+_1}\times\cdots\times\co0{mS^+_n}=
		\bigsqcup_{(i_1,i_2,\dots i_n)\in\mathbb \{0,1,\dots,m-1\}^n}S+(i_1,i_2,\dots,i_n)
	\end{equation*}
	Ma ora applicando quanto abbiamo ricavato per la misura di elementi con estremi a coordinate intere, è chiaro che la misura del membro destro dell'ultima uguaglianza coincide per $\mu,m_n$ e allora sfruttando ancora la loro invarianza per traslazione otteniamo che coincidono anche su $S$.
	
	Ora consideriamo $S\in\S_n$ generico e notiamo che vista la densita dei razionali nei reali posso scrivere $S$ come unione \emph{crescente} di elementi di $(Q_i)_{i\in\N}\subseteq S_n$ con estremi a coordinate razionali. 
	Da questa scrittura, sfruttando \cref{LimiteMonotonoCrescenteMisura}, e quanto già ottenuto sulla coincidenza tra $\mu$ e $m_n$ otteniamo
	\begin{equation*}
		m_n(S)=\lim_{n\to\infty}m_n(Q_i)=\lim_{n\to\infty}\mu(Q_i)=\mu(S)
	\end{equation*}
	cioè $\mu,m_n$ coincidono su $S$ e visto che questo è arbitrariamente scelto in $\S_n$ coincidono su tutto $\S_n$.
	
	Ora finalmente possiamo applicare \cref{prop:LebesgueUnicaEstensione} ed ottenere $\mu\equiv m_n$ che è la tesi.
\end{proof}

\begin{proposition}\label{prop:LebesgueProprietaIsometria}
	La misura di Lebesgue su $\R^n$ è invariante per isometria.
\end{proposition}
\begin{proof}
	Poichè ogni isometria si scrive come una rotazione composta con una traslazione e sappiamo già grazie alla \cref{nota:LebesgueProprieta} che la misura di Lebesgue è invariante per rotazione, è sufficiente dimostrare che la misura di Lebesgue è invariante per rotazione.
	Sia $O:\R^n\to\R^n$ una rotazione. Essendo una rotazione di $\R^n$ questa è in particolare un'applicazione lineare continua di un Banach in se stesso ed è perciò anche Lipschitziana (si può dimostrare anche più facilmente in questo caso).
	
	Allora per quanto visto in \cref{prop:LipschitzMisurabiliInMisurabili} la funzione $O$ manda misurabili in misurabili e perciò ha senso definire la funzione di insiemi $m_n \circ O=\mu:\M_n\to\M_n$. 
	Se dimostriamo che $\mu\equiv m_n$ abbiamo proprio che $m_n$ è invariante per rotazione.
	
	È facile notare che $\mu$ è invariante per traslazione, visto che dati $A\in \M_n$ e $v\in\R^n$ vale
	\begin{equation*}
		\mu(A+v)=m_n(O(A+v))=m_n(O(A)+O(v))=m_n(O(A))=\mu(A)
	\end{equation*}
	dove nei vari passaggi abbiamo usato l'invarianza per traslazione della misura di Lebesgue e la linearità di $O$.
	
	Inoltre, essendo $O$ anche bigettiva, come mostrato in \cref{prop:BigettivaInduceMisura}, la funzione $\mu$ risulta essere una misura.
	
	Allora unendo quanto ottenuto abbiamo che $\mu$ è una misura invariante per traslazione su $\M_n$ e quindi in particolare sui Boreliani di $\R^n$. Allora applicando \cref{thm:LebesgueUnicaInvarianteTraslazione} si ottiene che esiste $\lambda\in\Rpiu$ tale che $\mu=\lambda m_n$.
	
	Ora basta considerare la palla unitaria $B$ di $\R^n$ , che è misurabile poichè aperta.
	La palla $B$, che ha ovviamente misura di Lebesgue né nulla né infinita, è invariante rispetto a $O$, essendo questa una rotazione, e perciò $\mu(B)=m_n(B)$ da cui ricaviamo che $\lambda=1$ cioè $\mu\equiv m_n$ che era quanto si voleva dimostrare.
\end{proof}

\begin{proposition}\label{NumerabiliLebesgueTrascurabili}
	Dato un insieme numerabile $A\subseteq\R^n$, questo è misurabile con misura nulla.
\end{proposition}
\begin{proof}
	I singoli punti, essendo banalmente trascurabili, sono Lebesgue misurabili visto che la misura di Lebesgue è completa, come mostrato nella \cref{nota:LebesgueCompletezza}.
	
	L'insieme $A$ è però numerabile, quindi unione numerabile di singoli punti, che sono trascurabili. Perciò è trascurabile a sua volta $A$ che significa che $A$ è misurabile di misura nulla.
\end{proof}

\begin{proposition}\label{prop:SottospaziTrascurabili}
	Dato $U\subset\R^n$ se lo spazio \emph{affine} generato da $U$ ha dimensione strettamente minore di $n$ allora $U$ è trascurabile secondo Lebesgue.
\end{proposition}
\begin{proof}
	Definiamo $V=\{(x_1,x_2,\dots,x_{n-1},0):\ (x_i)_{1\le i\le n-1}\subseteq\R\}$.
	
	Poichè lo spazio generato da $U$ ha dimensione $\le n-1$, esiste un'isometria che manda $U$ in un sottoinsieme di $V$ ed essendo la misura di Lebesgue invariante per isometria, come dimostrato nella \cref{prop:LebesgueProprietaIsometria}, è sufficiente dimostrare che $V$ è trascurabile per concludere.
	
	Dato $\epsilon>0$ e $k\in\N$ chiamiamo $I_\epsilon(k)$
	\begin{equation*}
		\co{-k}k\times\co{-k}k\times\cdots\times \co{\frac{-\epsilon}{(2k)^{n-1}2^{k+1}}}{\frac{+\epsilon}{(2k)^{n-1}2^{k+1}}}
	\end{equation*}
	È banale, sfruttando la sola definizione della premisura $m_n$, che $m_n(I_\epsilon(k))=\frac{\epsilon}{2^k}$.
	
	Inoltre sfruttando la sola definizione di $V$ è facile verificare che per ogni $\epsilon>0$
	\begin{equation*}
		V\subset\bigcup_{k\in\N} I_\epsilon(k)\implies m_n^*(V)\le \sum m_n(I_\epsilon(k))=\epsilon
	\end{equation*}
	dove con $m_n^*$ abbiamo indicato la misura esterna associata a $m_n$.
	E questo conclude dato che implica, valendo per $\epsilon>0$, che la misura esterna di $V$ è nulla e perciò che $V$ è trascurabile.
\end{proof}


\begin{theorem}[Insieme di Vitali]\label{InsiemeVitali}
	Dato un sottoinsieme $A\in\M_n$ misurabile secondo Lebesgue con misura non nulla, esiste un sottoinsieme $B\subseteq A$ non misurabile secondo Lebesgue, cioè $B\not\in\M$.
\end{theorem}
\begin{proof}
	Dato $A$, notiamo che per la \sigadd[ità] della misura vale:
	\begin{equation*}
		0<m_n(A)=\sum_{(i_1,i_2,\dots,i_n)\in\mathbb Z^n} m_n\left(A\cap\left(\co{i_1}{i_1+1}\times\cdots\co{i_n}{i_n+1}\right)\right)
	\end{equation*}
	e perciò in particolare, visto che la serie ha somma positiva, ne ricaviamo che esistono $(i_1,\dots,i_n)\in \mathbb Z^n$ tale che:
	\begin{equation*}
		0<m_n\left(A\cap\left(\co{i_1}{i_1+1}\times\cdots\co{i_n}{i_n+1}\right)\right)\le
		m_n\left(\co{i_1}{i_1+1}\times\cdots\co{i_n}{i_n+1}\right)=1
	\end{equation*}
	quindi, a meno di intersecare con $\co{i_1}{i_1+1}\times\cdots\co{i_n}{i_n+1}$, possiamo assumere che $A$ sia contenuto in tale $n$-parallelepipedo poichè l'ipotesi di misura non nulla non si perde. 
	In particolare, poichè la misura di Lebesgue è invariante per traslazione come mostrato in \cref{nota:LebesgueProprieta}, senza perdita di generalità assumeremo $A\subseteq \co{0}{1}\times\cdots\co{0}{1}$.
	
	Poniamo su $\R^n$ la relazione di equivalenza seguente:
	\begin{equation*}
		\forall x,y\in\R^n:\ x\equiv y\iff x-y\in\mathbb Q^n
	\end{equation*}
	Ora quozientiamo $\R^n$ rispetto a questa relazione d'equivalenza. In particolare siano $(R_i)_{i\in I}$ le classi d'equivalenza, dove $I$ è un insieme di indici. Sia infine $J$ il sottoinsieme degli indici definito come $J=\{i\in I:\ R_i\cap A\not =\emptyset\}$. 

	Applichiamo ora l'assioma della scelta alla famiglia di insiemi $(R_j\cap A)_{j\in J}$ e estraiamone un insieme di rappresentanti privilegiati $(r_j)_{j\in J}$. Sia ora $B$ l'insieme dei rappresentanti privilegiati.
	
	Vale, per costruzione, $B\subseteq A$. 
	Inoltre, ancora per la definizione di $B$, risulta
	\begin{equation}\label{ContenimentiVitali}
		A\subseteq\bigsqcup_{q\in\mathbb Q^n\cap\left(\cc{-1}{1}\times\cdots \times\cc{-1}{1}\right)} B+q\subseteq \cc{-1}{2}\times\cdots\times\cc{-1}{2}
	\end{equation}
	dato che ogni elemento di $A$ si scrive \emph{in modo unico} (e questo spiega perchè l'unione è disgiunta) come un rappresentante privilegiato rispetto all'equivalenza più un elemento a coordinate razionali, che in questo caso possiamo scegliere tra $-1$ e $1$ poichè $A\subseteq \co 01\times\cdots \times\co 01$.
	
	Se assumiamo che $B$ sia misurabile, ora otteniamo un assurdo applicando \cref{ContenimentiVitali} e ricordando che $m_n$ è invariante per traslazione come mostrato in \cref{nota:LebesgueProprieta}:
	\begin{equation*}
		0<m_n(A)\le \sum_{q\in \mathbb Q^n\cap\left(\cc{-1}{1}\times\cdots \times\cc{-1}{1}\right)} m_n(B+q)=
		\sum_{q\in \mathbb Q^n\cap\left(\cc{-1}{1}\times\cdots \times\cc{-1}{1}\right)} m_n(B) \le 3^n
	\end{equation*}
	che non può essere visto che stiamo sommando una quantità numerabile di volte il valore $m_n(B)$ e perciò possiamo ottenere solo $+\infty$ oppure $0$.
	
	Allora l'insieme $B$ deve essere non misurabile come richiesto.
\end{proof}

\begin{exercise}
	Dato un insieme $A\subseteq\R^n$ di misura nulla secondo Lebesgue e un insieme numerabile $T\subset \R^n$, esiste $k\in \R^n$ tale che $A+k$ sia disgiunto da $T$.
\end{exercise}
\begin{proof}
	Consideriamo l'insieme $B=\bigcup_{t\in T} (A-t)$. 
	Essendo unione numerabile di trascurabili è a sua volta trascurabile.
	Perciò esiste $k\in\R^n\setminus B$ visto che $\R^n$ non è trascurabile.
	
	Assumendo per assurdo che $A-k$ intersechi $T$, avremmo che esistono $t\in T$ e $a\in A$ tali che:
	\begin{equation*}
		a-k=t\implies k=a-t \implies k\in A-t\implies k\in B	
	\end{equation*}
	ma questo non può essere per definizione di $k$ e perciò otteniamo che $A-k$ è disgiunto da $T$, che è la tesi.
\end{proof}

\begin{theorem}[Steinhaus]\label{Steinhaus}%TODO
	Dato un insieme $A\subset\R$ misurabile non trascurabile, l'insieme $A-A=\{a_1-a_2:\ a_1,a_2\in A\}$ contiene un intorno di $0$.
\end{theorem}
\begin{proof}
	TODO.
	Probabilmente da fare usando il solito risultato indimostrabile sulla chiusura monotona 3 o 4 volte.
\end{proof}


