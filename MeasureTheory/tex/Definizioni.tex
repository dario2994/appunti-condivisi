\section{Definizioni}
\begin{definition}[Algebra]
	Dato un insieme $X$, una famiglia $\mathcal A\subseteq\mathcal P(X)$ è un'algebra se valgono:
	\begin{itemize}
		\item $\emptyset\in\mathcal A$
		\item $\forall A\in\mathcal A:\ A^c\in\mathcal A$ cioè un'algebra è stabile per passaggio al complementare.
		\item $\forall A,B\in\mathcal A:\ A\cup B\in\mathcal A$ cioè un'algebra è stabile per unioni finite.
	\end{itemize}
\end{definition}
\begin{remark}\label{ProprietaAlg}
	Un'algebra è stabile anche per intersezioni finite e per differenza insiemistica.
\end{remark}
\begin{proof}
	Poichè vale la formula insiemistica:
	\begin{equation*}
		\bigcap_{i\in I} A_i = \left( \bigcup_{i\in I} A_i^c \right)^c
	\end{equation*}
	e un'algebra è stabile per unione finita e complementare, ottengo facilmente che lo è anche per intersezioni finite.
	
	Per la differenza si sfrutta la seguente relazione insiemistica $A\setminus B=A\cup B^c$ e visto che ho appena dimostrato che $\mathcal A$ è chiuso anche per intersezioni ho finito.
\end{proof}


\begin{definition}[\sigalg{}]
	Dato un insieme $X$, una famiglia $\mathcal A\subseteq\mathcal P (X)$ si dice \sigalg{} se valgono:
	\begin{itemize}
	\item $\emptyset\in \mathcal A$
	\item $\forall A\in \mathcal A:\ A^c\in \mathcal A$ cioè una \sigalg{} è stabile per passaggio al complementare.
	\item $\forall (A_n)_{n\in\mathbb N}\subseteq \mathcal A:\ \bigcup_{n\in\mathbb N} A_n\in \mathcal A$ cioè una \sigalg{} è stabile per unioni numerabili.  
	\end{itemize}
\end{definition}

\begin{remark}\label{ProprietaSigAlg}
	Una \sigalg{} è stabile anche per intersezioni numerabili e per differenza simmetrica.
\end{remark}
\begin{proof}
	Si dimostrano entrambe le proprietà in modo del tutto analogo a come abbiamo dimostrato \cref{ProprietaAlg}.
\end{proof}

\begin{definition}[\Semiring{}]
	Una famiglia $\mathcal S\subseteq \mathcal P(X)$ è detta \semiring{} se valgono le seguenti proprietà:
	\begin{itemize}
		\item $\emptyset\in \mathcal S$
		\item $\displaystyle\forall A,B\in \mathcal S: A\cap B, A\setminus B\in \sqcup \mathcal S$ dove
		$\displaystyle
		\sqcup{ \mathcal S }=\left\{\bigsqcup_{n\in \mathbb N} S_n\ |\ (S_n)_{n\in\mathbb N} \subseteq \mathcal S \wedge \forall i\not= j:\ S_i\cap S_j=\emptyset\right\}$ 
		cioè un \semiring{} non deve essere stabile per intersezione e differenza, ma queste si devono scrivere come unioni disgiunte.
	\end{itemize}
\end{definition}

\begin{proposition}\label{UnioneDisgiuntaQuasiAlgebra}
	Dato un \semiring{} $\mathcal S$ l'insieme $\sqcup\mathcal S$ è stabile per intersezione finita e unione numerabile.
\end{proposition}
\begin{proof}
	Noto intanto che, per definizione, $\sqcup\mathcal S$ è stabile per unione disgiunta numerabile.
	
	Per dimostrare la stabilità di $\sqcup\mathcal S$ per intersezione finita, basta ovviamente farlo per due soli insiemi $A,B\in\sqcup\mathcal S$. Per definizione posso scrivere $A=\bigsqcup_{n\in\mathbb N} A_n, B=\bigsqcup_{n\in\mathbb N} B_n$ dove $(A_n)_{n\in\mathbb N},(B_n)_{n\in\mathbb N}$ sono successioni in $\mathcal S$. Allora vale la seguente identità:
	\begin{equation*}
		A\cap B=\bigsqcup_{n\in\mathbb N} A_n\cap\bigsqcup_{n\in\mathbb N} B_n=
		\bigsqcup_{n,m\in\mathbb N} A_n\cap B_m\in\sqcup\mathcal S
	\end{equation*}
	dove nell'ultimo passaggio ho usato la stabilità di $\sqcup\mathcal S$ per unione disgiunta.
	
	Per l'unione, considero $A,B\in\mathcal S$. Poichè vale $A\cup B=(A\setminus B)\sqcup(A\cap B)$, viste le proprietà di un \semiring{}, risulta $A\cup B\in \sqcup\mathcal S$. Da questo è facile ottenere che anche unioni finite di elementi di $\mathcal S$ appartengono a $\sqcup\mathcal S$.
	
	Sfruttando quanto detto, fissata $(A_n)_{n\in\mathbb N}\subseteq\mathcal S$ vale:
	\begin{equation}\label{UnioneNumerabileDaS}
		\bigcup_{n\in\mathbb N} A_n=\bigsqcup_{n\in\mathbb N} A_n\setminus\cup_{i<n} A_i
		=\bigsqcup_{n\in\mathbb N} \bigcap_{i<n} (A_n\setminus A_i)\in\sqcup\mathcal S
	\end{equation}
	dove nell'ultimo passaggio ho usato il fatto che $\sqcup\mathcal S$ è stabile per unione disgiunta e intersezione finita.
	
	E ora finalmente dimostro la stabilità di $\sqcup\mathcal S$ per unioni numerabili. Sia $(S_n)_{n\in\mathbb N}$ una successione in $\sqcup\mathcal S$. Per definizione devono esistere le successioni $(A^n_i)_{i\in\mathbb N}\subseteq \mathcal S$ tali che $S_n=\bigsqcup_{i\in\mathbb N} A^n_i$.
	
	Allora applicando \cref{UnioneNumerabileDaS} ho:
	\begin{equation*}
		\bigcup_{n\in\mathbb N}S_n=\bigcup_{i,n\in\mathbb N}A^n_i\in\sqcup\mathcal S 
	\end{equation*}
	che è proprio la stabilità di $\sqcup\mathcal S$ per unioni numerabili.

\end{proof}

\begin{definition}[{\sigadd[ità]}]
	Una funzione $\mu:\mathcal F\to \Rpiu$, dove $\mathcal F$ è una famiglia di insiemi, si dice \sigadd{} se per ogni sottofamiglia numerabile $(F_n)_{n\in\mathbb N}\subseteq \mathcal F$ a due a due disgiunta, tale che l'unione appartiene a $\mathcal F$, vale l'addittività:
	\begin{equation*}
		\mu\left(\bigcup_{n\in\mathbb N}F_n \right)=\sum_{n\in\mathbb N} \mu(F_n) 
	\end{equation*}
\end{definition}
\begin{remark}
	Data $\mu:\mathcal F\to \Rpiu$ \sigadd{}, se $\emptyset\in \mathcal F$ allora $\mu(\emptyset)=0$
\end{remark}
\begin{proof}
	Usando la proprietà di \sigadd[ità] ottengo $\mu(\emptyset)=\mu(\emptyset)+\mu(\emptyset)$ che porta ovviamente alla tesi.
\end{proof}


\begin{definition}[Spazio di misura]
	Dati $X$ un insieme, $\mathcal A$ una famiglia di sottoinsiemi di $\mathcal A$ e $\mu:\mathcal A\to \Rpiu$ una funzione, la terna $(X,\mathcal A, \mu)$ si dice uno spazio di misura se:
	\begin{itemize}
		\item La famiglia $\mathcal A$ è una \sigalg{}.
		\item La funzione $\mu$ è \sigadd{}.
	\end{itemize}
	e in questo caso la funzione $\mu$ è detta \emph{misura}.
\end{definition}
%D'ora in poi, quando ci si riferirà ad una misura, si darà per scontato che questa si riferisce ad uno spazio misurato.

\begin{remark}\label{MonotoniaMisura}
	Dato $(X,\mathcal A,\mu)$ uno spazio di misura, $\mu$ è monotona, cioè se $A,B\in\mathcal A$ e $A\subseteq B$ allora $\mu(A)\le \mu(B)$.
\end{remark}
\begin{proof}
	Per quanto detto in \cref{ProprietaSigAlg} $B\setminus A\in\mathcal A$ e perciò sfruttando l'addittività su insiemi disgiunti di $\mu$ ho $\mu(B)=\mu(B\setminus A)+\mu(A)>\mu(A)$ che è la tesi.
\end{proof}

\begin{definition}\label{FinitezzaMisura}
	Dato $(X,\mathcal A,\mu)$ uno spazio di misura, la misura $\mu$ è detta finita se $\mu(X)<+\infty$.
\end{definition}


\begin{proposition}\label{LimiteMonotonoMisura}
	Dato $(X,\mathcal A,\mu)$ uno spazio di misura, allora data una successione $(A_n)_{n\in\mathbb N}\subseteq \mathcal A$ tale che $A_n\subseteq A_{n+1}$ vale:
	\begin{equation*}
		\mu\left(\bigcup_{n\in\mathbb N} A_n\right)=\lim_{n\in\mathbb N} \mu(A_n)
	\end{equation*}
\end{proposition}
\begin{proof}
	Definisco $B_n=A_n\setminus\bigcup_{i<n}A_i$ e per \cref{ProprietaSigAlg} so che $B_n\in\mathcal A$.
	Per facili ragionamenti insiemistici risulta che la successione $(B_n)_{n\in\mathbb N}$ è disgiunta a due a due ed inoltre $A_n=\bigcup_{i\le n}B_i$.
	Sfruttando tutte queste proprietà, e la \sigadd[ità] di $\mu$, ottengo:
	\begin{equation*}
		\mu\left(\bigcup_{n\in\mathbb N} A_n\right)=\mu\left(\bigcup_{n\in\mathbb N} B_n\right)=
		\sum_{n\in\mathbb N} \mu(B_n)=\lim_{n\to\infty} \sum_{i\le n} \mu(B_i)=
		\lim_{n\to\infty} \mu\left(\bigcup_{i\le n} B_i\right)=\lim_{n\to\infty} \mu(A_n)
	\end{equation*}
	che è proprio la tesi.
\end{proof}

\begin{definition}[Misura esterna]
	Dato un insieme $X$ e una funzione $\mu:\mathcal P(X)\to \Rpiu$ è detta una misura esterna se valgono:
	\begin{itemize}
		\item $\mu(\emptyset)=0$
		\item $\mu$ è monotona, cioè dati $A,B\subseteq X$ se vale $A\subseteq B$ allora $\mu(A)\le \mu(B)$
		\item $\mu$ è \sigsubadd{}, cioè  per ogni successione $(A_n)_{n\in\mathbb N}\subseteq \mathcal P(X)$ di sottoinsiemi di $X$ vale $\mu\left(\bigcup_{n\in\mathbb{N}}A_n\right)\le \sum_{n\in\mathbb N} \mu(A_n)$
	\end{itemize}
\end{definition}

\begin{remark}
	Dato $(X,\mathcal A,\mu)$ uno spazio di misura, la misura $\mu$ è \sigsubadd{}.
\end{remark}
\begin{proof}
	Data una successione di sottoinsiemi $(A_n)_{n\in\mathbb N}\subseteq \mathcal A$, considero, come nella dimostrazione di \cref{LimiteMonotonoMisura}, i sottoinsiemi $B_n=A_n\setminus\bigcup_{i<n}A_i\in\mathcal A$.
	Allora ho, lavorando analogamente alla dimostrazione di cui sopra:
	\begin{equation*}
		\mu\left(\bigcup_{n\in\mathbb N} A_n\right)=\mu\left(\bigcup_{n\in\mathbb N} B_n\right)=
		\sum_{n\in\mathbb N} \mu(B_n)\le \sum_{n\in\mathbb N} \mu(A_n)
	\end{equation*}
	dove nell'ultimo passaggio ho sfruttato la monotonia di $\mu$ dimostrata in \cref{MonotoniaMisura}.
\end{proof}

\begin{definition}
	Una terna $(X,\mathcal S,\mu)$ tale che $\mathcal S\subseteq\mathcal P(X)$ è un \semiring{} e $\mu:\mathcal S\to \Rpiu$ è \sigadd{}, la chiamo spazio di misura elementare e la funzione $\mu$ la chiamo misura elementare o premisura.
\end{definition}

\begin{lemma}\label{CoerenzaPremisura}
	Fissato $(X,\mathcal S,\mu)$ uno spazio di misura elementare, siano $(A_n)_{n\in\mathbb N},(B_n)_{n\in\mathbb N}\subseteq\mathcal S$ delle famiglie tali che l'unione sia la stessa, ma i $(B_n)_{n\in\mathbb N}$ siano disgiunti a due a due: $\bigcup_{n\in\mathbb N}A_n=\bigsqcup_{n\in\mathbb N}B_n$.
	Allora risulta $\sum_{n\in\mathbb N}\mu(A_n)\ge \sum_{n\in\mathbb N}\mu(B_n)$.
\end{lemma}
\begin{proof}
	Definisco $A'_n=A_n\setminus\bigcup_{i<n}A_i$. La successione $(A'_n)_{n\in\mathbb N}$ è disgiunta a due a due e ogni singolo elemento appartiene a $\sqcup \mathcal S$ visto che vale $A'_n=\bigcap_{i<n}A_n\setminus A_i$ e $\sqcup \mathcal S$ è chiuso per intersezione finita, come mostrato in \cref{UnioneDisgiuntaQuasiAlgebra}. Infine è chiaro che l'unione della nuova famiglia è uguale a quella di $(A_n)_{n\in\mathbb N}$.
	È importante notare che $A_n\setminus A'_n=A\cap\bigcup_{i<n}A_i\in\sqcup\mathcal S$ dove l'ultimo contenimento è per la stabilità di $\sqcup\mathcal S$ per unioni e intersezioni finite. Allora esistono $(E^n_i)_{i\in\mathbb N}\subseteq\mathcal S$ tali che in unione disgiunta mi $A_n\setminus A'_n$.
	
	Ora definisco $C_{ij}=A'_i\cap B_j$. Ovviamente la successione $(C_{ij})_{i,j\in\mathbb N}$ è disgiunta a due a due (perchè lo sono sia $(A'_n)_{n\in\mathbb N}$ che $(B_n)_{n\in\mathbb N}$) ed è un sottoinsieme di $\sqcup\mathcal S$ poichè intersezione di elementi che vi appartengono. Quindi esiste la famiglia $(F^{ij}_n)_{n\in\mathbb N}\subseteq\mathcal S$ la cui unione disgiunta realizza $C_{ij}$.
	
	Ora per costruzione e per le osservazioni fatte valgono:
	\begin{align*}
		A_n= A'_n\cup \bigsqcup_{i\in\mathbb N}E^n_i=\bigsqcup_{i\in\mathbb N}C_{ni}\cup \bigsqcup_{i\in\mathbb N}E^n_i
		=\bigsqcup_{i,j\in\mathbb N}F^{ni}_j\cup \bigsqcup_{i\in\mathbb N}E^n_i
		&\Longrightarrow \mu(A_n)\ge\sum_{i,j\in\mathbb N}\mu(F^{ni}_j)+\sum_{i\in\mathbb N}\mu(E^n_i)\\
		B_n=\bigsqcup_{i\in\mathbb N}C_{in}=\bigsqcup_{i,j\in\mathbb N}F^{in}_j
		&\Longrightarrow \mu(B_n)=\sum_{i,j\in\mathbb N}\mu(F^{in}_j)
	\end{align*}
	quindi sommando su $n$ arrivo a:
	\begin{align*}
		\sum_{n\in\mathbb N}\mu(A_n)&\ge \sum_{n\in\mathbb N}\sum_{i,j\in\mathbb N}\mu(F^{ni}_j)
		=\sum_{i,n,j\in\mathbb N}\mu(F^{ni}_j)\\
		\sum_{n\in\mathbb N}\mu(B_n)&=\sum_{n\in\mathbb N}\sum_{i,j\in\mathbb N}\mu(F^{in}_j)=\sum_{i,n,j\in\mathbb N}\mu(F^{in}_j)
	\end{align*}
	ma visto che l'ordine degli indici non conta, questi risultati implicano banalmente la tesi.


\end{proof}



\begin{lemma}\label{PiuCheMonotonaPremisura}
	Data $(X,\mathcal S,\mu)$ uno spazio di misura elementare, siano $A,(A_n)_{n\in\mathbb N}\subseteq \mathcal S$ tali che $A\subseteq\bigcup_{n\in\mathbb N}A_n$.
	Allora vale $\mu(A)\le \sum_{n\in\mathbb N}\mu(A_n)$.
\end{lemma}
\begin{proof}
	Visto che $A\subseteq\bigcup_{n\in\mathbb N}A_n$ vale la scrittura insiemistica:
	\begin{equation}\label{ScritturaDecenteUnionePremisura}
		\bigcup_{n\in\mathbb N}A_n=A\sqcup\bigcup_{n\in\mathbb N}A_n\setminus A
	\end{equation}
	Poichè $\mathcal S$ è un \semiring{} $A_n\setminus A\in \sqcup \mathcal S$, e visto che $\sqcup S$ è chiuso per unione numerabile, come mostrato in \cref{UnioneDisgiuntaQuasiAlgebra}, esiste una famiglia $(B_n)_{n\in\mathbb N}\subseteq\mathcal S$ disgiunta tale che $\bigcup_{n\in\mathbb N}A_n\setminus A=\bigsqcup_{n\in\mathbb N}B_n$.
	Allora sostituendo in \cref{ScritturaDecenteUnionePremisura} ottengo:
	\begin{equation*}
		\bigcup_{n\in\mathbb N}A_n=A\sqcup\bigsqcup_{n\in\mathbb N}B_n
	\end{equation*}
	quindi ricado nelle ipotesi di \cref{CoerenzaPremisura} ottenendo che:
	\begin{equation*}
		\sum_{n\in\mathbb N}\mu(A_n)\ge \mu(A)+\sum_{n\in\mathbb N}\mu(B_n)\ge \mu(A)
	\end{equation*}
	che è proprio quanto si voleva dimostrare.

\end{proof}


\begin{theorem}[Estensione di Caratheodory]\label{EstensioneCaratheodory}
	Data $\mu:\mathcal P(X)\to \Rpiu$ una misura esterna, sia $\mathcal A\subseteq \mathcal P(X)$ l'insieme così definito:
	\begin{equation*}
		\mathcal A=\{E\in\mathcal P(X):\ \mu(A)=\mu(A\cap E)+\mu(A\setminus E)\ \forall A\in \mathcal P(X)\}
	\end{equation*}
	allora $\mathcal A$ è una \sigalg{}, detta \sigalg{} di Caratheodory, e $\mu$ ridotta su $\mathcal A$ è \sigadd{}, cioè la terna $(X,\mathcal A,\mu|_{\mathcal A})$ è uno spazio di misura.
\end{theorem}
\begin{proof}
	La dimostrazione procede in due passi, prima dimostro che $\mathcal A$ è un'algebra di insiemi e poi passo a dimostrare che è una \sigalg{} e che $\mu$ è \sigadd{} ridotta su di questa.
	
	Il fatto che $\mathcal A$ sia stabile per complementare è ovvio per la definizione (che è simmetrica tra $E$ ed $E^c$).
	
	Sia $A\in\mathcal P(X)$ generico ed $E,F\in\mathcal A$ applicando la sola definizione di $\mathcal A$ ed alcuni passaggi insiemistici ottengo:
	\begin{align*}
		\mu(A)\stackrel{F\in\mathcal A}{=}&\mu(A\cap F)+\mu(A\setminus F)\stackrel{E\in\mathcal A}{=}
		\mu(A\cap F)+\mu\left((A\setminus F)\cap E\right)+\mu\left((A\setminus F)\setminus E\right)\\
		=\hspace{0.4em}&\mu\left((A\cap (E\cup F))\cap F\right)+\mu\left((A\cap (E\cup F))\setminus F\right)+
		\mu\left(A\setminus(E\cup F)\right)\\
		\stackrel{F\in\mathcal A}{=}&\mu(A\cap (E\cup F))+\mu\left(A\setminus(E\cup F)\right)
	\end{align*}
	e visto che questo vale per ogni scelta di $A\in\mathcal P(X)$ ho appena dimostrato che $\mathcal A$ è stabile per unione.
	
	Unendo quanto detto si ha facilmente che $\mathcal A$ è un'algebra di insiemi.
	
	Ora sia $(E_n)_{n\in\mathbb N}\subseteq \mathcal A$ una famiglia numerabile di insiemi ed $A\in\mathcal P(X)$ un generico sottoinsieme di $X$.
	
	Per la \sigsubadd[ità] di $\mu$ vale:
	\begin{equation}\label{DisuguaglianzaFacileCaratheodory}
		\mu(A)\le \mu\left(A\cap\bigcup_{n\in\mathbb N} E_n\right)+\mu\left(A\setminus\cap\bigcup_{n\in\mathbb N} E_n\right)
	\end{equation}
	Si vuole dimostrare che il $\le$ è in realtà un'identità. Se $\mu(A)=+\infty$ questo è ovvio, quindi tratteremo il caso in cui $\mu(A)<+\infty$. Chiamo $F_n=E_n\setminus \bigcup_{i<n} E_i$, ottenendo in maniera ovvia che gli $(F_n)_{n\in\mathbb N}$ sono a due a due disgiunti e che appartengono a $\mathcal A$ poichè quest'ultima è un'algebra.
	
	Per induzione è facile verificare, sfruttando unicamente il fatto che $F_n\in\mathcal A$ e $\mu(A)<+\infty$, che risulta:
	\begin{equation}\label{IdentitaDifferenzaCaratheodory}
		\mu\left(A\setminus \bigsqcup_{n\le m} F_n\right)=\mu(A)-\sum_{n\le m} \mu(A\cap F_n)
	\end{equation}
	e incidentalmente da questa formula si ha che la serie $\sum_{n\in\mathbb N}\mu(A\cap F_n)$ converge, visto che è a termini positivi e limitata (da $\mu(A)$).
	
	Per la \sigsubadd[ità] di $\mu$ vale:
	\begin{equation}\label{IntersezioneStimaCaratheodory}
		\mu\left(A\cap\bigcup_{n\in\mathbb N} E_n\right)=\mu\left(\bigsqcup_{n\in\mathbb N} A\cap F_n\right)\le
		\sum_{n\in\mathbb N} \mu(A\cap F_n)
	\end{equation}
	mentre, grazie alla monotonia e a \cref{IdentitaDifferenzaCaratheodory} ottengo:
	\begin{equation}\label{DifferenzaStimaCaratheodory}
		\mu\left(A\setminus\bigcup_{n\in\mathbb N} E_n\right) = \mu\left(A\setminus\bigsqcup_{n\in\mathbb N} F_n\right) \le \mu\left(A\setminus\bigsqcup_{n\le m} F_n\right) = 
		\mu(A)-\sum_{n\le m}\mu(A\cap F_n)
	\end{equation}
	
	Ora unendo \cref{IntersezioneStimaCaratheodory,DifferenzaStimaCaratheodory} arrivo ad avere che, per ogni $m\in\mathbb{N}$:
	\begin{equation*}
		\mu\left(A\cap\bigcup_{n\in\mathbb N} E_n\right)+\mu\left(A\setminus\bigcup_{n\in\mathbb N} E_n\right)\le
		\mu(A)+\sum_{m\le n}\mu(A\cap F_n) 
	\end{equation*}
	ma per la convergenza di $\sum_{n\in \mathbb N}\mu(A\cap F_n)$, estraendo l'$\inf$ da entrambe le parti finalmente arrivo a:
	\begin{equation*}
		\mu\left(A\cap\bigcup_{n\in\mathbb N} E_n\right)+\mu\left(A\setminus\bigcup_{n\in\mathbb N} E_n\right)\le
		\mu(A)
	\end{equation*}
	che unita a \cref{DisuguaglianzaFacileCaratheodory} mi assicura che vale l'identità tra i membri e, visto che ciò vale indipendentemente dalla scelta di $A\in\mathcal P(X)$ implica che $\bigcup_{n\in\mathbb N}E_n\in\mathcal A$ che equivale a dire che $\mathcal A$ è una \sigalg{}.
	
	Dimostrare che $\mu$ è \sigadd{} su $\mathcal A$ è ora molto facile.
	Considero $(E_n)_{n\in\mathbb N}\subset \mathcal A$ una famiglia numerabile di insiemi \emph{disgiunti}. Per facile induzione si ha che:
	\begin{equation*}
		\mu\left(\bigsqcup_{n\le m}E_n\right)=\sum_{n\le m} \mu(E_n)
	\end{equation*}
	e applicando questa e la monotonia di $\mu$ ottengo:
	\begin{equation*}
		\sum_{n\le m} \mu(E_n)=\mu\left(\bigsqcup_{n\le m}E_n\right)\le
		\mu\left(\bigsqcup_{n\in\mathbb N}E_n\right)\le \sum_{n\in\mathbb N} \mu(E_n)
	\end{equation*}
	e questa doppia disuguaglianza, per la definizione delle serie a termini positivi, mi implica che tutte le disuguaglianze sono identità. Ma allora questo dimostra che $\mu$ è \sigadd{} su $\mathcal A$ concludendo la dimostrazione.
\end{proof}

\begin{proposition}\label{MisuraEsternaDiPremisura}
	Dato $(X,\mathcal S,\mu)$ uno spazio di misura elementare si consideri la funzione che associa ad ogni sottoinsieme l'estremo inferiore delle misure dei ricoprimenti, cioè $\mu^*:\mathcal P(X)\to\Rpiu$ definita come 
	\begin{equation*}
		\mu^*(A)=\inf\left\{\sum_{n\in\mathbb N} A_n\ |\ (A_n)_{n\in\mathbb N}\subseteq\mathcal S\ \wedge
		\ A\subseteq\bigcup_{n\in\mathbb N}A_n\right\}
	\end{equation*}
	Allora $\mu^*$ è una misura esterna che estende $\mu$ (cioè $\mu^*|_{\mathcal S}=\mu$) ed inoltre $\mathcal S$ appartiene alla relativa \sigalg{} di Caratheodory (come definita in \cref{EstensioneCaratheodory}).
\end{proposition}
\begin{proof}
	Per ottenere che $\mu^*$ è una misura esterna basta verificare le proprietà che deve rispettare.
	Ovviamente, poichè $\mu(\emptyset)=0$, vale $\mu^*(\emptyset)=0$. 
	Inoltre, ancora facilmente, $\mu^*$ è monotona, visto che se $A\subseteq B$ un ricoprimento di $B$ ricopre anche $A$.
	E infine è anche \sigsubadd{} visto che l'unione di ricoprimenti (che risulta ancora un ricoprimento numerabile) è un ricoprimento dell'unione.
	
	Dato $S\in\mathcal S$ vale ovviamente $\mu*(S)\le\mu(S)$, poichè $S$ si ricopre da solo. Per dimostrare la disuguaglianza opposta considero $(S_n)_{n\in\mathbb N}\in \mathcal S$ un ricoprimento di $S$
	
	Ora perciò mi resta da dimostrare che se $E\subseteq \mathcal S$ allora per ogni $A\in\mathcal P(X)$ risulta:
	\begin{equation}\label{MisuraEsternaDisDifficile}
		\mu^*(A) \ge \mu^*(A\cap E)+\mu^*(A\setminus E)
	\end{equation}
	Questo è sufficiente ad avere che $\mathcal S$ è contenuto nella \sigalg{} di Caratheodory poichè l'altra disuguaglianza mi è assicurata dalla \sigsubadd[ità].
	
	Dato $(A_n)_{n\in\mathbb N}\subseteq\mathcal S$ un ricoprimento di $A$, considero $B_n=A_n\cap E$ e $C_n=A_n\setminus E$. Ovviamente $(B_n)_{n\in\mathbb N},(C_n)_{n\in\mathbb N}$ ricoprono rispettivamente $A\cap E,A\setminus E$. Poichè $\mathcal S$ è un \semiring{} riesco però a trovare $(B'^n_i)_{i\in\mathbb N},(C'^n_i)_{i\in\mathbb N} \subseteq \mathcal S$ tali che $B_n=\bigsqcup_{i\in\mathbb N}B'^n_i$ e analogo risultato per $C_n$. Quindi $(B'^n_i)_{n,i\in\mathbb N}, (C'^n_i)_{n,i\in\mathbb N}$ risultano ricoprimenti con elementi di $\mathcal S$ di $A\cap E,A\setminus E$ rispettivamente.
	Ora, sfruttando non più della sola \sigadd[ità] di $\mu$ concludo:
	\begin{align*}
		\sum_{n\in\mathbb N}\mu(A_n)=\sum_{n\in\mathbb N} \mu(B_n)+\mu(C_n)&=
		\sum_{n\in\mathbb N}\sum_{i\in\mathbb N}\mu(B'^n_i)+\mu(C'^n_i)\\
		&=
		\sum_{n,i\in\mathbb N}\mu(B'^n_i)+\sum_{n,i\in\mathbb N}\mu(C'^n_i)\ge \mu(A\cap E)+\mu(A\setminus E)
	\end{align*}
	ma questo implica facilmente \cref{MisuraEsternaDisDifficile} estraendo l'estremo inferiore a entrambi i membri sui ricoprimenti di $A$.
\end{proof}

\begin{theorem}
	Dato $(X,\mathcal S,\mu)$ uno spazio di misura elementare esiste una \sigalg{} $\mathcal A$ e una funzione $\mu':\mathcal A\to\Rpiu$ tali che $\mathcal S\subseteq \mathcal A$, $\mu'$ estende la premisura $\mu$ e $(X,\mathcal A,\mu')$ è uno spazio di misura.
\end{theorem}
\begin{proof}
	Considero la misura esterna $\mu^*:\mathcal P(X)\to\Rpiu$ definita nell'enunciato di \cref{MisuraEsternaDiPremisura}. Questa, come dimostrato 
\end{proof}




