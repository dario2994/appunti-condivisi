\section{Estendere una premisura ad una misura}
L'obiettivo ora è riuscire ad estendere una premisura definita su un semianello ad una misura su una \sigalg{}. Per fare questo il percorso sarà prima quello di estendere la premisura ad una misura esterna, per poi ridurre questa ad una misura canonica.


\begin{theorem}\label{RiduzionePreCaratheodory}
	Data $\mu:\mathcal P(X)\to \Rpiu$ una misura esterna, sia $\A\subseteq \mathcal P(X)$ l'insieme così definito:
	\begin{equation*}
		\A=\{E\in\mathcal P(X):\ \mu(A)=\mu(A\cap E)+\mu(A\setminus E)\ \forall A\in \mathcal P(X)\}
	\end{equation*}
	allora $\A$ è una \sigalg{}, detta \sigalg{} di Caratheodory, e $\mu$ ridotta su $\A$ è una misura completa.
\end{theorem}
\begin{proof}
	La dimostrazione procede in tre passi: prima mostriamo che $\A$ è un'algebra di insiemi, poi che è una \sigalg{} e infine che $\mu$ è \sigadd{} e completa ridotta su $\A$.
	
	Il fatto che $\A$ sia stabile per complementare è ovvio per la definizione (che è simmetrica tra $E$ ed $E^c$).
	
	Fissati $A\in\mathcal P(X)$ generico ed $E,F\in\A$, applicando la sola definizione di $\A$ ed alcuni passaggi insiemistici si ricava:
	\begin{align*}
		\mu(A)\stackrel{F\in\A}{=}&\mu(A\cap F)+\mu(A\setminus F)\stackrel{E\in\A}{=}
		\mu(A\cap F)+\mu\left((A\setminus F)\cap E\right)+\mu\left((A\setminus F)\setminus E\right)\\
		=\hspace{0.4em}&\mu\left((A\cap (E\cup F))\cap F\right)+\mu\left((A\cap (E\cup F))\setminus F\right)+
		\mu\left(A\setminus(E\cup F)\right)\\
		\stackrel{F\in\A}{=}&\mu(A\cap (E\cup F))+\mu\left(A\setminus(E\cup F)\right)
	\end{align*}
	e visto che questo vale per ogni scelta di $A\in\mathcal P(X)$ abbiamo dimostrato che $\A$ è stabile per unione.
	
	Unendo quanto detto si ha facilmente che $\A$ è un'algebra di insiemi.
	
	Ora sia $(E_n)_{n\in\N}\subseteq \A$ una famiglia numerabile di insiemi ed $A\in\mathcal P(X)$ un generico sottoinsieme di $X$.
	
	Per la \sigsubadd[ità] di $\mu$ vale:
	\begin{equation}\label{DisuguaglianzaFacileCaratheodory}
		\mu(A)\le \mu\left(A\cap\bigcup_{n\in\N} E_n\right)+\mu\left(A\setminus\bigcup_{n\in\N} E_n\right)
	\end{equation}
	Si vuole dimostrare che il $\le$ è in realtà un'uguaglianza. Se $\mu(A)=+\infty$ questo è ovvio, quindi tratteremo il caso in cui $\mu(A)<+\infty$. Chiamiamo $F_n=E_n\setminus \bigcup_{i<n} E_i$, ottenendo in maniera ovvia che gli $(F_n)_{n\in\N}$ sono a due a due disgiunti e che appartengono ad $\A$ poiché quest'ultima è un'algebra.
	
	Per induzione è facile verificare, sfruttando unicamente il fatto che $F_n\in\A$ e $\mu(A)<+\infty$, che risulta:
	\begin{equation}\label{IdentitaDifferenzaCaratheodory}
		\mu\left(A\setminus \bigsqcup_{n\le m} F_n\right)=\mu(A)-\sum_{n\le m} \mu(A\cap F_n)
	\end{equation}
	e incidentalmente da questa formula si ha che la serie $\sum_{n\in\N}\mu(A\cap F_n)$ converge, visto che è a termini positivi e limitata (da $\mu(A)$).
	
	Per la \sigsubadd[ità] di $\mu$ vale:
	\begin{equation}\label{IntersezioneStimaCaratheodory}
		\mu\left(A\cap\bigcup_{n\in\N} E_n\right)=\mu\left(\bigsqcup_{n\in\N} A\cap F_n\right)\le
		\sum_{n\in\N} \mu(A\cap F_n)
	\end{equation}
	mentre, grazie alla monotonia e a \cref{IdentitaDifferenzaCaratheodory}, otteniamo:
	\begin{equation}\label{DifferenzaStimaCaratheodory}
		\mu\left(A\setminus\bigcup_{n\in\N} E_n\right) = \mu\left(A\setminus\bigsqcup_{n\in\N} F_n\right) \le \mu\left(A\setminus\bigsqcup_{n\le m} F_n\right) = 
		\mu(A)-\sum_{n\le m}\mu(A\cap F_n)
	\end{equation}
	
	Ora unendo \cref{IntersezioneStimaCaratheodory,DifferenzaStimaCaratheodory} giungiamo ad avere che, per ogni $m\in\mathbb{N}$:
	\begin{equation*}
		\mu\left(A\cap\bigcup_{n\in\N} E_n\right)+\mu\left(A\setminus\bigcup_{n\in\N} E_n\right)\le
		\mu(A)+\sum_{m\le n}\mu(A\cap F_n) 
	\end{equation*}
	ma per la convergenza di $\sum_{n\in \N}\mu(A\cap F_n)$, estraendo l'$\inf$ da entrambe le parti finalmente arriviamo a:
	\begin{equation*}
		\mu\left(A\cap\bigcup_{n\in\N} E_n\right)+\mu\left(A\setminus\bigcup_{n\in\N} E_n\right)\le
		\mu(A)
	\end{equation*}
	che unita a \cref{DisuguaglianzaFacileCaratheodory} ci assicura che vale l'identità tra i membri e, visto che ciò vale indipendentemente dalla scelta di $A\in\mathcal P(X)$, risulta $\bigcup_{n\in\N}E_n\in\A$ che equivale a dire che $\A$ è una \sigalg{}.
	
	Dimostrare che $\mu$ è \sigadd{} su $\A$ è ora molto facile.
	Consideriamo $(E_n)_{n\in\N}\subseteq \A$ una famiglia numerabile di insiemi \emph{disgiunti}. Per facile induzione si ha che:
	\begin{equation*}
		\mu\left(\bigsqcup_{n\le m}E_n\right)=\sum_{n\le m} \mu(E_n)
	\end{equation*}
	e applicando questa e la monotonia di $\mu$ risulta:
	\begin{equation*}
		\sum_{n\le m} \mu(E_n)=\mu\left(\bigsqcup_{n\le m}E_n\right)\le
		\mu\left(\bigsqcup_{n\in\N}E_n\right)\le \sum_{n\in\N} \mu(E_n)
	\end{equation*}
	e questa doppia disuguaglianza, per la definizione delle serie a termini positivi, implica che tutte le disuguaglianze sono identità. Ma allora questo dimostra che $\mu$ è \sigadd{} su $\A$.
	
	Infine per dimostrare la completezza di $\mu|_{\A}$ basta mostrare che dato $E\in\mathcal P(X)$ trascurabile, vale $E\in\A$ (questo è sufficiente a mostrare la completezza, visto che per monotonia i sottinsiemi di un trascurabile sono a loro volta trascurabili).
	
	Fissato un generico $A\in\mathcal P(X)$, risulta per la monotonia di $\mu$:
	\begin{equation*}
		\mu(A\cap N)+\mu(A\setminus N)\le \mu(N)+\mu(A)=\mu(A)
	\end{equation*}
	che, unita alla \sigsubadd[ità] di $\mu$ mi assicura
	\begin{equation*}
		\mu(A)=\mu(A\cap N)+\mu(A\setminus N)
	\end{equation*}
	che è proprio la condizione di appartenenza ad $\A$.
\end{proof}

\begin{proposition}\label{MisuraEsternaDiPremisura}
	Dato $(X,\S,\mu)$ uno spazio di misura elementare si consideri la funzione che associa ad ogni sottoinsieme l'estremo inferiore delle misure dei ricoprimenti, cioè $\mu^*:\mathcal P(X)\to\Rpiu$ definita come 
	\begin{equation*}
		\mu^*(A)=\inf\left\{\sum_{n\in\N} \mu(A_n)\ |\ (A_n)_{n\in\N}\subseteq\S\ \wedge
		\ A\subseteq\bigcup_{n\in\N}A_n\right\}
	\end{equation*}
	Allora $\mu^*$ è una misura esterna che estende $\mu$ (cioè $\mu^*|_\S=\mu$) ed inoltre $\S$ appartiene alla relativa \sigalg{} di Caratheodory (come definita in \cref{RiduzionePreCaratheodory}).
\end{proposition}
\begin{proof}
	Per affermare che $\mu^*$ è una misura esterna sono sufficienti le seguenti verifiche.
	Ovviamente, poiché $\mu(\emptyset)=0$, vale $\mu^*(\emptyset)=0$. 
	Inoltre, ancora facilmente, $\mu^*$ è monotona, visto che se $A\subseteq B$ un ricoprimento di $B$ ricopre anche $A$.
	E infine è anche \sigsubadd{} visto che l'unione di ricoprimenti (che risulta ancora un ricoprimento numerabile) è un ricoprimento dell'unione.
	
	Dato $S\in\S$ vale ovviamente $\mu^*(S)\le\mu(S)$, poiché $S$ si ricopre da solo. Per dimostrare la disuguaglianza opposta, ottenendo così che $\mu^*$ estende $\mu$, consideriamo $(S_n)_{n\in\N}\in \S$ un ricoprimento di $S$. Per \cref{PiuCheMonotonaPremisura} vale:
	\begin{equation*}
		\mu(S)\ge \sum_{n\in\N} \mu(S_n)
	\end{equation*}
	e perciò passando all'estremo inferiore sui ricoprimenti otteniamo la disuguaglianza cercata.
	
	Ora perciò resta da dimostrare che se $E\in \S$ allora per ogni $A\in\mathcal P(X)$ risulta:
	\begin{equation}\label{MisuraEsternaDisDifficile}
		\mu^*(A) \ge \mu^*(A\cap E)+\mu^*(A\setminus E)
	\end{equation}
	Questo è sufficiente ad avere che $\S$ è contenuto nella \sigalg{} di Caratheodory, poiché l'altra disuguaglianza è assicurata dalla \sigsubadd[ità].
	
	Dato $(A_n)_{n\in\N}\subseteq\S$ un ricoprimento di $A$, chiamiamo $B_n=A_n\cap E$ e $C_n=A_n\setminus E$. Ovviamente $(B_n)_{n\in\N},(C_n)_{n\in\N}$ ricoprono rispettivamente $A\cap E,A\setminus E$. Poiché $\S$ è un \semiring{} riusciamo però a trovare $(B'^n_i)_{i\in\N},(C'^n_i)_{i\in\N} \subseteq \S$ tali che $B_n=\bigsqcup_{i\in\N}B'^n_i$ e analogo risultato per $C_n$. Quindi $(B'^n_i)_{n,i\in\N}, (C'^n_i)_{n,i\in\N}$ risultano ricoprimenti con elementi di $\S$ di $A\cap E,A\setminus E$ rispettivamente.
	Ora, sfruttando la \sigadd[ità] di $\mu$ e la \sigsubadd[ità] di $\mu^*$ concludiamo:
	\begin{align*}
		\sum_{n\in\N}\mu(A_n)&=
		\sum_{n\in\N} \mu\left(\bigsqcup_{i\in\N} B'^n_i\sqcup \bigsqcup_{i\in\N} C'^n_i\right)=
		\sum_{n\in\N}\sum_{i\in\N}\mu(B'^n_i)+\sum_{n\in\N}\sum_{i\in\N}\mu(C'^n_i)\\
		&\ge
		\sum_{n\in\N}\mu^*(B_n)+\sum_{n\in\N}\mu^*(C_n)
		\ge \mu^*(A\cap E)+\mu^*(A\setminus E)
	\end{align*}
	e questo implica facilmente \cref{MisuraEsternaDisDifficile} estraendo l'estremo inferiore a entrambi i membri sui ricoprimenti di $A$.
\end{proof}
\begin{remark}\label{PremisuraMassimaMisura}
	Sia $(X,\S,\mu)$ uno spazio di premisura e $\mu^*:\mathcal P(X)\to\Rpiu$ la misura esterna associata a $\mu$ (come definita nell'enunciato di \cref{MisuraEsternaDiPremisura}), allora una qualsiasi misura $\mu':\A\to\Rpiu$, dove $\A\supseteq \S$ è una \sigalg{}, che estenda $\mu$ è minore della misura esterna, cioè $\mu'(A)\le \mu^*(A)$ per ogni $A\in\A$.
\end{remark}
\begin{proof}
	Fissiamo $A\in\A$ e chiamiamo $(S_n)_{n\in\N}\in\S$ un suo ricoprimento numerabile.
	
	Essendo $\mu'$ una misura è \sigsubadd{} e decrescente per \cref{SubAdditivitaMisura,MonotoniaMisura} e perciò vale:
	\begin{equation*}
		\sum_{n\in\N}\mu(S_n)=\sum_{n\in\N}\mu'(S_n)\ge \mu'\left(\bigcup_{n\in\N}S_n\right)\ge \mu'(A)
	\end{equation*}
	ed ora passando all'estremo superiore sui ricoprimenti di $A$ otteniamo $\mu^*(A)\ge\mu'(A)$ che è la tesi, visto che la dimostrazione vale per ogni $A\in\A$.
\end{proof}



\begin{theorem}[Estensione di Caratheodory]\label{EstensioneCaratheodory}
	Dato $(X,\S,\mu)$ uno spazio di misura elementare esiste una \sigalg{} $\A$ e una funzione $\mu':\A\to\Rpiu$ tali che $\S\subseteq \A$, $\mu'$ estende la premisura $\mu$ e $(X,\A,\mu')$ è uno spazio di misura completo.
\end{theorem}
\begin{proof}
	Consideriamo la misura esterna $\mu^*:\mathcal P(X)\to\Rpiu$ definita nell'enunciato di \cref{MisuraEsternaDiPremisura}. Sempre \cref{MisuraEsternaDiPremisura} ci assicura che questa è un'estensione di $\mu$.
	
	Possiamo ora ridurre $\mu^*$ grazie al \cref{RiduzionePreCaratheodory} ad una misura completa $\mu':\A\to\Rpiu$ dove $\A$ è la \sigalg{} di Caratheodory. 
	
	Ma come dimostrato in \cref{MisuraEsternaDiPremisura} $\S\subseteq\A$ e inoltre vale $\mu'|_\S=\mu^*|_\S=\mu$, perciò lo spazio $(X,\A,\mu')$ rispetta tutte le richieste dell'enunciato.
\end{proof}


\begin{proposition}\label{EquivalenzeMisurabilitaSottoinsieme}
	Dato $(X,\S,\mu)$ una spazio di misura elementare, sia $\mu^*$ la misura esterna associata a $\mu$ (come definita nell'enunciato di \cref{MisuraEsternaDiPremisura}) e $\A$ la \sigalg{} di Caratheodory (la cui esistenza mi è assicurata dal \cref{EstensioneCaratheodory}) con la relativa estensione della misura $\mu':\A\to\Rpiu$.
	
	Sia $A\subseteq X$ tale che sia \emph{\sigfin[o] rispetto alla premisura}, cioè sia contenuto in un unione numerabile di elementi di $\S$ con misura finita.
	
	Allora le seguenti affermazioni su $A$ sono equivalenti:
	\begin{enumerate}[label=(\arabic*),ref=(\arabic*)]
		\item $A\in\A$ cioè l'insieme è un misurabile (secondo Caratheodory).\label{MisurabileEquivalenze}
		\item Fissato $\epsilon>0$ esiste $S\in\sqcup\S$ che contiene $A$ e tale che $\mu^*(S\setminus A)<\epsilon$.\label{UnioniDaFuoriEquivalenze}
		\item Esiste $B\in\sigma A(S)$ che $B$ che contiene $A$ e tale che $\mu^*(B\setminus A)=0$.\label{SigmaDaFuoriEquivalenze}
		\item Esiste $B\in\sigma A(S)$ che è contenuto in $A$ e tale che $\mu^*(A\setminus B)=0$.\label{SigmaDaDentroEquivalenze}
	\end{enumerate}
\end{proposition}
\newcommand{\ImplicationProof}[2]{$\text{\ref{#1}}\implies\text{\ref{#2}}$}% X => Y
\begin{proof}
	Per l'ipotesi di \sigfin[ezza] esiste $(X_n)_{n\in\N}\subseteq \S$ una successione disgiunta (posso sceglierla disgiunta poichè $\sqcup\S$ è chiuso per unioni numerabili come mostrato in \cref{UnioneDisgiuntaQuasiAlgebra}) tale che $A\subseteq \bigcup_{n\in\N}X_n$ e $\mu(X_n)<+\infty$. Chiamiamo inoltre $X'=\bigcup_{n\in\N}X_n$.
	
	Dimostriamo innanzitutto la catena di implicazioni
	$\text{\ref{MisurabileEquivalenze}}\implies
	\text{\ref{UnioniDaFuoriEquivalenze}}\implies
	\text{\ref{SigmaDaFuoriEquivalenze}}\implies
	\text{\ref{MisurabileEquivalenze}}$.
	\begin{description}
		\item[\ImplicationProof{MisurabileEquivalenze}{UnioniDaFuoriEquivalenze}] Definiamo $A_n=A\cap X_n$. Gli $A_n$ sono misurabili, perchè intersezioni di misurabili, e hanno misura finita dato che la misura è monotona.
		
		Sia $\epsilon>0$ fissato.
		
		Dato $n\in\N$ allora $\mu'(A_n)=\mu^*(A_n)$, ma per la definizione di $\mu^*$ questo implica che esiste $S_n\in\sqcup \S$ tale che $A_n\subseteq S_n$ e tale che valga:
		\begin{equation*}
			\mu^*(A_n)\le \mu^*(S_n) \le \mu^*(A_n)+\frac\epsilon{2^n}\implies
			0\le \mu^*(S_n)-\mu^*(A_n)\le \frac\epsilon{2^n}
		\end{equation*}
		dove nell'implicazione abbiamo sfruttato $\mu^*(A_n)=\mu'(A_n)<+\infty$.
		Infine dal fatto che $S_n,A_n$ sono misurabili, per l'addittività della misura, ricaviamo:
		\begin{equation}\label{DifficileFreccia12Misurabili}
			0\le \mu'(S_n\setminus A_n) \le \frac\epsilon{2^n} \implies \mu^*(S_n\setminus A_n)\le \frac\epsilon{2^n}
		\end{equation}
		
		Ora consideriamo $S=\bigcup_{n\in\N}S_n$, che appartiene ancora a $\sqcup\S$ poichè questo è chiuso per unione numerabile come dimostrato in \cref{UnioneDisgiuntaQuasiAlgebra}. Si ha $A\subseteq S$ e per la subaddittività della misura esterna  $\mu^*$, sfruttando \cref{DifficileFreccia12Misurabili}, si ottiene:
		\begin{equation*}
			\mu^*(S\setminus A)=\mu^*\left(\bigcup_{n\in\N}S_n\setminus A_n\right)\le
			\sum_{n\in\N}\mu^*(S_n\setminus A_n)\le\sum_{n\in\N}\frac\epsilon{2^n}\le \epsilon
		\end{equation*}
		e questo conclude, visto che $S$ soddisfa tutte le richieste dell'enunciato.
		\item[\ImplicationProof{UnioniDaFuoriEquivalenze}{SigmaDaFuoriEquivalenze}] L'ipotesi ci assicura l'esistenza di $(S_n)_{n\in\N}\subseteq \sqcup \S$ tali che $A\subseteq S_n$ e $\mu^*(S_n\setminus A)<\frac1n$.
		
		Definiamo $B=\cap_{n\in\N}S_n$, che appartiene a $\sigma A(\S)$ poichè generato a partire da $\S$ con sole unioni e intersezioni numerabili. Ovviamente vale $A\subseteq B$.
		
		Infine risulta vera la seguente:
		\begin{equation*}
			\forall n\in\N:\ B\subseteq B_n\implies B\setminus A\subseteq B_n\setminus A\implies \mu^*(B\setminus A)\le \mu^*(B_n\setminus A)\le \frac1n
		\end{equation*}
		e perciò $\mu^*(B\setminus A)=0$ e questo dimostra l'implicazione cercata.
		\item[\ImplicationProof{SigmaDaFuoriEquivalenze}{MisurabileEquivalenze}] Per ipotesi esiste $B\in\sigma A(\S)$ che contiene $A$ e tale che $\mu^*(B\setminus A)=0$.
		
		Poichè $\A$ è una \sigalg{} contenente $\S$ e $B\in\sigma A(\S)$, deve essere $B\in\A$. 
		Inoltre $\mu':\A\to\Rpiu$ è una misura completa, come visto nel \cref{RiduzionePreCaratheodory}, e inoltre i trascurabili secondo $\mu^*$ risultano essere proprio i trascurabili secondo $\mu'$ (come visto nella dimostrazione di \cref{RiduzionePreCaratheodory}) quindi anche $B\setminus A\in \A$ in quanto è un trascurabile e la misura è completa. 
		
		Di conseguenza, visto che $\A$ è una \sigalg, otteniamo che $A=B\setminus (B\setminus A)$ è misurabile in quanto differenza di misurabili. Perciò $A\in\A$ dimostrando l'implicazione.
	\end{description}
	
	Ora dimostriamo \ImplicationProof{MisurabileEquivalenze}{SigmaDaDentroEquivalenze} sfruttando un passaggio al complementare \emph{ristretto}.
	
	Se $A$ è misurabile allora anche $X'\setminus A$ è misurabile ed è anch'esso \sigfin[o] per $\mu$. Quindi sfruttando l'implicazione \ImplicationProof{MisurabileEquivalenze}{SigmaDaFuoriEquivalenze} otteniamo che esiste $C\in\sigma A(\S)$ tale che contiene $X'\setminus A$ e tale che $\mu^*(C\setminus(X'\setminus A))=0$. 
	
	Chiamiamo $B$ l'insieme $X'\setminus C$. L'insieme definito è contenuto in $A$ ed appartiene a $\sigma A(\S)$. Inoltre vale la seguente:
	\begin{equation*}
		A\setminus B\subseteq C\setminus (X'\setminus A) \implies \mu^*(A\setminus B)\le \mu^*\left(C\setminus(X'\setminus A)\right)\le \epsilon
	\end{equation*}
	e questo termina la dimostrazione, visto che $B$ rispetta tutte le richieste.
	
	Infine per l'implicazione \ImplicationProof{SigmaDaDentroEquivalenze}{MisurabileEquivalenze} si procede in maniera completamente analoga alla dimostrazione dell'implicazione \ImplicationProof{SigmaDaFuoriEquivalenze}{MisurabileEquivalenze}.
\end{proof}

\begin{corollary}\label{EquivalenzeMisurabilitaSpazio}
	La \cref{EquivalenzeMisurabilitaSottoinsieme} vale anche senza l'ipotesi che $A$ sia \sigfin[o] rispetto alla premisura, ammesso che lo spazio $X$ lo sia.
\end{corollary}
\begin{proof}
	La dimostrazione è ovvia, visto che se $X$ è \sigfin[o], lo è di conseguenza qualsiasi suo sottoinsieme.
\end{proof}

\begin{corollary}
	Nelle stesse ipotesi di \cref{EquivalenzeMisurabilitaSottoinsieme}, sia $A$ un misurabile \sigfin[o].
	
	Dato un insieme $K$ chiamo limite monotono di $K$, che indico con $\hat K$ l'insieme:
	\begin{equation*}
		\left\{\bigsqcup_{n\in\N}K_n:(K_n)_{n\in\N}\subseteq K\wedge K_n \text{ disgiunti}\right\}\cup
		\left\{\bigcap_{n\in\N}K_n:\ (K_n)_{n\in\N}\subseteq K\ \wedge\ K_{n+1}\subseteq K_n\right\}
	\end{equation*}

	Allora esiste $B\in\hat{\hat\S}$ che coincide con $A$ a meno di un sottoinsieme trascurabile.
\end{corollary}
\begin{proof}
	Poichè siamo nelle stesse ipotesi di \cref{EquivalenzeMisurabilitaSottoinsieme} ed $A$ è misurabile, di certo valgono le varie implicazioni, quindi assumiamone implicitamente la validità.
	
	Notiamo che nella dimostrazione di \ImplicationProof{UnioniDaFuoriEquivalenze}{SigmaDaFuoriEquivalenze} della \cref{EquivalenzeMisurabilitaSottoinsieme} scriviamo $B$, che è per l'appunto un insieme che coincide con $A$ a meno di un sottoinsieme trascurabile, come intersezione di numerabile di $(S_n)_{n\in\N}\subseteq \sqcup \S$.
	Avremmo finito se quest'intersezione fosse monotona, visto che $\sqcup \S\subseteq \hat \S$.
	Ma questo è facile da ottenere, sfruttando questa scrittura:
	\begin{equation*}
		\bigcap_{n\in\N} S_n = \bigcap_{n\in\N} \bigcap_{i\le n} S_n
	\end{equation*}
	dove gli elementi che interseco rispetto ad $n$ appartengono ancora a $\sqcup \S$ poichè questo è chiuso per intersezione finita (come mostrato in \cref{UnioneDisgiuntaQuasiAlgebra}).

\end{proof}



\begin{remark}
	Nell'enunciato del \cref{EquivalenzeMisurabilitaSottoinsieme} sembra mancare simmetria tra le equivalenze: manca la possibilità di approssimare dall'interno con elementi di $\sqcup\S$. Infatti è falso che si possa approssimare dall'interno con elementi di $\sqcup\S$.
\end{remark}
\begin{proof}
	Il controesempio più naturale si troverà studiando la misura di Lebesgue su $\R$. In questo ambiente basta considerare $\R\setminus\mathbb Q$ per trovare un insieme con misura infinita, ma che, se approssimato dall'interno con unioni numerabili di intervalli, appare avere misura nulla. Questo è dovuto al fatto che l'insieme proposto, pur avendo misura non nulla, è a parte interna vuota.
\end{proof}




\begin{proposition}
	Dato $(X,\S,\mu)$ una spazio di misura elementare \sigfin[o], esiste un'unica estensione della premisura ad una misura su $\sigma A(\S)$.
\end{proposition}
\begin{proof}
	L'esistenza di una possibile estensione ci è assicurata dal \cref{EstensioneCaratheodory}.
	
	Per dimostrare l'unicità consideriamo due misure $\mu',\mu'':\sigma A(\S)\to\Rpiu$ che estendono la premisura $\mu$ alla \sigalg{} generata. Fissiamo inoltre $A\in\sigma A(\S)$ e dimostriamo che $\mu'(A)=\mu''(A)$.
	
	Se $\mu'(A)=+\infty=\mu''(A)$ la tesi è già dimostrata,  quindi assumiamo, senza perdità di generalità, $\mu''(A)<+\infty$.
	
	Fissato $\epsilon>0$, grazie a \cref{EquivalenzeMisurabilitaSpazio} ricaviamo che esiste $S\in\sqcup S$ che contiene $A$ tale che $\mu^*(S\setminus A)< \epsilon$ (questo perchè di certo $A$ appartiene ai misurabili secondo Caratheodory, visto che appartiene alla \sigalg{} generata da $\S$).
	
	Ora, sfruttando \cref{PremisuraMassimaMisura}, ricaviamo anche $\mu^*(S\setminus A)\ge\mu'(S\setminus A)$ e un risultato analogo per $\mu''$. Quindi risultano valere $\mu'(S\setminus A)<\epsilon,\ \mu''(S\setminus A)<\epsilon$.
	
	Unendo i risultati ottenuti, ricordando che $\mu',\mu''$ sono estensioni di $\mu$ e che $\mu''(A)<+\infty$,  abbiamo:
	\begin{gather*}
		\mu'(A)+\mu'(S\setminus A)=\mu'(S)=\mu(S)=\mu''(S)=\mu''(A)+\mu''(S\setminus A) \\
		\Downarrow \\
		\lvert\mu'(A)-\mu''(A)\rvert=\lvert\mu''(S\setminus A)-\mu'(S\setminus A)\rvert<2\epsilon
	\end{gather*}
	e visto che questo deve valere per ogni $\epsilon>0$ otteniamo $\mu'(A)=\mu''(A)$ che è quanto si voleva.
\end{proof}

