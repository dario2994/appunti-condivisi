\section{Integrazione secondo Lebesgue}
In questa sezione definiremo la nozione di integrale secondo Lebesgue e dimostreremo alcuni risultati introduttivi. In particolare definiremo inizialmente l'integrale di funzioni misurabili positive per poi estenderlo facilmente alle funzioni misurabili che hanno integrale del loro valore assoluto finito. Chiameremo queste ultime funzioni integrabili.

\begin{remark}
	Indicheremo l'integrale di $f$ su una spazio di misura $(X,\A,\mu)$ con 
	\begin{equation*}
		\int_X f(x) \de \mu(x)=\int_X f \de \mu=\int f \de \mu=\int f
	\end{equation*}
	dove si applicheranno le varie omissioni quando saranno ovvi la variabile di integrazione $x$, lo spazio di misura $X$ o la misura $\mu$.
\end{remark}

\begin{remark}
	Chiameremo funzioni positive quelle che assumono solo valori maggiori o uguali a 0.
\end{remark}



\begin{lemma}
	Data una funzione $\simp$ semplice e positiva su $(X,\A,\mu)$, esistono $c_k>0$ reali, con $k=1,\dots, n$, tale che $\simp=\sum_{k=1}^nc_k\chi(E_k)$, dove $E_k\in \A$.
\end{lemma}


\begin{definition}\label{DefIntSemp}
	Sia $\simp$ una funzione misurabile, semplice e positiva, definita sullo spazio di misura $(X,\A,\mu)$. Definiamo l'integrale di Lebesgue della funzione il valore
	\begin{equation*}
		\int_X \simp d \mu = \sum_{k=1}^n c_k\mu(E_k)
	\end{equation*}
	dove $c_k\in \R$ e $E_k \in \A$ sono tali che $\simp=\sum_{k=1}^nc_k\chi(E_k)$.
\end{definition}

\begin{remark}\label{BuonaDefIntSemp}
	Quella appena enunciata è una buona definizione, cioè non dipende dalla scelta dei $c_k$ e degli $E_k$.
\end{remark}
\begin{proof}
	Consideriamo innanzitutto dei $c_k\in \R$ e degli $E_k\in \A$ tali che $\simp=\sum_{k=1}^nc_k\chi(E_k)$ e definiamo per ogni $\alpha\subseteq\{1,2,\dots,n\}$:
	\begin{equation*}
		\begin{cases}
			E_\alpha=\bigcap_{k\in\alpha}E_k\setminus \bigcup_{k\not\in \alpha} E_k\\
			c_\alpha=\sum_{k\in\alpha} c_k
		\end{cases}
	\end{equation*}
	D'ora in poi nelle sommatorie sottointenderemo l'insieme su cui cicla $\alpha$, cioè $\mathcal{P}(\{1,2,\dots,n\})$.
	
	Innanzitutto gli $E_\alpha$ sono ancora insiemi misurabili in quanto $\A$ è una \sigalg{} e sono insiemi disgiunti, inoltre vale facilmente che:
	\begin{equation*}
		\simp=\sum_{k=1}^nc_k\chi(E_k)=\sum_{k=1}^nc_k\sum_{\alpha\ni k}\chi(E_\alpha)=\sum_\alpha \chi(E_\alpha)\sum_{k\in\alpha}c_k=\sum_\alpha c_\alpha\chi(E_\alpha)
	\end{equation*}
	dove ho utilizzato che $E_k=\cup_{\alpha\ni k} E_\alpha$.

	Vogliamo dimostrare che $\sum_{k=1}^nc_k\mu(E_k)=\sum_{\alpha}c_\alpha\mu(E_\alpha)$, ma per l'additività di $\mu$ su insiemi disgiunti otteniamo:
	\begin{equation}\label{RaffinamentoIntegrale}
		\sum_\alpha c_\alpha\mu(E_\alpha)=\sum_\alpha \sum_{k\in\alpha}c_k\mu(E_\alpha)=\sum_{k=1}^n c_k\sum_{\alpha \ni k} \mu(E_\alpha)=\sum_{k=1}^n c_k \mu(E_k)
	\end{equation}
	che è proprio quello che volevamo.
	
	Consideriamo ora $c_k, b_k\in\R$ ed $E_k,F_k\in\A$ tali che $\simp=\sum_{k=1}^nc_k\chi(E_k)=\sum_{k=1}^mb_k\chi(F_k)$. A meno di aggiungere dei termini ad entrambe le sommatorie possiamo supporre $E_k=F_k$ e $n=m$, da cui $\simp=\sum_{k=1}^nc_k\chi(E_k)=\sum_{k=1}^nb_k\chi(E_k)$. Applichiamo la costruzione di prima e otteniamo:
	\begin{equation*}
		\begin{cases}
			E_\alpha=\bigcap_{k\in\alpha}E_k\setminus \bigcup_{k\not\in \alpha} E_k\\
			c_\alpha=\sum_{k\in\alpha} c_k\\
			b_\alpha=\sum_{k\in\alpha} b_k
		\end{cases}
	\end{equation*}
	da cui ricaviamo per quanto detto precedentemente che:
	\begin{equation}
		\sum_\alpha c_\alpha\chi(E_\alpha)=\sum_{k=1}^nc_k\chi(E_k)=\simp=\sum_{k=1}^nb_k\chi(E_k)=\sum_\alpha b_\alpha\chi(E_\alpha)
	\end{equation}
	che implica facilmente che $c_\alpha=b_\alpha$ per ogni $\alpha$, poichè gli $E_\alpha$ sono disgiunti.
	
	Infine da quest'ultima uguaglianza e dalla \cref{RaffinamentoIntegrale} otteniamo proprio quello che volevamo, in quanto:
	\begin{equation*}
		\sum_{k=1}^n c_k \mu(E_k)=\sum_\alpha c_\alpha\mu(E_\alpha)=\sum_\alpha b_\alpha\mu(E_\alpha)=\sum_{k=1}^n b_k \mu(E_k)
	\end{equation*}
\end{proof}

\begin{proposition}\label{PropIntSemp}
	Date $\simp_1,\simp_2$ funzioni semplici e positive su $(X,\A,\mu)$ e dati $a,b$ reali non negativi valgono:
	\begin{enumerate}
		\item $\int (a\simp_1+b\simp_2)\de\mu=a\int \simp_1\de\mu+b\int \simp_2\de\mu$; \label{PIS:add}
		\item $\simp_1\le \simp_2\Longrightarrow \int \simp_1\de\mu\le \int \simp_2\de\mu$. \label{PIS:mono}
	\end{enumerate}
\end{proposition}
\begin{proof}
	Seguono entrambe facilmente dalla definizione di integrale di funzioni semplici e dalla \cref{BuonaDefIntSemp}.
\end{proof}

\begin{proposition}\label{IntSempSpezzato}
	Siano $X_1,X_2,\dots,X_k$ misurabili e disgiunti tali che $X=\cup_{m=1}^kX_m$ e sia $\simp$ una funzione semplice e positiva, allora
	\begin{equation*}
		\int_X \simp \de \mu=\sum_{m=1}^k \int_{X_m}\simp\de \mu
	\end{equation*}
\end{proposition}
\begin{proof}
	Per ogni $X_m$ siano $c_{m,i}\in\R$ e $E_{m,i}\in\A$, per $i=1,2,\dots,n_m$, tali che 
	\begin{equation*}
		\simp|_{X_m}=\sum_{i=1}^{n_m}c_{m,i}\chi_{E_{m,i}}
	\end{equation*}
	allora vale
	\begin{equation*}
		\simp=\sum_{m=1}^k\sum_{i=1}^{n_m}c_{m,i}\chi_{E_{m,i}}
	\end{equation*}
	da cui quindi ottengo facilmente
	\begin{equation*}
		\int_X \simp \de \mu=\sum_{m=1}^k\sum_{i=1}^{n_m}c_{m,i}\mu(E_{m,i})=\sum_{m=1}^k\int_{X_m}\simp\de \mu
	\end{equation*}

\end{proof}



\begin{definition}\label{DefIntPos}
	Sia $f$ una funzione misurabile positiva su $(X,\A,\mu)$, allora definiamo l'integrale di $f$ come
	\begin{equation*}
		\int_Xf\de\mu=\sup_{n\in\N}\left\{\int_X\simp\de \mu:\ 0\le \simp\le f\ \wedge\ \simp \text{ semplice}\right\}
	\end{equation*}
\end{definition}

\begin{proposition}
	Data una successione crescente di funzioni semplici e positive $(\simp_n)$ tali che convergono puntualmente a $f$, vale 
	\begin{equation*}
		\int_X f\de\mu=\sup_{n\in\N}\left\{\int_X\simp_n\de\mu\right\}
	\end{equation*}
\end{proposition}
\begin{proof}
	Dimostriamo innanzitutto che dato un insieme $Y\subseteq X$ misurabile tale che $f\ge a$ in $Y$, allora vale $\sup_{n\in\N}\left\{\int_Y\simp_n\de\mu\right\}\ge a\mu(Y)$.
	
	Dato $\epsilon>0$ sia $A_n=\{\simp_n\ge a-\epsilon\}\cap Y$. Allora abbiamo che gli $A_n$ sono misurabili, vale $A_n\subseteq A_{n+1}$, perchè le $\simp_n$ sono crescenti, e $\cup_nA_n=Y$. Inoltre $\simp_n\ge (a-\epsilon)\chi_{A_n}$ in $Y$ per la definizione stessa di $A_n$. Perciò per \cref{PropIntSemp} ho che
	\begin{equation*}
		\int_Y\simp_n\de\mu\ge(a-\epsilon)\mu(A_n)
	\end{equation*}
	da cui passando al limite su $n$ ed utilizzando che $\lim_{n\to\infty}\int_Y\simp_n\de\mu=\sup_{n\in\N}\int_Y\simp_n\de\mu$, otteniamo
	\begin{equation*}
		\sup_{n\in\N}\left\{\int_Y\simp_n\de\mu\right\}\ge(a-\epsilon)\lim_{n\to\infty}\mu(A_n)=(a-\epsilon)\mu(Y)
	\end{equation*}
	dove in particolare nell'ultima uguaglianza abbiamo utilizzato \cref{LimiteMonotonoCrescenteMisura}. Poichè in quest'ultima relazione $\epsilon$ è arbitrario otteniamo quindi
	\begin{equation*}
		\sup_{n\in\N}\left\{\int_Y\simp_n\de\mu\right\}\ge a\mu(Y)
	\end{equation*}

	Consideriamo ora una qualsiasi funzione semplice $\simp$ tale che $0\le \simp\le f$. Per facile conseguenza della definizione di funzione semplice, esistono $k$ sottoinsiemi misurabili disgiunti di $X$, che chiamiamo $X_1,\dots,X_k$, tali che $\cup_{m=1}^kX_m=X$ e $\simp|_{X_m}$ è costante per ogni $m=1,\dots,k$.
	
	Per quanto dimostrato precedentemente abbiamo che $\sup_{n\in\N}\left\{\int_{X_m}\simp_n\de\mu\right\}\ge a_m\mu(X_m)$ per ogni $m=1,\dots,k$, dove $a_m=\simp|_{X_m}$. Unendo questo risultato a \cref{IntSempSpezzato} e ricordando che i $\sup$ sono in realtà dei limiti, otteniamo:
	\begin{equation*}
		\sup_{n\in\N}\left\{\int_X\simp_n\de\mu\right\}=\sup_{n\in\N}\left\{\sum_{m=1}^k \int_{X_m}\simp_n\de \mu\right\}=\sum_{m=1}^k\sup_{n\in\N}\left\{ \int_{X_m}\simp_n\de \mu\right\}\ge \sum_{m=1}^ka_m\mu(X_m)=\int_X\simp\de\mu
	\end{equation*}
	Passando ora all'estremo superiore per $0\le \simp \le f$ otteniamo proprio:
	\begin{equation*}
		\sup_{n\in\N}\left\{\int_X\simp_n\de\mu\right\}\ge \int_X f \de\mu\Longrightarrow \sup_{n\in\N}\left\{\int_X\simp_n\de\mu\right\}= \int_X f\de\mu
	\end{equation*}
	dove l'ultima implicazione è ovvia perchè banalmente $\sup_{n\in\N}\left\{\int_X\simp_n\de\mu\right\}\le \int_X f\de\mu$.
\end{proof}

\begin{theorem}[Beppo Levi]
	Sia $(f_n)$ una successione crescente di funzioni misurabili positive convergenti puntualmente ad una funzione $f$, allora $\int_Xf\de\mu=\lim_{n\to\infty}\int_Xf_n\de\mu$.
\end{theorem}

\begin{proof}
	Innanzitutto notiamo che ha senso parlare dell'integrale di $f$, perchè essendo $f=\sup_{n\in\N}f_n$ per la \cref{SupMis} $f$ è una funzione misurabile e positiva.
\end{proof}







