\section{Misura prodotto}
Trattiamo ora la possibilità di rendere uno spazio di misura il prodotto di due spazi di misura. 

Questo risulterà facile sfruttando, come strumento principale, il teorema di estensione di Caratheodory. 
Nella dimostrazione del primo teorema, che fondamentalmente verifica le ipotesi di Caratheodory, proponiamo una via tecnica (che sfrutta la teoria degli integrali costruita), che diverge da quello che potrebbe essere un modo standard di procedere. Questo è sia più breve di una dimostrazione fatta con le mani, sia rende chiara da subito la possibilità di avere scambi tra gli operatori integrali in uno spazio prodotto. 

Infine dimostreremo i teoremi di Fubini e Tonelli, cioè la possibilità di scambiare tra loro gli operatori di integrazione sotto opportune ipotesi.

\newcommand{\B}{\ensuremath{\mathscr B}}
\begin{theorem}
	Dati $(X,\A,\mu)$ e $(X,\B,\nu)$ spazi di misura, definiamo la funzione $\mu\nu:\A\times\B\to\Rpiu$ come il prodotto delle misure $\mu\nu(A\times B)=\mu(A)\nu(B)$.
	
	Allora $(X\times Y,\A\times\B,\mu\nu)$ è uno spazio di misura elementare.
\end{theorem}
