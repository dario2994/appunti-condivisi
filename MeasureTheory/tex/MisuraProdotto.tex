\section{Misura prodotto}
Trattiamo ora la possibilità di rendere uno spazio di misura il prodotto di due spazi di misura. 
Questo risulterà facile sfruttando, come strumento principale, il teorema di estensione di \carat{}. 
Si potrebbe decidere di esplicitare direttamente la misura prodotto, come integrale della misura delle sezioni, piuttosto che sfruttare il teorema di \carat{} per assicurarne l'esistenza. 
Questa via, più diretta e meno tecnica, è effettivamente presa da vari libri di teoria della misura. Qui si è deciso di procedere diversamente sia per ignoranza iniziale degli autori, sia perchè in effetti si dimostra, ed è a nostro parere molto istruttivo, che le due definizioni sono equivalenti.

L'idea portante di tutta questa sezione è quella di mostrare che la misura più ovvia da porre sullo spazio prodotto, cioè quella derivante dalla premisura sui rettangoli, è in effetti sufficientemente  ricca da assicurare molte proprietà allo spazio di misura a essa relativo ed in particolare ha sufficienti proprietà per dimostrare i teoremi di Fubini e Tonelli, che assicurano, sotto ipotesi relativamente lascive, la possibilità di scambiare tra loro gli operatori di integrazione.

È importante enfatizzare da subito come vi sia una certa arbitrarietà nel decidere quale sia la \sigalg{} dei misurabili sullo spazio prodotto. Si potrebbe decidere di lavorare sulla \sigalg{} di \carat{}, ma questa risulterebbe troppo ampia, oppure lavorare sulla \sigalg{} generata dai rettangoli, che invece risulta spesso (in particolare nel caso fondamentale della misura di Lebesgue) troppo ristretta.
L'ambiente giusto in cui lavorare risulterà infatti essere la \sigalg{} generata dai rettangoli nel caso in cui le misure iniziali non siano complete, mentre sarà più adatto lavorare nel completamento di questa \sigalg{} nel caso in cui le misure siano complete.



Nella dimostrazione del primo teorema, che fondamentalmente verifica le ipotesi di \carat{}, proponiamo una via tecnica (che sfrutta la teoria degli integrali), che diverge da quello che potrebbe essere un modo standard di procedere. Questo è sia più breve di una dimostrazione fatta con le mani, sia rende chiara da subito la possibilità di avere scambi tra gli integrali in uno spazio prodotto. 
\newcommand{\B}{\ensuremath{\mathscr B}}
\begin{definition}\label{def:SezioneProdotto}
	Dati due insiemi $X,Y$ e $x\in X$ fissato, indichiamo con la notazione $J^X_x$ la funzione $J^X_x:\mathcal P(X\times Y) \to \mathcal P(Y)$ che restituisce le sezioni di un insieme:
	\begin{equation*}
		\forall E\subseteq X\times Y:\ J^X_x(E)=\{y\in Y:\ (x,y)\in E\}\punto
	\end{equation*}
	Dove ovvio verrà omesso l'apice che indica l'insieme.
\end{definition}

\begin{definition}\label{def:SemianelloProdotto}
	Dati due spazi misurabili $(X,\A)$, $(Y,\B)$ definiamo, con abuso di notazione, $\A\times\B$ come la famiglia degli insiemi $A\times B$ dove $A\in\A$ e $B\in\B$, che chiameremo rettangoli.
\end{definition}

\begin{proposition}\label{prop:SemianelloProdotto}
	Dati due spazi misurabili $(X,\A)$, $(Y,\B)$, la famiglia $\A\times\B\subseteq \mathcal P(X\times Y)$ è un \semiring{}.
\end{proposition}
\begin{proof}
	Ovviamente $\emptyset\in\A\times\B$ visto che $\emptyset\in\A$ e $\emptyset\in\B$.
	
	Per concludere, dati $A\times B, A'\times B'\in\A\times\B$, mostriamo esplicitamente una scrittura dell'intersezione e della differenza che rispetti le proprietà che deve avere un \semiring{}:
	\begin{align*}
		(A\times B)\cap (A'\times B') &= (A\cap A')\times(B\cap B')\virgola \\
		(A\times B)\setminus (A'\times B') &= ((A\setminus A')\times B)\sqcup( (A\cap A')\times (B\setminus B') )\punto
	\end{align*}
\end{proof}

\begin{definition}\label{def:PremisuraProdotto}
	Dati $(X,\A,\mu)$ e $(Y,\B,\nu)$ spazi di misura, denotiamo con $\overline{\mu\nu}$ la funzione $\overline{\mu\nu}:\A\times\B\to\Rpiu$ definita in modo che 
	\begin{equation*}
		\forall A\times B\in\A\times\B:\ \overline{\mu\nu}(A\times B)=\mu(A)\nu(B)\punto
	\end{equation*}
\end{definition}

\begin{theorem}\label{thm:PremisuraProdotto}
	Dati $(X,\A,\mu)$ e $(Y,\B,\nu)$ spazi di misura, allora $(X\times Y,\A\times\B,\overline{\mu\nu})$ è uno spazio di misura elementare, dove $\A\times\B$ e $\overline{\mu\nu}$ sono definiti nelle \cref{def:SemianelloProdotto,def:PremisuraProdotto}.
\end{theorem}
\begin{proof}
	Avendo già verificato che $\A\times\B$ è un \semiring{} nella \cref{prop:SemianelloProdotto}, è sufficiente mostrare che $\overline{\mu\nu}$ è \sigadd{}.
	
	Sia $(A_n\times B_n)_{n\in\N}\subseteq \A\times \B$ una partizione disgiunta di $A\times B$.
	
	Per ogni $x\in X$, essendo gli $A_n\times B_n$ disgiunti, risulta
	\begin{equation}\label{eq:PremisuraProdottoSerie}
		\bigsqcup_{n\in\N} J_x(A_n\times B_n) = J_x(A\times B)  \implies \sum_{n\in\N}\nu\left(J_x(A_n\times B_n)\right)=\nu\left(J_x(A\times B)\right)\virgola
	\end{equation}
	dove nell'implicazione abbiamo applicato la \sigadd[ità] di $\nu$.
	
	Applicando la sola definizione della misura prodotto e dell'integrale per funzioni semplici, fissato $C\times D\in \A\times\B$, abbiamo
	\begin{equation}\label{eq:StupidaIdentitaIndicatrici}
		\overline{\mu\nu}(C\times D)=\mu(C)\nu(D)=\int_{X}\chi_C(x)\nu(D)\de \mu(x)=\int_X \nu\left(J_x(C\times D)\right)\de\mu(x)\virgola
	\end{equation}
	dove l'ultima uguaglianza segue dalla facile identità $\nu\left(J_x(C\times D)\right)=\chi_C(x)\nu(D)$.
	
	Ora unendo le \cref{eq:PremisuraProdottoSerie,eq:StupidaIdentitaIndicatrici}, sfruttando il \cref{cor:IntegrazionePerSeriePositive}, otteniamo
	\begin{align*}
		\overline{\mu\nu}(A\times B)&=\int_X\nu(J_x(A\times B))\de \mu(x)=\int_X\sum_{n\in\N}\nu\left(J_x(A_n\times B_n)\right)\de\mu(x)\\
		&=\sum_{n\in\N}\int_X\nu\left(J_x(A_n\times B_n)\right)\de\mu(x)=\sum_{n\in\N}\overline{\mu\nu}(A_n\times B_n)\virgola
	\end{align*}
	che è la tesi cercata.
\end{proof}

La proposizione seguente è la prima che rende chiaro il fatto che l'ambiente di lavoro finale potrà essere vario. Infatti piuttosto che dimostrarla per soli insiemi misurabili (non ancora definiti) la si deve dimostrare per insiemi trascurabili qualunque, per poi poterla sfruttare indipendentemente da quale \sigalg{} dei misurabili verrà scelta.

\begin{proposition}\label{prop:TrascurabiliProdotto}
	Dato lo spazio di premisura $(X\times Y,\A\times\B,\overline{\mu\nu})$, che si è mostrato essere di premisura nel \cref{thm:PremisuraProdotto}, sia $\mu\nu^*:\mathcal P(X\times Y)\to\Rpiu$ la misura esterna associata a $\overline{\mu\nu}$.
	
	Se $E\subseteq X\times Y$ è trascurabile rispetto a $\mu\nu^*$, allora per $\mu$-quasi ogni $x\in X$ l'insieme $J_x^X(E)$ è trascurabile\footnote{Il fatto che un insieme sia trascurabile non implica che sia misurabile come osservato nella \cref{def:TrascurabiliMisura}.} rispetto alla misura $\mu$.
\end{proposition}
\begin{proof}
	Chiamiamo $\mu^*,\nu^*$ le misure esterne associate alle misure $\mu,\nu$, definite come nella \cref{prop:MisuraEsternaDiPremisura} considerando $\mu,\nu$ delle premisure\footnote{Uno spazio di misura è sempre ovviamente anche uno spazio di premisura.}.
	
	Per $\delta>0$, sia $X_\delta$ l'insieme degli $x\in X$ tali che $\nu^*(J_x^X(E))\ge\delta$ e sia $m_\delta=\mu^*(X_\delta)$.
	
	Essendo $E$ trascurabile, se per assurdo $m_\delta>0$, esiste una famiglia di rettangoli $(A_i\times B_i)_{i\in\N}\subseteq \A\times\B$ la cui unione contiene $E$ e tali che
	\begin{equation*}
		\sum_{i\in\N}\mu(A_i)\nu(B_i)=\sum_{i\in\N}\overline{\mu\nu}(A_i\times B_i)< \delta m_\delta \punto
	\end{equation*}
	Chiamiamo $f:X\to\Rpiu$ la funzione definita come
	\begin{equation*}
		f(x)=\sum_{i\in\N}\chi_{A_i}(x)\nu(B_i)
	\end{equation*}
	e notiamo subito che $f$ è una funzione misurabile positiva, per la \cref{prop:SupDiMisurabili}, essendo una serie numerabile di funzioni indicatrici di misurabili.
	
	Per la sola definizione di misura esterna si ottiene che per ogni $x\in X$
	\begin{equation*}
		f(x)=\sum_{i\in\N}\chi_{A_i}(x)\nu(B_i)\ge\nu^*(J_x^X(E))\virgola
	\end{equation*}
	poichè i $B_i$, tali che i rispettivi $A_i$ contengano $x$, sono un ricoprimento numerabile di $J_x^X(E)$.
	
	Allora, per la monotonia della misura esterna, ne segue banalmente che l'insieme $\{x\in X:f(x)\ge\delta\}$ ha misura\footnote{Qui si può parlare di misura e non di misura esterna poichè $f$ è una funzione misurabile.} maggiore o uguale a $m_\delta$.
	Perciò applicando il \cref{thm:DisuguaglianzaChebyshev}, ne ricaviamo che $\LNorm f\ge \delta m_\delta$.

	Però, applicando il \cref{cor:IntegrazionePerSeriePositive}, otteniamo
	\begin{equation*}
		\delta m_\delta \le\LNorm f=\int_X \sum_{i\in\N}\chi_{A_i}(x)\nu(B_i)\de\mu(x)= \sum_{i\in\N}\int_X \chi_{A_i}(x)\nu(B_i)\de\mu(x)=\sum_{i\in\N}\mu(A_i)\nu(B_i)<\delta m_\delta\virgola
	\end{equation*}
	che mostra l'assurdo e dimostra $m_\delta=0$.
	
	Per concludere basta notare che l'insieme degli $x\in X$ tali che $\nu^*(J_x^X(E))>0$ si ottiene come unione numerabile degli $X_{\frac 1i}$, che sono tutti trascurabili, ed è perciò trascurabile anch'esso per la \cref{nota:UnioneTrascurabili} come voluto.
\end{proof}

\begin{definition}\label{def:SigAlgProdotto}
	Dati $(X,\A,\mu)$ e $(Y,\B,\nu)$ spazi di misura, chiamiamo $\A\otimes\B\subseteq \mathcal P(X\times Y)$ la \sigalg{} generata da $\A\times\B$. 
\end{definition}

\begin{proposition}\label{prop:SezioniMisurabili}
	Dati $(X,\A,\mu)$ e $(Y,\B,\nu)$ spazi di misura ed $E\in\A\otimes\B$, le sue sezioni trasversali sono misurabili, cioè
	\begin{align*}
		\forall x\in X:\ &J^X_x(E)\in\B \virgola \\
		\forall y\in Y:\ &J^Y_y(E)\in\A \punto
	\end{align*}
\end{proposition}
\begin{proof}
	\newcommand{\C}{\ensuremath{\mathscr C}}
	Sia $\C\subset\mathcal P(X\times Y)$, l'insieme dei sottoinsiemi di $X\times Y$ tali che tutte le sezioni trasversali (come definite implicitamente nell'enunciato) sono misurabili.
	Dimostriamo ora che $\A\times\B\subseteq \C$ e che $\C$ è una \sigalg{}.
	
	Dato un rettangolo $A\times B\in\A\times B$, è chiaro che le sue sezioni trasversali o sono $A$ o $B$ o vuote e in ogni caso risultano misurabili, perciò vale $A\times B\in\C$ e visto che il rettangolo è stato scelto arbitrariamente se ne ricava $\A\times\B\subseteq \C$.
	
	Dato $E\in\C$, è chiaro che $E^\mathsf{c}\in\C$, visto che le sezioni del complementare sono il complementare delle sezioni e perciò sono anch'esse misurabili.
	
	Data una successione $(E_i)_{i\in\N}$, le sezioni dell'unione $\bigcup_{i\in\N}E_i$ sono facilmente l'unione delle sezioni e quindi anch'esse misurabili. Perciò, per definizione di $\C$, anche $\bigcup_{i\in\N}E_i\in\C$.
	Allora unendo quanto detto si ricava come cercato che $\C$ è una \sigalg{}.
	
	Per concludere basta ricordare che $\A\otimes\B$ è la più piccola \sigalg{} che contiene $\A\times\B$ e visto che $\C$ è una \sigalg{} che contiene $\A\times\B$, deve essere $\A\otimes\B\subseteq\C$. Allora $E\in\A\otimes\B$ implica $E\in\C$ e per definizione di $\C$ questo comporta la tesi.
\end{proof}

\begin{definition}\label{def:MisuraProdotto}
	Dati $(X,\A,\mu)$ e $(Y,\B,\nu)$ spazi di misura, chiamiamo $(X\times Y,\A\otimes\B,\mu\nu)$ lo spazio di misura prodotto, dove $\mu\nu$ è la riduzione alla sola \sigalg{} generata da $\A\times\B$ della misura, associata alla premisura $\overline{\mu\nu}$, assicurataci dal \cref{thm:EstensioneCaratheodory}.
\end{definition}

Ora che abbiamo fissato la \sigalg{} dei misurabili, fermiamoci un attimo a riflettere sulla scelta. Perchè piuttosto non abbiamo scelto il suo completamento? Oppure perchè non abbiamo scelto la \sigalg{} degli insiemi che hanno ogni sezione misurabile? 
Il motivo è che nessuna di queste ci avrebbe assicurato, nel caso generale, la veridicità della \cref{prop:SezioniMisurabili} e dei teoremi che seguiranno.

Ora andiamo a dimostrare i teoremi di Fubini e Tonelli, che permettono di scambiare integrali, nel caso in cui i misurabili siano proprio $\A\otimes\B$. In seguito li ridimostreremo, in maniera più concisa, nel caso in cui le misure di partenza siano complete e i misurabili piuttosto che essere $\A\otimes\B$, sono il suo completamento.

Le dimostrazioni che seguono fondamentalmente vanno per passi e dimostrano la commutazione tra integrali per funzioni gradualmente più complicate: indicatrici di insiemi, semplici, misurabili positive ed infine integrabili.

D'ora in poi parleremo dello spazio di misura prodotto dando per scontato la notazione associata ai singoli spazi di misura, cioè se parliamo di $(X\times Y,\A\otimes\B,\mu\nu)$ è sottointeso che corrisponde agli spazi di misura $(X,\A,\mu)$ e $(Y,\B,\nu)$.

\begin{proposition}\label{prop:PreTonelli}
	Dato lo spazio di misura prodotto $(X\times Y,\A\otimes\B,\mu\nu)$, per ogni insieme misurabile $E\in\A\otimes\B$ \sigfin[o] vale la seguente espressione per la misura:
	\begin{equation*}
		\mu\nu(E)=\int_X\nu\left(J_x^X(E)\right)\de\mu(x)=\int_Y\mu\left(J_y^Y(E)\right)\de\nu(y)\punto
	\end{equation*}
\end{proposition}
\begin{proof}
	\newcommand{\E}{\ensuremath{\mathscr E}}
	Dimostriamo solo la prima uguaglianza della tesi, la seconda risulterà di conseguenza vera per simmetria.
	
	Sia $\E\subseteq \A\otimes\B$ l'insieme degli insiemi misurabili sul prodotto con \emph{misura finita} che rispettano la tesi\footnote{Più correttamente $\E$ contiene gli insiemi tali che gli integrali dell'enunciato sono ben definiti e coincidono} (solo la prima uguaglianza).
	
	Dimostriamo che $\E$ contiene i rettangoli ed è chiuso per intersezione numerabile monotona e unione numerabile disgiunta.
	
	Sfruttando solo le definizioni degli operatori in gioco, si ha
	\begin{equation*}
		\forall A\times B\in\A\times\B:\ \mu\nu(A\times B)=\mu(A)\nu(B)=\int_X \chi_A(x)\nu(B)\de\mu(x)=\int_X\nu\left(J_x^X(E)\right)\de\mu(x)
	\end{equation*}
	e questo dimostra che $\A\times\B\subseteq \E$ come voluto.
	
	Fissati $(E_i)_{i\in\N}\subseteq \E$ disgiunti, è banale verificare
	\begin{equation*}
		\bigsqcup_{i\in\N}J_x^X(E_i)=J_x^X\left(\bigsqcup_{i\in\N}E_i\right)\virgola
	\end{equation*}
	da cui segue, ricordando che la tesi vale per gli $E_i$ ed applicando la commutazione tra integrale e serie mostrata nel \cref{cor:IntegrazionePerSeriePositive}, l'uguaglianza
	\begin{align*}
		\mu\nu\left(\bigsqcup_{i\in\N}E_i\right)&=\sum_{i\in\N}\mu\nu(E_i)=\sum_{i\in\N}\int_X \nu\left(J_x^X(E_i)\right)\de\mu(x)\\
		&=\int_X\sum_{i\in\N}\nu\left(J_x^X\left(E_i\right)\right)\de\mu(x)=\int_X\nu\left(J_x^X\left(\bigsqcup_{i\in\N}E_i\right)\right)\punto
	\end{align*}
	Quest'ultima uguaglianza dimostra, nel caso in cui l'unione degli $E_i$ abbia misura finita, che $\bigsqcup_{i\in\N}E_i\in\E$ e perciò, vista la scelta arbitraria degli $E_i$, ne ricaviamo che $\E$ è chiuso per unione disgiunta.
	
	Infine siano $(E_i)_{i\in\N}\subseteq \E$ che si contengono decrescentemente $E_{i+1}\subseteq E_i$.
	
	Allora, ricordando il \cref{cor:LimiteMonotonoDecrescenteMisura} e il \cref{thm:ConvergenzaDominata}, otteniamo
	\begin{align*}
		\mu\nu\left(\bigcap_{i\in\N}E_i\right)&=\lim_{i\in\N}\mu\nu(E_i)=
		\lim_{i\in\N}\int_X\nu\left(J_x^X(E_i)\right)\de\mu(x)\\
		&=\int_X\lim_{i\in\N}\nu\left(J_x^X(E_i)\right)\de\mu(x)=
		\int_X\nu\left(J_x^X\left(\bigcap_{i\in\N}E_i\right)\right)\de\mu(x)\virgola
	\end{align*}
	dove nei vari passaggi abbiamo usato implicitamente che gli $E_i$ si contengono vicendevolmente.
	È facile verificare che quanto scritto equivale a dire $\bigcap_{i\in\N}E_i\in\E$, perciò, sempre per l'arbitrarietà della scelta degli $E_i$, ne segue che $\E$ è chiuso per intersezione monotona.
	
	Per quanto detto, seguendo la notazione del \cref{cor:ChiusuraMonotonaInsiemiFiniti}, abbiamo dimostrato che $\widehat{\widehat{{\A\times\B}^F}^F}$ è un sottoinsieme di $E$ e perciò appunto per il \cref{cor:ChiusuraMonotonaInsiemiFiniti} fissato $F\in\A\otimes\B$ di misura finita, visto che $\A\otimes\B$ è un sottoinsieme dei misurabili secondo \carat{}, esiste $E\in\E$ che contiene $F$ e tale che $\mu\nu(E\setminus F)=0$. 
	
	Però per la \cref{prop:TrascurabiliProdotto} l'insieme trascurabile $E\setminus F$ ha quasi ogni sezione trascurabile a sua volta.
	Allora applicando la \cref{prop:SezioniMisurabili} e ricordando il \cref{lemma:L1NullaAlloraNulla} ne discende banalmente che $E\setminus F\in\E$.
	Perciò vale
	\begin{align*}
		\mu\nu(F)&=\mu\nu(E)-\mu\nu(E\setminus F)=\int_X\nu\left(J_x^X(E)\right)\de\mu(x)-\int_X\nu\left(J_x^X(E\setminus F)\right)\de\mu(x)\\
		&=\int_X\nu\left(J_x^X(E\setminus F)\right)-\nu(\left(J_x^X(E)\right)\de\mu(x)
		=\int_X\nu\left(J_x^X(F)\right)\de\mu(x)\virgola
	\end{align*}
	dove nei passaggi intermedi abbiamo ripetutamente usato che $E$ contiene $F$.
	Ma l'identità equivale a $F\in\E$ e perciò, poichè $F$ è un qualsiasi insieme misurabile sul prodotto con misura finita, è dimostrato che ogni insieme di misura finita appartiene a $\E$.
	
	Per concludere la dimostrazione sia $G\in\A\otimes\B$ un insieme misurabile \sigfin[o]. Per l'ipotesi di \sigfin[ezza], esiste una famiglia $(F_i)_{i\in\N}\subseteq \A\otimes\B$ di insiemi disgiunti di misura finita la cui unione coincide con $G$.
	
	Ricordando la dimostrazione del fatto che $\E$ è chiuso per unione disgiunta, notiamo che abbiamo già dimostrato che l'unione disgiunta di elementi di $\E$ rispetta anch'essa la tesi, indipendentemente dal fatto che sia anch'essa di misura finita o meno. Questo però dimostra che la tesi vale anche per $G$ visto che $F_i\in\E$, essendo misurabili di misura finita, e $G$ si può scrivere come unione numerabile disgiunta degli $F_i$. 
\end{proof}

\begin{theorem}[Tonelli]\label{thm:Tonelli}
	Dato $(X\times Y,\A\otimes\B,\mu\nu)$ uno spazio prodotto, per ogni funzione $f:X\times Y\to\Rbar$ misurabile positiva a supporto \sigfin[o] vale
	\begin{equation*}
		\int_{X\times Y}f(u)\de\mu\nu(u)=\int_X\int_Y f(x,y)\de\nu(y)\de\mu(x)=\int_Y\int_X f(x,y)\de\mu(x)\de\nu(y)\punto
	\end{equation*}
\end{theorem}
\begin{proof}
	\newcommand{\E}{\ensuremath{\mathscr E}}
	Sia $\E$ l'insieme delle funzioni misurabili con supporto \sigfin[o] che rispettano la prima uguaglianza della tesi (è ovvio che se dimostriamo la prima uguaglianza, la seconda segue per simmetria).
	
	La \cref{prop:PreTonelli} implica che, per ogni $E\in\A\otimes\B$ a supporto \sigfin[o], vale
	\begin{equation*}
		\int_{X\times Y}\chi_E(u)\de\mu\nu(u)=\mu\nu(E)=\int_X\nu\left(J_x^X(E)\right)\de\mu(x)=\int_X\int_Y\chi_E(x,y)\de\nu(y)\de\mu(x)\virgola
	\end{equation*}
	che equivale a dire che le funzioni indicatrici dei misurabili \sigfin[i] appartengono a $\E$.
	
	Inoltre, per la linearità dell'operatore di integrale, è facile convincersi che $\E$, visto che contiene le indicatrici degli insiemi misurabili \sigfin[i], contiene anche le funzioni semplici positive a supporto \sigfin[o].
	
	Per concludere sia $f:X\times Y\to\Rbar$ una funzione misurabile a supporto \sigfin[o].
	Per il \cref{cor:LimSemCrescMis} esiste una famiglia crescente $(f_i)_{i\in\N}$ di funzioni misurabili positive che convergono puntualmente ad $f$. Essendo $f$ a supporto \sigfin[o] ed essendo le $f_i$ sempre minori di $f$ anche il loro supporto è \sigfin[o] e perciò $f_i\in\E$ per ogni $i\in\N$.
	
	Infine, applicando ripetutamente il \cref{thm:BeppoLevi} e ricordando la monotonia dell'integrale, otteniamo
	\begin{align*}
		\int_{X\times Y}f(u)\de\mu\nu(u)
		&=\lim_{i\in\N}\int_{X\times Y}f_i(u)\de\mu\nu(u)
		=\lim_{i\in\N}\int_X\int_Y f_i(x,y)\de\nu(y)\de\mu(x)\\
		&=\int_X\left(\lim_{i\in\N}\int_Y f_i(x,y)\de\nu(y)\right)\de\mu(x)
		=\int_X\int_Y\lim_{i\in\N} f_i(x,y)\de\nu(y)\de\mu(x)\\
		&=\int_X\int_Y f(x,y)\de\nu(y)\de\mu(x)
	\end{align*}
	e questo dimostra $f\in\E$ concludendo la dimostrazione.
\end{proof}

\begin{theorem}[Fubini]\label{thm:Fubini}
	Dato $(X\times Y,\A\otimes\B,\mu\nu)$ uno spazio prodotto, per ogni funzione $f:X\times Y\to\Rbar$ integrabile vale
	\begin{equation*}
		\int_{X\times Y}f(u)\de\mu\nu(u)=\int_X\int_Y f(x,y)\de\nu(y)\de\mu(x)=\int_Y\int_X f(x,y)\de\mu(x)\de\nu(y)\virgola
	\end{equation*}
	dove le funzioni $f(x,\cdot):Y\to\Rpiu$ e $f(\cdot,y):X\to\Rpiu$ risultano integrabili solo quasi ovunque e perciò anche gli integrali $\int_Y f(x,y)\de\nu(y)$ e $\int_X f(x,y)\de\mu(x)$ risultano definiti solo quasi ovunque, ma questo non cambia il valore dell'integrale esterno in virtù del \cref{cor:IntegraleAMenoDiTrascurabili}.
\end{theorem}
\begin{proof}
	Ancora una volta dimostriamo solo la prima uguaglianza, visto che la seconda segue poi per simmetria.
	
	Decompongo, come nella \cref{def:IntegraleIntegrabili}, la funzione $f$ nelle funzioni positive $f^+,f^-:X\times Y\to\Rbar$.
	Le funzioni $f^+,f^-$ sono integrabili a loro volta, come osservato nella \cref{nota:FpiuFmenoIntegrabili}, e perciò per il \cref{cor:SupportoIntegrabile} hanno supporto \sigfin[o].
	
	Allora da qui si conclude banalmente sfruttando la linearità dell'operatore di integrale e il \cref{thm:Tonelli}:
	\begin{align*}
		\int_{X\times Y}f(u)\de\mu\nu(u)
		& =\int_{X\times Y}f^+(u)\de\mu\nu(u)-\int_{X\times Y}f^-(u)\de\mu\nu(u)\\
		&=\int_X\int_Y f^+(x,y)\de\nu(y)\de\mu(x)-\int_X\int_Y f^-(x,y)\de\nu(y)\de\mu(x)\\
		&=\int_X\left(\int_Y f^+(x,y)\de\nu(y)-\int_Y f^-(x,y)\de\nu(y)\right)\de\mu(x)\\
		&=\int_X\int_Y f^+(x,y)-f^-(x,y)\de\nu(y)\de\mu(x)
		=\int_X\int_Y f(x,y)\de\nu(y)\de\mu(x)\virgola
	\end{align*}
	dove abbiamo implicitamente sfruttato che, come osservato nella \cref{prop:IntegraleIntegrabili}, il fatto che la somma di funzioni integrabili sia definita solo quasi ovunque non influisce sul valore dell'integrale doppio.
\end{proof}

Ora che abbiamo terminato la trattazione nel caso in cui i misurabili siano $\A\otimes\B$, tratteremo il caso in cui entrambe le misure iniziali sono complete e $\A\otimes\B$ piuttosto che essere la \sigalg{} generata dai rettangoli è il suo completamento rispetto alla misura prodotto.
Ridimostreremo perciò gli ultimi tre risultati soffermandoci unicamente sulle differenze negli enunciati e nelle dimostrazioni nel caso di misure complete.

\emph{D'ora in poi assumiamo implicitamente che $(X,\A,\mu)$ e $(Y,\B,\nu)$ siano spazi di misura completi.}

\begin{definition}\label{def:MisuraProdottoCompleta}
	Indichiamo con $(X\times Y,\overline{\A\otimes\B},\mu\nu)$\footnote{Compiamo ora un leggerissimo abuso di notazione, visto che $\mu\nu$ è già la misura su $\A\otimes\B$ piuttosto che sul suo completamento} il completamento dello spazio di misura $(X\times Y,\A\otimes\B,\mu\nu)$.
\end{definition}

\begin{proposition}\label{prop:PreTonelliCompleto}
	Dato lo spazio di misura prodotto $(X\times Y,\overline{\A\otimes\B},\mu\nu)$, per ogni insieme misurabile $E\in\overline{\A\otimes\B}$ \sigfin[o] vale la seguente espressione per la misura:
	\begin{equation*}
		\mu\nu(E)=\int_X\nu\left(J_x^X(E)\right)\de\mu(x)=\int_Y\mu\left(J_y^Y(E)\right)\de\nu(y)\virgola
	\end{equation*}
	dove le funzioni integrande $\nu\left(J_x^X(E)\right)$ e $\mu\left(J_y^Y(E)\right)$ sono definite solo quasi ovunque, ma questo non cambia il valore dell'integrale in virtù del \cref{cor:IntegraleAMenoDiTrascurabili}.
\end{proposition}
\begin{proof}
	La dimostrazione procede pedissequamente a quella della \cref{prop:PreTonelli}, tranne che quando si afferma che grazie alla \cref{prop:TrascurabiliProdotto} gli insiemi trascurabili risultano avere quasi tutte le sezioni misurabili di misura nulla. Infatti nel dimostrarlo si sfruttava la \cref{prop:SezioniMisurabili}, che in questo caso non si può più applicare.
	Ridimostriamo perciò il fatto citato in questo caso.
	
	Sia $N\in\overline{\A\otimes\B}$ trascurabile. Allora per la \cref{prop:TrascurabiliProdotto}, per $\mu$-quasi ogni $x\in X$ l'insieme $J_x^X(N)$ risulta trascurabile. Ma allora per la completezza della misura $\mu$,  si ottiene anche che quasi ovunque $J_x^X(N)$ è misurabile di misura nulla e questo è proprio quanto serve.
	
	È importante notare che è proprio in questo passaggio che si sfrutta la completezza delle misure ed è sempre qui che diventa necessaria la precisazione nell'enunciato dell'esistenza solo quasi ovunque delle funzioni integrande.
\end{proof}

\begin{theorem}[Tonelli]\label{thm:TonelliCompleto}
	Dato $(X\times Y,\overline{\A\otimes\B},\mu\nu)$ uno spazio prodotto, per ogni funzione $f:X\times Y\to\Rbar$ misurabile positiva a supporto \sigfin[o] vale
	\begin{equation*}
		\int_{X\times Y}f(u)\de\mu\nu(u)=\int_X\int_Y f(x,y)\de\nu(y)\de\mu(x)=\int_Y\int_X f(x,y)\de\mu(x)\de\nu(y)\virgola
	\end{equation*}
	dove le funzioni $f(x,\cdot):Y\to\Rpiu$ e $f(\cdot,y):X\to\Rpiu$ risultano misurabili solo quasi ovunque e perciò anche gli integrali $\int_Y f(x,y)\de\nu(y)$ e $\int_X f(x,y)\de\mu(x)$ risultano definiti solo quasi ovunque, ma questo non cambia il valore dell'integrale esterno in virtù del \cref{cor:IntegraleAMenoDiTrascurabili}.
\end{theorem}
\begin{proof}
	Anche in questo caso la dimostrazione ricopia quella del \cref{thm:Tonelli}, con la lieve differenza che, al posto di sfruttare la \cref{prop:PreTonelli}, sfrutta la \cref{prop:PreTonelliCompleto} che ha ovviamente enunciato analogo solo che tratta il caso di misure complete. 
	Proprio per il fatto che sfrutta la \cref{prop:PreTonelliCompleto}, che assicura la definizione delle funzioni solo quasi ovunque, nell'enunciato si deve specificare che le funzioni e i relativi integrali sono definiti solo quasi ovunque.
\end{proof}

\begin{theorem}[Fubini]\label{thm:FubiniCompleto}
	Dato $(X\times Y,\overline{\A\otimes\B},\mu\nu)$ uno spazio prodotto, per ogni funzione $f:X\times Y\to\Rbar$ integrabile vale
	\begin{equation*}
		\int_{X\times Y}f(u)\de\mu\nu(u)=\int_X\int_Y f(x,y)\de\nu(y)\de\mu(x)=\int_Y\int_X f(x,y)\de\mu(x)\de\nu(y)\virgola
	\end{equation*}
	dove le funzioni $f(x,\cdot):Y\to\Rpiu$ e $f(\cdot,y):X\to\Rpiu$ risultano misurabili solo quasi ovunque e perciò anche gli integrali $\int_Y f(x,y)\de\nu(y)$ e $\int_X f(x,y)\de\mu(x)$ risultano definiti solo quasi ovunque, ma questo non cambia il valore dell'integrale esterno in virtù del \cref{cor:IntegraleAMenoDiTrascurabili}.
\end{theorem}
\begin{proof}
	Si dimostra identicamente a come si è dimostrato il \cref{thm:Fubini}, solo che, al posto di applicare il \cref{thm:Tonelli}, si applica il \cref{thm:TonelliCompleto}.
\end{proof}

Per concludere la sezione, lasciamo per esercizio al lettore di verificare che le seguenti situazioni, ovviamente carenti delle ipotesi necessarie, falsificano le ipotesi dei teoremi di Fubini e Tonelli.

\begin{exercise}\label{ex:ControesempiFubini}
	Chiamando $(X,\A,\mu)$ e $(Y,\B,\nu)$ due spazi di misura e $f:X\times Y\to \Rbar$ una funzione dal prodotto degli spazi, mostrare che nei seguenti casi non vale almeno una parte degli enunciati di Fubini e Tonelli.
	\begin{enumerate}
		\item Se $X=Y=\R$, $\A=\B=\mathcal P(\R)$, $\mu$ è la misura contapunti (cioè la misura che restituisce il numero di elementi se l'insieme è finito ed $+\infty$ altrimenti), $\nu$ la misura che restituisce $0$ se l'insieme è finito e $+\infty$ se è infinito ed infine $f$ è la funzione indicatrice della diagonale $\{(x,y)\in X\times Y: x=y\}$.
		\item Se $X=Y=\R$, $\A,\B$ coincidono con i Boreliani di $\R$, $\mu=\nu=m_1$ e la funzione $f$ è la funzione indicatrice dell'insieme $\cc01\times V$ dove $V$ è un insieme trascurabile non misurabile per Lebesgue.
		\item Se $X=Y=\R$, $\A=\B=\M_1$, $\mu=\nu=m_1$ e la funzione $f$ è definita come $f(x,y)=\frac{x^2-y^2}{(x^2+y^2)^2}\cdot\chi_{\cc01\times\cc01}(x,y)$.
	\end{enumerate}

\end{exercise}

%TODO Aggiungere vari controesempi a Fubini Tonelli nel caso in cui vengano meno alcune ipotesi.
