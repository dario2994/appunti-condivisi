\section{Definizioni e fatti introduttivi}
In questa sezione definiremo tutti gli oggetti necessari per fondare la teoria della misura e dimostreremo alcuni fatti, perlopiù di carattere insiemistico su di essi.

In particolare definiremo alcune strutture insiemistiche (algebre, \sigalg[e], \semiring[i]) e ne dimostreremo alcune proprietà. Poi passeremo a trattare gli spazi di misura ed alcune loro \emph{versioni più deboli} come le misure esterne e le premisure.

L'idea generale di queste pagine è dare gli strumenti necessari a comprendere il teorema di estensione di una premisura, nella cui dimostrazione ed enunciato convergono tutti i fatti qui trattati. 

Da notare infine che la definizione di \semiring{} non è quella canonica (né forse si può parlare di una definizione canonica in letteratura) ed anzi è più debole di quella tipicamente usata per enunciare il teorema di estensione. Questo rende più generale il risultato, ma allo stesso tempo rende più ostico dimostrare le prime proprietà della premisura che infatti necessiteranno di vari lemmi tecnici per essere dimostrate (e sarebbero banali se al posto di un \semiring{} ci fosse un anello).

\begin{definition}[Algebra]
	Dato un insieme $X$, una famiglia $\A\subseteq\mathcal P(X)$ è un'algebra se valgono:
	\begin{itemize}
		\item $\emptyset\in\A$
		\item $\forall A\in\A:\ A^c\in\A$ cioè un'algebra è stabile per passaggio al complementare.
		\item $\forall A,B\in\A:\ A\cup B\in\A$ cioè un'algebra è stabile per unioni finite.
	\end{itemize}
\end{definition}
\begin{remark}\label{ProprietaAlg}
	Un'algebra è stabile anche per intersezioni finite e per differenza insiemistica.
\end{remark}
\begin{proof}
	Poiché vale la formula insiemistica:
	\begin{equation*}
		\bigcap_{i\in I} A_i = \left( \bigcup_{i\in I} A_i^c \right)^c
	\end{equation*}
	e un'algebra è stabile per unione finita e complementare, facilmente risulta esserlo anche per intersezioni finite.
	
	Per la differenza si sfrutta la seguente relazione insiemistica $A\setminus B=A\cap B^c$. Questa porta a concludere visto che abbiamo appena dimostrato che $\A$ è stabile per intersezioni.
\end{proof}


\begin{definition}[\sigalg{}]
	Dato un insieme $X$, una famiglia $\A\subseteq\mathcal P(X)$ si dice \sigalg{} se valgono:
	\begin{itemize}
	\item $\emptyset\in \A$
	\item $\forall A\in \A:\ A^c\in \A$ cioè una \sigalg{} è stabile per passaggio al complementare.
	\item $\forall (A_n)_{n\in\N}\subseteq \A:\ \bigcup_{n\in\N} A_n\in \A$ cioè una \sigalg{} è stabile per unioni numerabili.  
	\end{itemize}
\end{definition}

\begin{remark}\label{ProprietaSigAlg}
	Una \sigalg{} è stabile anche per intersezioni numerabili e per differenza insiemistica.
\end{remark}
\begin{proof}
	Si dimostrano entrambe le proprietà in modo del tutto analogo a come abbiamo dimostrato \cref{ProprietaAlg}.
\end{proof}

\begin{proposition}\label{SigmaAlgGenerata}
	Dato un insieme $X$, sia $\mathcal F\subseteq\mathcal P(X)$ una famiglia di sottoinsiemi di $X$.
	
	Allora esiste la più \emph{piccola} (cioè contenuta in ogni altra) \sigalg{} contenente la famiglia $\mathcal F$, che viene detta \sigalg{} generata da $\mathcal F$ e si indica con $\sigma A(\mathcal F)$.
\end{proposition}
\begin{proof}
	Consideriamo l'insieme $\mathcal I$ contenente tutte le \sigalg[e] $\A$ tali che $\mathcal F\subseteq \A \subseteq \mathcal P(X)$.
	L'insieme $\mathcal I$ è non vuoto poichè ovviamente $\mathcal P(X)\in\mathcal I$.
	
	Sia $\mathscr B=\bigcap_{\A\in\mathcal I} \A$.
	Vale che $\mathcal F\subseteq \mathscr B$ visto che appartiene ad ogni insieme in $\mathcal I$ e tale insieme è non vuoto.
	
	È di banale verifica che intersezione di \sigalg[e] è ancora una \sigalg{}, perciò $\mathscr B$ risulta essere una \sigalg{} contenente $\mathcal F$. Ma $\mathscr B$ è definita come l'intersezione di tutte le \sigalg{} contenenti $\mathcal F$, quindi è la più piccola e la tesi è dimostrata.
\end{proof}


\begin{definition}[\Semiring{}]
	Una famiglia $\S\subseteq \mathcal P(X)$ è detta \semiring{} se valgono le seguenti proprietà:
	\begin{itemize}
		\item $\emptyset\in \S$
		\item $\displaystyle\forall A,B\in \S: A\cap B, A\setminus B\in \sqcup \S$ dove
		$\displaystyle
		\sqcup{ \S }=\left\{\bigsqcup_{n\in \N} S_n\ |\ (S_n)_{n\in\N} \subseteq \S \wedge \forall i\not= j:\ S_i\cap S_j=\emptyset\right\}$ 
		cioè un \semiring{} non deve essere stabile per intersezione e differenza, ma queste si devono scrivere come unioni disgiunte.
	\end{itemize}
\end{definition}

\begin{proposition}\label{UnioneDisgiuntaQuasiAlgebra}
	Dato un \semiring{} $\S$ l'insieme $\sqcup\S$ è stabile per intersezione finita e unione numerabile.
\end{proposition}
\begin{proof}
	Notiamo intanto che, per definizione, $\sqcup\S$ è stabile per unione disgiunta numerabile.
	
	Per dimostrare la stabilità di $\sqcup\S$ per intersezione finita, basta ovviamente farlo per due soli insiemi $A,B\in\sqcup\S$. Per definizione possiamo scrivere $A=\bigsqcup_{n\in\N} A_n, B=\bigsqcup_{n\in\N} B_n$ dove $(A_n)_{n\in\N},(B_n)_{n\in\N}$ sono successioni in $\S$. Allora vale la seguente identità:
	\begin{equation*}
		A\cap B=\bigsqcup_{n\in\N} A_n\cap\bigsqcup_{n\in\N} B_n=
		\bigsqcup_{n,m\in\N} A_n\cap B_m\in\sqcup\S
	\end{equation*}
	dove nell'ultimo passaggio è stata usata la stabilità di $\sqcup\S$ per unione disgiunta.
	
	Per l'unione, consideriamo $A,B\in\S$. Poiché vale $A\cup B=(A\setminus B)\sqcup(A\cap B)$, viste le proprietà di un \semiring{}, risulta $A\cup B\in \sqcup\S$. Da questo è facile ottenere che anche unioni finite di elementi di $\S$ appartengono a $\sqcup\S$.
	
	Sfruttando quanto detto, fissata $(A_n)_{n\in\N}\subseteq\S$ vale:
	\begin{equation}\label{UnioneNumerabileDaS}
		\bigcup_{n\in\N} A_n=\bigsqcup_{n\in\N} A_n\setminus\cup_{i<n} A_i
		=\bigsqcup_{n\in\N} \bigcap_{i<n} (A_n\setminus A_i)\in\sqcup\S
	\end{equation}
	dove nell'ultimo passaggio abbiamo applicato la stabilità di $\sqcup\S$ per unione disgiunta e intersezione finita.
	
	E infine dimostriamo la stabilità di $\sqcup\S$ per unioni numerabili. Sia $(S_n)_{n\in\N}$ una successione in $\sqcup\S$. Per definizione devono esistere le successioni $(A^n_i)_{i\in\N}\subseteq \S$ tali che $S_n=\bigsqcup_{i\in\N} A^n_i$.
	
	Allora applicando \cref{UnioneNumerabileDaS} otteniamo:
	\begin{equation*}
		\bigcup_{n\in\N}S_n=\bigcup_{i,n\in\N}A^n_i\in\sqcup\S 
	\end{equation*}
	che è proprio la stabilità di $\sqcup\S$ per unioni numerabili.
\end{proof}

\begin{definition}[{\sigadd[ità]}]
	Una funzione $\mu:\mathcal F\to \Rpiu$, dove $\mathcal F$ è una famiglia di insiemi, si dice \sigadd{} se per ogni sottofamiglia numerabile $(F_n)_{n\in\N}\subseteq \mathcal F$ a due a due disgiunta, tale che l'unione appartiene a $\mathcal F$, vale l'addittività:
	\begin{equation*}
		\mu\left(\bigcup_{n\in\N}F_n \right)=\sum_{n\in\N} \mu(F_n) 
	\end{equation*}
\end{definition}

\begin{definition}[Spazio di misura]
	Dati $X$ un insieme, $\A$ una famiglia di sottoinsiemi di $X$ e $\mu:\A\to \Rpiu$ una funzione, la terna $(X,\A, \mu)$ si dice uno spazio di misura se:
	\begin{itemize}
		\item la famiglia $\A$ è una \sigalg{}.
		\item la funzione $\mu$ è \sigadd{}.
	\end{itemize}
	e in questo caso la funzione $\mu$ è detta \emph{misura} e gli insiemi in $\A$ sono detti misurabili.
\end{definition}

\begin{remark}\label{RiduzioneMisura}
	Se $(X,\A,\mu)$ è uno spazio di misura e $E\in\A$, allora ovviamente anche $(X,\A_E,\mu|_{\A_E})$ è uno spazio di misura, dove $\A_E=\{A\cap E:\ A\in\A\}$. Cioè riducendosi ad un sottoinsieme misurabile si mantengono le proprietà di spazio di misura.
\end{remark}

\begin{remark}\label{MonotoniaMisura}
	Dato $(X,\A,\mu)$ uno spazio di misura, $\mu$ è monotona, cioè se $A,B\in\A$ e $A\subseteq B$ allora $\mu(A)\le \mu(B)$.
\end{remark}
\begin{proof}
	Per quanto detto in \cref{ProprietaSigAlg}, $B\setminus A\in\A$ e perciò sfruttando l'addittività su insiemi disgiunti di $\mu$ otteniamo $\mu(B)=\mu(B\setminus A)+\mu(A)>\mu(A)$ che è la tesi.
\end{proof}
\begin{remark}\label{SubAdditivitaMisura}
	Dato $(X,\A,\mu)$ uno spazio misurato, la misura $\mu$ è \sigsubadd{}, cioè dati $(A_n)_{n\in\N}\subseteq\A$ risulta:
	\begin{equation*}
		\mu\left(\bigcup_{n\in\N} A_n \right)\le \sum_{n\in\N}\mu(A_n)
	\end{equation*}
\end{remark}
\begin{proof}
	La dimostrazione risulta molto facile sfruttando la monotonia, mostrata in \cref{MonotoniaMisura}, e la solita decomposizione dell'unione in un'unione disgiunta che permette di applicare la \sigadd[ità]:
	\begin{equation*}
		\mu\left(\bigcup_{n\in\N} A_n \right)=\mu\left(\bigsqcup_{n\in\N} A_n\setminus\bigcup_{i<n}A_i \right)=
		\sum_{n\in\N}\mu\left(A_n\setminus\bigcup_{i<n}A_i\right)\le \sum_{n\in\N}\mu(A_n)
	\end{equation*}
\end{proof}



\begin{definition}\label{TrascurabiliMisura}
	In uno spazio di misura un insieme con misura nulla è detto trascurabile.
\end{definition}
\begin{remark}\label{UnioneTrascurabili}
	Unione numerabile di insiemi trascurabile è a sua volta trascurabile.
\end{remark}
\begin{proof}
	È un'ovvia conseguenza della \sigsubadd[ità] della misura dimostrata in \cref{SubAdditivitaMisura}.
\end{proof}

\begin{definition}\label{def:QuasiOvunque}
	Nel seguito i termini \emph{quasi ovunque} e \emph{quasi sempre} e \emph{quasi per ogni}, implicitamente riferiti ad uno spazio di misura, definiranno proposizioni che valgono per ogni elemento tranne alpiù per un sottoinsieme trascurabile di elementi.
\end{definition}


\begin{definition}\label{FinitezzaMisura}
	Dato $(X,\A,\mu)$ uno spazio di misura, la misura $\mu$ è detta finita se $X\in\mathcal{A}$ e $\mu(X)<+\infty$.
\end{definition}

\begin{definition}\label{SigmaFinitezzaMisura}
	Dato $(X,\A,\mu)$ uno spazio di misura, la misura $\mu$ è detta \sigfin{} se esiste una famiglia numerabile $(A_n)_{n\in\N}\subseteq \A$ tale che l'unione sia $X$ e inoltre $\mu(A_n)<+\infty$ per ogni $n\in\N$.
\end{definition}

\begin{definition}\label{CompletezzaMisura}
	Dato $(X,\A,\mu)$ uno spazio di misura, la misura $\mu$ si dice completa se per ogni $A\in\A$ tale che $\mu(A)=0$, ogni suo sottoinsieme è anch'esso appartenente ad $\A$.
\end{definition}

\begin{remark}
	Le definizioni appena date (\cref{TrascurabiliMisura,def:QuasiOvunque,FinitezzaMisura,SigmaFinitezzaMisura,CompletezzaMisura}) si estendono anche agli spazi, che definiremo tra poco, che sono muniti di misure elementari o di misure esterne. Inoltre le qualità di finitezza, \sigfin[ezza] e completezza si riferiranno, col medesimo significato, sia alla misura che allo spazio. 
\end{remark}

\begin{proposition}\label{CompletamentoMisura}
	Dato uno spazio di misura $(X,\A,\mu)$, chiamo $\mu$-completamento di $\A$ l'insieme $\A^*$ formato dagli elementi di $\A$ uniti con un sottoinsieme di un trascurabile:
	\begin{equation*}
		\A^*=\{A\cup N\ |\ A\in\A\ \wedge \exists B:\ N\subseteq B,\ \mu(B)=0\}
	\end{equation*}
	Allora $\A^*$ è una \sigalg{} e $\mu$ si estende in maniera canonica su $\A^*$ ad una misura completa.
\end{proposition}
\begin{proof}
	Dimostriamo intanto che $\A^*$ è una \sigalg{}.
	Dato $A^*\in\A^*$, esistono per definizione $A\in\A$ e $N\subseteq B\in\A$ dove $\mu(B)=0$, tali che $A^*=A\cup N$. Possiamo facilmente imporre $B\cap A=\emptyset$, visto che se così non è si può ridefinire $N,B$ facendone la differenza con $A$. Assumiamo perciò $A,B$ disgiunti.
	Chiamando $M=B\setminus N$, vale facilmente:
	\begin{equation*}
		(A^*)^c=(A\cup N)^c=A^c\cap N^c=A^c\cap(B^c\cup M)=(A^c\cap B^c )\cup M \in\A^*
	\end{equation*}
	dove l'ultima uguaglianza vale perché ho supposto $A$ e $B$ disgiunti e l'ultima appartenenza è invece vera poiché $M$ è anch'esso sottoinsieme di un trascurabile. Quindi $\A^*$ è stabile per passaggio al complementare.
	
	Ora consideriamo $(A^*_n)_{n\in\N}$ una successione di insiemi, e scegliamo $(A_n)_{n\in\N},(B_n)_{n\in\N},(N_n)_{n\in\N}$ tali che valgano $A^*_n=A_n\cup N_n$ e $N_n\subseteq B_n\in\A$ con $\mu(B_n)=0$.
	
	Allora risulta:
	\begin{align*}
		\bigcup_{n\in\N}A^*_n &=\bigcup_{n\in\N}A_n \cup \bigcup_{n\in\N}N_n\\
		\bigcup_{n\in\N}N_n &\subseteq \bigcup_{n\in\N}B_n\in\A\ \wedge\ \mu\left(\bigcup_{n\in\N}B_n\right)=0
	\end{align*}
	dove l'ultima uguaglianza dipende dal fatto che la misura è \sigsubadd{} per \cref{SubAdditivitaMisura}. Perciò abbiamo dimostrato che $\A^*$ è stabile per unione numerabile.
	
	Unendo i due risultati arriviamo a dire che $\A^*$ è una \sigalg{}.
	
	Definiamo ora $\tilde\mu:\A^*\to\Rpiu$ di modo che dato $A^*\in\A^*$ valga $\tilde\mu(A^*)=\mu(A)$ dove $A^*=A\cup N$ con $N$ sottoinsieme di un trascurabile ed $A\in\A$.
	
	Dimostriamo innanzitutto la coerenza della definizione di $\tilde\mu$. Si deve dimostrare che per $A,A'\in\A$ e $N,N'$ sottoinsiemi di trascurabili tali che $A\cup N=A'\cup N'$ vale $\mu(A)=\mu(A')$. Siano $B,B'\in\A$ i trascurabili a cui appartengono $N,N'$, sia infine $Q=B\cup B'$ a sua volta trascurabile.
	
	Per facili ragionamenti di monotonia abbiamo:
	\begin{align*}
		\mu(A)\le \mu(A\cup Q) =\mu(A)+\mu(Q)=\mu(A) &\implies \mu(A)=\mu(A\cup Q)\\
		\mu(A')\le \mu(A'\cup Q) =\mu(A')+\mu(Q)=\mu(A') &\implies \mu(A')=\mu(A'\cup Q)
	\end{align*}
	ma per costruzione vale $A\cup Q=A'\cup Q$ e perciò otteniamo proprio $\mu(A)=\mu(A')$ che mostra la coerenza della definizione di $\tilde\mu$.
	
	La \sigadd[ità] è ovvia una volta che si è mostrata la coerenza. Presa $(A^*_n)_{n\in\N}\in\A^*$ famiglia disgiunta a due a due, sia $(A_n)_{n\in\N}$ la parte \emph{non trascurabile} della prima successione. Allora risulta, per definizione di $\tilde\mu$:
	\begin{equation*}
		\sum_{n\in\N} \tilde\mu(A^*_n)=\sum_{n\in\N} \mu(A_n)=\mu\left(\bigcup_{n\in\N} A_n\right)=\mu\left(\bigcup_{n\in\N} A^*_n\right)
	\end{equation*}
	dove nell'ultimo passaggio abbiamo implicitamente applicato \cref{UnioneTrascurabili}.
	
	Resta da verificare che $\tilde\mu$ sia completa, ma questo è ovvio visto che  $\tilde\mu(A^*)=0$ se e solo se $A^*$ è il sottoinsieme di un trascurabile di $\A$ e la relazione di essere sottoinsieme di un trascurabile è chiusa per estrazione di sottoinsiemi. 

\end{proof}



\begin{proposition}\label{LimiteMonotonoCrescenteMisura}
	Dato $(X,\A,\mu)$ uno spazio di misura e una successione $(A_n)_{n\in\N}\subseteq \A$ tale che $A_n\subseteq A_{n+1}$, risultano commutare limite e misura:
	\begin{equation*}
		\mu\left(\bigcup_{n\in\N} A_n\right)=\lim_{n\in\N} \mu(A_n)
	\end{equation*}
\end{proposition}
\begin{proof}
	Sia $B_n=A_n\setminus\bigcup_{i<n}A_i$. Applicando \cref{ProprietaSigAlg} si ottiene $B_n\in\A$.
	Per facili ragionamenti insiemistici risulta che la successione $(B_n)_{n\in\N}$ è disgiunta a due a due ed inoltre $A_n=\bigcup_{i\le n}B_i$.
	Sfruttando tutte queste proprietà e la \sigadd[ità] di $\mu$, otteniamo:
	\begin{equation*}
		\mu\left(\bigcup_{n\in\N} A_n\right)=\mu\left(\bigcup_{n\in\N} B_n\right)=
		\sum_{n\in\N} \mu(B_n)=\lim_{n\to\infty} \sum_{i\le n} \mu(B_i)=
		\lim_{n\to\infty} \mu\left(\bigcup_{i\le n} B_i\right)=\lim_{n\to\infty} \mu(A_n)
	\end{equation*}
	che è proprio la tesi.
\end{proof}

\begin{corollary}\label{LimiteMonotonoDecrescenteMisura}
	Dato $(X,\A,\mu)$ uno spazio di misura e una successione $(A_n)_{n\in\N}\subseteq \A$ tale che $A_{n+1}\subseteq A_n$ con $\mu(A_{n_0})<+\infty$, risultano commutare limite e misura:
	\begin{equation*}
		\mu\left(\bigcap_{n\in\N} A_n\right)=\lim_{n\in\N} \mu(A_n)
	\end{equation*}
\end{corollary}
\begin{proof}
	Intanto possiamo assumere $n_0=0$, tanto i primi elementi della successione $(A_n)_{n\in\N}$ non contano.
	
	Poniamo $B_n=A_0\setminus A_n$. Allora vale ovviamente $B_n\subseteq B_{n+1}$ e perciò possiamo applicare \cref{LimiteMonotonoCrescenteMisura} ottenendo:
	\begin{equation}\label{ContoLimiteMonotonoMisura}
		\mu\left(A_0\setminus\bigcap_{n\in\N}A_n\right)=\mu\left(\bigcup_{n\in\N}B_n\right)=\lim_{n\in\N}\mu(B_n)
		=\lim_{n\in\N}\mu(A_0\setminus A_n)
	\end{equation}
	Ma poichè $\mu(A_0)<+\infty$, dalla \sigadd[ità] di $\mu$, possiamo ottenere:
	\begin{align*}
		\mu\left(A_0\setminus\bigcap_{n\in\N}A_n\right)&=\mu(A_0)-\mu\left(\bigcap_{n\in\N}A_n\right)\\
		\mu(A_0\setminus A_n)&=\mu(A_0)-\mu(A_n)
	\end{align*}
	e applicando questi in \cref{ContoLimiteMonotonoMisura} arriviamo alla tesi ricordando ancora che $\mu(A_0)<+\infty$.
\end{proof}

\begin{proposition}\label{prop:BigettivaInduceMisura}
	Dato $(X,\A,\mu)$ uno spazio di misura, se $f:X\to X$ è una funzione bigettiva che manda misurabili in misurabili allora anche $(X,\A,\mu\circ f)$ è uno spazio di misura, vedendo $f$ come funzione tra insiemi.
\end{proposition}
\begin{proof}
	Il fatto che $f$ manda misurabili in misurabili assicura che la funzione $\mu\circ f$ è ben definita sugli elementi di $\A$.
	
	Per mostrare la tesi è più che sufficiente dimostrare che $\mu\circ f$ è \sigadd{}, quindi scegliamo $(A_n)_{n\in\N}\subseteq\A$ una successione di misurabili disgiunti.
	Grazie al fatto che $f$ è bigettiva anche $f(A_n)$ sono tutti disgiunti e perciò per la \sigadd[ità] di $\mu$ otteniamo
	\begin{equation*}
		\sum_{n\in\N}\mu(f(A_n))=\mu\left(\bigsqcup_{n\in\N}f(A_n)\right)=\mu\left(f\left(\bigsqcup_{n\in\N}A_n\right)\right)
	\end{equation*}
	che equivale proprio alla \sigadd[ità] di $\mu\circ f$ e conclude la dimostrazione.
\end{proof}



\begin{definition}[Misura esterna]
	Dato un insieme $X$ e una funzione $\mu^*:\mathcal P(X)\to \Rpiu$ è detta una misura esterna se valgono:
	\begin{itemize}
		\item $\mu^*(\emptyset)=0$
		\item $\mu^*$ è monotona, cioè dati $A,B\subseteq X$ se vale $A\subseteq B$ allora $\mu^*(A)\le \mu^*(B)$
		\item $\mu^*$ è \sigsubadd{}, cioè  per ogni successione $(A_n)_{n\in\N}\subseteq \mathcal P(X)$ di sottoinsiemi di $X$ vale $\mu^*\left(\bigcup_{n\in\mathbb{N}}A_n\right)\le \sum_{n\in\N} \mu^*(A_n)$
	\end{itemize}
\end{definition}

\begin{definition}
	Una terna $(X,\S,\mu)$ tale che $\S\subseteq\mathcal P(X)$ è un \semiring{} e $\mu:\S\to \Rpiu$ è \sigadd{}, la chiamo spazio di misura elementare e la funzione $\mu$ la chiamo misura elementare o premisura.
\end{definition}

\begin{lemma}\label{CoerenzaPremisura}
	Fissato $(X,\S,\mu)$ uno spazio di misura elementare, siano $(A_n)_{n\in\N},(B_n)_{n\in\N}\subseteq\S$ delle famiglie tali che l'unione sia la stessa, ma i $(B_n)_{n\in\N}$ siano disgiunti a due a due: $\bigcup_{n\in\N}A_n=\bigsqcup_{n\in\N}B_n$.
	Allora risulta $\sum_{n\in\N}\mu(A_n)\ge \sum_{n\in\N}\mu(B_n)$.
\end{lemma}
\begin{proof}
	Sia $A'_n=A_n\setminus\bigcup_{i<n}A_i$. La successione $(A'_n)_{n\in\N}$ è disgiunta a due a due e ogni singolo elemento appartiene a $\sqcup \S$ visto che vale $A'_n=\bigcap_{i<n}A_n\setminus A_i$ e $\sqcup \S$ è chiuso per intersezione finita, come mostrato in \cref{UnioneDisgiuntaQuasiAlgebra}. Infine è chiaro che l'unione della nuova famiglia è uguale a quella di $(A_n)_{n\in\N}$.
	È importante notare che $A_n\setminus A'_n=A_n\cap\bigcup_{i<n}A_i\in\sqcup\S$ dove l'ultima appartenenza è vera per la stabilità di $\sqcup\S$ per unioni e intersezioni finite. Allora esistono $(E^n_i)_{i\in\N}\subseteq\S$ tali che in unione disgiunta realizzano $A_n\setminus A'_n$.
	
	Siano quindi $C_{ij}=A'_i\cap B_j$. Ovviamente la successione $(C_{ij})_{i,j\in\N}$ è disgiunta a due a due (perché lo sono sia $(A'_n)_{n\in\N}$ che $(B_n)_{n\in\N}$) ed è un sottoinsieme di $\sqcup\S$ poiché intersezione di elementi che vi appartengono. Quindi esiste la famiglia $(F^{ij}_n)_{n\in\N}\subseteq\S$ la cui unione disgiunta realizza $C_{ij}$.
	
	Ora per costruzione e per le osservazioni fatte valgono:
	\begin{align*}
		A_n= A'_n\sqcup \bigsqcup_{i\in\N}E^n_i=\bigsqcup_{i\in\N}C_{ni}\sqcup \bigsqcup_{i\in\N}E^n_i
		=\bigsqcup_{i,j\in\N}F^{ni}_j\sqcup \bigsqcup_{i\in\N}E^n_i
		&\implies \mu(A_n)\ge\sum_{i,j\in\N}\mu(F^{ni}_j)+\sum_{i\in\N}\mu(E^n_i)\\
		B_n=\bigsqcup_{i\in\N}C_{in}=\bigsqcup_{i,j\in\N}F^{in}_j
		&\implies \mu(B_n)=\sum_{i,j\in\N}\mu(F^{in}_j)
	\end{align*}
	quindi sommando su $n$ arriviamo a:
	\begin{align*}
		\sum_{n\in\N}\mu(A_n)&\ge \sum_{n\in\N}\sum_{i,j\in\N}\mu(F^{ni}_j)
		=\sum_{i,n,j\in\N}\mu(F^{ni}_j)\\
		\sum_{n\in\N}\mu(B_n)&=\sum_{n\in\N}\sum_{i,j\in\N}\mu(F^{in}_j)=\sum_{i,n,j\in\N}\mu(F^{in}_j)
	\end{align*}
	ma visto che l'ordine degli indici non conta, questi risultati implicano banalmente la tesi.


\end{proof}



\begin{lemma}\label{PiuCheMonotonaPremisura}
	Dato $(X,\S,\mu)$ uno spazio di misura elementare, siano $A,(A_n)_{n\in\N}\subseteq \S$ tali che $A\subseteq\bigcup_{n\in\N}A_n$.
	Allora vale $\mu(A)\le \sum_{n\in\N}\mu(A_n)$.
\end{lemma}
\begin{proof}
	Visto che $A\subseteq\bigcup_{n\in\N}A_n$ vale la scrittura insiemistica:
	\begin{equation}\label{ScritturaDecenteUnionePremisura}
		\bigcup_{n\in\N}A_n=A\sqcup\bigcup_{n\in\N}A_n\setminus A
	\end{equation}
	Poiché $\S$ è un \semiring{} $A_n\setminus A\in \sqcup \S$, e visto che $\sqcup S$ è chiuso per unione numerabile, come mostrato in \cref{UnioneDisgiuntaQuasiAlgebra}, esiste una famiglia $(B_n)_{n\in\N}\subseteq\S$ disgiunta tale che $\bigcup_{n\in\N}A_n\setminus A=\bigsqcup_{n\in\N}B_n$.
	Allora sostituendo in \cref{ScritturaDecenteUnionePremisura} si ottiene:
	\begin{equation*}
		\bigcup_{n\in\N}A_n=A\sqcup\bigsqcup_{n\in\N}B_n
	\end{equation*}
	quindi si ricade nelle ipotesi di \cref{CoerenzaPremisura} ottenendo che:
	\begin{equation*}
		\sum_{n\in\N}\mu(A_n)\ge \mu(A)+\sum_{n\in\N}\mu(B_n)\ge \mu(A)
	\end{equation*}
	che è proprio quanto si voleva dimostrare.

\end{proof}



