\section{Definizioni e fatti introduttivi}
\begin{definition}[Algebra]
	Dato un insieme $X$, una famiglia $\mathcal A\subseteq\mathcal P(X)$ è un'algebra se valgono:
	\begin{itemize}
		\item $\emptyset\in\mathcal A$
		\item $\forall A\in\mathcal A:\ A^c\in\mathcal A$ cioè un'algebra è stabile per passaggio al complementare.
		\item $\forall A,B\in\mathcal A:\ A\cup B\in\mathcal A$ cioè un'algebra è stabile per unioni finite.
	\end{itemize}
\end{definition}
\begin{remark}\label{ProprietaAlg}
	Un'algebra è stabile anche per intersezioni finite e per differenza insiemistica.
\end{remark}
\begin{proof}
	Poichè vale la formula insiemistica:
	\begin{equation*}
		\bigcap_{i\in I} A_i = \left( \bigcup_{i\in I} A_i^c \right)^c
	\end{equation*}
	e un'algebra è stabile per unione finita e complementare, ottengo facilmente che lo è anche per intersezioni finite.
	
	Per la differenza si sfrutta la seguente relazione insiemistica $A\setminus B=A\cup B^c$ e visto che ho appena dimostrato che $\mathcal A$ è chiuso anche per intersezioni ho finito.
\end{proof}


\begin{definition}[\sigalg{}]
	Dato un insieme $X$, una famiglia $\mathcal A\subseteq\mathcal P (X)$ si dice \sigalg{} se valgono:
	\begin{itemize}
	\item $\emptyset\in \mathcal A$
	\item $\forall A\in \mathcal A:\ A^c\in \mathcal A$ cioè una \sigalg{} è stabile per passaggio al complementare.
	\item $\forall (A_n)_{n\in\mathbb N}\subseteq \mathcal A:\ \bigcup_{n\in\mathbb N} A_n\in \mathcal A$ cioè una \sigalg{} è stabile per unioni numerabili.  
	\end{itemize}
\end{definition}

\begin{remark}\label{ProprietaSigAlg}
	Una \sigalg{} è stabile anche per intersezioni numerabili e per differenza simmetrica.
\end{remark}
\begin{proof}
	Si dimostrano entrambe le proprietà in modo del tutto analogo a come abbiamo dimostrato \cref{ProprietaAlg}.
\end{proof}

\begin{definition}[\Semiring{}]
	Una famiglia $\mathcal S\subseteq \mathcal P(X)$ è detta \semiring{} se valgono le seguenti proprietà:
	\begin{itemize}
		\item $\emptyset\in \mathcal S$
		\item $\displaystyle\forall A,B\in \mathcal S: A\cap B, A\setminus B\in \sqcup \mathcal S$ dove
		$\displaystyle
		\sqcup{ \mathcal S }=\left\{\bigsqcup_{n\in \mathbb N} S_n\ |\ (S_n)_{n\in\mathbb N} \subseteq \mathcal S \wedge \forall i\not= j:\ S_i\cap S_j=\emptyset\right\}$ 
		cioè un \semiring{} non deve essere stabile per intersezione e differenza, ma queste si devono scrivere come unioni disgiunte.
	\end{itemize}
\end{definition}

\begin{proposition}\label{UnioneDisgiuntaQuasiAlgebra}
	Dato un \semiring{} $\mathcal S$ l'insieme $\sqcup\mathcal S$ è stabile per intersezione finita e unione numerabile.
\end{proposition}
\begin{proof}
	Noto intanto che, per definizione, $\sqcup\mathcal S$ è stabile per unione disgiunta numerabile.
	
	Per dimostrare la stabilità di $\sqcup\mathcal S$ per intersezione finita, basta ovviamente farlo per due soli insiemi $A,B\in\sqcup\mathcal S$. Per definizione posso scrivere $A=\bigsqcup_{n\in\mathbb N} A_n, B=\bigsqcup_{n\in\mathbb N} B_n$ dove $(A_n)_{n\in\mathbb N},(B_n)_{n\in\mathbb N}$ sono successioni in $\mathcal S$. Allora vale la seguente identità:
	\begin{equation*}
		A\cap B=\bigsqcup_{n\in\mathbb N} A_n\cap\bigsqcup_{n\in\mathbb N} B_n=
		\bigsqcup_{n,m\in\mathbb N} A_n\cap B_m\in\sqcup\mathcal S
	\end{equation*}
	dove nell'ultimo passaggio ho usato la stabilità di $\sqcup\mathcal S$ per unione disgiunta.
	
	Per l'unione, considero $A,B\in\mathcal S$. Poichè vale $A\cup B=(A\setminus B)\sqcup(A\cap B)$, viste le proprietà di un \semiring{}, risulta $A\cup B\in \sqcup\mathcal S$. Da questo è facile ottenere che anche unioni finite di elementi di $\mathcal S$ appartengono a $\sqcup\mathcal S$.
	
	Sfruttando quanto detto, fissata $(A_n)_{n\in\mathbb N}\subseteq\mathcal S$ vale:
	\begin{equation}\label{UnioneNumerabileDaS}
		\bigcup_{n\in\mathbb N} A_n=\bigsqcup_{n\in\mathbb N} A_n\setminus\cup_{i<n} A_i
		=\bigsqcup_{n\in\mathbb N} \bigcap_{i<n} (A_n\setminus A_i)\in\sqcup\mathcal S
	\end{equation}
	dove nell'ultimo passaggio ho usato il fatto che $\sqcup\mathcal S$ è stabile per unione disgiunta e intersezione finita.
	
	E ora finalmente dimostro la stabilità di $\sqcup\mathcal S$ per unioni numerabili. Sia $(S_n)_{n\in\mathbb N}$ una successione in $\sqcup\mathcal S$. Per definizione devono esistere le successioni $(A^n_i)_{i\in\mathbb N}\subseteq \mathcal S$ tali che $S_n=\bigsqcup_{i\in\mathbb N} A^n_i$.
	
	Allora applicando \cref{UnioneNumerabileDaS} ho:
	\begin{equation*}
		\bigcup_{n\in\mathbb N}S_n=\bigcup_{i,n\in\mathbb N}A^n_i\in\sqcup\mathcal S 
	\end{equation*}
	che è proprio la stabilità di $\sqcup\mathcal S$ per unioni numerabili.
\end{proof}

\begin{definition}[{\sigadd[ità]}]
	Una funzione $\mu:\mathcal F\to \Rpiu$, dove $\mathcal F$ è una famiglia di insiemi, si dice \sigadd{} se per ogni sottofamiglia numerabile $(F_n)_{n\in\mathbb N}\subseteq \mathcal F$ a due a due disgiunta, tale che l'unione appartiene a $\mathcal F$, vale l'addittività:
	\begin{equation*}
		\mu\left(\bigcup_{n\in\mathbb N}F_n \right)=\sum_{n\in\mathbb N} \mu(F_n) 
	\end{equation*}
\end{definition}
\begin{remark}
	Data $\mu:\mathcal F\to \Rpiu$ \sigadd{}, se $\emptyset\in \mathcal F$ allora $\mu(\emptyset)=0$
\end{remark}
\begin{proof}
	Usando la proprietà di \sigadd[ità] ottengo $\mu(\emptyset)=\mu(\emptyset)+\mu(\emptyset)$ che porta ovviamente alla tesi.
\end{proof}


\begin{definition}[Spazio di misura]
	Dati $X$ un insieme, $\mathcal A$ una famiglia di sottoinsiemi di $\mathcal A$ e $\mu:\mathcal A\to \Rpiu$ una funzione, la terna $(X,\mathcal A, \mu)$ si dice uno spazio di misura se:
	\begin{itemize}
		\item La famiglia $\mathcal A$ è una \sigalg{}.
		\item La funzione $\mu$ è \sigadd{}.
	\end{itemize}
	e in questo caso la funzione $\mu$ è detta \emph{misura}.
\end{definition}
%D'ora in poi, quando ci si riferirà ad una misura, si darà per scontato che questa si riferisce ad uno spazio misurato.

\begin{remark}\label{MonotoniaMisura}
	Dato $(X,\mathcal A,\mu)$ uno spazio di misura, $\mu$ è monotona, cioè se $A,B\in\mathcal A$ e $A\subseteq B$ allora $\mu(A)\le \mu(B)$.
\end{remark}
\begin{proof}
	Per quanto detto in \cref{ProprietaSigAlg} $B\setminus A\in\mathcal A$ e perciò sfruttando l'addittività su insiemi disgiunti di $\mu$ ho $\mu(B)=\mu(B\setminus A)+\mu(A)>\mu(A)$ che è la tesi.
\end{proof}

\begin{definition}\label{FinitezzaMisura}
	Dato $(X,\mathcal A,\mu)$ uno spazio di misura, la misura $\mu$ è detta finita se $\mu(X)<+\infty$.
\end{definition}


\begin{proposition}\label{LimiteMonotonoMisura}
	Dato $(X,\mathcal A,\mu)$ uno spazio di misura, allora data una successione $(A_n)_{n\in\mathbb N}\subseteq \mathcal A$ tale che $A_n\subseteq A_{n+1}$ vale:
	\begin{equation*}
		\mu\left(\bigcup_{n\in\mathbb N} A_n\right)=\lim_{n\in\mathbb N} \mu(A_n)
	\end{equation*}
\end{proposition}
\begin{proof}
	Definisco $B_n=A_n\setminus\bigcup_{i<n}A_i$ e per \cref{ProprietaSigAlg} so che $B_n\in\mathcal A$.
	Per facili ragionamenti insiemistici risulta che la successione $(B_n)_{n\in\mathbb N}$ è disgiunta a due a due ed inoltre $A_n=\bigcup_{i\le n}B_i$.
	Sfruttando tutte queste proprietà, e la \sigadd[ità] di $\mu$, ottengo:
	\begin{equation*}
		\mu\left(\bigcup_{n\in\mathbb N} A_n\right)=\mu\left(\bigcup_{n\in\mathbb N} B_n\right)=
		\sum_{n\in\mathbb N} \mu(B_n)=\lim_{n\to\infty} \sum_{i\le n} \mu(B_i)=
		\lim_{n\to\infty} \mu\left(\bigcup_{i\le n} B_i\right)=\lim_{n\to\infty} \mu(A_n)
	\end{equation*}
	che è proprio la tesi.
\end{proof}

\begin{definition}[Misura esterna]
	Dato un insieme $X$ e una funzione $\mu:\mathcal P(X)\to \Rpiu$ è detta una misura esterna se valgono:
	\begin{itemize}
		\item $\mu(\emptyset)=0$
		\item $\mu$ è monotona, cioè dati $A,B\subseteq X$ se vale $A\subseteq B$ allora $\mu(A)\le \mu(B)$
		\item $\mu$ è \sigsubadd{}, cioè  per ogni successione $(A_n)_{n\in\mathbb N}\subseteq \mathcal P(X)$ di sottoinsiemi di $X$ vale $\mu\left(\bigcup_{n\in\mathbb{N}}A_n\right)\le \sum_{n\in\mathbb N} \mu(A_n)$
	\end{itemize}
\end{definition}

\begin{remark}
	Dato $(X,\mathcal A,\mu)$ uno spazio di misura, la misura $\mu$ è \sigsubadd{}.
\end{remark}
\begin{proof}
	Data una successione di sottoinsiemi $(A_n)_{n\in\mathbb N}\subseteq \mathcal A$, considero, come nella dimostrazione di \cref{LimiteMonotonoMisura}, i sottoinsiemi $B_n=A_n\setminus\bigcup_{i<n}A_i\in\mathcal A$.
	Allora ho, lavorando analogamente alla dimostrazione di cui sopra:
	\begin{equation*}
		\mu\left(\bigcup_{n\in\mathbb N} A_n\right)=\mu\left(\bigcup_{n\in\mathbb N} B_n\right)=
		\sum_{n\in\mathbb N} \mu(B_n)\le \sum_{n\in\mathbb N} \mu(A_n)
	\end{equation*}
	dove nell'ultimo passaggio ho sfruttato la monotonia di $\mu$ dimostrata in \cref{MonotoniaMisura}.
\end{proof}

\begin{definition}
	Una terna $(X,\mathcal S,\mu)$ tale che $\mathcal S\subseteq\mathcal P(X)$ è un \semiring{} e $\mu:\mathcal S\to \Rpiu$ è \sigadd{}, la chiamo spazio di misura elementare e la funzione $\mu$ la chiamo misura elementare o premisura.
\end{definition}

\begin{lemma}\label{CoerenzaPremisura}
	Fissato $(X,\mathcal S,\mu)$ uno spazio di misura elementare, siano $(A_n)_{n\in\mathbb N},(B_n)_{n\in\mathbb N}\subseteq\mathcal S$ delle famiglie tali che l'unione sia la stessa, ma i $(B_n)_{n\in\mathbb N}$ siano disgiunti a due a due: $\bigcup_{n\in\mathbb N}A_n=\bigsqcup_{n\in\mathbb N}B_n$.
	Allora risulta $\sum_{n\in\mathbb N}\mu(A_n)\ge \sum_{n\in\mathbb N}\mu(B_n)$.
\end{lemma}
\begin{proof}
	Definisco $A'_n=A_n\setminus\bigcup_{i<n}A_i$. La successione $(A'_n)_{n\in\mathbb N}$ è disgiunta a due a due e ogni singolo elemento appartiene a $\sqcup \mathcal S$ visto che vale $A'_n=\bigcap_{i<n}A_n\setminus A_i$ e $\sqcup \mathcal S$ è chiuso per intersezione finita, come mostrato in \cref{UnioneDisgiuntaQuasiAlgebra}. Infine è chiaro che l'unione della nuova famiglia è uguale a quella di $(A_n)_{n\in\mathbb N}$.
	È importante notare che $A_n\setminus A'_n=A\cap\bigcup_{i<n}A_i\in\sqcup\mathcal S$ dove l'ultimo contenimento è per la stabilità di $\sqcup\mathcal S$ per unioni e intersezioni finite. Allora esistono $(E^n_i)_{i\in\mathbb N}\subseteq\mathcal S$ tali che in unione disgiunta mi $A_n\setminus A'_n$.
	
	Ora definisco $C_{ij}=A'_i\cap B_j$. Ovviamente la successione $(C_{ij})_{i,j\in\mathbb N}$ è disgiunta a due a due (perchè lo sono sia $(A'_n)_{n\in\mathbb N}$ che $(B_n)_{n\in\mathbb N}$) ed è un sottoinsieme di $\sqcup\mathcal S$ poichè intersezione di elementi che vi appartengono. Quindi esiste la famiglia $(F^{ij}_n)_{n\in\mathbb N}\subseteq\mathcal S$ la cui unione disgiunta realizza $C_{ij}$.
	
	Ora per costruzione e per le osservazioni fatte valgono:
	\begin{align*}
		A_n= A'_n\cup \bigsqcup_{i\in\mathbb N}E^n_i=\bigsqcup_{i\in\mathbb N}C_{ni}\cup \bigsqcup_{i\in\mathbb N}E^n_i
		=\bigsqcup_{i,j\in\mathbb N}F^{ni}_j\cup \bigsqcup_{i\in\mathbb N}E^n_i
		&\Longrightarrow \mu(A_n)\ge\sum_{i,j\in\mathbb N}\mu(F^{ni}_j)+\sum_{i\in\mathbb N}\mu(E^n_i)\\
		B_n=\bigsqcup_{i\in\mathbb N}C_{in}=\bigsqcup_{i,j\in\mathbb N}F^{in}_j
		&\Longrightarrow \mu(B_n)=\sum_{i,j\in\mathbb N}\mu(F^{in}_j)
	\end{align*}
	quindi sommando su $n$ arrivo a:
	\begin{align*}
		\sum_{n\in\mathbb N}\mu(A_n)&\ge \sum_{n\in\mathbb N}\sum_{i,j\in\mathbb N}\mu(F^{ni}_j)
		=\sum_{i,n,j\in\mathbb N}\mu(F^{ni}_j)\\
		\sum_{n\in\mathbb N}\mu(B_n)&=\sum_{n\in\mathbb N}\sum_{i,j\in\mathbb N}\mu(F^{in}_j)=\sum_{i,n,j\in\mathbb N}\mu(F^{in}_j)
	\end{align*}
	ma visto che l'ordine degli indici non conta, questi risultati implicano banalmente la tesi.


\end{proof}



\begin{lemma}\label{PiuCheMonotonaPremisura}
	Data $(X,\mathcal S,\mu)$ uno spazio di misura elementare, siano $A,(A_n)_{n\in\mathbb N}\subseteq \mathcal S$ tali che $A\subseteq\bigcup_{n\in\mathbb N}A_n$.
	Allora vale $\mu(A)\le \sum_{n\in\mathbb N}\mu(A_n)$.
\end{lemma}
\begin{proof}
	Visto che $A\subseteq\bigcup_{n\in\mathbb N}A_n$ vale la scrittura insiemistica:
	\begin{equation}\label{ScritturaDecenteUnionePremisura}
		\bigcup_{n\in\mathbb N}A_n=A\sqcup\bigcup_{n\in\mathbb N}A_n\setminus A
	\end{equation}
	Poichè $\mathcal S$ è un \semiring{} $A_n\setminus A\in \sqcup \mathcal S$, e visto che $\sqcup S$ è chiuso per unione numerabile, come mostrato in \cref{UnioneDisgiuntaQuasiAlgebra}, esiste una famiglia $(B_n)_{n\in\mathbb N}\subseteq\mathcal S$ disgiunta tale che $\bigcup_{n\in\mathbb N}A_n\setminus A=\bigsqcup_{n\in\mathbb N}B_n$.
	Allora sostituendo in \cref{ScritturaDecenteUnionePremisura} ottengo:
	\begin{equation*}
		\bigcup_{n\in\mathbb N}A_n=A\sqcup\bigsqcup_{n\in\mathbb N}B_n
	\end{equation*}
	quindi ricado nelle ipotesi di \cref{CoerenzaPremisura} ottenendo che:
	\begin{equation*}
		\sum_{n\in\mathbb N}\mu(A_n)\ge \mu(A)+\sum_{n\in\mathbb N}\mu(B_n)\ge \mu(A)
	\end{equation*}
	che è proprio quanto si voleva dimostrare.

\end{proof}



