\section{Misura superficiale}
Obiettivo di questa sezione è formalizzare i concetti di ``lunghezza'', ``superficie'', ``volume'', utilizzando gli strumenti di teoria della misura affrontati finora. Studieremo innanzitutto il caso di un sottospazio $k$-dimesionale di $\R^n$, per poi dare una definizione più generale che vedremo coincidere a quella precedente nel caso ristretto.

\begin{remark}\label{nota:IsometriaSottospazio}
	Dato $V$ sottospazio affine $k$-dimensionale di $\R^n$, allora esiste $U:\R^k\to V$ isometria lineare.
\end{remark}

\begin{definition}\label{def:MisuraSottospazio}
	Dato $V$ sottospazio affine $k$-dimensionale di $\R^n$ e $U$ come nella \cref{nota:IsometriaSottospazio}, definiamo i Boreliani $\mathcal{B}(V)$ di $V$ come le immagini attraverso $U$ dei Boreliani di $\R^k$. Inoltre per ogni $E\in\mathcal{B}(V)$, definiamo la misura di Lebesgue $k$-dimensionale come $\sigma(E)=m_n(U^{-1}(E))$.
\end{definition}

\begin{remark}
	La \cref{def:MisuraSottospazio} non dipende dalla scelta dell'isometria lineare.
\end{remark}
\begin{proof}
	Siano $U:\R^k\to V$ e $W:\R^k\to V$ isometrie lineari che parametrizzano $V$. Allora $T=U\circ W^{-1}:\R^k\to\R^k$ è un'isometria di $\R^k$, in quanto composizione di isometrie.
	
	Consideriamo ora $E\subseteq V$, allora vale che $W^{-1}(E)=T\circ U^{-1}(E)$. Per la \cref{prop:LebesgueProprietaIsometria}, abbiamo però che la misura di Lebesgue è invariante per isometria, quindi $W^{-1}(E)$ appartiene ai Boreliani di $\R^k$ se e solo se ci appartiene anche $U^{-1}(E)$, poichè si ottengono uno dall'altro mediante un'isometria. Analogamente dato $E\in\mathcal{B}(V)$ vale $m_n(W^{-1}(E))=m_n(T\circ U^{-1}(E))=m_n(U^{-1}(E))$, da cui quello che volevamo dimostrare.
\end{proof}

Notiamo che la \cref{def:MisuraSottospazio} rispecchia la nostra idea di quella che dovrebbe essere la misura su un sottospazio di $\R^n$. In particolare, la misura di Lebesgue $k$-dimensionale dipende fondamentalmente dalla metrica del sottospazio, che è quello che ci aspettiamo.

Consideriamo per esempio il piano $x+y+z=1$ di $\R^3$ e in particolare il triangolo di tale piano formato dalla base canonica di $\R^3$. Tale triangolo è un triangolo equilatero di lato $\sqrt{2}$, vorremmo quindi che la sua misura (cioè la sua ``area'') sia quella di un triangolo equilatero in $\R^2$ con lo stesso lato. Questo però risulta immediato in quanto definiamo la misura del triangolo come la misura di un suo isometrico, che quindi è un triangolo con lo stesso lato.


