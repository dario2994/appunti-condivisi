\section{Misura superficiale}
Obiettivo di questa sezione è formalizzare i concetti di ``lunghezza'', ``superficie'', ``volume'', utilizzando gli strumenti di teoria della misura affrontati finora. 

\begin{definition}
	Una funzione $f:\Omega \subseteq \R^k\to\R^n$ si dice immersione iniettiva se è differenziabile con continuità ed è iniettiva con $Df(x)$ inettivo per ogni $x\in\Omega$. In tal caso si dice che $f$ è una parametrizzazione $C^1$ della superficie $k$-dimensionale $f(\Omega)$.
\end{definition}

\begin{definition}
	$\Sigma\subseteq\R^n$ è una sottovarietà $C^1$ $k$-dimensionale di $\R^n$ se per ogni $x\in\Sigma$ esiste un intorno di $x$ in $\Sigma$ che ammette una parametrizzazione $C^1$ definita su un aperto di $\R^k$.
\end{definition}

La topologia naturalmente indotta su $\Sigma$ è quella che ha come aperti gli aperti di $\R^n$ intersecati con $\Sigma$. D'ora in poi sottoinderemo di star considerando questa topologia. Inoltre chiamiamo Boreliani di $\Sigma$ la \sigalg\ generata dagli aperti di $\Sigma$.

\begin{lemma}\label{lemma:SottovarietaUnioneNumerabile}
	Data $\Sigma$ sottovarietà $C^1$ $k$-dimensionale di $\R^n$, esistono $\{\Sigma_i\}_{i\in\N}\subseteq\Sigma$, aperti nella topologia di $\Sigma$, tali che $\cup_{i\in\N}\Sigma_i=\Sigma$ e $\Sigma_i$ ammette una parametrizzazione $C^1$ definita su un aperto $\Omega_i$ di $\R^k$, per ogni $i\in\N$.
\end{lemma}
\begin{proof}
	$\Sigma$ è separabile in quanto sottoinsieme di $\R^n$ che è separabile, ma di conseguenza è anche Lindelöf, poichè in uno spazio metrico la separabilità e la proprietà di Lindelöf sono equivalenti.
	
	Per la definizione stessa di sottovarietà $C^1$ $k$-dimensionale, per ogni punto $P\in\Sigma$ esiste un suo intorno aperto $\Sigma_P$ (nella topologia di $\Sigma$) che ammette parametrizzazione $C^1$. Abbiamo quindi che $\{\Sigma_P\}_{P\in\Sigma}$ è un ricoprimento aperto di $\Sigma$ e perciò per la proprietà di Lindelöf possiamo estrarne un ricoprimento numerabile, da cui la tesi.
\end{proof}

\begin{remark}\label{nota:BorelianiSottovarieta}
	Utilizzando la stessa notazione del \cref{lemma:SottovarietaUnioneNumerabile}, notiamo che $\Sigma_i$ è un Boreliano di $\R^n$ per ogni $i\in\N$. (AGGIUNGERE CHE IMMAGINE DI APERTI TRAMITE CONTINUE È BORELIANA) Di conseguenza abbiamo anche che i Boreliani di $\Sigma$ sono un sottoinsieme dei Boreliani di $\R^n$.
\end{remark}


\begin{lemma}\label{lemma:InvarianzaImmersione}
	Date $f:\Omega\subseteq\R^k\to M$ e $g:\Omega'\subseteq\R^k\to M$ parametrizzazioni $C^1$ della superficie $k$-dimensionale $M$, per ogni $E\in\mathcal{B}(M)$ sono definiti i seguenti integrali e vale che 
	\begin{equation*}
		\int_{g^{-1}(E)}\sqrt{\det(Dg(x)^TDg(x))}\de x=\int_{f^{-1}(E)}\sqrt{\det(Df(x)^TDf(x))}\de x
	\end{equation*}
\end{lemma}
\begin{proof}
	Innanzitutto notiamo che i due integrali sono definiti, in quanto $Df(x)$ e $Dg(x)$ sono continue e soprattutto $f^{-1}(E)$ e $g^{-1}(E)$ sono misurabili in $\R^k$. Infatti per la \cref{nota:BorelianiSottovarieta} $E$ è un Boreliano di $\R^n$, quindi $f^{-1}(E)$ e $g^{-1}(E)$ sono controimmagini di un Boreliano tramite funzioni misurabili e di conseguenza sono misurabili di $\R^k$ per (AGGIUNGERE CONTROIMMAGINE DI BORELIANI TRAMITE MISURABILI). 

	Sia $\varphi=g\circ f^{-1}$, allora $\varphi$ è un diffeomorfismo $C^1$ fra gli aperti $\Omega$ e $\Omega'$. Quindi per il \cref{thm:CambioVariabile} otteniamo
	\begin{multline*}
		\int_{f^{-1}(E)}\sqrt{\det(Df(x)^TDf(x))}\de x=\int_{\varphi\circ g^{-1}(E)}\sqrt{\det(Df(x)^TDf(x))}\de x=\\
		=\int_{g^{-1}(E)}\sqrt{\det(Df(\varphi(x))^TDf(\varphi(x)))}\ |\det D\varphi(x)|\de x=\int_{g^{-1}(E)}\sqrt{\det(Dg(x)^TDg(x))}\de x\virgola
	\end{multline*}
	dove abbiamo usato che se $g=f\circ \varphi$, allora $Dg(x)=Df(\varphi(x))\cdot D\varphi(x)$.
\end{proof}

\begin{definition}\label{def:MisuraKDimensionale}
	Sia $E$ Boreliano di $\Sigma$, sottovarietà $C^1$ $k$-dimensionale di $\R^n$, e siano $\{\Sigma_i\}_{i\in\N}$ e $\{\Omega_i\}_{i\in\N}$ come nel \cref{lemma:SottovarietaUnioneNumerabile}. Chiamiamo inoltre, per ogni $i\in\N$, $f_i:\Omega_i\to\Sigma_i$ l'immersione iniettiva che parametrizza $\Sigma_i$.
	
	Definiamo la misura di Lebesgue $k$-dimensionale di $E$ come
	\begin{equation*}
		\sigma(E)=\sum_{i\in\N} \sigma_i(E_i\cap \Sigma_i)\virgola
	\end{equation*}
	dove $E_i=E\setminus (\cup_{j<i}\Sigma_j)$ e $\sigma_i(F)=\int_{f_i^{-1}(F)}\sqrt{\det(Df_i(x)^TDf_i(x))} \de x$.
\end{definition}
\begin{theorem}
	La \cref{def:MisuraKDimensionale} è una buona definizione e non dipende dalla scelta del ricoprimento $\{\Sigma_i\}_{i\in\N}$ e delle immersioni iniettive $\{f_i\}_{i\in\N}$.
\end{theorem}
\begin{proof}
	Innanzitutto notiamo che la definizione è ben posta, in quanto tutti gli integrali sono ben definiti. Infatti $Df_i(x)$ è continua per ogni $i\in\N$ e inoltre si dimostra facilmente per induzione su $i$ che $E_i\cap\Sigma_i$ è un Boreliano di $\R^n$, utilizzando la \cref{nota:BorelianiSottovarieta}. Quindi $f_i^{-1}(E_i\cap\Sigma_i)$ è un misurabile di $\R^k$ (analogamente a quanto detto nel \cref{lemma:InvarianzaImmersione}).

	Consideriamo ora due diversi ricoprimenti $\{\Sigma_i\}_{i\in\N}$ e $\{\Sigma_i'\}_{i\in\N}$ di $\Sigma$, che rispettano le ipotesi del \cref{lemma:SottovarietaUnioneNumerabile} e che ammettono rispettivamente parametrizzazioni $\{f_i\}_{i\in\N}$ e $\{g_i\}_{i\in\N}$. Vogliamo dimostrare che le misure $\sigma$ e $\sigma'$ indotte dai due ricoprimenti coincidono.
	
	Sia $B_{i,j}=(\Sigma_i\setminus (\cup_{k<i}\Sigma_k))\cap (\Sigma_j'\setminus (\cup_{k<j}\Sigma_j'))$, allora $\{B_{i,j}\}_{i,j\in\N}$ è una partizione di $\Sigma$ con elementi disgiunti (è facile verificare che $B_{i,j}\cap B_{i',j'}=\emptyset$ per ogni $i,j,i',j'\in\N$). In particolare, poichè
	\begin{equation*}
		E_i\cap\Sigma_i=(E\setminus (\cup_{j<i}\Sigma_j))\cap \Sigma_i=E\cap (\Sigma_i\setminus (\cup_{k<i}\Sigma_k))=E\cap (\cup_{j\in\N}B_{i,j})=\cup_{j\in\N}(E\cap B_{i,j})\virgola
	\end{equation*}
	abbiamo che
	\begin{equation*}
		\sigma_i(E_i\cap \Sigma_i)=\sigma_i(\cup_{j\in\N}(E\cap B_{i,j}))=\sum_{j\in\N}\sigma_i(E\cap B_{i,j})\virgola
	\end{equation*}
	dove l'ultima uguaglianza è vera perchè dati $A_j=E\cap B_{i,j}$, che sono disgiunti fra loro, risulta
	\begin{align*}
		\sigma_i(\cup_{j\in\N}A_j)& =\int_{f_i^{-1}(\cup_{j\in\N}A_j)}\sqrt{\det(Df_i(x)^TDf_i(x))} \de x\\
		&=\sum_{j\in\N}\int_{f_i^{-1}(A_j)}\sqrt{\det(Df_i(x)^TDf_i(x))} \de x=\sum_{j\in\N}\sigma_i(E\cap B_{i,j})\punto
	\end{align*}
	
	Otteniamo quindi che
	\begin{equation*}
		\sigma(E)=\sum_{i\in\N}\sigma_i(E_i\cap \Sigma_i)=\sum_{i,j\in\N} \sigma_i(E\cap B_{i,j})
	\end{equation*}
	e analogamente per $\sigma'$.

	Ma per il \cref{lemma:InvarianzaImmersione} abbiamo che $\sigma_i(E\cap B_{i,j})=\sigma_j'(E\cap B_{i,j})$, da cui otteniamo quindi che $\sigma(E)=\sigma'(E)$.

\end{proof}





Studieremo innanzitutto il caso di un sottospazio $k$-dimesionale di $\R^n$, per poi dare una definizione più generale che vedremo coincidere a quella precedente nel caso ristretto.

\begin{remark}\label{nota:IsometriaSottospazio}
	Dato $V$ sottospazio affine $k$-dimensionale di $\R^n$, allora esiste $U:\R^k\to V$ isometria lineare.
\end{remark}

\begin{definition}\label{def:MisuraSottospazio}
	Dato $V$ sottospazio affine $k$-dimensionale di $\R^n$ e $U$ come nella \cref{nota:IsometriaSottospazio}, definiamo i Boreliani $\mathcal{B}(V)$ di $V$ come le immagini attraverso $U$ dei Boreliani di $\R^k$. Inoltre per ogni $E\in\mathcal{B}(V)$, definiamo la misura di Lebesgue $k$-dimensionale come $\sigma(E)=m_n(U^{-1}(E))$.
\end{definition}

\begin{remark}
	La \cref{def:MisuraSottospazio} non dipende dalla scelta dell'isometria lineare.
\end{remark}
\begin{proof}
	Siano $U:\R^k\to V$ e $W:\R^k\to V$ isometrie lineari che parametrizzano $V$. Allora $T=U\circ W^{-1}:\R^k\to\R^k$ è un'isometria di $\R^k$, in quanto composizione di isometrie.
	
	Consideriamo ora $E\subseteq V$, allora vale che $W^{-1}(E)=T\circ U^{-1}(E)$. Per la \cref{prop:LebesgueProprietaIsometria}, abbiamo però che la misura di Lebesgue è invariante per isometria, quindi $W^{-1}(E)$ appartiene ai Boreliani di $\R^k$ se e solo se ci appartiene anche $U^{-1}(E)$, poichè si ottengono uno dall'altro mediante un'isometria. Analogamente dato $E\in\mathcal{B}(V)$ vale $m_n(W^{-1}(E))=m_n(T\circ U^{-1}(E))=m_n(U^{-1}(E))$, da cui quello che volevamo dimostrare.
\end{proof}

Notiamo che la \cref{def:MisuraSottospazio} rispecchia la nostra idea di quella che dovrebbe essere la misura su un sottospazio di $\R^n$. In particolare, la misura di Lebesgue $k$-dimensionale dipende fondamentalmente dalla metrica del sottospazio, che è quello che ci aspettiamo.

Consideriamo per esempio il piano $x+y+z=1$ di $\R^3$ e in particolare il triangolo di tale piano formato dalla base canonica di $\R^3$. Tale triangolo è un triangolo equilatero di lato $\sqrt{2}$, vorremmo quindi che la sua misura (cioè la sua ``area'') sia quella di un triangolo equilatero in $\R^2$ con lo stesso lato. Questo però risulta immediato in quanto definiamo la misura del triangolo come la misura di un suo isometrico, che quindi è un triangolo con lo stesso lato.


