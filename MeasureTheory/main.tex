\documentclass[a4paper,12pt]{article}
\usepackage{stilebase}
% \usepackage{float}
% \usepackage{figure}

\title{Appunti di teoria della misura elementare}
\author{Federico Glaudo e Marco Trevisiol}

\begin{document}

\maketitle
% \clearpage


\begin{abstract}
	Trattiamo in queste brevissime dispense i fatti fondamentali (e fondazionali) di teoria della misura elementare.
	
	In particolare, dopo una prima sezione prevalentemente di definizioni e fatti preparatori, dimostreremo il teorema di estensione di Caratheodory, che permette di costruire misure a partire da strutture ben più semplici come le misure elementari.
	
	Il percorso seguito ricalca quello proposto dal professor Majer durante il corso di Analisi II dell'anno 2013-2014 alla facoltà di matematica a Pisa.
\end{abstract}
\clearpage

% \tableofcontents
% \clearpage

\section{Definizioni e fatti introduttivi}
In questa sezione definiremo tutti gli oggetti necessari per fondare la teoria della misura e dimostreremo alcuni fatti, perlopiù di carattare insiemistico su di essi.

In particolare definiremo alcune strutture insiemistiche (algebre, \sigalg[e], \semiring[i]) e ne dimostreremo alcune proprietà. Poi passeremo a trattare gli spazi di misura ed alcune loro \emph{versioni più deboli} come le misure esterne e le premisure.

L'idea generale di queste pagine è dare gli strumenti necessari a comprendere il teorema di estensione di una premisura, nella cui dimostrazione ed enunciato convergono tutti i fatti qui trattati. 

Da notare infine che la definizione di \semiring{} non è quella canonica (nè forse si può parlare di una definizione canonica in letteratura) ed anzi è più debole di quella tipicamente usata per enunciare il teorema di estensione. Questo rende più generale il risultato, ma allo stesso tempo rende più ostico dimostrare le prime proprietà della premisura che infatti necessiteranno di vari lemmi tecnici per essere dimostrate (e sarebbero banali se al posto di un \semiring{} ci fosse un anello).

\begin{definition}[Algebra]
	Dato un insieme $X$, una famiglia $\mathcal A\subseteq\mathcal P(X)$ è un'algebra se valgono:
	\begin{itemize}
		\item $\emptyset\in\mathcal A$
		\item $\forall A\in\mathcal A:\ A^c\in\mathcal A$ cioè un'algebra è stabile per passaggio al complementare.
		\item $\forall A,B\in\mathcal A:\ A\cup B\in\mathcal A$ cioè un'algebra è stabile per unioni finite.
	\end{itemize}
\end{definition}
\begin{remark}\label{ProprietaAlg}
	Un'algebra è stabile anche per intersezioni finite e per differenza insiemistica.
\end{remark}
\begin{proof}
	Poichè vale la formula insiemistica:
	\begin{equation*}
		\bigcap_{i\in I} A_i = \left( \bigcup_{i\in I} A_i^c \right)^c
	\end{equation*}
	e un'algebra è stabile per unione finita e complementare, facilmente risulta esserlo anche per intersezioni finite.
	
	Per la differenza si sfrutta la seguente relazione insiemistica $A\setminus B=A\cup B^c$. Questa porta a concludere visto che abbiamo appena dimostrato che $\mathcal A$ è stabile per intersezioni.
\end{proof}


\begin{definition}[\sigalg{}]
	Dato un insieme $X$, una famiglia $\mathcal A\subseteq\mathcal P (X)$ si dice \sigalg{} se valgono:
	\begin{itemize}
	\item $\emptyset\in \mathcal A$
	\item $\forall A\in \mathcal A:\ A^c\in \mathcal A$ cioè una \sigalg{} è stabile per passaggio al complementare.
	\item $\forall (A_n)_{n\in\mathbb N}\subseteq \mathcal A:\ \bigcup_{n\in\mathbb N} A_n\in \mathcal A$ cioè una \sigalg{} è stabile per unioni numerabili.  
	\end{itemize}
\end{definition}

\begin{remark}\label{ProprietaSigAlg}
	Una \sigalg{} è stabile anche per intersezioni numerabili e per differenza insiemistica.
\end{remark}
\begin{proof}
	Si dimostrano entrambe le proprietà in modo del tutto analogo a come abbiamo dimostrato \cref{ProprietaAlg}.
\end{proof}

\begin{definition}[\Semiring{}]
	Una famiglia $\mathcal S\subseteq \mathcal P(X)$ è detta \semiring{} se valgono le seguenti proprietà:
	\begin{itemize}
		\item $\emptyset\in \mathcal S$
		\item $\displaystyle\forall A,B\in \mathcal S: A\cap B, A\setminus B\in \sqcup \mathcal S$ dove
		$\displaystyle
		\sqcup{ \mathcal S }=\left\{\bigsqcup_{n\in \mathbb N} S_n\ |\ (S_n)_{n\in\mathbb N} \subseteq \mathcal S \wedge \forall i\not= j:\ S_i\cap S_j=\emptyset\right\}$ 
		cioè un \semiring{} non deve essere stabile per intersezione e differenza, ma queste si devono scrivere come unioni disgiunte.
	\end{itemize}
\end{definition}

\begin{proposition}\label{UnioneDisgiuntaQuasiAlgebra}
	Dato un \semiring{} $\mathcal S$ l'insieme $\sqcup\mathcal S$ è stabile per intersezione finita e unione numerabile.
\end{proposition}
\begin{proof}
	Notiamo intanto che, per definizione, $\sqcup\mathcal S$ è stabile per unione disgiunta numerabile.
	
	Per dimostrare la stabilità di $\sqcup\mathcal S$ per intersezione finita, basta ovviamente farlo per due soli insiemi $A,B\in\sqcup\mathcal S$. Per definizione possiamo scrivere $A=\bigsqcup_{n\in\mathbb N} A_n, B=\bigsqcup_{n\in\mathbb N} B_n$ dove $(A_n)_{n\in\mathbb N},(B_n)_{n\in\mathbb N}$ sono successioni in $\mathcal S$. Allora vale la seguente identità:
	\begin{equation*}
		A\cap B=\bigsqcup_{n\in\mathbb N} A_n\cap\bigsqcup_{n\in\mathbb N} B_n=
		\bigsqcup_{n,m\in\mathbb N} A_n\cap B_m\in\sqcup\mathcal S
	\end{equation*}
	dove nell'ultimo passaggio è stata usata la stabilità di $\sqcup\mathcal S$ per unione disgiunta.
	
	Per l'unione, consideriamo $A,B\in\mathcal S$. Poichè vale $A\cup B=(A\setminus B)\sqcup(A\cap B)$, viste le proprietà di un \semiring{}, risulta $A\cup B\in \sqcup\mathcal S$. Da questo è facile ottenere che anche unioni finite di elementi di $\mathcal S$ appartengono a $\sqcup\mathcal S$.
	
	Sfruttando quanto detto, fissata $(A_n)_{n\in\mathbb N}\subseteq\mathcal S$ vale:
	\begin{equation}\label{UnioneNumerabileDaS}
		\bigcup_{n\in\mathbb N} A_n=\bigsqcup_{n\in\mathbb N} A_n\setminus\cup_{i<n} A_i
		=\bigsqcup_{n\in\mathbb N} \bigcap_{i<n} (A_n\setminus A_i)\in\sqcup\mathcal S
	\end{equation}
	dove nell'ultimo passaggio abbiamo applicato la stabilità di $\sqcup\mathcal S$ per unione disgiunta e intersezione finita.
	
	E infine dimostriamo la stabilità di $\sqcup\mathcal S$ per unioni numerabili. Sia $(S_n)_{n\in\mathbb N}$ una successione in $\sqcup\mathcal S$. Per definizione devono esistere le successioni $(A^n_i)_{i\in\mathbb N}\subseteq \mathcal S$ tali che $S_n=\bigsqcup_{i\in\mathbb N} A^n_i$.
	
	Allora applicando \cref{UnioneNumerabileDaS} otteniamo:
	\begin{equation*}
		\bigcup_{n\in\mathbb N}S_n=\bigcup_{i,n\in\mathbb N}A^n_i\in\sqcup\mathcal S 
	\end{equation*}
	che è proprio la stabilità di $\sqcup\mathcal S$ per unioni numerabili.
\end{proof}

\begin{definition}[{\sigadd[ità]}]
	Una funzione $\mu:\mathcal F\to \Rpiu$, dove $\mathcal F$ è una famiglia di insiemi, si dice \sigadd{} se per ogni sottofamiglia numerabile $(F_n)_{n\in\mathbb N}\subseteq \mathcal F$ a due a due disgiunta, tale che l'unione appartiene a $\mathcal F$, vale l'addittività:
	\begin{equation*}
		\mu\left(\bigcup_{n\in\mathbb N}F_n \right)=\sum_{n\in\mathbb N} \mu(F_n) 
	\end{equation*}
\end{definition}
\begin{remark}
	Data $\mu:\mathcal F\to \Rpiu$ \sigadd{}, se $\emptyset\in \mathcal F$ allora $\mu(\emptyset)=0$
\end{remark}
\begin{proof}
	Usando la proprietà di \sigadd[ità] si ottiene $\mu(\emptyset)=\mu(\emptyset)+\mu(\emptyset)$ che porta ovviamente alla tesi.
\end{proof}


\begin{definition}[Spazio di misura]
	Dati $X$ un insieme, $\mathcal A$ una famiglia di sottoinsiemi di $X$ e $\mu:\mathcal A\to \Rpiu$ una funzione, la terna $(X,\mathcal A, \mu)$ si dice uno spazio di misura se:
	\begin{itemize}
		\item la famiglia $\mathcal A$ è una \sigalg{}.
		\item la funzione $\mu$ è \sigadd{}.
	\end{itemize}
	e in questo caso la funzione $\mu$ è detta \emph{misura}.
\end{definition}

D'ora in poi, quando ci si riferirà ad una misura, si darà per scontato che questa si riferisce ad uno spazio di misura.

\begin{remark}\label{MonotoniaMisura}
	Dato $(X,\mathcal A,\mu)$ uno spazio di misura, $\mu$ è monotona, cioè se $A,B\in\mathcal A$ e $A\subseteq B$ allora $\mu(A)\le \mu(B)$.
\end{remark}
\begin{proof}
	Per quanto detto in \cref{ProprietaSigAlg}, $B\setminus A\in\mathcal A$ e perciò sfruttando l'addittività su insiemi disgiunti di $\mu$ otteniamo $\mu(B)=\mu(B\setminus A)+\mu(A)>\mu(A)$ che è la tesi.
\end{proof}
\begin{remark}\label{SubAdditivitaMisura}
	Dato $(X,\mathcal A,\mu)$ uno spazio misurato, la misura $\mu$ è \sigsubadd{}, cioè dati $(A_n)_{n\in\mathbb N}\subseteq\mathcal A$ risulta:
	\begin{equation*}
		\mu\left(\bigcup_{n\in\mathbb N} A_n \right)\le \sum_{n\in\mathbb N}\mu(A_n)
	\end{equation*}
\end{remark}
\begin{proof}
	La dimostrazione risulta molto facile sfruttando la monotonia, mostrata in \cref{MonotoniaMisura}, e la solita decomposizione dell'unione in un'unione disgiunta che permette di applicare la \sigadd[ità]:
	\begin{equation*}
		\mu\left(\bigcup_{n\in\mathbb N} A_n \right)=\mu\left(\bigsqcup_{n\in\mathbb N} A_n\setminus\bigcup_{i<n}A_i \right)=
		\sum_{n\in\mathbb N}\mu\left(A_n\setminus\bigcup_{i<n}A_i\right)\le \sum_{n\in\mathbb N}\mu(A_n)
	\end{equation*}
\end{proof}



\begin{definition}\label{TrascurabiliMisura}
	In uno spazio di misura (o anch'è in ipotesi più deboli come premisura o misura esterna) un insieme con misura nulla è detto trascurabile.
\end{definition}
\begin{remark}\label{UnioneTrascurabili}
	Unione numerabile di insiemi trascurabile è a sua volta trascurabile.
\end{remark}
\begin{proof}
	È un'ovvia conseguenza della \sigsubadd[ità] della misura dimostrata in \cref{SubAdditivitaMisura}.
\end{proof}




\begin{definition}\label{FinitezzaMisura}
	Dato $(X,\mathcal A,\mu)$ uno spazio di misura, la misura $\mu$ è detta finita se $X\in\mathcal{A}$ e $\mu(X)<+\infty$.
\end{definition}

\begin{definition}\label{CompletezzaMisura}
	Dato $(X,\mathcal A,\mu)$ uno spazio di misura, la misura $\mu$ si dice completa se per ogni $A\in\mathcal A$ tale che $\mu(A)=0$, ogni suo sottoinsieme è anch'esso appartenente ad $\mathcal A$.
\end{definition}

\begin{proposition}\label{CompletamentoMisura}
	Dato uno spazio di misura $(X,\mathcal A,\mu)$, chiamo $\mu$-completamento di $\mathcal A$ l'insieme $\mathcal A^*$ formato dagli elementi di $\mathcal A$ uniti con un sottoinsieme di un trascurabile:
	\begin{equation*}
		\mathcal A^*=\{A\cup N\ |\ A\in\mathcal A\ \wedge \exists B:\ N\subseteq B,\ \mu(B)=0\}
	\end{equation*}
	Allora $\mathcal A^*$ è una \sigalg{} e $\mu$ si estende in maniera canonica su $\mathcal A^*$ ad una misura completa.
\end{proposition}
\begin{proof}
	Dimostriamo intanto che $\mathcal A^*$ è una \sigalg{}.
	Dato $A^*\in\mathcal A^*$, esistono per definizione $A\in\mathcal A$ e $N\subseteq B\in\mathcal A$ dove $\mu(B)=0$, tali che $A^*=A\cup N$. Possiamo facilmente imporre $B\cap A=\emptyset$, visto che se così non è si può ridefinire $N,B$ facendone la differenza con $A$. Assumiamo perciò $A,B$ disgiunti.
	Chiamando $M=B\setminus N$, vale facilmente:
	\begin{equation*}
		(A^*)^c=(A\cup N)^c=A^c\cap N^c=A^c\cap(B^c\cup M)=(A^c\cap B^c )\cup M \in\mathcal A^*
	\end{equation*}
	dove l'ultima uguaglianza vale perchè ho supposto $A$ e $B$ disgiunti e l'ultima appartenenza è invece vera poichè $M$ è anch'esso sottoinsieme di un trascurabile. Quindi $\mathcal A^*$ è stabile per passaggio al complementare.
	
	Ora consideriamo $(A^*_n)_{n\in\mathbb N}$ una successione di insiemi, e scegliamo $(A_n)_{n\in\mathbb N},(B_n)_{n\in\mathbb N},(N_n)_{n\in\mathbb N}$ tali che valgano $A^*_n=A_n\cup N_n$ e $N_n\subseteq B_n\in\mathcal A$ con $\mu(B_n)=0$.
	
	Allora risulta:
	\begin{align*}
		\bigcup_{n\in\mathbb N}A^*_n &=\bigcup_{n\in\mathbb N}A_n \cup \bigcup_{n\in\mathbb N}N_n\\
		\bigcup_{n\in\mathbb N}N_n &\subseteq \bigcup_{n\in\mathbb N}B_n\in\mathcal A\ \wedge\ \mu\left(\bigcup_{n\in\mathbb N}B_n\right)=0
	\end{align*}
	dove l'ultima uguaglianza dipende dal fatto che la misura è \sigsubadd{} per \cref{SubAdditivitaMisura}. Perciò abbiamo dimostrato che $\mathcal A^*$ è stabile per unione numerabile.
	
	Unendo i due risultati arriviamo a dire che $\mathcal A^*$ è una \sigalg{}.
	
	Definiamo ora $\tilde\mu:\mathcal A^*\to\Rpiu$ di modo che dato $A^*\in\mathcal A^*$ valga $\tilde\mu(A^*)=\mu(A)$ dove $A^*=A\cup N$ con $N$ sottoinsieme di un trascurabile ed $A\in\mathcal A$.
	
	Dimostriamo innanzitutto la coerenza della definizione di $\tilde\mu$. Si deve dimostrare che per $A,A'\in\mathcal A$ e $N,N'$ sottoinsiemi di trascurabili tali che $A\cup N=A'\cup N'$ vale $\mu(A)=\mu(A')$. Siano $B,B'\in\mathcal A$ i trascurabili a cui appartengono $N,N'$, sia infine $Q=B\cup B'$ a sua volta trascurabile.
	
	Per facili ragionamenti di monotonia abbiamo:
	\begin{align*}
		\mu(A)\le \mu(A\cup Q) =\mu(A)+\mu(Q)=\mu(A) &\Longrightarrow \mu(A)=\mu(A\cup Q)\\
		\mu(A')\le \mu(A'\cup Q) =\mu(A')+\mu(Q)=\mu(A') &\Longrightarrow \mu(A')=\mu(A'\cup Q)
	\end{align*}
	ma per costruzione vale $A\cup Q=A'\cup Q$ e perciò otteniamo proprio $\mu(A)=\mu(A')$ che mostra la coerenza della definizione di $\tilde\mu$.
	
	La \sigadd[ità] è ovvia una volta che si è mostrata la coerenza. Presa $(A^*_n)_{n\in\mathbb N}\in\mathcal A^*$ famiglia disgiunta a due a due, sia $(A_n)_{n\in\mathbb N}$ la parte \emph{non trascurabile} della prima successione. Allora risulta, per definizione di $\tilde\mu$:
	\begin{equation*}
		\sum_{n\in\mathbb N} \tilde\mu(A^*_n)=\sum_{n\in\mathbb N} \mu(A_n)=\mu\left(\bigcup_{n\in\mathbb N} A_n\right)=\mu\left(\bigcup_{n\in\mathbb N} A^*_n\right)
	\end{equation*}
	dove nell'ultimo passaggio abbiamo implicitamente applicato \cref{UnioneTrascurabili}.
	
	Resta da verificare che $\tilde\mu$ sia completa, ma questo è ovvio visto che  $\tilde\mu(A^*)=0$ se e solo se $A^*$ è il sottoinsieme di un trascurabile di $\mathcal A$ e la relazione di essere sottoinsieme di un trascurabile è chiusa per estrazione di sottoinsiemi. 

\end{proof}



\begin{proposition}\label{LimiteMonotonoMisura}
	Dato $(X,\mathcal A,\mu)$ uno spazio di misura, allora data una successione $(A_n)_{n\in\mathbb N}\subseteq \mathcal A$ tale che $A_n\subseteq A_{n+1}$ vale:
	\begin{equation*}
		\mu\left(\bigcup_{n\in\mathbb N} A_n\right)=\lim_{n\in\mathbb N} \mu(A_n)
	\end{equation*}
\end{proposition}
\begin{proof}
	Sia $B_n=A_n\setminus\bigcup_{i<n}A_i$. Applicando \cref{ProprietaSigAlg} si ottiene $B_n\in\mathcal A$.
	Per facili ragionamenti insiemistici risulta che la successione $(B_n)_{n\in\mathbb N}$ è disgiunta a due a due ed inoltre $A_n=\bigcup_{i\le n}B_i$.
	Sfruttando tutte queste proprietà e la \sigadd[ità] di $\mu$, otteniamo:
	\begin{equation*}
		\mu\left(\bigcup_{n\in\mathbb N} A_n\right)=\mu\left(\bigcup_{n\in\mathbb N} B_n\right)=
		\sum_{n\in\mathbb N} \mu(B_n)=\lim_{n\to\infty} \sum_{i\le n} \mu(B_i)=
		\lim_{n\to\infty} \mu\left(\bigcup_{i\le n} B_i\right)=\lim_{n\to\infty} \mu(A_n)
	\end{equation*}
	che è proprio la tesi.
\end{proof}

\begin{definition}[Misura esterna]
	Dato un insieme $X$ e una funzione $\mu^*:\mathcal P(X)\to \Rpiu$ è detta una misura esterna se valgono:
	\begin{itemize}
		\item $\mu^*(\emptyset)=0$
		\item $\mu^*$ è monotona, cioè dati $A,B\subseteq X$ se vale $A\subseteq B$ allora $\mu^*(A)\le \mu^*(B)$
		\item $\mu^*$ è \sigsubadd{}, cioè  per ogni successione $(A_n)_{n\in\mathbb N}\subseteq \mathcal P(X)$ di sottoinsiemi di $X$ vale $\mu^*\left(\bigcup_{n\in\mathbb{N}}A_n\right)\le \sum_{n\in\mathbb N} \mu^*(A_n)$
	\end{itemize}
\end{definition}

\begin{remark}
	Dato $(X,\mathcal A,\mu)$ uno spazio di misura, la misura $\mu$ è \sigsubadd{}.
\end{remark}
\begin{proof}
	Data una successione di sottoinsiemi $(A_n)_{n\in\mathbb N}\subseteq \mathcal A$, consideriamo, come nella dimostrazione di \cref{LimiteMonotonoMisura}, i sottoinsiemi $B_n=A_n\setminus\bigcup_{i<n}A_i\in\mathcal A$.
	Allora, lavorando analogamente alla dimostrazione di cui sopra, si ha:
	\begin{equation*}
		\mu\left(\bigcup_{n\in\mathbb N} A_n\right)=\mu\left(\bigcup_{n\in\mathbb N} B_n\right)=
		\sum_{n\in\mathbb N} \mu(B_n)\le \sum_{n\in\mathbb N} \mu(A_n)
	\end{equation*}
	dove nell'ultimo passaggio sfruttiamo la monotonia di $\mu$ dimostrata in \cref{MonotoniaMisura}.
\end{proof}

\begin{definition}
	Una terna $(X,\mathcal S,\mu)$ tale che $\mathcal S\subseteq\mathcal P(X)$ è un \semiring{} e $\mu:\mathcal S\to \Rpiu$ è \sigadd{}, la chiamo spazio di misura elementare e la funzione $\mu$ la chiamo misura elementare o premisura.
\end{definition}

\begin{lemma}\label{CoerenzaPremisura}
	Fissato $(X,\mathcal S,\mu)$ uno spazio di misura elementare, siano $(A_n)_{n\in\mathbb N},(B_n)_{n\in\mathbb N}\subseteq\mathcal S$ delle famiglie tali che l'unione sia la stessa, ma i $(B_n)_{n\in\mathbb N}$ siano disgiunti a due a due: $\bigcup_{n\in\mathbb N}A_n=\bigsqcup_{n\in\mathbb N}B_n$.
	Allora risulta $\sum_{n\in\mathbb N}\mu(A_n)\ge \sum_{n\in\mathbb N}\mu(B_n)$.
\end{lemma}
\begin{proof}
	Sia $A'_n=A_n\setminus\bigcup_{i<n}A_i$. La successione $(A'_n)_{n\in\mathbb N}$ è disgiunta a due a due e ogni singolo elemento appartiene a $\sqcup \mathcal S$ visto che vale $A'_n=\bigcap_{i<n}A_n\setminus A_i$ e $\sqcup \mathcal S$ è chiuso per intersezione finita, come mostrato in \cref{UnioneDisgiuntaQuasiAlgebra}. Infine è chiaro che l'unione della nuova famiglia è uguale a quella di $(A_n)_{n\in\mathbb N}$.
	È importante notare che $A_n\setminus A'_n=A_n\cap\bigcup_{i<n}A_i\in\sqcup\mathcal S$ dove l'ultima appartenenza è vera per la stabilità di $\sqcup\mathcal S$ per unioni e intersezioni finite. Allora esistono $(E^n_i)_{i\in\mathbb N}\subseteq\mathcal S$ tali che in unione disgiunta realizzano $A_n\setminus A'_n$.
	
	Siano quindi $C_{ij}=A'_i\cap B_j$. Ovviamente la successione $(C_{ij})_{i,j\in\mathbb N}$ è disgiunta a due a due (perchè lo sono sia $(A'_n)_{n\in\mathbb N}$ che $(B_n)_{n\in\mathbb N}$) ed è un sottoinsieme di $\sqcup\mathcal S$ poichè intersezione di elementi che vi appartengono. Quindi esiste la famiglia $(F^{ij}_n)_{n\in\mathbb N}\subseteq\mathcal S$ la cui unione disgiunta realizza $C_{ij}$.
	
	Ora per costruzione e per le osservazioni fatte valgono:
	\begin{align*}
		A_n= A'_n\sqcup \bigsqcup_{i\in\mathbb N}E^n_i=\bigsqcup_{i\in\mathbb N}C_{ni}\sqcup \bigsqcup_{i\in\mathbb N}E^n_i
		=\bigsqcup_{i,j\in\mathbb N}F^{ni}_j\sqcup \bigsqcup_{i\in\mathbb N}E^n_i
		&\Longrightarrow \mu(A_n)\ge\sum_{i,j\in\mathbb N}\mu(F^{ni}_j)+\sum_{i\in\mathbb N}\mu(E^n_i)\\
		B_n=\bigsqcup_{i\in\mathbb N}C_{in}=\bigsqcup_{i,j\in\mathbb N}F^{in}_j
		&\Longrightarrow \mu(B_n)=\sum_{i,j\in\mathbb N}\mu(F^{in}_j)
	\end{align*}
	quindi sommando su $n$ arriviamo a:
	\begin{align*}
		\sum_{n\in\mathbb N}\mu(A_n)&\ge \sum_{n\in\mathbb N}\sum_{i,j\in\mathbb N}\mu(F^{ni}_j)
		=\sum_{i,n,j\in\mathbb N}\mu(F^{ni}_j)\\
		\sum_{n\in\mathbb N}\mu(B_n)&=\sum_{n\in\mathbb N}\sum_{i,j\in\mathbb N}\mu(F^{in}_j)=\sum_{i,n,j\in\mathbb N}\mu(F^{in}_j)
	\end{align*}
	ma visto che l'ordine degli indici non conta, questi risultati implicano banalmente la tesi.


\end{proof}



\begin{lemma}\label{PiuCheMonotonaPremisura}
	Dato $(X,\mathcal S,\mu)$ uno spazio di misura elementare, siano $A,(A_n)_{n\in\mathbb N}\subseteq \mathcal S$ tali che $A\subseteq\bigcup_{n\in\mathbb N}A_n$.
	Allora vale $\mu(A)\le \sum_{n\in\mathbb N}\mu(A_n)$.
\end{lemma}
\begin{proof}
	Visto che $A\subseteq\bigcup_{n\in\mathbb N}A_n$ vale la scrittura insiemistica:
	\begin{equation}\label{ScritturaDecenteUnionePremisura}
		\bigcup_{n\in\mathbb N}A_n=A\sqcup\bigcup_{n\in\mathbb N}A_n\setminus A
	\end{equation}
	Poichè $\mathcal S$ è un \semiring{} $A_n\setminus A\in \sqcup \mathcal S$, e visto che $\sqcup S$ è chiuso per unione numerabile, come mostrato in \cref{UnioneDisgiuntaQuasiAlgebra}, esiste una famiglia $(B_n)_{n\in\mathbb N}\subseteq\mathcal S$ disgiunta tale che $\bigcup_{n\in\mathbb N}A_n\setminus A=\bigsqcup_{n\in\mathbb N}B_n$.
	Allora sostituendo in \cref{ScritturaDecenteUnionePremisura} si ottiene:
	\begin{equation*}
		\bigcup_{n\in\mathbb N}A_n=A\sqcup\bigsqcup_{n\in\mathbb N}B_n
	\end{equation*}
	quindi si ricade nelle ipotesi di \cref{CoerenzaPremisura} ottenendo che:
	\begin{equation*}
		\sum_{n\in\mathbb N}\mu(A_n)\ge \mu(A)+\sum_{n\in\mathbb N}\mu(B_n)\ge \mu(A)
	\end{equation*}
	che è proprio quanto si voleva dimostrare.

\end{proof}




\section{Estendere una premisura ad una misura}
L'obiettivo ora è riuscire ad estendere una premisura definita su un semianello ad una misura su una \sigalg{}. Per fare questo il percorso sarà prima quello di estendere la premisura ad una misura esterna, per poi ridurre questa ad una misura canonica.


\begin{theorem}\label{RiduzionePreCaratheodory}
	Data $\mu:\mathcal P(X)\to \Rpiu$ una misura esterna, sia $\A\subseteq \mathcal P(X)$ l'insieme così definito:
	\begin{equation*}
		\A=\{E\in\mathcal P(X):\ \mu(A)=\mu(A\cap E)+\mu(A\setminus E)\ \forall A\in \mathcal P(X)\}
	\end{equation*}
	allora $\A$ è una \sigalg{}, detta \sigalg{} di Caratheodory, e $\mu$ ridotta su $\A$ è una misura completa.
\end{theorem}
\begin{proof}
	La dimostrazione procede in tre passi: prima mostriamo che $\A$ è un'algebra di insiemi, poi che è una \sigalg{} e infine che $\mu$ è \sigadd{} e completa ridotta su $\A$.
	
	Il fatto che $\A$ sia stabile per complementare è ovvio per la definizione (che è simmetrica tra $E$ ed $E^c$).
	
	Fissati $A\in\mathcal P(X)$ generico ed $E,F\in\A$, applicando la sola definizione di $\A$ ed alcuni passaggi insiemistici si ricava:
	\begin{align*}
		\mu(A)\stackrel{F\in\A}{=}&\mu(A\cap F)+\mu(A\setminus F)\stackrel{E\in\A}{=}
		\mu(A\cap F)+\mu\left((A\setminus F)\cap E\right)+\mu\left((A\setminus F)\setminus E\right)\\
		=\hspace{0.4em}&\mu\left((A\cap (E\cup F))\cap F\right)+\mu\left((A\cap (E\cup F))\setminus F\right)+
		\mu\left(A\setminus(E\cup F)\right)\\
		\stackrel{F\in\A}{=}&\mu(A\cap (E\cup F))+\mu\left(A\setminus(E\cup F)\right)
	\end{align*}
	e visto che questo vale per ogni scelta di $A\in\mathcal P(X)$ abbiamo dimostrato che $\A$ è stabile per unione.
	
	Unendo quanto detto si ha facilmente che $\A$ è un'algebra di insiemi.
	
	Ora sia $(E_n)_{n\in\N}\subseteq \A$ una famiglia numerabile di insiemi ed $A\in\mathcal P(X)$ un generico sottoinsieme di $X$.
	
	Per la \sigsubadd[ità] di $\mu$ vale:
	\begin{equation}\label{DisuguaglianzaFacileCaratheodory}
		\mu(A)\le \mu\left(A\cap\bigcup_{n\in\N} E_n\right)+\mu\left(A\setminus\bigcup_{n\in\N} E_n\right)
	\end{equation}
	Si vuole dimostrare che il $\le$ è in realtà un'uguaglianza. Se $\mu(A)=+\infty$ questo è ovvio, quindi tratteremo il caso in cui $\mu(A)<+\infty$. Chiamiamo $F_n=E_n\setminus \bigcup_{i<n} E_i$, ottenendo in maniera ovvia che gli $(F_n)_{n\in\N}$ sono a due a due disgiunti e che appartengono ad $\A$ poiché quest'ultima è un'algebra.
	
	Per induzione è facile verificare, sfruttando unicamente il fatto che $F_n\in\A$ e $\mu(A)<+\infty$, che risulta:
	\begin{equation}\label{IdentitaDifferenzaCaratheodory}
		\mu\left(A\setminus \bigsqcup_{n\le m} F_n\right)=\mu(A)-\sum_{n\le m} \mu(A\cap F_n)
	\end{equation}
	e incidentalmente da questa formula si ha che la serie $\sum_{n\in\N}\mu(A\cap F_n)$ converge, visto che è a termini positivi e limitata (da $\mu(A)$).
	
	Per la \sigsubadd[ità] di $\mu$ vale:
	\begin{equation}\label{IntersezioneStimaCaratheodory}
		\mu\left(A\cap\bigcup_{n\in\N} E_n\right)=\mu\left(\bigsqcup_{n\in\N} A\cap F_n\right)\le
		\sum_{n\in\N} \mu(A\cap F_n)
	\end{equation}
	mentre, grazie alla monotonia e a \cref{IdentitaDifferenzaCaratheodory}, otteniamo:
	\begin{equation}\label{DifferenzaStimaCaratheodory}
		\mu\left(A\setminus\bigcup_{n\in\N} E_n\right) = \mu\left(A\setminus\bigsqcup_{n\in\N} F_n\right) \le \mu\left(A\setminus\bigsqcup_{n\le m} F_n\right) = 
		\mu(A)-\sum_{n\le m}\mu(A\cap F_n)
	\end{equation}
	
	Ora unendo \cref{IntersezioneStimaCaratheodory,DifferenzaStimaCaratheodory} giungiamo ad avere che, per ogni $m\in\mathbb{N}$:
	\begin{equation*}
		\mu\left(A\cap\bigcup_{n\in\N} E_n\right)+\mu\left(A\setminus\bigcup_{n\in\N} E_n\right)\le
		\mu(A)+\sum_{m\le n}\mu(A\cap F_n) 
	\end{equation*}
	ma per la convergenza di $\sum_{n\in \N}\mu(A\cap F_n)$, estraendo l'$\inf$ da entrambe le parti finalmente arriviamo a:
	\begin{equation*}
		\mu\left(A\cap\bigcup_{n\in\N} E_n\right)+\mu\left(A\setminus\bigcup_{n\in\N} E_n\right)\le
		\mu(A)
	\end{equation*}
	che unita a \cref{DisuguaglianzaFacileCaratheodory} ci assicura che vale l'identità tra i membri e, visto che ciò vale indipendentemente dalla scelta di $A\in\mathcal P(X)$, risulta $\bigcup_{n\in\N}E_n\in\A$ che equivale a dire che $\A$ è una \sigalg{}.
	
	Dimostrare che $\mu$ è \sigadd{} su $\A$ è ora molto facile.
	Consideriamo $(E_n)_{n\in\N}\subseteq \A$ una famiglia numerabile di insiemi \emph{disgiunti}. Per facile induzione si ha che:
	\begin{equation*}
		\mu\left(\bigsqcup_{n\le m}E_n\right)=\sum_{n\le m} \mu(E_n)
	\end{equation*}
	e applicando questa e la monotonia di $\mu$ risulta:
	\begin{equation*}
		\sum_{n\le m} \mu(E_n)=\mu\left(\bigsqcup_{n\le m}E_n\right)\le
		\mu\left(\bigsqcup_{n\in\N}E_n\right)\le \sum_{n\in\N} \mu(E_n)
	\end{equation*}
	e questa doppia disuguaglianza, per la definizione delle serie a termini positivi, implica che tutte le disuguaglianze sono identità. Ma allora questo dimostra che $\mu$ è \sigadd{} su $\A$.
	
	Infine per dimostrare la completezza di $\mu|_{\A}$ basta mostrare che dato $E\in\mathcal P(X)$ trascurabile, vale $E\in\A$ (questo è sufficiente a mostrare la completezza, visto che per monotonia i sottinsiemi di un trascurabile sono a loro volta trascurabili).
	
	Fissato un generico $A\in\mathcal P(X)$, risulta per la monotonia di $\mu$:
	\begin{equation*}
		\mu(A\cap N)+\mu(A\setminus N)\le \mu(N)+\mu(A)=\mu(A)
	\end{equation*}
	che, unita alla \sigsubadd[ità] di $\mu$ mi assicura
	\begin{equation*}
		\mu(A)=\mu(A\cap N)+\mu(A\setminus N)
	\end{equation*}
	che è proprio la condizione di appartenenza ad $\A$.
\end{proof}

\begin{proposition}\label{MisuraEsternaDiPremisura}
	Dato $(X,\S,\mu)$ uno spazio di misura elementare si consideri la funzione che associa ad ogni sottoinsieme l'estremo inferiore delle misure dei ricoprimenti, cioè $\mu^*:\mathcal P(X)\to\Rpiu$ definita come 
	\begin{equation*}
		\mu^*(A)=\inf\left\{\sum_{n\in\N} \mu(A_n)\ |\ (A_n)_{n\in\N}\subseteq\S\ \wedge
		\ A\subseteq\bigcup_{n\in\N}A_n\right\}
	\end{equation*}
	Allora $\mu^*$ è una misura esterna che estende $\mu$ (cioè $\mu^*|_\S=\mu$) ed inoltre $\S$ appartiene alla relativa \sigalg{} di Caratheodory (come definita in \cref{RiduzionePreCaratheodory}).
\end{proposition}
\begin{proof}
	Per affermare che $\mu^*$ è una misura esterna sono sufficienti le seguenti verifiche.
	Ovviamente, poiché $\mu(\emptyset)=0$, vale $\mu^*(\emptyset)=0$. 
	Inoltre, ancora facilmente, $\mu^*$ è monotona, visto che se $A\subseteq B$ un ricoprimento di $B$ ricopre anche $A$.
	E infine è anche \sigsubadd{} visto che l'unione di ricoprimenti (che risulta ancora un ricoprimento numerabile) è un ricoprimento dell'unione.
	
	Dato $S\in\S$ vale ovviamente $\mu^*(S)\le\mu(S)$, poiché $S$ si ricopre da solo. Per dimostrare la disuguaglianza opposta, ottenendo così che $\mu^*$ estende $\mu$, consideriamo $(S_n)_{n\in\N}\in \S$ un ricoprimento di $S$. Per \cref{PiuCheMonotonaPremisura} vale:
	\begin{equation*}
		\mu(S)\ge \sum_{n\in\N} \mu(S_n)
	\end{equation*}
	e perciò passando all'estremo inferiore sui ricoprimenti otteniamo la disuguaglianza cercata.
	
	Ora perciò resta da dimostrare che se $E\in \S$ allora per ogni $A\in\mathcal P(X)$ risulta:
	\begin{equation}\label{MisuraEsternaDisDifficile}
		\mu^*(A) \ge \mu^*(A\cap E)+\mu^*(A\setminus E)
	\end{equation}
	Questo è sufficiente ad avere che $\S$ è contenuto nella \sigalg{} di Caratheodory, poiché l'altra disuguaglianza è assicurata dalla \sigsubadd[ità].
	
	Dato $(A_n)_{n\in\N}\subseteq\S$ un ricoprimento di $A$, chiamiamo $B_n=A_n\cap E$ e $C_n=A_n\setminus E$. Ovviamente $(B_n)_{n\in\N},(C_n)_{n\in\N}$ ricoprono rispettivamente $A\cap E,A\setminus E$. Poiché $\S$ è un \semiring{} riusciamo però a trovare $(B'^n_i)_{i\in\N},(C'^n_i)_{i\in\N} \subseteq \S$ tali che $B_n=\bigsqcup_{i\in\N}B'^n_i$ e analogo risultato per $C_n$. Quindi $(B'^n_i)_{n,i\in\N}, (C'^n_i)_{n,i\in\N}$ risultano ricoprimenti con elementi di $\S$ di $A\cap E,A\setminus E$ rispettivamente.
	Ora, sfruttando non più della sola \sigadd[ità] di $\mu$ concludiamo:
	\begin{align*}
		\sum_{n\in\N}\mu(A_n)=\sum_{n\in\N} \mu(B_n)+\mu(C_n)&=
		\sum_{n\in\N}\sum_{i\in\N}\mu(B'^n_i)+\mu(C'^n_i)\\
		&=
		\sum_{n,i\in\N}\mu(B'^n_i)+\sum_{n,i\in\N}\mu(C'^n_i)\ge \mu(A\cap E)+\mu(A\setminus E)
	\end{align*}
	ma questo implica facilmente \cref{MisuraEsternaDisDifficile} estraendo l'estremo inferiore a entrambi i membri sui ricoprimenti di $A$.
\end{proof}

\begin{theorem}[Estensione di Caratheodory]\label{EstensioneCaratheodory}
	Dato $(X,\S,\mu)$ uno spazio di misura elementare esiste una \sigalg{} $\A$ e una funzione $\mu':\A\to\Rpiu$ tali che $\S\subseteq \A$, $\mu'$ estende la premisura $\mu$ e $(X,\A,\mu')$ è uno spazio di misura completo.
\end{theorem}
\begin{proof}
	Consideriamo la misura esterna $\mu^*:\mathcal P(X)\to\Rpiu$ definita nell'enunciato di \cref{MisuraEsternaDiPremisura}. Sempre \cref{MisuraEsternaDiPremisura} ci assicura che questa è un'estensione di $\mu$.
	
	Possiamo ora ridurre $\mu^*$ grazie al \cref{RiduzionePreCaratheodory} ad una misura completa $\mu':\A\to\Rpiu$ dove $\A$ è la \sigalg{} di Caratheodory. 
	
	Ma come dimostrato in \cref{MisuraEsternaDiPremisura} $\S\subseteq\A$ e inoltre vale $\mu'|_\S=\mu^*|_\S=\mu$, perciò lo spazio $(X,\A,\mu')$ rispetta tutte le richieste dell'enunciato.
\end{proof}


\begin{proposition}
	Dato $(X,\S,\mu)$ una spazio di misura elementare $\sigma$-finito, sia $\mu^*$ la misura esterna associata a $\mu$ (come definita nell'enunciato di \cref{MisuraEsternaDiPremisura}) e $\A$ la \sigalg{} di Caratheodory (la cui esistenza mi è assicurata dal \cref{EstensioneCaratheodory}) con la relativa estensione della misura $\mu':\A\to\Rpiu$.
	
	Scelto $A\in\mathcal P(X)$, le seguenti affermazioni sono equivalenti:
	\begin{enumerate}[label=(\arabic*),ref=(\arabic*)]
		\item $A\in\A$ cioè l'insieme è un misurabile (secondo Caratheodory).\label{MisurabileEquivalenze}
		\item Fissato $\epsilon>0$ esiste $S\in\sqcup\S$ che contiene $A$ e tale che $\mu^*(S\setminus A)<\epsilon$.\label{UnioniDaFuoriEquivalenze}
		\item Esiste $B\in\sigma A(S)$ che $B$ che contiene $A$ e tale che $\mu^*(B\setminus A)=0$.\label{SigmaDaFuoriEquivalenze}
		\item Fissato $\epsilon>0$ esiste $S$ contenuto in $A$ tale che $S^c\in\sqcup\S$  e $\mu^*(A\setminus S)<\epsilon$.\label{UnioniDaDentroEquivalenze}
		\item Esiste $B\in\sigma A(S)$ che è contenuto in $A$ e tale che $\mu^*(A\setminus B)=0$.\label{SigmaDaDentroEquivalenze}
	\end{enumerate}
\end{proposition}
\begin{proof}
	Dimostriamo innanzitutto la catena di implicazioni
	$\text{\ref{MisurabileEquivalenze}}\implies
	\text{\ref{UnioniDaFuoriEquivalenze}}\implies
	\text{\ref{SigmaDaFuoriEquivalenze}}\implies
	\text{\ref{MisurabileEquivalenze}}$.
	\newcommand{\ImplicationProof}[2]{$\text{\ref{#1}}\implies\text{\ref{#2}}$}% X => Y
	\begin{description}
		\item[\ImplicationProof{MisurabileEquivalenze}{UnioniDaFuoriEquivalenze}] Sia $A$ misurabile e $(X_n)_{n\in\N}\subseteq \S$ una partizione numerabile di $X$ tale che $\mu(X_n)<+\infty$ (questa esiste poichè $\mu$ è \sigfin{} e ogni unione numerabile di elementi in $\S$ è un'unione disgiunta, visto che $\sqcup\S$ è chiuso per unioni numerabili come mostrato in \cref{UnioneDisgiuntaQuasiAlgebra}).
		
		Definiamo $A_n=A\cap X_n$. Gli $A_n$ sono misurabili, perchè intersezioni di misurabili, e hanno misura finita dato che la misura è monotona.
		
		Sia $\epsilon>0$ fissato.
		
		Dato $n\in\N$ allora $\mu'(A_n)=\mu^*(A_n)$, ma per la definizione di $\mu^*$ questo implica che esiste $S_n\in\sqcup \S$ tale che $A_n\subseteq S_n$ e tale che valga:
		\begin{equation*}
			\mu^*(A_n)\le \mu^*(S_n) \le \mu^*(A_n)+\frac\epsilon{2^n}\implies
			0\le \mu^*(S_n)-\mu^*(A_n)\le \frac\epsilon{2^n}
		\end{equation*}
		dove nell'implicazione abbiamo sfruttato $\mu^*(A_n)=\mu'(A_n)<+\infty$.
		Infine dal fatto che $S_n,A_n$ sono misurabili, per l'addittività della misura, ricaviamo:
		\begin{equation}\label{DifficileFreccia12Misurabili}
			0\le \mu'(S_n\setminus A_n) \le \frac\epsilon{2^n} \implies \mu^*(S_n\setminus A_n)\le \frac\epsilon{2^n}
		\end{equation}
		
		Ora consideriamo $S=\bigcup_{n\in\N}S_n$, che appartiene ancora a $\sqcup\S$ poichè questo è chiuso per unione numerabile come dimostrato in \cref{UnioneDisgiuntaQuasiAlgebra}. Si ha $A\subseteq S$ e per la subaddittività della misura esterna  $\mu^*$, sfruttando \cref{DifficileFreccia12Misurabili}, si ottiene:
		\begin{equation*}
			\mu^*(S\setminus A)=\mu^*\left(\bigcup_{n\in\N}S_n\setminus A_n\right)\le
			\sum_{n\in\N}\mu^*(S_n\setminus A_n)\le\sum_{n\in\N}\frac\epsilon{2^n}\le \epsilon
		\end{equation*}
		e questo conclude, visto che $S$ soddisfa tutte le richieste dell'enunciato.
		\item[\ImplicationProof{UnioniDaFuoriEquivalenze}{SigmaDaFuoriEquivalenze}] L'ipotesi ci assicura l'esistenza di $(S_n)_{n\in\N}\subseteq \sqcup \S$ tali che $A\subseteq S_n$ e $\mu^*(S_n\setminus A)<\frac1n$.
		
		Definiamo $B=\cap_{n\in\N}S_n$, che appartiene a $\sigma A(\S)$ poichè generato a partire da $\S$ con sole unioni e intersezioni numerabili. Ovviamente vale $A\subseteq B$.
		
		Infine risulta vera la seguente:
		\begin{equation*}
			\forall n\in\N:\ B\subseteq B_n\implies B\setminus A\subseteq B_n\setminus A\implies \mu^*(B\setminus A)\le \mu^*(B_n\setminus A)\le \frac1n
		\end{equation*}
		e perciò $\mu^*(B\setminus A)=0$ e questo dimostra l'implicazione cercata.
		\item[\ImplicationProof{SigmaDaFuoriEquivalenze}{MisurabileEquivalenze}] Per ipotesi esiste $B\in\sigma A(\S)$ che contiene $A$ e tale che $\mu^*(B\setminus A)=0$.
		
		Poichè $\A$ è una \sigalg{} contenente $\S$ e $B\in\sigma A(\S)$, deve essere $B\in\A$. 
		Inoltre $\mu':\A\to\Rpiu$ è una misura completa, come visto nel \cref{RiduzionePreCaratheodory}, e inoltre i trascurabili secondo $\mu^*$ risultano essere proprio i trascurabili secondo $\mu'$ (come visto nella dimostrazione di \cref{RiduzionePreCaratheodory}) quindi anche $B\setminus A\in \A$ in quanto è un trascurabile e la misura è completa. 
		
		Di conseguenza, visto che $\A$ è una \sigalg, otteniamo che $A=B\setminus (B\setminus A)$ è misurabile in quanto differenza di misurabili. Perciò $A\in\A$ dimostrando l'implicazione.
	\end{description}
	
	Infine basta vedere, sfruttando il passaggio al complementare, che risultano $\text{\ref{UnioniDaFuoriEquivalenze}}\iff\text{\ref{UnioniDaDentroEquivalenze}}$ e $\text{\ref{SigmaDaFuoriEquivalenze}}\iff\text{\ref{SigmaDaDentroEquivalenze}}$ e questo chiude la dimostrazione.
\end{proof}

\section{Funzioni misurabili}
In questa sezione daremo la definizione di funzione misurabile e dimostreremo alcuni fatti basilari su di esse.

La teoria delle funzioni misurabili, oltre ad essere strettamente necessaria per la successiva teoria dell'integrazione, ci permetterà di dimostrare che i misurabili secondo Lebesgue, introdotti nella sezione precedente, non coincidono con i Boreliani usando strumenti propri della teoria della misura.
Questo fatto lo dimostriamo però solo nel caso unidimensionale sia perché il risultato è già stato dimostrato, sia perché la dimostrazione si adatta facilmente in dimensione maggiore e sia perché troviamo importante rendere chiara l'idea piuttosto che celarla in una notazione troppo pesante.

\begin{definition}[Retta reale estesa]
	Indichiamo con $\Rbar$ l'insieme $\R\cup\{-\infty,+\infty\}$ munito della topologia che ha come base tutti gli aperti tipici di $\R$ e gli insiemi del tipo $\{-\infty\}\cup\oo{-\infty}a$ e $\{+\infty\}\cup\oo{a}{+\infty}$ dove $a$ è un numero reale. 
	Quest'ultima categoria di insiemi aperti verrà indicata tramite la notazione $\co{-\infty}a$ e $\oc a{+\infty}$.
\end{definition}


\begin{definition}[Funzione misurabile]
	Dato uno spazio di misura $(X,\A,\mu)$, una funzione $f:X\rightarrow \Rbar$ si dice misurabile se
	$\forall A \subseteq \Rbar$ aperto si ha $f^{-1}(A)\in \A$.
\end{definition}

\begin{proposition}\label{prop:BasicMis}
	Dato uno spazio di misura $(X,\A,\mu)$, sia $f:X\rightarrow \Rbar$ una funzione, sono equivalenti i seguenti fatti
	\footnote{Qui introduciamo la notazione per i \textit{sovralivelli} di una funzione che useremo in tutti gli appunti:
		data una funzione $f$ di codominio reale e un certo reale $k$ indichiamo con $\{f>k\}$ l'insieme
		$\{x:f(x)>k\}=f^{-1}((k,+\infty])$; stessa notazione verrà usata anche per il sottolivello.}:
	\begin{enumerate}[label=(\arabic*),ref=(\arabic*)]
		\item $f$ è misurabile; \label{BM:mis}
		\item $\{f<a\}\in \A \quad \forall a\in \Rbar$; \label{BM:sot}
		\item $\{f\le a\}\in \A \quad \forall a\in \Rbar$; \label{BM:soteq}
		\item $\{f>a\}\in \A \quad \forall a\in \Rbar$; \label{BM:sov}
		\item $\{f\ge a\}\in \A \quad \forall a\in \Rbar$;  \label{BM:soveq}
		\item $\{a<f<b\}\in \A \quad \forall a,b\in \Rbar$. \label{BM:int}
	\end{enumerate}
\end{proposition}
\begin{proof}
	Sfruttando le proprietà di $\A$ come \sigalg, mostriamo a catena tutte le implicazioni:
	\begin{description}
	\item[\ImplicationProof{BM:mis}{BM:sot}] per definizione di misurabile, $\{f<a\}=f^{-1}\left(\co{-\infty}{a}\right)\in \A$,
		perché $\co{-\infty}{a}$ è aperto;
	\item[\ImplicationProof{BM:sot}{BM:soteq}] poiché $\cc{-\infty}{a}=\bigcap_{n\in \N}\co{-\infty}{a+\frac{1}{n}}$, allora otteniamo
		\begin{equation*}
			\{f\le a\}=f^{-1}\left(\cc{-\infty}{a}\right)=\bigcap_{n\in \N}f^{-1}\left(\co{-\infty}{a+\frac{1}{n}}\right)\in \A	
		\end{equation*}

	\item[\ImplicationProof{BM:soteq}{BM:sov}] passando al complementare, $\{f>a\}^\mathsf{c}=\{f\le a\}\in \A$;
	\item[\ImplicationProof{BM:sov}{BM:soveq}] analogamente a \ImplicationProof{BM:sot}{BM:soteq}; 
	\item[\ImplicationProof{BM:soveq}{BM:sot}] analogamente a \ImplicationProof{BM:soteq}{BM:sov};
	\item[$\text{\ref{BM:sot}}\ +\ \text{\ref{BM:sov}}\implies\text{\ref{BM:int}}$] perché
		$\{a<f<b\}=\{a<f\}\cap\{f<b\}\in \A$;
	\item[\ImplicationProof{BM:int}{BM:mis}] dato che un aperto $A\subseteq\Rbar$ si può scrivere come
		$A=\bigcup_{n\in \N}A_n$ dove ciascun $A_n$ è un intervallo aperto di $\Rbar$ (o una semiretta aperta),
		allora $f^{-1}(A)=\bigcup_{n\in \N}f^{-1}(A_n)\in \A$, dove abbiamo sfruttato che per il punto \ref{BM:int} $f^{-1}(A_n)\in \A\ \ \forall n$.
	\end{description}
\end{proof}

\begin{proposition}\label{prop:CounterImgMis}
	Dato uno spazio di misura $(X,\A,\mu)$, e data $f:X\rightarrow \Rbar$ una funzione misurabile, la famiglia di insiemi
	\[
		\mathcal{E} = \{ E\subseteq \Rbar : f^{-1}(E)\in \A \}
	\]
	è una \sigalg{} ed inoltre contiene i Boreliani.
\end{proposition}
\begin{proof}
	Verifichiamo che $\mathcal E$ è stabile per unioni numerabili e passaggio al complementare.
	
	Fissati $\{E_n\}_{n\in \N}\subseteq \mathcal{E}$ sia $E = \cup_{n\in \N}E_n$, vale:
	\begin{equation*}
		f^{-1}(E)=f^{-1}\left(\cup_{n\in \N}E_n\right) = \cup_{n\in \N}f^{-1}(E_n)\in \A \implies E \in \mathcal{E}
	\end{equation*}
	dove l'appartenenza ad $\A$ si ha per le proprietà di \sigalg{}, e questo dimostra la stabilità per unione numerabile.
	
	Per quanto riguarda il passaggio al complementare, fissato $E\in \mathcal{E}$, risulta:
	\begin{equation*}
		f^{-1}(E^\mathsf{c})= f^{-1}(E)^\mathsf{c} \in \A \implies E^\mathsf{c} \in \mathcal{E}.
	\end{equation*}
	
	Inoltre $\mathcal E$ contiene gli aperti per definizione di funzione misurabile, da cui, per quanto appena dimostrato, contiene la \sigalg{} generata da questi, cioè i Boreliani.
\end{proof}

\begin{definition}\label{def:FpiuFmeno}
	Sia $f:X\to\Rbar$ una funzione misurabile su uno spazio di misura $(X,\A,\mu)$. Definiamo $f^+ = \max\{f,0\}$ e $f^- = \max\{-f,0\}$.
\end{definition}
\begin{remark}\label{nota:ProprietaFpiuFmeno}
	Data una funzione $f$ misurabile, vale $f^+,f^-$ sono misurabile. Inoltre $f^+,f^-\ge 0$ e $f=f^+-f^-$. 
\end{remark}
\begin{proof}
	Abbiamo che per ogni $a\in\Rbar$ vale
	\begin{equation*}
		\{f^+>a\}=\left\{\begin{array}{ll}
			\{f>a\}\in\A &\text{se $a>0$}\\
			X\in\A &\text{se $a\le 0$}
	\end{array}\right.
	\end{equation*}
	Quindi, per la \ref{BM:sov} della \cref{prop:BasicMis}, $f^+$ è misurabile. Analogamente si dimostra che $f^-$ è misurabile.
\end{proof}


\begin{proposition}\label{prop:AlgMis}
	Dato uno spazio di misura $(X,\A,\mu)$, sia $\mathcal{M}$ l'insieme delle funzioni misurabili da $X$ in $\Rbar$.
	Allora $\mathcal{M}$ è un'algebra nel senso che, dove sono definite\footnote{Nel definire le operazioni algebriche su $\mathcal{M}$ adottiamo le seguenti convenzioni: la somma è definita se non accade che entrambe $f$ e $-g$ siano $\pm\infty$, per la moltiplicazione $0\cdot \infty = 0$.} ,
	valgono le seguenti:
	\begin{enumerate}[label=(\arabic*),ref=(\arabic*)]
		\item $f,g\in \mathcal{M} \Rightarrow f+g\in \mathcal{M}$; \label{AlM:sum}
		\item $f\in \mathcal{M}, \lambda \in \R \Rightarrow \lambda f\in \mathcal{M}$; \label{AlM:sca}
		\item $f,g\in \mathcal{M} \Rightarrow fg\in \mathcal{M}$. \label{AlM:pro}
	\end{enumerate}
\end{proposition}

\begin{proof}
	Mostriamo per ogni punto che vale la proposizione \ref{BM:sov} nella \cref{prop:BasicMis} (che come lì mostrato, equivale alla misurabilità),
	distinguendo vari casi di $a\in \Rbar$.
	\begin{description}
	\item[\ref{AlM:sum}]
	\[
		\{f+g>a\}=\left\{\begin{array}{ll}
			\{f\ge -\infty\}\cap\{g\ge -\infty\}\in \A &\qquad \text{se $a=-\infty$}\puntovirgola\\
			\bigcup_{q\in \Q}\left(\{f>q\}\cap\{g>a-q\}\right)\in \A &\qquad \text{se $a\in \R$}\puntovirgola\\
			\{f=+\infty\}\cup\{g=+\infty\}\in \A &\qquad \text{se $a=+\infty$}\punto
		\end{array}\right.
	\]
	\item[\ref{AlM:sca}]
	\[
		\{\lambda f>a\}=\left\{\begin{array}{ll}
			\left\{f<\frac{a}{\lambda}\right\}\in \A &\qquad \text{se $\lambda<0$}\puntovirgola \\
			X \in \A &\qquad \text{se $\lambda=0$ e $a< 0$}\puntovirgola\\
			\emptyset \in \A &\qquad \text{se $\lambda=0$ e $a\ge 0$}\puntovirgola\\
			\left\{f>\frac{a}{\lambda}\right\}\in \A &\qquad \text{se $\lambda>0$}\punto
		\end{array}\right.
	\]
	\item[\ref{AlM:pro}] Scomponiamo $f=f^+ - f^-$, $g=g^+- g^-$, quindi la funzione prodotto $fg$ si scrive come una qualche combinazione di prodotti di funzioni misurabili non negative (abbiamo già dimostrato nella \cref{nota:ProprietaFpiuFmeno} le proprietà di $f^+,f^-,g^+,g^-$ che ci servono). Grazie ai punti \ref{AlM:sum} e \ref{AlM:sca} e a questa osservazione ci basta mostrare il caso in cui $f,g\ge0$:
	\[
		\{fg>a\}=\left\{\begin{array}{ll}
			X\in \A &\enspace \text{se $a<0$}\puntovirgola\\
			\{f>0\}\cup\{g>0\}\in \A &\enspace \text{se $a=0$}\puntovirgola\\
			\bigcup_{q\in \Q^+}\left(\{q<f<+\infty\}\cap\left\{\frac{a}{q}<g<+\infty \right\} \right)\in \A &\enspace \text{se $0<a<+\infty$}\puntovirgola\\
			(\{f=+\infty\}\cap\{g>0\})\cup (\{f>0\}\cap\{g=+\infty\})\in \A &\enspace \text{se $a=+\infty$}\punto
		\end{array}\right.
	\]
	\end{description}
\end{proof}

\begin{remark}\label{nota:CarMis}
	È facile vedere che le funzioni caratteristiche degli insiemi misurabili sono misurabili.
\end{remark}
\begin{proof}
	Basta osservare che se $A\in \A$ è misurabile, $\{ \chi_A > a\}$ può essere solo $\emptyset$, $A$, $X$ (tutti e tre misurabili) a seconda che
	$a\ge 1$, $0\le a< 1$ oppure $a < 0$ rispettivamente.
\end{proof}

\begin{remark}\label{nota:ContinueMisurabili}
	Sia $X$ un insieme dotato sia di una topologia che di una misura su di esso, tali che in particolare la \sigalg{} dei misurabili contenga tutti gli aperti.
	Data una funzione $f:X\to\R$, se $f$ è continua è anche misurabile.
\end{remark}
\begin{proof}
	Basta notare che la controimmagine di un aperto è un aperto per continuità, ma gli aperti sono misurabili per ipotesi e di conseguenza la funzione è misurabile.
\end{proof}

\begin{remark}\label{nota:MonotoneMisurabili}
	Fissato $A\subseteq \R$ misurabile, ogni funzione $f:A\to\R$ monotona è misurabile, munendo $\R$ della misura di Lebesgue definita nella precedente sezione.
\end{remark}
\begin{proof}
	È sufficiente notare che la controimmagine di un intervallo\footnote{Definiamo, unicamente in questa dimostrazione, un intervallo come un generico sottoinsieme connesso di $\R$.} è a sua volta un intervallo intersecato $A$ poiché la funzione è monotona, perciò applicando la \cref{prop:BasicMis} ricaviamo che la funzione è misurabile visto che gli intervalli sono misurabili secondo Lebesgue (e lo è la loro interesezione con $A$, per la \cref{nota:RiduzioneMisura}).
\end{proof}

\begin{proposition}\label{prop:BorelianiNonMisurabili2}
	I Boreliani di $\R$ non coincidono con l'insieme $\M_1$ dei misurabili.
\end{proposition}
\begin{proof}
	Definiamo la funzione $f:\co{0}{1}\to \co{0}{1}$ in modo che $f(x)$ sia il numero che corrisponde alla lettura in base $3$ della scrittura in base $2$ di $x$.
	Poiché alcuni numeri hanno due scritture in base $2$, sceglieremo sempre quella che non ha una coda infinita di $1$.
	
	Qui di seguito un diagramma che mostra la definizione di $f$:
	\begin{equation*}
		x=\>\stackrel{\text{Scrittura in base $2$ di $x$}}{\overline{0.x_1x_2x_3\cdots}_2} \>  \longmapsto
		\> \stackrel{\text{Lettura in base $3$ di $x$ in base $2$}}{\overline{0.x_1x_2x_3\cdots}_3}\>=f(x)
	\end{equation*}

	La funzione $f$ appena definita è strettamente crescente, poiché lo è la funzione che associa ad un numero $x$ la sua lettura in qualche base (dove le sequenze di cifre sono ordinate lessicograficamente). Allora per la \cref{nota:MonotoneMisurabili} $f$ è misurabile.
	
	Inoltre l'immagine di $f$ è un insieme trascurabile, infatti questa coincide con l'insieme dei numeri tra $0$ e $1$ che si scrivono unicamente usando cifre $0,1$ in base $3$ ed è facile dimostrare che questo è trascurabile (esercizio per il lettore molto simile all'\cref{ex:CantorTrascurabile}).
	
	Per il \cref{thm:InsiemeVitali} esiste $A\subseteq \co{0}{1}$ che non è misurabile.
	Sia $B=f(A)$.
	
	Poiché $f$ è strettamente crescente è in particolare iniettiva e perciò $A=f^{-1}(B)$.
	Allora la \cref{prop:CounterImgMis} ci assicura che $B$ non appartiene ai Boreliani, altrimenti la sua controimmagine sarebbe misurabile. 
	Infine $B$ è sottoinsieme di $\co01$, che è trascurabile, perciò per la completezza della misura di Lebesgue $B$ è misurabile.
	
	Allora, unendo quanto detto, abbiamo che $B$ è un misurabile non Boreliano come voluto.
\end{proof}



\begin{definition}
	Una funzione $\simp:X \rightarrow \Rbar$ con dominio lo spazio di misura $(X,\A,\mu)$ si dice semplice se è combinazione lineare di
	funzioni caratteristiche di insiemi misurabili.
\end{definition}
\begin{remark}
	È immediato che le funzioni semplici sono misurabili.
\end{remark}
\begin{proof}
	Discende dalla \cref{nota:CarMis} e dalla \cref{prop:AlgMis}.
\end{proof}


\begin{proposition}\label{prop:SupDiMisurabili}
	Sia $\{f_n\}_{n\in \N}$ una famiglia di funzioni misurabili definite dallo spazio di misura $(X,\A,\mu)$ a $\Rbar$.
	Allora $F:X\rightarrow \Rbar$ definita da $F(x)=\sup\{f_n(x):n\in \N\}$ è misurabile.
\end{proposition}
\begin{proof}
	Consideriamo il sovralivello della funzione $F$: $\{F>a\}=\bigcup_{n\in \N}\{f_n>a\}$, allora è evidente che $\{F>a\}\in \A$ per le proprietà 
	di chiusura della \sigalg.
\end{proof}

\begin{remark}\label{nota:LimMis}
	Quest'ultima proposizione ha alcune notevoli conseguenze immediate:
	\begin{enumerate}
		\item $\inf$ di una famiglia numerabile di misurabili è misurabile;\label{LM:inf}
		\item $\limsup$ e $\liminf$ di una famiglia numerabile di misurabili sono misurabili;\label{LM:lim_infsup}
		\item limite puntuale di funzioni misurabili è misurabile.\label{LM:lim}
	\end{enumerate}
\end{remark}
\begin{proof}
	\begin{description}
		\item[\ref{LM:inf}] Per l'$\inf$ basta notare che $\inf\{f_n\}=-\sup\{-f_n\}$, quindi è misurabile per la \cref{prop:SupDiMisurabili};
		\item[\ref{LM:lim_infsup}] per definizione, $\limsup\{f_n\}$ e $\liminf\{f_n\}$ sono rispettivamente
			$\lim_n\{\sup\{f_n\}\}=\inf\{\sup\{f_n\}\}$ e
			$\lim_n\{\inf\{f_n\}\}=\sup\{\inf\{f_n\}\}$, quindi sono funzioni misurabili per il punto precedente;
		\item[\ref{LM:lim}] infine se esiste il limite $\lim_n\{f_n\}$ allora
			$\liminf\{f_n\}=\limsup\{f_n\}=\lim_n\{f_n\}$, pertanto è misurabile per il punto precedente.
	\end{description}
\end{proof}

\begin{proposition}\label{prop:LimSemMis}
	Sia $f:X \rightarrow \Rbar$ una funzione con dominio lo spazio di misura $(X,\A,\mu)$.
	Allora $f$ è misurabile se e solo se esiste una successione di funzioni semplici $\simp_n$ che converge puntualmente a $f$.
\end{proposition}
\begin{proof}
	Il se è mostrato nella \cref{nota:LimMis}.
	
	Per il solo se facciamo vedere che la seguente successione converge puntualmente a $f$:
	\[
		\simp_n(x) =
		\left\{ \begin{array}{ll}
			n &\qquad se\ f(x)>n;\\
			\frac{k}{2^n} &\qquad se\ \frac{k}{2^n}<f(x)\le \frac{k+1}{2^n} \qquad k=-n2^n,-n2^n+1,\dots,n2^n;\\
			-n &\qquad se\ f(x)\le-n;
		\end{array} \right.\ .
	\]
	Prima di tutto abbiamo 
	\[\simp_n=
		n\chi_{\{f>n\}}+
		\sum_{k=-n2^n}^{n2^n}\frac{k}{2^n}\chi_{\left\{ \frac{k}{2^n}<f\le \frac{k+1}{2^n} \right\}}
		-n\chi_{\{f<-n\}},
	\]
	che mostra che le $\simp_n$ sono funzioni semplici.
	
	Per mostrare la convergenza puntuale distinguiamo $f(x)$ a seconda che sia un numero finito o meno:
	nel primo caso abbiamo che $|f(x)-\simp_n(x)|\le \frac{1}{2^n}$ definitivamente, cioè $\forall n\ge |f(x)|$,
	nel secondo caso $\simp_n(x)=\pm n\rightarrow \pm\infty = f(x)$;
	quindi $\simp_n(x)\rightarrow f(x)$, $\forall x\in X$.
\end{proof}

\begin{definition}
	Una funzione $f$ definita su $(X,\A,\mu)$ e a valori in $\Rbar$ si dice positiva se assume solo valori maggiori o uguali a 0.
\end{definition}


\begin{corollary}\label{cor:LimSemCrescMis}
	Sia $f:X\rightarrow \Rbar$ misurabile e positiva su $(X,\A,\mu)$ spazio di misura, allora esiste una successione crescente di funzioni semplici e positive $(\simp_n)$ che converge puntualmente a $f$.
\end{corollary}
\begin{proof}
	Costruendo le $\simp_n$ come nella \cref{prop:LimSemMis}, se $f$ è positiva otteniamo facilmente che le $\simp_n$ sono anche crescenti, da cui la tesi.
\end{proof}

\begin{theorem}\label{thm:ChiusuraMonotonaFunzioni}
	Se $\mathcal F$ è una famiglia di funzioni da $X$ a $\Rbar$, dove $(X,\A,\mu)$ è uno spazio di misura, tale che
	\begin{itemize}
	 \item $\mathcal F$ è uno spazio vettoriale;
	 \item $\mathcal F$ contiene le funzioni indicatrivi di ogni insieme $A\in\A$;
	 \item se $(f_n)\subseteq \mathcal F$ è una successione di funzioni misurabili positive che converge crescentemente a $f$, allora $f\in \mathcal F$;
	\end{itemize}
	allora $\mathcal F$ contiene tutte le funzioni misurabili.
\end{theorem}
\begin{proof}
	Per prima cosa notiamo che $\mathcal F$ contiene le funzioni semplici: essendo $\mathcal F$ uno spazio vettoriale, contiene le combinazioni
	lineari delle funzioni caratteristiche, cioè le funzioni semplici.
	
	Notiamo allora che per il \cref{cor:LimSemCrescMis}, $\mathcal F$ contiene le funzioni misurabili positive. Allora, data $f$ misurabile,
	$f = f_+-f_-$ quindi è contenuta in $\mathcal F$, ancora per le proprietà di spazio vettoriale, poiché entrambe $f_+,f_-$ sono
	positive e misurabili per la \cref{nota:ProprietaFpiuFmeno}.
\end{proof}


\end{document}

\makeindex