\documentclass[a4paper,12pt]{article}
\usepackage{stilebase}
% \usepackage{float}
% \usepackage{figure}

\title{Appunti di teoria della misura elementare}
\author{Federico Glaudo e Marco Trevisiol}

\begin{document}

\maketitle
% \clearpage


\begin{abstract}
	Trattiamo in queste brevissime dispense i fatti fondamentali (e fondazionali) di teoria della misura elementare.
	
	In particolare, dopo una prima sezione prevalentemente di definizioni e fatti preparatori, dimostreremo il teorema di estensione di Caratheodory, che permette di costruire misure a partire da strutture ben più semplici come le misure elementari.
	
	Il percorso seguito ricalca quello proposto dal professor Majer durante il corso di Analisi II dell'anno 2013-2014 alla facoltà di matematica a Pisa.
\end{abstract}
\clearpage

% \tableofcontents
% \clearpage

\section{Definizioni e risultati introduttivi}
In questa sezione definiremo la distanza di Hausdorff fra i chiusi e limitati e mostreremo qualche risultato introduttivo.

In tutto il corso della trattazione indicheremo con $(X,d)$ uno spazio metrico generico.

\begin{definition}
	Sia $\mathcal{K}(X)=\{K\in \mathcal{P}(X) : K \text{ è compatto}\}$ l'insieme dei compatti di $X$.
\end{definition}

\begin{definition}
	Dato $A\in \mathcal{K}(X)$, definiamo $d_A: X\to [0,+\infty)$ tale che $d_A(x)=\inf_{a\in A}d(a,x)$ per ogni $x\in X$.
\end{definition}

\begin{remark}
	$d_A({}\cdot{})$ è effettivamente una funzione a valori in $[0,+\infty)$ perchè vale $\inf_{a\in A}d(a,x)<\infty$, poichè $A$ essendo compatto è limitato.
\end{remark}

\begin{lemma}\label{DistanzaCompattoRealizzata}
	Per ogni $x\in X$ e $A \in \mathcal{K}(X)$, esiste $a\in A$ tale che $d_A(x)=d(a,x)$.
\end{lemma}
\begin{proof}
	Poichè $d_A(x)=\inf_{a\in A} d(a,x)$, esiste $(a_n)$ a valori in $A$ tale che $\lim_{n\to\infty}d(a_n,x)=d_A(x)$. Dato che $A$ è un compatto esiste una sottosuccessione $(a_{n_k})$ di $(a_n)$ convergente ad $a\in A$. Tale sottosuccessione rispetta quindi che $a_{n_k}\to a$ e $d(a_{n_k},x)\to d_A(x)$, da cui facilmente $d(a,x)=d_A(x)$, poichè $d(a_{n_k},x)\to d(a,x)$.
\end{proof}
\begin{remark}\label{DistanzaCompattoAppartenenza}
	Dati $x\in X$ e $A \in \mathcal{K}(X)$, $d_A(x)=0$ se e solo se $x\in A$. Infatti per \cref{DistanzaCompattoRealizzata} esiste $a\in A$ tale che $d_A(x)=d(a,x)$, da cui $d_A(x)=0$ se e solo se $a=x$.
\end{remark}

\begin{definition}\label{DistanzaFraCompatti}
	Sia $\delta_A:\mathcal{K}(X)\to [0,+\infty)$ tale che $\delta_A(B)=\sup_{b\in B} d_A(b)$.
\end{definition}
\begin{remark}
	Anche in questo caso $\delta_A(B)$ ha valori in $[0,+\infty)$ perchè $A\cup B$ è limitato.
\end{remark}

\begin{definition}[Distanza di Hausdorff]\label{HausdorffDefinizione}
	Definiamo infine $\delta:\mathcal{K}(X)\times \mathcal{K}(X) \to [0,+\infty)$ come $\delta(A,B)=\max\{ \delta_A(B),\delta_B(A) \}$.
\end{definition}

\begin{theorem}
	La funzione definita in \cref{HausdorffDefinizione} è una distanza chiamata distanza di Hausdorff.
\end{theorem}

\begin{proof}
	Dimostriamo che $\delta({}\cdot{},{}\cdot{})$ è veramente una distanza.
	\begin{itemize}
		\item $\delta(A,B)=\delta(B,A)$ (simmetria)
		
		$\delta(A,B)=\max\{ \delta_A(B),\delta_B(A) \}=\delta(B,A)$.
		\item $\delta(A,B)=0 \iff A=B$
		
		Se $A=B$ vale banalmente $\delta(A,B)=0$; se invece esiste $a\in A$ tale che $a\not\in B$ ho che $\delta(A,B)\ge \delta_B(A)\ge \delta_B(a)>0$, dove l'ultima disuguaglianza è vera per \cref{DistanzaCompattoAppartenenza}.
		\item $\delta(A,C)\le \delta(A,B)+\delta(B,C)$ (disuguaglianza triangolare)
		
		Dimostro innanzitutto che per ogni $c\in C$ vale $d_A(c)\le \delta(A,B)+\delta(B,C)$.
		
 		Per \cref{DistanzaCompattoRealizzata} ho che esiste $a\in A$ tale che $d_A(c)=d(a,c)$ e che esiste $b\in B$ tale che $d_B(c)=d(b,c)$. Analogamente esiste $a'\in A$ tale che $d_A(b)=d(a',b)$. Utilizzando facili conseguenze delle definizioni date precedentemente, ho quindi che 
 		\begin{equation*}
 			d_A(c)=d(a,c)\le d(a',c) \le d(a',b)+d(b,c)=d_A(b)+d_B(c)\le \delta(A,B)+\delta(B,C)
 		\end{equation*}
 		
 		Passando ora al $\sup$ su $c\in C$ in quest'ultima disuguaglianza ottengo che $\delta_A(C)\le \delta(A,B)+\delta(B,C)$, ma del tutto  analogamente vale $\delta_C(A)\le \delta(A,B)+\delta(B,C)$, quindi
 		\begin{equation*}
 			\delta(A,C)=\max\{ \delta_A(B),\delta_B(A) \}\le \delta(A,B)+\delta(B,C)
 		\end{equation*}
		che è proprio la disuguaglianza triangolare.
	\end{itemize}
	Inoltre $\delta$ ha valori in $[0,+\infty)$ poichè per ogni $A,B\in \mathcal{K}(X)$ $\delta(A,B)<\infty$, dato che $A$ e $B$ sono limitati.
\end{proof}

\begin{remark}
	Del tutto analogamente si dimostra che la distanza di Hausdorff è una distanza anche sui chiusi e limitati, poichè anche in questo caso vale \cref{DistanzaCompattoAppartenenza}.
\end{remark}

\begin{definition}[Definizione equivalente della distanza di Hausdorff] \label{HausdorffDefinizioneEquivalente}
	La distanza di Hausdorff si può definire in modo equivalente nel seguente modo. Sia $\delta_A'(B)=\inf\{r:B\subseteq U_r(A)\}$, dove $U_r(A)=\{x\in X : d_A(x)\le r \}$. Allora definisco $\delta'(A,B)=\max\{\delta_A'(B),\delta_B'(A)\}$
\end{definition}

\begin{theorem}
	Le definizioni \cref{HausdorffDefinizione} e \cref{HausdorffDefinizioneEquivalente} sono equivalenti.
\end{theorem}
\begin{proof}
	Dimostro in particolare che $\delta_A(B)=\delta_A'(B)$, dove $\delta_A$ e $\delta_A'$ sono definite rispettivamente in \cref{DistanzaFraCompatti} e \cref{HausdorffDefinizioneEquivalente}, poichè da questo segue banalmente la tesi.
	
	Chiamo $P=\{ r : \exists b\in B \text{ tale che }d_A(b)=r \}$ e $Q=\{ r: \forall b\in B\text{ vale } d_A(b)\le r \}$, allora si nota facilmente che per ogni $p\in P$ e $q\in Q$ vale $p\le q$ e inoltre per ogni $x\in X$ ho che $x\in P$ o $x\in Q$. Da questo segue facilmente che $\sup\{ r: r\in P \}=\inf\{r: r\in Q \}$.
	
	Vale quindi che
	\begin{equation*}
		\delta_A(B)=\sup_{b\in B} d_A(b)=\sup\{ r : r\in P \}=\inf\{ r : r\in Q \}= \inf\{ r : B\subseteq U_r(A) \}=\delta_A'(B)
	\end{equation*}
	che è quello che volevo dimostrare
\end{proof}

\begin{lemma}\label{IsometriaCanonica}
	Esiste un'isometria canonica $\varphi: X\in \mathcal{K}(X)$, tale che $\varphi(X)$ è un chiuso in $\mathcal{K}(X)$.
\end{lemma}
\begin{proof}
	Definisco l'isometria $\varphi: X\in\mathcal{K}(X)$ tale che $\varphi(x)=\{x\}$ per ogni $x\in X$. 
	
	Innanzitutto vale banalmente che questa è un'isometria, perchè segue facilmente dalle definizioni che $\delta(\{ x \}, \{ x' \})=d(x,x')$.
	
	Dimostriamo ora che $\varphi(X)$ è un chiuso. Sia $(\{ x_n \})_{n\in \mathbb{N}}$ una successione convergente in $\mathcal{K}(X)$, allora $(x_n)_{n\in \mathbb{N}}$ converge in $X$ ad un valore $x$ (poichè $\varphi$ è un isometria). Voglio mostrare che allora $(\{ x_n \})_{n\in \mathbb{N}}$ converge a $\{ x \}$, ma questo è ovvio perchè $\lim_{n\to\infty} \delta(\{ x_n \}, \{x\})=\lim_{n\to\infty} d(x,x')=0$. Questo conclude la dimostrazione per l'unicità del limite.
\end{proof}










\section{Estendere una premisura ad una misura}
L'obiettivo ora è riuscire ad estendere una premisura definita su un semianello ad una misura su una \sigalg{}. Per fare questo il percorso sarà prima quello di estendere la premisura ad una misura esterna, per poi ridurre questa ad una misura canonica.


\begin{theorem}\label{RiduzionePreCaratheodory}
	Data $\mu:\mathcal P(X)\to \Rpiu$ una misura esterna, sia $\A\subseteq \mathcal P(X)$ l'insieme così definito:
	\begin{equation*}
		\A=\{E\in\mathcal P(X):\ \mu(A)=\mu(A\cap E)+\mu(A\setminus E)\ \forall A\in \mathcal P(X)\}
	\end{equation*}
	allora $\A$ è una \sigalg{}, detta \sigalg{} di Caratheodory, e $\mu$ ridotta su $\A$ è una misura completa.
\end{theorem}
\begin{proof}
	La dimostrazione procede in tre passi: prima mostriamo che $\A$ è un'algebra di insiemi, poi che è è una \sigalg{} e infine che $\mu$ è \sigadd{} e completa ridotta su $\A$.
	
	Il fatto che $\A$ sia stabile per complementare è ovvio per la definizione (che è simmetrica tra $E$ ed $E^c$).
	
	Fissati $A\in\mathcal P(X)$ generico ed $E,F\in\A$, applicando la sola definizione di $\A$ ed alcuni passaggi insiemistici si ricava:
	\begin{align*}
		\mu(A)\stackrel{F\in\A}{=}&\mu(A\cap F)+\mu(A\setminus F)\stackrel{E\in\A}{=}
		\mu(A\cap F)+\mu\left((A\setminus F)\cap E\right)+\mu\left((A\setminus F)\setminus E\right)\\
		=\hspace{0.4em}&\mu\left((A\cap (E\cup F))\cap F\right)+\mu\left((A\cap (E\cup F))\setminus F\right)+
		\mu\left(A\setminus(E\cup F)\right)\\
		\stackrel{F\in\A}{=}&\mu(A\cap (E\cup F))+\mu\left(A\setminus(E\cup F)\right)
	\end{align*}
	e visto che questo vale per ogni scelta di $A\in\mathcal P(X)$ abbiamo dimostrato che $\A$ è stabile per unione.
	
	Unendo quanto detto si ha facilmente che $\A$ è un'algebra di insiemi.
	
	Ora sia $(E_n)_{n\in\N}\subseteq \A$ una famiglia numerabile di insiemi ed $A\in\mathcal P(X)$ un generico sottoinsieme di $X$.
	
	Per la \sigsubadd[ità] di $\mu$ vale:
	\begin{equation}\label{DisuguaglianzaFacileCaratheodory}
		\mu(A)\le \mu\left(A\cap\bigcup_{n\in\N} E_n\right)+\mu\left(A\setminus\cap\bigcup_{n\in\N} E_n\right)
	\end{equation}
	Si vuole dimostrare che il $\le$ è in realtà un'uguaglianza. Se $\mu(A)=+\infty$ questo è ovvio, quindi tratteremo il caso in cui $\mu(A)<+\infty$. Chiamiamo $F_n=E_n\setminus \bigcup_{i<n} E_i$, ottenendo in maniera ovvia che gli $(F_n)_{n\in\N}$ sono a due a due disgiunti e che appartengono a $\A$ poiché quest'ultima è un'algebra.
	
	Per induzione è facile verificare, sfruttando unicamente il fatto che $F_n\in\A$ e $\mu(A)<+\infty$, che risulta:
	\begin{equation}\label{IdentitaDifferenzaCaratheodory}
		\mu\left(A\setminus \bigsqcup_{n\le m} F_n\right)=\mu(A)-\sum_{n\le m} \mu(A\cap F_n)
	\end{equation}
	e incidentalmente da questa formula si ha che la serie $\sum_{n\in\N}\mu(A\cap F_n)$ converge, visto che è a termini positivi e limitata (da $\mu(A)$).
	
	Per la \sigsubadd[ità] di $\mu$ vale:
	\begin{equation}\label{IntersezioneStimaCaratheodory}
		\mu\left(A\cap\bigcup_{n\in\N} E_n\right)=\mu\left(\bigsqcup_{n\in\N} A\cap F_n\right)\le
		\sum_{n\in\N} \mu(A\cap F_n)
	\end{equation}
	mentre, grazie alla monotonia e a \cref{IdentitaDifferenzaCaratheodory} otteniamo:
	\begin{equation}\label{DifferenzaStimaCaratheodory}
		\mu\left(A\setminus\bigcup_{n\in\N} E_n\right) = \mu\left(A\setminus\bigsqcup_{n\in\N} F_n\right) \le \mu\left(A\setminus\bigsqcup_{n\le m} F_n\right) = 
		\mu(A)-\sum_{n\le m}\mu(A\cap F_n)
	\end{equation}
	
	Ora unendo \cref{IntersezioneStimaCaratheodory,DifferenzaStimaCaratheodory} giungiamo ad avere che, per ogni $m\in\mathbb{N}$:
	\begin{equation*}
		\mu\left(A\cap\bigcup_{n\in\N} E_n\right)+\mu\left(A\setminus\bigcup_{n\in\N} E_n\right)\le
		\mu(A)+\sum_{m\le n}\mu(A\cap F_n) 
	\end{equation*}
	ma per la convergenza di $\sum_{n\in \N}\mu(A\cap F_n)$, estraendo l'$\inf$ da entrambe le parti finalmente arriviamo a:
	\begin{equation*}
		\mu\left(A\cap\bigcup_{n\in\N} E_n\right)+\mu\left(A\setminus\bigcup_{n\in\N} E_n\right)\le
		\mu(A)
	\end{equation*}
	che unita a \cref{DisuguaglianzaFacileCaratheodory} ci assicura che vale l'identità tra i membri e, visto che ciò vale indipendentemente dalla scelta di $A\in\mathcal P(X)$, risulta $\bigcup_{n\in\N}E_n\in\A$ che equivale a dire che $\A$ è una \sigalg{}.
	
	Dimostrare che $\mu$ è \sigadd{} su $\A$ è ora molto facile.
	Consideriamo $(E_n)_{n\in\N}\subset \A$ una famiglia numerabile di insiemi \emph{disgiunti}. Per facile induzione si ha che:
	\begin{equation*}
		\mu\left(\bigsqcup_{n\le m}E_n\right)=\sum_{n\le m} \mu(E_n)
	\end{equation*}
	e applicando questa e la monotonia di $\mu$ risulta:
	\begin{equation*}
		\sum_{n\le m} \mu(E_n)=\mu\left(\bigsqcup_{n\le m}E_n\right)\le
		\mu\left(\bigsqcup_{n\in\N}E_n\right)\le \sum_{n\in\N} \mu(E_n)
	\end{equation*}
	e questa doppia disuguaglianza, per la definizione delle serie a termini positivi, implica che tutte le disuguaglianze sono identità. Ma allora questo dimostra che $\mu$ è \sigadd{} su $\A$.
	
	Infine per dimostrare la completezza di $\mu|_{\A}$ basta mostrare che dato $E\in\mathcal P(X)$ trascurabile, vale $E\in\A$ (questo è sufficiente a mostrare la completezza, visto che per monotonia i sottinsiemi di un trascurabile sono a loro volta trascurabili).
	
	Fissato un generico $A\in\mathcal P(X)$, risulta per la monotonia di $\mu$:
	\begin{equation*}
		\mu(A\cap N)+\mu(A\setminus N)\le \mu(N)+\mu(A)=\mu(A)
	\end{equation*}
	che, unita alla \sigsubadd[ità] di $\mu$ mi assicura
	\begin{equation*}
		\mu(A)=\mu(A\cap N)+\mu(A\setminus N)
	\end{equation*}
	che è proprio la condizione di appartenenza ad $\A$.
\end{proof}

\begin{proposition}\label{MisuraEsternaDiPremisura}
	Dato $(X,\mathcal S,\mu)$ uno spazio di misura elementare si consideri la funzione che associa ad ogni sottoinsieme l'estremo inferiore delle misure dei ricoprimenti, cioè $\mu^*:\mathcal P(X)\to\Rpiu$ definita come 
	\begin{equation*}
		\mu^*(A)=\inf\left\{\sum_{n\in\N} A_n\ |\ (A_n)_{n\in\N}\subseteq\mathcal S\ \wedge
		\ A\subseteq\bigcup_{n\in\N}A_n\right\}
	\end{equation*}
	Allora $\mu^*$ è una misura esterna che estende $\mu$ (cioè $\mu^*|_{\mathcal S}=\mu$) ed inoltre $\mathcal S$ appartiene alla relativa \sigalg{} di Caratheodory (come definita in \cref{RiduzionePreCaratheodory}).
\end{proposition}
\begin{proof}
	Per ottenere che $\mu^*$ è una misura esterna basta verificare le proprietà che deve rispettare.
	Ovviamente, poiché $\mu(\emptyset)=0$, vale $\mu^*(\emptyset)=0$. 
	Inoltre, ancora facilmente, $\mu^*$ è monotona, visto che se $A\subseteq B$ un ricoprimento di $B$ ricopre anche $A$.
	E infine è anche \sigsubadd{} visto che l'unione di ricoprimenti (che risulta ancora un ricoprimento numerabile) è un ricoprimento dell'unione.
	
	Dato $S\in\mathcal S$ vale ovviamente $\mu*(S)\le\mu(S)$, poiché $S$ si ricopre da solo. Per dimostrare la disuguaglianza opposta consideriamo $(S_n)_{n\in\N}\in \mathcal S$ un ricoprimento di $S$
	
	Ora perciò resta da dimostrare che se $E\subseteq \mathcal S$ allora per ogni $A\in\mathcal P(X)$ risulta:
	\begin{equation}\label{MisuraEsternaDisDifficile}
		\mu^*(A) \ge \mu^*(A\cap E)+\mu^*(A\setminus E)
	\end{equation}
	Questo è sufficiente ad avere che $\mathcal S$ è contenuto nella \sigalg{} di Caratheodory poiché l'altra disuguaglianza è assicurata dalla \sigsubadd[ità].
	
	Dato $(A_n)_{n\in\N}\subseteq\mathcal S$ un ricoprimento di $A$, chiamiamo $B_n=A_n\cap E$ e $C_n=A_n\setminus E$. Ovviamente $(B_n)_{n\in\N},(C_n)_{n\in\N}$ ricoprono rispettivamente $A\cap E,A\setminus E$. Poiché $\mathcal S$ è un \semiring{} riusciamo però a trovare $(B'^n_i)_{i\in\N},(C'^n_i)_{i\in\N} \subseteq \mathcal S$ tali che $B_n=\bigsqcup_{i\in\N}B'^n_i$ e analogo risultato per $C_n$. Quindi $(B'^n_i)_{n,i\in\N}, (C'^n_i)_{n,i\in\N}$ risultano ricoprimenti con elementi di $\mathcal S$ di $A\cap E,A\setminus E$ rispettivamente.
	Ora, sfruttando non più della sola \sigadd[ità] di $\mu$ concludo:
	\begin{align*}
		\sum_{n\in\N}\mu(A_n)=\sum_{n\in\N} \mu(B_n)+\mu(C_n)&=
		\sum_{n\in\N}\sum_{i\in\N}\mu(B'^n_i)+\mu(C'^n_i)\\
		&=
		\sum_{n,i\in\N}\mu(B'^n_i)+\sum_{n,i\in\N}\mu(C'^n_i)\ge \mu(A\cap E)+\mu(A\setminus E)
	\end{align*}
	ma questo implica facilmente \cref{MisuraEsternaDisDifficile} estraendo l'estremo inferiore a entrambi i membri sui ricoprimenti di $A$.
\end{proof}

\begin{theorem}[Estensione di Caratheodory]\label{EstensionexCaratheodory}
	Dato $(X,\mathcal S,\mu)$ uno spazio di misura elementare esiste una \sigalg{} $\A$ e una funzione $\mu':\A\to\Rpiu$ tali che $\mathcal S\subseteq \A$, $\mu'$ estende la premisura $\mu$ e $(X,\A,\mu')$ è uno spazio di misura completo.
\end{theorem}
\begin{proof}
	Consideriamo la misura esterna $\mu^*:\mathcal P(X)\to\Rpiu$ definita nell'enunciato di \cref{MisuraEsternaDiPremisura}. Sempre \cref{MisuraEsternaDiPremisura} ci assicura che questa è un'estensione di $\mu$.
	
	Possiamo ora ridurre $\mu^*$ grazie al \cref{RiduzionePreCaratheodory} ad una misura completa $\mu':\A\to\Rpiu$ dove $\A$ è la \sigalg{} di Caratheodory. 
	
	Ma come dimostrato in \cref{MisuraEsternaDiPremisura} $\mathcal S\subseteq\A$ e inoltre vale $\mu'|_{\mathcal S}=\mu^*|_{\mathcal S}=\mu$, perciò lo spazio $(X,\A,\mu')$ rispetta tutte le richieste dell'enunciato.
\end{proof}

\section{Misura di Lebesgue}
Ora applicheremo i risultati astratti ottenuti nelle due precedenti sezioni al caso più tangibile della retta reale.

Definiremo la misura di Lebesgue e, oltre a chiarire come mai questa sia la misura più naturale su $\R$, studieremo i misurabili secondo Lebesgue mostrando sia che non coincidono con la \sigalg{} dei boreliani (cioè la \sigalg{} generata dagli aperti) sia che non coincidono con le parti di $\R$.

\begin{definition}
	I Boreliani sono la \sigalg{} generata dai sottoinsiemi aperti della retta reale.
\end{definition}

\begin{theorem}
	Sia $\S$ l'insieme degli intervalli semiaperti a destra di $\R$, cioè i sottoinsiemi della retta reale della forma $[a,b)$ con $a<b$. 
	Definiamo inoltre la funzione $\mu:\S\to\Rbar$ in modo che $f\left([a,b)\right)=b-a$.
	
	Allora $(\R,\S,m)$ è uno spazio di misura elementare.
\end{theorem}
\begin{proof}
	TODO
\end{proof}
\section{Funzioni misurabili}
In questa sezione daremo la definizione di funzione misurabile e dimostreremo alcuni fatti basilari su di esse.

La teoria delle funzioni misurabili, oltre ad essere strettamente necessaria per la successiva teoria dell'integrazione, ci permetterà di dimostrare che i misurabili secondo Lebesgue, introdotti nella sezione precedente, non coincidono con i Boreliani usando strumenti propri della teoria della misura.
Questo fatto lo dimostriamo però solo nel caso unidimensionale sia perché il risultato è già stato dimostrato, sia perché la dimostrazione si adatta facilmente in dimensione maggiore e sia perché troviamo importante rendere chiara l'idea piuttosto che celarla in una notazione troppo pesante.

\begin{definition}[Funzione misurabile]
	Dato uno spazio di misura $(X,\A,\mu)$, una funzione $f:X\rightarrow \Rbar$ si dice misurabile se
	$\forall A \subseteq \Rbar$ aperto si ha $f^{-1}(A)\in \A$.
\end{definition}

\begin{proposition}\label{prop:BasicMis}
	Dato uno spazio di misura $(X,\A,\mu)$, sia $f:X\rightarrow \Rbar$ una funzione, sono equivalenti i seguenti fatti
	\footnote{qui introduciamo la notazione per i \textit{sovralivelli} di una funzione che useremo in tutti gli appunti:
		data una funzione $f$ di codominio reale e un certo reale $k$ indichiamo con $\{f>k\}$ l'insieme
		$\{x:f(x)>k\}=f^{-1}((k,+\infty])$; stessa notazione verrà usata anche per il sottolivello}:
	\begin{enumerate}[label=(\arabic*),ref=(\arabic*)]
		\item $f$ è misurabile; \label{BM:mis}
		\item $\{f<a\}\in \A \quad \forall a\in \Rbar$; \label{BM:sot}
		\item $\{f\leq a\}\in \A \quad \forall a\in \Rbar$; \label{BM:soteq}
		\item $\{f>a\}\in \A \quad \forall a\in \Rbar$; \label{BM:sov}
		\item $\{f\geq a\}\in \A \quad \forall a\in \Rbar$;  \label{BM:soveq}
		\item $\{a<f<b\}\in \A \quad \forall a,b\in \Rbar$. \label{BM:int}
	\end{enumerate}
\end{proposition}
\begin{proof}
	Sfruttando le proprietà di $\A$ come \sigalg, mostriamo a catena tutte le implicazioni:
	\begin{description}
	\item[\ImplicationProof{BM:mis}{BM:sot}] per definizione di misurabile, $\{f<a\}=f^{-1}\left(\oo{-\infty}{a}\right)\in \A$,
		perché $\oo{-\infty}{a}$ è aperto;
	\item[\ImplicationProof{BM:sot}{BM:soteq}] poiché $\oc{-\infty}{a}=\bigcap_{n\in \N}\oc{-\infty}{a+\frac{1}{n}}$, allora otteniamo
		\begin{equation*}
			\{f\leq a\}=f^{-1}\left(\oc{-\infty}{a}\right)=\bigcap_{n\in \N}f^{-1}\left(\co{-\infty}{a+\frac{1}{n}}\right)\in \A	
		\end{equation*}

	\item[\ImplicationProof{BM:soteq}{BM:sov}] passando al complementare, $\{f>a\}^\mathsf{c}=\{f\leq a\}\in \A$;
	\item[\ImplicationProof{BM:sov}{BM:soveq}] analogamente a \ImplicationProof{BM:sot}{BM:soteq}; 
	\item[\ImplicationProof{BM:soveq}{BM:sot}] analogamente a \ImplicationProof{BM:soteq}{BM:sov};
	\item[$\text{\ref{BM:sot}}\ +\ \text{\ref{BM:sov}}\implies\text{\ref{BM:int}}$] perché
		$\{a<f<b\}=\{a<f\}\cap\{f<b\}\in \A$;
	\item[\ImplicationProof{BM:int}{BM:mis}] perché un aperto $A\subseteq\Rbar$ si può scrivere come
		$A=\bigcup_{n\in \N}A_n$ dove ciascun $A_n$ è un intervallo aperto di $\Rbar$ (o una semiretta aperta);
		quindi $f^{-1}(A)=\bigcup_{n\in \N}f^{-1}(A_n)\in \A$, perché per il punto \ref{BM:int} $f^{-1}(A_n)\in \A\ \ \forall n$.
	\end{description}
\end{proof}

\begin{proposition}\label{prop:CounterImgMis}
	Dato uno spazio di misura $(X,\A,\mu)$, e data $f:X\rightarrow \Rbar$ una funzione misurabile, la famiglia di insiemi
	\[
		\mathcal{E} = \{ E\subseteq \Rbar : f^{-1}(E)\in \A \}
	\]
	è una \sigalg{} ed inoltre contiene i Boreliani.
\end{proposition}
\begin{proof}
	Verifichiamo che $\mathcal E$ è stabile per unioni numerabili e passaggio al complementare.
	
	Fissati $\{E_n\}_{n\in \N}\subseteq \mathcal{E}$ sia $E = \cup_{n\in \N}E_n$, vale:
	\begin{equation*}
		f^{-1}(E)=f^{-1}\left(\cup_{n\in \N}E_n\right) = \cup_{n\in \N}f^{-1}(E_n)\in \A \implies E \in \mathcal{E}
	\end{equation*}
	dove l'appartenenza ad $\A$ si ha per le proprietà di \sigalg{}, e questo dimostra la stabilità per unione numerabile.
	
	Per quanto riguarda il passaggio al complementare, fissato $E\in \mathcal{E}$, risulta:
	\begin{equation*}
		f^{-1}(E^\mathsf{c})= f^{-1}(E)^\mathsf{c} \in \A \implies E^\mathsf{c} \in \mathcal{E}.
	\end{equation*}
	
	Inoltre $\mathcal E$ contiene gli aperti per definizione di funzione misurabile, da cui, per quanto appena dimostrato, contiene la \sigalg{} generata da questi, cioè i Boreliani.
\end{proof}

\begin{definition}\label{def:FpiuFmeno}
	Sia $f:X\to\Rbar$ una funzione misurabile su uno spazio di misura $(X,\A,\mu)$. Definiamo $f^+ = \max\{f,0\}$ e $f^- = \max\{-f,0\}$.
\end{definition}
\begin{remark}\label{nota:ProprietaFpiuFmeno}
	Data una funzione $f$ misurabile, vale $f^+,f^-$ sono misurabile. Inoltre $f^+,f^-\ge 0$ e $f=f^+-f^-$. 
\end{remark}
\begin{proof}
	Abbiamo che per ogni $a\in\Rbar$ vale
	\begin{equation*}
		\{f^+>a\}=\left\{\begin{array}{ll}
			\{f>a\}\in\A &\text{se }a>0\\
			X\in\A &\text{se }a\le 0
	\end{array}\right.
	\end{equation*}
	Quindi, per la \ref{BM:sov} della \cref{prop:BasicMis}, $f^+$ è misurabile. Analogamente si dimostra che $f^-$ è misurabile.
\end{proof}


\begin{proposition}\label{prop:AlgMis}
	Dato uno spazio di misura $(X,\A,\mu)$, sia $\mathcal{M}$ l'insieme delle funzioni misurabili da $X$ in $\Rbar$.
	Allora $\mathcal{M}$ è un'algebra nel senso che, dove sono definite \footnote{Nel definire le operazioni algebriche su $\mathcal{M}$ adottiamo le seguenti convenzioni: la somma è definita
		se non accade che entrambe $f$ e $-g$ siano $\pm\infty$, per la moltiplicazione $0\cdot \infty = 0$.},
	valgono le seguenti:
	\begin{enumerate}[label=(\arabic*),ref=(\arabic*)]
		\item $f,g\in \mathcal{M} \Rightarrow f+g\in \mathcal{M}$; \label{AlM:sum}
		\item $f\in \mathcal{M}, \lambda \in \R \Rightarrow \lambda f\in \mathcal{M}$; \label{AlM:sca}
		\item $f,g\in \mathcal{M} \Rightarrow fg\in \mathcal{M}$. \label{AlM:pro}
	\end{enumerate}
\end{proposition}

\begin{proof}
	Mostriamo per ogni punto che vale la proposizione \ref{BM:sov} nella \cref{prop:BasicMis} (che come lì mostrato, equivale alla misurabilità),
	distinguendo vari casi di $a\in \Rbar$.
	\begin{description}
	\item[\ref{AlM:sum}]
	\[
		\{f+g>a\}=\left\{\begin{array}{ll}
			\{f\ge -\infty\}\cap\{g\ge -\infty\}\in \A &\qquad \text{se}\ a=-\infty;\\
			\bigcup_{q\in \Q}\left(\{f>q\}\cap\{g>a-q\}\right)\in \A &\qquad \text{se}\ a\in \R;\\
			\{f=+\infty\}\cup\{g=+\infty\}\in \A &\qquad \text{se}\ a=+\infty.
		\end{array}\right.
	\]
	\item[\ref{AlM:sca}]
	\[
		\{\lambda f>a\}=\left\{\begin{array}{ll}
			\left\{f<\frac{a}{\lambda}\right\}\in \A &\qquad \text{se}\ \lambda<0;\\
			X \in \A &\qquad \text{se}\ \lambda=0\ \text{e}\ a< 0;\\
			\emptyset \in \A &\qquad \text{se}\ \lambda=0\ \text{e}\ a\geq 0;\\
			\left\{f>\frac{a}{\lambda}\right\}\in \A &\qquad \text{se}\ \lambda>0.
		\end{array}\right.
	\]
	\item[\ref{AlM:pro}] Scomponiamo $f=f^+ - f^-$, $g=g^+- g^-$, quindi la funzione prodotto $fg$ si scrive come una qualche combinazione di prodotti di funzioni misurabili non negative (abbiamo già dimostrato nella \cref{nota:ProprietaFpiuFmeno} le proprietà di $f^+,f^-,g^+,g^-$ che ci servono). Grazie ai punti \ref{AlM:sum} e \ref{AlM:sca} e a questa osservazione ci basta mostrare il caso in cui $f,g\geq0$:
	\[
		\{fg>a\}=\left\{\begin{array}{ll}
			X\in \A &\quad se\ a<0;\\
			\{f>0\}\cup\{g>0\}\in \A &\quad se\ a=0;\\
			\bigcup_{q\in \Q^+}\left(\{q<f<+\infty\}\cap\left\{\frac{a}{q}<g<+\infty \right\} \right)\in \A &\quad se\ 0<a<+\infty;\\
			(\{f=+\infty\}\cap\{g>0\})\cup (\{f>0\}\cap\{g=+\infty\})\in \A &\quad se\ a=+\infty.
		\end{array}\right.
	\]
	\end{description}
\end{proof}

\begin{remark}\label{nota:CarMis}
	È facile vedere che le funzioni caratteristiche degli insiemi misurabili sono misurabili.
\end{remark}
\begin{proof}
	Basta osservare che se $A\in \A$ è misurabile, $\{ \chi_A > a\}$ può valere solo $\emptyset$, $A$, $X$ (tutti e 3 misurabili) a seconda che
	$a\geq 1$, $a\geq 0$ oppure $a < 0$ rispettivamente.
\end{proof}

\begin{remark}\label{nota:ContinueMisurabili}
	Sia $X$ un insieme dotato sia di una topologia che di una misura su di esso, tali che in particolare la \sigalg{} dei misurabili contenga tutti gli aperti.
	Data una funzione $f:X\to\R$, se $f$ è continua è anche misurabile.
\end{remark}
\begin{proof}
	Basta notare che la controimmagine di un aperto è un aperto per continuità, ma gli aperti sono misurabili per ipotesi e di conseguenza la funzione è misurabile.
\end{proof}

\begin{remark}\label{nota:MonotoneMisurabili}
	Fissato $A\subseteq \R$ misurabile, ogni funzione $f:A\to\R$ monotona è misurabile, munendo $\R$ della misura di Lebesgue definita nella precedente sezione.
\end{remark}
\begin{proof}
	È sufficiente notare che la controimmagine di un intervallo\footnote{Definiamo, unicamente in questa dimostrazione, un intervallo come un generico sottoinsieme connesso di \R.} è a sua volta un intervallo intersecato $A$ poiché la funzione è monotona, perciò applicando la \cref{prop:BasicMis} ricaviamo che la funzione è misurabile visto che gli intervalli sono misurabili secondo Lebesgue (e lo è la loro interesezione con $A$, per la \cref{nota:RiduzioneMisura}).
\end{proof}

\begin{proposition}\label{prop:BorelianiNonMisurabili2}
	I Boreliani di $\R$ non coincidono con l'insieme $\M_1$ dei misurabili.
\end{proposition}
\begin{proof}
	Definiamo la funzione $f:\co{0}{1}\to \co{0}{1}$ in modo che $f(x)$ sia il numero che corrisponde alla lettura in base $3$ della scrittura in base $2$ di $x$.
	Poiché alcuni numeri hanno due scritture in base $2$, sceglieremo sempre quella che non ha una coda infinita di $1$.
	
	Qui di seguito un diagramma che mostra la definizione di $f$:
	\begin{equation*}
		x=\>\stackrel{\text{Scrittura in base $2$ di $x$}}{\overline{0.x_1x_2x_3\cdots}_2} \>  \longmapsto
		\> \stackrel{\text{Lettura in base $3$ di $x$ in base $2$}}{\overline{0.x_1x_2x_3\cdots}_3}\>=f(x)
	\end{equation*}

	La funzione $f$ appena definita è strettamente crescente, poiché lo è la funzione che associa ad un numero $x$ la sua lettura in qualche base (dove le sequenze di cifre sono ordinate lessicograficamente). Allora per la \cref{nota:MonotoneMisurabili} $f$ è misurabile.
	
	Inoltre l'immagine di $f$ è un insieme trascurabile, infatti questa coincide con i numeri tra $0$ e $1$ che si scrivono unicamente usando cifre $0,1$ in base $3$ e questo è facile dimostrare che è trascurabile (esercizio per il lettore molto simile all'\cref{ex:CantorTrascurabile}).
	
	Per il \cref{thm:InsiemeVitali} esiste $A\subseteq \co{0}{1}$ che non sia misurabile.
	Sia $B=f(A)$.
	
	Poiché $f$ è strettamente crescente è in particolare iniettiva e perciò $A=f^{-1}(B)$.
	Allora la \cref{prop:CounterImgMis} ci assicura che $B$ non appartiene ai Boreliani, altrimenti la sua controimmagine sarebbe misurabile. 
	Infine $B$ è sottoinsieme di $\co01$, che è trascurabile, perciò per la completezza della misura di Lebesgue $B$ è misurabile.
	
	Allora, unendo quanto detto, abbiamo che $B$ è un misurabile non Boreliano come voluto.
\end{proof}



\begin{definition}
	Una funzione $\simp:X \rightarrow \Rbar$ con dominio lo spazio di misura $(X,\A,\mu)$ si dice semplice se è combinazione lineare di
	funzioni caratteristiche di insiemi misurabili.
\end{definition}
\begin{remark}
	È immediato che le funzioni semplici sono misurabili.
\end{remark}
\begin{proof}
	Discende dalla \cref{nota:CarMis} e dalla \cref{prop:AlgMis}.
\end{proof}


\begin{proposition}\label{prop:SupDiMisurabili}
	Sia $\{f_n\}_{n\in \N}$ una famiglia di funzioni misurabili definite dallo spazio di misura $(X,\A,\mu)$ a $\Rbar$.
	Allora $F:X\rightarrow \Rbar$ definita da $F(x)=\sup\{f_n(x):n\in \N\}$ è misurabile.
\end{proposition}
\begin{proof}
	Consideriamo il sovralivello della funzione $F$: $\{F>a\}=\bigcup_{n\in \N}\{f_n>a\}$, ma allora $\{F>a\}\in \A$ per le proprietà 
	di chiusura della \sigalg.
\end{proof}

\begin{remark}\label{nota:LimMis}
	Quest'ultima proposizione ha alcune notevoli conseguenze immediate:
	\begin{enumerate}
		\item $\inf$ di una famiglia numerabile di misurabili è misurabile;\label{LM:inf}
		\item $\limsup$ e $\liminf$ di una famiglia numerabile di misurabili sono misurabili;\label{LM:lim_infsup}
		\item limite puntuale di funzioni misurabili è misurabile.\label{LM:lim}
	\end{enumerate}
\end{remark}
\begin{proof}
	\begin{description}
		\item[\ref{LM:inf}] Per l'$\inf$ basta notare che $\inf\{f_n\}=-\sup\{-f_n\}$, quindi è misurabile per la \cref{prop:SupDiMisurabili};
		\item[\ref{LM:lim_infsup}] per definizione, $\limsup\{f_n\}$ e $\liminf\{f_n\}$ sono rispettivamente
			$\lim_n\{\sup\{f_n\}\}=\inf\{\sup\{f_n\}\}$ e
			$\lim_n\{\inf\{f_n\}\}=\sup\{\inf\{f_n\}\}$, quindi sono funzioni misurabili per il punto precedente;
		\item[\ref{LM:lim}] infine se esiste il limite $\lim_n\{f_n\}$ allora
			$\liminf\{f_n\}=\limsup\{f_n\}=\lim_n\{f_n\}$, pertanto è misurabile per il punto precedente.
	\end{description}
\end{proof}

\begin{proposition}\label{prop:LimSemMis}
	Sia $f:X \rightarrow \Rbar$ una funzione con dominio lo spazio di misura $(X,\A,\mu)$.
	Allora $f$ è misurabile se e solo se esiste una successione di funzioni semplici $\simp_n$ che converge puntualmente a $f$.
\end{proposition}
\begin{proof}
	Il se è mostrato nella \cref{nota:LimMis}.
	
	Per il solo se facciamo vedere che la seguente successione converge puntualmente a $f$:
	\[
		\simp_n(x) =
		\left\{ \begin{array}{ll}
			n &\qquad se\ f(x)>n;\\
			\frac{k}{2^n} &\qquad se\ \frac{k}{2^n}<f(x)\leq \frac{k+1}{2^n} \qquad k=-n2^n,-n2^n+1,\dots,n2^n;\\
			-n &\qquad se\ f(x)\leq-n;
		\end{array} \right.\ .
	\]
	Prima di tutto abbiamo 
	\[\simp_n=
		n\chi_{\{f>n\}}+
		\sum_{k=-n2^n}^{n2^n}\frac{k}{2^n}\chi_{\left\{ \frac{k}{2^n}<f\leq \frac{k+1}{2^n} \right\}}
		-n\chi_{\{f<-n\}},
	\]
	che mostra che le $\simp_n$ sono funzioni semplici.
	
	Per mostrare la convergenza puntuale distinguiamo $f(x)$ a seconda che sia un numero finito o meno:
	nel primo caso abbiamo che $|f(x)-\simp_n(x)|\leq \frac{1}{2^n}$ definitivamente, cioè $\forall n\geq |f(x)|$,
	nel secondo caso $\simp_n(x)=\pm n\rightarrow \pm\infty = f(x)$;
	quindi $\simp_n(x)\rightarrow f(x)$, $\forall x\in X$.
\end{proof}

\begin{definition}
	Una funzione $f$ definita su $(X,\A,\mu)$ e a valori in $\Rbar$ si dice positiva se assume solo valori maggiori o uguali a 0.
\end{definition}


\begin{corollary}\label{cor:LimSemCrescMis}
	Sia $f:X\rightarrow \Rbar$ misurabile e positiva su $(X,\A,\mu)$ spazio di misura, allora esiste una successione crescente di funzioni semplici e positive $(\simp_n)$ che converge puntalmente a $f$.
\end{corollary}
\begin{proof}
	Costruendo le $\simp_n$ come nella \cref{prop:LimSemMis}, se $f$ è positiva otteniamo facilmente che le $\simp_n$ sono anche crescenti, da cui la tesi.
\end{proof}

\begin{theorem}\label{thm:ChiusuraMonotonaFunzioni}
	Se $\mathcal F$ è una famiglia di funzioni da $X$ a $\Rbar$, dove $(X,\A,\mu)$ è uno spazio di misura, tale che
	\begin{itemize}
	 \item $\mathcal F$ è uno spazio vettoriale;
	 \item $\mathcal F$ contiene le funzioni $\chi_A$ $\forall A\in \A$;
	 \item se $(f_n)\subseteq \mathcal F$ è una successione monotona che converge a $f$, allora $f\in \mathcal F$;
	\end{itemize}
	allora $\mathcal F$ contiene tutte le funzioni misurabili.
\end{theorem}
\begin{proof}
	Per prima cosa notiamo che $\mathcal F$ contiene le funzioni semplici: essendo $\mathcal F$ uno spazio vettoriale, contiene le combinazioni
	lineari delle funzioni caratteristiche, cioè le funzioni semplici.
	
	Notiamo allora che per il \cref{cor:LimSemCrescMis}, $\mathcal F$ contiene le funzioni misurabili positive. Allora, data $f$ misurabile,
	$f = f_+-f_-$ quindi è contenuta in $\mathcal F$, ancora per le proprietà di spazio vettoriale, poiché entrambe $f_+,f_-$ sono
	positive e misurabili per la \cref{nota:ProprietaFpiuFmeno}.
\end{proof}

\section{Integrazione secondo Lebesgue}
In questa sezione definiremo la nozione di integrale secondo Lebesgue e dimostreremo alcuni risultati introduttivi. In particolare definiremo inizialmente l'integrale di funzioni misurabili positive per poi estenderlo facilmente alle funzioni misurabili che hanno integrale del valore assoluto finito. Chiameremo queste ultime funzioni integrabili.

\begin{lemma}
	Data una funzione $\simp$ semplice e positiva su $(X,\A,\mu)$, esistono $c_k>0$ reali, con $k=1,\dots, n$, tale che $\simp=\sum_{k=1}^nc_k\chi(E_k)$, dove $E_k\in \A$.
\end{lemma}


\begin{definition}
	Sia $\simp$ una funzione misurabile, semplice e non negativa, definita sullo spazio di misura $(X,\A,\mu)$. Definiamo l'integrale di Lebesgue della funzione il valore
	\begin{equation*}
		\int_X \simp d \mu = \sum_{k=1}^n c_k\mu(E_k)
	\end{equation*}
	dove $c_k\in \R$ e $E_k \in \A$ sono tali che $\simp=\sum_{k=1}^nc_k\chi(E_k)$.
\end{definition}

\begin{remark}
	Quella appena enunciata è una buona definizione, cioè non dipende dalla scelta dei $c_k$ e degli $E_k$.
\end{remark}
\begin{proof}
	Consideriamo innanzitutto dei $c_k\in \R$ e degli $E_k\in \A$ tali che $\simp=\sum_{k=1}^nc_k\chi(E_k)$ e definiamo per ogni $\alpha\subseteq\{1,2,\dots,n\}$:
	\begin{equation*}
		\begin{cases}
			E_\alpha=\bigcap_{k\in\alpha}E_k\setminus \bigcup_{k\not\in \alpha} E_k\\
			c_\alpha=\sum_{k\in\alpha} c_k
		\end{cases}
	\end{equation*}
	Allora vogliamo dimostrare che $\sum_{k=1}^nc_k\mu(E_k)=\sum_{\alpha}c_\alpha\mu(E_\alpha)$

\end{proof}



\section{Misura prodotto}
Trattiamo ora la possibilità di rendere uno spazio di misura il prodotto di due spazi di misura. 

Questo risulterà facile sfruttando, come strumento principale, il teorema di estensione di Caratheodory. 
Nella dimostrazione del primo teorema, che fondamentalmente verifica le ipotesi di Caratheodory, proponiamo una via tecnica (che sfrutta la teoria degli integrali costruita), che diverge da quello che potrebbe essere un modo standard di procedere. Questo è sia più breve di una dimostrazione fatta con le mani, sia rende chiara da subito la possibilità di avere scambi tra gli operatori integrali in uno spazio prodotto. 

Infine dimostreremo i teoremi di Fubini e Tonelli, cioè la possibilità di scambiare tra loro gli operatori di integrazione sotto opportune ipotesi.

\newcommand{\B}{\ensuremath{\mathscr B}}
\begin{theorem}
	Dati $(X,\A,\mu)$ e $(X,\B,\nu)$ spazi di misura, definiamo la funzione $\mu\nu:\A\times\B\to\Rpiu$ come il prodotto delle misure $\mu\nu(A\times B)=\mu(A)\nu(B)$.
	
	Allora $(X\times Y,\A\times\B,\mu\nu)$ è uno spazio di misura elementare.
\end{theorem}


\end{document}

\makeindex