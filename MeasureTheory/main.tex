\documentclass[a4paper,12pt]{article}
\usepackage{stilebase}
% \usepackage{float}
% \usepackage{figure}

\title{Appunti di teoria della misura elementare}
\author{Giada Franz \and Federico Glaudo \and Marco Trevisiol}

\makeindex[title=Indice analitico]
\indexsetup{level=\section}

\begin{document}

\maketitle

\begin{abstract}
	Trattiamo in queste dispense i fatti fondamentali di teoria della misura.
	
	In particolare, dopo una prima sezione prevalentemente di definizioni e fatti preparatori, dimostreremo il teorema di estensione di \carat{}, che permette di costruire misure a partire da strutture ben più semplici come le misure elementari. Applicheremo poi questa costruzione generale per fondare su solide basi la misura di Lebesgue su $\R^n$.
	
	In seguito svilupperemo la teoria dell'integrazione nel caso generale, per poi approfondire in maniera più attenta il caso dell'integrazione negli spazi euclidei. Sarà proprio nell'ambientazione euclidea che definiremo anche la misura su sottovarietà di $\R^n$ e dimostreremo il teorema di Stokes, concludendo così la trattazione.
	
	Il percorso seguito ricalca quello proposto dai professori Pietro Majer e Paolo Acquistapace durante il corso di \textit{Analisi in più variabili I} dell'anno 2013-2014 alla facoltà di matematica a Pisa.
	
	Il layout di queste pagine è volutamente fuori dall'ordinario: le pagine hanno margini molto più stretti dello standard sia del \LaTeX{} che dei libri di matematica. 
	Abbiamo fatto questa scelta perché secondo noi questo formato è ottimale per la lettura su schermo e in quanto, pur non permettendo di prendere note sul foglio stesso, rende meno faticosa la lettura cartacea.
	
	Vogliamo infine ringraziare, per le molte correzioni segnalateci, i nostri colleghi e amici: Fabio Ferri, Guglielmo Nocera, Gioacchino Antonelli, Luigi Pagano, Luca Minutillo Menga.
\end{abstract}
\clearpage

\tableofcontents
\clearpage

\section{Definizioni e risultati introduttivi}
In questa sezione definiremo la distanza di Hausdorff fra i chiusi e limitati e mostreremo qualche risultato introduttivo.

In tutto il corso della trattazione indicheremo con $(X,d)$ uno spazio metrico generico.

\begin{definition}
	Sia $\mathcal{K}(X)=\{K\in \mathcal{P}(X) : K \text{ è compatto}\}$ l'insieme dei compatti di $X$.
\end{definition}

\begin{definition}
	Dato $A\in \mathcal{K}(X)$, definiamo $d_A: X\to [0,+\infty)$ tale che $d_A(x)=\inf_{a\in A}d(a,x)$ per ogni $x\in X$.
\end{definition}

\begin{remark}
	$d_A({}\cdot{})$ è effettivamente una funzione a valori in $[0,+\infty)$ perchè vale $\inf_{a\in A}d(a,x)<\infty$, poichè $A$ essendo compatto è limitato.
\end{remark}

\begin{lemma}\label{DistanzaCompattoRealizzata}
	Per ogni $x\in X$ e $A \in \mathcal{K}(X)$, esiste $a\in A$ tale che $d_A(x)=d(a,x)$.
\end{lemma}
\begin{proof}
	Poichè $d_A(x)=\inf_{a\in A} d(a,x)$, esiste $(a_n)$ a valori in $A$ tale che $\lim_{n\to\infty}d(a_n,x)=d_A(x)$. Dato che $A$ è un compatto esiste una sottosuccessione $(a_{n_k})$ di $(a_n)$ convergente ad $a\in A$. Tale sottosuccessione rispetta quindi che $a_{n_k}\to a$ e $d(a_{n_k},x)\to d_A(x)$, da cui facilmente $d(a,x)=d_A(x)$, poichè $d(a_{n_k},x)\to d(a,x)$.
\end{proof}
\begin{remark}\label{DistanzaCompattoAppartenenza}
	Dati $x\in X$ e $A \in \mathcal{K}(X)$, $d_A(x)=0$ se e solo se $x\in A$. Infatti per \cref{DistanzaCompattoRealizzata} esiste $a\in A$ tale che $d_A(x)=d(a,x)$, da cui $d_A(x)=0$ se e solo se $a=x$.
\end{remark}

\begin{definition}\label{DistanzaFraCompatti}
	Sia $\delta_A:\mathcal{K}(X)\to [0,+\infty)$ tale che $\delta_A(B)=\sup_{b\in B} d_A(b)$.
\end{definition}
\begin{remark}
	Anche in questo caso $\delta_A(B)$ ha valori in $[0,+\infty)$ perchè $A\cup B$ è limitato.
\end{remark}

\begin{definition}[Distanza di Hausdorff]\label{HausdorffDefinizione}
	Definiamo infine $\delta:\mathcal{K}(X)\times \mathcal{K}(X) \to [0,+\infty)$ come $\delta(A,B)=\max\{ \delta_A(B),\delta_B(A) \}$.
\end{definition}

\begin{theorem}
	La funzione definita in \cref{HausdorffDefinizione} è una distanza chiamata distanza di Hausdorff.
\end{theorem}

\begin{proof}
	Dimostriamo che $\delta({}\cdot{},{}\cdot{})$ è veramente una distanza.
	\begin{itemize}
		\item $\delta(A,B)=\delta(B,A)$ (simmetria)
		
		$\delta(A,B)=\max\{ \delta_A(B),\delta_B(A) \}=\delta(B,A)$.
		\item $\delta(A,B)=0 \iff A=B$
		
		Se $A=B$ vale banalmente $\delta(A,B)=0$; se invece esiste $a\in A$ tale che $a\not\in B$ ho che $\delta(A,B)\ge \delta_B(A)\ge \delta_B(a)>0$, dove l'ultima disuguaglianza è vera per \cref{DistanzaCompattoAppartenenza}.
		\item $\delta(A,C)\le \delta(A,B)+\delta(B,C)$ (disuguaglianza triangolare)
		
		Dimostro innanzitutto che per ogni $c\in C$ vale $d_A(c)\le \delta(A,B)+\delta(B,C)$.
		
 		Per \cref{DistanzaCompattoRealizzata} ho che esiste $a\in A$ tale che $d_A(c)=d(a,c)$ e che esiste $b\in B$ tale che $d_B(c)=d(b,c)$. Analogamente esiste $a'\in A$ tale che $d_A(b)=d(a',b)$. Utilizzando facili conseguenze delle definizioni date precedentemente, ho quindi che 
 		\begin{equation*}
 			d_A(c)=d(a,c)\le d(a',c) \le d(a',b)+d(b,c)=d_A(b)+d_B(c)\le \delta(A,B)+\delta(B,C)
 		\end{equation*}
 		
 		Passando ora al $\sup$ su $c\in C$ in quest'ultima disuguaglianza ottengo che $\delta_A(C)\le \delta(A,B)+\delta(B,C)$, ma del tutto  analogamente vale $\delta_C(A)\le \delta(A,B)+\delta(B,C)$, quindi
 		\begin{equation*}
 			\delta(A,C)=\max\{ \delta_A(B),\delta_B(A) \}\le \delta(A,B)+\delta(B,C)
 		\end{equation*}
		che è proprio la disuguaglianza triangolare.
	\end{itemize}
	Inoltre $\delta$ ha valori in $[0,+\infty)$ poichè per ogni $A,B\in \mathcal{K}(X)$ $\delta(A,B)<\infty$, dato che $A$ e $B$ sono limitati.
\end{proof}

\begin{remark}
	Del tutto analogamente si dimostra che la distanza di Hausdorff è una distanza anche sui chiusi e limitati, poichè anche in questo caso vale \cref{DistanzaCompattoAppartenenza}.
\end{remark}

\begin{definition}[Definizione equivalente della distanza di Hausdorff] \label{HausdorffDefinizioneEquivalente}
	La distanza di Hausdorff si può definire in modo equivalente nel seguente modo. Sia $\delta_A'(B)=\inf\{r:B\subseteq U_r(A)\}$, dove $U_r(A)=\{x\in X : d_A(x)\le r \}$. Allora definisco $\delta'(A,B)=\max\{\delta_A'(B),\delta_B'(A)\}$
\end{definition}

\begin{theorem}
	Le definizioni \cref{HausdorffDefinizione} e \cref{HausdorffDefinizioneEquivalente} sono equivalenti.
\end{theorem}
\begin{proof}
	Dimostro in particolare che $\delta_A(B)=\delta_A'(B)$, dove $\delta_A$ e $\delta_A'$ sono definite rispettivamente in \cref{DistanzaFraCompatti} e \cref{HausdorffDefinizioneEquivalente}, poichè da questo segue banalmente la tesi.
	
	Chiamo $P=\{ r : \exists b\in B \text{ tale che }d_A(b)=r \}$ e $Q=\{ r: \forall b\in B\text{ vale } d_A(b)\le r \}$, allora si nota facilmente che per ogni $p\in P$ e $q\in Q$ vale $p\le q$ e inoltre per ogni $x\in X$ ho che $x\in P$ o $x\in Q$. Da questo segue facilmente che $\sup\{ r: r\in P \}=\inf\{r: r\in Q \}$.
	
	Vale quindi che
	\begin{equation*}
		\delta_A(B)=\sup_{b\in B} d_A(b)=\sup\{ r : r\in P \}=\inf\{ r : r\in Q \}= \inf\{ r : B\subseteq U_r(A) \}=\delta_A'(B)
	\end{equation*}
	che è quello che volevo dimostrare
\end{proof}

\begin{lemma}\label{IsometriaCanonica}
	Esiste un'isometria canonica $\varphi: X\in \mathcal{K}(X)$, tale che $\varphi(X)$ è un chiuso in $\mathcal{K}(X)$.
\end{lemma}
\begin{proof}
	Definisco l'isometria $\varphi: X\in\mathcal{K}(X)$ tale che $\varphi(x)=\{x\}$ per ogni $x\in X$. 
	
	Innanzitutto vale banalmente che questa è un'isometria, perchè segue facilmente dalle definizioni che $\delta(\{ x \}, \{ x' \})=d(x,x')$.
	
	Dimostriamo ora che $\varphi(X)$ è un chiuso. Sia $(\{ x_n \})_{n\in \mathbb{N}}$ una successione convergente in $\mathcal{K}(X)$, allora $(x_n)_{n\in \mathbb{N}}$ converge in $X$ ad un valore $x$ (poichè $\varphi$ è un isometria). Voglio mostrare che allora $(\{ x_n \})_{n\in \mathbb{N}}$ converge a $\{ x \}$, ma questo è ovvio perchè $\lim_{n\to\infty} \delta(\{ x_n \}, \{x\})=\lim_{n\to\infty} d(x,x')=0$. Questo conclude la dimostrazione per l'unicità del limite.
\end{proof}










\section{Estendere una premisura ad una misura}
L'obiettivo ora è riuscire ad estendere una premisura definita su un semianello ad una misura su una \sigalg{}. Per fare questo il percorso sarà prima quello di estendere la premisura ad una misura esterna, per poi ridurre questa ad una misura canonica.


\begin{theorem}\label{RiduzionePreCaratheodory}
	Data $\mu:\mathcal P(X)\to \Rpiu$ una misura esterna, sia $\A\subseteq \mathcal P(X)$ l'insieme così definito:
	\begin{equation*}
		\A=\{E\in\mathcal P(X):\ \mu(A)=\mu(A\cap E)+\mu(A\setminus E)\ \forall A\in \mathcal P(X)\}
	\end{equation*}
	allora $\A$ è una \sigalg{}, detta \sigalg{} di Caratheodory, e $\mu$ ridotta su $\A$ è una misura completa.
\end{theorem}
\begin{proof}
	La dimostrazione procede in tre passi: prima mostriamo che $\A$ è un'algebra di insiemi, poi che è è una \sigalg{} e infine che $\mu$ è \sigadd{} e completa ridotta su $\A$.
	
	Il fatto che $\A$ sia stabile per complementare è ovvio per la definizione (che è simmetrica tra $E$ ed $E^c$).
	
	Fissati $A\in\mathcal P(X)$ generico ed $E,F\in\A$, applicando la sola definizione di $\A$ ed alcuni passaggi insiemistici si ricava:
	\begin{align*}
		\mu(A)\stackrel{F\in\A}{=}&\mu(A\cap F)+\mu(A\setminus F)\stackrel{E\in\A}{=}
		\mu(A\cap F)+\mu\left((A\setminus F)\cap E\right)+\mu\left((A\setminus F)\setminus E\right)\\
		=\hspace{0.4em}&\mu\left((A\cap (E\cup F))\cap F\right)+\mu\left((A\cap (E\cup F))\setminus F\right)+
		\mu\left(A\setminus(E\cup F)\right)\\
		\stackrel{F\in\A}{=}&\mu(A\cap (E\cup F))+\mu\left(A\setminus(E\cup F)\right)
	\end{align*}
	e visto che questo vale per ogni scelta di $A\in\mathcal P(X)$ abbiamo dimostrato che $\A$ è stabile per unione.
	
	Unendo quanto detto si ha facilmente che $\A$ è un'algebra di insiemi.
	
	Ora sia $(E_n)_{n\in\N}\subseteq \A$ una famiglia numerabile di insiemi ed $A\in\mathcal P(X)$ un generico sottoinsieme di $X$.
	
	Per la \sigsubadd[ità] di $\mu$ vale:
	\begin{equation}\label{DisuguaglianzaFacileCaratheodory}
		\mu(A)\le \mu\left(A\cap\bigcup_{n\in\N} E_n\right)+\mu\left(A\setminus\cap\bigcup_{n\in\N} E_n\right)
	\end{equation}
	Si vuole dimostrare che il $\le$ è in realtà un'uguaglianza. Se $\mu(A)=+\infty$ questo è ovvio, quindi tratteremo il caso in cui $\mu(A)<+\infty$. Chiamiamo $F_n=E_n\setminus \bigcup_{i<n} E_i$, ottenendo in maniera ovvia che gli $(F_n)_{n\in\N}$ sono a due a due disgiunti e che appartengono a $\A$ poiché quest'ultima è un'algebra.
	
	Per induzione è facile verificare, sfruttando unicamente il fatto che $F_n\in\A$ e $\mu(A)<+\infty$, che risulta:
	\begin{equation}\label{IdentitaDifferenzaCaratheodory}
		\mu\left(A\setminus \bigsqcup_{n\le m} F_n\right)=\mu(A)-\sum_{n\le m} \mu(A\cap F_n)
	\end{equation}
	e incidentalmente da questa formula si ha che la serie $\sum_{n\in\N}\mu(A\cap F_n)$ converge, visto che è a termini positivi e limitata (da $\mu(A)$).
	
	Per la \sigsubadd[ità] di $\mu$ vale:
	\begin{equation}\label{IntersezioneStimaCaratheodory}
		\mu\left(A\cap\bigcup_{n\in\N} E_n\right)=\mu\left(\bigsqcup_{n\in\N} A\cap F_n\right)\le
		\sum_{n\in\N} \mu(A\cap F_n)
	\end{equation}
	mentre, grazie alla monotonia e a \cref{IdentitaDifferenzaCaratheodory} otteniamo:
	\begin{equation}\label{DifferenzaStimaCaratheodory}
		\mu\left(A\setminus\bigcup_{n\in\N} E_n\right) = \mu\left(A\setminus\bigsqcup_{n\in\N} F_n\right) \le \mu\left(A\setminus\bigsqcup_{n\le m} F_n\right) = 
		\mu(A)-\sum_{n\le m}\mu(A\cap F_n)
	\end{equation}
	
	Ora unendo \cref{IntersezioneStimaCaratheodory,DifferenzaStimaCaratheodory} giungiamo ad avere che, per ogni $m\in\mathbb{N}$:
	\begin{equation*}
		\mu\left(A\cap\bigcup_{n\in\N} E_n\right)+\mu\left(A\setminus\bigcup_{n\in\N} E_n\right)\le
		\mu(A)+\sum_{m\le n}\mu(A\cap F_n) 
	\end{equation*}
	ma per la convergenza di $\sum_{n\in \N}\mu(A\cap F_n)$, estraendo l'$\inf$ da entrambe le parti finalmente arriviamo a:
	\begin{equation*}
		\mu\left(A\cap\bigcup_{n\in\N} E_n\right)+\mu\left(A\setminus\bigcup_{n\in\N} E_n\right)\le
		\mu(A)
	\end{equation*}
	che unita a \cref{DisuguaglianzaFacileCaratheodory} ci assicura che vale l'identità tra i membri e, visto che ciò vale indipendentemente dalla scelta di $A\in\mathcal P(X)$, risulta $\bigcup_{n\in\N}E_n\in\A$ che equivale a dire che $\A$ è una \sigalg{}.
	
	Dimostrare che $\mu$ è \sigadd{} su $\A$ è ora molto facile.
	Consideriamo $(E_n)_{n\in\N}\subset \A$ una famiglia numerabile di insiemi \emph{disgiunti}. Per facile induzione si ha che:
	\begin{equation*}
		\mu\left(\bigsqcup_{n\le m}E_n\right)=\sum_{n\le m} \mu(E_n)
	\end{equation*}
	e applicando questa e la monotonia di $\mu$ risulta:
	\begin{equation*}
		\sum_{n\le m} \mu(E_n)=\mu\left(\bigsqcup_{n\le m}E_n\right)\le
		\mu\left(\bigsqcup_{n\in\N}E_n\right)\le \sum_{n\in\N} \mu(E_n)
	\end{equation*}
	e questa doppia disuguaglianza, per la definizione delle serie a termini positivi, implica che tutte le disuguaglianze sono identità. Ma allora questo dimostra che $\mu$ è \sigadd{} su $\A$.
	
	Infine per dimostrare la completezza di $\mu|_{\A}$ basta mostrare che dato $E\in\mathcal P(X)$ trascurabile, vale $E\in\A$ (questo è sufficiente a mostrare la completezza, visto che per monotonia i sottinsiemi di un trascurabile sono a loro volta trascurabili).
	
	Fissato un generico $A\in\mathcal P(X)$, risulta per la monotonia di $\mu$:
	\begin{equation*}
		\mu(A\cap N)+\mu(A\setminus N)\le \mu(N)+\mu(A)=\mu(A)
	\end{equation*}
	che, unita alla \sigsubadd[ità] di $\mu$ mi assicura
	\begin{equation*}
		\mu(A)=\mu(A\cap N)+\mu(A\setminus N)
	\end{equation*}
	che è proprio la condizione di appartenenza ad $\A$.
\end{proof}

\begin{proposition}\label{MisuraEsternaDiPremisura}
	Dato $(X,\mathcal S,\mu)$ uno spazio di misura elementare si consideri la funzione che associa ad ogni sottoinsieme l'estremo inferiore delle misure dei ricoprimenti, cioè $\mu^*:\mathcal P(X)\to\Rpiu$ definita come 
	\begin{equation*}
		\mu^*(A)=\inf\left\{\sum_{n\in\N} A_n\ |\ (A_n)_{n\in\N}\subseteq\mathcal S\ \wedge
		\ A\subseteq\bigcup_{n\in\N}A_n\right\}
	\end{equation*}
	Allora $\mu^*$ è una misura esterna che estende $\mu$ (cioè $\mu^*|_{\mathcal S}=\mu$) ed inoltre $\mathcal S$ appartiene alla relativa \sigalg{} di Caratheodory (come definita in \cref{RiduzionePreCaratheodory}).
\end{proposition}
\begin{proof}
	Per ottenere che $\mu^*$ è una misura esterna basta verificare le proprietà che deve rispettare.
	Ovviamente, poiché $\mu(\emptyset)=0$, vale $\mu^*(\emptyset)=0$. 
	Inoltre, ancora facilmente, $\mu^*$ è monotona, visto che se $A\subseteq B$ un ricoprimento di $B$ ricopre anche $A$.
	E infine è anche \sigsubadd{} visto che l'unione di ricoprimenti (che risulta ancora un ricoprimento numerabile) è un ricoprimento dell'unione.
	
	Dato $S\in\mathcal S$ vale ovviamente $\mu*(S)\le\mu(S)$, poiché $S$ si ricopre da solo. Per dimostrare la disuguaglianza opposta consideriamo $(S_n)_{n\in\N}\in \mathcal S$ un ricoprimento di $S$
	
	Ora perciò resta da dimostrare che se $E\subseteq \mathcal S$ allora per ogni $A\in\mathcal P(X)$ risulta:
	\begin{equation}\label{MisuraEsternaDisDifficile}
		\mu^*(A) \ge \mu^*(A\cap E)+\mu^*(A\setminus E)
	\end{equation}
	Questo è sufficiente ad avere che $\mathcal S$ è contenuto nella \sigalg{} di Caratheodory poiché l'altra disuguaglianza è assicurata dalla \sigsubadd[ità].
	
	Dato $(A_n)_{n\in\N}\subseteq\mathcal S$ un ricoprimento di $A$, chiamiamo $B_n=A_n\cap E$ e $C_n=A_n\setminus E$. Ovviamente $(B_n)_{n\in\N},(C_n)_{n\in\N}$ ricoprono rispettivamente $A\cap E,A\setminus E$. Poiché $\mathcal S$ è un \semiring{} riusciamo però a trovare $(B'^n_i)_{i\in\N},(C'^n_i)_{i\in\N} \subseteq \mathcal S$ tali che $B_n=\bigsqcup_{i\in\N}B'^n_i$ e analogo risultato per $C_n$. Quindi $(B'^n_i)_{n,i\in\N}, (C'^n_i)_{n,i\in\N}$ risultano ricoprimenti con elementi di $\mathcal S$ di $A\cap E,A\setminus E$ rispettivamente.
	Ora, sfruttando non più della sola \sigadd[ità] di $\mu$ concludo:
	\begin{align*}
		\sum_{n\in\N}\mu(A_n)=\sum_{n\in\N} \mu(B_n)+\mu(C_n)&=
		\sum_{n\in\N}\sum_{i\in\N}\mu(B'^n_i)+\mu(C'^n_i)\\
		&=
		\sum_{n,i\in\N}\mu(B'^n_i)+\sum_{n,i\in\N}\mu(C'^n_i)\ge \mu(A\cap E)+\mu(A\setminus E)
	\end{align*}
	ma questo implica facilmente \cref{MisuraEsternaDisDifficile} estraendo l'estremo inferiore a entrambi i membri sui ricoprimenti di $A$.
\end{proof}

\begin{theorem}[Estensione di Caratheodory]\label{EstensionexCaratheodory}
	Dato $(X,\mathcal S,\mu)$ uno spazio di misura elementare esiste una \sigalg{} $\A$ e una funzione $\mu':\A\to\Rpiu$ tali che $\mathcal S\subseteq \A$, $\mu'$ estende la premisura $\mu$ e $(X,\A,\mu')$ è uno spazio di misura completo.
\end{theorem}
\begin{proof}
	Consideriamo la misura esterna $\mu^*:\mathcal P(X)\to\Rpiu$ definita nell'enunciato di \cref{MisuraEsternaDiPremisura}. Sempre \cref{MisuraEsternaDiPremisura} ci assicura che questa è un'estensione di $\mu$.
	
	Possiamo ora ridurre $\mu^*$ grazie al \cref{RiduzionePreCaratheodory} ad una misura completa $\mu':\A\to\Rpiu$ dove $\A$ è la \sigalg{} di Caratheodory. 
	
	Ma come dimostrato in \cref{MisuraEsternaDiPremisura} $\mathcal S\subseteq\A$ e inoltre vale $\mu'|_{\mathcal S}=\mu^*|_{\mathcal S}=\mu$, perciò lo spazio $(X,\A,\mu')$ rispetta tutte le richieste dell'enunciato.
\end{proof}

\section{Misura di Lebesgue}
Ora applicheremo i risultati astratti ottenuti nelle due precedenti sezioni al caso più tangibile della retta reale.

Definiremo la misura di Lebesgue e, oltre a chiarire come mai questa sia la misura più naturale su $\R$, studieremo i misurabili secondo Lebesgue mostrando sia che non coincidono con la \sigalg{} dei boreliani (cioè la \sigalg{} generata dagli aperti) sia che non coincidono con le parti di $\R$.

\begin{definition}
	I Boreliani sono la \sigalg{} generata dai sottoinsiemi aperti della retta reale.
\end{definition}

\begin{theorem}
	Sia $\S$ l'insieme degli intervalli semiaperti a destra di $\R$, cioè i sottoinsiemi della retta reale della forma $[a,b)$ con $a<b$. 
	Definiamo inoltre la funzione $\mu:\S\to\Rbar$ in modo che $f\left([a,b)\right)=b-a$.
	
	Allora $(\R,\S,m)$ è uno spazio di misura elementare.
\end{theorem}
\begin{proof}
	TODO
\end{proof}
\section{Funzioni misurabili}
In questa sezione daremo la definizione di funzione misurabile e dimostreremo alcuni fatti basilari su di esse.

La teoria delle funzioni misurabili, oltre ad essere strettamente necessaria per la successiva teoria dell'integrazione, ci permetterà di dimostrare che i misurabili secondo Lebesgue, introdotti nella sezione precedente, non coincidono con i Boreliani usando strumenti propri della teoria della misura.
Questo fatto lo dimostriamo però solo nel caso unidimensionale sia perché il risultato è già stato dimostrato, sia perché la dimostrazione si adatta facilmente in dimensione maggiore e sia perché troviamo importante rendere chiara l'idea piuttosto che celarla in una notazione troppo pesante.

\begin{definition}[Funzione misurabile]
	Dato uno spazio di misura $(X,\A,\mu)$, una funzione $f:X\rightarrow \Rbar$ si dice misurabile se
	$\forall A \subseteq \Rbar$ aperto si ha $f^{-1}(A)\in \A$.
\end{definition}

\begin{proposition}\label{prop:BasicMis}
	Dato uno spazio di misura $(X,\A,\mu)$, sia $f:X\rightarrow \Rbar$ una funzione, sono equivalenti i seguenti fatti
	\footnote{qui introduciamo la notazione per i \textit{sovralivelli} di una funzione che useremo in tutti gli appunti:
		data una funzione $f$ di codominio reale e un certo reale $k$ indichiamo con $\{f>k\}$ l'insieme
		$\{x:f(x)>k\}=f^{-1}((k,+\infty])$; stessa notazione verrà usata anche per il sottolivello}:
	\begin{enumerate}[label=(\arabic*),ref=(\arabic*)]
		\item $f$ è misurabile; \label{BM:mis}
		\item $\{f<a\}\in \A \quad \forall a\in \Rbar$; \label{BM:sot}
		\item $\{f\leq a\}\in \A \quad \forall a\in \Rbar$; \label{BM:soteq}
		\item $\{f>a\}\in \A \quad \forall a\in \Rbar$; \label{BM:sov}
		\item $\{f\geq a\}\in \A \quad \forall a\in \Rbar$;  \label{BM:soveq}
		\item $\{a<f<b\}\in \A \quad \forall a,b\in \Rbar$. \label{BM:int}
	\end{enumerate}
\end{proposition}
\begin{proof}
	Sfruttando le proprietà di $\A$ come \sigalg, mostriamo a catena tutte le implicazioni:
	\begin{description}
	\item[\ImplicationProof{BM:mis}{BM:sot}] per definizione di misurabile, $\{f<a\}=f^{-1}\left(\oo{-\infty}{a}\right)\in \A$,
		perché $\oo{-\infty}{a}$ è aperto;
	\item[\ImplicationProof{BM:sot}{BM:soteq}] poiché $\oc{-\infty}{a}=\bigcap_{n\in \N}\oc{-\infty}{a+\frac{1}{n}}$, allora otteniamo
		\begin{equation*}
			\{f\leq a\}=f^{-1}\left(\oc{-\infty}{a}\right)=\bigcap_{n\in \N}f^{-1}\left(\co{-\infty}{a+\frac{1}{n}}\right)\in \A	
		\end{equation*}

	\item[\ImplicationProof{BM:soteq}{BM:sov}] passando al complementare, $\{f>a\}^\mathsf{c}=\{f\leq a\}\in \A$;
	\item[\ImplicationProof{BM:sov}{BM:soveq}] analogamente a \ImplicationProof{BM:sot}{BM:soteq}; 
	\item[\ImplicationProof{BM:soveq}{BM:sot}] analogamente a \ImplicationProof{BM:soteq}{BM:sov};
	\item[$\text{\ref{BM:sot}}\ +\ \text{\ref{BM:sov}}\implies\text{\ref{BM:int}}$] perché
		$\{a<f<b\}=\{a<f\}\cap\{f<b\}\in \A$;
	\item[\ImplicationProof{BM:int}{BM:mis}] perché un aperto $A\subseteq\Rbar$ si può scrivere come
		$A=\bigcup_{n\in \N}A_n$ dove ciascun $A_n$ è un intervallo aperto di $\Rbar$ (o una semiretta aperta);
		quindi $f^{-1}(A)=\bigcup_{n\in \N}f^{-1}(A_n)\in \A$, perché per il punto \ref{BM:int} $f^{-1}(A_n)\in \A\ \ \forall n$.
	\end{description}
\end{proof}

\begin{proposition}\label{prop:CounterImgMis}
	Dato uno spazio di misura $(X,\A,\mu)$, e data $f:X\rightarrow \Rbar$ una funzione misurabile, la famiglia di insiemi
	\[
		\mathcal{E} = \{ E\subseteq \Rbar : f^{-1}(E)\in \A \}
	\]
	è una \sigalg{} ed inoltre contiene i Boreliani.
\end{proposition}
\begin{proof}
	Verifichiamo che $\mathcal E$ è stabile per unioni numerabili e passaggio al complementare.
	
	Fissati $\{E_n\}_{n\in \N}\subseteq \mathcal{E}$ sia $E = \cup_{n\in \N}E_n$, vale:
	\begin{equation*}
		f^{-1}(E)=f^{-1}\left(\cup_{n\in \N}E_n\right) = \cup_{n\in \N}f^{-1}(E_n)\in \A \implies E \in \mathcal{E}
	\end{equation*}
	dove l'appartenenza ad $\A$ si ha per le proprietà di \sigalg{}, e questo dimostra la stabilità per unione numerabile.
	
	Per quanto riguarda il passaggio al complementare, fissato $E\in \mathcal{E}$, risulta:
	\begin{equation*}
		f^{-1}(E^\mathsf{c})= f^{-1}(E)^\mathsf{c} \in \A \implies E^\mathsf{c} \in \mathcal{E}.
	\end{equation*}
	
	Inoltre $\mathcal E$ contiene gli aperti per definizione di funzione misurabile, da cui, per quanto appena dimostrato, contiene la \sigalg{} generata da questi, cioè i Boreliani.
\end{proof}

\begin{definition}\label{def:FpiuFmeno}
	Sia $f:X\to\Rbar$ una funzione misurabile su uno spazio di misura $(X,\A,\mu)$. Definiamo $f^+ = \max\{f,0\}$ e $f^- = \max\{-f,0\}$.
\end{definition}
\begin{remark}\label{nota:ProprietaFpiuFmeno}
	Data una funzione $f$ misurabile, vale $f^+,f^-$ sono misurabile. Inoltre $f^+,f^-\ge 0$ e $f=f^+-f^-$. 
\end{remark}
\begin{proof}
	Abbiamo che per ogni $a\in\Rbar$ vale
	\begin{equation*}
		\{f^+>a\}=\left\{\begin{array}{ll}
			\{f>a\}\in\A &\text{se }a>0\\
			X\in\A &\text{se }a\le 0
	\end{array}\right.
	\end{equation*}
	Quindi, per la \ref{BM:sov} della \cref{prop:BasicMis}, $f^+$ è misurabile. Analogamente si dimostra che $f^-$ è misurabile.
\end{proof}


\begin{proposition}\label{prop:AlgMis}
	Dato uno spazio di misura $(X,\A,\mu)$, sia $\mathcal{M}$ l'insieme delle funzioni misurabili da $X$ in $\Rbar$.
	Allora $\mathcal{M}$ è un'algebra nel senso che, dove sono definite \footnote{Nel definire le operazioni algebriche su $\mathcal{M}$ adottiamo le seguenti convenzioni: la somma è definita
		se non accade che entrambe $f$ e $-g$ siano $\pm\infty$, per la moltiplicazione $0\cdot \infty = 0$.},
	valgono le seguenti:
	\begin{enumerate}[label=(\arabic*),ref=(\arabic*)]
		\item $f,g\in \mathcal{M} \Rightarrow f+g\in \mathcal{M}$; \label{AlM:sum}
		\item $f\in \mathcal{M}, \lambda \in \R \Rightarrow \lambda f\in \mathcal{M}$; \label{AlM:sca}
		\item $f,g\in \mathcal{M} \Rightarrow fg\in \mathcal{M}$. \label{AlM:pro}
	\end{enumerate}
\end{proposition}

\begin{proof}
	Mostriamo per ogni punto che vale la proposizione \ref{BM:sov} nella \cref{prop:BasicMis} (che come lì mostrato, equivale alla misurabilità),
	distinguendo vari casi di $a\in \Rbar$.
	\begin{description}
	\item[\ref{AlM:sum}]
	\[
		\{f+g>a\}=\left\{\begin{array}{ll}
			\{f\ge -\infty\}\cap\{g\ge -\infty\}\in \A &\qquad \text{se}\ a=-\infty;\\
			\bigcup_{q\in \Q}\left(\{f>q\}\cap\{g>a-q\}\right)\in \A &\qquad \text{se}\ a\in \R;\\
			\{f=+\infty\}\cup\{g=+\infty\}\in \A &\qquad \text{se}\ a=+\infty.
		\end{array}\right.
	\]
	\item[\ref{AlM:sca}]
	\[
		\{\lambda f>a\}=\left\{\begin{array}{ll}
			\left\{f<\frac{a}{\lambda}\right\}\in \A &\qquad \text{se}\ \lambda<0;\\
			X \in \A &\qquad \text{se}\ \lambda=0\ \text{e}\ a< 0;\\
			\emptyset \in \A &\qquad \text{se}\ \lambda=0\ \text{e}\ a\geq 0;\\
			\left\{f>\frac{a}{\lambda}\right\}\in \A &\qquad \text{se}\ \lambda>0.
		\end{array}\right.
	\]
	\item[\ref{AlM:pro}] Scomponiamo $f=f^+ - f^-$, $g=g^+- g^-$, quindi la funzione prodotto $fg$ si scrive come una qualche combinazione di prodotti di funzioni misurabili non negative (abbiamo già dimostrato nella \cref{nota:ProprietaFpiuFmeno} le proprietà di $f^+,f^-,g^+,g^-$ che ci servono). Grazie ai punti \ref{AlM:sum} e \ref{AlM:sca} e a questa osservazione ci basta mostrare il caso in cui $f,g\geq0$:
	\[
		\{fg>a\}=\left\{\begin{array}{ll}
			X\in \A &\quad se\ a<0;\\
			\{f>0\}\cup\{g>0\}\in \A &\quad se\ a=0;\\
			\bigcup_{q\in \Q^+}\left(\{q<f<+\infty\}\cap\left\{\frac{a}{q}<g<+\infty \right\} \right)\in \A &\quad se\ 0<a<+\infty;\\
			(\{f=+\infty\}\cap\{g>0\})\cup (\{f>0\}\cap\{g=+\infty\})\in \A &\quad se\ a=+\infty.
		\end{array}\right.
	\]
	\end{description}
\end{proof}

\begin{remark}\label{nota:CarMis}
	È facile vedere che le funzioni caratteristiche degli insiemi misurabili sono misurabili.
\end{remark}
\begin{proof}
	Basta osservare che se $A\in \A$ è misurabile, $\{ \chi_A > a\}$ può valere solo $\emptyset$, $A$, $X$ (tutti e 3 misurabili) a seconda che
	$a\geq 1$, $a\geq 0$ oppure $a < 0$ rispettivamente.
\end{proof}

\begin{remark}\label{nota:ContinueMisurabili}
	Sia $X$ un insieme dotato sia di una topologia che di una misura su di esso, tali che in particolare la \sigalg{} dei misurabili contenga tutti gli aperti.
	Data una funzione $f:X\to\R$, se $f$ è continua è anche misurabile.
\end{remark}
\begin{proof}
	Basta notare che la controimmagine di un aperto è un aperto per continuità, ma gli aperti sono misurabili per ipotesi e di conseguenza la funzione è misurabile.
\end{proof}

\begin{remark}\label{nota:MonotoneMisurabili}
	Fissato $A\subseteq \R$ misurabile, ogni funzione $f:A\to\R$ monotona è misurabile, munendo $\R$ della misura di Lebesgue definita nella precedente sezione.
\end{remark}
\begin{proof}
	È sufficiente notare che la controimmagine di un intervallo\footnote{Definiamo, unicamente in questa dimostrazione, un intervallo come un generico sottoinsieme connesso di \R.} è a sua volta un intervallo intersecato $A$ poiché la funzione è monotona, perciò applicando la \cref{prop:BasicMis} ricaviamo che la funzione è misurabile visto che gli intervalli sono misurabili secondo Lebesgue (e lo è la loro interesezione con $A$, per la \cref{nota:RiduzioneMisura}).
\end{proof}

\begin{proposition}\label{prop:BorelianiNonMisurabili2}
	I Boreliani di $\R$ non coincidono con l'insieme $\M_1$ dei misurabili.
\end{proposition}
\begin{proof}
	Definiamo la funzione $f:\co{0}{1}\to \co{0}{1}$ in modo che $f(x)$ sia il numero che corrisponde alla lettura in base $3$ della scrittura in base $2$ di $x$.
	Poiché alcuni numeri hanno due scritture in base $2$, sceglieremo sempre quella che non ha una coda infinita di $1$.
	
	Qui di seguito un diagramma che mostra la definizione di $f$:
	\begin{equation*}
		x=\>\stackrel{\text{Scrittura in base $2$ di $x$}}{\overline{0.x_1x_2x_3\cdots}_2} \>  \longmapsto
		\> \stackrel{\text{Lettura in base $3$ di $x$ in base $2$}}{\overline{0.x_1x_2x_3\cdots}_3}\>=f(x)
	\end{equation*}

	La funzione $f$ appena definita è strettamente crescente, poiché lo è la funzione che associa ad un numero $x$ la sua lettura in qualche base (dove le sequenze di cifre sono ordinate lessicograficamente). Allora per la \cref{nota:MonotoneMisurabili} $f$ è misurabile.
	
	Inoltre l'immagine di $f$ è un insieme trascurabile, infatti questa coincide con i numeri tra $0$ e $1$ che si scrivono unicamente usando cifre $0,1$ in base $3$ e questo è facile dimostrare che è trascurabile (esercizio per il lettore molto simile all'\cref{ex:CantorTrascurabile}).
	
	Per il \cref{thm:InsiemeVitali} esiste $A\subseteq \co{0}{1}$ che non sia misurabile.
	Sia $B=f(A)$.
	
	Poiché $f$ è strettamente crescente è in particolare iniettiva e perciò $A=f^{-1}(B)$.
	Allora la \cref{prop:CounterImgMis} ci assicura che $B$ non appartiene ai Boreliani, altrimenti la sua controimmagine sarebbe misurabile. 
	Infine $B$ è sottoinsieme di $\co01$, che è trascurabile, perciò per la completezza della misura di Lebesgue $B$ è misurabile.
	
	Allora, unendo quanto detto, abbiamo che $B$ è un misurabile non Boreliano come voluto.
\end{proof}



\begin{definition}
	Una funzione $\simp:X \rightarrow \Rbar$ con dominio lo spazio di misura $(X,\A,\mu)$ si dice semplice se è combinazione lineare di
	funzioni caratteristiche di insiemi misurabili.
\end{definition}
\begin{remark}
	È immediato che le funzioni semplici sono misurabili.
\end{remark}
\begin{proof}
	Discende dalla \cref{nota:CarMis} e dalla \cref{prop:AlgMis}.
\end{proof}


\begin{proposition}\label{prop:SupDiMisurabili}
	Sia $\{f_n\}_{n\in \N}$ una famiglia di funzioni misurabili definite dallo spazio di misura $(X,\A,\mu)$ a $\Rbar$.
	Allora $F:X\rightarrow \Rbar$ definita da $F(x)=\sup\{f_n(x):n\in \N\}$ è misurabile.
\end{proposition}
\begin{proof}
	Consideriamo il sovralivello della funzione $F$: $\{F>a\}=\bigcup_{n\in \N}\{f_n>a\}$, ma allora $\{F>a\}\in \A$ per le proprietà 
	di chiusura della \sigalg.
\end{proof}

\begin{remark}\label{nota:LimMis}
	Quest'ultima proposizione ha alcune notevoli conseguenze immediate:
	\begin{enumerate}
		\item $\inf$ di una famiglia numerabile di misurabili è misurabile;\label{LM:inf}
		\item $\limsup$ e $\liminf$ di una famiglia numerabile di misurabili sono misurabili;\label{LM:lim_infsup}
		\item limite puntuale di funzioni misurabili è misurabile.\label{LM:lim}
	\end{enumerate}
\end{remark}
\begin{proof}
	\begin{description}
		\item[\ref{LM:inf}] Per l'$\inf$ basta notare che $\inf\{f_n\}=-\sup\{-f_n\}$, quindi è misurabile per la \cref{prop:SupDiMisurabili};
		\item[\ref{LM:lim_infsup}] per definizione, $\limsup\{f_n\}$ e $\liminf\{f_n\}$ sono rispettivamente
			$\lim_n\{\sup\{f_n\}\}=\inf\{\sup\{f_n\}\}$ e
			$\lim_n\{\inf\{f_n\}\}=\sup\{\inf\{f_n\}\}$, quindi sono funzioni misurabili per il punto precedente;
		\item[\ref{LM:lim}] infine se esiste il limite $\lim_n\{f_n\}$ allora
			$\liminf\{f_n\}=\limsup\{f_n\}=\lim_n\{f_n\}$, pertanto è misurabile per il punto precedente.
	\end{description}
\end{proof}

\begin{proposition}\label{prop:LimSemMis}
	Sia $f:X \rightarrow \Rbar$ una funzione con dominio lo spazio di misura $(X,\A,\mu)$.
	Allora $f$ è misurabile se e solo se esiste una successione di funzioni semplici $\simp_n$ che converge puntualmente a $f$.
\end{proposition}
\begin{proof}
	Il se è mostrato nella \cref{nota:LimMis}.
	
	Per il solo se facciamo vedere che la seguente successione converge puntualmente a $f$:
	\[
		\simp_n(x) =
		\left\{ \begin{array}{ll}
			n &\qquad se\ f(x)>n;\\
			\frac{k}{2^n} &\qquad se\ \frac{k}{2^n}<f(x)\leq \frac{k+1}{2^n} \qquad k=-n2^n,-n2^n+1,\dots,n2^n;\\
			-n &\qquad se\ f(x)\leq-n;
		\end{array} \right.\ .
	\]
	Prima di tutto abbiamo 
	\[\simp_n=
		n\chi_{\{f>n\}}+
		\sum_{k=-n2^n}^{n2^n}\frac{k}{2^n}\chi_{\left\{ \frac{k}{2^n}<f\leq \frac{k+1}{2^n} \right\}}
		-n\chi_{\{f<-n\}},
	\]
	che mostra che le $\simp_n$ sono funzioni semplici.
	
	Per mostrare la convergenza puntuale distinguiamo $f(x)$ a seconda che sia un numero finito o meno:
	nel primo caso abbiamo che $|f(x)-\simp_n(x)|\leq \frac{1}{2^n}$ definitivamente, cioè $\forall n\geq |f(x)|$,
	nel secondo caso $\simp_n(x)=\pm n\rightarrow \pm\infty = f(x)$;
	quindi $\simp_n(x)\rightarrow f(x)$, $\forall x\in X$.
\end{proof}

\begin{definition}
	Una funzione $f$ definita su $(X,\A,\mu)$ e a valori in $\Rbar$ si dice positiva se assume solo valori maggiori o uguali a 0.
\end{definition}


\begin{corollary}\label{cor:LimSemCrescMis}
	Sia $f:X\rightarrow \Rbar$ misurabile e positiva su $(X,\A,\mu)$ spazio di misura, allora esiste una successione crescente di funzioni semplici e positive $(\simp_n)$ che converge puntalmente a $f$.
\end{corollary}
\begin{proof}
	Costruendo le $\simp_n$ come nella \cref{prop:LimSemMis}, se $f$ è positiva otteniamo facilmente che le $\simp_n$ sono anche crescenti, da cui la tesi.
\end{proof}

\begin{theorem}\label{thm:ChiusuraMonotonaFunzioni}
	Se $\mathcal F$ è una famiglia di funzioni da $X$ a $\Rbar$, dove $(X,\A,\mu)$ è uno spazio di misura, tale che
	\begin{itemize}
	 \item $\mathcal F$ è uno spazio vettoriale;
	 \item $\mathcal F$ contiene le funzioni $\chi_A$ $\forall A\in \A$;
	 \item se $(f_n)\subseteq \mathcal F$ è una successione monotona che converge a $f$, allora $f\in \mathcal F$;
	\end{itemize}
	allora $\mathcal F$ contiene tutte le funzioni misurabili.
\end{theorem}
\begin{proof}
	Per prima cosa notiamo che $\mathcal F$ contiene le funzioni semplici: essendo $\mathcal F$ uno spazio vettoriale, contiene le combinazioni
	lineari delle funzioni caratteristiche, cioè le funzioni semplici.
	
	Notiamo allora che per il \cref{cor:LimSemCrescMis}, $\mathcal F$ contiene le funzioni misurabili positive. Allora, data $f$ misurabile,
	$f = f_+-f_-$ quindi è contenuta in $\mathcal F$, ancora per le proprietà di spazio vettoriale, poiché entrambe $f_+,f_-$ sono
	positive e misurabili per la \cref{nota:ProprietaFpiuFmeno}.
\end{proof}

\section{Integrazione secondo Lebesgue}
In questa sezione definiremo la nozione di integrale secondo Lebesgue e dimostreremo alcuni risultati introduttivi. In particolare definiremo inizialmente l'integrale di funzioni misurabili positive per poi estenderlo facilmente alle funzioni misurabili che hanno integrale del valore assoluto finito. Chiameremo queste ultime funzioni integrabili.

\begin{lemma}
	Data una funzione $\simp$ semplice e positiva su $(X,\A,\mu)$, esistono $c_k>0$ reali, con $k=1,\dots, n$, tale che $\simp=\sum_{k=1}^nc_k\chi(E_k)$, dove $E_k\in \A$.
\end{lemma}


\begin{definition}
	Sia $\simp$ una funzione misurabile, semplice e non negativa, definita sullo spazio di misura $(X,\A,\mu)$. Definiamo l'integrale di Lebesgue della funzione il valore
	\begin{equation*}
		\int_X \simp d \mu = \sum_{k=1}^n c_k\mu(E_k)
	\end{equation*}
	dove $c_k\in \R$ e $E_k \in \A$ sono tali che $\simp=\sum_{k=1}^nc_k\chi(E_k)$.
\end{definition}

\begin{remark}
	Quella appena enunciata è una buona definizione, cioè non dipende dalla scelta dei $c_k$ e degli $E_k$.
\end{remark}
\begin{proof}
	Consideriamo innanzitutto dei $c_k\in \R$ e degli $E_k\in \A$ tali che $\simp=\sum_{k=1}^nc_k\chi(E_k)$ e definiamo per ogni $\alpha\subseteq\{1,2,\dots,n\}$:
	\begin{equation*}
		\begin{cases}
			E_\alpha=\bigcap_{k\in\alpha}E_k\setminus \bigcup_{k\not\in \alpha} E_k\\
			c_\alpha=\sum_{k\in\alpha} c_k
		\end{cases}
	\end{equation*}
	Allora vogliamo dimostrare che $\sum_{k=1}^nc_k\mu(E_k)=\sum_{\alpha}c_\alpha\mu(E_\alpha)$

\end{proof}



\section{Misura prodotto}
Trattiamo ora la possibilità di rendere uno spazio di misura il prodotto di due spazi di misura. 

Questo risulterà facile sfruttando, come strumento principale, il teorema di estensione di Caratheodory. 
Nella dimostrazione del primo teorema, che fondamentalmente verifica le ipotesi di Caratheodory, proponiamo una via tecnica (che sfrutta la teoria degli integrali costruita), che diverge da quello che potrebbe essere un modo standard di procedere. Questo è sia più breve di una dimostrazione fatta con le mani, sia rende chiara da subito la possibilità di avere scambi tra gli operatori integrali in uno spazio prodotto. 

Infine dimostreremo i teoremi di Fubini e Tonelli, cioè la possibilità di scambiare tra loro gli operatori di integrazione sotto opportune ipotesi.

\newcommand{\B}{\ensuremath{\mathscr B}}
\begin{theorem}
	Dati $(X,\A,\mu)$ e $(X,\B,\nu)$ spazi di misura, definiamo la funzione $\mu\nu:\A\times\B\to\Rpiu$ come il prodotto delle misure $\mu\nu(A\times B)=\mu(A)\nu(B)$.
	
	Allora $(X\times Y,\A\times\B,\mu\nu)$ è uno spazio di misura elementare.
\end{theorem}

\section{L'integrazione negli spazi Euclidei}
Approfondiamo ora la teoria dell'integrazione nell'ambientazione più classica: lo spazio $\R^n$ munito della misura di Lebesgue.

Mostreremo che l'integrale di Lebesgue non è che una generalizzazione dell'integrale di Riemann e che la misura prodotto definibile su $\R^n$ a partire dalla misura sulla retta reale coincide proprio con la misura $m_n$. Questi due risultati forniranno, a meno di applicare i teoremi di Fubini e Tonelli, i metodi primari per calcolare gli integrali in più dimensioni.

La seconda parte della sezione sarà completamente votata alla dimostrazione della formula del cambio di variabile, che necessita di un gran numero di lemmi e di un teorema, il teorema di Radon-Nikodym, che lasceremo indimostrato.

\begin{proposition}\label{prop:MisuraProdottoEuclidea}
	La \sigalg{} prodotto completata $\overline{\M_n\otimes\M_m}$ dei misurabili di $\R^n$ e $\R^m$ coincide con la \sigalg{} dei misurabili di $\R^{n+m}$.
\end{proposition}
\begin{proof}
	Lavoriamo sulla \sigalg{} prodotto \emph{non completata} $\M_n\otimes\M_m$, per poi ottenere l'enunciato ricordando che il completamento dei Boreliani sono i misurabili.

	Ricordando la \cref{def:LebesgueSemiaperti}, poichè valgono ovviamente i contenimenti $S_n\subseteq M_n$ e $S_m\subseteq M_m$, è facile ricavarne che $S_{n+m}\subseteq M_n\otimes M_m$.
	Ma allora, applicando la \cref{prop:SigAlgUgualeBoreliani} e ricordando che $\M_n\otimes M_m$ è una \sigalg{}, otteniamo che i Boreliani sono contenuti in $\M_n\otimes\M_m$.
	
	Sia $E_1\times E_2\in \M_n\times\M_m$. Per il \cref{thm:LebesgueEquivalenzeMisurabilita}, esistono $B_1,B_2$ Boreliani e $N_1,N_2$ trascurabili, rispettivamente in $R^n$ e $R^m$, tali che $E_1=A_1\sqcup N_1$ e $E_2=A_2\sqcup N_2$.
	Perciò, grazie alla distributività del prodotto insiemistico rispetto all'unione disgiunta, otteniamo
	\begin{equation*}
		E_1\times E_2=(A_1\sqcup N_1)\times(A_2\sqcup N_2)=\left(A_1\times A_2\right)\sqcup\left(A_1\times N_2\sqcup A_2\times N_1 \sqcup N_1\times N_2\right) \punto
	\end{equation*}
	Però l'insieme $A_1\times N_2\sqcup A_2\times N_1 \sqcup N_1\times N_2$ è trascurabile per la \cref{prop:TrascurabilePerInsiemeTrascurabile} e l'insieme $A_1\times A_2$ è un Boreliano in quanto prodotto di Boreliani\footnote{Lasciamo al lettore la dimostrazione che il prodotto di Boreliani è un Boreliano.}, allora applicando ancora il \cref{thm:LebesgueEquivalenzeMisurabilita} si ottiene $E_1\times E_2\in\M_{n+m}$.
	Vista la generalità della scelta di $E_1\times E_2$, quanto appena mostrato implica che $\M_n\times\M_m\subseteq \M_{n+m}$.
	
	Quindi abbiamo dimostrato che $\M_n\otimes\M_m$ contiene i Boreliani ed è un sottoinsieme dei misurabili, perciò, ricordando la \cref{prop:CompletamentoBoreliani}, è ovvio concluderne che il suo completamento\footnote{È fondamentale notare che la misura indotta dal prodotto e la misura di Lebesgue coincidono in virtù della \cref{prop:UnicitaCaratheodory} e di conseguenza anche l'operatore di completamento di una \sigalg{} è coincidente.} $\overline{\M_n\otimes\M_m}$ coincide con l'insieme dei misurabili.
\end{proof}


\begin{proposition}\label{prop:IntegraleRiemannCoincide}
	Data una funzione $f:\cc ab\to \mathbb R$, se essa è integrabile secondo Riemann ed integrabile secondo Lebesgue allora i due integrali coincidono.
\end{proposition}
\begin{proof}
	Nel caso in cui la funzione sia integrabile secondo Riemann (e quindi anche limitata, dando senso agli $\inf,\sup$ che compariranno nelle formule), il valore dell'integrale secondo Riemann corrisponde all'estremo superiore delle somme inferiori alla Riemann\footnote{Indicheremo con $\int_a^b$ l'integrale di Riemann, e con $\int_{\cc ab}$ l'integrale di Lebesgue}
	\begin{equation*}
		\int_a^b f(x)\de x=\sup\left\{\sum_{i=0}^{k-1} (x_{i+1}-x_i)\left(\inf_{t\in\co{x_i}{x_{i+1}}}f(t)\right):\ a=x_0<x_1<\cdots<x_{k-1}<x_k=b\right\} \virgola
	\end{equation*}
	ed anche all'estremo inferiore delle somme superiori alla Riemann
	\begin{equation*}
		\int_a^b f(x)\de x=\inf\left\{\sum_{i=0}^{k-1} (x_{i+1}-x_i)\left(\sup_{t\in\co{x_i}{x_{i+1}}}f(t)\right):\ a=x_0<x_1<\cdots<x_{k-1}<x_k=b\right\} \punto
	\end{equation*}
	
	Data una partizione $a=x_0<x_1<\cdots<x_{k-1}<x_k=b$ dell'intervallo $\cc ab$, definiamo le funzioni
	\begin{align*}
		f^-_{x_0,x_1,\dots,x_k}(x)=\sum_{i=0}^{k-1}\chi_{\co{x_i}{x_{i+1}}}(x)\cdot\inf_{t\in\co{x_i}{x_{i+1}}}f(t) \virgola \\
		f^+_{x_0,x_1,\dots,x_k}(x)=\sum_{i=0}^{k-1}\chi_{\co{x_i}{x_{i+1}}}(x)\cdot\sup_{t\in\co{x_i}{x_{i+1}}}f(t) \punto
	\end{align*}

	Con le definizioni mostrate è facile accorgersi che l'integrale di Riemann allora è l'estremo superiore dell'integrale di Lebesgue delle funzioni semplici $f^-_{x_0,\dots,x_k}$ su tutte le partizioni e analogamente l'estremo inferiore dell'integrale di Lebesgue delle funzioni semplici $f^+_{x_0,\dots,x_k}$ ancora su tutte le partizioni.
	Allora, per la monotonia dell'integrale di Lebesgue è facile accorgersi che
	\begin{multline*}
		\int_{\cc ab}f(x)\de x\le \\
		\inf_{a=x_0<\dots<x_k=b}\left\{\int_{\cc ab}f^+_{x_0,\dots,x_k}(x)\de x\right\}
		=\int_a^b f(x)\de x= 
		\sup_{a=x_0<\dots<x_k=b}\left\{\int_{\cc ab}f^-_{x_0,\dots,x_k}(x)\de x\right\}\\
		 \le \int_{\cc ab} f(x)\de x
	\end{multline*}
	e ciò equivale ovviamente alla tesi.
\end{proof}

Dimostriamo ora un caso particolare, ma molto significativo, della formula del cambio di variabile. 
Proviamo la formula per il cambio di variabile lineare che verrà poi sfruttata, attraverso una sorta di passaggio al limite, per dimostrare la formula più generale.

In particolare, e tale procedimento verrà ripetuto varie volte in questa sezione, prima otterremo un risultato sulla misura di un insieme trasformato da un'applicazione lineare per poi ricavarne facilmente la formula per il cambio di variabile. L'idea che sta alla base di questo è che l'identità riguardante le misure è in realtà un cambio di variabile riguardante le funzioni indicatrici.

\begin{proposition}\label{prop:MisuraImmagineLineare}
	Fissata un'applicazione lineare $L:\R^n\to\R^n$, per ogni $E\in\M_n$ misurabile secondo Lebesgue risulta che $L(E)$ è misurabile e rispetta
	\begin{equation}
		m_n(L(E))=\lvert \det L\rvert\cdot m_n(E)\punto
	\end{equation}
\end{proposition}
\begin{proof}
	La funzione $L$ essendo lineare è in particolare Lipschitziana (con costante la propria norma) quindi, applicando la \cref{prop:LipschitzTengonoMisurabili}, otteniamo che per ogni $E$ misurabile, $L(E)$ è anch'esso misurabile.
	
	Consideriamo ora la funzione $\mu_L:\M_n\to\Rpiu$ definita come $\mu_L(E)=m_n(L(E))$. Per quanto appena mostrato questa è una buona definizione.
	Per la \cref{prop:BigettivaInduceMisura} la $\mu_L$ è una misura.
	Inoltre, ricordando che la misura di Lebesgue è invariante per traslazione come mostrato nella \cref{nota:LebesgueProprieta}, si ottiene
	\begin{equation*}
		\forall v\in\R^n:\ \mu_L(E+v)=m_n(L(E+v))=m_n(L(E)+Lv)=m_n(L(E))=\mu_L(E)
	\end{equation*}
	che equivale a dire che $\mu_L$ è invariante per traslazione.
	\newcommand{\linR}{\ensuremath{\mathcal L(\R^n,\R^n)}}
	Allora possiamo applicare il \cref{thm:LebesgueUnicaInvarianteTraslazione}\footnote{Il teorema ci assicura l'identità delle misure solo sui Boreliani, ma i misurabili sono il completamento dei Boreliani e perciò le due misure coincidono anche sui misurabili.} ottenendo che esiste una funzione $c:\linR\to\Rpiu$ con dominio le applicazioni lineari, tale che valga
	\begin{equation}\label{eq:DefQuasiDet}
		\forall L\in \linR,\ E\in\M_n:\ \mu_L(E)=c(L)m_n(E)\punto
	\end{equation}
	
	Mostriamo ora che $c(\cdot)$ è moltiplicativa e che coindice con il valore assoluto del determinante sulle applicazioni diagonali e ortogonali. Da questo, sfruttando una decomposizione nota delle applicazioni lineari di $\R^n$, seguirà che coincide con il valore assoluto del determinante su ogni applicazione lineare e questo è proprio quanto richiesto dalla tesi. 
	
	Fissate $A,B\in \linR$ applicazioni lineari ed $E\in\M_n$ un misurabili non trascurabile, applicando unicamente l'\cref{eq:DefQuasiDet} risulta
	\begin{align*}
		c(AB)m_n(E)&=\mu_{AB}(E)=m_n(AB(E))=m_n(A(B(E))\\
		&=\mu_A(B(E))=c(A)m_n(B(E))=c(A)\mu_B(E)=c(A)c(B)m_n(E)\virgola
	\end{align*}
	da cui si ricava $c(AB)=c(A)c(B)$ dividendo per $m_n(E)$. Perciò $c$ è moltiplicativa.
	
	Sia $D\in\linR$ un'applicazione diagonale, in particolare siano $(\lambda_i)_{1\le i\le n}$ i valori sulla diagonale.
	Allora è facile ricavare
	\begin{equation*}
		D\left(\co01\times\co01\times\dots\times\co01\right)=\co0{\lambda_1}\times\co0{\lambda_2}\times\cdots\times\co0{\lambda_n}\virgola
	\end{equation*}
	da cui, applicando ad entrambi i membri la misura di Lebesgue, si ottiene
	\begin{align*}
		c(D)&=c(D)m_n\left(\co01\times\dots\times\co01\right)=\mu_D\left(\co01\times\dots\times\co01\right)\\
		&=m_n\left(\co0{\lambda_1}\times\cdots\times\co0{\lambda_n}\right)=
		\lvert\lambda_1\rvert\cdot\lvert\lambda_2\rvert\cdots\lvert\lambda_n\rvert=\lvert\det(D)\rvert
	\end{align*}
	che equivale a dire che $c(\cdot)$ e $\lvert\det(\cdot)\rvert$ coincidono sulle matrici diagonali.
	
	Fissata un'applicazione $O\in\linR$ ortogonale, chiamando $P$ la palla unitaria di $\R^n$ è ovvio che $O(P)=P$.
	Da questo, applicando ad entrambi i membri la misura di Lebesgue, si ricava $\mu_O(P)=m_n(P)$ e, ricordando che $P$ non è trascurabile, ne discende $c(O)=1$.
	Ma $O$ è ortogonale, quindi $\det O=\pm 1$ e allora è dimostrato anche in questo caso $\lvert\det O\rvert =c(O)$.
	
	Infine, data un'applicazione lineare generica $L\in\linR$, sia $L=OS$ la sua decomposizione polare\footnote{La si ottiene notando che $LL^t$ è una matrice simmetrica definita positiva che perciò ammette una ``radice quadrata''.} dove $O$ è ortogonale e $S$ è simmetrica. Per il teorema spettrale esistono $P,D$ rispettivamente invertibile e simmetrica tali che $S=PDP^{-1}$.
	Ricordando le proprietà che rispetta la funzione $c$ e la moltiplicatività del determinante, ricaviamo
	\begin{align*}
		c(L)&=c(OPDP^{-1})=c(O)c(P)c(D)c(P^{-1})=c(O)c(D)c(P)c(P^{-1})\\
		&=c(O)c(D)c(PP^{-1})=\lvert\det O\rvert\cdot\lvert\det D\rvert=\lvert \det(OPDP^{-1})\rvert=\lvert\det L\rvert \virgola
	\end{align*}
	che è equivale a dire che $c(\cdot)$ e $\lvert\det(\cdot)\rvert$ coincidono come si voleva.
\end{proof}

\begin{corollary}\label{cor:CambioVariabileLineare}
	Fissata un'applicazione lineare $L:\R^n\to\R^n$, per ogni $E\in\M_n$ misurabile ed $f:\R^n\to\Rbar$ integrabile vale la formula per il cambio di variabile lineare
	\begin{equation*}
		\int_{L(E)}f(y)\de m_n(y) = \int_E f(L(x))\left\lvert\det L\right\rvert \de m_n(x) \punto
	\end{equation*}
\end{corollary}
\begin{proof}
	Sia $\mathcal F\subseteq \L(\R^n,\M_n,m_n)$ l'insieme delle funzioni che rispettano l'enunciato.
	
	Assumendo che $f$ sia l'indicatrice di un insieme $A\in\M_n$, applicando la \cref{prop:MisuraImmagineLineare}, otteniamo
	\begin{multline*}
		\int_{L(E)}f(y)\de m_n(y)=m_n\left( A\cap L(E) \right)=m_n\left(L(L^{-1}(A)\cap E)\right)\\
		=\left\lvert\det L\right\rvert m_n\left(L^{-1}(A)\cap E\right)=\int_E f(L(x))\left\lvert\det L\right\rvert \de m_n(x) \virgola
	\end{multline*}
	cioè le indicatrici appartengono a $\mathcal F$.
	
	Per la linearità dell'operatore integrale è chiaro che l'insieme delle funzioni che rispettano è uno spazio vettoriale ed inoltre per il \cref{thm:BeppoLevi} è ovvio che $\mathcal F$ è chiuso per convergenza monotona (ammesso di rimanere nelle integrabili).
	
	Unendo quanto detto, abbiamo tutte le ipotesi per applicare il \cref{thm:ChiusuraMonotonaFunzioni}\footnote{Bisogna notare che qui sfruttiamo una versione leggermente diversa dell'enunciato, la cui dimostrazione è analoga. Infatti noi ci interessiamo ad ottenere che $\mathcal F$ coincida con le funzioni integrabili, non con tutte le misurabili. 
	Perciò l'enunciato del teorema utilizzato andrebbe modificato aggiungendo il fatto che la convergenza monotona sia ristretta nelle integrabili.} e ottenerne quindi che $\mathcal F=\L(\R^n,\M_n,m_n)$, cioè la tesi.
\end{proof}

I fatti seguenti sono tutti mirati a fornire basi solide ai vari procedimenti di limite necessari per ottenere la formula per il cambio di variabile.
%TODO: aldilà o al di là
In particolare la continuità $L^1$, così è nota in letteratura la \cref{prop:ContinuitaL1}, oltre ad essere il fondamento di tutti gli altri lemmi è importante anche aldilà di questa dimostrazione. È infatti un fatto assolutamente non ovvio che rende possibile ottenere tutti i risultati riguardo la convoluzione integrale.

Bisogna porre un accento sul fatto che la dimostrazione della continuità $L^1$ ricalca quella che vuole essere la struttura di tutte le dimostrazioni fondazionali della teoria della misura: si inizia dimostrando l'enunciato per insiemi nel \semiring{} per poi concludere che vale per ogni misurabile attraverso l'applicazione di chiusure monotone.



\begin{lemma}\label{lemma:ContinuitaL1Semianello}
	Dato $T=\co{a_1}{b_1}\times\cdots\times\co{a_n}{b_n}\in\S_n$ e $c=(c_1,\dots,c_n)\in\R^n$, vale la stima
	\begin{equation*}
		\LNorm{\chi_T({}\cdot{}+c)-\chi_T({}\cdot{})}\le 2m_n(T)\sum_{i=1}^n \frac{\lvert c_j\rvert }{b_i-a_i} \punto
	\end{equation*}
\end{lemma}
\begin{proof}
	Definiamo la funzione misurabile $s_c:\R^n\to\R$ come $s_c(x)=\left\lvert\chi_T(x+c)-\chi_T(x)\right\rvert$.
	
	Sfruttando delle identità insiemistiche e la definizione delle funzioni caratteristiche, è facile verificare che
	\begin{equation*}
		s_c(x)=\chi_{T\setminus\left(T-c\right)}(x)+\chi_{\left(T-c\right)\setminus T}(x)\punto
	\end{equation*}
	Inoltre i due insiemi $T\setminus\left(T-c\right)$ e $\left(T-c\right)\setminus T$ hanno la stessa misura in virtù della \cref{prop:LebesgueProprietaIsometria}, in quanto si può ottenere l'uno dall'altro, a meno del bordo che è però trascurabile, con un'isometria.
	Unendo quanto detto e ricordando che l'integrale di una caratteristica è la misura dell'insieme, è evidente la validità della formula
	\begin{equation} \label{eq:IdentitaIntegraleContinuitaL1}
		\LNorm{s_c}=2m_n\left(T\setminus\left(T-c\right)\right)\punto
	\end{equation}
	
	Vogliamo quindi stimare la misura di $T\setminus\left(T-c\right)$. 
	Se $x=(x_1,\dots,x_n)\in\R^n$ appartiene a $T\setminus\left(T-c\right)$ allora $x\in T$ e $x\not\in T-c$. Ma tali appartenenze, passando in coordinate, divengono il seguente sistema:
	\begin{equation}\label{eq:SistemaSemianelloContinuitaL1}
		\begin{cases}
			\forall i\in\{1,\dots,n\}: &a_i\le x_i<b_i \virgola\\
			\exists j\in\{1,\dots,n\}: &x_j<a_j-c_j \vee b_j-c_j\le x_j \punto
		\end{cases}
	\end{equation}
	Definiamo quindi $T_j$, con $j\in\{1,\dots,n\}$, come l'insieme dei punti che rispettano il sistema dove la proprietà nella seconda riga è rispettata da $j$.
	Allora, per la subadditività della misura di Lebesgue, vale
	\begin{equation}\label{eq:StimaInsiemeConPezzettiContinuitaL1}
		T\setminus\left(T-c\right)\subseteq \bigcup_{j=1}^n T_j \implies m_n\left(T\setminus\left(T-c\right)\right)\le \sum_{j=1}^n m_n(T_j) \punto
	\end{equation}
	Preso $x\in T_j$ se $c_j\ge 0$, come conseguenza dell'\cref{eq:SistemaSemianelloContinuitaL1}, vale $b_j-c_j\le x_j<b_j$; mentre se $c_j<0$ risulta $a_j\le x_j<a_j-c_j$. Sia allora $I$ l'intervallo $\co{b_j-c_j}{b_j}$, nel caso in cui $c_j\ge 0$, e $\co{a_j}{a_j-c_j}$ altrimenti. In entrambi i casi la lunghezza di $I$ risulta essere $\lvert c_j \rvert$.
	Per quanto detto, ricordando ancora l'\cref{eq:SistemaSemianelloContinuitaL1} è facile accorgersi che vale
	\begin{equation*}
		T_j\subseteq \co{a_1}{b_1}\times\cdots\times\co{a_{j-1}}{b_{j-1}}\times I\times\co{a_{j+1}}{b_{j+1}}\times\cdots\times \co{a_n}{b_n}
	\end{equation*}
	e da questa, per la definizione della misura di Lebesgue su $\S_n$ è ovvio ricavarne
	\begin{equation}\label{eq:StimaFinalePezzettoContinuitaL1}
		m_n(T_j)\le \frac{\lvert c_j\rvert }{b_j-a_j} \prod_{i=1}^n (b_i-a_i)=\frac{\lvert c_j\rvert }{b_j-a_j} m_n(T) \punto
	\end{equation}
	
	Unendo le \cref{eq:IdentitaIntegraleContinuitaL1,eq:StimaInsiemeConPezzettiContinuitaL1,eq:StimaFinalePezzettoContinuitaL1} arriviamo finalmente ad avere
	\begin{equation*}
		\LNorm{s_c}\le 2m_n(T)\sum_{i=1}^n \frac{\lvert c_j\rvert }{b_i-a_i}\virgola
	\end{equation*}
	che implica che la tesi per definizione di $s_c$.
\end{proof}


\begin{proposition}[Continuità $L^1$]\label{prop:ContinuitaL1}
	Data una funzione $f:\R^n\to\Rbar$ integrabile nella misura di Lebesgue, le funzioni $f({}\cdot{}+c)$ convergono, per $c$ che tende a $0$, in norma $L^1$ ad $f$.
\end{proposition}
\begin{proof}
	Sia $\mathcal F\subseteq \L(\R^n,\M_n,m_n)$ l'insieme delle funzioni integrabili per cui l'enunciato è vero.
	Chiamiamo poi, per $c\in\R^n$, $\tau_c:\R^n\to\R^n$ la traslazione $\tau_c(x)=x+c$.
	
	Dimostreremo che le funzioni indicatrici di $\S_n$ appartengono ad $\mathcal F$, poi che $\mathcal F$ è uno spazio vettoriale chiuso per convergenga puntuale ed infine che se due funzioni coincidono quasi ovunque ed una appartiene ad $\mathcal F$ allora anche l'altra vi appartiene. 
	L'unione di questi fatti sarà sufficiente a mostrare che $\mathcal F=\L(\R^n,\M_n,m_n)$ dimostrando così la tesi.
	
	Il \cref{lemma:ContinuitaL1Semianello} dimostra banalmente, considerando il limite per $c\to 0$, che le indicatrici degli insiemi in $\S_n$ appartengono ad $\mathcal F$.
	
	Il fatto che $\mathcal F$ sia uno spazio vettoriale discende direttamente dalla linearità dell'integrale e dal fatto che il limite di una somma è la somma dei limiti.
	
	Sia $(f_n)_{n\in\N}$ una successione di funzioni integrabili che appartengono ad $\mathcal F$ e sia $f$ il limite puntuale di tale successione, che assumiamo essere integrabile.
	Per il \cref{thm:ConvergenzaDominata}, le cui ipotesi sono facilmente verificate poichè $f$ è integrabile, le $f_n$ convergono ad $f$ anche in norma $L^1$ e perciò, per ogni $\epsilon>0$, esiste $m\in\N$ tale che $\LNorm{f_m-f}<\epsilon$. 
	Inoltre, una facile conseguenza dell'invarianza per traslazione della misura di Lebesgue è 
	\begin{equation*}
		\LNorm{f_m\circ\tau_c-f\circ\tau_c}=\LNorm{f_m-f} \virgola
	\end{equation*}
	da cui, poichè $\LNorm{{}\cdot{}}$ rispetta la triangolare come mostrato nella \cref{prop:L1VettorialeConSeminorma}, ne otteniamo
	\begin{equation*}
		\LNorm{f\circ\tau_c-f}\le \LNorm{f_m\circ\tau_c-f\circ\tau_c}+\LNorm{f_m-f}+\LNorm{f_m\circ\tau_c-f_m} \le 2\epsilon+\LNorm{f_m\circ\tau_c-f_m}\virgola
	\end{equation*}
	che passando al massimo limite entrambi i membri e ricordando che $f_n\in\mathcal F$, implica la disuguaglianza
	\begin{equation*}
		\limsup_{c\to 0} \LNorm{f\circ\tau_c-f} \le 2\epsilon \punto
	\end{equation*}
	Ma quest'ultima disuguaglianza vale per ogni $\epsilon>0$ e ciò implica ovviamente che anche $f\in\mathcal F$.
	
	Date $f,g:\R^n\to\Rbar$ funzioni integrabili equivalenti in $L^1$, con $f\in\mathcal F$, dimostriamo che $g\in\mathcal F$. 
	Anche $f\circ\tau_c$ e $g\circ\tau_c$ sono equivalenti in $L^1$ e perciò anche $f\circ\tau_c-f$ coincide quasi ovunque con $g\circ\tau_c-g$.
	Unendo quanto detto si ottiene
	\begin{equation*}
		\LNorm{g\circ\tau_c-g}=\LNorm{f\circ\tau_c-f}\to 0 \virgola
	\end{equation*}
	che implica che anche $g$ appartiene a $\mathcal F$.
	
	Sia $\mathcal E$ la famiglia degli insiemi tali che le loro indicatrici appartengono ad $\mathcal F$. 
	
	Abbiamo mostrato che $\S_n\subseteq \mathcal E$. 
	Inoltre il fatto che $\mathcal F$ sia uno spazio vettoriale implica che $\mathcal E$ sia stabile per unione disgiunta e, essendo $\mathcal F$ chiuso per convergenza puntuale nelle funzioni integrabile, $\mathcal E$ risulta stabile per unione e intersezione monotona numerabile negli insiemi finiti.
	Quindi risultano verificate le ipotesi del \cref{cor:ChiusuraMonotonaInsiemiFiniti} e ne deduciamo che $\mathcal E$ contiene ogni misurabile finito a meno di un trascurabile. Ma il fatto che se $\mathcal F$ contiene una funzione allora contiene ogni funzione che coincide quasi ovunque con essa implica che $\mathcal E$ se contiene un insieme allora contiene anche quelli che coincidono con lui a meno di un trascurabile.
	Quindi questo implica $\mathcal E$ contiene tutti gli insiemi misurabili finiti e perciò $\mathcal F$ contiene le loro indicatrici.
	
	Concludiamo quindi applicando il \cref{thm:ChiusuraMonotonaFunzioni}\footnote{Di nuovo sfruttiamo l'enunciato nella forma con insiemi finiti e funzioni integrabili.}, di cui abbiamo controllato essere verificate tutte le ipotesi, e otteniamo che $\mathcal F$ contiene tutte le funzioni integrabili.
\end{proof}

\begin{definition}[Media integrale] \label{def:MediaIntegrale}
	Definiamo l'operatore $\aint$, chiamato media integrale, come
	\begin{equation*}
		\aint_E f(x)\de\mu(x)=\frac 1{\mu(E)}\int_E f(x)\de\mu(x)\virgola
	\end{equation*}
	dove $E$ è un insieme misurabile nello spazio di misura $(X,\A,\mu)$ e $f:X\to\Rbar$ è una funzione integrabile nel medesimo spazio di misura.
\end{definition}
\begin{remark}\label{nota:ProprietaMediaIntegrale}
	Fissato uno spazio di misura $(X,\A,\mu)$ e un insieme misurabile $E\in\A$ l'operatore di media integrale rispetta le seguenti proprietà:
	\begin{itemize}
		\item Date $f,g:X\to\Rbar$ funzioni misurabili e $a,b\in\R$ vale
		\begin{equation*}
			a\aint_E f(x)\de\mu(x)+b\aint_E g(x)\de\mu(x)=\aint_E af(x)+bg(x)\de\mu(x)
		\end{equation*}
		e se $f\le g$ allora
		\begin{equation*}
			\aint_E f(x)\de\mu(x)\le \aint_E g(x)\de\mu(x)\virgola
		\end{equation*}
		cioè la media integrale è lineare e monotona.
		\item Fissato $c\in\R$ vale
		\begin{equation*}
			\aint_E c\de\mu(x)=c\punto
		\end{equation*}
		\item Per ogni $f:X\to\Rbar$ misurabile valgono le stime
		\begin{equation*}
			\inf_{x\in E}f(x)\le \aint_E f(x)\de\mu(x) \le \sup_{x\in E}f(x)\punto
		\end{equation*}
	\end{itemize}
\end{remark}
\begin{proof}
	La linearità, la monotonia e l'identità riguardante la media integrale delle costanti discendono dalla \cref{def:MediaIntegrale} applicando le prime proprietà dell'integrale di Lebesgue.
	
	Per quanto riguarda l'ultima stima, è sufficiente applicare le proprietà già dimostrate per ottenere
	\begin{equation*}
		\inf_{x\in E}f(x) = \aint_E \inf_{x\in E} f(x)\de\mu(x)\le \aint_E f(x)\de\mu(x) \le \aint_E \sup_{x\in E} f(x)\de\mu(x) =\sup_{x\in E}f(x)\punto
	\end{equation*}
\end{proof}


\begin{lemma}\label{lemma:ContinuitaL1Palle}
	Fissato $E\in\M_n$ un insieme misurabile e $f:E\to\Rbar$ una funzione misurabile, definiamo, per ogni $r>0$, la funzione $f_r:E\to\R$ come
	\begin{equation*}
		\forall x\in E:\ f_r(x)=\frac{\int_{B_r(x)\cap E}f(t)\de t}{m_n\left(B_r(x)\right)}\virgola
	\end{equation*}
	dove $B_r(x)$ è la palla aperta di raggio $r$ e centro $x$.
	Allora le funzioni $f_r$, che risultano essere continue, convergono, per $r$ che tende a $0$, in norma $L^1$ alla funzione $f$.
\end{lemma}
\begin{proof}
	Innanzitutto, per comodità di notazione, allarghiamo il dominio della $f$ ponendo $f(E^{\mathsf c})=\{0\}$, così facendo otteniamo
	\begin{equation*}
		\forall r>0,\ x\in E:\ f_r(x)=\frac{\int_{B_r(x)}f(t)\de t}{m_n(B_r(x))}=\aint_{B_r(x)} f(t)\de t \punto
	\end{equation*}

	Vale facilmente, per ogni $x,y\in E$, l'identità
	\begin{equation*}
		\lvert f_r(x)-f_r(y)\rvert %= \left\lvert \aint_{B_r(x)}f(t)\de t-\aint_{B_r(y)} f(t)\de t\right\rvert 
		\le \aint_{B_r(x)}\lvert f(t)-f(t+(y-x))\rvert \de t \le \frac{ \LNorm{f({}\cdot{})-f({}\cdot{}+(y-x)} }{ m_n(B_r(0)) } \virgola
	\end{equation*}
	che grazie alla \cref{prop:ContinuitaL1} implica la continuità delle funzioni $f_r$.
	
	Ora verifichiamo, per poi poter applicare il teorema di Tonelli, che la funzione $g:\R^n\times\R^n\to\Rbar$ definita come $g(x,t)=f(x+t)-f(x)$ è misurabile nello spazio prodotto $\R^n\times\R^n$. 
	La funzione $u(x,t)=f(x)$ è misurabile nel prodotto grazie alla \cref{nota:FunzioniMisProdotto} e allora anche $u(x+t,t)$ lo è in quanto composizione di una misurabile con una lineare invertibile. Unendo quanto detto si ha, come cercato, che $g(x,t)=u(x+t,t)-u(x,t)$ è misurabile in quanto differenza di misurabili.
	
	Calcoliamo ora la norma $L^1$ della differenza tra $f$ e $f_r$:
	\begin{multline}\label{eq:ContinuitaPalleDis}
		\LNorm{f-f_r}=\int_E \left\lvert f(x)-\aint_{B_r(x)}f(t)\de t\right\rvert\de x\\
		=\int_E\left\lvert\aint_{B_r(x)}f(x)-f(t)\de x\right\rvert\de t\le \int_E\aint_{B_r(x)}\left\lvert f(x)-f(t)\right\rvert\de x\de t\\
		=\int_E\aint_{B_r(0)}\left\lvert f(x)-f(x+h)\right\rvert\de h\de x
		=\aint_{B_r(0)}\int_E \left\lvert f(x)-f(x+h)\right\rvert\de x\de h\\
		= \aint_{B_r(0)}\LNorm{f(\cdot)-f(\cdot+h)}\de h\le \sup_{h\in B_r(0)} \LNorm{f(\cdot)-f(\cdot+h)}\virgola
	\end{multline}
	dove nei vari passaggi abbiamo sfruttato il \cref{thm:TonelliCompleto}, che si può applicare ricordando la \cref{prop:MisuraProdottoEuclidea}, e la \cref{nota:ProprietaMediaIntegrale}.
	
	Per la \cref{prop:ContinuitaL1}, il valore $\sup_{h\in B_r(0)} \LNorm{f(\cdot)-f(\cdot+h)}$ tende a $0$ per $r\to 0$ e applicando ciò nell'\cref{eq:ContinuitaPalleDis} si ricava facilmente la tesi.
\end{proof}

Terminati i lemmi preliminari, mostriamo ora la formula per il cambio di variabile. La dimostrazione procede in tre passi. 
In un primo passo, cioè nel \cref{lemma:LimiteDeterminante}, proviamo solo una disuguaglianza, che poi mostreremo essere un'uguaglianza nel \cref{lemma:MisuraImmagine} e che sarà l'equivalente della \cref{prop:MisuraImmagineLineare} ma nel caso generale e non in quello lineare. Infine il terzo passo semplicente trasforma l'enunciato sulla misura di un insieme nella formula per il cambio di variabile.

Per passare da disuguaglianza a disuguaglianza sfrutteremo come strumento fondamentale il teorema di Radon-Nikodym, che asserisce fondamentalmente che ogni misura ammette una ``densità'' cioè una funzione tale che il suo integrale su un insieme, rispetto ad un'altra misura, coincide con la misura dell'insieme.
Tale teorema lo lasciamo indimostrato fondamentalmente per la non banalità della dimostrazione e poichè in queste dispense lo useremo solo come strumento tecnico, senza approfondire l'argomento.

\begin{lemma}\label{lemma:LimiteDeterminante}
	Fissato $\Omega\subseteq\R^n$ un aperto, sia $\varphi:\Omega\to\R^n$ una funzione differenziabile con continuità.
	Allora vale la seguente disuguaglianza:
	\begin{equation*}
		\forall x\in\Omega:\ \limsup_{r\to 0} \frac{ m_n\left(\varphi\left(B_r(x)\right)\right)} {m_n\left(B_r(x)\right)}\le \lvert\det D\varphi(x)\rvert \virgola 
	\end{equation*}
	dove $B_r(x)$ è la palla aperta di raggio $r$ e centro $x$ e $D\varphi$ è la matrice Jacobiana della funzione $\varphi$.
\end{lemma}
\begin{proof}
	Fissiamo $\bar x\in\Omega$ e chiamiamo $A=D \varphi(\bar x)$.
	Allora, per definizione di differenziale, risulta vero che
	\begin{equation*}
		\varphi(\bar x+h)=\varphi(\bar x)+Ah+\smallO(h)=\varphi(\bar x)+A(h+\smallO(h))\in \varphi(\bar x)+A\left(B_{|h|+\smallO(h)}(0)\right) \virgola
	\end{equation*}
	dove abbiamo implicitamente sfruttato che moltiplicare per la norma degli operatori di $A$ non cambia il fatto che una funzione sia $\smallO$-piccolo di un'altra.
	
	Da quanto appena detto discende facilmente che, per ogni $r>0$, vale
	\begin{equation}\label{eq:ContenimentoPallaLineare}
		\varphi(B_r(\bar x))\subseteq \varphi(\bar x)+A\left(B_{r+\smallO(r)}(0)\right).
	\end{equation}
	Essendo però $\varphi$ differenziabile con continuità, è anche localmente Lipschitziana e perciò, purchè $r$ sia sufficientemente piccolo, la funzione è Lipschitziana in $B_r(\bar x)$ e quindi, per la \cref{prop:LipschitzTengonoMisurabili}, l'insieme $\varphi(B_r(\bar x))$ è misurabile.
	Allora possiamo applicare la misura di Lebesgue ad entrambi i membri del contenimento mostrato nell'\cref{eq:ContenimentoPallaLineare} ottenendo
	\begin{equation}\label{eq:StimaImmaginePalla}
		m_n\left( \varphi(B_r(\bar x)) \right)\le m_n\left(\varphi(\bar x)+A\left(B_{r+\smallO(r)}(0)\right)\right)
		=\lvert \det A\rvert m_n\left(B_{r+\smallO(r)}(\bar x)\right)\virgola
	\end{equation}
	dove i passaggi sono giustificati dall'invarianza per traslazione della misura di Lebesgue e dalla \cref{prop:MisuraImmagineLineare}.
	Ma ora, grazie alla $n$-omogeneità della misura $m_n$, ricaviamo
	\begin{equation*} 
		m_n\left(B_{r+\smallO(r)}(\bar x)\right)=\left(1+\smallO(r)\right)^n m_n\left( B_r(\bar x)\right)
	\end{equation*}
	e perciò sostituendo questa nell'\cref{eq:StimaImmaginePalla} arriviamo a
	\begin{equation*}
		m_n\left( \varphi(B_r(\bar x)) \right)\le \left(1+\smallO(r)\right)^n m_n\left( B_r(\bar x)\right)\lvert \det A\rvert \virgola
	\end{equation*}
	che implica banalmente la tesi.
\end{proof}

\begin{definition}\label{def:AssolutamenteContinua}
	Date due misure $\mu:\A\to\Rpiu,\nu:\A\to\Rpiu$ sullo stesso spazio misurabile $(X,\A)$, diciamo che $\nu$ è assolutamente continua rispetto a $\mu$, indicandolo con $\nu\ll\mu$, se, per ogni $E\in\A$, se $\mu(E)=0$ allora $\nu(E)=0$, cioè se tutti gli insiemi trascurabili per $\mu$ sono trascurabili anche per $\nu$.
\end{definition}

\begin{theorem}[Radon-Nikodym] \label{thm:RadonNikodym}
	Dato uno spazio di misura $(X,\A,\mu)$ \sigfin[o], se la misura $\nu:\A\to\Rpiu$ è assolutamente continua rispetto a $\mu$, allora esiste una funzione $\rho:X\to\Rpiu$ misurabile rispetto a $\mu$, tale che
	\begin{equation*}
		\forall A\in\A:\ \nu(A)=\int_A \rho(x)\de\mu(x)\punto
	\end{equation*}
\end{theorem}

\begin{lemma}\label{lemma:MisuraImmagine}
	Fissati $\Omega,\Omega'\subseteq\R^n$ due aperti, sia $\varphi:\Omega\to\Omega'$ un diffeomorfismo $C^1$.
	Allora, per ogni $E\subseteq \Omega$ misurabile vale
	\begin{equation*}
		m_n(\varphi(E))=\int_E\left\lvert \det D\varphi(x) \right\rvert \de m_n(x)\virgola
	\end{equation*}
	dove $D\varphi$ è la matrice Jacobiana del diffeomorfismo $\varphi$.
\end{lemma}
\begin{proof}
	Innanzitutto scriviamo $\Omega=\bigcup_{n\in\N} K_n$ dove $(K_n)_{n\in\N}$ è una successione di compatti. 
	Allora, dato un insieme $N\subset\Omega$ trascurabile, vale
	\begin{equation*}
		\varphi(N)=\bigcup_{n\in\N} \varphi\left(N\cap K_n\right)\virgola
	\end{equation*}
	ma nei compatti $K_n$ la funzione $\varphi$, essendo $C^1$, è Lipschitziana e quindi per la \cref{prop:LipschitzTengonoMisurabili} manda trascurabili in trascurabili e perciò $\varphi\left(N\cap K_n\right)$ è trascurabile e da questo ne segue che $\varphi(N)$ è trascurabile in quanto unione numerabile di trascurabili.
	Riassumendo, la funzione $\varphi$ manda trascurabili in trascurabili e per la \cref{prop:ContinueSpecialiTengonoMisurabili} questo implica che manda misurabili in misurabili.
	
	Allora per la \cref{prop:BigettivaInduceMisura} la funzione di insiemi $\nu=m_n\circ \varphi$ è una misura, che risulta essere assolutamente continua rispetto alla misura di Lebesgue poichè $\varphi$ manda trascurabili in trascurabili.
	Perciò, per il \cref{thm:RadonNikodym}, esiste $\rho:\Omega\to\Rpiu$ integrabile tale che
	\begin{equation*}
		\forall E\in M_n\text{ t.c. } E\subseteq \Omega:\ m_n(\varphi(E))=\int_E \rho \de x \punto
	\end{equation*}
	Con una dimostrazione del tutto analoga a quella del \cref{cor:CambioVariabileLineare} questo implica che vale anche la formula
	\begin{equation}\label{eq:CambioRadonNikodym}
		\forall E\in M_n\text{ t.c. } E\subseteq\Omega:\ \int_{\varphi(E)} f(y)\de y = \int_{E} f(\varphi(x))\rho(x) \de x
	\end{equation}
	per ogni $f:\Omega\to\Rbar$ integrabile.
	
	A questo punto, usando la medesima notazione dell'enunciato del \cref{lemma:ContinuitaL1Palle}, notiamo che
	\begin{equation*}
		\rho_r(x)=\aint_{B_r(x)}\rho(t)\de t=\frac{\int_{B_r(x)} \rho(t)\de t}{m_n(B_r(0))}=\frac{m_n(\varphi(B_r(x)))}{m_n(B_r(0))}\virgola
	\end{equation*}
	e quindi, applicando proprio il \cref{lemma:ContinuitaL1Palle}, otteniamo che $\frac{m_n(\varphi(B_r(x)))}{m_n(B_r(0))}$ converge a $\rho$ in norma $L^1$ per $r$ che tende a $0$.
	Ricordando però la \cref{prop:L1ImplicaSottosuccessioneQuasiOvunque}, abbiamo anche che $\frac{m_n(\varphi(B_r(x)))}{m_n(B_r(0))}$ converge puntualmente quasi ovunque a $\rho$. Perciò, notando che se esiste il limite coincide con il limite superiore, possiamo applicare il \cref{lemma:LimiteDeterminante} ed ottenere che per quasi ogni $x\in\Omega$ vale
	\begin{equation}\label{eq:DisuguaglianzaRadonNikodym}
		\rho(x)=\limsup_{r\to 0} \frac{ m_n\left(\varphi\left(B_r(x)\right)\right)} {m_n\left(B_r(x)\right)}\le \lvert\det D\varphi(x)\rvert \punto
	\end{equation}
	
	Ora, unendo le due \cref{eq:CambioRadonNikodym,eq:DisuguaglianzaRadonNikodym}, arriviamo a dire che per ogni $f:\Omega\to\Rbar$ misurabile positiva vale
	\begin{equation*}
		\forall E\in M_n\text{ t.c. } E\subseteq\Omega:\ \int_{\varphi(E)} f(y)\de y\le 
		\int_E f(\varphi(x))\left\lvert\det D\varphi(x)\right\rvert\de x \punto
	\end{equation*}
	
	Ora giungiamo alla disuguaglianza conclusiva applicando quest'ultima stima sia alla funzione $\varphi$ sia alla funzione $\varphi^{-1}$ che essendo per ipotesi anch'esso un diffeomorfismo $C^1$ rispetterà un analogo enunciato.
	Seguendo lo schema enunciato otteniamo che per ogni $E\subseteq\Omega$ misurabile vale
	\begin{multline*}
		m_n(E) =\int_{\varphi(\varphi^{-1}(E))}1\de x\le \int_{\varphi^{-1}(E)} \left\lvert\det D\varphi(x)\right\rvert\de x \\
 		\le \int_E \left\lvert\det D\varphi(\varphi^{-1}(x))\right\rvert \cdot \left\lvert\det D\varphi^{-1}(x)\right\rvert\de x =\int_E 1\de x=m_n(E) \virgola
	\end{multline*}
	dove abbiamo applicato implicitamente la formula del differenziale dell'inversa e la moltiplicatività del determinante.
	Però, poichè primo e ultimo membro coincidono, tutte le disuguaglianze della catena devono essere uguaglianze e da questo ricaviamo
	\begin{equation*}
		m_n(E)=\int_{\varphi^{-1}(E)} \left\lvert\det D\varphi(x)\right\rvert\de x \virgola
	\end{equation*}
	che implica banalmente la tesi.
\end{proof}

\begin{theorem}\label{thm:CambioVariabile}
	Fissati $\Omega,\Omega'\subseteq\R^n$ due aperti, sia $\varphi:\Omega\to\Omega'$ un diffeomorfismo $C^1$.
	Allora, per ogni $E\subseteq \Omega$ misurabile e $f:\Omega'\to\R$ integrabile, vale
	\begin{equation*}
		\int_{\varphi(E)} f(y)\de m_n(y) = \int_{E} f(\varphi(x))\lvert \det D\varphi(x) \rvert \de m_n(x) \virgola
	\end{equation*}
	dove $D\varphi$ è la matrice Jacobiana del diffeomorfismo $\varphi$.
\end{theorem}
\begin{proof}
	La dimostrazione è completamente analoga a quella del \cref{cor:CambioVariabileLineare}, solo che al posto di sfruttare la \cref{prop:MisuraImmagineLineare} si sfrutta il \cref{lemma:MisuraImmagine}.
\end{proof}


\section{Misura su varietà differenziabili \texorpdfstring{di $\R^n$}{}}
Obiettivo di questa sezione è formalizzare i concetti di ``lunghezza'', ``superficie'', ``volume'', utilizzando gli strumenti di teoria della misura affrontati finora. 

Vogliamo introdurre innanzitutto il concetto di varietà differenziabile $k$-dimensionale di $\R^n$, che sarà la struttura che generalizza la nostra idea di ``superficie'' e su cui definiremo una misura. In particolare una varietà differenziabile sarà parametrizzata da funzioni, chiamate immersioni iniettive, che gli conferiscono le proprietà di regolarità necessarie a definire la misura.
Per fare tutto ciò sfrutteremo la topologia naturalmente indotta da $\R^n$ sui suoi sottoinsiemi.

\begin{definition}
	Una funzione $f:\Omega \subseteq \R^k\to\R^n$, con $\Omega$ aperto, si dice immersione iniettiva se è differenziabile con continuità ed è iniettiva con differenziale iniettivo in ogni punto. In tal caso si dice che $f$ è una parametrizzazione $C^1$ della superficie $k$-dimensionale $f(\Omega)$.
\end{definition}

\begin{remark}\label{nota:TopologiaIndotta}
	Dato un sottoinsieme $Y$ di uno spazio topologico $X$, esiste una naturale topologia indotta su $Y$. In particolare un sottoinsieme $U$ di $Y$ è un aperto di $Y$ nella topologia indotta se e solo esiste un aperto $V$ di $X$ tale che $U=V\cap Y$. 
	
	Si verifica facilmente che questa topologia indotta è effettivamente una topologia su $Y$.
\end{remark}

D'ora in poi, quando lavoreremo su un sottoinsieme di $\R^n$, sottoinderemo di star considerando la topologia indotta su quel sottoinsieme, come definita nella \cref{nota:TopologiaIndotta}.

\begin{definition}\label{def:BorelianiSottoinsieme}
	Ricalcando la \cref{def:Boreliani}, definiamo Boreliani di un sottoinsieme $X$ di $\R^n$ come la \sigalg\ generata dagli aperti di $X$ nella topologia indotta e li indichiamo con $\Borel(X)$.
\end{definition}

\begin{definition}
	Un sottoinsieme $\Sigma$ di $\R^n$ è una varietà differenziabile $k$-dimensionale di $\R^n$ se per ogni $x\in\Sigma$ esiste un intorno aperto di $x$ (nella topologia indotta) in $\Sigma$ che ammette una parametrizzazione $C^1$ definita su un aperto di $\R^k$.
\end{definition}

Di seguito enunceremo, fra gli altri risultati, dei fatti di topologia di base non del tutto banali, che però non ci soffermeremo a dimostrare in quanto non strettamente pertinenti alla trattazione. Tali fatti ci serviranno a dimostrare alcuni lemmi utili ad arrivare alla definizione di misura su una varietà differenziabile di $\R^n$.

\begin{definition}
	Uno spazio topologico si dice spazio di Lindelöf se ogni suo ricoprimento aperto ammette un sottoricoprimento numerabile. 
\end{definition}

\begin{definition}
	Uno spazio topologico si dice separabile se contiene un sottoinsieme numerabile denso.
\end{definition}

\begin{proposition}\label{prop:SottoinsiemeSeparabile}
	Ogni sottoinsieme di uno spazio metrico separabile (con topologia indotta dalla metrica) è separabile (nella sua topologia indotta).
\end{proposition}

\begin{proposition}\label{prop:SeparabileLindelofMetrico}
	Per uno spazio metrico sono equivalenti essere separabile ed essere uno spazio di Lindelöf.
\end{proposition}

\begin{lemma}\label{lemma:SottovarietaUnioneNumerabile}
	Data $\Sigma$ varietà differenziabile $k$-dimensionale di $\R^n$, esiste un ricoprimento numerabile  $\{\Sigma_i\}_{i\in\N}$ di $\Sigma$ tale che, per ogni $i\in\N$, $\Sigma_i$ è aperto nella topologia di $\Sigma$ e ammette una parametrizzazione $C^1$ definita su un aperto $\Omega_i$ di $\R^k$.
\end{lemma}
\begin{proof}
	La varietà $\Sigma$ è separabile per la \cref{prop:SottoinsiemeSeparabile}, in quanto sottoinsieme di $\R^n$ che è uno spazio metrico separabile, e di conseguenza è anche uno spazio di Lindelöf con la sua topologia indotta per la \cref{prop:SeparabileLindelofMetrico}.
	
	Per la definizione stessa di varietà differenziabile $k$-dimensionale, per ogni punto $P\in\Sigma$ esiste un suo intorno aperto $\Sigma_P$ (nella topologia di $\Sigma$) che ammette parametrizzazione $C^1$. Abbiamo quindi che $\{\Sigma_P\}_{P\in\Sigma}$ è un ricoprimento aperto di $\Sigma$ e perciò, poichè $\Sigma$ è uno spazio di Lindelöf, possiamo estrarne un ricoprimento numerabile, da cui la tesi.
\end{proof}

\begin{proposition}\label{prop:ApertoUnioneCompatti}
	Ogni aperto $A$ di $\R^n$ si può scrivere come unione numerabile crescente di compatti.
\end{proposition}

\begin{lemma}\label{lemma:ImmagineApertiContinua}
	Sia $f:\R^k\to\R^n$ una funzione continua, allora $f$ manda aperti di $\R^k$ in Boreliani di $\R^n$.
\end{lemma}
\begin{proof}
	Dato $A$ aperto di $\R^k$, per la \cref{prop:ApertoUnioneCompatti} esiste una successione numerabile crescente di compatti $\{K_n\}_{n\in\N}$ tale che $A=\bigcup_{n\in\N}K_n$. Perciò abbiamo che
	\begin{equation*}
		f(A)=f\left(\bigcup_{n\in\N}K_n\right)=\bigcup_{n\in\N}f(K_n)\punto
	\end{equation*}
	L'immagine tramite una funzione continua di un compatto è compatta, quindi $f(K_n)$ è compatto per ogni $n\in\N$ e di conseguenza è anche un Boreliano di $\R^n$.
	
	Abbiamo ottenuto perciò che $f(A)$ si può scrivere come unione numerabile di Boreliani, quindi è anch'esso un Boreliano, come volevamo dimostrare.	
\end{proof}

\begin{lemma}\label{lemma:ContinuaImplicaBoreliana}
	Sia $f:X\subseteq\R^k\to Y\subseteq\R^n$ una funzione continua, allora $f$ è Boreliana, cioè controimmagine di Boreliani è Boreliana, dove si intendono come Boreliani quelli ottenuti dalla topologia indotta, come nella \cref{def:BorelianiSottoinsieme}. 
\end{lemma}
\begin{proof}
	La dimostrazione ricalca fondamentalmente quella della \cref{prop:CounterImgMis}, sfruttando che controimmagine di aperti tramite funzioni continue è aperta.
\end{proof}

\begin{remark}\label{nota:SigmaBoreliano}
	Utilizzando la stessa notazione del \cref{lemma:SottovarietaUnioneNumerabile}, notiamo che per il \cref{lemma:ImmagineApertiContinua} l'insieme $\Sigma_i$ è un Boreliano di $\R^n$, per ogni $i\in\N$, in quanto immagine di un aperto tramite la funzione continua $f_i$. Di conseguenza abbiamo anche che $\Sigma=\cup_{i\in\N}\Sigma_i$ è un Boreliano, poichè unione numerabile di Boreliani.  
\end{remark}

\begin{remark}\label{nota:BorelianiSottovarieta}
	Per la \cref{nota:SigmaBoreliano}, abbiamo che gli aperti di $\Sigma$ sono anche Boreliani di $\R^n$, poichè sono intersezione fra aperti di $\R^n$ e $\Sigma$, che è un Boreliano di $\R^n$. Perciò in particolare otteniamo che i Boreliani di $\Sigma$ sono un sottoinsieme dei Boreliani di $\R^n$.
\end{remark}

Abbiamo finalmente tutti gli strumenti per definire la misura di Lebesgue $k$-dimensionale su una varietà differenziabile $k$-dimensionale di $\R^n$. Prima di dare la definizione vera e propria cerchiamo però di capire quale è la forma che ci aspettiamo da questa misura su delle strutture più semplici di una varietà differenziabile. A prima vista infatti la definizione di misura $k$-dimensionale su una varietà differenziabile può sembrare del tutto controintuitiva, senza aver studiato precedentemente dei casi più semplici.

Consideriamo quindi il caso di un sottospazio vettoriale $k$-dimensionale $V$ di $\R^n$. Questo è un esempio speciale di varietà differenziabile $k$-dimensionale di $\R^n$, in quando esiste una funzione lineare $L:\R^k\to \R^n$, tale che $L(\R^k)=V$, $L$ è bigettiva con la sua immagine e ha facilmente differenziale iniettivo, in quanto il differenziale di una funzione lineare è la funzione stessa, quindi $L$ è una parametrizzazione $C^1$ di $V$.

Data $L:\R^k\to\R^n$ applicazione lineare, esiste sempre la decomposizione polare della matrice $L$ che è della forma $L=U\cdot S$, con $S=(L^T\cdot L)^{\frac 12}:\R^k\to\R^k$ matrice simmetrica invertibile \footnote{Data $A$ matrice simmetrica definita positiva (che nel nostro caso è $L^T\cdot L$), esiste sempre una matrice denotata con $A^\frac 12$ che elevata al quadrato dà $A$.} e $U=L\cdot (L^T\cdot L)^{-\frac 12}:\R^k\to\R^n$ ortogonale, cioè tale che $U^T\cdot U=I$.

Quello che vogliamo da una misura su $V$ è che coincida con quella che potremmo costruire imitando quello che abbiamo fatto su $\R^k$. Infatti data una base ortonormale di $V$ nessuno ci vieta di definire l'insieme dei parallelepipedi $k$-dimensionali di $V$, come nella \cref{def:LebesgueSemiaperti}, e i loro volumi, come nella \cref{def:LebesgueElementare}, e procedere quindi alla costruzione della misura su $V$ esattamente come abbiamo fatto nella \cref{sezione:MisuraLebesgue}.
Questo equivale a dire che data $U:\R^k\to V$ isometria lineare (come quella che abbiamo ottenuto dalla decomposizione polare di $L$), vorremo che valga $\sigma(E)=m_k(U^{-1}(E))$ per ogni $E\in\Borel(V)$, dove abbiamo chiamato $\sigma$ la misura su $V$. Infatti scegliere $U:\R^k\to V$ isometria lineare equivale a prendere una base ortonormale di $V$, mentre imporre $\sigma(E)=m_k(U^{-1}(E))$ vuol dire ``copiare'' la misura di $\R^k$ su $V$.

Calcoliamo quindi a quanto equivale $m_k(U^{-1}(E))$, dato $E\in\Borel(V)$, in funzione della nostra ``immersione iniettiva'' $L$. Sfruttando la \cref{prop:MisuraImmagineLineare} e le proprietà del determinante, otteniamo
\begin{equation*}
	m_k(U^{-1}(E))=m_k(S\cdot L^{-1}(E))=\lvert\det S\rvert\cdot m_k(L^{-1}(E))=\sqrt{\det(L^T\cdot L)}\cdot m_k(L^{-1}(E))\virgola
\end{equation*}
in particolare vorremmo quindi che la misura $k$-dimensionale di $E$ sia uguale a
\begin{equation*}
	\sigma(E)=m_k(U^{-1}(E))=\int_{L^{-1}(E)}\sqrt{\det(L^T\cdot L)}\de x\punto
\end{equation*}

Dopo questa breve digressione, la definizione che stiamo per dare sembrerà meno arbitraria.

\begin{definition}\label{def:MisuraKDimensionale}
	Sia $E\in\Borel(\Sigma$), dove $\Sigma$ è una varietà differenziabile $k$-dimensionale di $\R^n$, e siano $\{\Sigma_i\}_{i\in\N}$ e $\{\Omega_i\}_{i\in\N}$ come nel \cref{lemma:SottovarietaUnioneNumerabile}. Chiamiamo inoltre, per ogni $i\in\N$, $f_i:\Omega_i\to\Sigma_i$ l'immersione iniettiva che parametrizza $\Sigma_i$.
	
	Definiamo la misura di Lebesgue $k$-dimensionale di $E$ come
	\begin{equation*}
		\sigma(E)=\sum_{i\in\N} \sigma_i(E\cap F_i)\virgola
	\end{equation*}
	dove $F_i=\Sigma_i\setminus (\cup_{j<i}\Sigma_j)$ e $\sigma_i$ è una funzione definita sui Boreliani di $\Sigma$ a valori in $\Rpiu$ tale che, dato $B\in\Borel(\Sigma)$, vale
	\begin{equation*}
		\sigma_i(B)=\int_{f_i^{-1}(B)}\sqrt{\det(\Diff f_i(x)^T\Diff f_i(x))} \de x\punto
	\end{equation*}

\end{definition}

Vogliamo ora dimostrare che quella appena data è una buona definizione, cioè non dipende dalla scelta di $\{\Sigma_i\}_{i\in\N}$ e dalle parametrizzazioni, tutte le quantità sono ben definite e $\sigma$ è veramente una misura.

\begin{lemma}\label{lemma:InvarianzaImmersione}
	Date $f:\Omega\subseteq\R^k\to M$ e $g:\Omega'\subseteq\R^k\to M$ parametrizzazioni $C^1$ della superficie $k$-dimensionale $M$, per ogni $E\in\Borel(M)$ sono definiti i seguenti integrali e vale che 
	\begin{equation*}
		\int_{g^{-1}(E)}\sqrt{\det(\Diff g(x)^T\Diff g(x))}\de x=\int_{f^{-1}(E)}\sqrt{\det(\Diff f(x)^T\Diff f(x))}\de x\punto
	\end{equation*}
\end{lemma}
\begin{proof}
	Innanzitutto notiamo che, dato $E\in\Borel(M)$, i due integrali sono definiti, in quanto $\Diff f(x)$ e $\Diff g(x)$ sono continue e $f^{-1}(E)$ e $g^{-1}(E)$ sono Boreliani di $\R^k$. Infatti per la \cref{nota:BorelianiSottovarieta} $E$ è un Boreliano di $\R^n$, quindi $f^{-1}(E)$ e $g^{-1}(E)$ sono controimmagini di un Boreliano tramite funzioni continue e di conseguenza sono Boreliani di $\R^k$ per il \cref{lemma:ContinuaImplicaBoreliana}.

	Sia $\varphi=f^{-1}\circ g$, allora $\varphi$ è un diffeomorfismo $C^1$ fra gli aperti $\Omega$ e $\Omega'$. Quindi per il \cref{thm:CambioVariabile} otteniamo
	\begin{multline*}
		\int_{f^{-1}(E)}\sqrt{\det(\Diff f(x)^T\Diff f(x))}\de x=\int_{\varphi\circ g^{-1}(E)}\sqrt{\det(\Diff f(x)^T\Diff f(x))}\de x=\\
		=\int_{g^{-1}(E)}\sqrt{\det(\Diff f(\varphi(x))^T\Diff f(\varphi(x)))}\ \lvert\det \Diff \varphi(x)\rvert\de x=\int_{g^{-1}(E)}\sqrt{\det(\Diff g(x)^T\Diff g(x))}\de x\virgola
	\end{multline*}
	dove abbiamo usato che se $g=f\circ \varphi$, allora $\Diff g(x)=\Diff f(\varphi(x))\cdot \Diff\varphi(x)$.
\end{proof}

\begin{remark}
	Notiamo che la funzione $\sigma_i:\Borel(\Sigma)\to\Rpiu$, introdotta nella \cref{def:MisuraKDimensionale}, è una misura sui Boreliani di $\Sigma$. DA SISTEMARE
\end{remark}
\begin{proof}
	Sia $f_i:\Omega_i\to\Sigma_i\subseteq\Sigma$ l'immersione iniettiva con cui definisco $\sigma_i$. La funzione $f_i$ è differenziabile con continuità
\end{proof}

\begin{proposition}\label{prop:DefMisuraKDimBuona}
	La \cref{def:MisuraKDimensionale} è una buona definizione e non dipende dalla scelta del ricoprimento $\{\Sigma_i\}_{i\in\N}$ e delle immersioni iniettive $\{f_i\}_{i\in\N}$.
\end{proposition}
\begin{proof}
	Innanzitutto notiamo che la definizione è ben posta, in quanto tutti gli integrali sono ben definiti. Infatti $\Diff f_i(x)$ è continua per ogni $i\in\N$ e inoltre è facile verificare che $E\cap F_i$ è un Boreliano di $\R^n$, utilizzando la \cref{nota:BorelianiSottovarieta}, in quanto intersezione di Boreliani. Quindi $f_i^{-1}(E\cap F_i)$ è un Boreliano di $\R^k$ per il \cref{lemma:ContinuaImplicaBoreliana} (analogamente a quanto detto nel \cref{lemma:InvarianzaImmersione}).

	Consideriamo ora due diversi ricoprimenti $\{\Sigma_i\}_{i\in\N}$ e $\{\Sigma_i'\}_{i\in\N}$ di $\Sigma$, tali che per ogni $i\in\N$ $\Sigma_i$ e $\Sigma_i'$ sono aperti nella topologia di $\Sigma$ e ammettono rispettivamente parametrizzazioni $f_i$ e $g_i$. Vogliamo dimostrare che le misure $\sigma$ e $\sigma'$ indotte dai due ricoprimenti coincidono.
	
	Chiamiamo $F_i=\Sigma_i\setminus (\cup_{j<i}\Sigma_j)$ ed $F_i'=\Sigma_i'\setminus (\cup_{j<i}\Sigma_j')$, come nella \cref{def:MisuraKDimensionale}. Sia poi $B_{ij}=F_i \cap F_j'$, allora $\{B_{ij}\}_{i,j\in\N}$ è una partizione di $\Sigma$ con elementi disgiunti (è facile verificare che $B_{ij}\cap B_{i'j'}=\emptyset$ per ogni scelta degli indici in $\N$). In particolare, poichè
	\begin{equation*}
		E\cap F_i=E\cap (\sqcup_{j\in\N}B_{ij})=\sqcup_{j\in\N}(E\cap B_{ij})\virgola
	\end{equation*}
	abbiamo che
	\begin{equation*}
		\sigma_i(E\cap F_i)=\sigma_i(\sqcup_{j\in\N}(E\cap B_{ij}))=\sum_{j\in\N}\sigma_i(E\cap B_{ij})\virgola
	\end{equation*}
	dove l'ultima uguaglianza è vera perchè dati $A_j=E\cap B_{ij}$, che sono disgiunti fra loro, risulta
	\begin{align*}
		\sigma_i(\cup_{j\in\N}A_j)& =\int_{f_i^{-1}(\cup_{j\in\N}A_j)}\sqrt{\det(\Diff f_i(x)^T\Diff f_i(x))} \de x\\
		&=\sum_{j\in\N}\int_{f_i^{-1}(A_j)}\sqrt{\det(\Diff f_i(x)^T\Diff f_i(x))} \de x=\sum_{j\in\N}\sigma_i(E\cap B_{ij})\punto
	\end{align*}
	
	Otteniamo quindi che
	\begin{equation*}
		\sigma(E)=\sum_{i\in\N}\sigma_i(E \cap F_i)=\sum_{i,j\in\N} \sigma_i(E\cap B_{ij})
	\end{equation*}
	e analogamente $\sigma'(E)=\sum_{i,j\in\N} \sigma_i'(E\cap B_{ij})$. Da notare che in queste sommatorie a due indici non ci dobbiamo preoccupare dell'ordine in cui si fanno le somme, perchè tutti i termini della sommatoria sono positivi (poichè integrali di funzioni positive).

	Per il \cref{lemma:InvarianzaImmersione} applicato alle parametrizzazioni $f_i$ e $g_j$ e al Boreliano $E\cap B_{ij}$, abbiamo però che $\sigma_i(E\cap B_{ij})=\sigma_j'(E\cap B_{ij})$, da cui otteniamo proprio che 
	\begin{equation*}
		\sigma(E)=\sum_{i,j\in\N} \sigma_i(E\cap B_{ij})=\sum_{i,j\in\N} \sigma_i'(E\cap B_{ij})=\sigma'(E)\punto
	\end{equation*}

\end{proof}

\begin{theorem}
	La misura di Lebesgue $k$-dimensionale definita sui Boreliani di una varietà differenziabile $k$-dimensionale di $\R^n$ $\Sigma$ è effettivamente una misura, cioè $(\Sigma,\Borel(\Sigma),\sigma)$ è uno spazio di misura.
\end{theorem}
\begin{proof}
	I Boreliani di $\Sigma$ sono per definizione una \sigalg\ e $\sigma(\emptyset)=0$ banalmente, quindi per dimostrare che $(\Sigma,\Borel(\Sigma),\sigma)$ è uno spazio di misura ci rimane solo da verificare che $\sigma$ sia \sigadd.
	
	Per la \sigadd[ità] di $\sigma_i$, già mostrata nella dimostrazione della \cref{prop:DefMisuraKDimBuona}, abbiamo che
	\begin{multline*}
		\sigma(\cup_{k\in\N}E_k)=\sum_{i\in\N}\sigma_i((\cup_{k\in\N}E_k)\cap F_i)=\sum_{i\in\N}\sigma_i(\cup_{k\in\N}(E_k\cap F_i))\\
		=\sum_{i\in\N}\sum_{k\in\N}\sigma_i(E_k\cap F_i)=\sum_{k\in\N}\sum_{i\in\N}\sigma_i(E_k\cap F_i)=\sum_{k\in\N}\sigma(E_k)\virgola
	\end{multline*}
	che è proprio quello che volevamo dimostrare.
\end{proof}
\section{Formule di Stokes}\label{sezione:FormuleStokes}

Durante tutta la sezione assumeremo sempre che $\Omega\subseteq\R^n$ sia un aperto.

\begin{theorem}\label{thm:PtRegEquiv}
	Per un punto $x\in \partial \Omega$ le seguenti $3$ condizioni sono equivalenti:
	\begin{enumerate}
		\item esistono $U$ intorno di $x$ e $f\in C^1(U,\R)$ tale che si abbia
			\begin{itemize}
				\item $U\cap \partial \Omega=f^{-1}(0)$,
				\item $U\cap \Omega=f^{-1}(\oo{-\infty}{0})$,
			\end{itemize}
			e tale che $\Diff f(x)\neq 0$ se $f(x)=0$;\label{PRE:i}
		\item esistono $U$ intorno di $x$ e $g$ diffeomorfismo $C^1$ da $\R^n$ in $U$ tale che\label{PRE:ii}
			\begin{itemize}
				\item $U\cap \partial \Omega = g(\R^{n-1}\times\{0\})$,
				\item $U\cap \Omega = g(\R^{n-1}\times\oo{-\infty}{0})$;
			\end{itemize}
		\item esistono, a meno di riordinare le coordinate, un intorno $U=V\times I$ di $x=(y,t)$ tale che $V\in\R^{n-1}$ è intorno di $y$
			e $I\in\R$ è intorno di $t$ e una funzione $\psi\in C^1(V,I)$ tale che
			\begin{itemize}
				\item $U\cap \partial \Omega = \{(z,s):s=\psi(z)\}$,
				\item $U\cap \Omega = \{(z,s):s<\psi(z)\}$.
			\end{itemize}\label{PRE:iii}
	\end{enumerate}
\end{theorem}

\begin{proof}
	Dimostriamo la catena in ordine inverso:
	\begin{description}
		\item [\ImplicationProof{PRE:i}{PRE:iii}] Dato che $\Diff f(x)\neq 0$, esiste una derivata parziale non nulla, supponiamo senza perdita
			di generalità che sia l'ultima e che sia positiva (se negativa il ragionamento è lo stesso).
			Allora le ipotesi del teorema della funzione implicita del Dini sono verificate, per cui otteniamo
			che esiste un intorno $V\times I\subseteq U$ di $x$ per cui esiste $\psi\in C^1(V,I)$ tale che $f(v,\psi(v))=0$ e $\psi(v)$
			è l'unico punto $t'\in I$ tale che $f(v,t')=0$.
			Inoltre, per la continuità di $\Diff f(x)$, esiste un intorno $V'\subseteq V$ di $v$ tale che $\Diff f(v')_n>\frac{1}{2}
			\Diff f(x)_n$ per tutti i $v'\in V'$. Ora, per la differenziabilità di $f$ in $v'$ si ha che
			\[
				f(v',\psi(v')+\delta)=f(v', \psi(v'))+\Diff f(v')_n\delta+\smallO(\delta)=\Diff f(v')\delta+\smallO(\delta)\virgola
			\]
			che, per $\delta>0$ e abbastanza piccolo, dà $f(v',\psi(v')+\delta)\geq \frac{1}{2}\Diff f(v')_n\geq \frac{1}{4}\Diff
			f(x)_n\delta>0$. Infine, non può esistere un $\delta>0$ tale che $\psi(v')+\delta\in I$ e
			$f(v',\psi(v')+\delta)<0$, altrimenti per la continuità di $f$ si avrebbe che esiste un punto $\psi(v')+\delta'\neq \psi(v')$
			che sia zero per la funzione $f(v',\cdot)$, contraddicendo il teorema della funzione implicita del Dini.
			Per $\delta<0$ si ha lo stesso risultato. Questo ci dice che $f^{-1}(0)\cap (V'\times I)=\{(v',s):s=\psi(v')\}$ e
			$f^{-1}(\oo{-\infty}{0})\cap (V'\times I)=\{(v',s):s<\psi(v')\}$, che conclude per le ipotesi sulle controimmagini di $f$.
		\item [\ImplicationProof{PRE:iii}{PRE:ii}] Se abbiamo $V\times I$ intorno di $x$ e $\psi$, detto $J=\oo{-\varepsilon}{\varepsilon}$
			possiamo ottenere un diffeomorfismo $g:V\times I \rightarrow V\times J$ ponendo $g(z,s)=(z,s-\psi(z))$. Allora,
			$g(V\times \{0\})=$graf$(\psi)=U\cap \partial \Omega$ e $g(V\times \oo{-\varepsilon}{0})=\{(z,s):s<\psi(z)\}=U\cap \Omega$.
			Per ottenere un diffeomorfismo di tutto $\R^n$ basta comporre $g$ con opportuni diffeomorfismi tra un intorno rettangolare
			di $x$ e $\R^n$.
		\item [\ImplicationProof{PRE:ii}{PRE:i}] Per la prima condizione basta considerare$f=(g^{-1})_n$ l'ultima componente di $g^{-1}$.
	\end{description}
\end{proof}


\begin{definition}[punto regolare]
	Diremo che un punto $x\in \partial \Omega$ è un punto regolare del bordo se verifica una delle $3$ condizioni di cui
	al \cref{thm:PtRegEquiv}.
\end{definition}

D'ora in poi assumeremo che $\Omega$ sia anche limitato.

\begin{definition}[aperto regolare]
	Diremo che $\Omega$ è un aperto regolare se tutti i punti del suo bordo sono regolari.
\end{definition}


\begin{theorem}[Partizione dell'unità]\label{thm:PartizioneUnita}
	Dato un ricoprimento di $\R^n$ con aperti $(\Omega_i)_{i\in I}$, esiste una famiglia di funzioni $f_i:\R^n\to\Rpiu$ infinitamente derivabili
	e non negative tali che:
	\begin{itemize}
		\item la chiusura dell'insieme in cui $f_i$ non si annulla è contenuta in $\Omega_i$ per ogni $i\in I$,
		\item per ogni $x\in\R^n$ vale che $\sum_{i\in I} f_i(x)=1$\footnote{La serie la definiamo come il $\sup$ delle somme finite.}.
	\end{itemize}
\end{theorem}
\begin{proof}
	TODO
\end{proof}

\begin{lemma}\label{lem:EquivRegolare}
	$\Omega$ è un aperto regolare se e solo se esiste una funzione $\phi\in C^1(\R^n,\R)$ tale che $\Diff \phi(x)\neq 0$ se $\phi(x)=0$ e tale che
	\begin{itemize}
		\item $\Omega = \phi^{-1}(\oo{-\infty}{0})$,
		\item $\partial \Omega = \phi^{-1}(0)$.
	\end{itemize}
\end{lemma}

\begin{proof}
	L'implicazione del ``se'' è banale per la caraterizzazione del punto \cref{PRE:i}. Per l'altra implicazione, per la regolarità dei punti
	del bordo di $\Omega$, consideriamo per ogni $x\in\partial \Omega$ un intorno $U_x$ tale che esista una funzione $\psi_x:U_x\rightarrow \R$
	come nel \cref{thm:PtRegEquiv}. Dato che $\Omega$ è limitato, $\partial\Omega$ è compatto e gli $U_x$ sono un ricoprimento aperto;
	quindi consideriamo un sottoricoprimento finito $U_1, \dots, U_n$  e prendiamo anche $U_0 = \R^n \setminus \partial \Omega$.
	Ora abbiamo che gli aperti $U_0,\dots,U_n$ sono un ricoprimento di $\R^n$;
	il \cref{thm:PartizioneUnita} fornisce quindi delle funzioni $f_i$ per ogni $i\in\{0,\dots, n\}$.
	Infine, definiamo $\psi_0:U_0\rightarrow \R$ tale che $\psi_0(x)=1$ se $x\notin \Omega$, $\psi_0(x)=-1$ altrimenti. Sia ora 
	\[
		\phi(x)=\sum_{i=0}^n f_i(x) \psi_i(x)\virgola
	\]
	dove, con una leggera imprecisione formale, poniamo $f_i(x) \psi_i(x) = 0$ se $x\notin U_i$, in modo che $f_i(x) \psi_i(x)$ sia una funzione
	$C^1$ in tutto $\R^n$. Infatti, in $U_i$, $\psi_i$ è $C^1$ e $f_i$ è addirittura $C^{\infty}$, altrove l'abbiamo posta uguale a zero.
	Ora abbiamo che $\phi\in C^1$ perché somma di funzioni $C^1$; $\phi(x)=0$ per ogni $x\in \partial\Omega$, dato che ognuna delle
	$\psi_i$ aveva questa proprietà se era definita in $x$ e la $\phi$ è una combinazione convessa di $\psi_i$;
	$\phi(x)<0$ per ogni $x \in \Omega$, perché $\phi$ è combinazione convessa di funzioni che, per ciascun $x\in\Omega$, sono non positive e
	almeno di una strettamente negativa; similmente, $\phi(x)>0$ per ogni $x \notin \bar{\Omega}$.
	
	Rimane da mostrare che $\Diff f(x)\neq 0$ se $f(x)=0$. Ma in ciascun punto $x$ del bordo, le funzioni $\psi_i$ non hanno differenziale nullo
	quando sono definite in $x$. Esisterà quindi una direzione $e_k$ per la quale esiste almeno un $i$ tale che $\Diff \psi_i(x)_k\neq 0$. Dato che
	le $\psi_i$ sono differenziabili, si ha che
	\[
		\psi_i(x+\delta e_k)=\psi_i(x)+\Diff \psi_i(x)_k \delta + \smallO(\delta) = \Diff \psi_i(x)_k \delta + \smallO(\delta)
	\]
	per cui, dato che $\psi_i(x+\delta e_k)<0$ solo se $x+\delta e_k\in \Omega$, quindi, per $u\neq i'$, $\Diff \psi_i(x)_k$ e
	$\Diff \psi_{i'}(x)_k$ non possono essere discordi, altrimenti si ha $\psi_i(x+\delta e_k)= \Diff \psi_i(x)_k \delta + \smallO(\delta)>0$ e
	$\psi_i'(x+\delta e_k)=\Diff \psi_i'(x)_k \delta + \smallO(\delta)<0$ pur avendo in entrambi i casi
	che $x+\delta e_k$ sta in $\Omega$. Quindi, sempre per lo stesso $x\in\partial \Omega$,
	\[ 
		\Diff \phi(x)=\sum_{i=0}^n \Diff (f_i \cdot\psi_i)(x)=\sum_{i=0}^n \Diff f_i(x)\cdot \psi_i(x)+\sum_{i=0}^n f_i(x)\cdot \Diff \psi_i(x)=
		\sum_{i=0}^n f_i(x)\cdot \Diff  \psi_i(x)\virgola
	\]
	poiché $\psi_i\equiv 0$ sul bordo; quindi $\Diff \phi(x)$ non è nullo perché la $j$-esima componente è combinazione convessa di
	$\Diff \psi_i(x)_k$ che sono non discordi.
\end{proof}

\begin{definition}
	Chiamiamo campo normale esterno la funzione $\nu:\partial\Omega\rightarrow \R^n$ tale che
	\[
		\nu(x)=\frac{\nabla \phi(x)}{\lVert\nabla \phi(x)\rVert}\punto
	\]
\end{definition}

\begin{remark}
	Il \cref{lem:EquivRegolare}, oltre che a fornire la funzione $\phi$ per l'aperto regolare $\Omega$, dice anche che $\nu(x)$ è ben definita
	su tutto $\partial \Omega$ e che è continua.
\end{remark}

\begin{remark}
	Si ha che per $t>0$ e $x\in\partial\Omega$, $x+t\nu(x)\notin \Omega$ per $t$ abbastanza piccoli e per $t<0$, $x+t\nu(x)\in \Omega$ per $t$
	abbastanza piccoli.
\end{remark}
\begin{proof}
	Stesso ragionamento fatto nella dimostrazione del \cref{lem:EquivRegolare}.
\end{proof}

\begin{remark}
	Per ogni $v$ tangente a $\partial\Omega$ in $x$ si ha che $<v,\nu(x)>=0$. Ma qui bisognerebbe spendere un paio di parole su cosa significhi
	essere tangente a $\partial\Omega$.
\end{remark}


Vogliamo ora formalizzare la definizione di integrale sulla superficie $\partial\Omega$ sfruttando anche i risultati della sezione precedente.

\begin{lemma}\label{lem:GrapVar}
	Sia $V\in\R^{n-1}$ aperto e $f\in C^1(V,\R)$, allora il grafico di $f$ è una varietà differenziabile di dimensione $n-1$ immersa in $\R^n$.
\end{lemma}
\begin{proof}
	Sia $g:V\rightarrow \R^n$ data da $g(x)=(x,f(x))$. Ora è chiaro che graf$(f)$ è l'immagine di $g$; inoltre $g$ è banalmente di classe $C^1$
	e, se vista come funzione da $V$ in graf$(f)$ è anche un omeomorfismo. Inoltre $\Diff g(x)[h]=(h,\Diff f(x)[h])$, in particolare il differenziale è
	iniettivo in ogni punto, quindi $g$ è una immersione iniettiva, cioè è una parametrizzazione $C^1$ di graf$(f)$.
\end{proof}

\begin{remark}
	Dalla dimostrazione di questo lemma, dato che $\nabla f(x)=\Diff f(x)$, segue immediatamente che $\Diff g(x)^T \Diff g(x)=
	(I \ \ \nabla f(x)^T)
	\left(\begin{smallmatrix}
	I \\
	\nabla f(x)
	\end{smallmatrix}\right)$, quindi per il \cref{lemma:DeterminanteMatriceQuasiIdentita} vale che $\det (\Diff g(x)^T \Diff g(x)) = 1+\lVert\nabla f(x)\rVert^2$. 
\end{remark}

Il \cref{lem:GrapVar} ci permette quindi di costruire un misura superficiale su graf$(f)$ in accordo con la \cref{def:MisuraKDimensionale}:
\begin{equation}\label{eq:MisuraSuperficialeGrafico}
	\sigma(E)=\int_{g^{-1}(E)} \sqrt{1+\lVert\nabla f(x)\rVert^2} \de x\punto
\end{equation}


\begin{proposition}\label{prop:FormulaStokes}
	Sia $u\in C^1(\bar \Omega)$ (ossia $u$ è la restrizione a $\bar \Omega$ di una funzione $C^1$ definita su un aperto
	$\Omega'\supseteq \bar \Omega$), sia $\nu:\partial \Omega \rightarrow \R^n$ la normale uscente da $\Omega$, allora vale che
	\begin{equation}\label{eq:FormulaStokes}
		\int_{\Omega} \Diff u(x)_j \de x = \int_{\partial \Omega} u(y)\nu(y)_j \de \sigma(y)\punto
	\end{equation}
\end{proposition}

\begin{proof}
	Per la dimostrazione si procederà in 2 step:
	\begin{enumerate}
		\item per ogni $p\in \bar \Omega$ esiste un intorno aperto rettangolare $U_p=\prod_{i=1}^n \oo{p_i-\varepsilon_i}{p_i+\varepsilon_i}$
			tale che l'\cref{eq:FormulaStokes} vale per ogni funzione $u$ tale che $supp(u)\subseteq U_p$;
		\item $\bar \Omega$ è ricoperto con un numero finito di intorni come sopra, quindi si applica il \cref{thm:PartizioneUnita} per ottenere
			la tesi per ogni $u$.
	\end{enumerate}
	
	Dimostriamo quindi il primo punto, distinguendo i casi in cui $p\in \Omega$ e $p\in \partial \Omega$.
	Se $p\in \Omega$ esisterà un intorno rettangolare $U_p=V\times \oo{a}{b}$ contenuto in $\Omega$, poiché quest'ultimo è un aperto. Se ora $supp(u)\subseteq U_p$,
	$u(\partial \Omega)=0$; quindi abbiamo che l'integrale a destra è $0$. Per l'intregrale a sinistra, invece, abbiamo che una funzione $u$
	con supporto contenuto in $V\times \oo{a}{b}$ è tale che $u(x)=0$ per ogni
	$x\in \partial (V\times \oo{a}{b})$. Allora, per il \cref{thm:Fubini} e il teorema fondamentale del calcolo integrale,
	\begin{align*}
		\int_{\Omega} \Diff u(x)_n \de x &= \int_{V\times \cc{a}{b}} \Diff u(v,t)_n \de (v,t) = \\
		& = \int_V \int_{\cc{a}{b}} \Diff u(v,t)_n \de t \de v = \int_V (u(v,b) - u(v,a)) \de v = 0\punto
	\end{align*}
	Quindi abbiamo la tesi del primo punto quando $p\in\Omega$. Se invece $p\in \partial \Omega$, allora $p$ è un punto regolare, cioè esiste
	un intorno $U=V\times I$ di $p$ e una $\phi\in C^1(V,I)$ tale che
	\begin{itemize}
		\item $U\cap \partial \Omega = \{(z,s):s=\psi(z)\}$,
		\item $U\cap \Omega = \{(z,s):s<\psi(z)\}$.
	\end{itemize}
	Distinguiamo ora i casi $j=n$ e $j\neq n$.
	\begin{description}
		\item [$j=n$:] Dato che stiamo supponendo che $supp(u)\subseteq U$, per il \cref{thm:Fubini} abbiamo che
			\begin{align*}
				\int_{\Omega} \Diff u(x)_n \de x  &= \int_{\Omega\cap U} \Diff u(v,t)_n\de x =
				\int_{U} \Diff u(v,t)_n \chi_{\{t<\psi(v)\}} \de (v,t) = \\
				& = \int_V \int_I \Diff u(v,t)_n \chi_{\{t<\psi(v)\}} \de t \de v =
				\int_V \int_{-\infty}^{\psi(v)} \Diff u(v,t)_n \de t \de v = \\
				& = \int_V (u(v,\psi(v)) - 0) \de v = \int_V u(v,\psi(v)) \de v\punto
			\end{align*}
			Ora abbiamo che la funzione $g:V\rightarrow \R^n$ data da $g(v)=(v,\psi(v))$ parametrizza l'ipersuperfice data dal grafico di
			$\psi$, quindi la misura superficiale è data dall' \cref{eq:MisuraSuperficialeGrafico}.
			Inoltre, detta $\nu:\partial \Omega \cap U \rightarrow \R^n$ la normale uscente da $\Omega$, sappiamo che
			$\nu(v,\psi(v))_n=(1+\psi(v)^2)^{-\frac{1}{2}}$. Quindi per definizione di integrale superficiale abbiamo che
			\begin{align*}
				\int_V u(v,\psi(v)) \de v &= \int_V u(v,\psi(v)) (1+\psi(v)^2)^{-\frac{1}{2}} (1+\psi(v)^2)^{\frac{1}{2}} \de v = \\
				& = \int_V u(v,\psi(v)) \nu(v,\psi(v))_n (1+\psi(v)^2)^{\frac{1}{2}} \de v = \\
				& = \int_{\partial \Omega \cap U} u(x) \nu(x)_n \de \sigma(x) = \\
				& = \int_{\partial \Omega} u(x) \nu(x)_n \de \sigma(x)\virgola
			\end{align*}
			dove nell'ultimo passaggio si è usato che $supp(u)\subseteq U$.
		\item [$j\neq n$:] Possiamo assumere (a meno di riordinare le coordinate) che $j=n-1$ e, a meno di considerare un intorno più piccolo,
			che $V=W \times J$. Ora, per il \cref{thm:Fubini} abbiamo che
			\[
				\int_{\Omega} \Diff_{n-1} u(x)_{n-1} \de x = \int_W \int_J \int_I \Diff_s u(w,s,t) \de t \de s \de w\virgola
			\]
			inoltre, dato che $\Diff_s \int_X u(v,s)\de v = \int_X \Diff_s u(v,s)\de v$ e che $\Diff_y \int_a^y u(v,t)\de t=u(v,y)$,
			utilizzando le regole di composizione del differenziale, abbiamo che
			\[
				\Diff_s \int_{-\infty}^{\psi(w,s)} u(w,s,t) \de t =
				\int_{-\infty}^{\psi(w,s)} \Diff_s u(w,s,t) \de t + u(w,s,\psi(w,s))\Diff_s \psi (w,s)\punto
			\]
			Per cui, integrando il termine a sinistra su $V=W\times J=W\times \oo{a}{b}$, per il teorema fondamentale del
			calcolo integrale,
			\begin{align*}
				&\int_W \int_a^b \Diff_s \int_{-\infty}^{\psi(w,s)} u(w,s,t) \de t \de s \de w = \\
				&\int_W \left(\int_{-\infty}^{\psi(w,b)} u(w,b,t) \de t - \int_{-\infty}^{\psi(w,a)} u(w,a,t) \de t\right) \de w
				= 0\virgola
			\end{align*}
			poiché i punti $(w,b,t)$ e $(w,a,t)$ sono sul bordo di $U$, dove la funzione $u$ è nulla per ipotesi. Usando questa relazione
			otteniamo che
			\[
				\int_W\int_J\int_I\Diff_s u(w,s,t) \de t \de s \de w =-\int_W \int_J u(w,s,\psi(w,s))\Diff_s \psi (w,s)\de s\de w\punto
			\]
			Ora, dato che $\nu(x)_{n-1}=\nu(v,\psi(v))_{n-1}=-\Diff\psi(v)_{n-1}(1+\psi(v)^2)^{-\frac{1}{2}}$, otteniamo
			\begin{align*}
				\int_{\Omega} \Diff_{n-1} u(x)_{n-1} \de x & = -\int_W \int_J u(w,s,\psi(w,s))\Diff_s \psi (w,s) \de s \de w\\
				& = -\int_V u(v,\psi(v))\Diff_{n-1} \psi(v) \de v\\
				& = -\int_V u(v,\psi(v))(-\nu(v,\psi(v))_{n-1})(1+\psi(v)^2)^{\frac{1}{2}} \de v\\
				& = \int_{\partial \Omega \cap U} u(x)\nu(x)_{n-1} \de \sigma(x)\punto
			\end{align*}
	\end{description}
	Questo conclude il primo punto.
	
	Per la seconda parte procediamo in questo modo. Intanto, osserviamo che se $f\in C^1(\R^n)$ ha supporto $U$ tale che $U\cap \bar\Omega =
	\emptyset$, la \cref{eq:FormulaStokes} è banalmente vera in quanto entrabi i membri sono banalmente nulli.
	Consideriamo ora, per ogni punto $p\in \bar\Omega$, un intorno $U_p$ della forma richiesta al punto precedente. È chiaro che $\{U_p\}$ è un
	ricoprimento aperto del compatto $\bar \Omega$, quindi consideriamo un sottoricoprimento finito $U_1,\dots,U_n$; consideriamo anche
	$U_0=\R^n \setminus \bar\Omega$. Il \cref{thm:PartizioneUnita} ci fornisce delle funzioni $f_i\in C^{\infty}$, per $0\leq i \leq n$,
	tali che la loro somma è la funzione costante $1$.
	Sia ora $u\in C^1(\R^n,\R)$ una qualsiasi funzione. Otteniamo che
	\[
		u = \sum_{i=0}^n f_i\cdot u\virgola
	\]
	ma ora abbiamo che $f_i \cdot u$ (che è una funzione $C^1$) soddisfa l'\cref{eq:FormulaStokes} per ogni $i$; infatti, per $i=0$ è vero
	per quanto appena osservato, per $i>0$, la funzione $f_i \cdot u$ ha supporto contenuto in $U_p$ per qualche $p\in \bar\Omega$ e, per quanto
	dimostrato nel punto precedente, abbiamo che per questo intorno $U_p$ ogni funzione $C^1$ soddisfa l'\cref{eq:FormulaStokes}.
	Per linearità degli integrali e quindi dei membri che compaiono in questa equazione, è chiaro che le funzioni che soddisfano formano
	un sottospazio vettoriale di $C^1$. In particolare $u$ soddisfa la formula di Stokes.
\end{proof}

Un'immediata conseguenza del teorema appena dimostrato è la seguente:

\begin{corollary}(Teorema della divergenza)
	\label{cor:thDivergenza}
	Sia $u:\bar\Omega \rightarrow \R^n$ una funzione $C^1$ (nello stesso senso di prima), allora si ha che
	\begin{equation}\label{eq:thDivergenza}
		\int_{\Omega} \text{div }u (x) \de x = \int_{\partial \Omega} <u(y),\nu(y)> \de \sigma(y)\puntovirgola
	\end{equation}
	dove l'operatore di divergenza è definito da $\text{div }u(x==\text{tr }\Diff u(x)$.
\end{corollary}

\begin{proof}
	Utilizzando la \cref{prop:FormulaStokes} sulla componente $j$-esima di $u$ che indichiamo con $u^j$ si ha che, per ogni $0\leq j\leq n$,
	\[
		\int_{\Omega} \Diff u^j(x)_j \de x = \int_{\partial \Omega} u^j(y)\nu(y)_j \de \sigma(y)\virgola
	\]
	da cui sommando otteniamo, per linearità degli integrali:
	\[
		\int_{\Omega} \sum_{j=0}^n\Diff u^j(x)_j \de x = \int_{\partial \Omega} \sum_{j=0}^n u^j(y)\nu(y)_j \de \sigma(y)\punto
	\]
	Ma ora, $\text{div}u (x)= \text{tr}\Diff u(x) = \sum_{j=0}^n \Diff u^j(x)_j$ e $<u(x),\nu(x)>=\sum_{j=0}^n u^j(y)\nu(y)_j$,
	quindi si ha la tesi.
\end{proof}

Giocherellando ancora con queste equazioni si ottengono anche le seguenti formule dette di Green:

\begin{corollary}(Prima formula di Green)
	Siano $u\in C^1(\bar\Omega, \R)$ e $v\in C^2(\bar\Omega, \R)$, allora si ha
	\[
		\int_{\Omega} \left( u \Delta v+< \text{grad }u,\text{grad }v> \right) de x=
		\int_{\bar \Omega} u \frac{\partial v}{\partial \nu} \de \sigma(y) \virgola
	\]
	dove l'operatore $\Delta v = \sum_{j=1}^n \Diff_{jj} v$ è il laplaciano e $\frac{\partial v}{\partial \nu}(x)=\Diff v(x)[\nu]$.
\end{corollary}

\begin{proof}
	Applicando la \cref{prop:FormulaStokes} sulle funzioni $v(x)\Diff u(x)_j$, dato che abbiamo 
	$\Diff_j (v(x)\Diff u(x)_j ) = \Diff v(x)_j \Diff u(x)_j + v(x)\Diff_{jj} u(x)$, sommando su $j$ otteniamo la tesi.
\end{proof}

\begin{corollary}(Seconda formula di Green)
	Siano $u,v\in C^2(\bar\Omega, \R)$ allora si ha che
	\[
		\int_{\Omega}\left(u\Delta v-v\Delta u\right) \de x=
		\int_{\bar\Omega}u\frac{\partial v}{\partial\nu}-v\frac{\partial u}{\partial\nu} \de \sigma(y)\punto
	\]
\end{corollary}

\begin{proof}
	Immediata conseguenza della prima formula applicata a $u,v$ e a $v,u$; poi basta sottrarre membro a membro.
\end{proof}




\printindex

\end{document}

\makeindex