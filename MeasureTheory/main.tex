\documentclass[a4paper,12pt]{article}
\usepackage{stilebase}
% \usepackage{float}
% \usepackage{figure}

\title{Appunti di teoria della misura elementare}
\author{Giada Franz \and Federico Glaudo \and Marco Trevisiol}

\makeindex[title=Indice analitico]
\indexsetup{level=\section}

\begin{document}

\maketitle

\begin{abstract}
	Trattiamo in queste dispense i fatti fondamentali di teoria della misura.
	
	In particolare, dopo una prima sezione prevalentemente di definizioni e fatti preparatori, dimostreremo il teorema di estensione di \carat{}, che permette di costruire misure a partire da strutture ben più semplici come le misure elementari. Applicheremo poi questa costruzione generale per fondare su solide basi la misura di Lebesgue su $\R^n$.
	
	In seguito svilupperemo la teoria dell'integrazione nel caso generale, per poi approfondire in maniera più attenta il caso dell'integrazione negli spazi euclidei.
	
	Il percorso seguito ricalca quello proposto dal professor Majer durante il corso di Analisi II dell'anno 2013-2014 alla facoltà di matematica a Pisa.
\end{abstract}
\clearpage

\tableofcontents
\clearpage

\section{Definizioni e fatti introduttivi}
In questa sezione definiremo tutti gli oggetti necessari per fondare la teoria della misura e dimostreremo alcuni fatti, perlopiù di carattare insiemistico su di essi.

In particolare definiremo alcune strutture insiemistiche (algebre, \sigalg[e], \semiring[i]) e ne dimostreremo alcune proprietà. Poi passeremo a trattare gli spazi di misura ed alcune loro \emph{versioni più deboli} come le misure esterne e le premisure.

L'idea generale di queste pagine è dare gli strumenti necessari a comprendere il teorema di estensione di una premisura, nella cui dimostrazione ed enunciato convergono tutti i fatti qui trattati. 

Da notare infine che la definizione di \semiring{} non è quella canonica (nè forse si può parlare di una definizione canonica in letteratura) ed anzi è più debole di quella tipicamente usata per enunciare il teorema di estensione. Questo rende più generale il risultato, ma allo stesso tempo rende più ostico dimostrare le prime proprietà della premisura che infatti necessiteranno di vari lemmi tecnici per essere dimostrate (e sarebbero banali se al posto di un \semiring{} ci fosse un anello).

\begin{definition}[Algebra]
	Dato un insieme $X$, una famiglia $\mathcal A\subseteq\mathcal P(X)$ è un'algebra se valgono:
	\begin{itemize}
		\item $\emptyset\in\mathcal A$
		\item $\forall A\in\mathcal A:\ A^c\in\mathcal A$ cioè un'algebra è stabile per passaggio al complementare.
		\item $\forall A,B\in\mathcal A:\ A\cup B\in\mathcal A$ cioè un'algebra è stabile per unioni finite.
	\end{itemize}
\end{definition}
\begin{remark}\label{ProprietaAlg}
	Un'algebra è stabile anche per intersezioni finite e per differenza insiemistica.
\end{remark}
\begin{proof}
	Poichè vale la formula insiemistica:
	\begin{equation*}
		\bigcap_{i\in I} A_i = \left( \bigcup_{i\in I} A_i^c \right)^c
	\end{equation*}
	e un'algebra è stabile per unione finita e complementare, facilmente risulta esserlo anche per intersezioni finite.
	
	Per la differenza si sfrutta la seguente relazione insiemistica $A\setminus B=A\cup B^c$. Questa porta a concludere visto che abbiamo appena dimostrato che $\mathcal A$ è stabile per intersezioni.
\end{proof}


\begin{definition}[\sigalg{}]
	Dato un insieme $X$, una famiglia $\mathcal A\subseteq\mathcal P (X)$ si dice \sigalg{} se valgono:
	\begin{itemize}
	\item $\emptyset\in \mathcal A$
	\item $\forall A\in \mathcal A:\ A^c\in \mathcal A$ cioè una \sigalg{} è stabile per passaggio al complementare.
	\item $\forall (A_n)_{n\in\mathbb N}\subseteq \mathcal A:\ \bigcup_{n\in\mathbb N} A_n\in \mathcal A$ cioè una \sigalg{} è stabile per unioni numerabili.  
	\end{itemize}
\end{definition}

\begin{remark}\label{ProprietaSigAlg}
	Una \sigalg{} è stabile anche per intersezioni numerabili e per differenza insiemistica.
\end{remark}
\begin{proof}
	Si dimostrano entrambe le proprietà in modo del tutto analogo a come abbiamo dimostrato \cref{ProprietaAlg}.
\end{proof}

\begin{definition}[\Semiring{}]
	Una famiglia $\mathcal S\subseteq \mathcal P(X)$ è detta \semiring{} se valgono le seguenti proprietà:
	\begin{itemize}
		\item $\emptyset\in \mathcal S$
		\item $\displaystyle\forall A,B\in \mathcal S: A\cap B, A\setminus B\in \sqcup \mathcal S$ dove
		$\displaystyle
		\sqcup{ \mathcal S }=\left\{\bigsqcup_{n\in \mathbb N} S_n\ |\ (S_n)_{n\in\mathbb N} \subseteq \mathcal S \wedge \forall i\not= j:\ S_i\cap S_j=\emptyset\right\}$ 
		cioè un \semiring{} non deve essere stabile per intersezione e differenza, ma queste si devono scrivere come unioni disgiunte.
	\end{itemize}
\end{definition}

\begin{proposition}\label{UnioneDisgiuntaQuasiAlgebra}
	Dato un \semiring{} $\mathcal S$ l'insieme $\sqcup\mathcal S$ è stabile per intersezione finita e unione numerabile.
\end{proposition}
\begin{proof}
	Notiamo intanto che, per definizione, $\sqcup\mathcal S$ è stabile per unione disgiunta numerabile.
	
	Per dimostrare la stabilità di $\sqcup\mathcal S$ per intersezione finita, basta ovviamente farlo per due soli insiemi $A,B\in\sqcup\mathcal S$. Per definizione possiamo scrivere $A=\bigsqcup_{n\in\mathbb N} A_n, B=\bigsqcup_{n\in\mathbb N} B_n$ dove $(A_n)_{n\in\mathbb N},(B_n)_{n\in\mathbb N}$ sono successioni in $\mathcal S$. Allora vale la seguente identità:
	\begin{equation*}
		A\cap B=\bigsqcup_{n\in\mathbb N} A_n\cap\bigsqcup_{n\in\mathbb N} B_n=
		\bigsqcup_{n,m\in\mathbb N} A_n\cap B_m\in\sqcup\mathcal S
	\end{equation*}
	dove nell'ultimo passaggio è stata usata la stabilità di $\sqcup\mathcal S$ per unione disgiunta.
	
	Per l'unione, consideriamo $A,B\in\mathcal S$. Poichè vale $A\cup B=(A\setminus B)\sqcup(A\cap B)$, viste le proprietà di un \semiring{}, risulta $A\cup B\in \sqcup\mathcal S$. Da questo è facile ottenere che anche unioni finite di elementi di $\mathcal S$ appartengono a $\sqcup\mathcal S$.
	
	Sfruttando quanto detto, fissata $(A_n)_{n\in\mathbb N}\subseteq\mathcal S$ vale:
	\begin{equation}\label{UnioneNumerabileDaS}
		\bigcup_{n\in\mathbb N} A_n=\bigsqcup_{n\in\mathbb N} A_n\setminus\cup_{i<n} A_i
		=\bigsqcup_{n\in\mathbb N} \bigcap_{i<n} (A_n\setminus A_i)\in\sqcup\mathcal S
	\end{equation}
	dove nell'ultimo passaggio abbiamo applicato la stabilità di $\sqcup\mathcal S$ per unione disgiunta e intersezione finita.
	
	E infine dimostriamo la stabilità di $\sqcup\mathcal S$ per unioni numerabili. Sia $(S_n)_{n\in\mathbb N}$ una successione in $\sqcup\mathcal S$. Per definizione devono esistere le successioni $(A^n_i)_{i\in\mathbb N}\subseteq \mathcal S$ tali che $S_n=\bigsqcup_{i\in\mathbb N} A^n_i$.
	
	Allora applicando \cref{UnioneNumerabileDaS} otteniamo:
	\begin{equation*}
		\bigcup_{n\in\mathbb N}S_n=\bigcup_{i,n\in\mathbb N}A^n_i\in\sqcup\mathcal S 
	\end{equation*}
	che è proprio la stabilità di $\sqcup\mathcal S$ per unioni numerabili.
\end{proof}

\begin{definition}[{\sigadd[ità]}]
	Una funzione $\mu:\mathcal F\to \Rpiu$, dove $\mathcal F$ è una famiglia di insiemi, si dice \sigadd{} se per ogni sottofamiglia numerabile $(F_n)_{n\in\mathbb N}\subseteq \mathcal F$ a due a due disgiunta, tale che l'unione appartiene a $\mathcal F$, vale l'addittività:
	\begin{equation*}
		\mu\left(\bigcup_{n\in\mathbb N}F_n \right)=\sum_{n\in\mathbb N} \mu(F_n) 
	\end{equation*}
\end{definition}
\begin{remark}
	Data $\mu:\mathcal F\to \Rpiu$ \sigadd{}, se $\emptyset\in \mathcal F$ allora $\mu(\emptyset)=0$
\end{remark}
\begin{proof}
	Usando la proprietà di \sigadd[ità] si ottiene $\mu(\emptyset)=\mu(\emptyset)+\mu(\emptyset)$ che porta ovviamente alla tesi.
\end{proof}


\begin{definition}[Spazio di misura]
	Dati $X$ un insieme, $\mathcal A$ una famiglia di sottoinsiemi di $X$ e $\mu:\mathcal A\to \Rpiu$ una funzione, la terna $(X,\mathcal A, \mu)$ si dice uno spazio di misura se:
	\begin{itemize}
		\item la famiglia $\mathcal A$ è una \sigalg{}.
		\item la funzione $\mu$ è \sigadd{}.
	\end{itemize}
	e in questo caso la funzione $\mu$ è detta \emph{misura}.
\end{definition}

D'ora in poi, quando ci si riferirà ad una misura, si darà per scontato che questa si riferisce ad uno spazio di misura.

\begin{remark}\label{MonotoniaMisura}
	Dato $(X,\mathcal A,\mu)$ uno spazio di misura, $\mu$ è monotona, cioè se $A,B\in\mathcal A$ e $A\subseteq B$ allora $\mu(A)\le \mu(B)$.
\end{remark}
\begin{proof}
	Per quanto detto in \cref{ProprietaSigAlg}, $B\setminus A\in\mathcal A$ e perciò sfruttando l'addittività su insiemi disgiunti di $\mu$ otteniamo $\mu(B)=\mu(B\setminus A)+\mu(A)>\mu(A)$ che è la tesi.
\end{proof}
\begin{remark}\label{SubAdditivitaMisura}
	Dato $(X,\mathcal A,\mu)$ uno spazio misurato, la misura $\mu$ è \sigsubadd{}, cioè dati $(A_n)_{n\in\mathbb N}\subseteq\mathcal A$ risulta:
	\begin{equation*}
		\mu\left(\bigcup_{n\in\mathbb N} A_n \right)\le \sum_{n\in\mathbb N}\mu(A_n)
	\end{equation*}
\end{remark}
\begin{proof}
	La dimostrazione risulta molto facile sfruttando la monotonia, mostrata in \cref{MonotoniaMisura}, e la solita decomposizione dell'unione in un'unione disgiunta che permette di applicare la \sigadd[ità]:
	\begin{equation*}
		\mu\left(\bigcup_{n\in\mathbb N} A_n \right)=\mu\left(\bigsqcup_{n\in\mathbb N} A_n\setminus\bigcup_{i<n}A_i \right)=
		\sum_{n\in\mathbb N}\mu\left(A_n\setminus\bigcup_{i<n}A_i\right)\le \sum_{n\in\mathbb N}\mu(A_n)
	\end{equation*}
\end{proof}



\begin{definition}\label{TrascurabiliMisura}
	In uno spazio di misura (o anch'è in ipotesi più deboli come premisura o misura esterna) un insieme con misura nulla è detto trascurabile.
\end{definition}
\begin{remark}\label{UnioneTrascurabili}
	Unione numerabile di insiemi trascurabile è a sua volta trascurabile.
\end{remark}
\begin{proof}
	È un'ovvia conseguenza della \sigsubadd[ità] della misura dimostrata in \cref{SubAdditivitaMisura}.
\end{proof}




\begin{definition}\label{FinitezzaMisura}
	Dato $(X,\mathcal A,\mu)$ uno spazio di misura, la misura $\mu$ è detta finita se $X\in\mathcal{A}$ e $\mu(X)<+\infty$.
\end{definition}

\begin{definition}\label{CompletezzaMisura}
	Dato $(X,\mathcal A,\mu)$ uno spazio di misura, la misura $\mu$ si dice completa se per ogni $A\in\mathcal A$ tale che $\mu(A)=0$, ogni suo sottoinsieme è anch'esso appartenente ad $\mathcal A$.
\end{definition}

\begin{proposition}\label{CompletamentoMisura}
	Dato uno spazio di misura $(X,\mathcal A,\mu)$, chiamo $\mu$-completamento di $\mathcal A$ l'insieme $\mathcal A^*$ formato dagli elementi di $\mathcal A$ uniti con un sottoinsieme di un trascurabile:
	\begin{equation*}
		\mathcal A^*=\{A\cup N\ |\ A\in\mathcal A\ \wedge \exists B:\ N\subseteq B,\ \mu(B)=0\}
	\end{equation*}
	Allora $\mathcal A^*$ è una \sigalg{} e $\mu$ si estende in maniera canonica su $\mathcal A^*$ ad una misura completa.
\end{proposition}
\begin{proof}
	Dimostriamo intanto che $\mathcal A^*$ è una \sigalg{}.
	Dato $A^*\in\mathcal A^*$, esistono per definizione $A\in\mathcal A$ e $N\subseteq B\in\mathcal A$ dove $\mu(B)=0$, tali che $A^*=A\cup N$. Possiamo facilmente imporre $B\cap A=\emptyset$, visto che se così non è si può ridefinire $N,B$ facendone la differenza con $A$. Assumiamo perciò $A,B$ disgiunti.
	Chiamando $M=B\setminus N$, vale facilmente:
	\begin{equation*}
		(A^*)^c=(A\cup N)^c=A^c\cap N^c=A^c\cap(B^c\cup M)=(A^c\cap B^c )\cup M \in\mathcal A^*
	\end{equation*}
	dove l'ultima uguaglianza vale perchè ho supposto $A$ e $B$ disgiunti e l'ultima appartenenza è invece vera poichè $M$ è anch'esso sottoinsieme di un trascurabile. Quindi $\mathcal A^*$ è stabile per passaggio al complementare.
	
	Ora consideriamo $(A^*_n)_{n\in\mathbb N}$ una successione di insiemi, e scegliamo $(A_n)_{n\in\mathbb N},(B_n)_{n\in\mathbb N},(N_n)_{n\in\mathbb N}$ tali che valgano $A^*_n=A_n\cup N_n$ e $N_n\subseteq B_n\in\mathcal A$ con $\mu(B_n)=0$.
	
	Allora risulta:
	\begin{align*}
		\bigcup_{n\in\mathbb N}A^*_n &=\bigcup_{n\in\mathbb N}A_n \cup \bigcup_{n\in\mathbb N}N_n\\
		\bigcup_{n\in\mathbb N}N_n &\subseteq \bigcup_{n\in\mathbb N}B_n\in\mathcal A\ \wedge\ \mu\left(\bigcup_{n\in\mathbb N}B_n\right)=0
	\end{align*}
	dove l'ultima uguaglianza dipende dal fatto che la misura è \sigsubadd{} per \cref{SubAdditivitaMisura}. Perciò abbiamo dimostrato che $\mathcal A^*$ è stabile per unione numerabile.
	
	Unendo i due risultati arriviamo a dire che $\mathcal A^*$ è una \sigalg{}.
	
	Definiamo ora $\tilde\mu:\mathcal A^*\to\Rpiu$ di modo che dato $A^*\in\mathcal A^*$ valga $\tilde\mu(A^*)=\mu(A)$ dove $A^*=A\cup N$ con $N$ sottoinsieme di un trascurabile ed $A\in\mathcal A$.
	
	Dimostriamo innanzitutto la coerenza della definizione di $\tilde\mu$. Si deve dimostrare che per $A,A'\in\mathcal A$ e $N,N'$ sottoinsiemi di trascurabili tali che $A\cup N=A'\cup N'$ vale $\mu(A)=\mu(A')$. Siano $B,B'\in\mathcal A$ i trascurabili a cui appartengono $N,N'$, sia infine $Q=B\cup B'$ a sua volta trascurabile.
	
	Per facili ragionamenti di monotonia abbiamo:
	\begin{align*}
		\mu(A)\le \mu(A\cup Q) =\mu(A)+\mu(Q)=\mu(A) &\Longrightarrow \mu(A)=\mu(A\cup Q)\\
		\mu(A')\le \mu(A'\cup Q) =\mu(A')+\mu(Q)=\mu(A') &\Longrightarrow \mu(A')=\mu(A'\cup Q)
	\end{align*}
	ma per costruzione vale $A\cup Q=A'\cup Q$ e perciò otteniamo proprio $\mu(A)=\mu(A')$ che mostra la coerenza della definizione di $\tilde\mu$.
	
	La \sigadd[ità] è ovvia una volta che si è mostrata la coerenza. Presa $(A^*_n)_{n\in\mathbb N}\in\mathcal A^*$ famiglia disgiunta a due a due, sia $(A_n)_{n\in\mathbb N}$ la parte \emph{non trascurabile} della prima successione. Allora risulta, per definizione di $\tilde\mu$:
	\begin{equation*}
		\sum_{n\in\mathbb N} \tilde\mu(A^*_n)=\sum_{n\in\mathbb N} \mu(A_n)=\mu\left(\bigcup_{n\in\mathbb N} A_n\right)=\mu\left(\bigcup_{n\in\mathbb N} A^*_n\right)
	\end{equation*}
	dove nell'ultimo passaggio abbiamo implicitamente applicato \cref{UnioneTrascurabili}.
	
	Resta da verificare che $\tilde\mu$ sia completa, ma questo è ovvio visto che  $\tilde\mu(A^*)=0$ se e solo se $A^*$ è il sottoinsieme di un trascurabile di $\mathcal A$ e la relazione di essere sottoinsieme di un trascurabile è chiusa per estrazione di sottoinsiemi. 

\end{proof}



\begin{proposition}\label{LimiteMonotonoMisura}
	Dato $(X,\mathcal A,\mu)$ uno spazio di misura, allora data una successione $(A_n)_{n\in\mathbb N}\subseteq \mathcal A$ tale che $A_n\subseteq A_{n+1}$ vale:
	\begin{equation*}
		\mu\left(\bigcup_{n\in\mathbb N} A_n\right)=\lim_{n\in\mathbb N} \mu(A_n)
	\end{equation*}
\end{proposition}
\begin{proof}
	Sia $B_n=A_n\setminus\bigcup_{i<n}A_i$. Applicando \cref{ProprietaSigAlg} si ottiene $B_n\in\mathcal A$.
	Per facili ragionamenti insiemistici risulta che la successione $(B_n)_{n\in\mathbb N}$ è disgiunta a due a due ed inoltre $A_n=\bigcup_{i\le n}B_i$.
	Sfruttando tutte queste proprietà e la \sigadd[ità] di $\mu$, otteniamo:
	\begin{equation*}
		\mu\left(\bigcup_{n\in\mathbb N} A_n\right)=\mu\left(\bigcup_{n\in\mathbb N} B_n\right)=
		\sum_{n\in\mathbb N} \mu(B_n)=\lim_{n\to\infty} \sum_{i\le n} \mu(B_i)=
		\lim_{n\to\infty} \mu\left(\bigcup_{i\le n} B_i\right)=\lim_{n\to\infty} \mu(A_n)
	\end{equation*}
	che è proprio la tesi.
\end{proof}

\begin{definition}[Misura esterna]
	Dato un insieme $X$ e una funzione $\mu^*:\mathcal P(X)\to \Rpiu$ è detta una misura esterna se valgono:
	\begin{itemize}
		\item $\mu^*(\emptyset)=0$
		\item $\mu^*$ è monotona, cioè dati $A,B\subseteq X$ se vale $A\subseteq B$ allora $\mu^*(A)\le \mu^*(B)$
		\item $\mu^*$ è \sigsubadd{}, cioè  per ogni successione $(A_n)_{n\in\mathbb N}\subseteq \mathcal P(X)$ di sottoinsiemi di $X$ vale $\mu^*\left(\bigcup_{n\in\mathbb{N}}A_n\right)\le \sum_{n\in\mathbb N} \mu^*(A_n)$
	\end{itemize}
\end{definition}

\begin{remark}
	Dato $(X,\mathcal A,\mu)$ uno spazio di misura, la misura $\mu$ è \sigsubadd{}.
\end{remark}
\begin{proof}
	Data una successione di sottoinsiemi $(A_n)_{n\in\mathbb N}\subseteq \mathcal A$, consideriamo, come nella dimostrazione di \cref{LimiteMonotonoMisura}, i sottoinsiemi $B_n=A_n\setminus\bigcup_{i<n}A_i\in\mathcal A$.
	Allora, lavorando analogamente alla dimostrazione di cui sopra, si ha:
	\begin{equation*}
		\mu\left(\bigcup_{n\in\mathbb N} A_n\right)=\mu\left(\bigcup_{n\in\mathbb N} B_n\right)=
		\sum_{n\in\mathbb N} \mu(B_n)\le \sum_{n\in\mathbb N} \mu(A_n)
	\end{equation*}
	dove nell'ultimo passaggio sfruttiamo la monotonia di $\mu$ dimostrata in \cref{MonotoniaMisura}.
\end{proof}

\begin{definition}
	Una terna $(X,\mathcal S,\mu)$ tale che $\mathcal S\subseteq\mathcal P(X)$ è un \semiring{} e $\mu:\mathcal S\to \Rpiu$ è \sigadd{}, la chiamo spazio di misura elementare e la funzione $\mu$ la chiamo misura elementare o premisura.
\end{definition}

\begin{lemma}\label{CoerenzaPremisura}
	Fissato $(X,\mathcal S,\mu)$ uno spazio di misura elementare, siano $(A_n)_{n\in\mathbb N},(B_n)_{n\in\mathbb N}\subseteq\mathcal S$ delle famiglie tali che l'unione sia la stessa, ma i $(B_n)_{n\in\mathbb N}$ siano disgiunti a due a due: $\bigcup_{n\in\mathbb N}A_n=\bigsqcup_{n\in\mathbb N}B_n$.
	Allora risulta $\sum_{n\in\mathbb N}\mu(A_n)\ge \sum_{n\in\mathbb N}\mu(B_n)$.
\end{lemma}
\begin{proof}
	Sia $A'_n=A_n\setminus\bigcup_{i<n}A_i$. La successione $(A'_n)_{n\in\mathbb N}$ è disgiunta a due a due e ogni singolo elemento appartiene a $\sqcup \mathcal S$ visto che vale $A'_n=\bigcap_{i<n}A_n\setminus A_i$ e $\sqcup \mathcal S$ è chiuso per intersezione finita, come mostrato in \cref{UnioneDisgiuntaQuasiAlgebra}. Infine è chiaro che l'unione della nuova famiglia è uguale a quella di $(A_n)_{n\in\mathbb N}$.
	È importante notare che $A_n\setminus A'_n=A_n\cap\bigcup_{i<n}A_i\in\sqcup\mathcal S$ dove l'ultima appartenenza è vera per la stabilità di $\sqcup\mathcal S$ per unioni e intersezioni finite. Allora esistono $(E^n_i)_{i\in\mathbb N}\subseteq\mathcal S$ tali che in unione disgiunta realizzano $A_n\setminus A'_n$.
	
	Siano quindi $C_{ij}=A'_i\cap B_j$. Ovviamente la successione $(C_{ij})_{i,j\in\mathbb N}$ è disgiunta a due a due (perchè lo sono sia $(A'_n)_{n\in\mathbb N}$ che $(B_n)_{n\in\mathbb N}$) ed è un sottoinsieme di $\sqcup\mathcal S$ poichè intersezione di elementi che vi appartengono. Quindi esiste la famiglia $(F^{ij}_n)_{n\in\mathbb N}\subseteq\mathcal S$ la cui unione disgiunta realizza $C_{ij}$.
	
	Ora per costruzione e per le osservazioni fatte valgono:
	\begin{align*}
		A_n= A'_n\sqcup \bigsqcup_{i\in\mathbb N}E^n_i=\bigsqcup_{i\in\mathbb N}C_{ni}\sqcup \bigsqcup_{i\in\mathbb N}E^n_i
		=\bigsqcup_{i,j\in\mathbb N}F^{ni}_j\sqcup \bigsqcup_{i\in\mathbb N}E^n_i
		&\Longrightarrow \mu(A_n)\ge\sum_{i,j\in\mathbb N}\mu(F^{ni}_j)+\sum_{i\in\mathbb N}\mu(E^n_i)\\
		B_n=\bigsqcup_{i\in\mathbb N}C_{in}=\bigsqcup_{i,j\in\mathbb N}F^{in}_j
		&\Longrightarrow \mu(B_n)=\sum_{i,j\in\mathbb N}\mu(F^{in}_j)
	\end{align*}
	quindi sommando su $n$ arriviamo a:
	\begin{align*}
		\sum_{n\in\mathbb N}\mu(A_n)&\ge \sum_{n\in\mathbb N}\sum_{i,j\in\mathbb N}\mu(F^{ni}_j)
		=\sum_{i,n,j\in\mathbb N}\mu(F^{ni}_j)\\
		\sum_{n\in\mathbb N}\mu(B_n)&=\sum_{n\in\mathbb N}\sum_{i,j\in\mathbb N}\mu(F^{in}_j)=\sum_{i,n,j\in\mathbb N}\mu(F^{in}_j)
	\end{align*}
	ma visto che l'ordine degli indici non conta, questi risultati implicano banalmente la tesi.


\end{proof}



\begin{lemma}\label{PiuCheMonotonaPremisura}
	Dato $(X,\mathcal S,\mu)$ uno spazio di misura elementare, siano $A,(A_n)_{n\in\mathbb N}\subseteq \mathcal S$ tali che $A\subseteq\bigcup_{n\in\mathbb N}A_n$.
	Allora vale $\mu(A)\le \sum_{n\in\mathbb N}\mu(A_n)$.
\end{lemma}
\begin{proof}
	Visto che $A\subseteq\bigcup_{n\in\mathbb N}A_n$ vale la scrittura insiemistica:
	\begin{equation}\label{ScritturaDecenteUnionePremisura}
		\bigcup_{n\in\mathbb N}A_n=A\sqcup\bigcup_{n\in\mathbb N}A_n\setminus A
	\end{equation}
	Poichè $\mathcal S$ è un \semiring{} $A_n\setminus A\in \sqcup \mathcal S$, e visto che $\sqcup S$ è chiuso per unione numerabile, come mostrato in \cref{UnioneDisgiuntaQuasiAlgebra}, esiste una famiglia $(B_n)_{n\in\mathbb N}\subseteq\mathcal S$ disgiunta tale che $\bigcup_{n\in\mathbb N}A_n\setminus A=\bigsqcup_{n\in\mathbb N}B_n$.
	Allora sostituendo in \cref{ScritturaDecenteUnionePremisura} si ottiene:
	\begin{equation*}
		\bigcup_{n\in\mathbb N}A_n=A\sqcup\bigsqcup_{n\in\mathbb N}B_n
	\end{equation*}
	quindi si ricade nelle ipotesi di \cref{CoerenzaPremisura} ottenendo che:
	\begin{equation*}
		\sum_{n\in\mathbb N}\mu(A_n)\ge \mu(A)+\sum_{n\in\mathbb N}\mu(B_n)\ge \mu(A)
	\end{equation*}
	che è proprio quanto si voleva dimostrare.

\end{proof}




\section{Estendere una premisura ad una misura}
L'obiettivo ora è riuscire ad estendere una premisura definita su un semianello ad una misura su una \sigalg{}. Per fare questo il percorso sarà prima quello di estendere la premisura ad una misura esterna, per poi ridurre questa ad una misura canonica.


\begin{theorem}\label{RiduzionePreCaratheodory}
	Data $\mu:\mathcal P(X)\to \Rpiu$ una misura esterna, sia $\A\subseteq \mathcal P(X)$ l'insieme così definito:
	\begin{equation*}
		\A=\{E\in\mathcal P(X):\ \mu(A)=\mu(A\cap E)+\mu(A\setminus E)\ \forall A\in \mathcal P(X)\}
	\end{equation*}
	allora $\A$ è una \sigalg{}, detta \sigalg{} di Caratheodory, e $\mu$ ridotta su $\A$ è una misura completa.
\end{theorem}
\begin{proof}
	La dimostrazione procede in tre passi: prima mostriamo che $\A$ è un'algebra di insiemi, poi che è una \sigalg{} e infine che $\mu$ è \sigadd{} e completa ridotta su $\A$.
	
	Il fatto che $\A$ sia stabile per complementare è ovvio per la definizione (che è simmetrica tra $E$ ed $E^c$).
	
	Fissati $A\in\mathcal P(X)$ generico ed $E,F\in\A$, applicando la sola definizione di $\A$ ed alcuni passaggi insiemistici si ricava:
	\begin{align*}
		\mu(A)\stackrel{F\in\A}{=}&\mu(A\cap F)+\mu(A\setminus F)\stackrel{E\in\A}{=}
		\mu(A\cap F)+\mu\left((A\setminus F)\cap E\right)+\mu\left((A\setminus F)\setminus E\right)\\
		=\hspace{0.4em}&\mu\left((A\cap (E\cup F))\cap F\right)+\mu\left((A\cap (E\cup F))\setminus F\right)+
		\mu\left(A\setminus(E\cup F)\right)\\
		\stackrel{F\in\A}{=}&\mu(A\cap (E\cup F))+\mu\left(A\setminus(E\cup F)\right)
	\end{align*}
	e visto che questo vale per ogni scelta di $A\in\mathcal P(X)$ abbiamo dimostrato che $\A$ è stabile per unione.
	
	Unendo quanto detto si ha facilmente che $\A$ è un'algebra di insiemi.
	
	Ora sia $(E_n)_{n\in\N}\subseteq \A$ una famiglia numerabile di insiemi ed $A\in\mathcal P(X)$ un generico sottoinsieme di $X$.
	
	Per la \sigsubadd[ità] di $\mu$ vale:
	\begin{equation}\label{DisuguaglianzaFacileCaratheodory}
		\mu(A)\le \mu\left(A\cap\bigcup_{n\in\N} E_n\right)+\mu\left(A\setminus\bigcup_{n\in\N} E_n\right)
	\end{equation}
	Si vuole dimostrare che il $\le$ è in realtà un'uguaglianza. Se $\mu(A)=+\infty$ questo è ovvio, quindi tratteremo il caso in cui $\mu(A)<+\infty$. Chiamiamo $F_n=E_n\setminus \bigcup_{i<n} E_i$, ottenendo in maniera ovvia che gli $(F_n)_{n\in\N}$ sono a due a due disgiunti e che appartengono ad $\A$ poiché quest'ultima è un'algebra.
	
	Per induzione è facile verificare, sfruttando unicamente il fatto che $F_n\in\A$ e $\mu(A)<+\infty$, che risulta:
	\begin{equation}\label{IdentitaDifferenzaCaratheodory}
		\mu\left(A\setminus \bigsqcup_{n\le m} F_n\right)=\mu(A)-\sum_{n\le m} \mu(A\cap F_n)
	\end{equation}
	e incidentalmente da questa formula si ha che la serie $\sum_{n\in\N}\mu(A\cap F_n)$ converge, visto che è a termini positivi e limitata (da $\mu(A)$).
	
	Per la \sigsubadd[ità] di $\mu$ vale:
	\begin{equation}\label{IntersezioneStimaCaratheodory}
		\mu\left(A\cap\bigcup_{n\in\N} E_n\right)=\mu\left(\bigsqcup_{n\in\N} A\cap F_n\right)\le
		\sum_{n\in\N} \mu(A\cap F_n)
	\end{equation}
	mentre, grazie alla monotonia e a \cref{IdentitaDifferenzaCaratheodory}, otteniamo:
	\begin{equation}\label{DifferenzaStimaCaratheodory}
		\mu\left(A\setminus\bigcup_{n\in\N} E_n\right) = \mu\left(A\setminus\bigsqcup_{n\in\N} F_n\right) \le \mu\left(A\setminus\bigsqcup_{n\le m} F_n\right) = 
		\mu(A)-\sum_{n\le m}\mu(A\cap F_n)
	\end{equation}
	
	Ora unendo \cref{IntersezioneStimaCaratheodory,DifferenzaStimaCaratheodory} giungiamo ad avere che, per ogni $m\in\mathbb{N}$:
	\begin{equation*}
		\mu\left(A\cap\bigcup_{n\in\N} E_n\right)+\mu\left(A\setminus\bigcup_{n\in\N} E_n\right)\le
		\mu(A)+\sum_{m\le n}\mu(A\cap F_n) 
	\end{equation*}
	ma per la convergenza di $\sum_{n\in \N}\mu(A\cap F_n)$, estraendo l'$\inf$ da entrambe le parti finalmente arriviamo a:
	\begin{equation*}
		\mu\left(A\cap\bigcup_{n\in\N} E_n\right)+\mu\left(A\setminus\bigcup_{n\in\N} E_n\right)\le
		\mu(A)
	\end{equation*}
	che unita a \cref{DisuguaglianzaFacileCaratheodory} ci assicura che vale l'identità tra i membri e, visto che ciò vale indipendentemente dalla scelta di $A\in\mathcal P(X)$, risulta $\bigcup_{n\in\N}E_n\in\A$ che equivale a dire che $\A$ è una \sigalg{}.
	
	Dimostrare che $\mu$ è \sigadd{} su $\A$ è ora molto facile.
	Consideriamo $(E_n)_{n\in\N}\subseteq \A$ una famiglia numerabile di insiemi \emph{disgiunti}. Per facile induzione si ha che:
	\begin{equation*}
		\mu\left(\bigsqcup_{n\le m}E_n\right)=\sum_{n\le m} \mu(E_n)
	\end{equation*}
	e applicando questa e la monotonia di $\mu$ risulta:
	\begin{equation*}
		\sum_{n\le m} \mu(E_n)=\mu\left(\bigsqcup_{n\le m}E_n\right)\le
		\mu\left(\bigsqcup_{n\in\N}E_n\right)\le \sum_{n\in\N} \mu(E_n)
	\end{equation*}
	e questa doppia disuguaglianza, per la definizione delle serie a termini positivi, implica che tutte le disuguaglianze sono identità. Ma allora questo dimostra che $\mu$ è \sigadd{} su $\A$.
	
	Infine per dimostrare la completezza di $\mu|_{\A}$ basta mostrare che dato $E\in\mathcal P(X)$ trascurabile, vale $E\in\A$ (questo è sufficiente a mostrare la completezza, visto che per monotonia i sottinsiemi di un trascurabile sono a loro volta trascurabili).
	
	Fissato un generico $A\in\mathcal P(X)$, risulta per la monotonia di $\mu$:
	\begin{equation*}
		\mu(A\cap N)+\mu(A\setminus N)\le \mu(N)+\mu(A)=\mu(A)
	\end{equation*}
	che, unita alla \sigsubadd[ità] di $\mu$ mi assicura
	\begin{equation*}
		\mu(A)=\mu(A\cap N)+\mu(A\setminus N)
	\end{equation*}
	che è proprio la condizione di appartenenza ad $\A$.
\end{proof}

\begin{proposition}\label{MisuraEsternaDiPremisura}
	Dato $(X,\S,\mu)$ uno spazio di misura elementare si consideri la funzione che associa ad ogni sottoinsieme l'estremo inferiore delle misure dei ricoprimenti, cioè $\mu^*:\mathcal P(X)\to\Rpiu$ definita come 
	\begin{equation*}
		\mu^*(A)=\inf\left\{\sum_{n\in\N} \mu(A_n)\ |\ (A_n)_{n\in\N}\subseteq\S\ \wedge
		\ A\subseteq\bigcup_{n\in\N}A_n\right\}
	\end{equation*}
	Allora $\mu^*$ è una misura esterna che estende $\mu$ (cioè $\mu^*|_\S=\mu$) ed inoltre $\S$ appartiene alla relativa \sigalg{} di Caratheodory (come definita in \cref{RiduzionePreCaratheodory}).
\end{proposition}
\begin{proof}
	Per affermare che $\mu^*$ è una misura esterna sono sufficienti le seguenti verifiche.
	Ovviamente, poiché $\mu(\emptyset)=0$, vale $\mu^*(\emptyset)=0$. 
	Inoltre, ancora facilmente, $\mu^*$ è monotona, visto che se $A\subseteq B$ un ricoprimento di $B$ ricopre anche $A$.
	E infine è anche \sigsubadd{} visto che l'unione di ricoprimenti (che risulta ancora un ricoprimento numerabile) è un ricoprimento dell'unione.
	
	Dato $S\in\S$ vale ovviamente $\mu^*(S)\le\mu(S)$, poiché $S$ si ricopre da solo. Per dimostrare la disuguaglianza opposta, ottenendo così che $\mu^*$ estende $\mu$, consideriamo $(S_n)_{n\in\N}\in \S$ un ricoprimento di $S$. Per \cref{PiuCheMonotonaPremisura} vale:
	\begin{equation*}
		\mu(S)\ge \sum_{n\in\N} \mu(S_n)
	\end{equation*}
	e perciò passando all'estremo inferiore sui ricoprimenti otteniamo la disuguaglianza cercata.
	
	Ora perciò resta da dimostrare che se $E\in \S$ allora per ogni $A\in\mathcal P(X)$ risulta:
	\begin{equation}\label{MisuraEsternaDisDifficile}
		\mu^*(A) \ge \mu^*(A\cap E)+\mu^*(A\setminus E)
	\end{equation}
	Questo è sufficiente ad avere che $\S$ è contenuto nella \sigalg{} di Caratheodory, poiché l'altra disuguaglianza è assicurata dalla \sigsubadd[ità].
	
	Dato $(A_n)_{n\in\N}\subseteq\S$ un ricoprimento di $A$, chiamiamo $B_n=A_n\cap E$ e $C_n=A_n\setminus E$. Ovviamente $(B_n)_{n\in\N},(C_n)_{n\in\N}$ ricoprono rispettivamente $A\cap E,A\setminus E$. Poiché $\S$ è un \semiring{} riusciamo però a trovare $(B'^n_i)_{i\in\N},(C'^n_i)_{i\in\N} \subseteq \S$ tali che $B_n=\bigsqcup_{i\in\N}B'^n_i$ e analogo risultato per $C_n$. Quindi $(B'^n_i)_{n,i\in\N}, (C'^n_i)_{n,i\in\N}$ risultano ricoprimenti con elementi di $\S$ di $A\cap E,A\setminus E$ rispettivamente.
	Ora, sfruttando non più della sola \sigadd[ità] di $\mu$ concludiamo:
	\begin{align*}
		\sum_{n\in\N}\mu(A_n)=\sum_{n\in\N} \mu(B_n)+\mu(C_n)&=
		\sum_{n\in\N}\sum_{i\in\N}\mu(B'^n_i)+\mu(C'^n_i)\\
		&=
		\sum_{n,i\in\N}\mu(B'^n_i)+\sum_{n,i\in\N}\mu(C'^n_i)\ge \mu(A\cap E)+\mu(A\setminus E)
	\end{align*}
	ma questo implica facilmente \cref{MisuraEsternaDisDifficile} estraendo l'estremo inferiore a entrambi i membri sui ricoprimenti di $A$.
\end{proof}

\begin{theorem}[Estensione di Caratheodory]\label{EstensioneCaratheodory}
	Dato $(X,\S,\mu)$ uno spazio di misura elementare esiste una \sigalg{} $\A$ e una funzione $\mu':\A\to\Rpiu$ tali che $\S\subseteq \A$, $\mu'$ estende la premisura $\mu$ e $(X,\A,\mu')$ è uno spazio di misura completo.
\end{theorem}
\begin{proof}
	Consideriamo la misura esterna $\mu^*:\mathcal P(X)\to\Rpiu$ definita nell'enunciato di \cref{MisuraEsternaDiPremisura}. Sempre \cref{MisuraEsternaDiPremisura} ci assicura che questa è un'estensione di $\mu$.
	
	Possiamo ora ridurre $\mu^*$ grazie al \cref{RiduzionePreCaratheodory} ad una misura completa $\mu':\A\to\Rpiu$ dove $\A$ è la \sigalg{} di Caratheodory. 
	
	Ma come dimostrato in \cref{MisuraEsternaDiPremisura} $\S\subseteq\A$ e inoltre vale $\mu'|_\S=\mu^*|_\S=\mu$, perciò lo spazio $(X,\A,\mu')$ rispetta tutte le richieste dell'enunciato.
\end{proof}


\begin{proposition}
	Dato $(X,\S,\mu)$ una spazio di misura elementare $\sigma$-finito, sia $\mu^*$ la misura esterna associata a $\mu$ (come definita nell'enunciato di \cref{MisuraEsternaDiPremisura}) e $\A$ la \sigalg{} di Caratheodory (la cui esistenza mi è assicurata dal \cref{EstensioneCaratheodory}) con la relativa estensione della misura $\mu':\A\to\Rpiu$.
	
	Scelto $A\in\mathcal P(X)$, le seguenti affermazioni sono equivalenti:
	\begin{enumerate}[label=(\arabic*),ref=(\arabic*)]
		\item $A\in\A$ cioè l'insieme è un misurabile (secondo Caratheodory).\label{MisurabileEquivalenze}
		\item Fissato $\epsilon>0$ esiste $S\in\sqcup\S$ che contiene $A$ e tale che $\mu^*(S\setminus A)<\epsilon$.\label{UnioniDaFuoriEquivalenze}
		\item Esiste $B\in\sigma A(S)$ che $B$ che contiene $A$ e tale che $\mu^*(B\setminus A)=0$.\label{SigmaDaFuoriEquivalenze}
		\item Fissato $\epsilon>0$ esiste $S$ contenuto in $A$ tale che $S^c\in\sqcup\S$  e $\mu^*(A\setminus S)<\epsilon$.\label{UnioniDaDentroEquivalenze}
		\item Esiste $B\in\sigma A(S)$ che è contenuto in $A$ e tale che $\mu^*(A\setminus B)=0$.\label{SigmaDaDentroEquivalenze}
	\end{enumerate}
\end{proposition}
\begin{proof}
	Dimostriamo innanzitutto la catena di implicazioni
	$\text{\ref{MisurabileEquivalenze}}\implies
	\text{\ref{UnioniDaFuoriEquivalenze}}\implies
	\text{\ref{SigmaDaFuoriEquivalenze}}\implies
	\text{\ref{MisurabileEquivalenze}}$.
	\newcommand{\ImplicationProof}[2]{$\text{\ref{#1}}\implies\text{\ref{#2}}$}% X => Y
	\begin{description}
		\item[\ImplicationProof{MisurabileEquivalenze}{UnioniDaFuoriEquivalenze}] Sia $A$ misurabile e $(X_n)_{n\in\N}\subseteq \S$ una partizione numerabile di $X$ tale che $\mu(X_n)<+\infty$ (questa esiste poichè $\mu$ è \sigfin{} e ogni unione numerabile di elementi in $\S$ è un'unione disgiunta, visto che $\sqcup\S$ è chiuso per unioni numerabili come mostrato in \cref{UnioneDisgiuntaQuasiAlgebra}).
		
		Definiamo $A_n=A\cap X_n$. Gli $A_n$ sono misurabili, perchè intersezioni di misurabili, e hanno misura finita dato che la misura è monotona.
		
		Sia $\epsilon>0$ fissato.
		
		Dato $n\in\N$ allora $\mu'(A_n)=\mu^*(A_n)$, ma per la definizione di $\mu^*$ questo implica che esiste $S_n\in\sqcup \S$ tale che $A_n\subseteq S_n$ e tale che valga:
		\begin{equation*}
			\mu^*(A_n)\le \mu^*(S_n) \le \mu^*(A_n)+\frac\epsilon{2^n}\implies
			0\le \mu^*(S_n)-\mu^*(A_n)\le \frac\epsilon{2^n}
		\end{equation*}
		dove nell'implicazione abbiamo sfruttato $\mu^*(A_n)=\mu'(A_n)<+\infty$.
		Infine dal fatto che $S_n,A_n$ sono misurabili, per l'addittività della misura, ricaviamo:
		\begin{equation}\label{DifficileFreccia12Misurabili}
			0\le \mu'(S_n\setminus A_n) \le \frac\epsilon{2^n} \implies \mu^*(S_n\setminus A_n)\le \frac\epsilon{2^n}
		\end{equation}
		
		Ora consideriamo $S=\bigcup_{n\in\N}S_n$, che appartiene ancora a $\sqcup\S$ poichè questo è chiuso per unione numerabile come dimostrato in \cref{UnioneDisgiuntaQuasiAlgebra}. Si ha $A\subseteq S$ e per la subaddittività della misura esterna  $\mu^*$, sfruttando \cref{DifficileFreccia12Misurabili}, si ottiene:
		\begin{equation*}
			\mu^*(S\setminus A)=\mu^*\left(\bigcup_{n\in\N}S_n\setminus A_n\right)\le
			\sum_{n\in\N}\mu^*(S_n\setminus A_n)\le\sum_{n\in\N}\frac\epsilon{2^n}\le \epsilon
		\end{equation*}
		e questo conclude, visto che $S$ soddisfa tutte le richieste dell'enunciato.
		\item[\ImplicationProof{UnioniDaFuoriEquivalenze}{SigmaDaFuoriEquivalenze}] L'ipotesi ci assicura l'esistenza di $(S_n)_{n\in\N}\subseteq \sqcup \S$ tali che $A\subseteq S_n$ e $\mu^*(S_n\setminus A)<\frac1n$.
		
		Definiamo $B=\cap_{n\in\N}S_n$, che appartiene a $\sigma A(\S)$ poichè generato a partire da $\S$ con sole unioni e intersezioni numerabili. Ovviamente vale $A\subseteq B$.
		
		Infine risulta vera la seguente:
		\begin{equation*}
			\forall n\in\N:\ B\subseteq B_n\implies B\setminus A\subseteq B_n\setminus A\implies \mu^*(B\setminus A)\le \mu^*(B_n\setminus A)\le \frac1n
		\end{equation*}
		e perciò $\mu^*(B\setminus A)=0$ e questo dimostra l'implicazione cercata.
		\item[\ImplicationProof{SigmaDaFuoriEquivalenze}{MisurabileEquivalenze}] Per ipotesi esiste $B\in\sigma A(\S)$ che contiene $A$ e tale che $\mu^*(B\setminus A)=0$.
		
		Poichè $\A$ è una \sigalg{} contenente $\S$ e $B\in\sigma A(\S)$, deve essere $B\in\A$. 
		Inoltre $\mu':\A\to\Rpiu$ è una misura completa, come visto nel \cref{RiduzionePreCaratheodory}, e inoltre i trascurabili secondo $\mu^*$ risultano essere proprio i trascurabili secondo $\mu'$ (come visto nella dimostrazione di \cref{RiduzionePreCaratheodory}) quindi anche $B\setminus A\in \A$ in quanto è un trascurabile e la misura è completa. 
		
		Di conseguenza, visto che $\A$ è una \sigalg, otteniamo che $A=B\setminus (B\setminus A)$ è misurabile in quanto differenza di misurabili. Perciò $A\in\A$ dimostrando l'implicazione.
	\end{description}
	
	Infine basta vedere, sfruttando il passaggio al complementare, che risultano $\text{\ref{UnioniDaFuoriEquivalenze}}\iff\text{\ref{UnioniDaDentroEquivalenze}}$ e $\text{\ref{SigmaDaFuoriEquivalenze}}\iff\text{\ref{SigmaDaDentroEquivalenze}}$ e questo chiude la dimostrazione.
\end{proof}

\section{Misura di Lebesgue}
Ora applicheremo i risultati astratti ottenuti nelle due precedenti sezioni al caso più tangibile della retta reale e dello spazio vettoriale $\R^n$.

Definiremo la misura di Lebesgue e, oltre a chiarire come mai questa sia la misura più naturale su $\R^n$, studieremo i misurabili secondo Lebesgue mostrando sia che non coincidono con la \sigalg{} dei Boreliani (cioè la \sigalg{} generata dagli aperti) sia che non coincidono con le parti di $\R^n$.

Nella prima parte di questa sezione, quella riguardante degli aspetti fondamentalmente combinatorici degli $n$-cubi di $\R^n$ lasceremo due dimostrazioni al lettore.
Queste vengono omesse in quanto, oltre ad essere molto pesanti notazionalmente e molto facili intuitivamente, non aggiungono nulla alla comprensione che si mira ad avere della misura di Lebesgue.


\begin{definition}\label{def:LebesgueSemiaperti}
	Indicheremo con $\S_n\subseteq\mathcal P(\R^n)$ l'insieme dei parallelepipedi $n$-dimensionali semiaperti a destra\footnote{Se $a\ge b$ con la scrittura $\co{a}{b}$ si intenderà l'insieme vuoto.}:
	\begin{equation*}
		\S_n=\left\{\co{a_1}{b_1}\times\cdots\times\co{a_n}{b_n}:\ (a_i)_{1\le i\le n},(b_i)_{1\le i\le n}\subseteq \R\right\}
	\end{equation*}
	
	Inoltre, dato $S=\co{a_1}{b_1}\times\cdots\times\co{a_n}{b_n}$, definiamo $S^-_i=a_i$ e $S^+_i=b_i$.
\end{definition}

\begin{proposition}\label{prop:SpaccareUnioneSemiaperti}
	Dati $A,B\in\S_n$ esiste una famiglia finita $(S_i)_{i\in I}\subseteq\S_n$ di insiemi disgiunti la cui unione disgiunta dà $A\cup B$ e tale che per ogni $i\in I$ vale una, ed una sola, delle seguenti:
	\begin{align*}
		S_i\subseteq &A\setminus B\\
		S_i\subseteq &B\setminus A\\
		S_i\subseteq &A\cap B
	\end{align*}
\end{proposition}
\begin{proof}
	Lasciata al lettore.
\end{proof}

\begin{lemma}\label{lem:SemianelloSemiAperti}
	La famiglia $\S_n$ è un \semiring{}.
\end{lemma}
\begin{proof}
	Ovviamente $\emptyset\in\S_n$.
	Inoltre dati $A,B\in\S_n$ si ricava facilmente che
	\begin{equation*}
		A\cap B=\co{\max(A^-_1,B^-_1)}{\min(A^+_1,B^+_1)}\times\cdots\times\co{\max(A^-_n,B^-_n)}{\min(A^+_n,B^+_n)}\in\S_n
	\end{equation*}

	Mentre per la differenza $A\setminus B$ applichiamo la \cref{prop:SpaccareUnioneSemiaperti} sugli insiemi $A,B$ per ottenere la famiglia $(S_i)_{i\in I}\subseteq \S_n$ come descritta dall'enunciato.
	Ora basta considerare la famiglia
	\begin{equation*}
		\mathcal F=\left\{S_i:\ i\in I\wedge S_i\subseteq A\setminus B\right\}
	\end{equation*}
	e notare che, viste le proprietà che ha la famiglia degli $(S_i)_{i\in I}$, l'unione disgiunta degli elementi della famiglia $\mathcal F$ è proprio $A\setminus B$.
	
	Ma visto che sia intersezione che differenza si scrivono come unione disgiunta di elementi di $\S_n$ abbiamo dimostrato che $\S_n$ è un \semiring{}.
\end{proof}

\begin{definition}\label{def:LebesgueElementare}
	Indicherò con $m_n:\S_n\to\Rpiu$ la funzione che associa ad ogni parallelepipedo il suo volume $n$-dimensionale:
	\begin{equation*}
		m_n(S)=\prod_{i=1}^n\max(0,S^+_i-S^-_i)
	\end{equation*}
\end{definition}
\begin{remark}\label{nota:LebesgueElementareProprieta}
	La funzione di insiemi $m_n$ è invariante per traslazione, cioè risulta che per ogni $S\in\S_n$ e $v\in\R^n$ vale\footnote{Definiamo lo shift di un insieme come $A+k=\{a+k:\ a\in A\}$.}
	\begin{equation*}
		m_n(S)=m_n(S+v)
	\end{equation*}
	ed è anche $n$-omogenea, cioè per ogni $\lambda>0$ e $S\in\S_n$ vale\footnote{Definisco la moltiplicazione per scalare come $\lambda S=\{\lambda s:\ s\in S\}$}
	\begin{equation*}
		m_n(\lambda S)=\lambda^n m_n(S)
	\end{equation*}
\end{remark}
\begin{proof}
	Entrambe le proprietà sono di facile verifica:
	\begin{multline*}
		m_n(S+v)=m_n\left(\co{S^-_1+v_1}{S^+_1+v_1}\times\cdots\times\co{S^-_n+v_n}{S^+_n+v_n}\right)\\
		=\prod_{i=1}^n\left((S^+_i+v_i)-(S^-_i+v_i)\right)=\prod_{i=1}^n\left(S^+_i-S^-_i\right)=m_n(S)
	\end{multline*}
	
	\begin{multline*}
		m_n(\lambda S)=m_n\left(\co{\lambda S^-_1}{\lambda S^+_1}\times\cdots\times\co{\lambda S^-_n}{\lambda S^+_n}\right)\\
		=\prod_{i=1}^n\left(\lambda S^+_i-\lambda S^-_i\right)=\lambda^n\prod_{i=1}^n\left(S^+_i-S^-_i\right)=\lambda^n m_n(S)
	\end{multline*}
\end{proof}


\begin{lemma}\label{lem:LebesgueElementareFinita}
	Dati $(S_i)_{1\le i\le k}\subseteq \S_n$ e $S\in\S_n$, valgono le seguenti disuguaglianze:
	\begin{itemize}
		\item Se gli $S_i$ sono disgiunti e $\bigsqcup_{i=1}^k S_i\subseteq S$ allora risulta
		\begin{equation*}
			\sum_{i=1}^k m_n(S_i)\le m_n(S)
		\end{equation*}
		\item Se vale il contenimento $S\subseteq\bigcup_{i=1}^k S_i$ allora risulta
		\begin{equation*}
			\sum_{i=1}^k m_n(S_i)\ge m_n(S)
		\end{equation*}
	\end{itemize}
\end{lemma}
\begin{proof}
	Lasciata al lettore.
\end{proof}

\begin{definition}\label{def:AllargamentoSemiaperti}
	Fissato $\lambda>0$ definiamo $F_\lambda:\S_n\to\S_n$ come l'operatore che associa ad $S\in\S_n$ l'insieme nullo se $S=\emptyset$ e altrimenti
	\begin{equation*}
		F_\lambda(S)=\co{S^-_1-\epsilon}{S^+_1}\times\cdots\times\co{S^-_n-\epsilon}{S^+_n}
	\end{equation*}
	dove $\epsilon>0$ è definito come
	\begin{equation*}
		\epsilon=\min\left(1,\frac{\lambda}{2^n\cdot m_n(S)}\right)\cdot\min_{1\le i\le n}\{S^+_i-S^-_i\}
	\end{equation*}
\end{definition}
\begin{remark}\label{nota:ParteInternaAllargamento}
	Segue banalmente dalla definizione che $S\in\S_n$ è un sottoinsieme della parte interna di $F_\lambda(S)$ per ogni $\lambda>0$.
\end{remark}

\begin{proposition}\label{prop:MisuraAllargamento}
	Fissati $\lambda>0$ e $S\in\S_n$ vale la seguente stima:
	\begin{equation*}
		m_n(S)\le m_n(F_\lambda(S))\le m_n(S)+\lambda
	\end{equation*}
\end{proposition}
\begin{proof}
	La prima disuguaglianza è ovvia.
	
	La seconda è ovvia se $S$ è vuoto, quindi assumiamo $S\not=\emptyset$.
	
	Ponendo $\epsilon$ come nella \cref{def:AllargamentoSemiaperti}, risulta vero
	\begin{multline*}
		\frac{m_n(F_\lambda(S))}{m_n(S)}=\prod_{i=1}^n\frac{S^+_i-S^-_i+\epsilon}{S^+_i-S^-_i}=
		\prod_{i=1}^n\left(1+\frac{\epsilon}{S^+_i-S^-_i}\right)\\\le
		\prod_{i=1}^n\left(1+\min\left(1,\frac{\lambda}{2^n\cdot m_n(S)}\right)\right)\le
		1+2^n\frac{\lambda}{2^n\cdot m_n(S)}=1+\frac{\lambda}{m_n(S)}
	\end{multline*}
	che implica la tesi moltiplicando ambo i membri per $m_n(S)$.
\end{proof}

\begin{theorem}\label{thm:LebesguePremisura}
	La terna $(\R^n,\S_n,m_n)$ è uno spazio di misura elementare.
\end{theorem}
\begin{proof}
	Per quanto mostrato nel \cref{lem:SemianelloSemiAperti} la famiglia $\S_n$ è un \semiring{}.
	
	Resta da dimostrare solo che $m_n$ è \sigadd{} su $\S_n$.
	Perciò fisso $(S_i)_{i\in\N}\subseteq \S_n$ disgiunti tali che la loro unione disgiunta dà $S\in\S_n$.
	
	Applicando il \cref{lem:LebesgueElementareFinita} abbiamo facilmente
	\begin{equation*}
		\forall k\in\N:\ \sqcup_{i=1}^k S_i\subseteq S\implies \forall k\in\N:\ \sum_{i=1}^k m_n(S_i)\le m_n(S) \implies \sum_{i\in\N}m_n(S_i)\le m_n(S)
	\end{equation*}

	
	Per ottenere la disuguaglianza opposta, concludendo quindi la dimostrazione, sfrutteremo la compattezza dei chiusi e limitati di $\R^n$, in particolare la possibilità di estrarre ricoprimenti finiti di aperti a partire da ricoprimenti numerabili.
	
	Fissiamo $\epsilon>0$ arbitrario.
	
	Grazie alla \cref{nota:ParteInternaAllargamento} abbiamo che la successione composta dalle parti interne degli elementi della successione $\left(F_{\frac\epsilon{2^i}}(S_i)\right)_{i\in\N}$ è un ricoprimento del compatto $S$.
	Perciò esiste un insieme finito di indici $I\subseteq \N$ tale che $\left(F_{\frac\epsilon{2^i}}(S_i)\right)_{i\in I}$ è un ricoprimento finito di $S$.
	
	Sfruttando ancora il \cref{lem:LebesgueElementareFinita} e applicando la \cref{prop:MisuraAllargamento} otteniamo
	\begin{equation*}
		\sum_{i\in\N}m_N(S_i)\ge \sum_{i\in I}m_N(S_i)\ge \sum_{i\in I}\left(m_n\left(F_{\frac\epsilon{2^i}}(S_i)\right)-\frac\epsilon{2^i}\right)
		\ge -\epsilon+\sum_{i\in I}m_n\left(F_{\frac\epsilon{2^i}}(S_i)\right)\ge m_n(S)-\epsilon
	\end{equation*}
	e visto che questo vale per ogni $\epsilon>0$ ne ricaviamo:
	\begin{equation*}
		\sum_{i\in\N}m_N(S_i)\ge m_n(S)
	\end{equation*}
	che conclude la dimostrazione della \sigadd[ità] di $m_n$.
\end{proof}

\begin{proposition}\label{prop:LebesguePremisuraSigFin}
	Lo spazio di misura elementare $(\R^n,\S_n,m_n)$ è \sigfin[o].
\end{proposition}
\begin{proof}
	Sfruttando che $\mathbb Z^n$ è numerabile, mostriamo esplicitamente la \sigfin[ezza]:
	\begin{equation*}
		\R^n=\bigsqcup_{i_1,i_2,\dots,i_n\in\mathbb Z^n} \co{i_1}{i_1+1}\times\cdots \times\co{i_n}{i_n+1}
	\end{equation*}
	
\end{proof}

\begin{proposition}\label{prop:ApertiUnioneDiSemiaperti}
	Ogni aperto di $\R^n$ si scrive come unione numerabile disgiunta di elementi di $\S_n$.
\end{proposition}
\begin{proof}
	Dato un aperto $A$, consideriamo tutti gli $n$-cubi con centro razionale e lato razionale contenuti in $A$.
	È facile vedere che l'unione di questi, che è ovviamente numerabile visto che $\mathbb Q^2$ lo è, è tutto l'aperto.
	
	Per ottenere che si può scrivere $A$ come unione disgiunta è sufficiente applicare la \cref{UnioneDisgiuntaQuasiAlgebra} ricordando che $\S_n$ è un \semiring{} come dimostrato nel \cref{lem:SemianelloSemiAperti}.
\end{proof}

\begin{definition}\label{def:Boreliani}
	I Boreliani di $\R^n$ sono la \sigalg{} generata dai sottoinsiemi aperti di $\R^n$.
\end{definition}

\begin{proposition}\label{prop:SigAlgUgualeBoreliani}
	La \sigalg{} generata da $\S_n$ coincide con i Boreliani di $\R^n$.
\end{proposition}
\begin{proof}
	Per quanto mostrato nella \cref{prop:ApertiUnioneDiSemiaperti} di certo la \sigalg{} generata da $\S_n$ contiene gli aperti di $\R^n$.
	Ma allora la \sigalg{} generata dagli aperti e quella generata da $\S_n$ coincidono e si ottiene la tesi.
\end{proof}

Ora che abbiamo ottenuto una premisura su $\R^n$, applicheremo il teorema di Caratheodory per ricavarne quindi una misura su $\R^n$. 
Continueremo dimostrando le proprietà dei misurabili in questa nuova misura per poi caratterizzarla come l'unica misura che coincida con la nozione intuitiva di volume, cioè sia invariante per traslazione e sia normalizzata sul cubo unitario.

\begin{definition}\label{def:LebesgueMisura}
	Dato lo spazio $(\R^n,\S_n,m_n)$, che si è mostrato essere di misura elementare nel \cref{thm:LebesguePremisura}, sia $\M_n$, che verrà chiamato l'insieme dei misurabili secondo Lebesgue, la relativa \sigalg{} di Caratheodory e $m_n:\M_n\to\Rpiu$ \footnote{Qui abusiamo leggermente di notazione, visto che con $m_n$ si indicava la premisura.}, che verrà chiamata misura di Lebesgue, la misura associata, la cui esistenza ci è assicurata dal \cref{EstensioneCaratheodory}. 
	Inoltre indicheremo con $m_n^*:\mathcal P(\R^n)\to\Rpiu$ la misura esterna associata a $m_n$, che quindi ridotta su $\M_n$ coincide con la misura di Lebesgue.
\end{definition}

\begin{remark}\label{nota:LebesgueCompletezza}
	La misura di Lebesgue su $\R^n$ è completa.
\end{remark}
\begin{proof}
	È un'ovvia conseguenza del \cref{EstensioneCaratheodory}.
\end{proof}

\begin{remark}\label{nota:LebesgueSigFin}
	La misura di Lebesgue è \sigfin{}.
\end{remark}
\begin{proof}
	È una banale conseguenza della \cref{prop:LebesguePremisuraSigFin}
\end{proof}

\begin{proposition}\label{prop:CompletamentoBoreliani}
	I Boreliani di $\R^n$ sono un sottoinsieme di $\M_n$ ed in particolare il loro completamento rispetto alla misura di Lebesgue è proprio $\M_n$.
\end{proposition}
\begin{proof}
	Il \cref{EstensioneCaratheodory} ci assicura che la \sigalg{} $\M_n$ dei Boreliani contiene la \sigalg{} generata da $\S_n$ e perciò, sfruttando \cref{prop:SigAlgUgualeBoreliani}, otteniamo i Boreliani sono misurabili secondo Lebesgue come voluto.
	
	Per avere che il completamento dei Boreliani coincide con $\M_n$ basta applicare \cref{prop:CaratheodoryCompletamentoSigAlg} ricordando \cref{prop:LebesguePremisuraSigFin}.
\end{proof}

\begin{theorem}\label{thm:LebesgueEquivalenzeMisurabilita}
	Dato $A\subseteq \R^n$ sono equivalenti le seguenti proposizioni:
	\begin{enumerate}[label=(\arabic*),ref=(\arabic*)]
		\item $A$ è misurabile secondo Lebesgue.\label{it:LEMMisurabile}
		\item Fissato $\epsilon>0$ esiste $C$ aperto che contiene $A$ e tale che $m_n^*(C\setminus A)<\epsilon$. \label{it:LEMApertoFuori}
		\item Fissato $\epsilon>0$ esiste $C$ chiuso contenuto in $A$ e tale che $m_n^*(A\setminus C)<\epsilon$. \label{it:LEMChiusoDentro}
		\item Esiste $B$ Boreliano che contiene $A$ e tale che $m_n^*(B\setminus A)=0$. \label{it:LEMBorelFuori}
		\item Esiste $B$ Boreliano contenuto in $A$ e tale che $m_n^*(A\setminus B)=0$. \label{it:LEMBorelDentro}
		\item Esiste una successione $(K_i)_{i\in\N}$ di compatti di $\R^n$ tutti contenuti in $A$, che si contengono in maniera crescente $K_i\subseteq K_{i+1}$ e la cui unione rispetta $m_n^*\left(A\setminus\bigcup_{i\in\N}K_i\right)=0$. \label{it:LEMCompattiDentro}
	\end{enumerate}
\end{theorem}
\begin{proof}
	È chiaro che questo teorema richiama, specializzandola al caso della misura di Lebesgue, il \cref{thm:EquivalenzeMisurabilitaSottoinsieme}.
	
	In particolare applicando il \cref{thm:EquivalenzeMisurabilitaSottoinsieme} si ha da subito che \ref{it:LEMMisurabile},\ref{it:LEMBorelFuori},\ref{it:LEMBorelDentro} sono equivalenti (poichè i Boreliani sono la \sigalg{} generata da $\S_n$ come mostrato nella \cref{prop:SigAlgUgualeBoreliani}) e anche che \ImplicationProof{it:LEMApertoFuori}{it:LEMMisurabile} poichè per quanto dimostrato nella \cref{prop:ApertiUnioneDiSemiaperti} un aperto è unione disgiunta di elementi di $\S_n$.
	
	Ancora per il \cref{thm:EquivalenzeMisurabilitaSottoinsieme}, assumendo la \ref{it:LEMMisurabile}, sappiamo che esistono $(S_i)_{i\in\N}\subseteq \S_n$ disgiunti, la cui unione $S$ contiene $A$ ed è tale che $m_n^*(S\setminus A)<\epsilon$.
	Allora definendo $C$ come la parte interna di $\bigcup_{i\in\N}F_\frac{\epsilon}{2^i}(S_i)$ abbiamo in effetti che, grazie alla \cref{prop:MisuraAllargamento}, $m_n^*(C\setminus A)\le \epsilon$.
	Unendo quanto detto e sfruttando la \sigsubadd[ità] della misura esterna si ottiene proprio $m_n^*(C\setminus A)<\epsilon$ e questo conclude la dimostrazione di \ImplicationProof{it:LEMMisurabile}{it:LEMApertoFuori}.
	
	Abbiamo quindi dimostrato che \ref{it:LEMMisurabile},\ref{it:LEMApertoFuori},\ref{it:LEMBorelFuori},\ref{it:LEMBorelDentro} sono tutte equivalenti.
	
	Per concludere dimostriamo la serie di implicazioni \ImplicationProof{it:LEMMisurabile}{it:LEMChiusoDentro}$\,\implies\,\,$\ImplicationProof{it:LEMCompattiDentro}{it:LEMBorelDentro}.
	\begin{description}
		\item[\ImplicationProof{it:LEMMisurabile}{it:LEMChiusoDentro}] Visto che $A\in\M_n$, allora anche il suo complementare è misurabile e visto che abbiamo già mostrato \ImplicationProof{it:LEMMisurabile}{it:LEMApertoFuori} otteniamo che esiste $D$ aperto tale che $A^\mathsf{c}\subseteq D$ e $m_n^*(D\setminus A^c)< \epsilon$.
		
		Allora chiamando $C$ il chiuso complementare di $D$ risulta banale verificare che $C$ rispetta le richieste di \ref{it:LEMChiusoDentro}.
		\item[\ImplicationProof{it:LEMChiusoDentro}{it:LEMCompattiDentro}] Sia $C_i$ un chiuso tale che $C_i\subseteq A$ e $m_n^*(A\setminus C_i)<\frac1i$.
		Chiamiamo $K_i$ l'insieme ottenuto intersecando la palla chiusa di raggio $i$ centrata nell'origine con il chiuso $\bigcup_{j\le i}C_i$.
		
		I $K_i$ sono compatti, poichè definiti come intersezione tra un chiuso ed un chiuso e limitato. 
		Ancora per definizione $K_i\subseteq K_{i+1}$ e inoltre l'unione dei $K_i$ coincide con l'unione dei $C_i$.
		
		Per concludere è sufficiente perciò verificare che l'unione dei $C_i$, oltre ad essere ovviamente contenuta in $A$, lo approssima a meno di un trascurabile.
		La verifica risulta agevole sfruttando la monotonia della misura esterna:
		\begin{equation*}
			\forall i_0\in\N:\ m_n^*(A\setminus \bigcup_{i\in\N} C_i)\le m_n^*(A\setminus C_{i_0})< \frac 1{i_0}\implies m_n^*(A\setminus \bigcup_{i\in\N} C_i)=0
		\end{equation*}

		\item[\ImplicationProof{it:LEMCompattiDentro}{it:LEMBorelDentro}] Presi i compatti descritti dalla proposizione, l'unione è ovviamente un Boreliano e rispetta le richieste della \ref{it:LEMBorelDentro}.
	\end{description}

\end{proof}

\begin{proposition}\label{prop:LebesgueUnicaEstensione}
	La misura di Lebesgue è l'unica possibile estensione alla \sigalg{} dei Boreliani della premisura $m_n$ definita su $\S_n$.
\end{proposition}
\begin{proof}
	È una banale conseguenza della \cref{UnicitaCaratheodory}, che si può applicare ricordando la \cref{prop:LebesguePremisuraSigFin} e notando che la \sigalg{} generata da $\S_n$ sono i Boreliani come mostrato nella \cref{prop:SigAlgUgualeBoreliani}.
\end{proof}

\begin{remark}\label{nota:LebesgueProprieta}
	La misura di Lebesgue in $\R^n$ ha le seguenti proprietà:
	\begin{itemize}
		\item È invariante per traslazione.
		\item È $n$-omogenea.
	\end{itemize}
\end{remark}
\begin{proof}
	Ricordiamo innanzitutto che la misura di Lebesgue è stata costruita appplicando il teorema di Caratheodory e perciò nasce come la riduzione della misura esterna associata a $m_n$ (questa volta da vedere come ridotta su $\S_n$). 
	Basta quindi mostrare che la misura esterna rispetta le proprietà richieste.
	
	Ma è chiaro che le proprietà richieste, poichè gli operatori in gioco commutano con l'unione, vengono ereditate dalla misura esterna se la premisura le rispetta.
	
	Ma nel nostro caso la premisura le rispetta come dimostrato nella \cref{nota:LebesgueElementareProprieta} e questo chiude la dimostrazione.
\end{proof}

\begin{theorem}\label{thm:LebesgueUnicaInvarianteTraslazione}
	Se $\mu$ è una misura sui Boreliani di $\R^n$ invariante per traslazione tale che la misura del cubo unitario valga $1$, cioè
	\begin{equation*}
		\mu\left(\co01\times\co01\times\cdots\times\co01\right)=1
	\end{equation*}
	allora $\mu\equiv m_n$, in altre parole la misura di Lebesgue è l'unica misura con queste proprietà (l'invarianza per traslazione è dimostrata nella \cref{nota:LebesgueProprieta}).
\end{theorem}
\begin{proof}
	L'idea della dimostrazione è provare che $\mu$ e $m_n$ coincidono sugli elementi di $\S_n$ con estremi interi, poi con estremi razionali ed infine con estremi generici.
	Una volta ottenuto questo, basterà applicare la \cref{prop:LebesgueUnicaEstensione} per avere la tesi.
	
	Chiamiamo $C$ il cubo unitario come definito nell'enunciato.
	
	Fissato $S\in\S_n$ tale che per ogni $1\le i\le n$ valga $S^-_i,S^+_i\in \mathbb Z$, sia $\mathbb Z(S)$ l'insieme dei punti con tutte le coordinate intere appartenenti a $S$.
	
	È facile vedere che
	\begin{equation*}
		\bigsqcup_{v\in\mathbb Z(S)} C+v=S
	\end{equation*}
	da cui otteniamo, visto che $\mu,m_n$ sono misure invarianti per traslazione che coincidono su $C$, che $\mu,m_n$ coincidono anche su $S$.
	
	Ora invece scegliamo $S\in\S_n$ tale che per ogni $1\le i\le n$ valga $S^-_i,S^+_i\in \mathbb Q$. Possiamo assumere, vista l'invarianza per traslazione di tutto quanto, che per ogni $1\le i\le n$ valga $S^-_i=0$.
	
	Sia $m\in\mathbb Z$ tale che per ogni $1\le i\le n$ valga $mS^+_i\in \mathbb Z$, cioè ad esempio si può porre $m$ come il minimo comune multiplo di tutti i denominatori.
	
	Allora, analogamente a quanto già fatto nel caso intero, vale la seguente identità:
	\begin{equation*}
		\co0{mS^+_1}\times\cdots\times\co0{mS^+_n}=
		\bigsqcup_{(i_1,i_2,\dots i_n)\in\mathbb \{0,1,\dots,m-1\}^n}S+(i_1S_1^+,i_2S_2^+,\dots,i_nS_n^+)
	\end{equation*}
	Ma ora applicando quanto abbiamo ricavato per la misura di elementi con estremi a coordinate intere, è chiaro che la misura del membro destro dell'ultima uguaglianza coincide per $\mu,m_n$ e allora sfruttando ancora la loro invarianza per traslazione otteniamo che coincidono anche su $S$.
	
	Ora consideriamo $S\in\S_n$ generico e notiamo che vista la densità dei razionali nei reali posso scrivere $S$ come unione \emph{crescente} di elementi $(Q_i)_{i\in\N}\subseteq \S_n$ con estremi a coordinate razionali. 
	Da questa scrittura, sfruttando la \cref{LimiteMonotonoCrescenteMisura} e quanto già ottenuto sulla coincidenza tra $\mu$ e $m_n$, otteniamo
	\begin{equation*}
		m_n(S)=\lim_{n\to\infty}m_n(Q_i)=\lim_{n\to\infty}\mu(Q_i)=\mu(S)
	\end{equation*}
	cioè $\mu,m_n$ coincidono su $S$ e visto che questo è arbitrariamente scelto in $\S_n$ coincidono su tutto $\S_n$.
	
	Ora finalmente possiamo applicare la \cref{prop:LebesgueUnicaEstensione} ed ottenere $\mu\equiv m_n$, che è la tesi.
\end{proof}
\begin{remark}
	Se togliamo dalle ipotesi del \cref{thm:LebesgueUnicaInvarianteTraslazione} la normalizzazione sul cubo unitario è banale, dividendo per il corretto fattore di normalizzazione, ricavare che allora la misura $\mu$ non coincide con Lebesgue ma è un suo multiplo di un fattore costante (proprio la misura $\mu$ del cubo unitario).
	
	Spesso useremo implicitamente questa versione del \cref{thm:LebesgueUnicaInvarianteTraslazione} piuttosto che metterci nelle ipotesi esatte del teorema.
\end{remark}



Ora passiamo a dimostrare che la misura di Lebesgue è anche invariante per isometria, oltre che per rotazione, e che gli insiemi intuitivamente \emph{piccoli} sono trascurabili secondo Lebesgue.

In particolare le prime due proposizioni tecniche, concernenti le funzioni che mandano misurabili in misurabili, oltre ad avere una veloce applicazione nel dimostrare che la misura di Lebesgue è invariante per isometria, torneranno utili quando si vorranno studiare i cambi di variabili lineari nella teoria dell'integrazione.

\begin{proposition}\label{prop:ContinueSpecialiTengonoMisurabili}
	Data una funzione $f:\R^n\to\R^n$ continua tale che per ogni $N\in\M_n$ trascurabile anche l'immagine $f(N)$ è trascurabile, allora per ogni $A\in\M_n$ misurabile anche $f(A)$ è misurabile.
\end{proposition}
\begin{proof}
	Fissato $A$ misurabile, applicando il \cref{thm:LebesgueEquivalenzeMisurabilita}, otteniamo l'esistenza di $(K_i)_{i\in\N}$ compatti di $\R^n$ e di $N$ trascurabile tali che la loro unione (crescente sui compatti) è $A$. Applicando quanto detto abbiamo
	\begin{equation*}
		f(A)=f\left(N\cup\bigcup_{i\in\N}K_i\right)=f(N)\cup\bigcup_{i\in\N}f(K_i)
	\end{equation*}
	ma $f(N)$ è trascurabile a sua volta per ipotesi, e $f(K_i)$ è compatto poichè è immagine continua di un compatto.
	
	Allora abbiamo scritto anche $f(A)$ come unione di un trascurabile e di un unione crescente di compatti e questo, ancora per il \cref{thm:LebesgueEquivalenzeMisurabilita}, implica che $f(A)$ è misurabile come voluto.
\end{proof}

\begin{proposition}\label{prop:LipschitzTengonoMisurabili}
	Data una funzione $f:\R^n\to\R^n$ Lipschitziana di costante $\lambda$, la funzione $f$ manda misurabili in misurabili ed in particolare vale la seguente stima sulla misura:
	\begin{equation*}
		\forall A\in\M_n:\ \lambda^n m_n(A) \ge m_n(f(A))
	\end{equation*}
\end{proposition}
\begin{proof}
	Dimostriamo innanzitutto la stima richiesta nel caso della misura esterna piuttosto che della misura.
	
	Dato un ricoprimento $(S_i)_{i\in\N}\subseteq \S_n$ di un insieme $A$ misurabile, siano $(v_i)_{i\in\N}$ i centri degli $n$-parallelepipedi.
	Chiamiamo $S'_i=\lambda(S_i-v_i)+f(v_i)$.
	Allora è facile convincersi, sfruttando la Lipschitzianità della funzione, che l'immagine di $S_i$ tramite $f$ è un sottoinsieme di $S'_i$, cioè $f(S_i)\subseteq S'_i$.
	
	Inoltre per la \cref{nota:LebesgueElementareProprieta} abbiamo anche che $m_n(S'_i)=\lambda^n m_n(S_i)$. 
	E quindi unendo quanto detto arriviamo ad avere
	\begin{equation}\label{eq:StimaMisuraEsternaLipschitz}
		\lambda^n m_n^*(A) \ge m_n^*(f(A)) 
	\end{equation}
	che implica in particolare che se $A$ è trascurabile, anche $f(A)$ è trascurabile.
	Rientriamo perciò nelle ipotesi della \cref{prop:ContinueSpecialiTengonoMisurabili} e quindi otteniamo che $f$ manda misurabili in misurabili, cioè $f(A)$ è misurabile.
	
	Per concludere basta notare che l'\cref{eq:StimaMisuraEsternaLipschitz} ci assicura la stima richiesta dal testo, visto che nel caso in cui $A$ sia misurabile anche $f(A)$ lo è e perciò le misure esterne coincidono con le misure.
\end{proof}

\begin{proposition}\label{prop:LebesgueProprietaIsometria}
	La misura di Lebesgue su $\R^n$ è invariante per isometria.
\end{proposition}
\begin{proof}
	Poichè ogni isometria si scrive come una rotazione composta con una traslazione e sappiamo già grazie alla \cref{nota:LebesgueProprieta} che la misura di Lebesgue è invariante per traslazione, è sufficiente dimostrare che la misura di Lebesgue è invariante per rotazione.
	Sia $O:\R^n\to\R^n$ una rotazione. Essendo una rotazione di $\R^n$, questa è in particolare un'applicazione lineare continua di un Banach in se stesso ed è perciò anche Lipschitziana (si può dimostrare anche più facilmente in questo caso).
	
	Allora per quanto visto nella \cref{prop:LipschitzTengonoMisurabili} la funzione $O$ manda misurabili in misurabili e perciò ha senso definire la funzione di insiemi $m_n \circ O=\mu:\M_n\to\M_n$. 
	Se dimostriamo che $\mu\equiv m_n$ abbiamo proprio che $m_n$ è invariante per rotazione.
	
	È facile notare che $\mu$ è invariante per traslazione, visto che dati $A\in \M_n$ e $v\in\R^n$ vale
	\begin{equation*}
		\mu(A+v)=m_n(O(A+v))=m_n(O(A)+O(v))=m_n(O(A))=\mu(A)
	\end{equation*}
	dove nei vari passaggi abbiamo usato l'invarianza per traslazione della misura di Lebesgue e la linearità di $O$.
	
	Inoltre, essendo $O$ anche bigettiva, come mostrato nella \cref{prop:BigettivaInduceMisura}, la funzione $\mu$ risulta essere una misura.
	
	Allora unendo quanto ottenuto abbiamo che $\mu$ è una misura invariante per traslazione su $\M_n$ e quindi in particolare sui Boreliani di $\R^n$. Allora applicando il \cref{thm:LebesgueUnicaInvarianteTraslazione} si ottiene che esiste $\lambda\in\Rpiu$ tale che $\mu=\lambda m_n$.
	
	Ora basta considerare la palla unitaria $B$ di $\R^n$, che è misurabile poichè aperta.
	La palla $B$, che ha ovviamente misura di Lebesgue né nulla né infinita, è invariante rispetto a $O$, essendo questa una rotazione, e perciò $\mu(B)=m_n(B)$ da cui ricaviamo che $\lambda=1$ cioè $\mu\equiv m_n$ che era quanto si voleva dimostrare.
\end{proof}

\begin{proposition}\label{prop:NumerabiliLebesgueTrascurabili}
	Dato un insieme numerabile $A\subseteq\R^n$, questo è misurabile con misura nulla.
\end{proposition}
\begin{proof}
	I singoli punti, essendo banalmente trascurabili, sono Lebesgue misurabili visto che la misura di Lebesgue è completa, come mostrato nella \cref{nota:LebesgueCompletezza}.
	
	L'insieme $A$ è però numerabile, quindi unione numerabile di singoli punti, che sono trascurabili. Perciò $A$ è trascurabile a sua volta che significa che $A$ è misurabile di misura nulla.
\end{proof}

\begin{proposition}\label{prop:SottospaziTrascurabili}
	Dato $U\subset\R^n$, se lo spazio \emph{affine} generato da $U$ ha dimensione strettamente minore di $n$, allora $U$ è trascurabile secondo Lebesgue.
\end{proposition}
\begin{proof}
	Definiamo $V=\{(x_1,x_2,\dots,x_{n-1},0):\ (x_i)_{1\le i\le n-1}\subseteq\R\}$.
	
	Poichè lo spazio generato da $U$ ha dimensione $\le n-1$, esiste un'isometria che manda $U$ in un sottoinsieme di $V$ ed essendo la misura di Lebesgue invariante per isometria, come dimostrato nella \cref{prop:LebesgueProprietaIsometria}, è sufficiente dimostrare che $V$ è trascurabile per concludere.
	
	Dato $\epsilon>0$ e $k\in\N$ chiamiamo
	\begin{equation*}
		I_\epsilon(k)=\co{-k}k\times\co{-k}k\times\cdots\times \co{\frac{-\epsilon}{(2k)^{n-1}2^{k+1}}}{\frac{+\epsilon}{(2k)^{n-1}2^{k+1}}}
	\end{equation*}
	È banale, sfruttando la sola definizione della premisura $m_n$, che $m_n(I_\epsilon(k))=\frac{\epsilon}{2^k}$.
	
	Inoltre sfruttando la sola definizione di $V$ è facile verificare che per ogni $\epsilon>0$
	\begin{equation*}
		V\subset\bigcup_{k\in\N} I_\epsilon(k)\implies m_n^*(V)\le \sum m_n(I_\epsilon(k))=\epsilon
	\end{equation*}
	e questo conclude dato che implica, valendo per $\epsilon>0$, che la misura esterna di $V$ è nulla e perciò che $V$ è trascurabile.
\end{proof}

\begin{exercise}\label{ex:CantorTrascurabile}
	L'insieme di Cantor è trascurabile secondo la misura di Lebesgue $m_1$ su $\R$.
\end{exercise}

Concludiamo questa sezione, oltre che con un paio di esercizi, con due importanti risultati che mostrano esplicitamente che l'insieme dei misurabili è strettamente contenuto tra i Boreliani e le parti di $\R^n$.

In questa sezione mostriamo che i Boreliani non coincidono con i misurabili, ragionando sulle rispettive cardinalità e applicando risultati che vengono dalla teoria degli ordinali. Nel seguito di queste dispense proporremo un'altra dimostrazione di questo stesso risultato che sfrutta unicamente fatti collegati con la teoria della misura.

\begin{proposition}\label{prop:CardBoreliani}
	I Boreliani di $\R^n$ hanno la cardinalità del continuo.
\end{proposition}
\begin{remark}
	I Boreliani si possono ottenere a partire dall'insieme degli aperti $\tau$ con una successione di insiemi, ottenendo uno dal precedente
	mediante aggiunta dei complementari e delle intersezioni numerabili. Tuttavia non basta una successione finita, né l'unione degli insiemi
	così ottenuti. Vediamo allora come formalizzare questa costruzione utilizzando un'indicizzazione sugli ordinali.
	
	Questa costruzione permetterà poi di ottenere facilmente la cardinalità dei Boreliani.
\end{remark}
\begin{proof}
	Per comodità, definiamo le seguenti applicazioni su famiglie di sottoinsiemi di $X$:
	$\cdot_{\delta}:\mathcal{P}(\mathcal{P}(X)) \rightarrow \mathcal{P}(\mathcal{P}(X))$ tale che
	\[
		A_{\delta} = \left\{ \bigcap_{n\in\N}E_n : E_n\in A \right\};
	\]
	e $\cdot_{\mathsf{c}}:\mathcal{P}(\mathcal{P}(X)) \rightarrow \mathcal{P}(\mathcal{P}(X))$ tale che
	\[
		A_{\mathsf{c}} = \left\{ E^\mathsf{c} : E\in A \right\}.
	\]
	
	Consideriamo quindi la seguente funzione, definita come $f:ON \rightarrow \mathcal{P}(\mathcal{P}(\R^n))$:
	\[
	f(\alpha) = \left\{
		\begin{array}{ll}
			\tau & \quad se\ \alpha=0, \\
			f(\beta)_{\delta}\cup f(\beta)_{\mathsf{c}} & \quad se\ \alpha=\beta+1,\\
			\bigcup_{\beta < \alpha}f(\alpha) & \quad se\ \alpha\ \grave{e}\ un\ ordinale\ limite.
		\end{array}
	\right.
	\]
	Questa $f$ rappresenta la successione di famiglie di insiemi che, partendo dagli aperti, raggiungerà una \sigalg,
	che è quella generata dagli aperti.
	
	Per prima cosa osserviamo che se $\beta < \alpha$, $f(\beta)\subseteq f(\alpha)$. Altrimenti, consideriamo il più piccolo $\alpha$
	tale che esiste $\beta<\alpha$ per cui non vale questa proposizione. Chiaramente $\alpha\neq 0$ e $\alpha$ non è un ordinale limite,
	quindi è successore di un ordinale $\gamma\geq \beta$. Per minimalità di $\alpha$, $f(\beta)\subseteq f(\gamma)$, tuttavia è sempre vero
	che $A\subseteq A_{\delta}$ (basta considerare intersezione di famiglie costanti), allora si ottiene $f(\gamma)\subseteq f(\alpha)$, assurdo.
	
	Ora dimostriamo che $f(\omega_1)$\footnote{Ricordiamo che $\omega_1$ è il primo ordinale più che numerabile.} è una \sigalg.
	\begin{description}
	 \item[Chiusura rispetto complementare] Se $E\in f(\omega_1)$, esiste $\alpha<\omega_1$ tale che $E\in f(\alpha)$.
		Quindi $E^\mathsf{c}\in f(\alpha+1) \subseteq f(\omega_1)$.
	 \item[Chiusura rispetto intersezione numerabile] Se $(E_n)_{n\in\N}\subseteq f(\omega_1)$, allora $\forall n\in\N$, esiste $\alpha_n<\omega_1$
		tale che $E_n\in f(\alpha_n)$. Inoltre $\lambda=\bigcup_{n\in\N}\alpha_n$ è un ordinale numerabile, perché unione numerabile di ordinali
		numerabili, cioè $\lambda<\omega_1$. Ma allora vale che $E_n\in f(\lambda)$, per ogni $n\in\N$.
		Quindi $\bigcap_{n\in\N}E_n\in f(\lambda+1)\subseteq f(\omega_1)$.
	\end{description}
	
	Avendo mostrato che $f(\omega_1)$ è una \sigalg, abbiamo che $f(\omega_1+1)=f(\omega_1)$, quindi $f$ è definitivamente costante.
	
	Mostriamo quindi che $f(\omega_1)$ sono i Boreliani.
	Per fare ciò, basta mostrare che i Boreliani contengono $f(0)=\tau$, che se contengono $A$ contengono $A_{\delta}$ e $A_{\mathsf{c}}$ e che
	se contengono tutti gli elementi di una famiglia $\{A_i\}_{i\in I}$ allora contengono anche la loro unione $\bigcup_{i\in I}A_i$.
	È però ovvio che la \sigalg \ generata dagli aperti soddisfi tutte queste proprietà.
	
	Ora che sappiamo che $f(\omega_1)$ sono i Boreliani, calcoliamone la cardinalità.
	Intanto, $|f(0)|=|\tau|=|\R|$\footnote{È fatto noto che gli aperti di $\R^n$ si scrivono come unione numerabile di palle di centro e raggio razionali e da questa osservazione discende che hanno la cardinalità del continuo.}. Poi, $|A_{\delta}|\leq |A|^{|\N|}$ e $|A_{\mathsf{c}}|=|A|$.
	Infine $|\bigsqcup_{\beta<\alpha}E_{\beta}| \leq |E||\alpha|$, con $|E_{\beta}|\leq |E|$ per ogni $\beta < \alpha$. Tutto ciò, porta a dire che
	\begin{equation}\label{eq:CardBorel}
		|\R|=|f(0)|\leq |f(\omega_1)| \leq |\omega_1||E| \leq |E|,
	\end{equation}
	purché $E$ sia un insieme di cardinalità maggiore od uguale a quella di tutti gli $f(\alpha)$ con $\alpha<\omega_1$;
	l'ultima disuguaglianza si ha perché è sicuramente vero che $|E|\geq |f(0)| = |\R| \geq |\omega_1|$.
	Ma ora, per tutti gli ordinali al più numerabili, $|f(\alpha)|=|\R|$, altrimenti consideriamo il più piccolo ordinale per cui non vale;
	non può essere $0$, non può essere un successore:
	\[
		|f(\alpha+1)|\leq |f(\alpha)|+|f(\alpha)|^{\N} = |\R|^{|\N|} = |\R|
	\]
	e non può essere un ordinale limite:
	\[
		|f(\alpha)|\leq |\alpha| |\R| \leq |\N||\R| = |\R|.
	\]
	Quindi $E$ basta che sia $\R$, allora dall'\cref{eq:CardBorel} si ha che i Boreliani hanno la cardinalità del continuo.
\end{proof}
\begin{corollary}\label{cor:BorelianiNonMisurabili}
	I Boreliani sono \emph{strettamente} contenuti nei misurabili.
\end{corollary}
\begin{proof}
	Il contenimento debole è mostrato nella \cref{prop:CompletamentoBoreliani}.
	Per concludere che non coincidono è sufficiente mostrare che hanno cardinalità differenti.
	
	Per quanto riguarda la misura di Lebesgue su $\R$, visto che Cantor è trascurabile, come enunciato nell'\cref{ex:CantorTrascurabile}, sia esso che tutti i suoi sottoinsiemi sono misurabili. 
	Ma Cantor ha, per fatto noto, la cardinalità del continuo e quindi l'insieme delle sue parti ha cardinalità $2^{|\R|}$, e perciò i misurabili hanno cardinalità $2^{|\R|}$ (per mostrare che non è maggiore basta notare che $\M_n\subseteq\mathcal P(\R)$).
	
	In dimensione $n>1$, basta considerare un insieme i cui elementi siano nulli su tutte le coordinate eccetto che sulla prima. Questo è trascurabile per la \cref{prop:SottospaziTrascurabili}, e perciò si può procedere analogamente a quanto fatto nel caso unidimensionale e ottenere che la cardinalità dei misurabili di $\R^n$ è $2^{|\R|}$ (per mostrare l'uguaglianza serve usare $|\R^n|=|\R|$).
	
	Ma la \cref{prop:CardBoreliani} ci assicura che i Boreliani hanno cardinalità $|\R|$, indifferentemente dalla dimensione in cui lavoriamo, e questo conclude la dimostrazione, visto che $2^{|\R|}\succ |\R|$.
\end{proof}

\begin{theorem}[Insieme di Vitali]\label{thm:InsiemeVitali}
	Dato un sottoinsieme $A\in\M_n$ misurabile secondo Lebesgue con misura non nulla, esiste un sottoinsieme $B\subseteq A$ non misurabile secondo Lebesgue, cioè $B\not\in\M_n$.
\end{theorem}
\begin{proof}
	Dato $A$, notiamo che per la \sigadd[ità] della misura vale:
	\begin{equation*}
		0<m_n(A)=\sum_{(i_1,i_2,\dots,i_n)\in\mathbb Z^n} m_n\left(A\cap\left(\co{i_1}{i_1+1}\times\cdots\co{i_n}{i_n+1}\right)\right)
	\end{equation*}
	e perciò in particolare, visto che la serie ha somma positiva, ne ricaviamo che esistono $(i_1,\dots,i_n)\in \mathbb Z^n$ tale che:
	\begin{equation*}
		0<m_n\left(A\cap\left(\co{i_1}{i_1+1}\times\cdots\co{i_n}{i_n+1}\right)\right)\le
		m_n\left(\co{i_1}{i_1+1}\times\cdots\co{i_n}{i_n+1}\right)=1
	\end{equation*}
	quindi, a meno di intersecare con $\co{i_1}{i_1+1}\times\cdots\co{i_n}{i_n+1}$, possiamo assumere che $A$ sia contenuto in tale $n$-parallelepipedo poichè l'ipotesi di misura non nulla non si perde. 
	In particolare, poichè la misura di Lebesgue è invariante per traslazione come mostrato nella \cref{nota:LebesgueProprieta}, senza perdita di generalità assumeremo $A\subseteq \co{0}{1}\times\cdots\co{0}{1}$.
	
	Poniamo su $\R^n$ la relazione di equivalenza seguente:
	\begin{equation*}
		\forall x,y\in\R^n:\ x\equiv y\iff x-y\in\mathbb Q^n
	\end{equation*}
	Ora quozientiamo $\R^n$ rispetto a questa relazione d'equivalenza. In particolare siano $(R_i)_{i\in I}$ le classi d'equivalenza, dove $I$ è un insieme di indici. Sia infine $J$ il sottoinsieme degli indici definito come $J=\{i\in I:\ R_i\cap A\not =\emptyset\}$. 

	Applichiamo ora l'assioma della scelta alla famiglia di insiemi $(R_j\cap A)_{j\in J}$ e estraiamone un insieme di rappresentanti privilegiati $(r_j)_{j\in J}$. Sia ora $B$ l'insieme dei rappresentanti privilegiati.
	
	Vale, per costruzione, $B\subseteq A$. 
	Inoltre, ancora per la definizione di $B$, risulta
	\begin{equation}\label{eq:ContenimentiVitali}
		A\subseteq\bigsqcup_{q\in\mathbb Q^n\cap\left(\cc{-1}{1}\times\cdots \times\cc{-1}{1}\right)} B+q\subseteq \cc{-1}{2}\times\cdots\times\cc{-1}{2}
	\end{equation}
	dato che ogni elemento di $A$ si scrive \emph{in modo unico} (e questo spiega perchè l'unione è disgiunta) come un rappresentante privilegiato rispetto all'equivalenza più un elemento a coordinate razionali, che in questo caso possiamo scegliere tra $-1$ e $1$ poichè $A\subseteq \co 01\times\cdots \times\co 01$.
	
	Se assumiamo che $B$ sia misurabile, ora otteniamo un assurdo applicando l'\cref{eq:ContenimentiVitali} e ricordando che $m_n$ è invariante per traslazione come mostrato nella \cref{nota:LebesgueProprieta}:
	\begin{equation*}
		0<m_n(A)\le \sum_{q\in \mathbb Q^n\cap\left(\cc{-1}{1}\times\cdots \times\cc{-1}{1}\right)} m_n(B+q)=
		\sum_{q\in \mathbb Q^n\cap\left(\cc{-1}{1}\times\cdots \times\cc{-1}{1}\right)} m_n(B) \le 3^n
	\end{equation*}
	che non può essere visto che stiamo sommando una quantità numerabile di volte il valore $m_n(B)$ e perciò possiamo ottenere solo $+\infty$ oppure $0$.
	
	Allora l'insieme $B$ deve essere non misurabile come richiesto.
\end{proof}
\begin{corollary}\label{cor:MisurabiliNonParti}
	Grazie al \cref{thm:InsiemeVitali}, è ovvio mostrare che i misurabili di $\R^n$ non coincidono con le parti di $\R^n$.
\end{corollary}

\begin{exercise}
	Dato un insieme $A\subseteq\R^n$ di misura nulla secondo Lebesgue e un insieme numerabile $T\subset \R^n$, esiste $k\in \R^n$ tale che $A+k$ sia disgiunto da $T$.
\end{exercise}
\begin{proof}
	Consideriamo l'insieme $B=\bigcup_{t\in T} (A-t)$. 
	Essendo unione numerabile di trascurabili è a sua volta trascurabile.
	Perciò esiste $k\in\R^n\setminus B$ visto che $\R^n$ non è trascurabile.
	
	Assumendo per assurdo che $A-k$ intersechi $T$, avremmo che esistono $t\in T$ e $a\in A$ tali che:
	\begin{equation*}
		a-k=t\implies k=a-t \implies k\in A-t\implies k\in B	
	\end{equation*}
	ma questo non può essere per definizione di $k$ e perciò otteniamo che $A-k$ è disgiunto da $T$, che è la tesi.
\end{proof}

\begin{theorem}[Steinhaus]\label{thm:Steinhaus}
	Dato un misurabile $A\in\M_n$ non trascurabile, l'insieme $A-A=\{a_1-a_2:\ a_1,a_2\in A\}$ contiene un intorno di $0$.
\end{theorem}
\begin{proof}
	Innanzitutto notiamo che si può supporre $A$ limitato e di misura finita. Consideriamo infatti una numerazione $(B_n)$ degli $n$-cubi chiusi di $\R^n$ di lato 1, con vertici a valori negli interi (cioè ogni $B_n$ sarà della forma $B_n=[m_1,m_1+1]\times\cdots\times[m_n,m_n+1]$ con $m_1,\dots,m_n\in\Z$). Allora abbiamo che $A_n=A\cap B_n$ è misurabile di misura finita per ogni $n\in\N$. Inoltre almeno uno degli $A_n$ deve avere misura diversa da 0 per la \cref{UnioneTrascurabili}. Tale $A_n$ rispetta quindi esso stesso le ipotesi del teorema e se la tesi valesse per questo $A_n$ limitato varrebbe banalmente anche per $A$. Ci basta quindi dimostrare la tesi per $A$ limitato e di conseguenza anche di misura finita.
	
	Per il \cref{thm:LebesgueEquivalenzeMisurabilita} e poichè $A$ ha misura finita, fissato $\epsilon>0$, esistono $H\supseteq A$ aperto e $C\subseteq A$ chiuso tali che
	\[
		m_n(C)+\epsilon > m_n(A) > m_n(H)-\epsilon\punto
	\]
	Inoltre, dato che $C$ è chiuso e limitato (quindi compatto), $H^\mathsf{c}$ è chiuso e $C \cap H^\mathsf{c} = \emptyset$, esiste $\rho>0$ tale che
	$(C+B_{\rho}(0)) \cap H^\mathsf{c} = \emptyset$ (basta prendere $\rho = \frac{1}{2}\inf\{|c-h|:c\in C\virgola\ h\in H\}$, che è maggiore di 0 per la compattezza di $C$).
	Allora si ha che $\forall x\in B_{\rho}(0)$, $C \cap (C+x) \neq \emptyset$; altrimenti per monotonia della misura si avrebbe
	\[
		2m_n(C) = m_n(C)+m_n(C+x) \leq m_n(C+B_{\rho}(0)) \leq m_n(H)\virgola
	\]
	ma questo contraddirebbe la disuguaglianza precedente, pur di considerare $\epsilon< \frac{1}{2}m_n(C)$. E ciò è sempre possibile, dato che
	$m_n(A) > 0$ per ipotesi.
	
	La contraddizione implica che $\forall x\in B_{\rho}(0)$ esistono $c_1,c_2$ tali che $c_1+x=c_2$, cioè che $B_{\rho}(0) \subseteq A-A$.
\end{proof}



\section{Funzioni misurabili}
In questa sezione daremo la definizione di funzione misurabile e dimostreremo alcuni fatti basilari su di esse.

La teoria delle funzioni misurabili, oltre ad essere strettamente necessaria per la successiva teoria dell'integrazione, ci permetterà di dimostrare che i misurabili secondo Lebesgue, introdotti nella sezione precedente, non coincidono con i Boreliani usando strumenti propri della teoria della misura.
Questo fatto lo dimostriamo però solo nel caso unidimensionale sia perché il risultato è già stato dimostrato, sia perché la dimostrazione si adatta facilmente in dimensione maggiore e sia perché troviamo importante rendere chiara l'idea piuttosto che celarla in una notazione troppo pesante.

\begin{definition}[Retta reale estesa]
	Indichiamo con $\Rbar$ l'insieme $\R\cup\{-\infty,+\infty\}$ munito della topologia che ha come base tutti gli aperti tipici di $\R$ e gli insiemi del tipo $\{-\infty\}\cup\oo{-\infty}a$ e $\{+\infty\}\cup\oo{a}{+\infty}$ dove $a$ è un numero reale. 
	Quest'ultima categoria di insiemi aperti verrà indicata tramite la notazione $\co{-\infty}a$ e $\oc a{+\infty}$.
\end{definition}


\begin{definition}[Funzione misurabile]
	Dato uno spazio di misura $(X,\A,\mu)$, una funzione $f:X\rightarrow \Rbar$ si dice misurabile se
	$\forall A \subseteq \Rbar$ aperto si ha $f^{-1}(A)\in \A$.
\end{definition}

\begin{proposition}\label{prop:BasicMis}
	Dato uno spazio di misura $(X,\A,\mu)$, sia $f:X\rightarrow \Rbar$ una funzione, sono equivalenti i seguenti fatti
	\footnote{Qui introduciamo la notazione per i \textit{sovralivelli} di una funzione che useremo in tutti gli appunti:
		data una funzione $f$ di codominio reale e un certo reale $k$ indichiamo con $\{f>k\}$ l'insieme
		$\{x:f(x)>k\}=f^{-1}((k,+\infty])$; stessa notazione verrà usata anche per il sottolivello.}:
	\begin{enumerate}[label=(\arabic*),ref=(\arabic*)]
		\item $f$ è misurabile; \label{BM:mis}
		\item $\{f<a\}\in \A \quad \forall a\in \Rbar$; \label{BM:sot}
		\item $\{f\le a\}\in \A \quad \forall a\in \Rbar$; \label{BM:soteq}
		\item $\{f>a\}\in \A \quad \forall a\in \Rbar$; \label{BM:sov}
		\item $\{f\ge a\}\in \A \quad \forall a\in \Rbar$;  \label{BM:soveq}
		\item $\{a<f<b\}\in \A \quad \forall a,b\in \Rbar$. \label{BM:int}
	\end{enumerate}
\end{proposition}
\begin{proof}
	Sfruttando le proprietà di $\A$ come \sigalg, mostriamo a catena tutte le implicazioni:
	\begin{description}
	\item[\ImplicationProof{BM:mis}{BM:sot}] per definizione di misurabile, $\{f<a\}=f^{-1}\left(\co{-\infty}{a}\right)\in \A$,
		perché $\co{-\infty}{a}$ è aperto;
	\item[\ImplicationProof{BM:sot}{BM:soteq}] poiché $\cc{-\infty}{a}=\bigcap_{n\in \N}\co{-\infty}{a+\frac{1}{n}}$, allora otteniamo
		\begin{equation*}
			\{f\le a\}=f^{-1}\left(\cc{-\infty}{a}\right)=\bigcap_{n\in \N}f^{-1}\left(\co{-\infty}{a+\frac{1}{n}}\right)\in \A	
		\end{equation*}

	\item[\ImplicationProof{BM:soteq}{BM:sov}] passando al complementare, $\{f>a\}^\mathsf{c}=\{f\le a\}\in \A$;
	\item[\ImplicationProof{BM:sov}{BM:soveq}] analogamente a \ImplicationProof{BM:sot}{BM:soteq}; 
	\item[\ImplicationProof{BM:soveq}{BM:sot}] analogamente a \ImplicationProof{BM:soteq}{BM:sov};
	\item[$\text{\ref{BM:sot}}\ +\ \text{\ref{BM:sov}}\implies\text{\ref{BM:int}}$] perché
		$\{a<f<b\}=\{a<f\}\cap\{f<b\}\in \A$;
	\item[\ImplicationProof{BM:int}{BM:mis}] dato che un aperto $A\subseteq\Rbar$ si può scrivere come
		$A=\bigcup_{n\in \N}A_n$ dove ciascun $A_n$ è un intervallo aperto di $\Rbar$ (o una semiretta aperta),
		allora $f^{-1}(A)=\bigcup_{n\in \N}f^{-1}(A_n)\in \A$, dove abbiamo sfruttato che per il punto \ref{BM:int} $f^{-1}(A_n)\in \A\ \ \forall n$.
	\end{description}
\end{proof}

\begin{proposition}\label{prop:CounterImgMis}
	Dato uno spazio di misura $(X,\A,\mu)$, e data $f:X\rightarrow \Rbar$ una funzione misurabile, la famiglia di insiemi
	\[
		\mathcal{E} = \{ E\subseteq \Rbar : f^{-1}(E)\in \A \}
	\]
	è una \sigalg{} ed inoltre contiene i Boreliani.
\end{proposition}
\begin{proof}
	Verifichiamo che $\mathcal E$ è stabile per unioni numerabili e passaggio al complementare.
	
	Fissati $\{E_n\}_{n\in \N}\subseteq \mathcal{E}$ sia $E = \cup_{n\in \N}E_n$, vale:
	\begin{equation*}
		f^{-1}(E)=f^{-1}\left(\cup_{n\in \N}E_n\right) = \cup_{n\in \N}f^{-1}(E_n)\in \A \implies E \in \mathcal{E}
	\end{equation*}
	dove l'appartenenza ad $\A$ si ha per le proprietà di \sigalg{}, e questo dimostra la stabilità per unione numerabile.
	
	Per quanto riguarda il passaggio al complementare, fissato $E\in \mathcal{E}$, risulta:
	\begin{equation*}
		f^{-1}(E^\mathsf{c})= f^{-1}(E)^\mathsf{c} \in \A \implies E^\mathsf{c} \in \mathcal{E}.
	\end{equation*}
	
	Inoltre $\mathcal E$ contiene gli aperti per definizione di funzione misurabile, da cui, per quanto appena dimostrato, contiene la \sigalg{} generata da questi, cioè i Boreliani.
\end{proof}

\begin{definition}\label{def:FpiuFmeno}
	Sia $f:X\to\Rbar$ una funzione misurabile su uno spazio di misura $(X,\A,\mu)$. Definiamo $f^+ = \max\{f,0\}$ e $f^- = \max\{-f,0\}$.
\end{definition}
\begin{remark}\label{nota:ProprietaFpiuFmeno}
	Data una funzione $f$ misurabile, vale $f^+,f^-$ sono misurabile. Inoltre $f^+,f^-\ge 0$ e $f=f^+-f^-$. 
\end{remark}
\begin{proof}
	Abbiamo che per ogni $a\in\Rbar$ vale
	\begin{equation*}
		\{f^+>a\}=\left\{\begin{array}{ll}
			\{f>a\}\in\A &\text{se $a>0$}\\
			X\in\A &\text{se $a\le 0$}
	\end{array}\right.
	\end{equation*}
	Quindi, per la \ref{BM:sov} della \cref{prop:BasicMis}, $f^+$ è misurabile. Analogamente si dimostra che $f^-$ è misurabile.
\end{proof}


\begin{proposition}\label{prop:AlgMis}
	Dato uno spazio di misura $(X,\A,\mu)$, sia $\mathcal{M}$ l'insieme delle funzioni misurabili da $X$ in $\Rbar$.
	Allora $\mathcal{M}$ è un'algebra nel senso che, dove sono definite\footnote{Nel definire le operazioni algebriche su $\mathcal{M}$ adottiamo le seguenti convenzioni: la somma è definita se non accade che entrambe $f$ e $-g$ siano $\pm\infty$, per la moltiplicazione $0\cdot \infty = 0$.} ,
	valgono le seguenti:
	\begin{enumerate}[label=(\arabic*),ref=(\arabic*)]
		\item $f,g\in \mathcal{M} \Rightarrow f+g\in \mathcal{M}$; \label{AlM:sum}
		\item $f\in \mathcal{M}, \lambda \in \R \Rightarrow \lambda f\in \mathcal{M}$; \label{AlM:sca}
		\item $f,g\in \mathcal{M} \Rightarrow fg\in \mathcal{M}$. \label{AlM:pro}
	\end{enumerate}
\end{proposition}

\begin{proof}
	Mostriamo per ogni punto che vale la proposizione \ref{BM:sov} nella \cref{prop:BasicMis} (che come lì mostrato, equivale alla misurabilità),
	distinguendo vari casi di $a\in \Rbar$.
	\begin{description}
	\item[\ref{AlM:sum}]
	\[
		\{f+g>a\}=\left\{\begin{array}{ll}
			\{f\ge -\infty\}\cap\{g\ge -\infty\}\in \A &\qquad \text{se $a=-\infty$}\puntovirgola\\
			\bigcup_{q\in \Q}\left(\{f>q\}\cap\{g>a-q\}\right)\in \A &\qquad \text{se $a\in \R$}\puntovirgola\\
			\{f=+\infty\}\cup\{g=+\infty\}\in \A &\qquad \text{se $a=+\infty$}\punto
		\end{array}\right.
	\]
	\item[\ref{AlM:sca}]
	\[
		\{\lambda f>a\}=\left\{\begin{array}{ll}
			\left\{f<\frac{a}{\lambda}\right\}\in \A &\qquad \text{se $\lambda<0$}\puntovirgola \\
			X \in \A &\qquad \text{se $\lambda=0$ e $a< 0$}\puntovirgola\\
			\emptyset \in \A &\qquad \text{se $\lambda=0$ e $a\ge 0$}\puntovirgola\\
			\left\{f>\frac{a}{\lambda}\right\}\in \A &\qquad \text{se $\lambda>0$}\punto
		\end{array}\right.
	\]
	\item[\ref{AlM:pro}] Scomponiamo $f=f^+ - f^-$, $g=g^+- g^-$, quindi la funzione prodotto $fg$ si scrive come una qualche combinazione di prodotti di funzioni misurabili non negative (abbiamo già dimostrato nella \cref{nota:ProprietaFpiuFmeno} le proprietà di $f^+,f^-,g^+,g^-$ che ci servono). Grazie ai punti \ref{AlM:sum} e \ref{AlM:sca} e a questa osservazione ci basta mostrare il caso in cui $f,g\ge0$:
	\[
		\{fg>a\}=\left\{\begin{array}{ll}
			X\in \A &\enspace \text{se $a<0$}\puntovirgola\\
			\{f>0\}\cup\{g>0\}\in \A &\enspace \text{se $a=0$}\puntovirgola\\
			\bigcup_{q\in \Q^+}\left(\{q<f<+\infty\}\cap\left\{\frac{a}{q}<g<+\infty \right\} \right)\in \A &\enspace \text{se $0<a<+\infty$}\puntovirgola\\
			(\{f=+\infty\}\cap\{g>0\})\cup (\{f>0\}\cap\{g=+\infty\})\in \A &\enspace \text{se $a=+\infty$}\punto
		\end{array}\right.
	\]
	\end{description}
\end{proof}

\begin{remark}\label{nota:CarMis}
	È facile vedere che le funzioni caratteristiche degli insiemi misurabili sono misurabili.
\end{remark}
\begin{proof}
	Basta osservare che se $A\in \A$ è misurabile, $\{ \chi_A > a\}$ può essere solo $\emptyset$, $A$, $X$ (tutti e tre misurabili) a seconda che
	$a\ge 1$, $0\le a< 1$ oppure $a < 0$ rispettivamente.
\end{proof}

\begin{remark}\label{nota:ContinueMisurabili}
	Sia $X$ un insieme dotato sia di una topologia che di una misura su di esso, tali che in particolare la \sigalg{} dei misurabili contenga tutti gli aperti.
	Data una funzione $f:X\to\R$, se $f$ è continua è anche misurabile.
\end{remark}
\begin{proof}
	Basta notare che la controimmagine di un aperto è un aperto per continuità, ma gli aperti sono misurabili per ipotesi e di conseguenza la funzione è misurabile.
\end{proof}

\begin{remark}\label{nota:MonotoneMisurabili}
	Fissato $A\subseteq \R$ misurabile, ogni funzione $f:A\to\R$ monotona è misurabile, munendo $\R$ della misura di Lebesgue definita nella precedente sezione.
\end{remark}
\begin{proof}
	È sufficiente notare che la controimmagine di un intervallo\footnote{Definiamo, unicamente in questa dimostrazione, un intervallo come un generico sottoinsieme connesso di $\R$.} è a sua volta un intervallo intersecato $A$ poiché la funzione è monotona, perciò applicando la \cref{prop:BasicMis} ricaviamo che la funzione è misurabile visto che gli intervalli sono misurabili secondo Lebesgue (e lo è la loro interesezione con $A$, per la \cref{nota:RiduzioneMisura}).
\end{proof}

\begin{proposition}\label{prop:BorelianiNonMisurabili2}
	I Boreliani di $\R$ non coincidono con l'insieme $\M_1$ dei misurabili.
\end{proposition}
\begin{proof}
	Definiamo la funzione $f:\co{0}{1}\to \co{0}{1}$ in modo che $f(x)$ sia il numero che corrisponde alla lettura in base $3$ della scrittura in base $2$ di $x$.
	Poiché alcuni numeri hanno due scritture in base $2$, sceglieremo sempre quella che non ha una coda infinita di $1$.
	
	Qui di seguito un diagramma che mostra la definizione di $f$:
	\begin{equation*}
		x=\>\stackrel{\text{Scrittura in base $2$ di $x$}}{\overline{0.x_1x_2x_3\cdots}_2} \>  \longmapsto
		\> \stackrel{\text{Lettura in base $3$ di $x$ in base $2$}}{\overline{0.x_1x_2x_3\cdots}_3}\>=f(x)
	\end{equation*}

	La funzione $f$ appena definita è strettamente crescente, poiché lo è la funzione che associa ad un numero $x$ la sua lettura in qualche base (dove le sequenze di cifre sono ordinate lessicograficamente). Allora per la \cref{nota:MonotoneMisurabili} $f$ è misurabile.
	
	Inoltre l'immagine di $f$ è un insieme trascurabile, infatti questa coincide con l'insieme dei numeri tra $0$ e $1$ che si scrivono unicamente usando cifre $0,1$ in base $3$ ed è facile dimostrare che questo è trascurabile (esercizio per il lettore molto simile all'\cref{ex:CantorTrascurabile}).
	
	Per il \cref{thm:InsiemeVitali} esiste $A\subseteq \co{0}{1}$ che non è misurabile.
	Sia $B=f(A)$.
	
	Poiché $f$ è strettamente crescente è in particolare iniettiva e perciò $A=f^{-1}(B)$.
	Allora la \cref{prop:CounterImgMis} ci assicura che $B$ non appartiene ai Boreliani, altrimenti la sua controimmagine sarebbe misurabile. 
	Infine $B$ è sottoinsieme di $\co01$, che è trascurabile, perciò per la completezza della misura di Lebesgue $B$ è misurabile.
	
	Allora, unendo quanto detto, abbiamo che $B$ è un misurabile non Boreliano come voluto.
\end{proof}



\begin{definition}
	Una funzione $\simp:X \rightarrow \Rbar$ con dominio lo spazio di misura $(X,\A,\mu)$ si dice semplice se è combinazione lineare di
	funzioni caratteristiche di insiemi misurabili.
\end{definition}
\begin{remark}
	È immediato che le funzioni semplici sono misurabili.
\end{remark}
\begin{proof}
	Discende dalla \cref{nota:CarMis} e dalla \cref{prop:AlgMis}.
\end{proof}


\begin{proposition}\label{prop:SupDiMisurabili}
	Sia $\{f_n\}_{n\in \N}$ una famiglia di funzioni misurabili definite dallo spazio di misura $(X,\A,\mu)$ a $\Rbar$.
	Allora $F:X\rightarrow \Rbar$ definita da $F(x)=\sup\{f_n(x):n\in \N\}$ è misurabile.
\end{proposition}
\begin{proof}
	Consideriamo il sovralivello della funzione $F$: $\{F>a\}=\bigcup_{n\in \N}\{f_n>a\}$, allora è evidente che $\{F>a\}\in \A$ per le proprietà 
	di chiusura della \sigalg.
\end{proof}

\begin{remark}\label{nota:LimMis}
	Quest'ultima proposizione ha alcune notevoli conseguenze immediate:
	\begin{enumerate}
		\item $\inf$ di una famiglia numerabile di misurabili è misurabile;\label{LM:inf}
		\item $\limsup$ e $\liminf$ di una famiglia numerabile di misurabili sono misurabili;\label{LM:lim_infsup}
		\item limite puntuale di funzioni misurabili è misurabile.\label{LM:lim}
	\end{enumerate}
\end{remark}
\begin{proof}
	\begin{description}
		\item[\ref{LM:inf}] Per l'$\inf$ basta notare che $\inf\{f_n\}=-\sup\{-f_n\}$, quindi è misurabile per la \cref{prop:SupDiMisurabili};
		\item[\ref{LM:lim_infsup}] per definizione, $\limsup\{f_n\}$ e $\liminf\{f_n\}$ sono rispettivamente
			$\lim_n\{\sup\{f_n\}\}=\inf\{\sup\{f_n\}\}$ e
			$\lim_n\{\inf\{f_n\}\}=\sup\{\inf\{f_n\}\}$, quindi sono funzioni misurabili per il punto precedente;
		\item[\ref{LM:lim}] infine se esiste il limite $\lim_n\{f_n\}$ allora
			$\liminf\{f_n\}=\limsup\{f_n\}=\lim_n\{f_n\}$, pertanto è misurabile per il punto precedente.
	\end{description}
\end{proof}

\begin{proposition}\label{prop:LimSemMis}
	Sia $f:X \rightarrow \Rbar$ una funzione con dominio lo spazio di misura $(X,\A,\mu)$.
	Allora $f$ è misurabile se e solo se esiste una successione di funzioni semplici $\simp_n$ che converge puntualmente a $f$.
\end{proposition}
\begin{proof}
	Il se è mostrato nella \cref{nota:LimMis}.
	
	Per il solo se facciamo vedere che la seguente successione converge puntualmente a $f$:
	\[
		\simp_n(x) =
		\left\{ \begin{array}{ll}
			n &\qquad se\ f(x)>n;\\
			\frac{k}{2^n} &\qquad se\ \frac{k}{2^n}<f(x)\le \frac{k+1}{2^n} \qquad k=-n2^n,-n2^n+1,\dots,n2^n;\\
			-n &\qquad se\ f(x)\le-n;
		\end{array} \right.\ .
	\]
	Prima di tutto abbiamo 
	\[\simp_n=
		n\chi_{\{f>n\}}+
		\sum_{k=-n2^n}^{n2^n}\frac{k}{2^n}\chi_{\left\{ \frac{k}{2^n}<f\le \frac{k+1}{2^n} \right\}}
		-n\chi_{\{f<-n\}},
	\]
	che mostra che le $\simp_n$ sono funzioni semplici.
	
	Per mostrare la convergenza puntuale distinguiamo $f(x)$ a seconda che sia un numero finito o meno:
	nel primo caso abbiamo che $|f(x)-\simp_n(x)|\le \frac{1}{2^n}$ definitivamente, cioè $\forall n\ge |f(x)|$,
	nel secondo caso $\simp_n(x)=\pm n\rightarrow \pm\infty = f(x)$;
	quindi $\simp_n(x)\rightarrow f(x)$, $\forall x\in X$.
\end{proof}

\begin{definition}
	Una funzione $f$ definita su $(X,\A,\mu)$ e a valori in $\Rbar$ si dice positiva se assume solo valori maggiori o uguali a 0.
\end{definition}


\begin{corollary}\label{cor:LimSemCrescMis}
	Sia $f:X\rightarrow \Rbar$ misurabile e positiva su $(X,\A,\mu)$ spazio di misura, allora esiste una successione crescente di funzioni semplici e positive $(\simp_n)$ che converge puntualmente a $f$.
\end{corollary}
\begin{proof}
	Costruendo le $\simp_n$ come nella \cref{prop:LimSemMis}, se $f$ è positiva otteniamo facilmente che le $\simp_n$ sono anche crescenti, da cui la tesi.
\end{proof}

\begin{theorem}\label{thm:ChiusuraMonotonaFunzioni}
	Se $\mathcal F$ è una famiglia di funzioni da $X$ a $\Rbar$, dove $(X,\A,\mu)$ è uno spazio di misura, tale che
	\begin{itemize}
	 \item $\mathcal F$ è uno spazio vettoriale;
	 \item $\mathcal F$ contiene le funzioni indicatrivi di ogni insieme $A\in\A$;
	 \item se $(f_n)\subseteq \mathcal F$ è una successione di funzioni misurabili positive che converge crescentemente a $f$, allora $f\in \mathcal F$;
	\end{itemize}
	allora $\mathcal F$ contiene tutte le funzioni misurabili.
\end{theorem}
\begin{proof}
	Per prima cosa notiamo che $\mathcal F$ contiene le funzioni semplici: essendo $\mathcal F$ uno spazio vettoriale, contiene le combinazioni
	lineari delle funzioni caratteristiche, cioè le funzioni semplici.
	
	Notiamo allora che per il \cref{cor:LimSemCrescMis}, $\mathcal F$ contiene le funzioni misurabili positive. Allora, data $f$ misurabile,
	$f = f_+-f_-$ quindi è contenuta in $\mathcal F$, ancora per le proprietà di spazio vettoriale, poiché entrambe $f_+,f_-$ sono
	positive e misurabili per la \cref{nota:ProprietaFpiuFmeno}.
\end{proof}

\section{Integrazione secondo Lebesgue}
In questa sezione definiremo la nozione di integrale secondo Lebesgue e dimostreremo alcuni risultati introduttivi. In particolare definiremo inizialmente l'integrale di funzioni misurabili positive per poi estenderlo facilmente alle funzioni misurabili che hanno integrale del loro valore assoluto finito. Chiameremo queste ultime funzioni integrabili.

\begin{remark}
	Indicheremo l'integrale di $f$ su una spazio di misura $(X,\A,\mu)$ con 
	\begin{equation*}
		\int_X f(x) \de \mu(x)=\int_X f \de \mu=\int f \de \mu=\int f
	\end{equation*}
	dove si applicheranno le varie omissioni quando saranno ovvi la variabile di integrazione $x$, lo spazio di misura $X$ o la misura $\mu$.
\end{remark}

\begin{definition}\label{def:IntegraleSemplici}
	Sia $\simp:X\to \Rpiu$ una funzione misurabile, semplice e positiva, definita sullo spazio di misura $(X,\A,\mu)$. Definiamo l'integrale di Lebesgue della funzione il valore
	\begin{equation*}
		\int_X \simp d \mu = \sum_{k=1}^n c_k\mu(E_k)\virgola
	\end{equation*}
	dove $c_k\in \Rbar$ e $E_k \in \A$ sono tali che $\simp=\sum_{k=1}^nc_k\chi_{E_k}$.
\end{definition}

\begin{remark}\label{nota:BuonaDefIntSemp}
	Quella appena enunciata è una buona definizione, cioè non dipende dalla scelta dei $c_k$ e degli $E_k$.
\end{remark}
\begin{proof}
	Consideriamo innanzitutto dei $c_k\in \Rbar$ e degli $E_k\in \A$ tali che $\simp=\sum_{k=1}^nc_k\chi_{E_k}$ e definiamo per ogni $\alpha\subseteq\{1,2,\dots,n\}$:
	\begin{equation*}
		\begin{cases}
			E_\alpha=\bigcap_{k\in\alpha}E_k\setminus \bigcup_{k\not\in \alpha} E_k\\
			c_\alpha=\sum_{k\in\alpha} c_k
		\end{cases}
	\end{equation*}
	D'ora in poi nelle sommatorie sottointenderemo l'insieme su cui cicla $\alpha$, cioè $\mathcal{P}(\{1,2,\dots,n\})$.
	
	Innanzitutto gli $E_\alpha$ sono ancora insiemi misurabili (in quanto $\A$ è una \sigalg{}) e sono insiemi disgiunti, inoltre vale facilmente che
	\begin{equation*}
		\simp=\sum_{k=1}^nc_k\chi_{E_k}=\sum_{k=1}^nc_k\sum_{\alpha\ni k}\chi_{E_\alpha}=\sum_\alpha \chi_{E_\alpha}\sum_{k\in\alpha}c_k=\sum_\alpha c_\alpha\chi_{E_\alpha}\virgola
	\end{equation*}
	dove abbiamo utilizzato che $E_k=\cup_{\alpha\ni k} E_\alpha$.

	Vogliamo dimostrare che $\sum_{k=1}^nc_k\mu(E_k)=\sum_{\alpha}c_\alpha\mu(E_\alpha)$, ma per l'additività di $\mu$ su insiemi disgiunti otteniamo
	\begin{equation}\label{eq:RaffinamentoIntegrale}
		\sum_\alpha c_\alpha\mu(E_\alpha)=\sum_\alpha \sum_{k\in\alpha}c_k\mu(E_\alpha)=\sum_{k=1}^n c_k\sum_{\alpha \ni k} \mu(E_\alpha)=\sum_{k=1}^n c_k \mu(E_k)\virgola
	\end{equation}
	che è proprio quello che volevamo.
	
	Consideriamo ora $c_k, b_k\in\Rbar$ ed $E_k,F_k\in\A$ tali che $\simp=\sum_{k=1}^nc_k\chi_{E_k}=\sum_{k=1}^mb_k\chi_{F_k}$. A meno di aggiungere dei termini ad entrambe le sommatorie possiamo supporre $n=m$ e $E_k=F_k$ per ogni $1\le k\le n$, da cui $\simp=\sum_{k=1}^nc_k\chi_{E_k}=\sum_{k=1}^nb_k\chi_{E_k}$. Applichiamo la costruzione di prima e otteniamo:
	\begin{equation*}
		\begin{cases}
			E_\alpha=\bigcap_{k\in\alpha}E_k\setminus \bigcup_{k\not\in \alpha} E_k\\
			c_\alpha=\sum_{k\in\alpha} c_k\\
			b_\alpha=\sum_{k\in\alpha} b_k
		\end{cases}
	\end{equation*}
	da cui ricaviamo per quanto detto precedentemente che
	\begin{equation}
		\sum_\alpha c_\alpha\chi_{E_\alpha}=\sum_{k=1}^nc_k\chi_{E_k}=\simp=\sum_{k=1}^nb_k\chi_{E_k}=\sum_\alpha b_\alpha\chi_{E_\alpha}\virgola
	\end{equation}
	che implica facilmente che $c_\alpha=b_\alpha$ per ogni $\alpha$, poiché gli $E_\alpha$ sono disgiunti.
	
	Infine da quest'ultima uguaglianza e dall'\cref{eq:RaffinamentoIntegrale} otteniamo proprio quello che volevamo, in quanto
	\begin{equation*}
		\sum_{k=1}^n c_k \mu(E_k)=\sum_\alpha c_\alpha\mu(E_\alpha)=\sum_\alpha b_\alpha\mu(E_\alpha)=\sum_{k=1}^n b_k \mu(E_k)\punto
	\end{equation*}
\end{proof}

\begin{proposition}\label{prop:IntegraleSemplici}
	Date $\simp_1,\simp_2:X\to\Rpiu$ funzioni misurabili, semplici e positive con $(X,\A,\mu)$ spazio di misura e dati $a,b$ reali non negativi, valgono:
	\begin{enumerate}
		\item $\int (a\simp_1+b\simp_2)\de\mu=a\int \simp_1\de\mu+b\int \simp_2\de\mu$ \label{PIS:add}
		\item $\simp_1\le \simp_2\Longrightarrow \int \simp_1\de\mu\le \int \simp_2\de\mu$ \label{PIS:mono}
	\end{enumerate}
\end{proposition}
\begin{proof}
	Seguono entrambe facilmente dalla definizione di integrale di funzioni semplici e dalla \cref{nota:BuonaDefIntSemp}.
\end{proof}

\begin{remark}
	Osserviamo che data una funzione misurabile, semplice e positiva $\simp:X\to\Rpiu$, con $(X,\A,\mu)$ spazio di misura, e dato $E\in\A$ misurabile, possiamo definire l'integrale di $\simp$ su $E$. 
	
	Infatti per la \cref{nota:RiduzioneMisura} anche $(E,\A_E,\mu|_{\A_E})$ (come definito nella \cref{nota:RiduzioneMisura}) è uno spazio di misura e vale facilmente che $\simp$ è misurabile, semplice e positiva anche ridotta su $(E,\A_E,\mu|_{\A_E})$.
\end{remark}

\begin{proposition}\label{prop:IntegraleSempliciSuMisurabili}
	Data una funzione $\simp:X\to\Rpiu$ misurabile, semplice e positiva, con $(X,\A,\mu)$ spazio di misura, e dato $E\in\A$, vale che
	\begin{equation*}
		\int_E \simp\de\mu=\int_X \simp \cdot \chi_E \de\mu\punto
	\end{equation*}
\end{proposition}
\begin{proof}
	Siano $c_k\in\Rbar$ e $E_k\in\A$, per $1\le k\le n$, tali che $\simp=\sum_{i=1}^nc_k\chi_{E_k}$. Chiamiamo $E'_k=E_k\cap E$, che è ancora un insieme misurabile per ogni $1\le k\le n$. Allora vale che
	\begin{equation*}
		\int_X \simp \cdot \chi_E \de\mu=\int_X \sum_{i=1}^nc_k\chi_{E_k}\cdot \chi_E \de\mu=\int_X \sum_{i=1}^nc_k\chi_{E'_k} \de\mu = \sum_{i=1}^n c_k \mu(E'_k)=\int_E \simp\de\mu \punto
	\end{equation*}
\end{proof}

\begin{corollary}\label{cor:IntegraleSempliciSpezzato}
	Siano $X_1,X_2,\dots,X_k$ misurabili e disgiunti tali che $X=\cup_{m=1}^kX_m$ e sia $\simp$ una funzione semplice e positiva, allora
	\begin{equation*}
		\int_X \simp \de \mu=\sum_{m=1}^k \int_{X_m}\simp\de \mu
	\end{equation*}
\end{corollary}
\begin{proof}
	Sfruttando le \cref{prop:IntegraleSempliciSuMisurabili,prop:IntegraleSemplici}, otteniamo che
	\begin{equation*}
		\int_X \simp \de \mu=\int_X\sum_{m=1}^k\simp\cdot \chi_{X_m}\de\mu=\sum_{m=1}^k\int_X \simp\cdot \chi_{X_m}\de\mu=\sum_{m=1}^k \int_{X_m}\simp\de \mu\punto
	\end{equation*}
\end{proof}



\begin{definition}\label{def:IntegralePositive}
	Sia $f:X\to\Rpiu$ una funzione misurabile e positiva, con $(X,\A,\mu)$ spazio di misura, allora definiamo l'integrale di $f$ come
	\begin{equation*}
		\int_Xf\de\mu=\sup\left\{\int_X\simp\de \mu:\ 0\le \simp\le f\ \wedge\ \simp \text{ semplice}\right\}\punto
	\end{equation*}
\end{definition}

\begin{proposition} \label{prop:DefinizioneEquivalenteIntegralePositive}
	Data una funzione $f:X\to\Rpiu$ misurabile e positiva, con $(X,\A,\mu)$ spazio di misura, e data una successione crescente di funzioni semplici e positive $(\simp_n)$ tali che convergono puntualmente a $f$, vale 
	\begin{equation*}
		\int_X f\de\mu=\sup_{n\in\N}\left\{\int_X\simp_n\de\mu\right\}\punto
	\end{equation*}
\end{proposition}
\begin{proof}
	Dimostriamo innanzitutto che dato un insieme $Y\in \A$ misurabile tale che $f\ge a$ in $Y$, allora vale $\sup_{n\in\N}\left\{\int_Y\simp_n\de\mu\right\}\ge a\mu(Y)$.
	
	Dato $\epsilon>0$ sia $A_n=\{\simp_n\ge a-\epsilon\}\cap Y$. Allora abbiamo che gli $A_n$ sono misurabili, vale $A_n\subseteq A_{n+1}$, perché le $\simp_n$ sono crescenti, e $\cup_nA_n=Y$. Inoltre $\simp_n\ge (a-\epsilon)\chi_{A_n}$ in $Y$ per la definizione stessa di $A_n$. Perciò per la \cref{prop:IntegraleSemplici} abbiamo che
	\begin{equation*}
		\int_Y\simp_n\de\mu\ge(a-\epsilon)\mu(A_n)\virgola
	\end{equation*}
	da cui passando al limite su $n$ ed utilizzando che $\lim_{n\to\infty}\int_Y\simp_n\de\mu=\sup_{n\in\N}\int_Y\simp_n\de\mu$, otteniamo
	\begin{equation*}
		\sup_{n\in\N}\left\{\int_Y\simp_n\de\mu\right\}\ge(a-\epsilon)\lim_{n\to\infty}\mu(A_n)=(a-\epsilon)\mu(Y)\virgola
	\end{equation*}
	dove in particolare nell'ultima uguaglianza abbiamo utilizzato la \cref{prop:LimiteMonotonoCrescenteMisura}. Poiché in quest'ultima relazione $\epsilon$ è arbitrario otteniamo quindi
	\begin{equation*}
		\sup_{n\in\N}\left\{\int_Y\simp_n\de\mu\right\}\ge a\mu(Y)\punto
	\end{equation*}

	Consideriamo ora una qualsiasi funzione semplice $\simp$ tale che $0\le \simp\le f$. Per facile conseguenza della definizione di funzione semplice, esistono $k$ sottoinsiemi misurabili disgiunti di $X$, che chiamiamo $X_1,\dots,X_k$, tali che $\cup_{m=1}^kX_m=X$ e $\simp|_{X_m}$ è costante per ogni $m=1,\dots,k$.
	
	Per quanto dimostrato precedentemente abbiamo che $\sup_{n\in\N}\left\{\int_{X_m}\simp_n\de\mu\right\}\ge a_m\mu(X_m)$ per ogni $m=1,\dots,k$, dove $a_m=\simp|_{X_m}$. Unendo questo risultato al \cref{cor:IntegraleSempliciSpezzato} e ricordando che i $\sup$ sono in realtà dei limiti, otteniamo:
	\begin{equation*}
		\sup_{n\in\N}\left\{\int_X\simp_n\de\mu\right\}=\sup_{n\in\N}\left\{\sum_{m=1}^k \int_{X_m}\simp_n\de \mu\right\}=\sum_{m=1}^k\sup_{n\in\N}\left\{ \int_{X_m}\simp_n\de \mu\right\}\ge \sum_{m=1}^ka_m\mu(X_m)=\int_X\simp\de\mu\punto
	\end{equation*}
	Passando ora all'estremo superiore per $0\le \simp \le f$ otteniamo proprio:
	\begin{equation*}
		\sup_{n\in\N}\left\{\int_X\simp_n\de\mu\right\}\ge \int_X f \de\mu\Longrightarrow \sup_{n\in\N}\left\{\int_X\simp_n\de\mu\right\}= \int_X f\de\mu\virgola
	\end{equation*}
	dove l'ultima implicazione è ovvia perché banalmente $\sup_{n\in\N}\left\{\int_X\simp_n\de\mu\right\}\le \int_X f\de\mu$.
\end{proof}

\begin{remark}\label{nota:ApprossimazioneIntegralePositiveConSemplici}
	Sfruttando la \cref{prop:DefinizioneEquivalenteIntegralePositive}, notiamo che data una funzione $f:X\to\Rpiu$ misurabile e positiva, con $(X,\A,\mu)$ spazio di misura, possiamo sempre scrivere l'integrale di $f$ come $\sup_{n\in\N}\left\{\int_X\simp_n\de\mu\right\}$, per qualche successione crescente $(\simp_n)$ di funzioni semplici e positive che converge puntalmente ad $f$, che esiste per il \cref{cor:LimSemCrescMis}.
\end{remark}


\begin{proposition}\label{prop:IntegralePositive}
	Date $f,g:X\to\Rpiu$ funzioni misurabili positive con $(X,\A,\mu)$ spazio di misura e dati $a,b$ reali non negativi valgono:
	\begin{enumerate}
		\item $\int (af+bg)\de\mu=a\int f\de\mu+b\int g\de\mu$ \label{PIP:add}
		\item $f\le g\Longrightarrow \int f\de\mu\le \int g\de\mu$ \label{PIP:mono}
	\end{enumerate}
\end{proposition}
\begin{proof}
	\begin{description}
		\item[\ref{PIP:add}] Siano $(\simp_n^f)_{n\in\N}$ e $(\simp_n^g)_{n\in\N}$ due successioni crescenti di funzioni semplici e positive convergenti rispettivamente ad $f$ e a $g$. Vale facilmente che la successione crescente di funzioni semplici e positive $(a\simp_n^f+b\simp_n^g)_{n\in\N}$ converge a $af+bg$. Allora per le \cref{prop:DefinizioneEquivalenteIntegralePositive,prop:IntegraleSemplici} abbiamo che
		\begin{equation*}
		\begin{split}
			a\int f\de\mu+b\int g\de\mu&=a\sup_{n\in\N}\int \simp_n^f\de\mu+b\sup_{n\in\N}\int \simp_n^g\de\mu \\
			&=\sup_{n\in\N}\int (a\simp_n^f+b\simp_n^g)\de\mu=\int (af+bg)\de\mu\punto
		\end{split}
		\end{equation*}
		
		\item[\ref{PIP:mono}] Segue facilmente dalla \cref{def:IntegralePositive}, in quanto se $\simp$ è una funzione semplice e positiva minore o uguale a $f$ è anche minore o uguale a $g$, perciò
		\begin{equation*}
		\int f\de\mu=\sup_{n\in\N}\left\{\int\simp\de \mu:\ 0\le \simp\le f\le g\ \wedge\ \simp \text{ semplice}\right\}\le \int g\de\mu\punto
	\end{equation*}
	\end{description}
\end{proof}

\begin{theorem}[Beppo Levi]\label{thm:BeppoLevi}
	Sia $(f_n)$ una successione crescente di funzioni misurabili positive, definita su uno spazio di misura $(X,\A,\mu)$ e a valori in $\Rpiu$, convergenti puntualmente ad una funzione $f$. Allora 
	\begin{equation*}
	\int_Xf\de\mu=\lim_{n\to\infty}\int_Xf_n\de\mu\punto
	\end{equation*}
\end{theorem}

\begin{proof}
	Innanzitutto notiamo che ha senso parlare dell'integrale di $f$, perché essendo $f(x)=\sup_{n\in\N}f_n(x)$ per ogni $x\in X$, per la \cref{prop:SupDiMisurabili} $f$ è una funzione misurabile e positiva.
	
	Come osservato nella \cref{nota:ApprossimazioneIntegralePositiveConSemplici}, per ogni $n$ esiste una successione crescente di funzioni semplici positive $(\simp_{n,k})_{k\in\N}$ che converge puntalmente a $f_n$. In particolare utilizzando la stessa costruzione del \cref{cor:LimSemCrescMis} notiamo che è possibile scegliere le $\simp_{n,k}$ crescenti anche rispetto a $n$. Allora la successione $(\simp_{n,n})$ converge puntalmente a $f$. Di conseguenza otteniamo proprio
	\begin{equation*}
		\int_X f\de\mu=\sup_{n\in\N}\int_X\simp_{n,n}\de\mu=\sup_{k,n\in\N}\int_X \simp_{n,k}\de\mu=\sup_{n\in\N}\int_X\punto f_n\de\mu=\lim_{n\to\infty}\int_Xf_n\de\mu\punto
	\end{equation*}
\end{proof}

\begin{corollary}\label{cor:IntegrazionePerSeriePositive}
	Sia $(f_n)$ una successione di funzioni misurabili e positive, definite su uno spazio di misura $(X,\A,\mu)$ e a valori in $\Rpiu$. Allora esiste $f:X\to\Rpiu$ misurabile e positiva tale che $f(x)=\sum_{n=0}^\infty f_n(x)$, per ogni $x\in X$, e $\int f\de\mu=\sum_{n=0}^\infty\int f_n\de\mu$.
\end{corollary}

\begin{proof}
	Per ogni $x\in X$, sia $s_n(x)=\sum_{k=0}^nf_n(x)$. Innanzitutto $(s_n)$ è una successione crescente di funzioni misurabili positive e $f(x)=\sum_{n=0}^\infty f_n(x)=\sup_{n\in\N}s_n(x)$, quindi $f$ è una funzione misurabile e positiva per la \cref{prop:SupDiMisurabili}. Dato che le $s_n$ covergono puntalmente a $f$, sono rispettate tutte le ipotesi del \cref{thm:BeppoLevi} e perciò vale
	\begin{equation*}
		\int f \de\mu=\lim_{n\to\infty}\int s_n\de\mu=\lim_{n\to\infty}\int \sum_{k=0}^nf_n\de\mu=\lim_{n\to\infty}\sum_{k=0}^n \int f_n\de\mu=\sum_{k=0}^\infty\int f_n\de\mu\virgola
	\end{equation*}
	che è proprio quello che volevamo dimostrare.
\end{proof}

\begin{lemma}\label{lemma:MisuraIntegrale}
	Sia $f:X\to\Rpiu$ una funzione misurabile positiva, con $(X,\A,\mu)$ spazio di misura, e sia $g:X\to Y$, con $(Y,\B)$ spazio misurabile, tale che $g^{-1}(B)\in\A$ per ogni $B\in\B$. Definiamo la funzione $\nu:\mathscr B\to\Rpiu$ ponendo, per ogni $B\in\B$,
	\begin{equation*}
		\nu(B)=\int_{g^{-1}(B)}f\de\mu\punto
	\end{equation*}
	Allora $\nu$ è una misura su $\B$.
\end{lemma}
\begin{proof}
	Per dimostrare che $(Y,\B,\nu)$ è uno spazio di misura ci basta dimostrare che $\nu$ è \sigadd, in quanto sappiamo già che $\B$ è una \sigalg\ e $\nu(\emptyset)=0$.
	
	Consideriamo quindi una sottofamiglia numerabile $(B_n)_{n\in\N}$ di $\B$, allora utilizzando il \cref{cor:IntegrazionePerSeriePositive} otteniamo
	\begin{multline*}
		\nu\left(\bigsqcup_{n\in\N}B_n\right)=\int_{g^{-1}(\sqcup B_n)}f\de\mu=\int_{\sqcup g^{-1}(B_n)}\sum_{n\in\N}\chi_{g^{-1}(B_n)}\cdot f\de\mu\\
		=\sum_{n\in\N}\int_{\sqcup g^{-1}(B_n)}\chi_{g^{-1}(B_n)}\cdot f\de\mu=
		\sum_{n\in\N}\int_{g^{-1}(B_n)}f\de\mu=\sum_{n\in\N}\nu(B_n)\virgola
	\end{multline*}
	che dimostra proprio la \sigadd\ di $\nu$.
\end{proof}

\begin{lemma}[Fatou] \label{lemma:Fatou}
	Sia $(f_n)$ una successione di funzioni misurabili positive, allora 
	\begin{equation*}
		\int_X\left(\liminf_{n\to\infty}f_n(x)\right)\de\mu\le \liminf_{n\to\infty}\int_Xf_n\de\mu\punto
	\end{equation*}
\end{lemma}

\begin{proof}
	Applichiamo il \cref{thm:BeppoLevi} alla successione di funzioni $(\inf_{k\ge n}f_k(x))_{n\in\N}$ che converge puntalmente in modo crescente alla funzione $\liminf_{n\to\infty}f_n(x)$ e otteniamo
	\begin{equation*}
		\int_X\left(\liminf_{n\to\infty}f_n\right)\de\mu=\int_X \sup_{n\in\N}\left\{\inf_{k\ge n}f_k\right\}\de\mu=\sup_{n\in\N}\int_X \inf_{k\ge n}f_k\de\mu\le \liminf_{n\to\infty}\int_Xf_n\de\mu\virgola
	\end{equation*}
	dove l'ultima disuguaglianza è vera per la monotonia dell'integrale. Infatti per la \cref{prop:IntegralePositive}, abbiamo che
	\begin{equation*}
		\int_X \inf_{k\ge n}f_k\de\mu\le \int_Xf_k\de\mu \ \forall k\ge n \Longrightarrow \int_X \inf_{k\ge n}f_k\de\mu\le \inf_{k\ge n}\int_Xf_k\de\mu\punto
	\end{equation*}
\end{proof}

\begin{definition}
	Una funzione misurabile $f:X\to\Rbar$, con $(X,\A,\mu)$ spazio di misura, si dice integrabile se 
	\begin{equation*}
		\LNorm f\doteqdot \int_X |f|\de\mu
	\end{equation*}
	ha valore finito. In particolare chiamiamo $\LNorm f$ norma integrale di $f$.
	
% 	La quantità $\LNorm f$ è detta norma integrale, anche se non è effettivamente una norma.
\end{definition}

\begin{definition}\label{def:IntegraleIntegrabili}
	Data una funzione integrabile $f:X\to\Rbar$, con $(X,\A,\mu)$ spazio di misura, definiamo il suo integrale come
	\begin{equation*}
		\int_X f\de\mu=\int_Xf_+\de\mu-\int_Xf_-\de\mu
	\end{equation*}
	dove $f_+$ ed $f_-$ sono definite nella \cref{def:FpiuFmeno}.
\end{definition}

\begin{remark}\label{nota:FpiuFmenoIntegrabili}
	Notiamo in particolare che le funzioni $f_+$ ed $f_-$, oltre ad essere positive e misurabili (per la \cref{nota:ProprietaFpiuFmeno}), sono anche integrabili.
\end{remark}

\begin{lemma}\label{lemma:DefinizioneEquivalenteIntegraleIntegrabili}
	Data una funzione integrabile $f:X\to\Rbar$, con $(X,\A,\mu)$ spazio di misura, e date due funzioni $g,h:X\to \Rbar$ integrabili e positive tali che $f=g-h$, vale che
	\begin{equation*}
		\int_X f\de\mu=\int_Xg\de\mu-\int_Xh\de\mu\punto
	\end{equation*}
\end{lemma}
\begin{proof}
	Poiché $f_++h=f_-+g$, per la \cref{prop:IntegralePositive} e ricordando che gli integrali di $f_+,f_-,g,h$ sono finiti, otteniamo
	\begin{gather*}
		\int  f_+\de\mu+\int  h\de\mu=\int  f_-\de\mu+\int  g\de\mu \\
		\Longrightarrow \int f\de\mu=\int f_+\de\mu-\int f_-\de\mu=\int g\de\mu-\int h\de\mu\punto
	\end{gather*}
\end{proof}



\begin{remark}\label{nota:MonotoniaIntegraleIntegrabili}
	Dalla definizione di funzioni integrabili e dalla \cref{prop:IntegralePositive}, segue facilmente che date $f,g:X\to\Rbar$ funzioni integrabili, con $(X,\A,\mu)$ spazio di misura, vale che se $f\le g$, allora $\int f\de\mu\le \int g\de\mu$.
\end{remark}

\begin{remark}\label{nota:IntegraleIntegrabiliSuMisurabili}
	Dalla \cref{prop:IntegraleSempliciSuMisurabili}, dimostrandolo prima per le funzioni positive, si ricava che data una funzione integrabile $f:X\to\Rbar$, con $(X,\A,\mu)$ spazio di misura, e dato $E\in\A$, vale che
	\begin{equation*}
		\int_E f \de\mu=\int_X f\cdot\chi_E\de\mu\punto
	\end{equation*}

\end{remark}



\begin{theorem}[Disuguaglianza di Chebyshev]\label{thm:DisuguaglianzaChebyshev}
	Data una funzione $f:X\to\Rbar$ integrabile, dove $(X,\A,\mu)$ è uno spazio di misura, e dato $\lambda>0$ reale vale
	\begin{equation*}
		\mu\left(\left\{x:\ |f(x)|>\lambda\right\}\right) \le \frac{\LNorm f}{\lambda}\punto
	\end{equation*}
\end{theorem}
\begin{proof}
	Sia $A_\lambda$ l'insieme degli $x$ tali che $|f(x)|>\lambda$. Ovviamente $A_\lambda$ è un insieme misurabile per la definizione di funzione integrabile.
	
	Grazie alla \cref{nota:MonotoniaIntegraleIntegrabili} e per la definizione di integrale di una funzione semplice abbiamo
	\begin{equation*}
		|f|\ge\lambda \cdot \chi_{A_\lambda} \implies \int_X |f|\de\mu \ge \int_X \lambda\cdot\chi_{A_\lambda}\de\mu=\lambda\mu(A_\lambda)\implies
		\LNorm{f}\ge \lambda\mu(A_\lambda)\virgola
	\end{equation*}
	che è equivalente alla tesi.
\end{proof}

\begin{corollary}\label{cor:SupportoIntegrabile}
	Sia $f:X\to\Rbar$ integrabile, dove $(X,\A,\mu)$ è uno spazio di misura, allora l'insieme degli $x\in X$ in cui $f$ non si annulla\footnote{In teoria della misura chiamiamo tale insieme supporto.} è \sigfin[o].
\end{corollary}
\begin{proof}
	Sia $A_n=\left\{x:\ |f(x)|>\frac 1n \right\}$, allora per il \cref{thm:DisuguaglianzaChebyshev} vale che $\mu(A_n)\le n \LNorm f$, quindi $A_n$ ha misura finita. Vale inoltre che $A\doteqdot\left\{x:\ f(x)\not=0 \right\}=\bigcup_{n\in\N}A_n$, quindi $A$ si scrive come unione numerabile di misurabili con misura finita e perciò è \sigfin[o], che è quello che volevamo.
\end{proof}


\begin{lemma}\label{lemma:L1NullaAlloraNulla}
	Data $f:X\to\Rbar$ integrabile, dove $(X,\A,\mu)$ è uno spazio di misura, risulta che $\LNorm{f}=0$ se e solo se
	\begin{equation*}
		\mu\left(\left\{x:\ f(x)\not=0\right\}\right)=0\punto
	\end{equation*}
\end{lemma}
\begin{proof}
	Dimostriamo innanzitutto che se $\LNorm{f}=0$ allora $\mu\left(\left\{x:\ f(x)\not=0\right\}\right)=0$.

	Per ogni $n\in\N$, chiamiamo $A_n$ l'insieme degli $x\in X$ tali che $|f(x)|>\frac 1n$.
	
	Dal \cref{thm:DisuguaglianzaChebyshev}, sfruttando $\LNorm{f}=0$ è facile ottenere $\mu(A_n)=0$. 
	Sfruttando quanto appena ottenuto insieme alla \sigsubadd[ità] della misura otteniamo facilmente la tesi:
	\begin{equation*}
		\mu\left(\left\{x:\ f(x)\not=0\right\}\right)=\mu\left(\bigcup_{n\in\N}\left\{x:\ |f(x)|>\frac1n\right\}\right)\le\sum_{n\in\N}\mu(A_n)=0\punto
	\end{equation*}
	
	Dimostriamo ora invece la freccia opposta dell'implicazione.
	
	Per ogni $0\le \simp \le |f|$ con $\simp$ semplice, vale che $\left\{x:\ \simp(x)\not=0\right\}\subseteq\left\{x:\ f(x)\not=0\right\}$ e quindi $\mu\left(\left\{x:\ \simp(x)\not=0\right\}\right)=0$. 
	
	Consideriamo ora la scrittura di $\simp$ come $\simp=\sum_{k=1}^n c_k\chi_{E_k}$, dove $c_k\in\Rbar$ e $E_k\in\A$. Abbiamo che, chiamati $F_k=E_k\cap \left\{x:\ \simp(x)\not=0\right\}$, $F_k\in\A$ perché intersezione di misurabili e inoltre 
	\begin{equation*}
		\simp=\sum_{k=1}^n c_k\chi_{F_k} \Longrightarrow \int_X\simp\de\mu=\sum_{k=1}^n c_k\mu(F_k)=0\virgola
	\end{equation*}
	dove l'ultima uguaglianza è vera perché $F_k\subseteq \left\{x:\ \simp(x)\not=0\right\}$ e perciò $\mu(F_k)=0$.
	
	Da questo discende facilmente, per la definizione stessa di $\LNorm{f}$, che
	\begin{equation*}
		\LNorm f=\sup\left\{ \int_X \simp\de\mu\ :\ 0\le \simp \le |f| \wedge \simp \text{ semplice}\right\}=0\punto
	\end{equation*}
\end{proof}

\begin{corollary}\label{cor:IntegraleAMenoDiTrascurabili}
	Sia $f:X\to\Rbar$ una funzione misurabile, dove $(X,\A,\mu)$ è uno spazio di misura, e sia $N\in\A$ trascurabile. Allora $f$ è integrabile su $X$ se e solo se è integrabile su $X\setminus N$ e in tal caso vale che
	\begin{equation*}
		\int_X f\de\mu=\int_{X\setminus N}f\de\mu\punto
	\end{equation*}
\end{corollary}
\begin{proof}
	Per l'additività dell'integrale e per la \cref{nota:IntegraleIntegrabiliSuMisurabili} abbiamo che
	\begin{equation*}
		\int_X f\de\mu=\int_{X\setminus N} f \de\mu+\int_N f\de\mu=\int_{X\setminus N}f\de\mu\virgola
	\end{equation*}
	dove l'ultima disuguaglianza è vera per il \cref{lemma:L1NullaAlloraNulla}, poiché infatti
	\begin{equation*}
		\left|\int_N f\de\mu\right|\le \int_N |f|\de\mu=0 \Longrightarrow \int_N f\de\mu=0\punto
	\end{equation*}
\end{proof}


Il fatto che l'integrale di una funzione integrabile sia invariante a meno di insiemi trascurabili sarà fondamentale e molto utilizzato nel resto della trattazione. Spesso parleremo infatti di integrale di una funzione anche se questa non è definita su un sottoinsieme trascurabile del dominio (infatti l'integrale sarebbe lo stesso per qualsiasi scelta di valori su quel trascurabile).


\begin{proposition}\label{prop:IntegraleIntegrabili}
	Date $f,g:X\to\Rbar$ funzioni integrabili con $(X,\A,\mu)$ spazio di misura e dati $a,b,\lambda\in\R$ valgono:
	\begin{enumerate}
		\item $af+bg$\footnote{La somma $af+bg$ di due funzioni integrabili è definita a meno che $af$ e $-bg$ siano entrambe $\pm \infty$, ma ciò accade al più in un insieme trascurabile del dominio (per il \cref{thm:DisuguaglianzaChebyshev}). Perciò per il \cref{cor:IntegraleAMenoDiTrascurabili}, come già detto, non ce ne dobbiamo preoccupare. } è integrabile e $\int (af+bg)\de\mu=a\int f\de\mu+b\int g\de\mu$ \label{PII:add}
		\item $\LNorm{\lambda f}=|\lambda|\LNorm f$ \label{PII:moltScalare}
		\item $|f|\le |g|\Longrightarrow \LNorm f\le \LNorm g$ \label{PII:mono}
		\item $\LNorm {f+g}\le\LNorm f+\LNorm g$ \label{PII:triang}
	\end{enumerate}
\end{proposition}
\begin{proof}
	La \ref{PII:add} è una facile conseguenza della definizione di integrale e dalla \cref{prop:IntegralePositive}.
	La \ref{PII:moltScalare} e la \ref{PII:mono} seguono facilmente dalla \ref{PII:add} e dalla \cref{nota:MonotoniaIntegraleIntegrabili}, mentre la \ref{PII:triang} si ottiene dalla \ref{PII:mono} in quanto $|f+g|\le|f|+|g|$.
\end{proof}



\begin{definition}
	Dato uno spazio di misura $(X,\A,\mu)$ definiamo
	\begin{equation*}
		\L(X,\A,\mu)=\{ f:X\to\Rbar \text{ integrabili} \}\punto
	\end{equation*}
	A volte utilizzeremo la notazione più compatta $\L$, quando sarà ovvio lo spazio di misura su cui stiamo lavorando.
\end{definition}


\begin{proposition}\label{prop:L1VettorialeConSeminorma}
	Lo spazio di funzioni $\L(X,\A,\mu)$ è uno spazio vettoriale e $\LNorm{\cdot}$ è una seminorma su $\L(X,\A,\mu)$.
\end{proposition}
\begin{proof}
	Dalla \cref{prop:IntegraleIntegrabili} segue facilmente che $\L(X,\A,\mu)$ è uno spazio vettoriale. 
	
	Altrettanto facilmente abbiamo anche che $\LNorm{\cdot}$ è una seminorma su $\L(X,\A,\mu)$, poiché la \cref{prop:IntegraleIntegrabili} dimostra che $\LNorm{\cdot}$ rispetta sia l'omogeneità che la disuguaglianza triangolare.
	
	Dal \cref{lemma:L1NullaAlloraNulla} ricaviamo però che esistono funzioni $f$ non identicamente nulle tali che $\LNorm{f}=0$. Quindi $\LNorm{\cdot}$ non è una norma su $\L(X,\A,\mu)$ ma solo una seminorma.
\end{proof}

\begin{remark}
	Dato che $\LNorm{\cdot}$ è una seminorma sullo spazio $\L$, possiamo parlare di convergenza in tale spazio. In particolare diremo che una successione di funzioni integrabili $(f_n)\subseteq \L(X,\A,\mu)$ converge ad $f$ in $\L$ se $\lim_{n\to\infty}\LNorm{f-f_n}=0$.
\end{remark}

\begin{theorem}[Convergenza dominata di Lebesgue]\label{thm:ConvergenzaDominata}
	Sia $(f_n)$ una successione di funzioni integrabili su $(X,\A,\mu)$ spazio di misura convergenti puntalmente a $f$ e sia $g$ integrabile tale che $|f_n(x)|\le g(x)$ per ogni $x\in X$ e per ogni $n\in\N$. Allora
	\begin{equation*}
		\int f \de\mu=\lim_{n\to\infty}\int f_n\de\mu\punto
	\end{equation*}
\end{theorem}
\begin{proof}
	Consideriamo le due successioni di funzioni misurabili positive $(g\pm f_n)$ (rispettivamente con il segno $+$ e il segno $-$). Tali funzioni sono banalmente positive in quanto $|f_n(x)|\le g(x)$, inoltre convergono puntualmente alle funzioni $g\pm f$. Per il \cref{lemma:Fatou} vale che
	\begin{equation*}
		\int g\de\mu\pm\int f\de\mu=\int (g\pm f) \de\mu \le \liminf_{n\to\infty} \int (g\pm f_n) \de\mu=\int g\de\mu +\liminf_{n\to\infty}\left( \pm\int f_n \de \mu\right)\punto
	\end{equation*}
	Perciò scegliendo rispettivamente il $+$ e il $-$ otteniamo
	\begin{equation*}
		\begin{cases}
			\int f \de\mu\le \liminf_{n\to\infty}\int f_n \de \mu \\
			\int f\de\mu \ge \limsup_{n\to\infty}\int f_n \de \mu
		\end{cases}
		\Longrightarrow \int f\de\mu = \lim_{n\to\infty}\int f_n \de \mu\punto
	\end{equation*}
\end{proof}

\begin{remark}\label{nota:ConvergenzaL1Dominata}
	Nelle stesse ipotesi del teorema precedente (\cref{thm:ConvergenzaDominata}), vale in particolare che $f_n\to f$ in $\L(X,\A,\mu)$. Infatti abbiamo che
	\begin{equation*}
		\int f \de\mu=\lim_{n\to\infty}\int f_n\de\mu \Longrightarrow \LNorm{f-f_n}=\int |f-f_n|\de\mu\to 0\punto
	\end{equation*}
\end{remark}

\begin{lemma}\label{lemma:SottospazioTrascurabili}
	L'insieme definito da
	\begin{equation*}
		\mathcal{N}=\{ f\in\L(X,\A,\mu) : \LNorm{f}=0\}
	\end{equation*}
	è un sottospazio vettoriale di $\L(X,\A,\mu)$.
\end{lemma}
\begin{proof}
	Per la \cref{prop:IntegraleIntegrabili} abbiamo che $\mathcal{N}$ è chiuso sia per moltiplicazione per scalare che per somma, quindi è uno spazio vettoriale in quanto contiene facilmente la funzione nulla.
\end{proof}

\begin{definition}
	Definiamo lo spazio vettoriale
	\begin{equation*}
		L^1(X,\A,\mu)\doteqdot \L(X,\A,\mu)/\mathcal{N}
	\end{equation*}
	dove $\mathcal{N}$ è il sottospazio vettoriale di $\L(X,\A,\mu)$ definito nel \cref{lemma:SottospazioTrascurabili}.
	
	Definiamo inoltre su tale spazio $L^1(X,\A,\mu)$ la norma indotta dalla seminorma già definita su $\L(X,\A,\mu)$. Cioè in particolare dato $f+\mathcal{N}\in L^1$ abbiamo che $\LNorm{f+\mathcal{N}}=\LNorm f$.
\end{definition}

\begin{remark}
	$L^1(X,\A,\mu)$, con la norma appena definita, è uno spazio vettoriale normato, cioè $\LNorm{\cdot}$ è effettivamente una norma su di esso. Questo è ovvio per il \cref{lemma:SottospazioTrascurabili} e per la definizion stessa di $L^1$
\end{remark}


\begin{theorem}[Integrazione per serie]\label{thm:IntegrazionePerSerie}
	Sia $(f_n)$ una successione di funzioni integrabili definite su uno spazio di misura $(X,\A,\mu)$ e a valori in $\Rbar$, tale che $\sum_{n=0}^{\infty}\LNorm{f_n}<\infty$. Allora vale che per quasi ogni $x\in X$ la serie $\sum_{n=0}^\infty f_n(x)$ è assolutamente convergente. 
	
	Detta poi $f:X\to\Rbar$ tale che $f(x)=\sum_{n=0}^\infty f_n(x)$ vale che $f\in \L(X,\A,\mu)$ e $\int_X f\de\mu=\sum_{n=0}^\infty \int_X f_n\de\mu$.
\end{theorem}

\begin{proof}
	Dato $n\in \N$, sia $g_n(x)=\sum_{k=0}^n|f_n(x)|$ per ogni $x\in X$. 
	
	Le $g_n$ sono funzioni misurabili (perché somma di misurabili), positive e crescenti, quindi convergono puntualmente ad una funzione $g:X\to\Rbar$. In particolare per il \cref{thm:BeppoLevi} vale inoltre che $\int g\de\mu=\lim_{n\to\infty}\int g_n\de\mu$.
	
	D'altro canto sappiamo che 
	\begin{equation*}
		\int g_n\de\mu=\sum_{k=0}^n\LNorm{ f_k} \Longrightarrow \lim_{n\to\infty}\int g_n\de\mu=\sum_{k=0}^\infty \LNorm{ f_k}<\infty\virgola
	\end{equation*}
	quindi $g$ è integrabile.
	
	Chiamiamo ora $N=\{ x\in X:g(x)=\infty\}$, allora per il \cref{thm:DisuguaglianzaChebyshev} abbiamo che $\mu(N)=0$, cioè $g$ assume valori finiti quasi ovunque. Questo implica quindi che $\sum_{n=0}^\infty f_n(x)$ converge assolutamente quasi ovunque, cioè in $X\setminus N$.
	
	Sia ora $s_n=\sum_{k=0}^n f_k$, allora vale che
	\begin{equation*}
		|s_n(x)|=\left|\sum_{k=0}^n f_k(x)\right|\le \sum_{k=0}^n |f_k(x)|=g_n(x)\le g(x)\virgola
	\end{equation*}
	quindi la successione $(s_n)$ è dominata dalla funzione $g$.
	
	Inoltre, poiché $\sum_{n=0}^\infty f_n(x)$ converge assolutamente in $X\setminus N$, $s_n=\sum_{k=0}^n f_k$ converge puntalmente alla funzione $f(x)=\sum_{n=0}^\infty f_n(x)$ nello stesso insieme.
	
	Sono quindi rispettate tutte le ipotesi del \cref{thm:ConvergenzaDominata} per la successione di funzioni $(s_n)$ con dominio $X\setminus N$, perciò
	\begin{equation*}
		\int_{X\setminus N} f\de\mu=\lim_{n\to\infty}\int_{X\setminus N}s_n\de\mu=\lim_{n\to\infty}\sum_{k=0}^n\int_{X\setminus N}  f_k\de\mu= \sum_{k=0}^\infty\int_{X\setminus N}f_k\de\mu\punto
	\end{equation*}

	Per quanto già osservato come conseguenza del \cref{cor:IntegraleAMenoDiTrascurabili}, è indifferente fare l'integrale su $X$ e su $X\setminus N$ (su $N$ possiamo definire $f$ arbitrariamente); perciò otteniamo proprio quello che volevamo
	\begin{equation*}
		\int_X f\de\mu=\int_{X\setminus N} f\de\mu=\sum_{n=0}^\infty\int_{X\setminus N}f_n\de\mu=\sum_{n=0}^\infty\int_X f_n\de\mu\punto
	\end{equation*}
\end{proof}

\begin{remark}\label{nota:ConvergenzaL1IntegrazionePerSerie}
	Da notare che dalla dimostrazione, ricordando la \cref{nota:ConvergenzaL1Dominata}, segue anche che la convergenza di $s_n$ ad $f$ è in $\L(X,\A,\mu)$ ed è dominata, infatti $|s_n|\le g$.
\end{remark}


\begin{corollary}
	Dato uno spazio di misura $(X,\A,\mu)$, lo spazio $L^1(X,\A,\mu)$ è completo.
\end{corollary}

\begin{proof}
	Segue immediatamente dal \cref{thm:IntegrazionePerSerie}, ricordando che uno spazio normato è completo se e solo se ogni serie totalmente convergente è convergente nello spazio.
\end{proof}

\begin{proposition}\label{prop:L1ImplicaSottosuccessioneQuasiOvunque}
	Sia $(f_n)$ una successione di funzioni integrabili convergenti ad $f$ nello spazio $\L(X,\A,\mu)$. Allora esiste una sottosuccessione $(f_{n_k})_{k\in\N}$ che converge quasi ovunque ad $f$ ed è dominata in $\L(X,\A,\mu)$.
\end{proposition}

\begin{proof}
	Per la \cref{nota:ConvergenzaL1IntegrazionePerSerie} $\LNorm{f-f_n}\to 0$, quindi possiamo definire $(f_{n_k})_{k\in\N}$ tale che $\LNorm{f-f_{n_k}}<2^{-k}$ e la successione $(n_k)_{k\in\N}$ sia crescente.
	
	Sia ora $h_k=f_{n_{k+1}}-f_{n_k}$ per ogni $k\in\N$. La successione $(h_k)$ rispetta le ipotesi del \cref{thm:IntegrazionePerSerie}, in quanto $\sum_{k=0}^\infty \LNorm {h_k}\le \sum_{k=0}^\infty 2^{-k}$, quindi la serie $\sum_{k=0}^\infty h_k$ converge quasi ovunque ed è dominata in $\L(X,\A,\mu)$.
	
	Abbiamo quindi ottenuto che $f_{n_k}=f_{n_0}+\sum_{i=0}^{k-1}(f_{n_{i+1}}-f_{n_i})=f_{n_0}+\sum_{i=0}^{k-1} h_i$ converge quasi ovunque alla funzione $f$ e la successione $(f_{n_k})_{k\in\N}$ è dominata.
\end{proof}

\begin{remark}
	Notiamo che la tesi della \cref{prop:L1ImplicaSottosuccessioneQuasiOvunque} non può essere migliorata. Infatti in genere non è vero che se una successione di funzioni integrabili $(f_n)$ converge ad $f$ in $\L(X,\A,\mu)$, allora converge quasi ovunque ad $f$.
\end{remark}
\begin{proof}
	Sia $(f_n)$ la successione di funzioni integrabili definita da $f_n=\chi_{I_n}$, dove $I_n=\left[\frac{n-2^k}{2^k},\frac{n-2^k+1}{2^k}\right]$ se $2^k\le n<2^{k+1}$. In particolare quindi i primi termini della successione $(I_n)$ sono $I_1=\left[0,1\right],I_2=\left[0,\frac 12\right],I_3=\left[\frac 12,1\right],I_4=\left[0,\frac 14\right],\cdots$. 
	
	È facile verificare che in $\L(X,\A,\mu)$ le $f_n$ convergono a $f=0$, ma è altrettanto facile vedere che non convergono quasi ovunque.

\end{proof}






\section{Misura prodotto}
Trattiamo ora la possibilità di rendere il prodotto di due spazi di misura uno spazio di misura. 
Questo risulterà facile sfruttando, come strumento principale, il teorema di estensione di \carat{}. 
Si potrebbe decidere di esplicitare direttamente la misura prodotto, come integrale della misura delle sezioni, piuttosto che sfruttare il teorema di \carat{} per assicurarne l'esistenza. 
Questa via, più diretta e meno tecnica, è effettivamente presa da vari libri di teoria della misura. Qui si è deciso di procedere diversamente sia per ignoranza iniziale degli autori, sia perché in effetti si dimostra, ed è a nostro parere molto istruttivo, che le due definizioni sono equivalenti.

L'idea portante di tutta questa sezione è quella di mostrare che la misura più ovvia da porre sullo spazio prodotto, cioè quella derivante dalla premisura sui rettangoli, è in effetti sufficientemente  ricca da assicurare molte proprietà allo spazio di misura a essa relativo ed in particolare ha sufficienti proprietà per dimostrare i teoremi di Fubini e Tonelli, che assicurano, sotto ipotesi relativamente lascive, la possibilità di scambiare tra loro gli operatori di integrazione.

È importante enfatizzare da subito come vi sia una certa arbitrarietà nel decidere quale sia la \sigalg{} dei misurabili sullo spazio prodotto. Si potrebbe decidere di lavorare sulla \sigalg{} di \carat{}, ma questa risulterebbe troppo ampia, oppure lavorare sulla \sigalg{} generata dai rettangoli, che invece risulta spesso (in particolare nel caso fondamentale della misura di Lebesgue) troppo ristretta.
L'ambiente giusto in cui lavorare risulterà infatti essere la \sigalg{} generata dai rettangoli nel caso in cui le misure iniziali non siano complete, mentre sarà più adatto lavorare nel completamento di questa \sigalg{} nel caso in cui le misure siano complete.



Nella dimostrazione del primo teorema, che fondamentalmente verifica le ipotesi del teorema di \carat{}, proponiamo una via tecnica (che sfrutta la teoria degli integrali), che diverge da quello che potrebbe essere un modo standard di procedere. Questo è più breve di una dimostrazione fatta con le mani e allo stesso tempo rende chiara da subito la possibilità di avere scambi tra gli integrali in uno spazio prodotto. 

\begin{definition}\label{def:SezioneProdotto}
	Dati due insiemi $X,Y$ e $x\in X$ fissato, indichiamo con la notazione $J^X_x$ la funzione $J^X_x:\mathcal P(X\times Y) \to \mathcal P(Y)$ che restituisce le sezioni di un insieme:
	\begin{equation*}
		\forall E\subseteq X\times Y:\ J^X_x(E)=\{y\in Y:\ (x,y)\in E\}\punto
	\end{equation*}
	Dove ovvio verrà omesso l'apice che indica l'insieme.
\end{definition}

\begin{definition}\label{def:SemianelloProdotto}
	Dati due spazi misurabili $(X,\A)$, $(Y,\B)$ definiamo, con abuso di notazione, $\A\times\B$ come la famiglia degli insiemi $A\times B$ dove $A\in\A$ e $B\in\B$, che chiameremo rettangoli.
\end{definition}

\begin{proposition}\label{prop:SemianelloProdotto}
	Dati due spazi misurabili $(X,\A)$, $(Y,\B)$, la famiglia $\A\times\B\subseteq \mathcal P(X\times Y)$ è un \semiring{}.
\end{proposition}
\begin{proof}
	Ovviamente $\emptyset\in\A\times\B$ visto che $\emptyset\in\A$ e $\emptyset\in\B$.
	
	Per concludere, dati $A\times B, A'\times B'\in\A\times\B$, mostriamo esplicitamente una scrittura dell'intersezione e della differenza che rispetti le proprietà che deve avere un \semiring{}:
	\begin{align*}
		(A\times B)\cap (A'\times B') &= (A\cap A')\times(B\cap B')\virgola \\
		(A\times B)\setminus (A'\times B') &= ((A\setminus A')\times B)\sqcup( (A\cap A')\times (B\setminus B') )\punto
	\end{align*}
\end{proof}

\begin{definition}\label{def:PremisuraProdotto}
	Dati $(X,\A,\mu)$ e $(Y,\B,\nu)$ spazi di misura, denotiamo con $\overline{\mu\nu}$ la funzione $\overline{\mu\nu}:\A\times\B\to\Rpiu$ definita in modo che 
	\begin{equation*}
		\forall A\times B\in\A\times\B:\ \overline{\mu\nu}(A\times B)=\mu(A)\nu(B)\punto
	\end{equation*}
\end{definition}

\begin{theorem}\label{thm:PremisuraProdotto}
	Dati $(X,\A,\mu)$ e $(Y,\B,\nu)$ spazi di misura, allora $(X\times Y,\A\times\B,\overline{\mu\nu})$ è uno spazio di misura elementare, dove $\A\times\B$ e $\overline{\mu\nu}$ sono definiti nelle \cref{def:SemianelloProdotto,def:PremisuraProdotto}.
\end{theorem}
\begin{proof}
	Avendo già verificato che $\A\times\B$ è un \semiring{} nella \cref{prop:SemianelloProdotto}, è sufficiente mostrare che $\overline{\mu\nu}$ è \sigadd{}.
	
	Sia $(A_n\times B_n)_{n\in\N}\subseteq \A\times \B$ una partizione disgiunta di $A\times B$.
	
	Per ogni $x\in X$, essendo gli $A_n\times B_n$ disgiunti, risulta
	\begin{equation}\label{eq:PremisuraProdottoSerie}
		\bigsqcup_{n\in\N} J_x(A_n\times B_n) = J_x(A\times B)  \implies \sum_{n\in\N}\nu\left(J_x(A_n\times B_n)\right)=\nu\left(J_x(A\times B)\right)\virgola
	\end{equation}
	dove nell'implicazione abbiamo applicato la \sigadd[ità] di $\nu$.
	
	Applicando la sola definizione della misura prodotto e dell'integrale per funzioni semplici, fissato $C\times D\in \A\times\B$, abbiamo
	\begin{equation}\label{eq:StupidaIdentitaIndicatrici}
		\overline{\mu\nu}(C\times D)=\mu(C)\nu(D)=\int_{X}\chi_C(x)\nu(D)\de \mu(x)=\int_X \nu\left(J_x(C\times D)\right)\de\mu(x)\virgola
	\end{equation}
	dove l'ultima uguaglianza segue dalla facile identità $\nu\left(J_x(C\times D)\right)=\chi_C(x)\nu(D)$.
	
	Ora unendo le \cref{eq:PremisuraProdottoSerie,eq:StupidaIdentitaIndicatrici}, sfruttando il \cref{cor:IntegrazionePerSeriePositive}, otteniamo
	\begin{align*}
		\overline{\mu\nu}(A\times B)&=\int_X\nu(J_x(A\times B))\de \mu(x)=\int_X\sum_{n\in\N}\nu\left(J_x(A_n\times B_n)\right)\de\mu(x)\\
		&=\sum_{n\in\N}\int_X\nu\left(J_x(A_n\times B_n)\right)\de\mu(x)=\sum_{n\in\N}\overline{\mu\nu}(A_n\times B_n)\virgola
	\end{align*}
	che è la tesi cercata.
\end{proof}

La proposizione seguente è la prima che rende chiaro il fatto che l'ambiente di lavoro finale potrà essere vario. Infatti piuttosto che dimostrarla per soli insiemi misurabili (non ancora definiti) la si deve dimostrare per insiemi trascurabili qualunque, per poi poterla sfruttare indipendentemente da quale \sigalg{} dei misurabili verrà scelta.

\begin{proposition}\label{prop:TrascurabiliProdotto}
	Dato lo spazio di premisura $(X\times Y,\A\times\B,\overline{\mu\nu})$, che si è mostrato essere di premisura nel \cref{thm:PremisuraProdotto}, sia $\mu\nu^*:\mathcal P(X\times Y)\to\Rpiu$ la misura esterna associata a $\overline{\mu\nu}$.
	
	Se $E\subseteq X\times Y$ è trascurabile rispetto a $\mu\nu^*$, allora per $\mu$-quasi ogni $x\in X$ l'insieme $J_x^X(E)$ è trascurabile\footnote{Il fatto che un insieme sia trascurabile non implica che sia misurabile come osservato nella \cref{def:TrascurabiliMisura}.} rispetto alla misura $\nu$.
\end{proposition}
\begin{proof}
	Chiamiamo $\mu^*,\nu^*$ le misure esterne, definite come nella \cref{prop:MisuraEsternaDiPremisura}, associate alle misure\footnote{Uno spazio di misura è sempre ovviamente anche uno spazio di premisura.} $\mu,\nu$.
	
	Per $\delta>0$, sia $X_\delta$ l'insieme degli $x\in X$ tali che $\nu^*(J_x^X(E))\ge\delta$ e sia $m_\delta=\mu^*(X_\delta)$.
	
	Essendo $E$ trascurabile, se per assurdo $m_\delta>0$, esiste una famiglia di rettangoli $(A_i\times B_i)_{i\in\N}\subseteq \A\times\B$ la cui unione contiene $E$ e tali che
	\begin{equation*}
		\sum_{i\in\N}\mu(A_i)\nu(B_i)=\sum_{i\in\N}\overline{\mu\nu}(A_i\times B_i)< \delta m_\delta \punto
	\end{equation*}
	Chiamiamo $f:X\to\Rpiu$ la funzione definita come
	\begin{equation*}
		f(x)=\sum_{i\in\N}\chi_{A_i}(x)\nu(B_i)
	\end{equation*}
	e notiamo subito che $f$ è una funzione misurabile positiva, per la \cref{prop:SupDiMisurabili}, essendo una serie numerabile di funzioni indicatrici di misurabili.
	
	Per la sola definizione di misura esterna si ottiene che per ogni $x\in X$
	\begin{equation*}\label{eq:DisTrascurabiliProdotto}
		f(x)=\sum_{i\in\N}\chi_{A_i}(x)\nu(B_i)\ge\nu^*(J_x^X(E))\virgola
	\end{equation*}
	poiché i $B_i$, tali che i rispettivi $A_i$ contengano $x$, sono un ricoprimento numerabile di $J_x^X(E)$.
	
	Allora, applicando quest'ultima disuguaglianza aggiungendo l'ipotesi $x\in X_\delta$, otteniamo il contenimento $X_\delta\subseteq \{x\in X:f(x)\ge\delta\}$ e quindi ne ricaviamo che l'insieme $\{x\in X:f(x)\ge\delta\}$ ha misura\footnote{Qui si può parlare di misura e non di misura esterna poiché $f$ è una funzione misurabile.} maggiore o uguale a $m_\delta$.
	Quindi il \cref{thm:DisuguaglianzaChebyshev} ci assicura che $\LNorm f\ge \delta m_\delta$.

	Però, applicando il \cref{cor:IntegrazionePerSeriePositive}, otteniamo
	\begin{equation*}
		\delta m_\delta \le\LNorm f=\int_X \sum_{i\in\N}\chi_{A_i}(x)\nu(B_i)\de\mu(x)= \sum_{i\in\N}\int_X \chi_{A_i}(x)\nu(B_i)\de\mu(x)=\sum_{i\in\N}\mu(A_i)\nu(B_i)<\delta m_\delta\virgola
	\end{equation*}
	che mostra l'assurdo e dimostra $m_\delta=0$.
	
	Per concludere basta notare che l'insieme degli $x\in X$ tali che $\nu^*(J_x^X(E))>0$ si ottiene come unione numerabile degli $X_{\frac 1i}$, che sono tutti trascurabili, ed è perciò trascurabile anch'esso per la \cref{nota:UnioneTrascurabili} come voluto.
\end{proof}

\begin{definition}\label{def:SigAlgProdotto}
	Dati $(X,\A,\mu)$ e $(Y,\B,\nu)$ spazi di misura, chiamiamo $\A\otimes\B\subseteq \mathcal P(X\times Y)$ la \sigalg{} generata da $\A\times\B$. 
\end{definition}

\begin{proposition}\label{prop:SezioniMisurabili}
	Dati $(X,\A,\mu)$ e $(Y,\B,\nu)$ spazi di misura ed $E\in\A\otimes\B$, le sue sezioni trasversali sono misurabili, cioè
	\begin{align*}
		\forall x\in X:\ &J^X_x(E)\in\B \virgola \\
		\forall y\in Y:\ &J^Y_y(E)\in\A \punto
	\end{align*}
\end{proposition}
\begin{proof}
	\newcommand{\C}{\ensuremath{\mathscr C}}
	Sia $\C\subset\mathcal P(X\times Y)$, l'insieme dei sottoinsiemi di $X\times Y$ tali che tutte le sezioni trasversali (come definite implicitamente nell'enunciato) sono misurabili.
	Dimostriamo ora che $\A\times\B\subseteq \C$ e che $\C$ è una \sigalg{}.
	
	Dato un rettangolo $A\times B\in\A\times\B$, è chiaro che le sue sezioni trasversali o sono $A$ o $B$ o vuote e in ogni caso risultano misurabili, perciò vale $A\times B\in\C$ e visto che il rettangolo è stato scelto arbitrariamente se ne ricava $\A\times\B\subseteq \C$.
	
	Dato $E\in\C$, è chiaro che $E^\mathsf{c}\in\C$, visto che le sezioni del complementare sono il complementare delle sezioni e perciò sono anch'esse misurabili.
	
	Data una successione $(E_i)_{i\in\N}$, le sezioni dell'unione $\bigcup_{i\in\N}E_i$ sono facilmente l'unione delle sezioni e quindi anch'esse misurabili. Perciò, per definizione di $\C$, anche $\bigcup_{i\in\N}E_i\in\C$.
	Allora unendo quanto detto si ricava come cercato che $\C$ è una \sigalg{}.
	
	Per concludere basta ricordare che $\A\otimes\B$ è la più piccola \sigalg{} che contiene $\A\times\B$ e visto che $\C$ è una \sigalg{} che contiene $\A\times\B$, deve essere $\A\otimes\B\subseteq\C$. Allora $E\in\A\otimes\B$ implica $E\in\C$ e per definizione di $\C$ questo comporta la tesi.
\end{proof}

\begin{definition}\label{def:MisuraProdotto}
	Dati $(X,\A,\mu)$ e $(Y,\B,\nu)$ spazi di misura, chiamiamo $(X\times Y,\A\otimes\B,\mu\nu)$ lo spazio di misura prodotto, dove $\mu\nu$ è la riduzione alla sola \sigalg{} generata da $\A\times\B$ della misura, associata alla premisura $\overline{\mu\nu}$, assicurataci dal \cref{thm:EstensioneCaratheodory}.
\end{definition}


Ora che abbiamo fissato la \sigalg{} dei misurabili, fermiamoci un attimo a riflettere sulla scelta. Perché piuttosto non abbiamo scelto il suo completamento? Oppure perché non abbiamo scelto la \sigalg{} degli insiemi che hanno ogni sezione misurabile? 
Il motivo è che nessuna di queste ci avrebbe assicurato, nel caso generale, la veridicità della \cref{prop:SezioniMisurabili} e dei teoremi che seguiranno.

Ora andiamo a dimostrare i teoremi di Fubini e Tonelli, che permettono di scambiare integrali, nel caso in cui i misurabili siano proprio $\A\otimes\B$. In seguito li ridimostreremo, in maniera più concisa, nel caso in cui le misure di partenza siano complete e i misurabili piuttosto che essere $\A\otimes\B$, sono il suo completamento.

Le dimostrazioni che seguono fondamentalmente vanno per passi e dimostrano la commutazione tra integrali per funzioni gradualmente più complicate: indicatrici di insiemi, semplici, misurabili positive ed infine integrabili.

D'ora in poi parleremo dello spazio di misura prodotto dando per scontato la notazione associata ai singoli spazi di misura, cioè se parliamo di $(X\times Y,\A\otimes\B,\mu\nu)$ è sottointeso che corrisponde agli spazi di misura $(X,\A,\mu)$ e $(Y,\B,\nu)$.

\begin{remark}\label{nota:FunzioniMisProdotto}
	Dato lo spazio di misura prodotto $(X\times Y,\A\otimes\B,\mu\nu)$ ed una funzione $f:X\to\Rbar$ misurabile in $(X,\A,\mu)$, la funzione associata $\hat f:X\times Y\to\Rbar$ definita come $\hat f(x,y)=f(x)$ è misurabile.
\end{remark}
\begin{proof}
	Sfruttando le proprietà basilari della controimmagine e la definizione di $\hat f$, per ogni $A$ aperto di $\R$, otteniamo
	\begin{equation*}
		\hat f^{-1}(A)=f^{-1}(A)\times Y \punto
	\end{equation*}
	Allora, essendo $f^{-1}(A)$ misurabile in $(X,\A,\mu)$ per l'ipotesi di misurabilità di $f$ ed $Y\in\B$, abbiamo l'appartenenza $\hat f^{-1}(A)\in \A\times\B$ e cioè $\hat f^{-1}(A)$ misurabile nel prodotto.
	Ma poiché questo è valido per ogni $A$ aperto di $\R$ ne concludiamo che $\hat f$ è misurabile nel prodotto.
\end{proof}


\begin{proposition}\label{prop:PreTonelli}
	Dato lo spazio di misura prodotto $(X\times Y,\A\otimes\B,\mu\nu)$, per ogni insieme misurabile $E\in\A\otimes\B$ \sigfin[o] vale la seguente espressione per la misura:
	\begin{equation*}
		\mu\nu(E)=\int_X\nu\left(J_x^X(E)\right)\de\mu(x)=\int_Y\mu\left(J_y^Y(E)\right)\de\nu(y)\punto
	\end{equation*}
\end{proposition}
\begin{proof}
	\newcommand{\E}{\ensuremath{\mathscr E}}
	Dimostriamo solo la prima uguaglianza della tesi, la seconda risulterà di conseguenza vera per simmetria.
	
	Sia $\E\subseteq \A\otimes\B$ l'insieme degli insiemi misurabili sul prodotto con \emph{misura finita} che rispettano la tesi\footnote{Più correttamente $\E$ contiene gli insiemi tali che gli integrali dell'enunciato sono ben definiti e coincidono.} (solo la prima uguaglianza).
	
	Dimostriamo che $\E$ contiene i rettangoli ed è chiuso per intersezione numerabile monotona e unione numerabile disgiunta.
	
	Sfruttando solo le definizioni degli operatori in gioco, si ha
	\begin{equation*}
		\forall A\times B\in\A\times\B:\ \mu\nu(A\times B)=\mu(A)\nu(B)=\int_X \chi_A(x)\nu(B)\de\mu(x)=\int_X\nu\left(J_x^X(A\times B)\right)\de\mu(x)
	\end{equation*}
	e questo dimostra che $\A\times\B\subseteq \E$ come voluto.
	
	Fissati $(E_i)_{i\in\N}\subseteq \E$ disgiunti, è banale verificare
	\begin{equation*}
		\bigsqcup_{i\in\N}J_x^X(E_i)=J_x^X\left(\bigsqcup_{i\in\N}E_i\right)\virgola
	\end{equation*}
	da cui segue, ricordando che la tesi vale per gli $E_i$ ed applicando la commutazione tra integrale e serie mostrata nel \cref{cor:IntegrazionePerSeriePositive}, l'uguaglianza
	\begin{align*}
		\mu\nu\left(\bigsqcup_{i\in\N}E_i\right)&=\sum_{i\in\N}\mu\nu(E_i)=\sum_{i\in\N}\int_X \nu\left(J_x^X(E_i)\right)\de\mu(x)\\
		&=\int_X\sum_{i\in\N}\nu\left(J_x^X\left(E_i\right)\right)\de\mu(x)=\int_X\nu\left(J_x^X\left(\bigsqcup_{i\in\N}E_i\right)\right)\punto
	\end{align*}
	Quest'ultima uguaglianza dimostra, nel caso in cui l'unione degli $E_i$ abbia misura finita, che $\bigsqcup_{i\in\N}E_i\in\E$ e perciò, vista la scelta arbitraria degli $E_i$, ne ricaviamo che $\E$ è chiuso per unione disgiunta.
	
	Infine siano $(E_i)_{i\in\N}\subseteq \E$ che si contengono decrescentemente $E_{i+1}\subseteq E_i$.
	
	Allora, ricordando il \cref{cor:LimiteMonotonoDecrescenteMisura} e il \cref{thm:ConvergenzaDominata}, otteniamo
	\begin{align*}
		\mu\nu\left(\bigcap_{i\in\N}E_i\right)&=\lim_{i\in\N}\mu\nu(E_i)=
		\lim_{i\in\N}\int_X\nu\left(J_x^X(E_i)\right)\de\mu(x)\\
		&=\int_X\lim_{i\in\N}\nu\left(J_x^X(E_i)\right)\de\mu(x)=
		\int_X\nu\left(J_x^X\left(\bigcap_{i\in\N}E_i\right)\right)\de\mu(x)\virgola
	\end{align*}
	dove nei vari passaggi abbiamo usato implicitamente che gli $E_i$ si contengono vicendevolmente.
	È facile verificare che quanto scritto equivale a dire $\bigcap_{i\in\N}E_i\in\E$, perciò, sempre per l'arbitrarietà della scelta degli $E_i$, ne segue che $\E$ è chiuso per intersezione monotona.
	
	Per quanto detto, seguendo la notazione del \cref{cor:ChiusuraMonotonaInsiemiFiniti}, abbiamo dimostrato che $\widehat{\widehat{{\A\times\B}^F}^F}$ è un sottoinsieme di $\E$ e perciò appunto per il \cref{cor:ChiusuraMonotonaInsiemiFiniti} fissato $F\in\A\otimes\B$ di misura finita, visto che $\A\otimes\B$ è un sottoinsieme dei misurabili secondo \carat{}, esiste $E\in\E$ che contiene $F$ e tale che $\mu\nu(E\setminus F)=0$. 
	
	Però per la \cref{prop:TrascurabiliProdotto} l'insieme trascurabile $E\setminus F$ ha quasi ogni sezione trascurabile a sua volta.
	Allora applicando la \cref{prop:SezioniMisurabili} e ricordando il \cref{lemma:L1NullaAlloraNulla} ne discende banalmente che $E\setminus F\in\E$.
	Perciò vale
	\begin{align*}
		\mu\nu(F)&=\mu\nu(E)-\mu\nu(E\setminus F)=\int_X\nu\left(J_x^X(E)\right)\de\mu(x)-\int_X\nu\left(J_x^X(E\setminus F)\right)\de\mu(x)\\
		&=\int_X\nu\left(J_x^X(E\setminus F)\right)-\nu(\left(J_x^X(E)\right)\de\mu(x)
		=\int_X\nu\left(J_x^X(F)\right)\de\mu(x)\virgola
	\end{align*}
	dove nei passaggi intermedi abbiamo ripetutamente usato che $E$ contiene $F$.
	Ma l'identità equivale a $F\in\E$ e perciò, poiché $F$ è un qualsiasi insieme misurabile sul prodotto con misura finita, è dimostrato che ogni insieme di misura finita appartiene a $\E$.
	
	Per concludere la dimostrazione sia $G\in\A\otimes\B$ un insieme misurabile \sigfin[o]. Per l'ipotesi di \sigfin[ezza], esiste una famiglia $(F_i)_{i\in\N}\subseteq \A\otimes\B$ di insiemi disgiunti di misura finita la cui unione coincide con $G$.
	
	Ricordando la dimostrazione del fatto che $\E$ è chiuso per unione disgiunta, notiamo che abbiamo già dimostrato che l'unione disgiunta di elementi di $\E$ rispetta anch'essa la tesi, indipendentemente dal fatto che sia anch'essa di misura finita o meno. Questo però dimostra che la tesi vale anche per $G$ visto che $F_i\in\E$, essendo misurabili di misura finita, e $G$ si può scrivere come unione numerabile disgiunta degli $F_i$. 
\end{proof}

\begin{theorem}[Tonelli]\label{thm:Tonelli}
	Dato $(X\times Y,\A\otimes\B,\mu\nu)$ uno spazio prodotto, per ogni funzione $f:X\times Y\to\Rbar$ misurabile positiva a supporto \sigfin[o] vale
	\begin{equation*}
		\int_{X\times Y}f(u)\de\mu\nu(u)=\int_X\int_Y f(x,y)\de\nu(y)\de\mu(x)=\int_Y\int_X f(x,y)\de\mu(x)\de\nu(y)\punto
	\end{equation*}
\end{theorem}
\begin{proof}
	\newcommand{\E}{\ensuremath{\mathscr E}}
	Sia $\E$ l'insieme delle funzioni misurabili con supporto \sigfin[o] che rispettano la prima uguaglianza della tesi (è ovvio che se dimostriamo la prima uguaglianza, la seconda segue per simmetria).
	
	La \cref{prop:PreTonelli} implica che, per ogni $E\in\A\otimes\B$ a supporto \sigfin[o], vale
	\begin{equation*}
		\int_{X\times Y}\chi_E(u)\de\mu\nu(u)=\mu\nu(E)=\int_X\nu\left(J_x^X(E)\right)\de\mu(x)=\int_X\int_Y\chi_E(x,y)\de\nu(y)\de\mu(x)\virgola
	\end{equation*}
	che equivale a dire che le funzioni indicatrici dei misurabili \sigfin[i] appartengono a $\E$.
	
	Inoltre, per la linearità dell'operatore di integrale, è facile convincersi che $\E$, visto che contiene le indicatrici degli insiemi misurabili \sigfin[i], contiene anche le funzioni semplici positive a supporto \sigfin[o].
	
	Per concludere sia $f:X\times Y\to\Rbar$ una funzione misurabile a supporto \sigfin[o].
	Per il \cref{cor:LimSemCrescMis} esiste una famiglia crescente $(f_i)_{i\in\N}$ di funzioni semplici positive che convergono puntualmente ad $f$. Essendo $f$ a supporto \sigfin[o] ed essendo le $f_i$ sempre minori di $f$ anche il loro supporto è \sigfin[o] e perciò $f_i\in\E$ per ogni $i\in\N$.
	
	Infine, applicando ripetutamente il \cref{thm:BeppoLevi} e ricordando la monotonia dell'integrale, otteniamo
	\begin{align*}
		\int_{X\times Y}f(u)\de\mu\nu(u)
		&=\lim_{i\in\N}\int_{X\times Y}f_i(u)\de\mu\nu(u)
		=\lim_{i\in\N}\int_X\int_Y f_i(x,y)\de\nu(y)\de\mu(x)\\
		&=\int_X\left(\lim_{i\in\N}\int_Y f_i(x,y)\de\nu(y)\right)\de\mu(x)
		=\int_X\int_Y\lim_{i\in\N} f_i(x,y)\de\nu(y)\de\mu(x)\\
		&=\int_X\int_Y f(x,y)\de\nu(y)\de\mu(x)
	\end{align*}
	e questo dimostra $f\in\E$ concludendo la dimostrazione.
\end{proof}

\begin{theorem}[Fubini]\label{thm:Fubini}
	Dato $(X\times Y,\A\otimes\B,\mu\nu)$ uno spazio prodotto, per ogni funzione $f:X\times Y\to\Rbar$ integrabile vale
	\begin{equation*}
		\int_{X\times Y}f(u)\de\mu\nu(u)=\int_X\int_Y f(x,y)\de\nu(y)\de\mu(x)=\int_Y\int_X f(x,y)\de\mu(x)\de\nu(y)\virgola
	\end{equation*}
	dove le funzioni $f(x,\cdot):Y\to\Rpiu$ e $f(\cdot,y):X\to\Rpiu$ risultano integrabili solo quasi ovunque e perciò anche gli integrali $\int_Y f(x,y)\de\nu(y)$ e $\int_X f(x,y)\de\mu(x)$ risultano definiti solo quasi ovunque, ma questo non cambia il valore dell'integrale esterno in virtù del \cref{cor:IntegraleAMenoDiTrascurabili}.
\end{theorem}
\begin{proof}
	Ancora una volta dimostriamo solo la prima uguaglianza, visto che la seconda segue poi per simmetria.
	
	Decomponiamo, come nella \cref{def:IntegraleIntegrabili}, la funzione $f$ nelle funzioni positive $f^+,f^-:X\times Y\to\Rbar$.
	Le funzioni $f^+,f^-$ sono integrabili a loro volta, come osservato nella \cref{nota:FpiuFmenoIntegrabili}, e perciò per il \cref{cor:SupportoIntegrabile} hanno supporto \sigfin[o].
	
	Allora da qui si conclude banalmente sfruttando la linearità dell'operatore di integrale e il \cref{thm:Tonelli}:
	\begin{align*}
		\int_{X\times Y}f(u)\de\mu\nu(u)
		& =\int_{X\times Y}f^+(u)\de\mu\nu(u)-\int_{X\times Y}f^-(u)\de\mu\nu(u)\\
		&=\int_X\int_Y f^+(x,y)\de\nu(y)\de\mu(x)-\int_X\int_Y f^-(x,y)\de\nu(y)\de\mu(x)\\
		&=\int_X\left(\int_Y f^+(x,y)\de\nu(y)-\int_Y f^-(x,y)\de\nu(y)\right)\de\mu(x)\\
		&=\int_X\int_Y f^+(x,y)-f^-(x,y)\de\nu(y)\de\mu(x)
		=\int_X\int_Y f(x,y)\de\nu(y)\de\mu(x)\virgola
	\end{align*}
	dove abbiamo implicitamente sfruttato che, come osservato nella \cref{prop:IntegraleIntegrabili}, il fatto che la somma di funzioni integrabili sia definita solo quasi ovunque non influisce sul valore dell'integrale doppio.
\end{proof}

Ora che abbiamo terminato la trattazione nel caso in cui i misurabili siano $\A\otimes\B$, tratteremo il caso in cui entrambe le misure iniziali sono complete e $\A\otimes\B$ piuttosto che essere la \sigalg{} generata dai rettangoli è il suo completamento rispetto alla misura prodotto.
Ridimostreremo perciò gli ultimi tre risultati soffermandoci unicamente sulle differenze negli enunciati e nelle dimostrazioni nel caso di misure complete.

\emph{D'ora in poi assumiamo implicitamente che $(X,\A,\mu)$ e $(Y,\B,\nu)$ siano spazi di misura completi.}

\begin{definition}\label{def:MisuraProdottoCompleta}
	Indichiamo con $(X\times Y,\overline{\A\otimes\B},\mu\nu)$\footnote{Compiamo ora un leggerissimo abuso di notazione, visto che $\mu\nu$ è già la misura su $\A\otimes\B$ piuttosto che sul suo completamento.} il completamento dello spazio di misura $(X\times Y,\A\otimes\B,\mu\nu)$.
\end{definition}

\begin{proposition}\label{prop:PreTonelliCompleto}
	Dato lo spazio di misura prodotto $(X\times Y,\overline{\A\otimes\B},\mu\nu)$, per ogni insieme misurabile $E\in\overline{\A\otimes\B}$ \sigfin[o] vale la seguente espressione per la misura:
	\begin{equation*}
		\mu\nu(E)=\int_X\nu\left(J_x^X(E)\right)\de\mu(x)=\int_Y\mu\left(J_y^Y(E)\right)\de\nu(y)\virgola
	\end{equation*}
	dove le funzioni integrande $\nu\left(J_x^X(E)\right)$ e $\mu\left(J_y^Y(E)\right)$ sono definite solo quasi ovunque, ma questo non cambia il valore dell'integrale in virtù del \cref{cor:IntegraleAMenoDiTrascurabili}.
\end{proposition}
\begin{proof}
	La dimostrazione procede pedissequamente a quella della \cref{prop:PreTonelli}, tranne che quando si afferma che grazie alla \cref{prop:TrascurabiliProdotto} gli insiemi trascurabili risultano avere quasi tutte le sezioni misurabili di misura nulla. Infatti nel dimostrarlo si sfruttava la \cref{prop:SezioniMisurabili}, che in questo caso non si può più applicare.
	Ridimostriamo perciò il fatto citato in questo caso.
	
	Sia $N\in\overline{\A\otimes\B}$ trascurabile. Allora per la \cref{prop:TrascurabiliProdotto}, per $\mu$-quasi ogni $x\in X$ l'insieme $J_x^X(N)$ risulta trascurabile. Ma allora per la completezza della misura $\mu$,  si ottiene anche che quasi ovunque $J_x^X(N)$ è misurabile di misura nulla e questo è proprio quanto serve.
	
	È importante notare che è proprio in questo passaggio che si sfrutta la completezza delle misure ed è sempre qui che diventa necessaria la precisazione nell'enunciato dell'esistenza solo quasi ovunque delle funzioni integrande.
\end{proof}

\begin{theorem}[Tonelli]\label{thm:TonelliCompleto}
	Dato $(X\times Y,\overline{\A\otimes\B},\mu\nu)$ uno spazio prodotto, per ogni funzione $f:X\times Y\to\Rbar$ misurabile positiva a supporto \sigfin[o] vale
	\begin{equation*}
		\int_{X\times Y}f(u)\de\mu\nu(u)=\int_X\int_Y f(x,y)\de\nu(y)\de\mu(x)=\int_Y\int_X f(x,y)\de\mu(x)\de\nu(y)\virgola
	\end{equation*}
	dove le funzioni $f(x,\cdot):Y\to\Rpiu$ e $f(\cdot,y):X\to\Rpiu$ risultano misurabili solo quasi ovunque e perciò anche gli integrali $\int_Y f(x,y)\de\nu(y)$ e $\int_X f(x,y)\de\mu(x)$ risultano definiti solo quasi ovunque, ma questo non cambia il valore dell'integrale esterno in virtù del \cref{cor:IntegraleAMenoDiTrascurabili}.
\end{theorem}
\begin{proof}
	Anche in questo caso la dimostrazione ricopia quella del \cref{thm:Tonelli}, con la lieve differenza che, al posto di sfruttare la \cref{prop:PreTonelli}, sfrutta la \cref{prop:PreTonelliCompleto} che ha ovviamente enunciato analogo solo che tratta il caso di misure complete. 
	Proprio per il fatto che sfrutta la \cref{prop:PreTonelliCompleto}, che assicura la definizione delle funzioni solo quasi ovunque, nell'enunciato si deve specificare che le funzioni e i relativi integrali sono definiti solo quasi ovunque.
\end{proof}

\begin{theorem}[Fubini]\label{thm:FubiniCompleto}
	Dato $(X\times Y,\overline{\A\otimes\B},\mu\nu)$ uno spazio prodotto, per ogni funzione $f:X\times Y\to\Rbar$ integrabile vale
	\begin{equation*}
		\int_{X\times Y}f(u)\de\mu\nu(u)=\int_X\int_Y f(x,y)\de\nu(y)\de\mu(x)=\int_Y\int_X f(x,y)\de\mu(x)\de\nu(y)\virgola
	\end{equation*}
	dove le funzioni $f(x,\cdot):Y\to\Rpiu$ e $f(\cdot,y):X\to\Rpiu$ risultano misurabili solo quasi ovunque e perciò anche gli integrali $\int_Y f(x,y)\de\nu(y)$ e $\int_X f(x,y)\de\mu(x)$ risultano definiti solo quasi ovunque, ma questo non cambia il valore dell'integrale esterno in virtù del \cref{cor:IntegraleAMenoDiTrascurabili}.
\end{theorem}
\begin{proof}
	Si dimostra identicamente a come si è dimostrato il \cref{thm:Fubini}, solo che, al posto di applicare il \cref{thm:Tonelli}, si applica il \cref{thm:TonelliCompleto}.
\end{proof}

Per concludere la sezione, lasciamo per esercizio al lettore di verificare che le seguenti situazioni, ovviamente carenti delle ipotesi necessarie, falsificano le ipotesi dei teoremi di Fubini e Tonelli.

\begin{exercise}\label{ex:ControesempiFubini}
	Chiamando $(X,\A,\mu)$ e $(Y,\B,\nu)$ due spazi di misura e $f:X\times Y\to \Rbar$ una funzione dal prodotto degli spazi, mostrare che nei seguenti casi non vale almeno una parte degli enunciati di Fubini e Tonelli.
	\begin{enumerate}
		\item Se $X=Y=\R$, $\A=\B=\mathcal P(\R)$, $\mu$ è la misura contapunti (cioè la misura che restituisce il numero di elementi se l'insieme è finito ed $+\infty$ altrimenti), $\nu$ la misura che restituisce $0$ se l'insieme è finito e $+\infty$ se è infinito ed infine $f$ è la funzione indicatrice della diagonale $\{(x,y)\in X\times Y: x=y\}$.
		\item Se $X=Y=\R$, $\A,\B$ coincidono con i Boreliani di $\R$, $\mu=\nu=m_1$ e la funzione $f$ è la funzione indicatrice dell'insieme $\cc01\times V$ dove $V$ è un insieme trascurabile non misurabile per Lebesgue.
		\item Se $X=Y=\R$, $\A=\B=\M_1$, $\mu=\nu=m_1$ e la funzione $f$ è definita come $f(x,y)=\frac{x^2-y^2}{(x^2+y^2)^2}\cdot\chi_{\cc01\times\cc01}(x,y)$.
	\end{enumerate}
\end{exercise}

\section{L'integrazione negli spazi euclidei}
Approfondiamo ora la teoria dell'integrazione nell'ambientazione più classica: lo spazio $\R^n$ munito della misura di Lebesgue.

Mostreremo che l'integrale di Lebesgue non è che una generalizzazione dell'integrale di Riemann e che la misura prodotto definibile su $\R^n$ a partire dalla misura sulla retta reale coincide proprio con la misura $m_n$. Questi due risultati forniranno, a meno di applicare i teoremi di Fubini e Tonelli, i metodi primari per calcolare gli integrali in più dimensioni.

La seconda parte della sezione sarà completamente votata alla dimostrazione della formula del cambio di variabile, che necessita di un gran numero di lemmi e di un teorema, il teorema di Radon-Nikodym, che lasceremo indimostrato.

\begin{proposition}\label{prop:MisuraProdottoEuclidea}
	La \sigalg{} prodotto completata $\overline{\M_n\otimes\M_m}$ dei misurabili di $\R^n$ e $\R^m$ coincide con la \sigalg{} dei misurabili di $\R^{n+m}$.
\end{proposition}
\begin{proof}
	Lavoriamo sulla \sigalg{} prodotto \emph{non completata} $\M_n\otimes\M_m$, per poi ottenere l'enunciato ricordando che il completamento dei Boreliani sono i misurabili.

	Ricordando la \cref{def:LebesgueSemiaperti}, poiché valgono ovviamente i contenimenti $S_n\subseteq \M_n$ e $S_m\subseteq \M_m$, è facile ricavarne che $S_{n+m}\subseteq \M_n\otimes \M_m$.
	Ma allora, applicando la \cref{prop:SigAlgUgualeBoreliani} e ricordando che $\M_n\otimes \M_m$ è una \sigalg{}, otteniamo che i Boreliani sono contenuti in $\M_n\otimes\M_m$.
	
	Sia $E_1\times E_2\in \M_n\times\M_m$. Per il \cref{thm:LebesgueEquivalenzeMisurabilita}, esistono $A_1,A_2$ Boreliani e $N_1,N_2$ trascurabili, rispettivamente in $\R^n$ e $\R^m$, tali che $E_1=A_1\sqcup N_1$ e $E_2=A_2\sqcup N_2$.
	Perciò, grazie alla distributività del prodotto insiemistico rispetto all'unione disgiunta, otteniamo
	\begin{equation*}
		E_1\times E_2=(A_1\sqcup N_1)\times(A_2\sqcup N_2)=\left(A_1\times A_2\right)\sqcup\left(A_1\times N_2\sqcup A_2\times N_1 \sqcup N_1\times N_2\right) \punto
	\end{equation*}
	Però l'insieme $A_1\times N_2\sqcup A_2\times N_1 \sqcup N_1\times N_2$ è trascurabile per la \cref{prop:TrascurabilePerInsiemeTrascurabile} e l'insieme $A_1\times A_2$ è un Boreliano in quanto prodotto di Boreliani\footnote{Lasciamo al lettore la dimostrazione che il prodotto di Boreliani è un Boreliano.}, allora applicando ancora il \cref{thm:LebesgueEquivalenzeMisurabilita} si ottiene $E_1\times E_2\in\M_{n+m}$.
	Vista la generalità della scelta di $E_1\times E_2$, quanto appena mostrato implica che $\M_n\times\M_m\subseteq \M_{n+m}$ e visto che $\M_{n+m}$ è una \sigalg{} se ne ricava che $\M_n\otimes\M_m\subseteq \M_{n+m}$.
	
	Per quanto appena detto e poiché $\M_n\otimes\M_m$ contiene i Boreliani, ricordando la \cref{prop:CompletamentoBoreliani}, è ovvio concluderne che il suo completamento\footnote{È fondamentale notare che la misura indotta dal prodotto e la misura di Lebesgue coincidono in virtù della \cref{prop:UnicitaCaratheodory} e di conseguenza anche l'operatore di completamento di una \sigalg{} è coincidente.} $\overline{\M_n\otimes\M_m}$ coincide con l'insieme dei misurabili.
\end{proof}


\begin{proposition}\label{prop:IntegraleRiemannCoincide}
	Data una funzione $f:\cc ab\to \mathbb R$, se essa è integrabile secondo Riemann ed integrabile secondo Lebesgue allora i due integrali coincidono.
\end{proposition}
\begin{proof}
	Nel caso in cui la funzione sia integrabile secondo Riemann (e quindi anche limitata, dando senso agli $\inf,\sup$ che compariranno nelle formule), il valore dell'integrale secondo Riemann corrisponde all'estremo superiore delle somme inferiori alla Riemann\footnote{Indicheremo con $\int_a^b$ l'integrale di Riemann, e con $\int_{\cc ab}$ l'integrale di Lebesgue.}
	\begin{equation*}
		\int_a^b f(x)\de x=\sup\left\{\sum_{i=0}^{k-1} (x_{i+1}-x_i)\left(\inf_{t\in\co{x_i}{x_{i+1}}}f(t)\right):\ a=x_0<x_1<\cdots<x_{k-1}<x_k=b\right\} \virgola
	\end{equation*}
	ed anche all'estremo inferiore delle somme superiori alla Riemann
	\begin{equation*}
		\int_a^b f(x)\de x=\inf\left\{\sum_{i=0}^{k-1} (x_{i+1}-x_i)\left(\sup_{t\in\co{x_i}{x_{i+1}}}f(t)\right):\ a=x_0<x_1<\cdots<x_{k-1}<x_k=b\right\} \punto
	\end{equation*}
	
	Data una partizione $a=x_0<x_1<\cdots<x_{k-1}<x_k=b$ dell'intervallo $\cc ab$, definiamo le funzioni
	\begin{align*}
		f^-_{x_0,x_1,\dots,x_k}(x)=\sum_{i=0}^{k-1}\chi_{\co{x_i}{x_{i+1}}}(x)\cdot\inf_{t\in\co{x_i}{x_{i+1}}}f(t) \virgola \\
		f^+_{x_0,x_1,\dots,x_k}(x)=\sum_{i=0}^{k-1}\chi_{\co{x_i}{x_{i+1}}}(x)\cdot\sup_{t\in\co{x_i}{x_{i+1}}}f(t) \punto
	\end{align*}

	Con le definizioni mostrate è facile accorgersi che l'integrale di Riemann allora è l'estremo superiore dell'integrale di Lebesgue delle funzioni semplici $f^-_{x_0,\dots,x_k}$ su tutte le partizioni e analogamente l'estremo inferiore dell'integrale di Lebesgue delle funzioni semplici $f^+_{x_0,\dots,x_k}$ ancora su tutte le partizioni.
	Allora, per la monotonia dell'integrale di Lebesgue è facile accorgersi che
	\begin{multline*}
		\int_{\cc ab}f(x)\de x\le \\
		\inf_{a=x_0<\dots<x_k=b}\left\{\int_{\cc ab}f^+_{x_0,\dots,x_k}(x)\de x\right\}
		=\int_a^b f(x)\de x= 
		\sup_{a=x_0<\dots<x_k=b}\left\{\int_{\cc ab}f^-_{x_0,\dots,x_k}(x)\de x\right\}\\
		 \le \int_{\cc ab} f(x)\de x
	\end{multline*}
	e ciò equivale ovviamente alla tesi.
\end{proof}

Dimostriamo ora un caso particolare, ma molto significativo, della formula del cambio di variabile. 
Proviamo la formula per il cambio di variabile lineare che verrà poi sfruttata, attraverso una sorta di passaggio al limite, per dimostrare la formula più generale.

In particolare, e tale procedimento verrà ripetuto varie volte in questa sezione, prima otterremo un risultato sulla misura di un insieme trasformato da un'applicazione lineare per poi ricavarne facilmente la formula per il cambio di variabile attraverso l'applicazione della \cref{prop:IdentitaMisuraImplicaCambioVariabile}. L'idea che sta alla base di questo ragionamento è che l'identità riguardante le misure è in realtà un cambio di variabile riguardante le funzioni indicatrici.

\begin{proposition}\label{prop:IdentitaMisuraImplicaCambioVariabile}
	Fissato $A\in\M_n$, siano $\varphi:A\to\R^n$ una funzione che manda misurabili in misurabili e $\rho:A\to\Rpiu$ una funzione misurabile positiva tali che per ogni $E\subseteq A$ misurabile valga
	\begin{equation*}
		m_n\left(\varphi(E)\right)=\int_E\rho(x)\de x \punto
	\end{equation*}
	Allora per ogni funzione integrabile $f:\R^n\to\Rbar$ ed $E\subseteq A$ misurabile sono definiti i seguenti integrali e vale l'identità
	\begin{equation*}
		\int_{\varphi(E)}f(y)\de m_n(y)=\int_E f(\varphi(x))\rho(x)\de m_n(x)\punto
	\end{equation*}
\end{proposition}
\begin{proof}
	Fissato $E\subseteq A$ misurabile, sia $\mathcal F\subseteq \L(\R^n,\M_n,m_n)$ l'insieme delle funzioni che rispettano l'enunciato con tale $E$.
	
	Assumendo che $f$ sia l'indicatrice di un insieme misurabile finito $I\subseteq A$, applicando l'ipotesi sulla relazione tra $\varphi$ e $\rho$, otteniamo
	\begin{multline*}
		\int_{\varphi(E)} f(y)\de m_n(y)=\int_{\varphi(E)} \chi_I(y)\de m_n(y)=m_n(\varphi(E)\cap I)=m_n\left(\varphi(E\cap \varphi^{-1}(I))\right) \\
		=\int_{\varphi^{-1}(I)\cap E}\rho(x)\de m_n(x)=\int_E \chi_{\varphi^{-1}(I)}(x)\rho(x)\de m_n(x)=\int_E f(\varphi(x))\rho(x)\de m_n(x) \virgola
	\end{multline*}
	cioè $f$ appartiene a $\mathcal F$ e, vista la scelta arbitraria di $f$, tutte le funzioni indicatrici di insiemi finiti appartengono a $\mathcal F$.
	
	Per la linearità dell'operatore integrale è chiaro che l'insieme delle funzioni che rispettano è uno spazio vettoriale.
	
	Sia ora $(f_n)_{n\in\N}\subseteq \mathcal F$ una successione di funzioni integrabili positive che converge crescentemente alla funzione integrabile $f$.
	
	Dall'ipotesi $f_n\in\mathcal F$ abbiamo che
	\begin{equation}\label{eq:IMICV_UguaglianzaFinita}
		\int_{\varphi(E)} f_n(y)\de y = \int_E f_n(\varphi(x))\rho(x)\de x\punto
	\end{equation}
	È evidente però che sia le funzioni $f_n$ che le funzioni $f_n\circ\varphi\cdot\rho$ sono funzioni misurabili positive che convergono in modo crescente. Allora passando al limite l'\cref{eq:IMICV_UguaglianzaFinita}, e applicando il \cref{thm:BeppoLevi}, ricaviamo
	\begin{equation*}
		\int_{\varphi(E)} f(y) \de y= \int_{E} f(\varphi(x))\rho(x)\de x \virgola
	\end{equation*}
	cioè $f\in\mathcal F$ e quindi, per l'arbitrarietà della successione $f_n$ scelta, abbiamo dimostrato che $\mathcal F$ è chiusa per convergenza crescente di funzioni integrabili positive (ammesso di rimanere nelle integrabili).
	
	Unendo quanto detto, abbiamo tutte le ipotesi per applicare il \cref{thm:ChiusuraMonotonaFunzioni}\footnote{Bisogna notare che qui sfruttiamo una versione leggermente diversa dell'enunciato, la cui dimostrazione è analoga. Infatti noi ci interessiamo ad ottenere che $\mathcal F$ coincida con le funzioni integrabili, non con tutte le misurabili. 
	Perciò l'enunciato del teorema utilizzato andrebbe modificato aggiungendo il fatto che la convergenza monotona sia ristretta nelle integrabili.} e ottenerne quindi che $\mathcal F=\L(\R^n,\M_n,m_n)$, cioè la tesi.
\end{proof}

\begin{proposition}\label{prop:MisuraImmagineLineare}
	Fissata un'applicazione lineare $L:\R^n\to\R^n$, per ogni $E\in\M_n$ misurabile secondo Lebesgue risulta che $L(E)$ è misurabile e rispetta
	\begin{equation}
		m_n(L(E))=\lvert \det L\rvert\cdot m_n(E)\punto
	\end{equation}
\end{proposition}
\begin{proof}
	La funzione $L$ essendo lineare è in particolare lipschitziana (con costante la propria norma) quindi, applicando la \cref{prop:LipschitzTengonoMisurabili}, otteniamo che per ogni $E$ misurabile, $L(E)$ è anch'esso misurabile.
	
	Consideriamo ora la funzione $\mu_L:\M_n\to\Rpiu$ definita come $\mu_L(E)=m_n(L(E))$. Per quanto appena mostrato questa è una buona definizione.
	Per la \cref{prop:BigettivaInduceMisura} la $\mu_L$ è una misura.
	Inoltre, ricordando che la misura di Lebesgue è invariante per traslazione come mostrato nella \cref{nota:LebesgueProprieta}, si ottiene
	\begin{equation*}
		\forall v\in\R^n:\ \mu_L(E+v)=m_n(L(E+v))=m_n(L(E)+Lv)=m_n(L(E))=\mu_L(E)
	\end{equation*}
	che equivale a dire che $\mu_L$ è invariante per traslazione.
	\newcommand{\linR}{\ensuremath{\mathcal L(\R^n,\R^n)}}
	Allora possiamo applicare il \cref{thm:LebesgueUnicaInvarianteTraslazione}\footnote{Il teorema ci assicura l'identità delle misure solo sui Boreliani, ma i misurabili sono il completamento dei Boreliani e perciò le due misure coincidono anche sui misurabili.} ottenendo che esiste una funzione $c:\linR\to\Rpiu$ con dominio le applicazioni lineari, tale che valga
	\begin{equation}\label{eq:DefQuasiDet}
		\forall L\in \linR,\ E\in\M_n:\ \mu_L(E)=c(L)m_n(E)\punto
	\end{equation}
	
	Mostriamo ora che $c(\cdot)$ è moltiplicativa e che coindice con il valore assoluto del determinante sulle applicazioni diagonali e ortogonali. Da questo, sfruttando una decomposizione nota delle applicazioni lineari di $\R^n$, seguirà che coincide con il valore assoluto del determinante su ogni applicazione lineare e questo è proprio quanto richiesto dalla tesi. 
	
	Fissate $A,B\in \linR$ applicazioni lineari ed $E\in\M_n$ un misurabile non trascurabile, applicando unicamente l'\cref{eq:DefQuasiDet} risulta
	\begin{align*}
		c(AB)m_n(E)&=\mu_{AB}(E)=m_n(AB(E))=m_n(A(B(E))\\
		&=\mu_A(B(E))=c(A)m_n(B(E))=c(A)\mu_B(E)=c(A)c(B)m_n(E)\virgola
	\end{align*}
	da cui si ricava $c(AB)=c(A)c(B)$ dividendo per $m_n(E)$. Perciò $c$ è moltiplicativa.
	
	Sia $D\in\linR$ un'applicazione diagonale, in particolare siano $(\lambda_i)_{1\le i\le n}$ i valori sulla diagonale.
	Allora è facile ricavare
	\begin{equation*}
		D\left(\co01\times\co01\times\dots\times\co01\right)=\co0{\lambda_1}\times\co0{\lambda_2}\times\cdots\times\co0{\lambda_n}\virgola
	\end{equation*}
	da cui, applicando ad entrambi i membri la misura di Lebesgue, si ottiene
	\begin{align*}
		c(D)&=c(D)m_n\left(\co01\times\dots\times\co01\right)=\mu_D\left(\co01\times\dots\times\co01\right)\\
		&=m_n\left(\co0{\lambda_1}\times\cdots\times\co0{\lambda_n}\right)=
		\lvert\lambda_1\rvert\cdot\lvert\lambda_2\rvert\cdots\lvert\lambda_n\rvert=\lvert\det(D)\rvert
	\end{align*}
	che equivale a dire che $c(\cdot)$ e $\lvert\det(\cdot)\rvert$ coincidono sulle matrici diagonali.
	
	Fissata un'applicazione $O\in\linR$ ortogonale, chiamando $P$ la palla unitaria di $\R^n$ è ovvio che $O(P)=P$.
	Da questo, applicando ad entrambi i membri la misura di Lebesgue, si ricava $\mu_O(P)=m_n(P)$ e, ricordando che $P$ non è trascurabile, ne discende $c(O)=1$.
	Ma $O$ è ortogonale, quindi $\det O=\pm 1$ e allora è dimostrato anche in questo caso $\lvert\det O\rvert =c(O)$.
	
	Infine, data un'applicazione lineare generica $L\in\linR$, sia $L=OS$ la sua decomposizione polare\footnote{La si ottiene notando che $LL^t$ è una matrice simmetrica semidefinita positiva che perciò ammette una ``radice quadrata''.} dove $O$ è ortogonale e $S$ è simmetrica. Per il teorema spettrale esistono $P,D$ rispettivamente invertibile e diagonale tali che $S=PDP^{-1}$.
	Ricordando le proprietà che rispetta la funzione $c$ e la moltiplicatività del determinante, ricaviamo
	\begin{align*}
		c(L)&=c(OPDP^{-1})=c(O)c(P)c(D)c(P^{-1})=c(O)c(D)c(P)c(P^{-1})\\
		&=c(O)c(D)c(PP^{-1})=\lvert\det O\rvert\cdot\lvert\det D\rvert=\lvert \det(OPDP^{-1})\rvert=\lvert\det L\rvert \virgola
	\end{align*}
	che è equivale a dire che $c(\cdot)$ e $\lvert\det(\cdot)\rvert$ coincidono come si voleva.
\end{proof}

\begin{corollary}\label{cor:CambioVariabileLineare}
	Fissata un'applicazione lineare $L:\R^n\to\R^n$, per ogni $E\in\M_n$ misurabile ed $f:\R^n\to\Rbar$ integrabile vale la formula per il cambio di variabile lineare
	\begin{equation*}
		\int_{L(E)}f(y)\de m_n(y) = \int_E f(L(x))\left\lvert\det L\right\rvert \de m_n(x) \punto
	\end{equation*}
\end{corollary}
\begin{proof}
	È un'ovvia conseguenza della \cref{prop:IdentitaMisuraImplicaCambioVariabile}, visto che le ipotesi di quest'ultima sono verificate dalla \cref{prop:MisuraImmagineLineare}.
\end{proof}

I fatti seguenti sono tutti mirati a fornire basi solide ai vari procedimenti di limite necessari per ottenere la formula per il cambio di variabile.
In particolare la continuità $L^1$, così è nota in letteratura la \cref{prop:ContinuitaL1}, oltre ad essere il fondamento di tutti gli altri lemmi è importante anche al di là di questa dimostrazione. È infatti un fatto assolutamente non ovvio che rende possibile ottenere tutti i risultati riguardo la convoluzione integrale.

Bisogna porre un accento sul fatto che la dimostrazione della continuità $L^1$ ricalca quella che vuole essere la struttura di tutte le dimostrazioni fondazionali della teoria della misura: si inizia dimostrando l'enunciato per insiemi nel \semiring{} per poi concludere che vale per ogni misurabile attraverso l'applicazione di chiusure monotone.



\begin{lemma}\label{lemma:ContinuitaL1Semianello}
	Dato $T=\co{a_1}{b_1}\times\cdots\times\co{a_n}{b_n}\in\S_n$ e $c=(c_1,\dots,c_n)\in\R^n$, vale la stima
	\begin{equation*}
		\LNorm{\chi_T({}\cdot{}+c)-\chi_T({}\cdot{})}\le 2m_n(T)\sum_{i=1}^n \frac{\lvert c_i\rvert }{b_i-a_i} \punto
	\end{equation*}
\end{lemma}
\begin{proof}
	Definiamo la funzione misurabile $s_c:\R^n\to\R$ come $s_c(x)=\left\lvert\chi_T(x+c)-\chi_T(x)\right\rvert$.
	
	Sfruttando delle identità insiemistiche e la definizione delle funzioni caratteristiche, è facile verificare che
	\begin{equation*}
		s_c(x)=\chi_{T\setminus\left(T-c\right)}(x)+\chi_{\left(T-c\right)\setminus T}(x)\punto
	\end{equation*}
	Inoltre i due insiemi $T\setminus\left(T-c\right)$ e $\left(T-c\right)\setminus T$ hanno la stessa misura in virtù della \cref{prop:LebesgueProprietaIsometria}, in quanto si può ottenere l'uno dall'altro, a meno del bordo che è però trascurabile, con un'isometria.
	Unendo quanto detto e ricordando che l'integrale di una caratteristica è la misura dell'insieme, è evidente la validità della formula
	\begin{equation} \label{eq:IdentitaIntegraleContinuitaL1}
		\LNorm{s_c}=2m_n\left(T\setminus\left(T-c\right)\right)\punto
	\end{equation}
	
	Vogliamo quindi stimare la misura di $T\setminus\left(T-c\right)$. 
	Se $x=(x_1,\dots,x_n)\in\R^n$ appartiene a $T\setminus\left(T-c\right)$ allora $x\in T$ e $x\not\in T-c$. Ma tali appartenenze, passando in coordinate, divengono il seguente sistema:
	\begin{equation}\label{eq:SistemaSemianelloContinuitaL1}
		\begin{cases}
			\forall i\in\{1,\dots,n\}: &a_i\le x_i<b_i \virgola\\
			\exists j\in\{1,\dots,n\}: &x_j<a_j-c_j \vee b_j-c_j\le x_j \punto
		\end{cases}
	\end{equation}
	Definiamo quindi $T_j$, con $j\in\{1,\dots,n\}$, come l'insieme dei punti che rispettano il sistema dove la proprietà nella seconda riga è rispettata da $j$.
	Allora, per la subadditività della misura di Lebesgue, vale
	\begin{equation}\label{eq:StimaInsiemeConPezzettiContinuitaL1}
		T\setminus\left(T-c\right)\subseteq \bigcup_{j=1}^n T_j \implies m_n\left(T\setminus\left(T-c\right)\right)\le \sum_{j=1}^n m_n(T_j) \punto
	\end{equation}
	Preso $x\in T_j$ se $c_j\ge 0$, come conseguenza dell'\cref{eq:SistemaSemianelloContinuitaL1}, vale $b_j-c_j\le x_j<b_j$; mentre se $c_j<0$ risulta $a_j\le x_j<a_j-c_j$. Sia allora $I$ l'intervallo $\co{b_j-c_j}{b_j}$, nel caso in cui $c_j\ge 0$, e $\co{a_j}{a_j-c_j}$ altrimenti. In entrambi i casi la lunghezza di $I$ risulta essere $\lvert c_j \rvert$.
	Per quanto detto, ricordando ancora l'\cref{eq:SistemaSemianelloContinuitaL1} è facile accorgersi che vale
	\begin{equation*}
		T_j\subseteq \co{a_1}{b_1}\times\cdots\times\co{a_{j-1}}{b_{j-1}}\times I\times\co{a_{j+1}}{b_{j+1}}\times\cdots\times \co{a_n}{b_n}
	\end{equation*}
	e da questa, per la definizione della misura di Lebesgue su $\S_n$ è ovvio ricavarne
	\begin{equation}\label{eq:StimaFinalePezzettoContinuitaL1}
		m_n(T_j)\le \frac{\lvert c_j\rvert }{b_j-a_j} \prod_{i=1}^n (b_i-a_i)=\frac{\lvert c_j\rvert }{b_j-a_j} m_n(T) \punto
	\end{equation}
	
	Unendo le \cref{eq:IdentitaIntegraleContinuitaL1,eq:StimaInsiemeConPezzettiContinuitaL1,eq:StimaFinalePezzettoContinuitaL1} arriviamo finalmente ad avere
	\begin{equation*}
		\LNorm{s_c}\le 2m_n(T)\sum_{i=1}^n \frac{\lvert c_i\rvert }{b_i-a_i}\virgola
	\end{equation*}
	che implica che la tesi per definizione di $s_c$.
\end{proof}


\begin{proposition}[Continuità $L^1$]\label{prop:ContinuitaL1}
	Data una funzione $f:\R^n\to\Rbar$ integrabile nella misura di Lebesgue, le funzioni $f({}\cdot{}+c)$ convergono, per $c$ che tende a $0$, in norma $L^1$ ad $f$.
\end{proposition}
\begin{proof}
	Sia $\mathcal F\subseteq \L(\R^n,\M_n,m_n)$ l'insieme delle funzioni integrabili per cui l'enunciato è vero.
	Chiamiamo poi, per $c\in\R^n$, $\tau_c:\R^n\to\R^n$ la traslazione $\tau_c(x)=x+c$.
	
	Dimostreremo che le funzioni indicatrici di $\S_n$ appartengono ad $\mathcal F$, poi che $\mathcal F$ è uno spazio vettoriale chiuso per convergenza puntuale monotona ed infine che se due funzioni coincidono quasi ovunque ed una appartiene ad $\mathcal F$ allora anche l'altra vi appartiene. 
	L'unione di questi fatti sarà sufficiente a mostrare che $\mathcal F=\L(\R^n,\M_n,m_n)$ dimostrando così la tesi.
	
	Il \cref{lemma:ContinuitaL1Semianello} dimostra banalmente, considerando il limite per $c\to 0$, che le indicatrici degli insiemi in $\S_n$ appartengono ad $\mathcal F$.
	
	Il fatto che $\mathcal F$ sia uno spazio vettoriale discende direttamente dalla linearità dell'integrale e dal fatto che il limite di una somma è la somma dei limiti.
	
	Sia $(f_n)_{n\in\N}$ una successione di funzioni integrabili che appartengono ad $\mathcal F$ convergenti monotonamente alla funzione $f$, che assumiamo essere integrabile.
	Per il \cref{thm:ConvergenzaDominata}, le cui ipotesi sono facilmente verificate poiché le funzioni sono dominate dalla funzione integrabile $\max(f,f_1)$, le $f_n$ convergono ad $f$ anche in norma $L^1$ e perciò, per ogni $\epsilon>0$, esiste $m\in\N$ tale che $\LNorm{f_m-f}<\epsilon$. 
	Inoltre, una facile conseguenza dell'invarianza per traslazione della misura di Lebesgue è 
	\begin{equation*}
		\LNorm{f_m\circ\tau_c-f\circ\tau_c}=\LNorm{f_m-f} \virgola
	\end{equation*}
	da cui, poiché $\LNorm{{}\cdot{}}$ rispetta la disuguaglianza triangolare come mostrato nella \cref{prop:L1VettorialeConSeminorma}, ne otteniamo
	\begin{equation*}
		\LNorm{f\circ\tau_c-f}\le \LNorm{f_m\circ\tau_c-f\circ\tau_c}+\LNorm{f_m-f}+\LNorm{f_m\circ\tau_c-f_m} \le 2\epsilon+\LNorm{f_m\circ\tau_c-f_m}\virgola
	\end{equation*}
	che passando al massimo limite entrambi i membri e ricordando che $f_m\in\mathcal F$, implica la disuguaglianza
	\begin{equation*}
		\limsup_{c\to 0} \LNorm{f\circ\tau_c-f} \le 2\epsilon \punto
	\end{equation*}
	Ma quest'ultima disuguaglianza vale per ogni $\epsilon>0$ e ciò implica ovviamente che anche $f\in\mathcal F$.
	
	Date $f,g:\R^n\to\Rbar$ funzioni integrabili equivalenti in $L^1$, con $f\in\mathcal F$, dimostriamo che $g\in\mathcal F$. 
	Anche $f\circ\tau_c$ e $g\circ\tau_c$ sono equivalenti in $L^1$ e perciò anche $f\circ\tau_c-f$ coincide quasi ovunque con $g\circ\tau_c-g$.
	Unendo quanto detto si ottiene
	\begin{equation*}
		\LNorm{g\circ\tau_c-g}=\LNorm{f\circ\tau_c-f}\to 0 \virgola
	\end{equation*}
	che implica che anche $g$ appartiene a $\mathcal F$.
	
	Sia $\mathcal E$ la famiglia degli insiemi tali che le loro indicatrici appartengono ad $\mathcal F$. 
	
	Abbiamo mostrato che $\S_n\subseteq \mathcal E$. 
	Inoltre il fatto che $\mathcal F$ sia uno spazio vettoriale implica che $\mathcal E$ sia stabile per unione disgiunta e, essendo $\mathcal F$ chiuso per convergenza puntuale monotona nelle funzioni integrabili, $\mathcal E$ risulta stabile per unione e intersezione monotona numerabile negli insiemi finiti.
	Quindi risultano verificate le ipotesi del \cref{cor:ChiusuraMonotonaInsiemiFiniti} e ne deduciamo che $\mathcal E$ contiene ogni misurabile finito a meno di un trascurabile. Ma il fatto che se $\mathcal F$ contiene una funzione allora contiene ogni funzione che coincide quasi ovunque con essa implica che $\mathcal E$ se contiene un insieme allora contiene anche quelli che coincidono con lui a meno di un trascurabile.
	Quindi questo implica $\mathcal E$ contiene tutti gli insiemi misurabili finiti e perciò $\mathcal F$ contiene le loro indicatrici.
	
	Concludiamo quindi applicando il \cref{thm:ChiusuraMonotonaFunzioni}\footnote{Di nuovo sfruttiamo l'enunciato nella forma con insiemi finiti e funzioni integrabili.}, di cui abbiamo controllato essere verificate tutte le ipotesi, e otteniamo che $\mathcal F$ contiene tutte le funzioni integrabili.
\end{proof}

\begin{definition}[Media integrale] \label{def:MediaIntegrale}
	Definiamo l'operatore $\aint$, chiamato media integrale, come
	\begin{equation*}
		\aint_E f(x)\de\mu(x)=\frac 1{\mu(E)}\int_E f(x)\de\mu(x)\virgola
	\end{equation*}
	dove $E$ è un insieme misurabile nello spazio di misura $(X,\A,\mu)$ e $f:X\to\Rbar$ è una funzione integrabile nel medesimo spazio di misura.
\end{definition}
\begin{remark}\label{nota:ProprietaMediaIntegrale}
	Fissato uno spazio di misura $(X,\A,\mu)$ e un insieme misurabile $E\in\A$ l'operatore di media integrale rispetta le seguenti proprietà:
	\begin{itemize}
		\item Date $f,g:X\to\Rbar$ funzioni misurabili e $a,b\in\R$ vale
		\begin{equation*}
			a\aint_E f(x)\de\mu(x)+b\aint_E g(x)\de\mu(x)=\aint_E af(x)+bg(x)\de\mu(x)
		\end{equation*}
		e se $f\le g$ allora
		\begin{equation*}
			\aint_E f(x)\de\mu(x)\le \aint_E g(x)\de\mu(x)\virgola
		\end{equation*}
		cioè la media integrale è lineare e monotona.
		\item Fissato $c\in\R$ vale
		\begin{equation*}
			\aint_E c\de\mu(x)=c\punto
		\end{equation*}
		\item Per ogni $f:X\to\Rbar$ misurabile valgono le stime
		\begin{equation*}
			\inf_{x\in E}f(x)\le \aint_E f(x)\de\mu(x) \le \sup_{x\in E}f(x)\punto
		\end{equation*}
	\end{itemize}
\end{remark}
\begin{proof}
	La linearità, la monotonia e l'identità riguardante la media integrale delle costanti discendono dalla \cref{def:MediaIntegrale} applicando le prime proprietà dell'integrale di Lebesgue.
	
	Per quanto riguarda l'ultima stima, è sufficiente applicare le proprietà già dimostrate per ottenere
	\begin{equation*}
		\inf_{x\in E}f(x) = \aint_E \inf_{x\in E} f(x)\de\mu(x)\le \aint_E f(x)\de\mu(x) \le \aint_E \sup_{x\in E} f(x)\de\mu(x) =\sup_{x\in E}f(x)\punto
	\end{equation*}
\end{proof}


\begin{lemma}\label{lemma:ContinuitaL1Palle}
	Fissato $E\in\M_n$ un insieme misurabile e $f:E\to\Rbar$ una funzione integrabile, definiamo, per ogni $r>0$, la funzione $f_r:E\to\Rbar$ come
	\begin{equation*}
		\forall x\in E:\ f_r(x)=\frac{\int_{B_r(x)\cap E}f(t)\de t}{m_n\left(B_r(x)\right)}\virgola
	\end{equation*}
	dove $B_r(x)$ è la palla aperta di raggio $r$ e centro $x$.
	Allora le funzioni $f_r$, che risultano essere continue, convergono, per $r$ che tende a $0$, in norma $L^1$ alla funzione $f$.
\end{lemma}
\begin{proof}
	Innanzitutto, per comodità di notazione, allarghiamo il dominio della $f$ ponendo $f(E^{\mathsf c})=\{0\}$, così facendo otteniamo
	\begin{equation*}
		\forall r>0,\ x\in E:\ f_r(x)=\frac{\int_{B_r(x)}f(t)\de t}{m_n(B_r(x))}=\aint_{B_r(x)} f(t)\de t \punto
	\end{equation*}

	Vale facilmente, per ogni $x,y\in E$, l'identità
	\begin{equation*}
		\lvert f_r(x)-f_r(y)\rvert %= \left\lvert \aint_{B_r(x)}f(t)\de t-\aint_{B_r(y)} f(t)\de t\right\rvert 
		\le \aint_{B_r(x)}\lvert f(t)-f(t+(y-x))\rvert \de t \le \frac{ \LNorm{f({}\cdot{})-f({}\cdot{}+(y-x))} }{ m_n(B_r(0)) } \virgola
	\end{equation*}
	che grazie alla \cref{prop:ContinuitaL1} implica la continuità delle funzioni $f_r$.
	
	Ora verifichiamo, per poi poter applicare il teorema di Tonelli, che la funzione $g:\R^n\times\R^n\to\Rbar$ definita come $g(x,t)=f(x+t)-f(x)$ è misurabile nello spazio prodotto $\R^n\times\R^n$. 
	La funzione $u(x,t)=f(x)$ è misurabile nel prodotto grazie alla \cref{nota:FunzioniMisProdotto} e allora anche $u(x+t,t)$ lo è in quanto composizione di una misurabile con una lineare invertibile. Unendo quanto detto si ha, come cercato, che $g(x,t)=u(x+t,t)-u(x,t)$ è misurabile in quanto differenza di misurabili.
	
	Calcoliamo ora la norma $L^1$ della differenza tra $f$ e $f_r$:
	\begin{multline}\label{eq:ContinuitaPalleDis}
		\LNorm{f-f_r}=\int_E \left\lvert f(x)-\aint_{B_r(x)}f(t)\de t\right\rvert\de x\\
		=\int_E\left\lvert\aint_{B_r(x)}f(x)-f(t)\de t\right\rvert\de x\le \int_E\aint_{B_r(x)}\left\lvert f(x)-f(t)\right\rvert\de t\de x\\
		=\int_E\aint_{B_r(0)}\left\lvert f(x)-f(x+h)\right\rvert\de h\de x
		=\aint_{B_r(0)}\int_E \left\lvert f(x)-f(x+h)\right\rvert\de x\de h\\
		= \aint_{B_r(0)}\LNorm{f(\cdot)-f(\cdot+h)}\de h\le \sup_{h\in B_r(0)} \LNorm{f(\cdot)-f(\cdot+h)}\virgola
	\end{multline}
	dove nei vari passaggi abbiamo sfruttato il \cref{thm:TonelliCompleto}, che si può applicare ricordando la \cref{prop:MisuraProdottoEuclidea}, e la \cref{nota:ProprietaMediaIntegrale}.
	
	Per la \cref{prop:ContinuitaL1}, il valore $\sup_{h\in B_r(0)} \LNorm{f(\cdot)-f(\cdot+h)}$ tende a $0$ per $r\to 0$ e applicando ciò nell'\cref{eq:ContinuitaPalleDis} si ricava facilmente la tesi.
\end{proof}

Terminati i lemmi preliminari, mostriamo ora la formula per il cambio di variabile. La dimostrazione procede in tre passi. 
In un primo passo, cioè nel \cref{lemma:LimiteDeterminante}, proviamo solo una disuguaglianza, che poi mostreremo essere un'uguaglianza nel \cref{lemma:MisuraImmagine} e che sarà l'equivalente della \cref{prop:MisuraImmagineLineare} ma nel caso generale e non in quello lineare. Infine il terzo passo semplicente trasforma l'enunciato sulla misura di un insieme nella formula per il cambio di variabile.

Per passare da disuguaglianza a uguaglianza sfrutteremo come strumento fondamentale il teorema di Radon-Nikodym, che asserisce che ogni misura ammette una ``densità'', cioè una funzione tale che il suo integrale su un insieme, rispetto ad un'altra misura, coincide con la misura dell'insieme.
Tale teorema lo lasciamo indimostrato fondamentalmente per la non banalità della dimostrazione e poiché in queste dispense lo useremo solo come strumento tecnico, senza approfondire l'argomento.

\begin{lemma}\label{lemma:LimiteDeterminante}
	Fissato $\Omega\subseteq\R^n$ un aperto, sia $\varphi:\Omega\to\R^n$ una funzione differenziabile con continuità.
	Allora vale la seguente disuguaglianza:
	\begin{equation*}
		\forall x\in\Omega:\ \limsup_{r\to 0} \frac{ m_n\left(\varphi\left(B_r(x)\right)\right)} {m_n\left(B_r(x)\right)}\le \lvert\det \Diff \varphi(x)\rvert \virgola 
	\end{equation*}
	dove $B_r(x)$ è la palla aperta di raggio $r$ e centro $x$ e $\Diff \varphi$ è la matrice jacobiana della funzione $\varphi$.
\end{lemma}
\begin{proof}
	Fissiamo $\bar x\in\Omega$ e chiamiamo $A=\Diff  \varphi(\bar x)$.
	Allora, per definizione di differenziale, risulta vero che
	\begin{equation*}
		\varphi(\bar x+h)=\varphi(\bar x)+Ah+\smallO(h)=\varphi(\bar x)+A(h+\smallO(h))\in \varphi(\bar x)+A\left(B_{|h|+\smallO(h)}(0)\right) \virgola
	\end{equation*}
	dove abbiamo implicitamente sfruttato che moltiplicare per la norma degli operatori di $A$ non cambia il fatto che una funzione sia $\smallO$-piccolo di un'altra.
	
	Da quanto appena detto discende facilmente che, per ogni $r>0$, vale
	\begin{equation}\label{eq:ContenimentoPallaLineare}
		\varphi(B_r(\bar x))\subseteq \varphi(\bar x)+A\left(B_{r+\smallO(r)}(0)\right).
	\end{equation}
	Essendo però $\varphi$ differenziabile con continuità, è anche localmente lipschitziana e perciò, per il \cref{cor:LocLipschitzTengonoMisurabili}, l'insieme $\varphi(B_r(\bar x))$ è misurabile.
	Allora possiamo applicare la misura di Lebesgue ad entrambi i membri del contenimento mostrato nell'\cref{eq:ContenimentoPallaLineare} ottenendo
	\begin{equation}\label{eq:StimaImmaginePalla}
		m_n\left( \varphi(B_r(\bar x)) \right)\le m_n\left(\varphi(\bar x)+A\left(B_{r+\smallO(r)}(0)\right)\right)
		=\lvert \det A\rvert m_n\left(B_{r+\smallO(r)}(\bar x)\right)\virgola
	\end{equation}
	dove i passaggi sono giustificati dall'invarianza per traslazione della misura di Lebesgue e dalla \cref{prop:MisuraImmagineLineare}.
	Ma ora, grazie alla $n$-omogeneità della misura $m_n$, ricaviamo
	\begin{equation*} 
		m_n\left(B_{r+\smallO(r)}(\bar x)\right)=\left(1+\smallO(1)\right)^n m_n\left( B_r(\bar x)\right)
	\end{equation*}
	e perciò sostituendo questa nell'\cref{eq:StimaImmaginePalla} arriviamo a
	\begin{equation*}
		m_n\left( \varphi(B_r(\bar x)) \right)\le \left(1+\smallO(1)\right)^n m_n\left( B_r(\bar x)\right)\lvert \det A\rvert \virgola
	\end{equation*}
	che implica banalmente la tesi.
\end{proof}

\begin{definition}\label{def:AssolutamenteContinua}
	Date due misure $\mu:\A\to\Rpiu,\ \nu:\A\to\Rpiu$ sullo stesso spazio misurabile $(X,\A)$, diciamo che $\nu$ è assolutamente continua rispetto a $\mu$, indicandolo con $\nu\ll\mu$, se per ogni $E\in\A$ tale che $\mu(E)=0$ vale $\nu(E)=0$, cioè se tutti gli insiemi trascurabili per $\mu$ sono trascurabili anche per $\nu$.
\end{definition}

\begin{theorem}[Radon-Nikodym] \label{thm:RadonNikodym}
	Dato uno spazio di misura $(X,\A,\mu)$ \sigfin[o], se la misura $\nu:\A\to\Rpiu$ è assolutamente continua rispetto a $\mu$, allora esiste una funzione $\rho:X\to\Rpiu$ misurabile rispetto a $\mu$, tale che
	\begin{equation*}
		\forall A\in\A:\ \nu(A)=\int_A \rho(x)\de\mu(x)\punto
	\end{equation*}
\end{theorem}

\begin{lemma}\label{lemma:MisuraImmagine}
	Fissati $\Omega,\Omega'\subseteq\R^n$ due aperti, sia $\varphi:\Omega\to\Omega'$ un diffeomorfismo $C^1$.
	Allora, per ogni $E\subseteq \Omega$ misurabile vale
	\begin{equation*}
		m_n(\varphi(E))=\int_E\left\lvert \det \Diff \varphi(x) \right\rvert \de m_n(x)\virgola
	\end{equation*}
	dove $\Diff \varphi$ è la matrice jacobiana del diffeomorfismo $\varphi$.
\end{lemma}
\begin{proof}
	La funzione $\varphi$, essendo $C^1$, è localmente lipschitziana e di conseguenza, applicando il \cref{cor:LocLipschitzTengonoMisurabili}, manda misurabili in misurabili e trascurabili in trascurabili.
	
	Allora per la \cref{prop:BigettivaInduceMisura} la funzione di insiemi $\nu=m_n\circ \varphi$ è una misura, che risulta essere assolutamente continua rispetto alla misura di Lebesgue poiché $\varphi$ manda trascurabili in trascurabili.
	Perciò, per il \cref{thm:RadonNikodym}, esiste una funzione $\rho:\Omega\to\Rpiu$ misurabile positiva tale che
	\begin{equation*}
		\forall E\in \M_n\virgola E\subseteq \Omega:\ m_n(\varphi(E))=\int_E \rho \de x \punto
	\end{equation*}
	Per la \cref{prop:IdentitaMisuraImplicaCambioVariabile} questo implica che vale anche la formula
	\begin{equation}\label{eq:CambioRadonNikodym}
		\forall E\in \M_n\virgola E\subseteq\Omega:\ \int_{\varphi(E)} f(y)\de y = \int_{E} f(\varphi(x))\rho(x) \de x
	\end{equation}
	per ogni $f:\Omega'\to\Rbar$ integrabile.
	
	Fissato $K\subseteq\Omega$ compatto, poiché $\varphi$ è continua, anche $\varphi(K)$ è compatto e perciò in particolare ha misura finita.
	Allora, per la definizione stessa di $\rho$, la funzione $\bar\rho=\chi_K\cdot\rho$ è integrabile.
	
	A questo punto, usando la medesima notazione dell'enunciato del \cref{lemma:ContinuitaL1Palle}, definiamo le funzioni $\rho_r, \bar\rho_r$ e notiamo che
	\begin{equation*}
		\rho_r(x)=\aint_{B_r(x)}\rho(t)\de t=\frac{\int_{B_r(x)} \rho(t)\de t}{m_n(B_r(0))}=\frac{m_n(\varphi(B_r(x)))}{m_n(B_r(0))}\punto
	\end{equation*}
	
	Applichiamo allora il \cref{lemma:ContinuitaL1Palle} alla funzione $\bar\rho$, ottenendo che $\bar\rho_r$ converge a $\bar\rho$ in norma $L^1$.
	E di conseguenza, per la \cref{prop:L1ImplicaSottosuccessioneQuasiOvunque}, questo implica che, a meno di estrarre sottosuccessioni, $\bar\rho_r$ converge puntualmente quasi ovunque a $\bar\rho$.
	Allora, visto che il limite superiore è maggiore del limite di ogni sottosuccessione, abbiamo ottenuto che per quasi ogni $x\in K$ vale
	\begin{equation*}
		\bar\rho(x)\le \limsup_{r\to 0} \bar\rho_r(x)\virgola
	\end{equation*}
	ma per $x$ nella parte interna di $K$ i valori $\bar\rho_r(x)$ e $\rho_r(x)$ coincidono per $r$ sufficientemente piccolo, quindi per quasi ogni $x$ nella parte interna di $K$ abbiamo la disuguaglianza
	\begin{equation*}
		\rho(x)\le \limsup_{r\to 0} \rho_r(x)=\limsup_{r\to 0} \frac{m_n(\varphi(B_r(x)))}{m_n(B_r(0))}\virgola
	\end{equation*}
	da cui, applicando il \cref{lemma:LimiteDeterminante}, otteniamo che per quasi ogni $x$ nella parte interna di $K$
	\begin{equation}\label{eq:DisuguaglianzaRadonNikodym}
		\rho(x)\le \lvert \det \Diff \varphi(x)\rvert \punto
	\end{equation}
	Vista l'arbitrarietà della scelta di $K$ abbiamo dimostrato che fissato un qualunque compatto, per quasi ogni valore $x$ nella sua parte interna, vale l'\cref{eq:DisuguaglianzaRadonNikodym}. 
	
	Chiamiamo $N\subseteq\Omega$ l'insieme degli $x$ per cui non è valida l'\cref{eq:DisuguaglianzaRadonNikodym} e sia $(K_n)_{n\in\N}$ una successione di compatti, tali che $K_n\subseteq \mathring K_{n+1}$, la cui unione \emph{crescente} è $\Omega$\footnote{È un risultato elementare di topologia reale, noto come esaustione tramite compatti, l'esistenza di una tale successione di compatti per ogni aperto.}.
	Ogni elemento $x\in\Omega$ è definitivamente contenuto nella parte interna di $K_n$, quindi risulta
	\begin{equation*}
		N=N\cap\Omega=N\cap\bigcup_{n\in\N} \mathring K_n=\bigcup_{n\in\N}N\cap\mathring K_n\virgola
	\end{equation*}
	che, poiché $N\cap\mathring{K_n}$ è trascurabile per quanto affermato sopra, dimostra che $N$ è trascurabile.
	
	Ma il fatto che $N$ sia trascurabile implica che l'\cref{eq:DisuguaglianzaRadonNikodym} vale per quasi ogni $x\in\Omega$ e ciò, unito all'\cref{eq:CambioRadonNikodym}, implica che per ogni $f:\Omega\to\Rbar$ misurabile positiva vale
	\begin{equation*}
		\forall E\in \M_n\virgola E\subseteq\Omega:\ \int_{\varphi(E)} f(y)\de y\le 
		\int_E f(\varphi(x))\left\lvert\det \Diff \varphi(x)\right\rvert\de x \punto
	\end{equation*}
	
	Ora giungiamo alla disuguaglianza conclusiva applicando quest'ultima stima sia alla funzione $\varphi$ sia alla funzione $\varphi^{-1}$ che, essendo per ipotesi anch'esso un diffeomorfismo $C^1$, rispetterà un analogo enunciato.
	Seguendo lo schema annunciato otteniamo che per ogni $E\subseteq\Omega$ misurabile vale
	\begin{multline*}
		m_n(E) =\int_{\varphi(\varphi^{-1}(E))}1\de x\le \int_{\varphi^{-1}(E)} \left\lvert\det \Diff \varphi(x)\right\rvert\de x \\
 		\le \int_E \left\lvert\det \Diff \varphi(\varphi^{-1}(x))\right\rvert \cdot \left\lvert\det \Diff \varphi^{-1}(x)\right\rvert\de x =\int_E 1\de x=m_n(E) \virgola
	\end{multline*}
	dove abbiamo applicato implicitamente la formula del differenziale dell'inversa e la moltiplicatività del determinante.
	Però, poiché primo e ultimo membro coincidono, tutte le disuguaglianze della catena devono essere uguaglianze e da questo ricaviamo
	\begin{equation*}
		m_n(E)=\int_{\varphi^{-1}(E)} \left\lvert\det \Diff \varphi(x)\right\rvert\de x \virgola
	\end{equation*}
	che implica banalmente la tesi.
\end{proof}

\begin{theorem}\label{thm:CambioVariabile}
	Fissati $\Omega,\Omega'\subseteq\R^n$ due aperti, sia $\varphi:\Omega\to\Omega'$ un diffeomorfismo $C^1$.
	Allora, per ogni $E\subseteq \Omega$ misurabile e $f:\Omega'\to\Rbar$ integrabile, vale
	\begin{equation*}
		\int_{\varphi(E)} f(y)\de m_n(y) = \int_{E} f(\varphi(x))\lvert \det \Diff \varphi(x) \rvert \de m_n(x) \virgola
	\end{equation*}
	dove $\Diff \varphi$ è la matrice jacobiana del diffeomorfismo $\varphi$.
\end{theorem}
\begin{proof}
	È un'ovvia conseguenza della \cref{prop:IdentitaMisuraImplicaCambioVariabile}, visto che le ipotesi di questa sono verificate dal \cref{lemma:MisuraImmagine}.
\end{proof}

Ci si accorge subito, svolgendo qualche esercizio in cui è richiesto di applicare il cambio di variabile, che è spesso ostico applicare il \cref{thm:CambioVariabile} in quanto richiede ipotesi molto stringenti (in particolare il fatto che $\varphi$ sia un diffeomorfismo tra aperti). 
Diamo quindi ora una versione più \emph{sporca}, ma più utile nella pratica, del medesimo teorema mettendo in luce una delle idee chiave della teoria della misura: gli insiemi trascurabili sono davvero trascurabili.

\begin{corollary}\label{cor:CambioVariabileSporco}
	Sia $\varphi:A\to\R^n$ una funzione localmente lipschitziana con $A$ sottoinsieme di $\R^n$.
	Assumiamo che esista un aperto $\Omega\subseteq A$ tale che valgano le seguenti proprietà:
	\begin{enumerate}
		\item L'insieme $A\setminus \Omega$ è trascurabile;
		\item La funzione $\varphi$ ridotta su $\Omega$ ha regolarità $C^1$;
		\item Chiamando $C$ l'insieme dei punti in $\Omega$ tali che il differenziale di $\varphi$ non sia iniettivo, risulta che $C$ è un insieme trascurabile e $\varphi$ ridotta su $\Omega\setminus C$ è un omeomorfismo con l'immagine.
	\end{enumerate}
	
	Allora, per ogni insieme misurabile $E\subseteq \varphi(A)$ e per ogni $f:E\to\Rbar$ integrabile, vale la formula per il cambio di variabile
	\begin{equation*}
		\int_{\varphi(E)} f(y)\de m_n(y) = \int_{E} f(\varphi(x))\lvert \det \Diff \varphi(x) \rvert \de m_n(x) \virgola
	\end{equation*}
	dove $\Diff \varphi$ è la matrice jacobiana di $\varphi$.
\end{corollary}
\begin{proof}
	Per le proprietà del determinante, vale l'identità insiemistica
	\begin{equation*}
		C=\left\{x\in\Omega:\det \Diff \varphi(x)=0\right\}
	\end{equation*}
	e perciò l'insieme $C$ risulta essere chiuso in quanto coincide con l'insieme degli zeri di una funzione continua (il determinante è continuo, ed anche $\Diff \varphi$ lo è per ipotesi).
	Allora abbiamo dimostrato che $C$ è un chiuso trascurabile.
	
	Definiamo ora $\overline\Omega=\Omega\setminus C$. 
	Per quanto detto su $C$ l'insieme $\overline\Omega$ è un aperto e l'insieme $A\setminus\overline\Omega$ è trascurabile, in quanto è un sottoinsieme di $(A\setminus\Omega)\cup C$ che è per ipotesi trascurabile.
	
	Definiamo inoltre $\overline E=E\cap\overline\Omega$. 
	Vale, grazie alle proprietà di $\overline\Omega$ evidenziate, la catena di contenimenti insiemistici
	\begin{equation*}
		E\setminus\overline E=E\setminus\overline\Omega\subseteq A\setminus\overline\Omega
	\end{equation*}
	e questa dimostra che $E\setminus\overline E$ è trascurabile in quanto sottoinsieme di un trascurabile.
	
	Inoltre, essendo $\varphi$ localmente lipschitziana per ipotesi, in virtù del \cref{cor:LocLipschitzTengonoMisurabili}, anche l'immagine di $E\setminus\overline E$ tramite $\varphi$ risulta essere trascurabile. Di conseguenza anche l'insieme $\varphi(E)\setminus\varphi(\overline E)$ è trascurabile in quanto è un sottoinsieme di $\varphi(E\setminus\overline E)$.
	
	Per concludere le definizioni, chiamiamo $\overline\varphi:\overline\Omega\to\overline\Omega'=\varphi(\overline\Omega)$ la restrizione di $\varphi$ all'aperto $\overline\Omega$.
	La funzione $\overline\varphi$ è $C^1$ in quanto restrizione ad un aperto di una $C^1$ e, per la definizione di $\overline\Omega$, ha differenziale iniettivo in ogni punto ed è un omeomorfismo con l'immagine.
	Ma allora, per un corollario del teorema di inversione locale\footnote{Tale teorema, vero in una forma più generale di quella qui usata, afferma che una funzione $C^1$ con differenziale bigettivo in un punto ammette un'inversa locale in quel punto anch'essa $C^1$.}, ne deduciamo che $\overline\Omega'$ è un aperto e $\overline\varphi$ è un diffeomorfismo $C^1$.
	
	Ricordando le proprietà di $\overline\varphi$, $\overline\Omega$ e $\overline E$ evidenziate, il \cref{cor:IntegraleAMenoDiTrascurabili} e, nell'identità centrale, il \cref{thm:CambioVariabile} (di cui abbiamo verificato tutte le ipotesi), otteniamo la tesi attraverso la catena d'uguaglianze
	\begin{multline*}
		\int_{\varphi(E)} f(y)\de y=\int_{\varphi(\overline E)} f(y)\de y+\int_{\varphi(\overline E)\setminus\varphi(E)}f(y)\de y \\
		=\int_{\overline\varphi(\overline E)} f(y)\de y=\int_{\overline E} f(\overline\varphi(x))\lvert \det \Diff \overline\varphi(x) \rvert \de x \\
		=\int_{\overline E}f(\varphi(x))\lvert \det \Diff \varphi(x) \rvert \de x+
		 \int_{E\setminus\overline E}f(\varphi(x))\lvert \det \Diff \varphi(x) \rvert \de x=\int_{E}f(\varphi(x))\lvert \det \Diff \varphi(x) \rvert \de x \punto
	\end{multline*}
\end{proof}

\begin{exercise}
	Calcolare il volume della palla unitaria di $\R^3$.
\end{exercise}
\begin{proof}
	Chiamiamo $B$ la palla chiusa di raggio $1$ centrata nell'origine di $\R^3$.
	Chiamiamo $\varphi:\cc{0}{1}\times\cc{0}{\pi}\times\cc{0}{2\pi}\to B$ la funzione definita come
	\begin{equation*}
		\varphi(r,\theta,\phi)=(r\cos\theta,r\sin\theta\cos\phi,r\sin\theta\sin\phi)\punto
	\end{equation*}
	La funzione $\varphi$ è suriettiva, ed è anche iniettiva a meno di non considerare i punti $(r,\theta,\phi)$ in cui una delle variabili assume uno dei suoi due valori estremali.
	
	Facendo qualche derivata parziale, calcoliamo che lo Jacobiano vale
	\begin{equation*}
		\Diff\varphi=
		\begin{pmatrix}
		\cos\theta 			&	-r\sin\theta		&	0						\\
		\sin\theta\cos\phi	&	r\cos\theta\cos\phi	&	-r\sin\theta\sin\phi	\\
		\sin\theta\sin\phi	&	r\cos\theta\sin\phi	&	r\sin\theta\cos\phi
		\end{pmatrix}
	\end{equation*}
	e di conseguenza otteniamo
	\begin{equation*}
		\det \Diff\varphi(r,\theta,\phi)= r^2\sin\theta \punto
	\end{equation*}
	
	Ora è facile verificare che la funzione $\varphi$ rispetta tutte le richieste del \cref{cor:CambioVariabileSporco} e di conseguenza, applicando la formula per il cambio di variabile ed il \cref{thm:TonelliCompleto}, otteniamo
	\begin{multline*}
		m_3(B)=\int_B \de(x,y,z)=\int_{\varphi\left(\cc{0}{1}\times\cc{0}{\pi}\times\cc{0}{2\pi}\right)}\de(x,y,z)\\
		=\int_{\cc{0}{1}\times\cc{0}{\pi}\times\cc{0}{2\pi}} \lvert r^2\sin\theta\rvert\de(r,\theta,\phi)=\int_0^1\int_0^\pi\int_0^{2\pi}\lvert r^2\sin\theta\rvert\de r\de\theta\de\phi\\
		=2\pi\left(\int_0^1 r^2\de r\right)\left(\int_0^\pi\sin\theta\de\theta\right)=2\pi\cdot\frac 13\cdot 2=\frac 43\pi
	\end{multline*}
	che è proprio il volume richiesto dall'esercizio.
\end{proof}
\begin{remark}\label{nota:CoordinateSferiche}
	Il cambio di variabile che associa alla tripla $(r,\theta,\phi)$ il punto di $\R^3$ con coordinate $(r\cos\theta,r\sin\theta\cos\phi,r\sin\theta\sin\phi)$ è noto con il nome di coordinate sferiche.
	
	Questo cambio di variabile è spesso utile nel calcolo di integrali in $\R^3$ poiché il determinante dello Jacobiano è straordinariamente semplice e permette quindi di sfruttare, evitando lunghi conti, le eventuali simmetrie che un problema può presentare.
\end{remark}



\section{Misura su varietà differenziabili \texorpdfstring{di $\R^n$}{}}
Obiettivo di questa sezione è formalizzare i concetti di ``lunghezza'', ``superficie'', ``volume'' utilizzando gli strumenti di teoria della misura affrontati finora. 

Vogliamo introdurre innanzitutto il concetto di varietà differenziabile $k$-dimensionale di $\R^n$, che sarà la struttura che generalizza la nostra idea di ``superficie'' e su cui definiremo una misura. In particolare una varietà differenziabile sarà parametrizzata da funzioni, chiamate immersioni iniettive, che gli conferiscono le proprietà di regolarità necessarie a definire la misura.
Per fare tutto ciò sfrutteremo la topologia naturalmente indotta da $\R^n$ sui suoi sottoinsiemi.

\begin{definition}
	Una funzione $f:\Omega \subseteq \R^k\to\R^n$, con $\Omega$ aperto, si dice immersione iniettiva se è differenziabile con continuità ed è iniettiva con differenziale iniettivo in ogni punto. In tal caso si dice che $f$ è una parametrizzazione $C^1$ della superficie $k$-dimensionale $f(\Omega)$.
\end{definition}
% 
% \begin{remark}\label{nota:TopologiaIndotta}
% 	Dato un sottoinsieme $Y$ di uno spazio topologico $X$, esiste una naturale topologia indotta su $Y$. In particolare un sottoinsieme $U$ di $Y$ è un aperto di $Y$ nella topologia indotta se e solo esiste un aperto $V$ di $X$ tale che $U=V\cap Y$. 
% 	
% 	Si verifica facilmente che questa topologia indotta è effettivamente una topologia su $Y$.
% \end{remark}

D'ora in poi, quando lavoreremo su un sottoinsieme di $\R^n$, sottoinderemo di star considerando la topologia di sottospazio indotta su quel sottoinsieme.

\begin{definition}\label{def:BorelianiSottoinsieme}
	Ricalcando la \cref{def:Boreliani}, definiamo Boreliani di un sottoinsieme $X$ di $\R^n$ come la \sigalg\ generata dagli aperti di $X$ nella topologia indotta e li indichiamo con $\Borel(X)$.
\end{definition}

\begin{definition}
	Un sottoinsieme $\Sigma$ di $\R^n$ è una varietà differenziabile $k$-dimensionale di $\R^n$ se per ogni $x\in\Sigma$ esiste un intorno aperto di $x$ (nella topologia indotta) in $\Sigma$ che ammette una parametrizzazione $C^1$ definita su un aperto di $\R^k$.
\end{definition}

% DA AGGIUSTARE
% Di seguito enunceremo, fra gli altri risultati, dei fatti di topologia di base non del tutto banali, che però non ci soffermeremo a dimostrare in quanto non strettamente pertinenti alla trattazione. Tali fatti ci serviranno a dimostrare alcuni lemmi utili ad arrivare alla definizione di misura su una varietà differenziabile di $\R^n$.

\begin{lemma}\label{lemma:SottovarietaUnioneNumerabile}
	Data $\Sigma$ varietà differenziabile $k$-dimensionale di $\R^n$, esiste un ricoprimento numerabile  $\{\Sigma_i\}_{i\in\N}$ di $\Sigma$ tale che, per ogni $i\in\N$, $\Sigma_i$ è aperto nella topologia di $\Sigma$ e ammette una parametrizzazione $C^1$ definita su un aperto $\Omega_i$ di $\R^k$.
\end{lemma}
\begin{proof}
	Per il \cref{cor:RLindelof} la varietà $\Sigma$ è uno spazio di Lindelöf con la sua topologia di sottospazio.
	
	Per la definizione stessa di varietà differenziabile $k$-dimensionale, per ogni punto $P\in\Sigma$ esiste un suo intorno aperto $\Sigma_P$ (nella topologia di $\Sigma$) che ammette parametrizzazione $C^1$. Abbiamo quindi che $\{\Sigma_P\}_{P\in\Sigma}$ è un ricoprimento aperto di $\Sigma$ e perciò, poiché $\Sigma$ è uno spazio di Lindelöf, possiamo estrarne un ricoprimento numerabile, da cui la tesi.
\end{proof}

\begin{proposition}\label{prop:ApertoUnioneCompatti}
	Ogni aperto $A$ di $\R^n$ si può scrivere come unione numerabile crescente di compatti.
\end{proposition}

\begin{lemma}\label{lemma:ImmagineApertiContinua}
	Sia $f:\R^k\to\R^n$ una funzione continua, allora $f$ manda aperti di $\R^k$ in Boreliani di $\R^n$.
\end{lemma}
\begin{proof}
	Dato $A$ aperto di $\R^k$, per la \cref{prop:ApertoUnioneCompatti} esiste una successione numerabile crescente di compatti $\{K_n\}_{n\in\N}$ tale che $A=\bigcup_{n\in\N}K_n$. Perciò abbiamo che
	\begin{equation*}
		f(A)=f\left(\bigcup_{n\in\N}K_n\right)=\bigcup_{n\in\N}f(K_n)\punto
	\end{equation*}
	L'immagine tramite una funzione continua di un compatto è compatta, quindi $f(K_n)$ è compatto per ogni $n\in\N$ e di conseguenza è anche un Boreliano di $\R^n$.
	
	Abbiamo ottenuto perciò che $f(A)$ si può scrivere come unione numerabile di Boreliani, quindi è anch'esso un Boreliano, come volevamo dimostrare.	
\end{proof}

\begin{lemma}\label{lemma:ContinuaImplicaBoreliana}
	Sia $f:X\subseteq\R^k\to Y\subseteq\R^n$ una funzione continua, allora $f$ è Boreliana, cioè controimmagine di Boreliani è Boreliana, dove si intendono come Boreliani quelli ottenuti dalla topologia indotta, come nella \cref{def:BorelianiSottoinsieme}. 
\end{lemma}
\begin{proof}
	La dimostrazione ricalca fondamentalmente quella della \cref{prop:CounterImgMis}, sfruttando che controimmagine di aperti tramite funzioni continue è aperta.
\end{proof}

\begin{remark}\label{nota:SigmaBoreliano}
	Utilizzando la stessa notazione del \cref{lemma:SottovarietaUnioneNumerabile}, notiamo che per il \cref{lemma:ImmagineApertiContinua} l'insieme $\Sigma_i$ è un Boreliano di $\R^n$, per ogni $i\in\N$, in quanto immagine di un aperto tramite la funzione continua $f_i$. Di conseguenza abbiamo anche che $\Sigma=\cup_{i\in\N}\Sigma_i$ è un Boreliano, poiché unione numerabile di Boreliani.  
\end{remark}

\begin{remark}\label{nota:BorelianiSottovarieta}
	Per la \cref{nota:SigmaBoreliano}, abbiamo che gli aperti di $\Sigma$ sono anche Boreliani di $\R^n$, poiché sono intersezione fra aperti di $\R^n$ e $\Sigma$, che è un Boreliano di $\R^n$. Perciò in particolare otteniamo che i Boreliani di $\Sigma$ sono un sottoinsieme dei Boreliani di $\R^n$.
\end{remark}

Abbiamo finalmente tutti gli strumenti per definire la misura di Lebesgue $k$-dimensionale su una varietà differenziabile $k$-dimensionale di $\R^n$. Prima di dare la definizione vera e propria cerchiamo però di capire quale è la forma che ci aspettiamo da questa misura su delle strutture più semplici di una varietà differenziabile. A prima vista infatti la definizione di misura $k$-dimensionale su una varietà differenziabile può sembrare del tutto controintuitiva, senza aver studiato precedentemente dei casi più semplici.

Consideriamo quindi il caso di un sottospazio vettoriale $k$-dimensionale $V$ di $\R^n$. Questo è un esempio speciale di varietà differenziabile $k$-dimensionale di $\R^n$, in quando esiste una funzione lineare $L:\R^k\to \R^n$, tale che $L(\R^k)=V$, $L$ è bigettiva con la sua immagine e ha facilmente differenziale iniettivo, in quanto il differenziale di una funzione lineare è la funzione stessa, quindi $L$ è una parametrizzazione $C^1$ di $V$.

Data $L:\R^k\to\R^n$ applicazione lineare, esiste sempre la decomposizione polare della matrice $L$ che è della forma $L=U\cdot S$, con $S=(L^T\cdot L)^{\frac 12}:\R^k\to\R^k$ matrice simmetrica invertibile \footnote{Data $A$ matrice simmetrica definita positiva (che nel nostro caso è $L^T\cdot L$), esiste sempre una matrice denotata con $A^\frac 12$ che elevata al quadrato dà $A$.} e $U=L\cdot (L^T\cdot L)^{-\frac 12}:\R^k\to\R^n$ ortogonale, cioè tale che $U^T\cdot U=I$.

Quello che vogliamo da una misura su $V$ è che coincida con quella che potremmo costruire imitando quello che abbiamo fatto su $\R^k$. Infatti data una base ortonormale di $V$ nessuno ci vieta di definire l'insieme dei parallelepipedi $k$-dimensionali di $V$, come nella \cref{def:LebesgueSemiaperti}, e i loro volumi, come nella \cref{def:LebesgueElementare}, e procedere quindi alla costruzione della misura su $V$ esattamente come abbiamo fatto nella \cref{sezione:MisuraLebesgue}.
Questo equivale a dire che data $U:\R^k\to V$ isometria lineare (come quella che abbiamo ottenuto dalla decomposizione polare di $L$), vorremo che valga $\sigma(E)=m_k(U^{-1}(E))$ per ogni $E\in\Borel(V)$, dove abbiamo chiamato $\sigma$ la misura su $V$. Infatti scegliere $U:\R^k\to V$ isometria lineare equivale a prendere una base ortonormale di $V$, mentre imporre $\sigma(E)=m_k(U^{-1}(E))$ vuol dire ``copiare'' la misura di $\R^k$ su $V$.

Calcoliamo quindi a quanto equivale $m_k(U^{-1}(E))$, dato $E\in\Borel(V)$, in funzione della nostra ``immersione iniettiva'' $L$. Sfruttando la \cref{prop:MisuraImmagineLineare} e le proprietà del determinante, otteniamo
\begin{equation*}
	m_k(U^{-1}(E))=m_k(S\cdot L^{-1}(E))=\lvert\det S\rvert\cdot m_k(L^{-1}(E))=\sqrt{\det(L^T\cdot L)}\cdot m_k(L^{-1}(E))\virgola
\end{equation*}
in particolare vorremmo quindi che la misura $k$-dimensionale di $E$ sia uguale a
\begin{equation*}
	\sigma(E)=m_k(U^{-1}(E))=\int_{L^{-1}(E)}\sqrt{\det(L^T\cdot L)}\de x\punto
\end{equation*}

Dopo questa breve digressione, la definizione che stiamo per dare sembrerà meno arbitraria.

\begin{definition}\label{def:MisuraKDimensionale}
	Sia $E\in\Borel(\Sigma$), dove $\Sigma$ è una varietà differenziabile $k$-dimensionale di $\R^n$, e siano $\{\Sigma_i\}_{i\in\N}$ e $\{\Omega_i\}_{i\in\N}$ come nel \cref{lemma:SottovarietaUnioneNumerabile}. Chiamiamo inoltre, per ogni $i\in\N$, $f_i:\Omega_i\to\Sigma_i$ l'immersione iniettiva che parametrizza $\Sigma_i$.
	
	Definiamo la misura di Lebesgue $k$-dimensionale di $E$ come
	\begin{equation*}
		\sigma(E)=\sum_{i\in\N} \sigma_i(E\cap F_i)\virgola
	\end{equation*}
	dove $F_i=\Sigma_i\setminus (\cup_{j<i}\Sigma_j)$ e $\sigma_i$ è una funzione definita sui Boreliani di $\Sigma$ a valori in $\Rpiu$ tale che, dato $B\in\Borel(\Sigma)$, vale
	\begin{equation*}
		\sigma_i(B)=\int_{f_i^{-1}(B)}\sqrt{\det(\Diff f_i(x)^T\Diff f_i(x))} \de x\punto
	\end{equation*}

\end{definition}

Vogliamo ora dimostrare che quella appena data è una buona definizione, cioè non dipende dalla scelta di $\{\Sigma_i\}_{i\in\N}$ e dalle parametrizzazioni, tutte le quantità sono ben definite e $\sigma$ è veramente una misura.

\begin{lemma}\label{lemma:InvarianzaImmersione}
	Date $f:\Omega\subseteq\R^k\to M$ e $g:\Omega'\subseteq\R^k\to M$ parametrizzazioni $C^1$ della superficie $k$-dimensionale $M$, per ogni $E\in\Borel(M)$ sono definiti i seguenti integrali e vale che 
	\begin{equation*}
		\int_{g^{-1}(E)}\sqrt{\det(\Diff g(x)^T\Diff g(x))}\de x=\int_{f^{-1}(E)}\sqrt{\det(\Diff f(x)^T\Diff f(x))}\de x\punto
	\end{equation*}
\end{lemma}
\begin{proof}
	Innanzitutto notiamo che, dato $E\in\Borel(M)$, i due integrali sono definiti, in quanto $\Diff f(x)$ e $\Diff g(x)$ sono continue e $f^{-1}(E)$ e $g^{-1}(E)$ sono Boreliani di $\R^k$. Infatti per la \cref{nota:BorelianiSottovarieta} $E$ è un Boreliano di $\R^n$, quindi $f^{-1}(E)$ e $g^{-1}(E)$ sono controimmagini di un Boreliano tramite funzioni continue e di conseguenza sono Boreliani di $\R^k$ per il \cref{lemma:ContinuaImplicaBoreliana}.

	Sia $\varphi=f^{-1}\circ g$, allora è ovvio che $\varphi$ è una funzione bigettiva tra $\Omega'$ e $\Omega$. Inoltre, se dimostrassimo che $\varphi$ è differenziabile con continuità,  ne seguirebbe, per assoluta simmetria tra $f$ e $g$, che anche $\varphi^{-1}=g^{-1}\circ f$ è differenziabile con continuità e perciò avremmo ottenuto che $\varphi$ è un diffeomorfismo $C^1$ fra gli aperti $\Omega'$ e $\Omega$. Il fatto che $\varphi$ sia differenziabile con continuità lo lasciamo da dimostrare al lettore poiché è non banale con strumenti elementari\footnote{Un modo per dimostrarlo può essere mostrare che la mappa $f^{-1}$ si estende in ogni punto ad una funzione da un intorno del punto in $\Omega'$ che sia differenziabile con continuità. Avendo dimostrato questo, $\varphi$ sarebbe $C^1$ in quanto composizione di funzioni $C^1$.} e centra ben poco con la teoria che stiamo costruendo.
	
	Quindi, poichè $\varphi$ è un diffeomorfismo $C^1$, per il \cref{thm:CambioVariabile} otteniamo
	\begin{multline*}
		\int_{f^{-1}(E)}\sqrt{\det(\Diff f(x)^T\Diff f(x))}\de x=\int_{\varphi\circ g^{-1}(E)}\sqrt{\det(\Diff f(x)^T\Diff f(x))}\de x=\\
		=\int_{g^{-1}(E)}\sqrt{\det(\Diff f(\varphi(x))^T\Diff f(\varphi(x)))}\ \lvert\det \Diff \varphi(x)\rvert\de x=\int_{g^{-1}(E)}\sqrt{\det(\Diff g(x)^T\Diff g(x))}\de x\virgola
	\end{multline*}
	dove abbiamo usato che se $g=f\circ \varphi$, allora $\Diff g(x)=\Diff f(\varphi(x))\cdot \Diff\varphi(x)$.
\end{proof}

\begin{remark}\label{nota:SigmaIMisura}
	La funzione $\sigma_i:\Borel(\Sigma)\to\Rpiu$, introdotta nella \cref{def:MisuraKDimensionale}, è una misura sui Boreliani di $\Sigma$.
\end{remark}
\begin{proof}
	Sia $f_i:\Omega_i\to\Sigma_i\subseteq\Sigma$ l'immersione iniettiva con cui definisco $\sigma_i$. La funzione $f_i^{-1}$ manda Boreliani di $\Sigma$ in Boreliani di $\R^k$, quindi per il \cref{lemma:MisuraIntegrale} $(\Sigma,\Borel(\Sigma),\sigma_i)$ è uno spazio di misura.
\end{proof}

\begin{proposition}\label{prop:DefMisuraKDimBuona}
	La \cref{def:MisuraKDimensionale} è una buona definizione e non dipende dalla scelta del ricoprimento $\{\Sigma_i\}_{i\in\N}$ e delle immersioni iniettive $\{f_i\}_{i\in\N}$.
\end{proposition}
\begin{proof}
	Innanzitutto notiamo che la definizione è ben posta, in quanto tutti gli integrali sono ben definiti. Infatti $\Diff f_i(x)$ è continua per ogni $i\in\N$ e inoltre è facile verificare che $E\cap F_i$ è un Boreliano di $\R^n$, utilizzando la \cref{nota:BorelianiSottovarieta}, in quanto intersezione di Boreliani. Quindi $f_i^{-1}(E\cap F_i)$ è un Boreliano di $\R^k$ per il \cref{lemma:ContinuaImplicaBoreliana} (analogamente a quanto detto nel \cref{lemma:InvarianzaImmersione}).

	Consideriamo ora due diversi ricoprimenti $\{\Sigma_i\}_{i\in\N}$ e $\{\Sigma_i'\}_{i\in\N}$ di $\Sigma$, tali che per ogni $i\in\N$ $\Sigma_i$ e $\Sigma_i'$ sono aperti nella topologia di $\Sigma$ e ammettono rispettivamente parametrizzazioni $f_i$ e $g_i$. Vogliamo dimostrare che le misure $\sigma$ e $\sigma'$ indotte dai due ricoprimenti coincidono.
	
	Chiamiamo $F_i=\Sigma_i\setminus (\cup_{j<i}\Sigma_j)$ ed $F_i'=\Sigma_i'\setminus (\cup_{j<i}\Sigma_j')$, come nella \cref{def:MisuraKDimensionale}. Sia poi $B_{ij}=F_i \cap F_j'$, allora $\{B_{ij}\}_{i,j\in\N}$ è una partizione di $\Sigma$ con elementi disgiunti (è facile verificare che $B_{ij}\cap B_{i'j'}=\emptyset$ per ogni scelta degli indici in $\N$). In particolare, poiché
	\begin{equation*}
		E\cap F_i=E\cap (\sqcup_{j\in\N}B_{ij})=\sqcup_{j\in\N}(E\cap B_{ij})\virgola
	\end{equation*}
	abbiamo che
	\begin{equation*}
		\sigma_i(E\cap F_i)=\sigma_i(\sqcup_{j\in\N}(E\cap B_{ij}))=\sum_{j\in\N}\sigma_i(E\cap B_{ij})\virgola
	\end{equation*}
	in quanto $\sigma_i$ è una misura per la \cref{nota:SigmaIMisura}.
	
	Otteniamo quindi che
	\begin{equation*}
		\sigma(E)=\sum_{i\in\N}\sigma_i(E \cap F_i)=\sum_{i,j\in\N} \sigma_i(E\cap B_{ij})
	\end{equation*}
	e analogamente $\sigma'(E)=\sum_{i,j\in\N} \sigma_i'(E\cap B_{ij})$. Da notare che in queste sommatorie a due indici non ci dobbiamo preoccupare dell'ordine in cui si fanno le somme, perché tutti i termini della sommatoria sono positivi (poiché integrali di funzioni positive).

	Per il \cref{lemma:InvarianzaImmersione} applicato alle parametrizzazioni $f_i$ e $g_j$ e al Boreliano $E\cap B_{ij}$, abbiamo però che $\sigma_i(E\cap B_{ij})=\sigma_j'(E\cap B_{ij})$, da cui otteniamo proprio che 
	\begin{equation*}
		\sigma(E)=\sum_{i,j\in\N} \sigma_i(E\cap B_{ij})=\sum_{i,j\in\N} \sigma_i'(E\cap B_{ij})=\sigma'(E)\punto
	\end{equation*}

\end{proof}

\begin{theorem}
	La misura di Lebesgue $k$-dimensionale definita sui Boreliani di una varietà differenziabile $k$-dimensionale di $\R^n$ è effettivamente una misura; cioè, chiamata $\Sigma$ la varietà differenziabile, $(\Sigma,\Borel(\Sigma),\sigma)$ è uno spazio di misura.
\end{theorem}
\begin{proof}
	Per la \cref{nota:SigmaIMisura}, $(\Sigma,\Borel(\Sigma),\sigma_i)$ è uno spazio di misura, per ogni $i\in\N$. Inoltre $\sigma=\sum_{i\in\N}\sigma_i$, quindi è facile verificare che $\sigma$ è a sua volta una misura essendo somma di misure.
\end{proof}

Mostriamo ora degli esempi e delle formule utili per il calcolo della misura $k$-dimensionale nella pratica. In particolare trattiamo dei casi particolare del calcolo di $\det(L^TL)$ per una data funzione lineare $L:\R^k\to\R^n$, in quanto ciò risulta spesso utile (come vedremo poi nell'\cref{es:MisuraTetraedro}) per il calcolo della misura $k$-dimensionale su una varietà.

\begin{lemma}\label{lemma:DeterminanteProdottoTrasposta}
	Sia $L:\R^k\to\R^n$ lineare con $k\le n$, allora $\det(L^TL)$ equivale alla somma dei quadrati dei determinanti dei minori massimali di $L$.
\end{lemma}
\begin{proof}
	TODO
\end{proof}

\begin{remark}\label{nota:DeterminanteMatriceQuasiIdentita}
	Sia $L:\R^n\to\R^{n+1}$ lineare tale che $L=\begin{pmatrix} v^T \\ I_n \end{pmatrix}$, dove $v$ è un vettore colonna e $I_n$ è la matrice identità $n\times n$. Allora $\det(L^TL)=1+\lvert v \rvert ^2$.
\end{remark}
\begin{proof}
	Lasciamo al lettore la dimostrazione, che segue dal \cref{lemma:DeterminanteProdottoTrasposta} con un facile conto.
\end{proof}

\begin{lemma}
	Sia $L:\R^2\to\R^3$ lineare e siano $u$ ed $v$ le colonne di $L$, vista come matrice. Allora $\det(L^TL)=\lvert u\rvert^2\cdot\lvert v \rvert^2- <u,v>^2$.
\end{lemma}
\begin{proof}
	Sfruttando che $L=\begin{pmatrix}u& v\end{pmatrix}$, otteniamo
	\begin{equation*}
		L^TL=\begin{pmatrix}u^T\\v^T\end{pmatrix}\begin{pmatrix}u&v\end{pmatrix}=\begin{pmatrix}
		                                                                         	u^Tu & u^Tv\\
		                                                                         	v^Tu & v^Tv
		                                                                         \end{pmatrix}\virgola
	\end{equation*}
	da cui ricaviamo che
	\begin{equation*}
		\det(L^TL)=\det\begin{pmatrix}u^Tu & u^Tv\\v^Tu & v^Tv\end{pmatrix}=\lvert u \rvert^2\cdot \lvert v \rvert^2-<u,v>^2\punto
	\end{equation*}
\end{proof}


Concludiamo questa sezione con un esercizio, per chiarire i concetti appena trattati.

\begin{exercise}\label{es:MisuraTetraedro}
	Calcolare la misura di un tetraedro $n$-dimensionale di lato 1\footnote{Il tetraedro $n$-dimensionale di lato $d$ è l'inviluppo convesso di $n+1$ punti in $\R^n$ a distanza $d$ uno dall'altro.}.
\end{exercise}
\begin{proof}
	Innanzitutto notiamo che un tetraedro $n$-dimensionale è misurabile in $\R^n$ in quanto è chiuso.
	Inoltre  dati due tetraedri $n$-dimensionali $T$ e $T'$ di lato rispettivamente $d$ e $d'$, poichè possiamo ottenere l'uno dall'altro con un'isometria composta un'omotetia di fattore $\frac{d'}d$, grazie alla \cref{prop:LebesgueProprietaIsometria} e alla $n$-omogeneità della misura di Lebesgue, vale che 
	\begin{equation*} 
		m_n(T')=\left(\frac{d'}{d}\right)^nm_n(T)\punto
	\end{equation*}

	Definiamo l'iperpiano affine $A=\{(x_1,x_2,\ldots,x_{n+1}) \mid x_1+x_2+\ldots +x_{n+1}=1\}$ in $\R^{n+1}$, consideriamo su di esso i punti $e_1,e_2,\ldots,e_{n+1}$ della base canonica di $\R^{n+1}$ e chiamiamo $T$ l'inviluppo convesso di questi $n+1$ punti. 
	L'insieme $T$ si trova sull iperpiano $A$ ed è l'inviluppo convesso di $n+1$ punti a distanza $\sqrt 2$ uno dall'altro, quindi, per quanto detto sulla misura superficiale, $\sigma(T)$ equivale alla misura di un tetraedro $n$-dimensionale di lato $\sqrt 2$. %TODO: Perché è vero questo?

	Perciò ci siamo ricondotti a calcolare $\sigma(T)$, poichè il volume di un tetraedro $n$-dimensionale di lato 1 sarà poi dato da $2^{-\frac n2}\sigma(T)$.

	Sia $L:\R^n\to\R^{n+1}$ la funzione tale che $L(x_1,x_2,\ldots,x_n)=(1-\sum_ix_i,x_1,x_2,\ldots,x_n)$, allora banalmente $L$ è un'immersione iniettiva che parametrizza $A$. Inoltre vale facilmente che 
	\begin{equation*}
		\Diff L=\begin{pmatrix}
		        	-1	& -1 & \cdots & -1\\
					1 & & & \\
					 & 1 & & \\
					 & &\ddots& \\
					 & & & 1
		        \end{pmatrix}\virgola
	\end{equation*}
	quindi per la definizione di misura $n$-dimensionale abbiamo che
	\begin{equation*}
		\sigma(T)=\int_{L^{-1}(T)}\sqrt{\det(\Diff L^T\Diff L)}\de x=\int_{L^{-1}(T)}\sqrt{\det(I_n+vv^T)}\de x\virgola
	\end{equation*}
	dove $I_n$ è la matrice identità $n\times n$ e $v$ è il vettore in cui tutte le $n$ entrate sono uguali a -1. Lasciando al lettore la dimostrazione del fatto che $\det(I_n+vv^T)=1+n$, otteniamo quindi %TODO: La dimostrazione lasciata al lettore probabilmente comparirà nella sezione su Stokes!
	\begin{equation*}
		\sigma(T)=\int_{L^{-1}(T)}\sqrt{1+n}\de x=m_n(L^{-1}(T))\sqrt{1+n}\punto
	\end{equation*}

	È molto facile verificare che $L^{-1}(T)$ è il poliedro che ha come vertici la base canonica e lo zero di $\R^n$. Perciò per trovare $m_n(L^{-1}(T))$, calcoliamo la misura del poliedro in $\R^m$ che ha come vertici la base canonica e lo zero. Chiamiamo tale misura $A_m$ e dimostriamo per induzione su $m$ che $A_m=1/m!$.
	
	Per $m=1$ il poliedro è semplicemente il segmento di lato 1 in $\R$, quindi è verificato che $A_1=1$. Dimostriamo ora che se la tesi vale per $m=k$, allora vale per $m=k+1$. Se consideriamo il poliedro in $\R^{k+1}$ formato dallo zero e da tutti i vertici della base canonica tranne $e_{k+1}$, otteniamo un poliedro che sta sull'iperpiano $\{x_{k+1}=0\}\subseteq\R^{k+1}$ e che equivale al poliedro nella dimensione precedente a meno dell'identificazione fra l'iperpiano e $\R^k$. 
	Integrando sulla componente $x_{k+1}$, ricordando il teorema di Tonelli, otteniamo quindi
	\begin{equation*}
		A_{k+1}=\int_0^1t^kA_k\de x=\int_0^1t^k\frac 1{k!}\de x=\frac 1{(k+1)!}\virgola
	\end{equation*}
	poichè ad altezza $t$ la misura della sezione del poliedro è $t^kA_k$, in quanto tale sezione è un poliedro nella dimensione minore di lato $t$ (per definizione di inviluppo convesso la \emph{fetta} deve essere un'omotetia di fattore $t$ della base).
	
	Riunendo quanto detto, abbiamo che il volume di un tetraedro $n$-dimensionale di lato 1 è
	\begin{equation*}
		2^{-\frac n2}\sigma(T)=2^{-\frac n2}m_n(L^{-1}(T))\sqrt{1+n}=\frac{\sqrt{1+n}}{2^{\frac n2}n!}\punto		
	\end{equation*}
		
\end{proof}


\section{Formule di Stokes}\label{sezione:FormuleStokes}

Durante tutta la sezione assumeremo sempre che $\Omega\subseteq\R^n$ sia un aperto.

\begin{theorem}\label{thm:PtRegEquiv}
	Per un punto $x\in \partial \Omega$ le seguenti $3$ condizioni sono equivalenti:
	\begin{enumerate}
		\item esistono $U$ intorno di $x$ e $f\in C^1(U,\R)$ tale che si abbia
			\begin{itemize}
				\item $U\cap \partial \Omega=f^{-1}(0)$,
				\item $U\cap \Omega=f^{-1}(\oo{-\infty}{0})$,
			\end{itemize}
			e tale che $\grad f(x)\neq 0$ se $f(x)=0$;\label{PRE:i}
		\item esistono $U$ intorno di $x$ e $g$ diffeomorfismo $C^1$ da $\R^n$ in $U$ tale che\label{PRE:ii}
			\begin{itemize}
				\item $U\cap \partial \Omega = g(\R^{n-1}\times\{0\})$,
				\item $U\cap \Omega = g(\R^{n-1}\times\oo{-\infty}{0})$;
			\end{itemize}
		\item esistono, a meno di riordinare le coordinate, un intorno $U=V\times I$ di $x=(y,t)$ tale che $V\in\R^{n-1}$ è intorno di $y$
			e $I\in\R$ è intorno di $t$ e una funzione $\psi\in C^1(V,I)$ tale che
			\begin{itemize}
				\item $U\cap \partial \Omega = \{(z,s):s=\psi(z)\}$,
				\item $U\cap \Omega = \{(z,s):s<\psi(z)\}$.
			\end{itemize}\label{PRE:iii}
	\end{enumerate}
\end{theorem}

\begin{proof}
	Dimostriamo la catena in ordine inverso:
	\begin{description}
		\item [\ImplicationProof{PRE:i}{PRE:iii}] Dato che $\grad f(x)\neq 0$, esiste una derivata parziale non nulla, supponiamo senza perdita
			di generalità che sia l'ultima e che sia positiva (se negativa il ragionamento è lo stesso).
			Allora le ipotesi del teorema della funzione implicita del Dini sono verificate, per cui otteniamo
			che esiste un intorno $V\times I\subseteq U$ di $x$ per cui esiste $\psi\in C^1(V,I)$ tale che $f(v,\psi(v))=0$ e $\psi(v)$
			è l'unico punto $t'\in I$ tale che $f(v,t')=0$.
			Inoltre, per la continuità di $\grad f(x)$, esiste un intorno $V'\subseteq V$ di $v$ tale che $\grad f(v')_n>\frac{1}{2}
			\grad f(x)_n$ per tutti i $v'\in V'$. Ora, per la differenziabilità di $f$ in $v'$ si ha che %TODO: Chi è \delta?
			\[
				f(v',\psi(v')+\delta)=f(v', \psi(v'))+\grad f(v')_n\delta+\smallO(\delta)=\grad f(v')_n\delta+\smallO(\delta)\virgola
			\]
			che, per $\delta>0$ e abbastanza piccolo, dà
			\begin{equation*}
				f(v',\psi(v')+\delta)\geq \frac{1}{2}\grad f(v')_n\delta\geq \frac{1}{4}\grad f(x)_n\delta>0\punto
			\end{equation*}
			Infine, non può esistere un $\delta>0$ tale che $\psi(v')+\delta\in I$ e
			$f(v',\psi(v')+\delta)<0$, altrimenti per la continuità di $f$ si avrebbe che esiste un punto $\psi(v')+\delta'\neq \psi(v')$
			che sia zero per la funzione $f(v',\cdot)$, contraddicendo il teorema della funzione implicita del Dini.
			Per $\delta<0$ si ha lo stesso risultato. Questo ci dice che 
			\begin{align*}
				f^{-1}(0)\cap (V'\times I)&=\{(v',s):s=\psi(v')\} \virgola\\
				f^{-1}(\oo{-\infty}{0})\cap (V'\times I)&=\{(v',s):s<\psi(v')\}\virgola
			\end{align*}
			che conclude per le ipotesi sulle controimmagini di $f$.
		\item [\ImplicationProof{PRE:iii}{PRE:ii}] Se abbiamo $V\times I$ intorno di $x$ e $\psi$, detto $J=\oo{-\varepsilon}{\varepsilon}$
			possiamo ottenere un diffeomorfismo $g:V\times I \rightarrow V\times J$ ponendo $g(z,s)=(z,s-\psi(z))$. 
			Di conseguenza otteniamo
			\begin{align*}
				g(V\times \{0\})&=\graf(\psi)=U\cap \partial \Omega \virgola\\
				g(V\times \oo{-\varepsilon}{0})&=\{(z,s):s<\psi(z)\}=U\cap \Omega\punto
			\end{align*}
			Per ottenere un diffeomorfismo di tutto $\R^n$ basta comporre $g$ con opportuni diffeomorfismi tra un intorno rettangolare
			di $x$ e $\R^n$.
		\item [\ImplicationProof{PRE:ii}{PRE:i}] Per la prima condizione è sufficiente scegliere $f=(g^{-1})_n$. %TODO: Perchè funziona questa scelta?
	\end{description}
\end{proof}


\begin{definition}[Punto regolare]
	Diremo che un punto $x\in \partial \Omega$ è un punto regolare del bordo se verifica una delle $3$ condizioni di cui
	al \cref{thm:PtRegEquiv}.
\end{definition}

D'ora in poi assumeremo che $\Omega$ sia anche limitato.

\begin{definition}[Aperto regolare]
	Diremo che $\Omega$ è un aperto regolare se tutti i punti del suo bordo sono regolari.
\end{definition}


\begin{theorem}[Partizione dell'unità]\label{thm:PartizioneUnita}
	Dato un ricoprimento di $\R^n$ con aperti $(\Omega_i)_{i\in I}$, esiste una famiglia di funzioni $f_i:\R^n\to\Rpiu$ infinitamente derivabili
	e non negative tali che:
	\begin{itemize}
		\item la chiusura dell'insieme in cui $f_i$ non si annulla è contenuta in $\Omega_i$ per ogni $i\in I$,
		\item per ogni $x\in\R^n$ vale che $\sum_{i\in I} f_i(x)=1$\footnote{La serie la definiamo come il $\sup$ delle somme finite.}.
	\end{itemize}
\end{theorem}
\begin{proof}
	TODO
\end{proof}

\begin{lemma}\label{lemma:EquivRegolare}
	$\Omega$ è un aperto regolare se e solo se esiste una funzione $\phi\in C^1(\R^n,\R)$ tale che $\grad \phi(x)\neq 0$ se $\phi(x)=0$ e tale che
	\begin{itemize}
		\item $\Omega = \phi^{-1}(\oo{-\infty}{0})$,
		\item $\partial \Omega = \phi^{-1}(0)$.
	\end{itemize}
\end{lemma}

\begin{proof}
	L'implicazione del ``se'' è banale per la caratterizzazione del punto \ref{PRE:i}. Per l'altra implicazione, per la regolarità dei punti
	del bordo di $\Omega$, consideriamo per ogni $x\in\partial \Omega$ un intorno $U_x$ tale che esista una funzione $\psi_x:U_x\rightarrow \R$
	come nel \cref{thm:PtRegEquiv}. Dato che $\Omega$ è limitato, $\partial\Omega$ è compatto e gli $U_x$ sono un ricoprimento aperto;
	quindi consideriamo un sottoricoprimento finito $U_1, \dots, U_n$  e prendiamo anche $U_0 = \R^n \setminus \partial \Omega$.
	Ora abbiamo che gli aperti $U_0,\dots,U_n$ sono un ricoprimento di $\R^n$;
	il \cref{thm:PartizioneUnita} fornisce quindi delle funzioni $f_i$ per ogni $i\in\{0,\dots, n\}$.
	Infine, definiamo $\psi_0:U_0\rightarrow \R$ tale che $\psi_0(x)=1$ se $x\notin \Omega$, $\psi_0(x)=-1$ altrimenti. Sia ora 
	\[
		\phi(x)=\sum_{i=0}^n f_i(x) \psi_i(x)\virgola
	\] %TODO: Chi cazzo è \psi_i?
	dove, con una leggera imprecisione formale, poniamo $f_i(x) \psi_i(x) = 0$ se $x\notin U_i$, in modo che $f_i(x) \psi_i(x)$ sia una funzione
	$C^1$ in tutto $\R^n$. Infatti, in $U_i$, $\psi_i$ è $C^1$ e $f_i$ è addirittura $C^{\infty}$, altrove l'abbiamo posta uguale a zero.
	Ora abbiamo che $\phi\in C^1$ perché somma di funzioni $C^1$; $\phi(x)=0$ per ogni $x\in \partial\Omega$, dato che ognuna delle
	$\psi_i$ aveva questa proprietà se era definita in $x$ e la $\phi$ è una combinazione convessa di $\psi_i$;
	$\phi(x)<0$ per ogni $x \in \Omega$, perché $\phi$ è combinazione convessa di funzioni che, per ciascun $x\in\Omega$, sono non positive e
	almeno di una strettamente negativa; similmente, $\phi(x)>0$ per ogni $x \notin \overline\Omega$.
	
	Rimane da mostrare che $\grad f(x)\neq 0$ se $f(x)=0$. Ma in ciascun punto $x$ del bordo, le funzioni $\psi_i$ non hanno differenziale nullo
	quando sono definite in $x$. Esisterà quindi una direzione $e_k$ per la quale esiste almeno un $i$ tale che $\grad \psi_i(x)_k\neq 0$. Dato che
	le $\psi_i$ sono differenziabili, si ha che
	\[
		\psi_i(x+\delta e_k)=\psi_i(x)+\grad \psi_i(x)_k \delta + \smallO(\delta) = \grad \psi_i(x)_k \delta + \smallO(\delta)
	\]
	per cui, dato che $\psi_i(x+\delta e_k)<0$ solo se $x+\delta e_k\in \Omega$, quindi, per $u\neq i'$, $\grad \psi_i(x)_k$ e
	$\grad \psi_{i'}(x)_k$ non possono essere discordi, altrimenti si ha $\psi_i(x+\delta e_k)= \grad \psi_i(x)_k \delta + \smallO(\delta)>0$ e
	$\psi_i'(x+\delta e_k)=\grad \psi_i'(x)_k \delta + \smallO(\delta)<0$ pur avendo in entrambi i casi
	che $x+\delta e_k$ sta in $\Omega$. %TODO: Questa frase è incomprensibile. Suggerimento generale: Spendi più tempo e più righe e vai a capo e dì cosa stai facendo.
	
	Quindi, sempre per lo stesso $x\in\partial \Omega$,
	\[ 
		\grad \phi(x)=\sum_{i=0}^n \grad (f_i \cdot\psi_i)(x)=\sum_{i=0}^n \grad f_i(x)\cdot \psi_i(x)+\sum_{i=0}^n f_i(x)\cdot \grad \psi_i(x)=
		\sum_{i=0}^n f_i(x)\cdot \grad  \psi_i(x)\virgola
	\]
	poiché $\psi_i\equiv 0$ sul bordo; quindi $\grad \phi(x)$ non è nullo perché la $j$-esima componente è combinazione convessa di
	$\grad \psi_i(x)_k$ che sono non discordi.
\end{proof}

\begin{definition}
	Chiamiamo campo normale esterno la funzione $\nu:\partial\Omega\rightarrow \R^n$ tale che
	\[
		\nu(x)=\frac{\grad \phi(x)}{\lVert\grad \phi(x)\rVert}\punto
	\] %TODO: Ma chi cazzo è \phi?
\end{definition}

\begin{remark}
	Il \cref{lemma:EquivRegolare}, oltre che a fornire la funzione $\phi$ per l'aperto regolare $\Omega$, dice anche che $\nu(x)$ è ben definita
	su tutto $\partial \Omega$ e che è continua.
\end{remark}

\begin{remark}
	Si ha che per $t>0$ e $x\in\partial\Omega$, $x+t\nu(x)\notin \Omega$ per $t$ abbastanza piccoli e per $t<0$, $x+t\nu(x)\in \Omega$ per $t$
	abbastanza piccoli.
\end{remark}
\begin{proof}
	Stesso ragionamento fatto nella dimostrazione del \cref{lemma:EquivRegolare}.
\end{proof}

\begin{remark}
	Per ogni $v$ tangente a $\partial\Omega$ in $x$ si ha che $v\cdot\nu(x)=0$. Ma qui bisognerebbe spendere un paio di parole su cosa significhi
	essere tangente a $\partial\Omega$. %TODO: Questa nota, scritta così, è controproducente. O la si toglie, o la si scrive diversamente o si approfondisce un poco la questione. Così non si capisce se è lasciata qui per sbaglio o no
\end{remark}


Vogliamo ora formalizzare la definizione di integrale sulla superficie $\partial\Omega$ sfruttando anche i risultati della sezione precedente.

\begin{lemma}\label{lemma:GrapVar}
	Sia $V\in\R^{n-1}$ aperto e $f\in C^1(V,\R)$, allora il grafico di $f$ è una varietà differenziabile di dimensione $n-1$ immersa in $\R^n$.
\end{lemma}
\begin{proof}
	Sia $g:V\rightarrow \R^n$ data da $g(x)=(x,f(x))$. Ora è chiaro che graf$(f)$ è l'immagine di $g$; inoltre $g$ è banalmente di classe $C^1$
	e, se vista come funzione da $V$ in graf$(f)$ è anche un omeomorfismo. Inoltre $\Diff g(x)[h]=(h,\Diff f(x)[h])$, in particolare il differenziale è
	iniettivo in ogni punto, quindi $g$ è una immersione iniettiva, cioè è una parametrizzazione $C^1$ di graf$(f)$.
\end{proof}

\begin{remark}
	Dalla dimostrazione di questo lemma, dato che $\grad f(x)=\Diff f(x)$, segue immediatamente che $\Diff g(x)^T \Diff g(x)=
	(I \ \ \grad f(x)^T)
	\left(\begin{smallmatrix}
	I \\
	\grad f(x)
	\end{smallmatrix}\right)$, quindi per il \cref{lemma:DeterminanteMatriceQuasiIdentita} vale che $\det (\Diff g(x)^T \Diff g(x)) = 1+\lVert\grad f(x)\rVert^2$. 
\end{remark}

Il \cref{lemma:GrapVar} ci permette quindi di costruire un misura superficiale su graf$(f)$ in accordo con la \cref{def:MisuraKDimensionale}:
\begin{equation}\label{eq:MisuraSuperficialeGrafico}
	\sigma(E)=\int_{g^{-1}(E)} \sqrt{1+\lVert\grad f(x)\rVert^2} \de x\punto
\end{equation}


\begin{theorem}[Formula di Stokes]\label{thm:Stokes}
	Sia $u\in C^1(\overline\Omega)$ (ossia $u$ è la restrizione a $\overline\Omega$ di una funzione $C^1$ definita su un aperto
	$\Omega'\supseteq \overline\Omega$), sia $\nu:\partial \Omega \rightarrow \R^n$ la normale uscente da $\Omega$, allora vale che
	\begin{equation}\label{eq:FormulaStokes}
		\int_{\Omega} \Diff u(x)_j \de x = \int_{\partial \Omega} u(y)\nu(y)_j \de \sigma(y)\punto
	\end{equation}
\end{theorem}

\begin{proof}
	Per la dimostrazione si procederà in 2 step:
	\begin{enumerate}
		\item per ogni $p\in \overline\Omega$ esiste un intorno aperto rettangolare $U_p=\prod_{i=1}^n \oo{p_i-\varepsilon_i}{p_i+\varepsilon_i}$
			tale che l'\cref{eq:FormulaStokes} vale per ogni funzione $u$ tale che $\supp(u)\subseteq U_p$;
		\item $\overline\Omega$ è ricoperto con un numero finito di intorni come sopra, quindi si applica il \cref{thm:PartizioneUnita} per ottenere
			la tesi per ogni $u$.
	\end{enumerate}
	
	Dimostriamo quindi il primo punto, distinguendo i casi in cui $p\in \Omega$ e $p\in \partial \Omega$.
	Se $p\in \Omega$ esisterà un intorno rettangolare $U_p=V\times \oo{a}{b}$ contenuto in $\Omega$, poiché quest'ultimo è un aperto. Se ora $\supp(u)\subseteq U_p$,
	$u(\partial \Omega)=0$; quindi abbiamo che l'integrale a destra è $0$. Per l'integrale a sinistra, invece, abbiamo che una funzione $u$
	con supporto contenuto in $V\times \oo{a}{b}$ è tale che $u(x)=0$ per ogni
	$x\in \partial (V\times \oo{a}{b})$. Allora, per il \cref{thm:Fubini} e il teorema fondamentale del calcolo integrale,
	\begin{align*}
		\int_{\Omega} \Diff u(x)_n \de x &= \int_{V\times \cc{a}{b}} \Diff u(v,t)_n \de (v,t) = \\
		& = \int_V \int_{\cc{a}{b}} \Diff u(v,t)_n \de t \de v = \int_V (u(v,b) - u(v,a)) \de v = 0\punto
	\end{align*}
	Quindi abbiamo la tesi del primo punto quando $p\in\Omega$. Se invece $p\in \partial \Omega$, allora $p$ è un punto regolare, cioè esiste
	un intorno $U=V\times I$ di $p$ e una $\phi\in C^1(V,I)$ tale che
	\begin{itemize}
		\item $U\cap \partial \Omega = \{(z,s):s=\psi(z)\}$,
		\item $U\cap \Omega = \{(z,s):s<\psi(z)\}$.
	\end{itemize}
	Distinguiamo ora i casi $j=n$ e $j\neq n$.
	\begin{description}
		\item [$j=n$:] Dato che stiamo supponendo che $\supp(u)\subseteq U$, per il \cref{thm:Fubini} abbiamo che
			\begin{align*}
				\int_{\Omega} \Diff u(x)_n \de x  &= \int_{\Omega\cap U} \Diff u(v,t)_n\de x =
				\int_{U} \Diff u(v,t)_n \chi_{\{t<\psi(v)\}} \de (v,t) = \\
				& = \int_V \int_I \Diff u(v,t)_n \chi_{\{t<\psi(v)\}} \de t \de v =
				\int_V \int_{-\infty}^{\psi(v)} \Diff u(v,t)_n \de t \de v = \\
				& = \int_V (u(v,\psi(v)) - 0) \de v = \int_V u(v,\psi(v)) \de v\punto
			\end{align*}
			Ora abbiamo che la funzione $g:V\rightarrow \R^n$ data da $g(v)=(v,\psi(v))$ parametrizza l'ipersuperfice data dal grafico di
			$\psi$, quindi la misura superficiale è data dall' \cref{eq:MisuraSuperficialeGrafico}.
			Inoltre, detta $\nu:\partial \Omega \cap U \rightarrow \R^n$ la normale uscente da $\Omega$, sappiamo che
			$\nu(v,\psi(v))_n=(1+\psi(v)^2)^{-\frac{1}{2}}$. Quindi per definizione di integrale superficiale abbiamo che
			\begin{align*}
				\int_V u(v,\psi(v)) \de v &= \int_V u(v,\psi(v)) (1+\psi(v)^2)^{-\frac{1}{2}} (1+\psi(v)^2)^{\frac{1}{2}} \de v = \\
				& = \int_V u(v,\psi(v)) \nu(v,\psi(v))_n (1+\psi(v)^2)^{\frac{1}{2}} \de v = \\
				& = \int_{\partial \Omega \cap U} u(x) \nu(x)_n \de \sigma(x) = \\
				& = \int_{\partial \Omega} u(x) \nu(x)_n \de \sigma(x)\virgola
			\end{align*}
			dove nell'ultimo passaggio si è usato che $\supp(u)\subseteq U$.
		\item [$j\neq n$:] Possiamo assumere (a meno di riordinare le coordinate) che $j=n-1$ e, a meno di considerare un intorno più piccolo,
			che $V=W \times J$. Ora, per il \cref{thm:Fubini} abbiamo che
			\[
				\int_{\Omega} \Diff_{n-1} u(x)_{n-1} \de x = \int_W \int_J \int_I \Diff_s u(w,s,t) \de t \de s \de w\virgola
			\]
			inoltre, dato che $\Diff_s \int_X u(v,s)\de v = \int_X \Diff_s u(v,s)\de v$ e che $\Diff_y \int_a^y u(v,t)\de t=u(v,y)$,
			utilizzando le regole di composizione del differenziale, abbiamo che
			\[
				\Diff_s \int_{-\infty}^{\psi(w,s)} u(w,s,t) \de t =
				\int_{-\infty}^{\psi(w,s)} \Diff_s u(w,s,t) \de t + u(w,s,\psi(w,s))\Diff_s \psi (w,s)\punto
			\]
			Per cui, integrando il termine a sinistra su $V=W\times J=W\times \oo{a}{b}$, per il teorema fondamentale del
			calcolo integrale,
			\begin{align*}
				&\int_W \int_a^b \Diff_s \int_{-\infty}^{\psi(w,s)} u(w,s,t) \de t \de s \de w = \\
				&\int_W \left(\int_{-\infty}^{\psi(w,b)} u(w,b,t) \de t - \int_{-\infty}^{\psi(w,a)} u(w,a,t) \de t\right) \de w
				= 0\virgola
			\end{align*}
			poiché i punti $(w,b,t)$ e $(w,a,t)$ sono sul bordo di $U$, dove la funzione $u$ è nulla per ipotesi. Usando questa relazione
			otteniamo che
			\[
				\int_W\int_J\int_I\Diff_s u(w,s,t) \de t \de s \de w =-\int_W \int_J u(w,s,\psi(w,s))\Diff_s \psi (w,s)\de s\de w\punto
			\]
			Ora, dato che $\nu(x)_{n-1}=\nu(v,\psi(v))_{n-1}=-\Diff\psi(v)_{n-1}(1+\psi(v)^2)^{-\frac{1}{2}}$, otteniamo
			\begin{align*}
				\int_{\Omega} \Diff_{n-1} u(x)_{n-1} \de x & = -\int_W \int_J u(w,s,\psi(w,s))\Diff_s \psi (w,s) \de s \de w\\
				& = -\int_V u(v,\psi(v))\Diff_{n-1} \psi(v) \de v\\
				& = -\int_V u(v,\psi(v))(-\nu(v,\psi(v))_{n-1})(1+\psi(v)^2)^{\frac{1}{2}} \de v\\
				& = \int_{\partial \Omega \cap U} u(x)\nu(x)_{n-1} \de \sigma(x)\punto
			\end{align*}
	\end{description}
	Questo conclude il primo punto.
	
	Per la seconda parte procediamo in questo modo. Intanto, osserviamo che se $f\in C^1(\R^n)$ ha supporto $U$ tale che $U\cap \overline\Omega =
	\emptyset$, la \cref{eq:FormulaStokes} è banalmente vera in quanto entrabi i membri sono banalmente nulli.
	Consideriamo ora, per ogni punto $p\in \overline\Omega$, un intorno $U_p$ della forma richiesta al punto precedente. È chiaro che $\{U_p\}$ è un
	ricoprimento aperto del compatto $\overline\Omega$, quindi consideriamo un sottoricoprimento finito $U_1,\dots,U_n$; consideriamo anche
	$U_0=\R^n \setminus \overline\Omega$. Il \cref{thm:PartizioneUnita} ci fornisce delle funzioni $f_i\in C^{\infty}$, per $0\leq i \leq n$,
	tali che la loro somma è la funzione costante $1$.
	Sia ora $u\in C^1(\R^n,\R)$ una qualsiasi funzione. Otteniamo che
	\[
		u = \sum_{i=0}^n f_i\cdot u\virgola
	\]
	ma ora abbiamo che $f_i \cdot u$ (che è una funzione $C^1$) soddisfa l'\cref{eq:FormulaStokes} per ogni $i$; infatti, per $i=0$ è vero
	per quanto appena osservato, per $i>0$, la funzione $f_i \cdot u$ ha supporto contenuto in $U_p$ per qualche $p\in \overline\Omega$ e, per quanto
	dimostrato nel punto precedente, abbiamo che per questo intorno $U_p$ ogni funzione $C^1$ soddisfa l'\cref{eq:FormulaStokes}.
	Per linearità degli integrali e quindi dei membri che compaiono in questa equazione, è chiaro che le funzioni che soddisfano formano
	un sottospazio vettoriale di $C^1$. In particolare $u$ soddisfa la formula di Stokes.
\end{proof}

Un'immediata conseguenza del teorema appena dimostrato è la seguente:

\begin{corollary}(Teorema della divergenza)
	\label{cor:thDivergenza}
	Sia $u:\overline\Omega \rightarrow \R^n$ una funzione $C^1$ (nello stesso senso di prima), allora si ha che
	\begin{equation}\label{eq:thDivergenza}
		\int_{\Omega} \div u (x) \de x = \int_{\partial \Omega} u(y)\cdot\nu(y) \de \sigma(y)\puntovirgola
	\end{equation}
	dove l'operatore di divergenza è definito da $\div u(x)=\tr\Diff u(x)$.
\end{corollary}

\begin{proof}
	Utilizzando il \cref{thm:Stokes} sulla componente $j$-esima di $u$ che indichiamo con $u^j$ si ha che, per ogni $0\leq j\leq n$,
	\[
		\int_{\Omega} \Diff u^j(x)_j \de x = \int_{\partial \Omega} u^j(y)\nu(y)_j \de \sigma(y)\virgola
	\]
	da cui sommando otteniamo, per linearità degli integrali:
	\[
		\int_{\Omega} \sum_{j=0}^n\Diff u^j(x)_j \de x = \int_{\partial \Omega} \sum_{j=0}^n u^j(y)\nu(y)_j \de \sigma(y)\punto
	\]
	Ma ora, $\div u (x)= \tr\Diff u(x) = \sum_{j=0}^n \Diff u^j(x)_j$ e $u(x)\cdot\nu(x)=\sum_{j=0}^n u^j(y)\nu(y)_j$,
	quindi si ha la tesi.
\end{proof}

Giocherellando ancora con queste equazioni si ottengono anche le seguenti formule dette di Green:

\begin{corollary}(Prima formula di Green)
	Siano $u\in C^1(\overline\Omega, \R)$ e $v\in C^2(\overline\Omega, \R)$, allora si ha
	\[
		\int_{\Omega} \left( u \Delta v+ \grad u\cdot\grad v \right) \de x=
		\int_{\overline\Omega} u \frac{\partial v}{\partial \nu} \de \sigma(y) \virgola
	\]
	dove l'operatore $\Delta v = \sum_{j=1}^n \Diff_{jj} v$ %TODO: Questa definizione del Laplaciano o è sbagliata oppure usa una notazione tipo quella di gtd... in ogni caso non si capisce.
	è il laplaciano e $\frac{\partial v}{\partial \nu}(x)=\Diff v(x)[\nu]$.
\end{corollary}
\begin{proof}
	Dato che abbiamo 
	\begin{equation*}
		\Diff_j (v(x)\Diff u(x)_j ) = \Diff v(x)_j \Diff u(x)_j + v(x)\Diff_{jj} u(x)\virgola
	\end{equation*} %TODO: Stesso problema dell'enunciato... non si capisce cosa vuol dire questa notazione compatta.
	applicando il \cref{thm:Stokes} sulle funzioni $v(x)\Diff u(x)_j$ e sommando su $j$ otteniamo la tesi. %TODO: Troppo conciso.
\end{proof}

\begin{corollary}(Seconda formula di Green)
	Siano $u,v\in C^2(\overline\Omega, \R)$ allora si ha che
	\[
		\int_{\Omega}\left(u\Delta v-v\Delta u\right) \de x=
		\int_{\overline\Omega}u\frac{\partial v}{\partial\nu}-v\frac{\partial u}{\partial\nu} \de \sigma(y)\punto
	\]
\end{corollary}

\begin{proof}
	Immediata conseguenza della prima formula applicata a $u,v$ e a $v,u$; poi basta sottrarre membro a membro.
\end{proof}




\printindex

\end{document}

\makeindex