\documentclass[a4paper,12pt]{article}
\usepackage{stilebase}

\title{Dispense sulla distanza di Hausdorff}
\author{Fabio Ferri \and Giada Franz \and Federico Glaudo}

\begin{document}

\maketitle
% \clearpage


\begin{abstract}
	In questo documento studieremo le proprietà della distanza di Hausdorff, la naturale distanza indotta sullo spazio dei sottoinsiemi (ci ridurremo ai chiusi e limitati nel nostro studio) di uno spazio metrico. 
	
	Nelle prime sezioni ci concentremo sulle proprietà della distanza di Hausdorff sui chiusi e limitati, in particolare su quali proprietà del metrico di base (completezza, totale limitatezza, compattezza) vengano ereditate da questo spazio. 
	Invece nell'ultima sezione evidenzieremo come i risultati ottenuti per i chiusi e limitati si estendono senza grande sforzo anche ai chiusi e \emph{totalmente} limitati (che sono \emph{strettamente} collegati ai compatti).
\end{abstract}
\clearpage

% \tableofcontents
% \clearpage

\section{Definizioni e risultati introduttivi}
In questa sezione definiremo la distanza di Hausdorff fra i chiusi e limitati e mostreremo qualche risultato introduttivo.

In tutto il corso della trattazione indicheremo con $(X,d)$ uno spazio metrico generico.

\begin{definition}
	Sia $\mathcal{K}(X)=\{K\in \mathcal{P}(X) : K \text{ è compatto}\}$ l'insieme dei compatti di $X$.
\end{definition}

\begin{definition}
	Dato $A\in \mathcal{K}(X)$, definiamo $d_A: X\to [0,+\infty)$ tale che $d_A(x)=\inf_{a\in A}d(a,x)$ per ogni $x\in X$.
\end{definition}

\begin{remark}
	$d_A({}\cdot{})$ è effettivamente una funzione a valori in $[0,+\infty)$ perchè vale $\inf_{a\in A}d(a,x)<\infty$, poichè $A$ essendo compatto è limitato.
\end{remark}

\begin{lemma}\label{DistanzaCompattoRealizzata}
	Per ogni $x\in X$ e $A \in \mathcal{K}(X)$, esiste $a\in A$ tale che $d_A(x)=d(a,x)$.
\end{lemma}
\begin{proof}
	Poichè $d_A(x)=\inf_{a\in A} d(a,x)$, esiste $(a_n)$ a valori in $A$ tale che $\lim_{n\to\infty}d(a_n,x)=d_A(x)$. Dato che $A$ è un compatto esiste una sottosuccessione $(a_{n_k})$ di $(a_n)$ convergente ad $a\in A$. Tale sottosuccessione rispetta quindi che $a_{n_k}\to a$ e $d(a_{n_k},x)\to d_A(x)$, da cui facilmente $d(a,x)=d_A(x)$, poichè $d(a_{n_k},x)\to d(a,x)$.
\end{proof}
\begin{remark}\label{DistanzaCompattoAppartenenza}
	Dati $x\in X$ e $A \in \mathcal{K}(X)$, $d_A(x)=0$ se e solo se $x\in A$. Infatti per \cref{DistanzaCompattoRealizzata} esiste $a\in A$ tale che $d_A(x)=d(a,x)$, da cui $d_A(x)=0$ se e solo se $a=x$.
\end{remark}

\begin{definition}\label{DistanzaFraCompatti}
	Sia $\delta_A:\mathcal{K}(X)\to [0,+\infty)$ tale che $\delta_A(B)=\sup_{b\in B} d_A(b)$.
\end{definition}
\begin{remark}
	Anche in questo caso $\delta_A(B)$ ha valori in $[0,+\infty)$ perchè $A\cup B$ è limitato.
\end{remark}

\begin{definition}[Distanza di Hausdorff]\label{HausdorffDefinizione}
	Definiamo infine $\delta:\mathcal{K}(X)\times \mathcal{K}(X) \to [0,+\infty)$ come $\delta(A,B)=\max\{ \delta_A(B),\delta_B(A) \}$.
\end{definition}

\begin{theorem}
	La funzione definita in \cref{HausdorffDefinizione} è una distanza chiamata distanza di Hausdorff.
\end{theorem}

\begin{proof}
	Dimostriamo che $\delta({}\cdot{},{}\cdot{})$ è veramente una distanza.
	\begin{itemize}
		\item $\delta(A,B)=\delta(B,A)$ (simmetria)
		
		$\delta(A,B)=\max\{ \delta_A(B),\delta_B(A) \}=\delta(B,A)$.
		\item $\delta(A,B)=0 \iff A=B$
		
		Se $A=B$ vale banalmente $\delta(A,B)=0$; se invece esiste $a\in A$ tale che $a\not\in B$ ho che $\delta(A,B)\ge \delta_B(A)\ge \delta_B(a)>0$, dove l'ultima disuguaglianza è vera per \cref{DistanzaCompattoAppartenenza}.
		\item $\delta(A,C)\le \delta(A,B)+\delta(B,C)$ (disuguaglianza triangolare)
		
		Dimostro innanzitutto che per ogni $c\in C$ vale $d_A(c)\le \delta(A,B)+\delta(B,C)$.
		
 		Per \cref{DistanzaCompattoRealizzata} ho che esiste $a\in A$ tale che $d_A(c)=d(a,c)$ e che esiste $b\in B$ tale che $d_B(c)=d(b,c)$. Analogamente esiste $a'\in A$ tale che $d_A(b)=d(a',b)$. Utilizzando facili conseguenze delle definizioni date precedentemente, ho quindi che 
 		\begin{equation*}
 			d_A(c)=d(a,c)\le d(a',c) \le d(a',b)+d(b,c)=d_A(b)+d_B(c)\le \delta(A,B)+\delta(B,C)
 		\end{equation*}
 		
 		Passando ora al $\sup$ su $c\in C$ in quest'ultima disuguaglianza ottengo che $\delta_A(C)\le \delta(A,B)+\delta(B,C)$, ma del tutto  analogamente vale $\delta_C(A)\le \delta(A,B)+\delta(B,C)$, quindi
 		\begin{equation*}
 			\delta(A,C)=\max\{ \delta_A(B),\delta_B(A) \}\le \delta(A,B)+\delta(B,C)
 		\end{equation*}
		che è proprio la disuguaglianza triangolare.
	\end{itemize}
	Inoltre $\delta$ ha valori in $[0,+\infty)$ poichè per ogni $A,B\in \mathcal{K}(X)$ $\delta(A,B)<\infty$, dato che $A$ e $B$ sono limitati.
\end{proof}

\begin{remark}
	Del tutto analogamente si dimostra che la distanza di Hausdorff è una distanza anche sui chiusi e limitati, poichè anche in questo caso vale \cref{DistanzaCompattoAppartenenza}.
\end{remark}

\begin{definition}[Definizione equivalente della distanza di Hausdorff] \label{HausdorffDefinizioneEquivalente}
	La distanza di Hausdorff si può definire in modo equivalente nel seguente modo. Sia $\delta_A'(B)=\inf\{r:B\subseteq U_r(A)\}$, dove $U_r(A)=\{x\in X : d_A(x)\le r \}$. Allora definisco $\delta'(A,B)=\max\{\delta_A'(B),\delta_B'(A)\}$
\end{definition}

\begin{theorem}
	Le definizioni \cref{HausdorffDefinizione} e \cref{HausdorffDefinizioneEquivalente} sono equivalenti.
\end{theorem}
\begin{proof}
	Dimostro in particolare che $\delta_A(B)=\delta_A'(B)$, dove $\delta_A$ e $\delta_A'$ sono definite rispettivamente in \cref{DistanzaFraCompatti} e \cref{HausdorffDefinizioneEquivalente}, poichè da questo segue banalmente la tesi.
	
	Chiamo $P=\{ r : \exists b\in B \text{ tale che }d_A(b)=r \}$ e $Q=\{ r: \forall b\in B\text{ vale } d_A(b)\le r \}$, allora si nota facilmente che per ogni $p\in P$ e $q\in Q$ vale $p\le q$ e inoltre per ogni $x\in X$ ho che $x\in P$ o $x\in Q$. Da questo segue facilmente che $\sup\{ r: r\in P \}=\inf\{r: r\in Q \}$.
	
	Vale quindi che
	\begin{equation*}
		\delta_A(B)=\sup_{b\in B} d_A(b)=\sup\{ r : r\in P \}=\inf\{ r : r\in Q \}= \inf\{ r : B\subseteq U_r(A) \}=\delta_A'(B)
	\end{equation*}
	che è quello che volevo dimostrare
\end{proof}

\begin{lemma}\label{IsometriaCanonica}
	Esiste un'isometria canonica $\varphi: X\in \mathcal{K}(X)$, tale che $\varphi(X)$ è un chiuso in $\mathcal{K}(X)$.
\end{lemma}
\begin{proof}
	Definisco l'isometria $\varphi: X\in\mathcal{K}(X)$ tale che $\varphi(x)=\{x\}$ per ogni $x\in X$. 
	
	Innanzitutto vale banalmente che questa è un'isometria, perchè segue facilmente dalle definizioni che $\delta(\{ x \}, \{ x' \})=d(x,x')$.
	
	Dimostriamo ora che $\varphi(X)$ è un chiuso. Sia $(\{ x_n \})_{n\in \mathbb{N}}$ una successione convergente in $\mathcal{K}(X)$, allora $(x_n)_{n\in \mathbb{N}}$ converge in $X$ ad un valore $x$ (poichè $\varphi$ è un isometria). Voglio mostrare che allora $(\{ x_n \})_{n\in \mathbb{N}}$ converge a $\{ x \}$, ma questo è ovvio perchè $\lim_{n\to\infty} \delta(\{ x_n \}, \{x\})=\lim_{n\to\infty} d(x,x')=0$. Questo conclude la dimostrazione per l'unicità del limite.
\end{proof}










\section{Trasmissione della completezza}



In questa sezione dimostrerò che se $\left (X,d\right )$ è completo allora lo è anche $\left (\mathcal{K}(X),\delta\right  ) $.


\begin{lemma} \label {pereppeppeppe}
Sia $K_n$ una successione di compatti in $\left (X,d\right )$ tali che $K_{n+1}\subseteq K_n$ $ \forall n \in \mathbb{N}$. Allora la successione ammette limite in $\left (\mathcal{K}(X),\delta\right  ) $ uguale a $ \bigcap_{n \in \mathbb{N}}K_n $.
\end{lemma}

\begin{proof}
Dimostro innanzitutto che l'intersezione dei compatti non è vuota. Considero una successione $\left (x_n\right ) \subseteq X$ tale che $x_n \in K_n$ $ \forall n \in \mathbb{N}$. In particolare  $x_n \in K_1$ $ \forall n \in \mathbb{N}$ e dunque esiste una sottosuccessione  $\left (x_n(k)\right )$ convergente a $x\in K_1$. Per ogni $n$  $x_{n(k)}\in K_n$ definitivamente, ergo $x\in K_n$ per la chiusura in $X$ di $K_n$. Di conseguenza $x \in K_n$ $ \forall n \in \mathbb{N}$, il che implica che l'intersezione dei compatti è non vuota.

Dato che per ogni $n$ $K_n\supseteq K$ allora ovviamente $\delta\left  (K_n,K\right )=\delta_{K}\left (K_n\right )$. Inoltre $\delta_{K}\left (K_n\right )$ è decrescente in $n$, quindi $ \forall n \in \mathbb{N}$ $\exists x_n \in K_n$ tale che $d_K\left (X_n\right )\geq r$ , dove 
\begin{equation*}
r=\lim_{n \to \infty}\delta_{K}\left (K_n\right )>0.
\end{equation*}

Dunque esiste una sottosuccessione $x_{n(k)}$ convergente a $x\in K_1$, ovvero, per lo stesso argomento precedente, a  $x\in K$. Ma allora 
\begin{equation*}
\lim_{n(k) \to \infty}d(x_{n(k)},x)=0,
\end{equation*}
contraddicendo il fatto che
\begin{equation*}
\lim_{n(k) \to \infty}\delta_{K}(K_{n(k)})=r>0.
\end{equation*}
\end{proof}

\begin{theorem}
Sia $(K_n)$ una successione di Cauchy in $\left (\mathcal{K}(X),\delta\right  ) $, dove $\left (X, d\right )$ è uno spazio metrico completo. Allora $\forall k \in \mathbb{N}$ vale che
\begin{equation*}
B_k:=\overline{\bigcup_{n\geq k} K_n}
\end{equation*}
è compatto.
\end{theorem}

\begin{proof}
In uno spazio metrico la compattezza per successioni è equivalente a completezza e totale limitatezza. Dunque mi basta dimostrare quest'ultima proprietà. 

Dato che lo spazio metrico è completo e $B_k$ è chiuso, allora $B_k$ è anche completo.

Ora mi resta da dimostrare che, dato $\varepsilon>0$, esiste un insieme finito di palle di raggio $\varepsilon$ che ricopre tutto $B_k$. Dato che i compatti formano una successione di Cauchy, esiste $n_0$ tale che $\delta\left (K_{n_0}, K_n\right )< \frac{\varepsilon}{4}$. Inoltre esiste un numero finito di palle di raggio $\frac{\epsilon}{4}$ che ricopre $K_{n_0}$. Ma allora se raddoppio il raggio delle palle considerate ricopro $\cup_{n\geq n_0} K_n$; infatti per ogni $x \in K_n$ c'è un elemento di $K_{n_0}$ a distanza minore di $\frac{\varepsilon}{4}$, il quale vede un centro $y$ di almeno una delle palle a distanza minore di $\frac{\varepsilon}{4}$, da cui $d(x,y)<\frac{\varepsilon}{2}$.

Inoltre mi basta un numero finito di palle di raggio $\frac{\varepsilon}{2}$ per ricoprire $\cup_{k \leq n< n_0} K_n$, visto che sono un numero finito di compatti. Ora se raddoppio il raggio di tutte le palle di sicuro mi vado a prendere anche la chiusura di $\cup_{n\geq k} K_n$, e dunque ho trovato un ricoprimento finito di palle di raggio $\varepsilon$.
\end{proof}

Dato che per ogni $k\in \mathbb{N}$ $B_{k+1}\subseteq B_k$ per il teorema appena dimostrato e per il \cref{pereppeppeppe} esiste $B=\lim_{k \to \infty} B_k$.

Adesso dimostro che la successione di compatti ha limite, mostrando che esso non è altro che $B$.

\begin{theorem}
Sia $\left (K_n\right )$ una successione di Cauchy in $\left (\mathcal{K}(X),\delta \right ) $, dove $\left (X, d\right )$ è uno spazio metrico completo. Allora $\lim_{n \to \infty}K_n=B$, ove
\begin{equation*}
B=\lim_{k \to \infty} \left(\overline{\bigcup_{n\geq k} K_n} \right).
\end{equation*}
\end{theorem}

\begin{proof}
Dato $\varepsilon$, esiste $n_0$ tale che $\delta\left  (K_{n_0}, K_n\right )< \varepsilon$ $\forall n\geq n_0$, il che implica $\delta _{K_{n_0}}\left  (K_n\right )<\varepsilon$ $\forall n\geq n_0$. Ma allora vale anche che $\delta _{K_{n_0}}\left  (\cup_{n\geq n_0} K_n\right )<\varepsilon$, e anche che $\delta _{K_{n_0}}\left  (\overline {\cup_{n\geq n_0} K_n} \right ) \leq \varepsilon$. Dunque, dato che per tutti gli $n$ $K_n \subseteq B_n$,  definitivamente $\delta\left  (K_n, B_n\right ) \leq \varepsilon$. Ma allora accade anche che $\delta\left  (K_n, B\right ) \to 0$.
\end{proof}

Dunque in uno spazio metrico completo $\left (X,d\right )$ vale effettivamente che una successione di Cauchy di compatti converge ad un compatto, ovvero che $\left (\mathcal{K}(X),\delta \right ) $ è a sua volta completo.


Si può dimostrare anche il viceversa:
\begin{theorem}
Sia  $\left (X,d\right )$ uno spazio metrico e $\left (\mathcal{K}(X),\delta \right ) $ lo spazio metrico dei compatti. Allora se $\left (\mathcal{K}(X),\delta \right ) $ è completo lo è anche $\left (X,d\right )$.
\end{theorem}
\begin{proof}
Sia $\left( x_n \right)$ una successione di Cauchy di $\left (X,d\right )$. Dato che $\forall y,z \in X$ vale che $d\left( y,z\right)=\delta \left(\left \{y\right \},\left \{z \right \} \right)$, allora anche $\left(\left \{ x_n\right \} \right)\subseteq \mathcal{K}(X)$ è una successione di Cauchy in  $\left (\mathcal{K}(X),\delta \right ) $. Allora per ipotesi la successione dei singoletti tende ad un compatto, che può essere soltanto a sua volta un singoletto $\left \{x \right \}$. Ma allora, dato che $\delta \left ( \left \{ x_n\right \},\left \{x \right \} \right ) \to 0$, è anche vero che $d\left (  x_n,x  \right ) \to 0$.
\end{proof}

\section{Trasmissione della compattezza}
In questa sezione dimostreremo che se $X$ è un compatto allora anche $\CL(X)$ lo è. 

In particolare proveremo che la proprietà di totale limitatezza di $X$ viene ereditata da $\CL(X)$ e quindi, sfruttando il \cref{CompletezzaCL}, anche la compattezza viene ereditata.

\begin{lemma}\label{TotaleLimitatezzaCL}
	Se $(X,d)$ è uno spazio metrico totalmente limitato, allora anche $(\CL(X),\delta)$ lo è.
\end{lemma}
\begin{proof}
	Fissato $\varepsilon>0$, per l'ipotesi di totale limitatezza di $X$, esiste un insieme $S\subseteq X$ finito, tale che:
	\begin{equation}\label{TotaleLimitatezzaX}
		X=\bigcup_{s\in S} B_{\varepsilon}(s) \Longrightarrow 
		\forall x\in X\ \exists s\in S:\ d(x,s)\le \varepsilon
	\end{equation}

	Ora consideriamo $\mathcal{R}=\mathcal{P}(S)$. Questo è un insieme finito di sottoinsiemi finiti (quindi chiusi e limitati) di $X$, perciò è un sottoinsieme finito di $\CL(X)$. 
	
	Dimostriamo che fissato $K\in\CL(X)$ esiste $R\in \mathcal{R}$ tale che $\delta(K,R)\le \varepsilon$, e questo è equivalente alla tesi di totale limitatezza di $\CL(X)$. In particolare la scelta di $R$ è costruttiva, visto che poniamo:
	\begin{equation*}
		R=\{s\in S\ |\ d_K(s)\le \varepsilon\}
	\end{equation*}
	
	Sfruttando la sola definizione di $R$ otteniamo che vale:
	\begin{equation}\label{DeltaKR}
		\delta_K(R)=\sup_{r\in R} d_K(r) \le \varepsilon
	\end{equation}
	
	Fissato $k\in K$ per l'\cref{TotaleLimitatezzaX} esiste $s\in S$ tale che $d(s,k)\le \varepsilon$. Di conseguenza risulta:
	\begin{equation*}
		d_K(s)\le d(s,k) \le \varepsilon \Longrightarrow s\in R \Longrightarrow d_R(k)\le d(s,k)\le \varepsilon
	\end{equation*}
	dove nella prima implicazione abbiamo usato la definizione di $R$. 
	Sfruttando l'ultima disuguaglianza mostrata è ovvio ottenere:
	\begin{equation}\label{DeltaRK}
		\delta_R(K)=\sup_{k\in K} d_R(k) \le  \varepsilon
	\end{equation}
	
	Unendo le \cref{DeltaKR,DeltaRK} e applicando la definizione di $\delta({}\cdot{},{}\cdot{})$ otteniamo:
	\begin{equation*}
		\delta(K,R)=\max\left(\delta_K(R),\delta_R(K)\right)\le \varepsilon
	\end{equation*}
	che è quanto volevamo e, come già annunciato, dimostra la totale limitatezza di $\CL(X)$.
\end{proof}

\begin{remark}\label{TotaleLimitatezzaInverso}
	Vale anche l'implicazione inversa di \cref{TotaleLimitatezzaCL}, cioè che se  $(\CL(X),\delta)$ è totalmente limitato, allora anche $(X,d)$ lo è.
\end{remark}
\begin{proof}
	Basta ricordare che per il \cref{IsometriaCanonica} esiste un'isometria tra $X$ e un sottoinsieme di $\CL(X)$ e che la proprietà di totale limitatezza di $\CL(X)$ si trasmette ad ogni suo sottoinsieme.
\end{proof}



\begin{theorem} \label{CompattezzaCL}
	Se $(X,d)$ è uno spazio metrico compatto, allora anche $(\CL(X),\delta)$ lo è.
\end{theorem}
\begin{proof}
	Se $X$ è compatto allora è in particolare anche completo e totalmente limitato, perciò risultano verificate le ipotesi di \cref{CompletezzaCL,TotaleLimitatezzaCL}. Applicando tali risultati otteniamo che $\CL(X)$ è completo e totalmente limitato, quindi è compatto e il teorema è dimostrato.
\end{proof}

\begin{remark} \label{CompattezzaInverso}
	Vale anche l'implicazione inversa del \cref{CompattezzaCL}, cioè che se  $(\CL(X),\delta)$ è compatto allora anche $(X,d)$ lo è.
\end{remark}
\begin{proof}
	Essendo $\CL(X)$ un compatto è in particolare completo e totalmente limitato, perciò sono verificate le ipotesi delle \cref{CompletezzaInverso,TotaleLimitatezzaInverso}. Applicando quindi tali risultati otteniamo che $X$ è completo e totalmente limitato, quindi è compatto.
\end{proof}


\section{Proprietà dei chiusi e \emph{totalmente} limitati}
In questa parte conclusiva mostriamo che quanto già dimostrato per lo spazio $\CL(X)$ vale anche per lo spazio dei chiusi e totalmente limitati. Infine facciamo notare come le proprietà principali vengano ereditate anche dallo spazio dei compatti di $X$.

\begin{definition}
	Sia $\CTL(X)$ lo spazio dei chiusi e totalmente limitati di $X$.
\end{definition}
\begin{remark}
	Ovviamente risulta $\CTL(X)\subseteq \CL(X)$
\end{remark}



\end{document}

\makeindex