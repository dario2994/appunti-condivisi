\documentclass[a4paper,12pt]{article}
\usepackage{stilebase}
% \usepackage{float}
% \usepackage{figure}

\title{Dispense sulla distanza di Hausdorff}
\author{Fabio Ferri \and Giada Franz \and Federico Glaudo}

\begin{document}

\maketitle
% \clearpage


\begin{abstract}
	In questo documento studieremo le proprietà della distanza di Hausdorff, la naturale distanza indotta sullo spazio dei sottoinsiemi (ci ridurremo ai chiusi e limitati nel nostro studio) di uno spazio metrico. 
	
	Nelle prime sezioni ci concentremo sulle proprietà della distanza di Hausdorff sui chiusi e limitati, in particolare su quali proprietà del metrico di base (completezza, totale limitatezza, compattezza) vengano ereditate da questo spazio. 
	Invece nella penultima sezione evidenzieremo come i risultati ottenuti per i chiusi e limitati si estendano senza grande sforzo anche ai chiusi e \emph{totalmente} limitati (che sono strettamente collegati ai compatti).
	Infine nell'ultima sezione sposteremo l'attenzione sulle proprietà di connessione nella distanza di Hausdorff ed in particolare sulla conservazione della connessione per passaggio al limite.

	Per sviluppare questo argomento ci siamo basati sugli esercizi proposti durante l'esercitazione del corso di Complementi di Matematica. In particolare ci siamo riferiti agli esercizi dal \texttt{6.20} al \texttt{6.25} del documento \url{http://cvgmt.sns.it/HomePages/cm/Ianno2014/esercitazioni.pdf}.
\end{abstract}
\clearpage

% \tableofcontents
% \clearpage

\section{Definizioni e fatti introduttivi}
In questa sezione definiremo tutti gli oggetti necessari per fondare la teoria della misura e dimostreremo alcuni fatti, perlopiù di carattare insiemistico su di essi.

In particolare definiremo alcune strutture insiemistiche (algebre, \sigalg[e], \semiring[i]) e ne dimostreremo alcune proprietà. Poi passeremo a trattare gli spazi di misura ed alcune loro \emph{versioni più deboli} come le misure esterne e le premisure.

L'idea generale di queste pagine è dare gli strumenti necessari a comprendere il teorema di estensione di una premisura, nella cui dimostrazione ed enunciato convergono tutti i fatti qui trattati. 

Da notare infine che la definizione di \semiring{} non è quella canonica (nè forse si può parlare di una definizione canonica in letteratura) ed anzi è più debole di quella tipicamente usata per enunciare il teorema di estensione. Questo rende più generale il risultato, ma allo stesso tempo rende più ostico dimostrare le prime proprietà della premisura che infatti necessiteranno di vari lemmi tecnici per essere dimostrate (e sarebbero banali se al posto di un \semiring{} ci fosse un anello).

\begin{definition}[Algebra]
	Dato un insieme $X$, una famiglia $\mathcal A\subseteq\mathcal P(X)$ è un'algebra se valgono:
	\begin{itemize}
		\item $\emptyset\in\mathcal A$
		\item $\forall A\in\mathcal A:\ A^c\in\mathcal A$ cioè un'algebra è stabile per passaggio al complementare.
		\item $\forall A,B\in\mathcal A:\ A\cup B\in\mathcal A$ cioè un'algebra è stabile per unioni finite.
	\end{itemize}
\end{definition}
\begin{remark}\label{ProprietaAlg}
	Un'algebra è stabile anche per intersezioni finite e per differenza insiemistica.
\end{remark}
\begin{proof}
	Poichè vale la formula insiemistica:
	\begin{equation*}
		\bigcap_{i\in I} A_i = \left( \bigcup_{i\in I} A_i^c \right)^c
	\end{equation*}
	e un'algebra è stabile per unione finita e complementare, facilmente risulta esserlo anche per intersezioni finite.
	
	Per la differenza si sfrutta la seguente relazione insiemistica $A\setminus B=A\cup B^c$. Questa porta a concludere visto che abbiamo appena dimostrato che $\mathcal A$ è stabile per intersezioni.
\end{proof}


\begin{definition}[\sigalg{}]
	Dato un insieme $X$, una famiglia $\mathcal A\subseteq\mathcal P (X)$ si dice \sigalg{} se valgono:
	\begin{itemize}
	\item $\emptyset\in \mathcal A$
	\item $\forall A\in \mathcal A:\ A^c\in \mathcal A$ cioè una \sigalg{} è stabile per passaggio al complementare.
	\item $\forall (A_n)_{n\in\mathbb N}\subseteq \mathcal A:\ \bigcup_{n\in\mathbb N} A_n\in \mathcal A$ cioè una \sigalg{} è stabile per unioni numerabili.  
	\end{itemize}
\end{definition}

\begin{remark}\label{ProprietaSigAlg}
	Una \sigalg{} è stabile anche per intersezioni numerabili e per differenza insiemistica.
\end{remark}
\begin{proof}
	Si dimostrano entrambe le proprietà in modo del tutto analogo a come abbiamo dimostrato \cref{ProprietaAlg}.
\end{proof}

\begin{definition}[\Semiring{}]
	Una famiglia $\mathcal S\subseteq \mathcal P(X)$ è detta \semiring{} se valgono le seguenti proprietà:
	\begin{itemize}
		\item $\emptyset\in \mathcal S$
		\item $\displaystyle\forall A,B\in \mathcal S: A\cap B, A\setminus B\in \sqcup \mathcal S$ dove
		$\displaystyle
		\sqcup{ \mathcal S }=\left\{\bigsqcup_{n\in \mathbb N} S_n\ |\ (S_n)_{n\in\mathbb N} \subseteq \mathcal S \wedge \forall i\not= j:\ S_i\cap S_j=\emptyset\right\}$ 
		cioè un \semiring{} non deve essere stabile per intersezione e differenza, ma queste si devono scrivere come unioni disgiunte.
	\end{itemize}
\end{definition}

\begin{proposition}\label{UnioneDisgiuntaQuasiAlgebra}
	Dato un \semiring{} $\mathcal S$ l'insieme $\sqcup\mathcal S$ è stabile per intersezione finita e unione numerabile.
\end{proposition}
\begin{proof}
	Notiamo intanto che, per definizione, $\sqcup\mathcal S$ è stabile per unione disgiunta numerabile.
	
	Per dimostrare la stabilità di $\sqcup\mathcal S$ per intersezione finita, basta ovviamente farlo per due soli insiemi $A,B\in\sqcup\mathcal S$. Per definizione possiamo scrivere $A=\bigsqcup_{n\in\mathbb N} A_n, B=\bigsqcup_{n\in\mathbb N} B_n$ dove $(A_n)_{n\in\mathbb N},(B_n)_{n\in\mathbb N}$ sono successioni in $\mathcal S$. Allora vale la seguente identità:
	\begin{equation*}
		A\cap B=\bigsqcup_{n\in\mathbb N} A_n\cap\bigsqcup_{n\in\mathbb N} B_n=
		\bigsqcup_{n,m\in\mathbb N} A_n\cap B_m\in\sqcup\mathcal S
	\end{equation*}
	dove nell'ultimo passaggio è stata usata la stabilità di $\sqcup\mathcal S$ per unione disgiunta.
	
	Per l'unione, consideriamo $A,B\in\mathcal S$. Poichè vale $A\cup B=(A\setminus B)\sqcup(A\cap B)$, viste le proprietà di un \semiring{}, risulta $A\cup B\in \sqcup\mathcal S$. Da questo è facile ottenere che anche unioni finite di elementi di $\mathcal S$ appartengono a $\sqcup\mathcal S$.
	
	Sfruttando quanto detto, fissata $(A_n)_{n\in\mathbb N}\subseteq\mathcal S$ vale:
	\begin{equation}\label{UnioneNumerabileDaS}
		\bigcup_{n\in\mathbb N} A_n=\bigsqcup_{n\in\mathbb N} A_n\setminus\cup_{i<n} A_i
		=\bigsqcup_{n\in\mathbb N} \bigcap_{i<n} (A_n\setminus A_i)\in\sqcup\mathcal S
	\end{equation}
	dove nell'ultimo passaggio abbiamo applicato la stabilità di $\sqcup\mathcal S$ per unione disgiunta e intersezione finita.
	
	E infine dimostriamo la stabilità di $\sqcup\mathcal S$ per unioni numerabili. Sia $(S_n)_{n\in\mathbb N}$ una successione in $\sqcup\mathcal S$. Per definizione devono esistere le successioni $(A^n_i)_{i\in\mathbb N}\subseteq \mathcal S$ tali che $S_n=\bigsqcup_{i\in\mathbb N} A^n_i$.
	
	Allora applicando \cref{UnioneNumerabileDaS} otteniamo:
	\begin{equation*}
		\bigcup_{n\in\mathbb N}S_n=\bigcup_{i,n\in\mathbb N}A^n_i\in\sqcup\mathcal S 
	\end{equation*}
	che è proprio la stabilità di $\sqcup\mathcal S$ per unioni numerabili.
\end{proof}

\begin{definition}[{\sigadd[ità]}]
	Una funzione $\mu:\mathcal F\to \Rpiu$, dove $\mathcal F$ è una famiglia di insiemi, si dice \sigadd{} se per ogni sottofamiglia numerabile $(F_n)_{n\in\mathbb N}\subseteq \mathcal F$ a due a due disgiunta, tale che l'unione appartiene a $\mathcal F$, vale l'addittività:
	\begin{equation*}
		\mu\left(\bigcup_{n\in\mathbb N}F_n \right)=\sum_{n\in\mathbb N} \mu(F_n) 
	\end{equation*}
\end{definition}
\begin{remark}
	Data $\mu:\mathcal F\to \Rpiu$ \sigadd{}, se $\emptyset\in \mathcal F$ allora $\mu(\emptyset)=0$
\end{remark}
\begin{proof}
	Usando la proprietà di \sigadd[ità] si ottiene $\mu(\emptyset)=\mu(\emptyset)+\mu(\emptyset)$ che porta ovviamente alla tesi.
\end{proof}


\begin{definition}[Spazio di misura]
	Dati $X$ un insieme, $\mathcal A$ una famiglia di sottoinsiemi di $X$ e $\mu:\mathcal A\to \Rpiu$ una funzione, la terna $(X,\mathcal A, \mu)$ si dice uno spazio di misura se:
	\begin{itemize}
		\item la famiglia $\mathcal A$ è una \sigalg{}.
		\item la funzione $\mu$ è \sigadd{}.
	\end{itemize}
	e in questo caso la funzione $\mu$ è detta \emph{misura}.
\end{definition}

D'ora in poi, quando ci si riferirà ad una misura, si darà per scontato che questa si riferisce ad uno spazio di misura.

\begin{remark}\label{MonotoniaMisura}
	Dato $(X,\mathcal A,\mu)$ uno spazio di misura, $\mu$ è monotona, cioè se $A,B\in\mathcal A$ e $A\subseteq B$ allora $\mu(A)\le \mu(B)$.
\end{remark}
\begin{proof}
	Per quanto detto in \cref{ProprietaSigAlg}, $B\setminus A\in\mathcal A$ e perciò sfruttando l'addittività su insiemi disgiunti di $\mu$ otteniamo $\mu(B)=\mu(B\setminus A)+\mu(A)>\mu(A)$ che è la tesi.
\end{proof}
\begin{remark}\label{SubAdditivitaMisura}
	Dato $(X,\mathcal A,\mu)$ uno spazio misurato, la misura $\mu$ è \sigsubadd{}, cioè dati $(A_n)_{n\in\mathbb N}\subseteq\mathcal A$ risulta:
	\begin{equation*}
		\mu\left(\bigcup_{n\in\mathbb N} A_n \right)\le \sum_{n\in\mathbb N}\mu(A_n)
	\end{equation*}
\end{remark}
\begin{proof}
	La dimostrazione risulta molto facile sfruttando la monotonia, mostrata in \cref{MonotoniaMisura}, e la solita decomposizione dell'unione in un'unione disgiunta che permette di applicare la \sigadd[ità]:
	\begin{equation*}
		\mu\left(\bigcup_{n\in\mathbb N} A_n \right)=\mu\left(\bigsqcup_{n\in\mathbb N} A_n\setminus\bigcup_{i<n}A_i \right)=
		\sum_{n\in\mathbb N}\mu\left(A_n\setminus\bigcup_{i<n}A_i\right)\le \sum_{n\in\mathbb N}\mu(A_n)
	\end{equation*}
\end{proof}



\begin{definition}\label{TrascurabiliMisura}
	In uno spazio di misura (o anch'è in ipotesi più deboli come premisura o misura esterna) un insieme con misura nulla è detto trascurabile.
\end{definition}
\begin{remark}\label{UnioneTrascurabili}
	Unione numerabile di insiemi trascurabile è a sua volta trascurabile.
\end{remark}
\begin{proof}
	È un'ovvia conseguenza della \sigsubadd[ità] della misura dimostrata in \cref{SubAdditivitaMisura}.
\end{proof}




\begin{definition}\label{FinitezzaMisura}
	Dato $(X,\mathcal A,\mu)$ uno spazio di misura, la misura $\mu$ è detta finita se $X\in\mathcal{A}$ e $\mu(X)<+\infty$.
\end{definition}

\begin{definition}\label{CompletezzaMisura}
	Dato $(X,\mathcal A,\mu)$ uno spazio di misura, la misura $\mu$ si dice completa se per ogni $A\in\mathcal A$ tale che $\mu(A)=0$, ogni suo sottoinsieme è anch'esso appartenente ad $\mathcal A$.
\end{definition}

\begin{proposition}\label{CompletamentoMisura}
	Dato uno spazio di misura $(X,\mathcal A,\mu)$, chiamo $\mu$-completamento di $\mathcal A$ l'insieme $\mathcal A^*$ formato dagli elementi di $\mathcal A$ uniti con un sottoinsieme di un trascurabile:
	\begin{equation*}
		\mathcal A^*=\{A\cup N\ |\ A\in\mathcal A\ \wedge \exists B:\ N\subseteq B,\ \mu(B)=0\}
	\end{equation*}
	Allora $\mathcal A^*$ è una \sigalg{} e $\mu$ si estende in maniera canonica su $\mathcal A^*$ ad una misura completa.
\end{proposition}
\begin{proof}
	Dimostriamo intanto che $\mathcal A^*$ è una \sigalg{}.
	Dato $A^*\in\mathcal A^*$, esistono per definizione $A\in\mathcal A$ e $N\subseteq B\in\mathcal A$ dove $\mu(B)=0$, tali che $A^*=A\cup N$. Possiamo facilmente imporre $B\cap A=\emptyset$, visto che se così non è si può ridefinire $N,B$ facendone la differenza con $A$. Assumiamo perciò $A,B$ disgiunti.
	Chiamando $M=B\setminus N$, vale facilmente:
	\begin{equation*}
		(A^*)^c=(A\cup N)^c=A^c\cap N^c=A^c\cap(B^c\cup M)=(A^c\cap B^c )\cup M \in\mathcal A^*
	\end{equation*}
	dove l'ultima uguaglianza vale perchè ho supposto $A$ e $B$ disgiunti e l'ultima appartenenza è invece vera poichè $M$ è anch'esso sottoinsieme di un trascurabile. Quindi $\mathcal A^*$ è stabile per passaggio al complementare.
	
	Ora consideriamo $(A^*_n)_{n\in\mathbb N}$ una successione di insiemi, e scegliamo $(A_n)_{n\in\mathbb N},(B_n)_{n\in\mathbb N},(N_n)_{n\in\mathbb N}$ tali che valgano $A^*_n=A_n\cup N_n$ e $N_n\subseteq B_n\in\mathcal A$ con $\mu(B_n)=0$.
	
	Allora risulta:
	\begin{align*}
		\bigcup_{n\in\mathbb N}A^*_n &=\bigcup_{n\in\mathbb N}A_n \cup \bigcup_{n\in\mathbb N}N_n\\
		\bigcup_{n\in\mathbb N}N_n &\subseteq \bigcup_{n\in\mathbb N}B_n\in\mathcal A\ \wedge\ \mu\left(\bigcup_{n\in\mathbb N}B_n\right)=0
	\end{align*}
	dove l'ultima uguaglianza dipende dal fatto che la misura è \sigsubadd{} per \cref{SubAdditivitaMisura}. Perciò abbiamo dimostrato che $\mathcal A^*$ è stabile per unione numerabile.
	
	Unendo i due risultati arriviamo a dire che $\mathcal A^*$ è una \sigalg{}.
	
	Definiamo ora $\tilde\mu:\mathcal A^*\to\Rpiu$ di modo che dato $A^*\in\mathcal A^*$ valga $\tilde\mu(A^*)=\mu(A)$ dove $A^*=A\cup N$ con $N$ sottoinsieme di un trascurabile ed $A\in\mathcal A$.
	
	Dimostriamo innanzitutto la coerenza della definizione di $\tilde\mu$. Si deve dimostrare che per $A,A'\in\mathcal A$ e $N,N'$ sottoinsiemi di trascurabili tali che $A\cup N=A'\cup N'$ vale $\mu(A)=\mu(A')$. Siano $B,B'\in\mathcal A$ i trascurabili a cui appartengono $N,N'$, sia infine $Q=B\cup B'$ a sua volta trascurabile.
	
	Per facili ragionamenti di monotonia abbiamo:
	\begin{align*}
		\mu(A)\le \mu(A\cup Q) =\mu(A)+\mu(Q)=\mu(A) &\Longrightarrow \mu(A)=\mu(A\cup Q)\\
		\mu(A')\le \mu(A'\cup Q) =\mu(A')+\mu(Q)=\mu(A') &\Longrightarrow \mu(A')=\mu(A'\cup Q)
	\end{align*}
	ma per costruzione vale $A\cup Q=A'\cup Q$ e perciò otteniamo proprio $\mu(A)=\mu(A')$ che mostra la coerenza della definizione di $\tilde\mu$.
	
	La \sigadd[ità] è ovvia una volta che si è mostrata la coerenza. Presa $(A^*_n)_{n\in\mathbb N}\in\mathcal A^*$ famiglia disgiunta a due a due, sia $(A_n)_{n\in\mathbb N}$ la parte \emph{non trascurabile} della prima successione. Allora risulta, per definizione di $\tilde\mu$:
	\begin{equation*}
		\sum_{n\in\mathbb N} \tilde\mu(A^*_n)=\sum_{n\in\mathbb N} \mu(A_n)=\mu\left(\bigcup_{n\in\mathbb N} A_n\right)=\mu\left(\bigcup_{n\in\mathbb N} A^*_n\right)
	\end{equation*}
	dove nell'ultimo passaggio abbiamo implicitamente applicato \cref{UnioneTrascurabili}.
	
	Resta da verificare che $\tilde\mu$ sia completa, ma questo è ovvio visto che  $\tilde\mu(A^*)=0$ se e solo se $A^*$ è il sottoinsieme di un trascurabile di $\mathcal A$ e la relazione di essere sottoinsieme di un trascurabile è chiusa per estrazione di sottoinsiemi. 

\end{proof}



\begin{proposition}\label{LimiteMonotonoMisura}
	Dato $(X,\mathcal A,\mu)$ uno spazio di misura, allora data una successione $(A_n)_{n\in\mathbb N}\subseteq \mathcal A$ tale che $A_n\subseteq A_{n+1}$ vale:
	\begin{equation*}
		\mu\left(\bigcup_{n\in\mathbb N} A_n\right)=\lim_{n\in\mathbb N} \mu(A_n)
	\end{equation*}
\end{proposition}
\begin{proof}
	Sia $B_n=A_n\setminus\bigcup_{i<n}A_i$. Applicando \cref{ProprietaSigAlg} si ottiene $B_n\in\mathcal A$.
	Per facili ragionamenti insiemistici risulta che la successione $(B_n)_{n\in\mathbb N}$ è disgiunta a due a due ed inoltre $A_n=\bigcup_{i\le n}B_i$.
	Sfruttando tutte queste proprietà e la \sigadd[ità] di $\mu$, otteniamo:
	\begin{equation*}
		\mu\left(\bigcup_{n\in\mathbb N} A_n\right)=\mu\left(\bigcup_{n\in\mathbb N} B_n\right)=
		\sum_{n\in\mathbb N} \mu(B_n)=\lim_{n\to\infty} \sum_{i\le n} \mu(B_i)=
		\lim_{n\to\infty} \mu\left(\bigcup_{i\le n} B_i\right)=\lim_{n\to\infty} \mu(A_n)
	\end{equation*}
	che è proprio la tesi.
\end{proof}

\begin{definition}[Misura esterna]
	Dato un insieme $X$ e una funzione $\mu^*:\mathcal P(X)\to \Rpiu$ è detta una misura esterna se valgono:
	\begin{itemize}
		\item $\mu^*(\emptyset)=0$
		\item $\mu^*$ è monotona, cioè dati $A,B\subseteq X$ se vale $A\subseteq B$ allora $\mu^*(A)\le \mu^*(B)$
		\item $\mu^*$ è \sigsubadd{}, cioè  per ogni successione $(A_n)_{n\in\mathbb N}\subseteq \mathcal P(X)$ di sottoinsiemi di $X$ vale $\mu^*\left(\bigcup_{n\in\mathbb{N}}A_n\right)\le \sum_{n\in\mathbb N} \mu^*(A_n)$
	\end{itemize}
\end{definition}

\begin{remark}
	Dato $(X,\mathcal A,\mu)$ uno spazio di misura, la misura $\mu$ è \sigsubadd{}.
\end{remark}
\begin{proof}
	Data una successione di sottoinsiemi $(A_n)_{n\in\mathbb N}\subseteq \mathcal A$, consideriamo, come nella dimostrazione di \cref{LimiteMonotonoMisura}, i sottoinsiemi $B_n=A_n\setminus\bigcup_{i<n}A_i\in\mathcal A$.
	Allora, lavorando analogamente alla dimostrazione di cui sopra, si ha:
	\begin{equation*}
		\mu\left(\bigcup_{n\in\mathbb N} A_n\right)=\mu\left(\bigcup_{n\in\mathbb N} B_n\right)=
		\sum_{n\in\mathbb N} \mu(B_n)\le \sum_{n\in\mathbb N} \mu(A_n)
	\end{equation*}
	dove nell'ultimo passaggio sfruttiamo la monotonia di $\mu$ dimostrata in \cref{MonotoniaMisura}.
\end{proof}

\begin{definition}
	Una terna $(X,\mathcal S,\mu)$ tale che $\mathcal S\subseteq\mathcal P(X)$ è un \semiring{} e $\mu:\mathcal S\to \Rpiu$ è \sigadd{}, la chiamo spazio di misura elementare e la funzione $\mu$ la chiamo misura elementare o premisura.
\end{definition}

\begin{lemma}\label{CoerenzaPremisura}
	Fissato $(X,\mathcal S,\mu)$ uno spazio di misura elementare, siano $(A_n)_{n\in\mathbb N},(B_n)_{n\in\mathbb N}\subseteq\mathcal S$ delle famiglie tali che l'unione sia la stessa, ma i $(B_n)_{n\in\mathbb N}$ siano disgiunti a due a due: $\bigcup_{n\in\mathbb N}A_n=\bigsqcup_{n\in\mathbb N}B_n$.
	Allora risulta $\sum_{n\in\mathbb N}\mu(A_n)\ge \sum_{n\in\mathbb N}\mu(B_n)$.
\end{lemma}
\begin{proof}
	Sia $A'_n=A_n\setminus\bigcup_{i<n}A_i$. La successione $(A'_n)_{n\in\mathbb N}$ è disgiunta a due a due e ogni singolo elemento appartiene a $\sqcup \mathcal S$ visto che vale $A'_n=\bigcap_{i<n}A_n\setminus A_i$ e $\sqcup \mathcal S$ è chiuso per intersezione finita, come mostrato in \cref{UnioneDisgiuntaQuasiAlgebra}. Infine è chiaro che l'unione della nuova famiglia è uguale a quella di $(A_n)_{n\in\mathbb N}$.
	È importante notare che $A_n\setminus A'_n=A_n\cap\bigcup_{i<n}A_i\in\sqcup\mathcal S$ dove l'ultima appartenenza è vera per la stabilità di $\sqcup\mathcal S$ per unioni e intersezioni finite. Allora esistono $(E^n_i)_{i\in\mathbb N}\subseteq\mathcal S$ tali che in unione disgiunta realizzano $A_n\setminus A'_n$.
	
	Siano quindi $C_{ij}=A'_i\cap B_j$. Ovviamente la successione $(C_{ij})_{i,j\in\mathbb N}$ è disgiunta a due a due (perchè lo sono sia $(A'_n)_{n\in\mathbb N}$ che $(B_n)_{n\in\mathbb N}$) ed è un sottoinsieme di $\sqcup\mathcal S$ poichè intersezione di elementi che vi appartengono. Quindi esiste la famiglia $(F^{ij}_n)_{n\in\mathbb N}\subseteq\mathcal S$ la cui unione disgiunta realizza $C_{ij}$.
	
	Ora per costruzione e per le osservazioni fatte valgono:
	\begin{align*}
		A_n= A'_n\sqcup \bigsqcup_{i\in\mathbb N}E^n_i=\bigsqcup_{i\in\mathbb N}C_{ni}\sqcup \bigsqcup_{i\in\mathbb N}E^n_i
		=\bigsqcup_{i,j\in\mathbb N}F^{ni}_j\sqcup \bigsqcup_{i\in\mathbb N}E^n_i
		&\Longrightarrow \mu(A_n)\ge\sum_{i,j\in\mathbb N}\mu(F^{ni}_j)+\sum_{i\in\mathbb N}\mu(E^n_i)\\
		B_n=\bigsqcup_{i\in\mathbb N}C_{in}=\bigsqcup_{i,j\in\mathbb N}F^{in}_j
		&\Longrightarrow \mu(B_n)=\sum_{i,j\in\mathbb N}\mu(F^{in}_j)
	\end{align*}
	quindi sommando su $n$ arriviamo a:
	\begin{align*}
		\sum_{n\in\mathbb N}\mu(A_n)&\ge \sum_{n\in\mathbb N}\sum_{i,j\in\mathbb N}\mu(F^{ni}_j)
		=\sum_{i,n,j\in\mathbb N}\mu(F^{ni}_j)\\
		\sum_{n\in\mathbb N}\mu(B_n)&=\sum_{n\in\mathbb N}\sum_{i,j\in\mathbb N}\mu(F^{in}_j)=\sum_{i,n,j\in\mathbb N}\mu(F^{in}_j)
	\end{align*}
	ma visto che l'ordine degli indici non conta, questi risultati implicano banalmente la tesi.


\end{proof}



\begin{lemma}\label{PiuCheMonotonaPremisura}
	Dato $(X,\mathcal S,\mu)$ uno spazio di misura elementare, siano $A,(A_n)_{n\in\mathbb N}\subseteq \mathcal S$ tali che $A\subseteq\bigcup_{n\in\mathbb N}A_n$.
	Allora vale $\mu(A)\le \sum_{n\in\mathbb N}\mu(A_n)$.
\end{lemma}
\begin{proof}
	Visto che $A\subseteq\bigcup_{n\in\mathbb N}A_n$ vale la scrittura insiemistica:
	\begin{equation}\label{ScritturaDecenteUnionePremisura}
		\bigcup_{n\in\mathbb N}A_n=A\sqcup\bigcup_{n\in\mathbb N}A_n\setminus A
	\end{equation}
	Poichè $\mathcal S$ è un \semiring{} $A_n\setminus A\in \sqcup \mathcal S$, e visto che $\sqcup S$ è chiuso per unione numerabile, come mostrato in \cref{UnioneDisgiuntaQuasiAlgebra}, esiste una famiglia $(B_n)_{n\in\mathbb N}\subseteq\mathcal S$ disgiunta tale che $\bigcup_{n\in\mathbb N}A_n\setminus A=\bigsqcup_{n\in\mathbb N}B_n$.
	Allora sostituendo in \cref{ScritturaDecenteUnionePremisura} si ottiene:
	\begin{equation*}
		\bigcup_{n\in\mathbb N}A_n=A\sqcup\bigsqcup_{n\in\mathbb N}B_n
	\end{equation*}
	quindi si ricade nelle ipotesi di \cref{CoerenzaPremisura} ottenendo che:
	\begin{equation*}
		\sum_{n\in\mathbb N}\mu(A_n)\ge \mu(A)+\sum_{n\in\mathbb N}\mu(B_n)\ge \mu(A)
	\end{equation*}
	che è proprio quanto si voleva dimostrare.

\end{proof}




\section{Trasmissione della completezza}
In questa sezione dimostrerò che se $(X,d)$ è completo allora lo è anche $(\CL(X),\delta) $.
In particolare nel primo lemma mostrerò la convergenza di una successione di Cauchy i cui elementi si contengono, per poi estendere il risultato ad una generica successione di Cauchy.

\begin{lemma} \label {inglobati}
Sia $\left (X,d\right )$ uno spazio metrico completo e $\left (K_n \right )$ una successione di Cauchy di chiusi e limitati in $\left (X,d\right )$ tali che $K_{n+1}\subseteq K_n$ $ \forall n \in \mathbb{N}$. Allora la successione ammette limite in $\left (\CL (X),\delta\right  ) $ uguale a $ K=\bigcap_{n \in \mathbb{N}}K_n $.
\end{lemma}
\begin{proof}
	È ovvio che $K$ è a sua volta un chiuso e limitato.
	
	Assumendo che, fissato $\varepsilon>0$, per $n$ sufficientemente grande, per ogni $\bar{x}\in K_n$ esista un elemento $x\in K$ tale che $d(\bar{x},x)\le \varepsilon$, ricavo in particolare $d_K(\bar{x})\le \varepsilon$ e perciò: 
	\begin{equation*}
		\delta(K,K_n)=\max\left\{\delta_{K_n}(K),\delta_K(K_n)\right\}=\max\left\{0,{}\sup_{\bar{x}\in K_n} d_K(\bar x)\right\}\le \varepsilon
	\end{equation*}
	perciò ho proprio $K_n\to K$.
	
	Non resta ora che dimostrare l'assunzione fatta. Sia $N$ tale che per ogni $n,m\ge N$ vale $\delta(K_n,K_m) < \frac{\varepsilon}2$, tale $N$ esiste poichè $(K_n)$ è di Cauchy. Fisso $\bar{n}\ge N$ e $\bar x\in K_{\bar n}$.
	
	Definisco la successione $(n_i)$ di modo che valgano le seguenti (anche questa esiste perchè $(K_n)$ è di Cauchy):
	\begin{equation*}
	\begin{cases}
		n_1=\bar n\\
		\forall i\ge 1:\ n_i<n_{i+1} \\
		\forall n\ge n_i:\ \delta(K_n,K_{n_i}) < \frac{\varepsilon}{2^i}
	\end{cases}\end{equation*}
	e definisco anche la successione $(x_n)_{n\ge \bar n}$ in modo che:
	\begin{equation*}\begin{cases}
		x_{\bar n}=\bar x\\
		x_n\in K_n\\
		n_i<n\le n_{i+1}\Longrightarrow d(x_{n_i},x_n)\le \frac{\varepsilon}{2^i}
	\end{cases}\end{equation*}
	Riesco sempre a trovare $x_n$ come richiesto poichè la condizione $\delta(K_n,K_{n_i}) < \frac{\varepsilon}{2^i}$, sfruttando \cref{ApprossimazioneDistanzaAsimmetrica}, me ne assicura l'esistenza.
	
	Allora ora vale la seguente stima:
	\begin{align*}
		n_i<n\le n_{i+1}\Longrightarrow d(\bar x, x_n)&\le d(\bar x, x_2)+d(x_2,x_3)+\cdots+d(x_{n_{i-1}},x_{n_i})+
		d(x_{n_i},x_n)\\
		&\le\frac{\varepsilon}2+\frac{\varepsilon}4+\cdots+\frac{\varepsilon}{2^{i-1}}
		+\frac{\varepsilon}{2^i}<\varepsilon
	\end{align*}
	e con un ragionamento del tutto analogo si ricava che la successione è definitivamente contenuta nella palla $B_{\frac{\varepsilon}{2^{i-1}}}(x_{n_i})$ e perciò è di Cauchy. 
	Ma $X$ è completo per ipotesi, quindi $x_n\to x\in X$ e $x\in K_n$ per ogni $n$ poichè definitivamente $x_i\in K_i \subseteq K_n$ e $K_n$ è chiuso. Perciò $x\in K$.
	
	Inoltre, poichè $d(\bar x,x_n)\le \varepsilon$, risulta che $d(\bar x, x)\le \varepsilon$ e per quanto appena detto implica che ho trovato, come cercavo, un punto in $K$ che dista meno di $\varepsilon$ da $\bar x$.
	
\end{proof}

\begin{lemma} \label{UnioniDiCauchy}
Sia $(K_n)$ una successione di Cauchy di chiusi e limitati in $\left (\CL(X),\delta\right  ) $. 
Allora $\forall k \in \mathbb{N}$ vale che
\begin{equation*}
A_k:=\overline{\bigcup_{n\geq k} K_n}
\end{equation*}
è chiuso e limitato, ed inoltre gli $A_k$ formano a loro volta una successione di Cauchy.
\end{lemma}

\begin{proof}
Per definizione $A_k$ è chiuso per ogni $k$. 

Dato $\varepsilon$, esiste $n_0$ tale che $\delta(K_{n_0}, K_n)< \varepsilon$ per ogni $n\geq n_0$. 
Inoltre siano $x_0\in X,r>0$, che esistono per la limitatezza di $K_{n_0}$, tali che $K_{n_0}\subseteq B_r(x_0)$. 
Allora, con una facile applicazione della disuguaglianza triangolare ottengo:
\begin{equation*}
	\forall n\ge n_0:\ K_n\subseteq B_{r+\varepsilon}(x_0) \Rightarrow A_{n_0}\subseteq \overline{B_{r+\varepsilon}(x_0)}
\end{equation*}
Quindi $A_{n_0}$ è limitato, ma allora lo è anche $A_1$ visto che lo posso esprimere come $\bigcup_{n<n_0}K_n \bigcup A_{n_0}$ e unione finita di limitati è limitata. Da questo discende che $A_k$ è limitato per ogni $k$ visto che $A_k\subseteq A_1$.

Unendo quanto detto ottengo che gli $A_k$ sono tutti chiusi e limitati.

Se $\delta(K_n, K_m)\leq \varepsilon$ $\forall n,m \geq n_0$  allora vale:
\begin{equation*}
	\forall n,m\ge n_0:\ \delta_{A_n}(A_m)\le \delta_{K_n}\left(\overline{\bigcup_{i\ge m}K_i} \right)
	=\sup_{i\ge m} \delta_{K_n}(K_i) \le \varepsilon
\end{equation*}
E questo dimostra che $(A_k)$ è una successione di Cauchy.
\end{proof}

\begin{theorem} \label{CompletezzaCL}
	Se $(X,d)$ è uno spazio metrico completo allora anche $(\CL(X),\delta)$ lo è.
\end{theorem}
\begin{proof}
	Sia $(K_n)$ una successione di Cauchy in $(\CL(X),\delta)$.
	Per \cref{UnioniDiCauchy} gli insiemi $A_n$ (definiti come nell'enunciato del lemma) formano una successione di Cauchy.
	Inoltre vale ovviamente $A_{n+1}\subseteq A_n$ quindi posso applicare \cref{inglobati} e ottenere che la successione $(A_n)$ converge ad $A\in\CL(X)$.
	
	Vale però che $K_n\subseteq A_n$ e quindi $\delta_{A_n}(K_n)=0$.
	
	D'altra parte, fissato $\varepsilon > 0$, esiste $n_0$ tale che per ogni $n,m\ge n_0$ vale $\delta(K_n,K_m)\le \varepsilon$, e perciò ricavo:
	\begin{equation*}
		\forall n\ge n_0:\ \delta_{K_n}(A_n)=\delta_{K_n}\left(\overline{\bigcup_{i\ge n} K_i}\right)
		=\sup_{i\ge n}\delta_{K_n}(K_i)\le \varepsilon
	\end{equation*}
	
	Perciò unendo le ultime due affermazioni ottengo $\delta(K_n,A_n)\to 0$, e questo implica (come fatto generale riguardo le successioni in un metrico) che $\lim_{n\to\infty}K_n=\lim_{n\to\infty}A_n=A$ e quindi ho dimostrato che la generica successione di Cauchy $K_n$ converge, ottenendo la tesi.
\end{proof}

\begin{remark}\label{CompletezzaInverso}
Se $\left (\CL(X),\delta \right ) $ è completo lo è anche $(X,d)$.
\end{remark}
\begin{proof}
Per quanto dimostrato in \cref{IsometriaCanonica} $\varphi(X)$ è un chiuso di $\CL(X)$, che però è per ipotesi un completo, perciò $\varphi(X)$ è un completo. Ma ancora sfruttando quanto detto in \cref{IsometriaCanonica} so che $(\varphi(X),\delta)$ è isometrico a $(X,d)$ e quindi ne ricavo che anche $(X,d)$ è un completo.
\end{proof}

\section{Trasmissione della compattezza}
In questa sezione dimostreremo, sfruttando i risultati precedentemente ottenuti, che se $X$ è un compatto allora anche $\mathcal{K}(X)$ lo è.

\begin{lemma}\label{TotaleLimitatezzaKX}
	Se $(X,d)$ è uno spazio metrico totalmente limitato, allora anche $(\CL(X),\delta)$ lo è.
\end{lemma}
\begin{proof}
	Fissato $\varepsilon$, per l'ipotesi di totale limitatezza di $X$, esiste un insieme $S\subset X$ finito, tale che:
	\begin{equation}\label{TotaleLimitatezzaX}
		X=\bigcup_{s\in S} B_{s}(\varepsilon) \Longrightarrow 
		\forall x\in X\ \exists s\in S:\ d(x,s)\le \varepsilon
	\end{equation}

	Ora considero $\mathcal{R}=\mathcal{P}(S)$. Questo è un insieme finito di sottinsiemi finiti (quindi chiusi e limitati) di $X$, perciò è un sottoinsieme finito di $\CL(X)$. 
	
	Dimostro che fissato $K\in\CL(X)$ esiste $R\in \mathcal{R}$ tale che $\delta(K,R)\le \varepsilon$, e questo è equivalente alla tesi di totale limitatezza di $\CL(X)$. In particolare la scelta di $R$ è costruttiva, visto che pongo:
	\begin{equation*}
		R=\{s\in S\ |\ d_K(s)\le \varepsilon\}
	\end{equation*}
	
	Sfruttando la sola definizione di $R$ ottengo che vale:
	\begin{equation}\label{DeltaKR}
		\delta_K(R)=\sup_{r\in R} d_K(r) \le \varepsilon
	\end{equation}
	
	Fissato $x\in K$ per \cref{TotaleLimitatezzaX} ottengo esiste $s\in S$ tale che $d(s,x)\le \varepsilon$. Di conseguenza risulta:
	\begin{equation*}
		d_K(s)\le d(s,x) \le \varepsilon \Longrightarrow s\in R
	\end{equation*}
	dove nell'implicazione ho usato la definizione di $R$.
	Perciò, ricordando che $x\in K$ era scelto arbitrariamente per ottenere quest'ultimo fatto, ho dimostrato che per ogni $k\in K$ esiste $r_k\in R$ tale che $d_K(r_k)\le \varepsilon$.
	
	Sfruttando quanto detto vale:
	\begin{equation}\label{DeltaRK}
		\delta_R(K)=\sup_{k\in K} d_R(k) \le \sup_{k\in K} d(r_k,k) \le \varepsilon
	\end{equation}
	
	Unendo \cref{DeltaKR,DeltaRK} e applicando la definizione di $\delta({}\cdot{},{}\cdot{})$ ottengo:
	\begin{equation*}
		\delta(K,R)=\max\left(\delta_K(R),\delta_R(K)\right)\le \varepsilon
	\end{equation*}
	che è quanto volevo e, come già annunciato, dimostra la totale limitatezza di $\CL(X)$.
\end{proof}

\begin{remark}\label{TotaleLimitatezzaInverso}
	Vale anche l'implicazione inversa di \cref{TotaleLimitatezzaKX}, cioè che se  $(\CL(X),\delta)$ è totalmente limitato, allora anche $X$ lo è.
\end{remark}
\begin{proof}
	Basta ricordare che esiste un'isometria tra $X$ e un sottinsieme di $\CL(X)$ e la proprietà di totale limitatezza di $\CL(X)$ si trasmette ad ogni suo sottinsieme.
\end{proof}



\begin{theorem} \label{CompattezzaKX}
	Se $(X,d)$ è uno spazio metrico compatto, allora anche $(\CL(X),\delta)$ lo è.
\end{theorem}
\begin{proof}
	Se $X$ è compatto allora è in particolare anche completo e totalmente limitato, perciò risultano verificate le ipotesi dei \cref{CompletezzaKX,TotaleLimitatezzaKX}. Applicando tali risultati ottengo che $\CL(X)$ è completo e totalmente limitato, quindi è compatto e il teorema è dimostrato.
\end{proof}

\begin{remark} \label{CompattezzaInverso}
	Vale anche l'implicazione inversa di \cref{CompattezzaKX}, cioè che se  $(\CL(X),\delta)$ è compatto allora anche $(X,d)$ lo è.
\end{remark}
\begin{proof}
	Essendo $\CL(X)$ un compatto è in particolare completo e totalmente limitato, perciò sono verificate le ipotesi dei \cref{CompletezzaInverso,TotaleLimitatezzaInverso}. Applicando quindi tali risultati ottengo che $X$ è completo e totalmente limitato, quindi è compatto.
\end{proof}


\section{Proprietà dei chiusi e \emph{totalmente} limitati}
In questa parte conclusiva mostriamo che quanto già dimostrato per lo spazio $\CL(X)$ vale anche per lo spazio dei chiusi e totalmente limitati. Infine facciamo notare come le proprietà principali vengano ereditate anche dallo spazio dei compatti di $X$. In particolare l'ultimo teorema appare in letteratura a volte con il nome di \emph{Teorema di selezione di Blaschke} (che però sembra riferirsi anche ad un altro risultato riguardante gli insiemi convessi).

Il passaggio principale, da cui tutto discende come facile corollario, è che i totalmente limitati sono un chiuso nello spazio $\CL(X)$.

\begin{definition}
	Dato uno spazio metrico $(X,d)$, sia $\CTL(X)$ lo spazio dei chiusi e totalmente limitati non vuoti di $X$.
\end{definition}
\begin{remark}\label{SottoinsiemeCTL}
	Ovviamente risulta $\CTL(X)\subseteq \CL(X)$, visto che essere totalmente limitato implica essere limitato.
\end{remark}

\begin{lemma}\label{ChiusoCTL}
	Lo spazio $(\CTL(X),\delta)$ è un chiuso dello spazio $(\CL(X),\delta)$.
\end{lemma}
\begin{proof}
	Poichè la \cref{SottoinsiemeCTL} mi assicura che $\CTL(X)$ è un sottoinsieme di $\CL(X)$, mi è sufficiente dimostrare che se una successione $(K_n) \subseteq \CTL(X)$ converge a $K\in\CL(X)$ allora $K\in\CTL(X)$.
	
	Fissato $\varepsilon$ la convergenza $K_n\to K$ mi assicura che:
	\begin{equation*}
		\exists n_0\in\mathbb{N}:\ \delta(K_{n_0},K)\le \frac{\varepsilon}3
		\Longrightarrow \delta_{K_{n_0}}(K)\le \frac{\varepsilon}3
	\end{equation*}
	e perciò applicando \cref{ApprossimazioneDistanzaAsimmetrica} ottengo:
	\begin{equation}\label{TL1}
		\forall k\in K\ \exists x\in K_{n_0}:
		\ d(k,x)\le \delta_{K_{n_0}}(K)+\frac{\varepsilon}3\le \frac{2\varepsilon}3
	\end{equation}
	
	Inoltre poichè $K_{n_0}$ è totalmente limitato per ipotesi, esiste un insieme finito $S \subseteq X$ tale che:
	\begin{equation}\label{TL2}
		K_{n_0}\subseteq \bigcup_{s\in S} B_{\frac{\varepsilon}3}(s)
		\Longrightarrow \forall x\in K_{n_0}\ \exists s\in S:\ d(x,s)\le \frac{\varepsilon}3
	\end{equation}
	
	Unendo \cref{TL1,TL2} arrivo a dire:
	\begin{equation}
		\forall k\in K\ \exists x\in K_{n_0},s\in S: d(k,x)\le \frac{2\varepsilon}3 \wedge d(x,s)\le \frac{\varepsilon}3
		\Longrightarrow d(k,s)\le d(k,x)+d(x,s) \le \varepsilon
	\end{equation}
	e questo è proprio equivalente a dire che $K$ è totalmente limitato, che è quanto si voleva, visto che ogni suo elemento dista meno di $\varepsilon$ da un elemento dell'insieme finito $S$.
\end{proof}

\begin{corollary}\label{CompletezzaCTL}
	Se lo spazio $(X,d)$ è completo anche $(\CTL(X),\delta)$ lo è.
\end{corollary}
\begin{proof}
	Per il risultato \cref{CompletezzaCL} vale che $(\CL(X),\delta)$ è completo, ma per \cref{ChiusoCTL} so che $\CTL(X)$ è un chiuso in $\CL(X)$. Ma i chiusi di un completo sono a loro volta completi e perciò la tesi è dimostrata.
\end{proof}

\begin{corollary}\label{CompattezzaCTL}
	Se lo spazio $(X,d)$ è compatto anche $(\CTL(X),\delta)$ lo è.
\end{corollary}
\begin{proof}
	Essendo $(X,d)$ compatto, è in particolare completo e perciò per quanto appena detto in \cref{CompletezzaCTL} anche $\CTL(X)$ è completo. Inoltre $(X,d)$ è per ipotesi anche totalmente limitato e perciò per \cref{CompattezzaCL} risulta $\CL(X)$ totalmente limitato. Ma $\CTL(X)$ è un sottoinsieme di $\CL(X)$ come osservato in \cref{SottoinsiemeCTL}, e la proprietà di totale limitatezza passa ai sottoinsiemi, quindi $\CTL(X)$ è totalmente limitato.
	
	Unendo quanto detto ho che $(\CTL(X),\delta)$ è completo e totalmente limitato, quindi è un compatto.
\end{proof}

\begin{definition}
	Dato uno spazio metrico $(X,d)$, sia $\K(X)$ lo spazio dei compatti non vuoti di $X$.
\end{definition}
\begin{remark}
	Ovviamente risulta $\K(X)\subseteq \CTL(X)$ visto che un compatto è di certo chiuso e totalmente limitato.
\end{remark}

\begin{corollary}
	Se $(X,d)$ è uno spazio completo anche $(\K(X),\delta)$ lo è.
\end{corollary}
\begin{proof}
	Essendo $X$ completo i compatti coincidono con i chiusi e totalmente limitati, perciò $\K(X)=\CTL(X)$ e allora basta applicare \cref{CompletezzaCTL} per avere la tesi.
\end{proof}


\begin{corollary}[Teorema di selezione di Blaschke]
	Se $(X,d)$ è uno spazio compatto anche $(\K(X),\delta)$ lo è.
\end{corollary}
\begin{proof}
	Se $(X,d)$ è compatto allora è in particolare completo e perciò vale ancora $\K(X)=\CTL(X)$ e quindi applicando \cref{CompattezzaCTL} ho la tesi. 
\end{proof}







\section{Stabilità della connessione per passaggio al limite}
In questa sezione conclusiva noteremo che nello spazio $(\CL(X),\delta)$ la proprietà di connessione può andare perduta per passaggio al limite, mentre ciò non succede se si lavora nello spazio dei compatti $(\K(X),\delta)$.

\begin{theorem}
	Sia $(K_n)$ una successione di chiusi e limitati connessi che converge a $K\in\CL(X)$. Il chiuso e limitato $K$ può non essere connesso.
\end{theorem}
\begin{proof}
	Mostriamo un controesempio all'affermazione.
	
	Considero i sottinsiemi di $\mathbb R^2$ così definiti:
	\begin{align*}
		A&=\left\{(x,y)\in\mathbb R^2:\ y=0\right\}\\
		B&=\left\{(x,y)\in\mathbb R^2:\ x\ge 0\ \wedge\ y=\frac 1x \right\}
	\end{align*}
	cioè si tratta dell'asse delle ascisse e del grafico, nel primo quadrante, della funzione inverso.
	
	Ovviamente $A,B$ sono chiusi e connessi ed altrettanto ovviamente non sono limitati e $A\cup B$ non è connesso.
	
	Per rendere $A\cup B$ connesso consideriamo la seguente successione di sottinsiemi chiusi di $\mathbb R^2$:
	\begin{equation}
		C_n=\left\{(x,y)\in\mathbb R^2:\ x=n\ \wedge\ 0\le y\le \frac 1n\right\}
	\end{equation}
	cioè un segmentino verticale di lunghezza $\frac 1n$ che connette $A$ e $B$ (li connette banalmente per archi, quindi anche per aperti).
	
	Infine definiamo $K_n=A\cup B\cup C_n$. Per quanto detto la successione $(C_n)$ è formata da sottoinsiemi chiusi e connessi di $\mathbb R^2$.
	
	Per rendere gli elementi della successione anche limitati scegliamo una metrica differente per $\mathbb R^2$, in modo che tutto lo spazio sia limitato ma la topologia non cambi. Per farlo basta scegliere il minimo tra la distanza canonica ed il valore $1$. Chiamiamo $d:\mathbb R^2\to [0,1]$ questa nuova distanza.
	
	Allora ora $(K_n)$ è una successione di chiusi e limitati connessi di $(\mathbb R^2, d)$ (visto che le proprietà di chiusura e connessione sono prettamente topologiche).
	
	Per concludere dimostriamo che la successione $(K_n)$ converge a $A\cup B$ vista come successione in $(\CL(\mathbb R^2),\delta)$, e questo chiuderebbe il controesempio visto che, come si cercava, $A\cup B$ non è un connesso.
	
	Notiamo che vale $A\cup B\in K_n$ e perciò $\delta_{K_n}(A\cup B)=0$. 
	Inoltre fissato $x\in K_n$ o questo appartiene ad $A\cup B$ e perciò $d_{A\cup B}(x)=0$ oppure appartiene a $C_n$. In quest'ultimo caso però vale, per definizione di $C_n$, $d_{A\cup B}(x)\le d\left((n,0),x\right)\le \frac 1n$. Riassumendo giungiamo a:
	\begin{gather*}
		\delta_{A\cup B}(K_n)=\sup_{x\in K_n} d_{A\cup B}(x)\le \frac 1n \\
		\Downarrow \\
		\delta(A\cup B,K_n)=\max(\delta_{K_n}\left(A\cup B),\delta_{A\cup B}(K_n)\right)\le \max\left(0,\frac1n\right)=\frac 1n
	\end{gather*}
	e quindi abbiamo ottenuto come volevamo che $K_n\to A\cup B$ visto che la distanza $\frac 1n$ tende a $0$.
\end{proof}

\begin{lemma}\label{CompattoInAperto}
	Siano $K,A$ rispettivamente un compatto ed un aperto di uno spazio metrico $(X,d)$ tali che $K\subseteq A$. Allora esiste $r>0$ tale che $U_r(K)\subseteq A$ dove $U_r$ è l'operatore definito nell'enunciato del \cref{EquivalenzaAllargamento}.
\end{lemma}
\begin{proof}
	Assumiamo per assurdo sia falsa la tesi, allora esiste una successione di coppie $(k_n,b_n)\in K\times A^c$ tali che $d(k_n,b_n)\le \frac 1n$.
	Sfruttando la compattezza di di $K$ possiamo assumere inoltre, a meno di estrarre una sottosuccessione, che $k_n$ sia una successione convergente a $k\in K$.
	
	Per una facile applicazione della disuguaglianza triangolare otteniamo:
	\begin{equation}
		d(k,b_n)\le d(k,k_n)+d(k_n,b_n)\le d(k,k_n)+\frac 1n \implies d(k,b_n)\to 0
	\end{equation}
	quindi la successione $(b_n)\in A^c$ converge a $k\in K\subseteq A$, ma questo mostra l'assurdo visto che abbiamo trovato una successione di elementi nel complementare di un aperto che converge nell'aperto.

\end{proof}



\begin{theorem}
	Sia $(K_n)$ una successione di compatti connessi nello spazio metrico $(X,d)$ che converge a $K\in\K(X)$ nella metrica di Hausdorff. Allora il compatto $K$ è connesso a sua volta.
\end{theorem}
\begin{proof}
	Assumiamo per assurdo che $K$ non sia connesso, allora esistono due aperti $A_1,A_2$ disgiunti, entrambi con intersezione non nulla con $K$, tali che $K\subseteq A\cup B$.
	
	Innanzitutto $A\cap K, B\cap K$ sono ancora dei compatti. Questo perchè presa una successione $(a_n)$ in $A\cap K$, so che esiste una sottosuccessione che converge a $k\in K$, ma questa deve anche convergere nel complementare di $B$ visto che $B$ è un aperto e gli $a_n$ non vi appartengono. Quindi $k\in K\setminus B = K\cap A$ e questo dimostra quanto voluto.
	
	Ora per il \cref{CompattoInAperto} sappiamo che esistono $r_A,r_B$ tali che $U_{r_A}(K\cap A)\subseteq A$ e $U_{r_B}(K\cap B)\subseteq B$. Chiamiamo $r=\min(r_A,r_B)$.

	Per il \cref{EquivalenzaAllargamento} definitivamente vale $K_n\subseteq U_r(K)$ e, per quanto detto sopra, allora $K_n\subseteq A\cup B$. 
	Ma tutti i $K_n$ sono connessi per ipotesi, quindi $K_n\subseteq A\cup B$ implica $K_n\subseteq A$ oppure $K_n\subseteq B$.

	Esisterà un $K_{n_0}$ tale che, senza perdita di generalità, è un sottoinsieme di $A$ e tale che $\delta_{K_{n_0}}(K)\le r$ (dove sfruttiamo il fatto che $K_n\to K$). Perciò in particolare $\delta_{K_{n_0}}(K\cap B)\le r$.

	Ma allora, applicando il \cref{ApprossimazioneDistanzaAsimmetrica}, abbiamo due punti $a\in K_{n_0}\subseteq A,\ b\in K\cap B\subseteq B$ tali che $d(a,b)<r$. E questo porta all'assurdo visto che ne segue:
	\begin{equation*}
		d(a,b)<r\implies b\in U_r(K\cap A) \subseteq A \implies b\in A
	\end{equation*}
	

	
\end{proof}





\end{document}

\makeindex