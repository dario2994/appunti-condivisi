\section{Proprietà dei chiusi e \emph{totalmente} limitati}
In questa parte conclusiva mostriamo che quanto già dimostrato per lo spazio $\CL(X)$ vale anche per lo spazio dei chiusi e totalmente limitati. Infine facciamo notare come le proprietà principali vengano ereditate anche dallo spazio dei compatti di $X$. In particolare l'ultimo teorema appare in letteratura a volte con il nome di \emph{Teorema di selezione di Blaschke} (che però sembra riferirsi anche ad un altro risultato riguardante gli insiemi convessi).

Il passaggio principale, da cui tutto discende come facile corollario, è che i totalmente limitati sono un chiuso nello spazio $\CL(X)$.

\begin{definition}
	Dato uno spazio metrico $(X,d)$, sia $\CTL(X)$ lo spazio dei chiusi e totalmente limitati non vuoti di $X$.
\end{definition}
\begin{remark}\label{SottoinsiemeCTL}
	Ovviamente risulta $\CTL(X)\subseteq \CL(X)$, visto che essere totalmente limitato implica essere limitato.
\end{remark}

\begin{lemma}\label{ChiusoCTL}
	Lo spazio $(\CTL(X),\delta)$ è un chiuso dello spazio $(\CL(X),\delta)$.
\end{lemma}
\begin{proof}
	Poichè la \cref{SottoinsiemeCTL} mi assicura che $\CTL(X)$ è un sottoinsieme di $\CL(X)$, mi è sufficiente dimostrare che se una successione $(K_n) \subseteq \CTL(X)$ converge a $K\in\CL(X)$ allora $K\in\CTL(X)$.
	
	Fissato $\varepsilon$ la convergenza $K_n\to K$ mi assicura che:
	\begin{equation*}
		\exists n_0\in\mathbb{N}:\ \delta(K_{n_0},K)\le \frac{\varepsilon}3
		\Longrightarrow \delta_{K_{n_0}}(K)\le \frac{\varepsilon}3
	\end{equation*}
	e perciò applicando \cref{ApprossimazioneDistanzaAsimmetrica} ottengo:
	\begin{equation}\label{TL1}
		\forall k\in K\ \exists x\in K_{n_0}:
		\ d(k,x)\le \delta_{K_{n_0}}(K)+\frac{\varepsilon}3\le \frac{2\varepsilon}3
	\end{equation}
	
	Inoltre poichè $K_{n_0}$ è totalmente limitato per ipotesi, esiste un insieme finito $S \subseteq X$ tale che:
	\begin{equation}\label{TL2}
		K_{n_0}\subseteq \bigcup_{s\in S} B_{\frac{\varepsilon}3}(s)
		\Longrightarrow \forall x\in K_{n_0}\ \exists s\in S:\ d(x,s)\le \frac{\varepsilon}3
	\end{equation}
	
	Unendo \cref{TL1,TL2} arrivo a dire:
	\begin{equation}
		\forall k\in K\ \exists x\in K_{n_0},s\in S: d(k,x)\le \frac{2\varepsilon}3 \wedge d(x,s)\le \frac{\varepsilon}3
		\Longrightarrow d(k,s)\le d(k,x)+d(x,s) \le \varepsilon
	\end{equation}
	e questo è proprio equivalente a dire che $K$ è totalmente limitato, che è quanto si voleva, visto che ogni suo elemento dista meno di $\varepsilon$ da un elemento dell'insieme finito $S$.
\end{proof}

\begin{corollary}\label{CompletezzaCTL}
	Se lo spazio $(X,d)$ è completo anche $(\CTL(X),\delta)$ lo è.
\end{corollary}
\begin{proof}
	Per il risultato \cref{CompletezzaCL} vale che $(\CL(X),\delta)$ è completo, ma per \cref{ChiusoCTL} so che $\CTL(X)$ è un chiuso in $\CL(X)$. Ma i chiusi di un completo sono a loro volta completi e perciò la tesi è dimostrata.
\end{proof}

\begin{corollary}\label{CompattezzaCTL}
	Se lo spazio $(X,d)$ è compatto anche $(\CTL(X),\delta)$ lo è.
\end{corollary}
\begin{proof}
	Essendo $(X,d)$ compatto, è in particolare completo e perciò per quanto appena detto in \cref{CompletezzaCTL} anche $\CTL(X)$ è completo. Inoltre $(X,d)$ è per ipotesi anche totalmente limitato e perciò per \cref{CompattezzaCL} risulta $\CL(X)$ totalmente limitato. Ma $\CTL(X)$ è un sottoinsieme di $\CL(X)$ come osservato in \cref{SottoinsiemeCTL}, e la proprietà di totale limitatezza passa ai sottoinsiemi, quindi $\CTL(X)$ è totalmente limitato.
	
	Unendo quanto detto ho che $(\CTL(X),\delta)$ è completo e totalmente limitato, quindi è un compatto.
\end{proof}

\begin{definition}
	Dato uno spazio metrico $(X,d)$, sia $\K(X)$ lo spazio dei compatti non vuoti di $X$.
\end{definition}
\begin{remark}
	Ovviamente risulta $\K(X)\subseteq \CTL(X)$ visto che un compatto è di certo chiuso e totalmente limitato.
\end{remark}

\begin{corollary}
	Se $(X,d)$ è uno spazio completo anche $(\K(X),\delta)$ lo è.
\end{corollary}
\begin{proof}
	Essendo $X$ completo i compatti coincidono con i chiusi e totalmente limitati, perciò $\K(X)=\CTL(X)$ e allora basta applicare \cref{CompletezzaCTL} per avere la tesi.
\end{proof}


\begin{corollary}[Teorema di selezione di Blaschke]
	Se $(X,d)$ è uno spazio compatto anche $(\K(X),\delta)$ lo è
\end{corollary}
\begin{proof}
	Se $(X,d)$ è compatto allora è in particolare completo e perciò vale ancora $\K(X)=\CTL(X)$ e quindi applicando \cref{CompattezzaCTL} ho la tesi. 
\end{proof}






