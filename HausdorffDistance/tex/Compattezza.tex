\section{Trasmissione della compattezza}
In questa sezione dimostreremo che se $X$ è un compatto allora anche $\CL(X)$ lo è. 

In particolare proveremo che la proprietà di totale limitatezza di $X$ viene ereditata da $\CL(X)$ e quindi, sfruttando il \cref{CompletezzaCL}, anche la compattezza viene ereditata.

\begin{lemma}\label{TotaleLimitatezzaCL}
	Se $(X,d)$ è uno spazio metrico totalmente limitato, allora anche $(\CL(X),\delta)$ lo è.
\end{lemma}
\begin{proof}
	Fissato $\varepsilon>0$, per l'ipotesi di totale limitatezza di $X$, esiste un insieme $S\subseteq X$ finito, tale che:
	\begin{equation}\label{TotaleLimitatezzaX}
		X=\bigcup_{s\in S} B_{\varepsilon}(s) \Longrightarrow 
		\forall x\in X\ \exists s\in S:\ d(x,s)\le \varepsilon
	\end{equation}

	Ora consideriamo $\mathcal{R}=\mathcal{P}(S)$. Questo è un insieme finito di sottoinsiemi finiti (quindi chiusi e limitati) di $X$, perciò è un sottoinsieme finito di $\CL(X)$. 
	
	Dimostriamo che fissato $K\in\CL(X)$ esiste $R\in \mathcal{R}$ tale che $\delta(K,R)\le \varepsilon$, e questo è equivalente alla tesi di totale limitatezza di $\CL(X)$. In particolare la scelta di $R$ è costruttiva, visto che poniamo:
	\begin{equation*}
		R=\{s\in S\ |\ d_K(s)\le \varepsilon\}
	\end{equation*}
	
	Sfruttando la sola definizione di $R$ otteniamo che vale:
	\begin{equation}\label{DeltaKR}
		\delta_K(R)=\sup_{r\in R} d_K(r) \le \varepsilon
	\end{equation}
	
	Fissato $k\in K$ per l'\cref{TotaleLimitatezzaX} esiste $s\in S$ tale che $d(s,k)\le \varepsilon$. Di conseguenza risulta:
	\begin{equation*}
		d_K(s)\le d(s,k) \le \varepsilon \Longrightarrow s\in R \Longrightarrow d_R(k)\le d(s,k)\le \varepsilon
	\end{equation*}
	dove nella prima implicazione abbiamo usato la definizione di $R$. 
	Sfruttando l'ultima disuguaglianza mostrata è ovvio ottenere:
	\begin{equation}\label{DeltaRK}
		\delta_R(K)=\sup_{k\in K} d_R(k) \le  \varepsilon
	\end{equation}
	
	Unendo le \cref{DeltaKR,DeltaRK} e applicando la definizione di $\delta({}\cdot{},{}\cdot{})$ otteniamo:
	\begin{equation*}
		\delta(K,R)=\max\left(\delta_K(R),\delta_R(K)\right)\le \varepsilon
	\end{equation*}
	che è quanto volevamo e, come già annunciato, dimostra la totale limitatezza di $\CL(X)$.
\end{proof}

\begin{remark}\label{TotaleLimitatezzaInverso}
	Vale anche l'implicazione inversa di \cref{TotaleLimitatezzaCL}, cioè che se  $(\CL(X),\delta)$ è totalmente limitato, allora anche $(X,d)$ lo è.
\end{remark}
\begin{proof}
	Basta ricordare che per il \cref{IsometriaCanonica} esiste un'isometria tra $X$ e un sottoinsieme di $\CL(X)$ e che la proprietà di totale limitatezza di $\CL(X)$ si trasmette ad ogni suo sottoinsieme.
\end{proof}



\begin{theorem} \label{CompattezzaCL}
	Se $(X,d)$ è uno spazio metrico compatto, allora anche $(\CL(X),\delta)$ lo è.
\end{theorem}
\begin{proof}
	Se $X$ è compatto allora è in particolare anche completo e totalmente limitato, perciò risultano verificate le ipotesi di \cref{CompletezzaCL,TotaleLimitatezzaCL}. Applicando tali risultati otteniamo che $\CL(X)$ è completo e totalmente limitato, quindi è compatto e il teorema è dimostrato.
\end{proof}

\begin{remark} \label{CompattezzaInverso}
	Vale anche l'implicazione inversa del \cref{CompattezzaCL}, cioè che se  $(\CL(X),\delta)$ è compatto allora anche $(X,d)$ lo è.
\end{remark}
\begin{proof}
	Essendo $\CL(X)$ un compatto è in particolare completo e totalmente limitato, perciò sono verificate le ipotesi delle \cref{CompletezzaInverso,TotaleLimitatezzaInverso}. Applicando quindi tali risultati otteniamo che $X$ è completo e totalmente limitato, quindi è compatto.
\end{proof}

