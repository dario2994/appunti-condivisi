\section{Definizioni e risultati introduttivi}
In questa sezione definiremo la distanza di Hausdorff fra i chiusi e limitati e mostreremo qualche risultato introduttivo.

In tutto il corso della trattazione indicheremo con $(X,d)$ uno spazio metrico generico.

\begin{definition}
	Sia $\mathcal{K}(X)=\{K\in \mathcal{P}(X) : K \text{ è compatto}\}$ l'insieme dei compatti di $X$.
\end{definition}

\begin{definition}
	Dato $A\in \mathcal{K}(X)$, definiamo $d_A: X\to [0,+\infty)$ tale che $d_A(x)=\inf_{a\in A}d(a,x)$ per ogni $x\in X$.
\end{definition}

\begin{remark}
	$d_A({}\cdot{})$ è effettivamente una funzione a valori in $[0,+\infty)$ perchè vale $\inf_{a\in A}d(a,x)<\infty$, poichè $A$ essendo compatto è limitato.
\end{remark}

\begin{lemma}\label{DistanzaCompattoRealizzata}
	Per ogni $x\in X$ e $A \in \mathcal{K}(X)$, esiste $a\in A$ tale che $d_A(x)=d(a,x)$.
\end{lemma}
\begin{proof}
	Poichè $d_A(x)=\inf_{a\in A} d(a,x)$, esiste $(a_n)$ a valori in $A$ tale che $\lim_{n\to\infty}d(a_n,x)=d_A(x)$. Dato che $A$ è un compatto esiste una sottosuccessione $(a_{n_k})$ di $(a_n)$ convergente ad $a\in A$. Tale sottosuccessione rispetta quindi che $a_{n_k}\to a$ e $d(a_{n_k},x)\to d_A(x)$, da cui facilmente $d(a,x)=d_A(x)$, poichè $d(a_{n_k},x)\to d(a,x)$.
\end{proof}
\begin{remark}\label{DistanzaCompattoAppartenenza}
	Dati $x\in X$ e $A \in \mathcal{K}(X)$, $d_A(x)=0$ se e solo se $x\in A$. Infatti per \cref{DistanzaCompattoRealizzata} esiste $a\in A$ tale che $d_A(x)=d(a,x)$, da cui $d_A(x)=0$ se e solo se $a=x$.
\end{remark}

\begin{definition}\label{DistanzaFraCompatti}
	Sia $\delta_A:\mathcal{K}(X)\to [0,+\infty)$ tale che $\delta_A(B)=\sup_{b\in B} d_A(b)$.
\end{definition}
\begin{remark}
	Anche in questo caso $\delta_A(B)$ ha valori in $[0,+\infty)$ perchè $A\cup B$ è limitato.
\end{remark}

\begin{definition}[Distanza di Hausdorff]\label{HausdorffDefinizione}
	Definiamo infine $\delta:\mathcal{K}(X)\times \mathcal{K}(X) \to [0,+\infty)$ come $\delta(A,B)=\max\{ \delta_A(B),\delta_B(A) \}$.
\end{definition}

\begin{theorem}
	La funzione definita in \cref{HausdorffDefinizione} è una distanza chiamata distanza di Hausdorff.
\end{theorem}

\begin{proof}
	Dimostriamo che $\delta({}\cdot{},{}\cdot{})$ è veramente una distanza.
	\begin{itemize}
		\item $\delta(A,B)=\delta(B,A)$ (simmetria)
		
		$\delta(A,B)=\max\{ \delta_A(B),\delta_B(A) \}=\delta(B,A)$.
		\item $\delta(A,B)=0 \iff A=B$
		
		Se $A=B$ vale banalmente $\delta(A,B)=0$; se invece esiste $a\in A$ tale che $a\not\in B$ ho che $\delta(A,B)\ge \delta_B(A)\ge \delta_B(a)>0$, dove l'ultima disuguaglianza è vera per \cref{DistanzaCompattoAppartenenza}.
		\item $\delta(A,C)\le \delta(A,B)+\delta(B,C)$ (disuguaglianza triangolare)
		
		Dimostro innanzitutto che per ogni $c\in C$ vale $d_A(c)\le \delta(A,B)+\delta(B,C)$.
		
 		Per \cref{DistanzaCompattoRealizzata} ho che esiste $a\in A$ tale che $d_A(c)=d(a,c)$ e che esiste $b\in B$ tale che $d_B(c)=d(b,c)$. Analogamente esiste $a'\in A$ tale che $d_A(b)=d(a',b)$. Utilizzando facili conseguenze delle definizioni date precedentemente, ho quindi che 
 		\begin{equation*}
 			d_A(c)=d(a,c)\le d(a',c) \le d(a',b)+d(b,c)=d_A(b)+d_B(c)\le \delta(A,B)+\delta(B,C)
 		\end{equation*}
 		
 		Passando ora al $\sup$ su $c\in C$ in quest'ultima disuguaglianza ottengo che $\delta_A(C)\le \delta(A,B)+\delta(B,C)$, ma del tutto  analogamente vale $\delta_C(A)\le \delta(A,B)+\delta(B,C)$, quindi
 		\begin{equation*}
 			\delta(A,C)=\max\{ \delta_A(B),\delta_B(A) \}\le \delta(A,B)+\delta(B,C)
 		\end{equation*}
		che è proprio la disuguaglianza triangolare.
	\end{itemize}
	Inoltre $\delta$ ha valori in $[0,+\infty)$ poichè per ogni $A,B\in \mathcal{K}(X)$ $\delta(A,B)<\infty$, dato che $A$ e $B$ sono limitati.
\end{proof}

\begin{remark}
	Del tutto analogamente si dimostra che la distanza di Hausdorff è una distanza anche sui chiusi e limitati, poichè anche in questo caso vale \cref{DistanzaCompattoAppartenenza}.
\end{remark}

\begin{definition}[Definizione equivalente della distanza di Hausdorff] \label{HausdorffDefinizioneEquivalente}
	La distanza di Hausdorff si può definire in modo equivalente nel seguente modo. Sia $\delta_A'(B)=\inf\{r:B\subseteq U_r(A)\}$, dove $U_r(A)=\{x\in X : d_A(x)\le r \}$. Allora definisco $\delta'(A,B)=\max\{\delta_A'(B),\delta_B'(A)\}$
\end{definition}

\begin{theorem}
	Le definizioni \cref{HausdorffDefinizione} e \cref{HausdorffDefinizioneEquivalente} sono equivalenti.
\end{theorem}
\begin{proof}
	Dimostro in particolare che $\delta_A(B)=\delta_A'(B)$, dove $\delta_A$ e $\delta_A'$ sono definite rispettivamente in \cref{DistanzaFraCompatti} e \cref{HausdorffDefinizioneEquivalente}, poichè da questo segue banalmente la tesi.
	
	Chiamo $P=\{ r : \exists b\in B \text{ tale che }d_A(b)=r \}$ e $Q=\{ r: \forall b\in B\text{ vale } d_A(b)\le r \}$, allora si nota facilmente che per ogni $p\in P$ e $q\in Q$ vale $p\le q$ e inoltre per ogni $x\in X$ ho che $x\in P$ o $x\in Q$. Da questo segue facilmente che $\sup\{ r: r\in P \}=\inf\{r: r\in Q \}$.
	
	Vale quindi che
	\begin{equation*}
		\delta_A(B)=\sup_{b\in B} d_A(b)=\sup\{ r : r\in P \}=\inf\{ r : r\in Q \}= \inf\{ r : B\subseteq U_r(A) \}=\delta_A'(B)
	\end{equation*}
	che è quello che volevo dimostrare
\end{proof}

\begin{lemma}\label{IsometriaCanonica}
	Esiste un'isometria canonica $\varphi: X\in \mathcal{K}(X)$, tale che $\varphi(X)$ è un chiuso in $\mathcal{K}(X)$.
\end{lemma}
\begin{proof}
	Definisco l'isometria $\varphi: X\in\mathcal{K}(X)$ tale che $\varphi(x)=\{x\}$ per ogni $x\in X$. 
	
	Innanzitutto vale banalmente che questa è un'isometria, perchè segue facilmente dalle definizioni che $\delta(\{ x \}, \{ x' \})=d(x,x')$.
	
	Dimostriamo ora che $\varphi(X)$ è un chiuso. Sia $(\{ x_n \})_{n\in \mathbb{N}}$ una successione convergente in $\mathcal{K}(X)$, allora $(x_n)_{n\in \mathbb{N}}$ converge in $X$ ad un valore $x$ (poichè $\varphi$ è un isometria). Voglio mostrare che allora $(\{ x_n \})_{n\in \mathbb{N}}$ converge a $\{ x \}$, ma questo è ovvio perchè $\lim_{n\to\infty} \delta(\{ x_n \}, \{x\})=\lim_{n\to\infty} d(x,x')=0$. Questo conclude la dimostrazione per l'unicità del limite.
\end{proof}









