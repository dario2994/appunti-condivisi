\section{Definizioni e risultati introduttivi}
In questa sezione definiremo la distanza di Hausdorff fra i chiusi e limitati e mostreremo qualche risultato introduttivo.

In tutto il corso della trattazione indicheremo con $(X,d)$ uno spazio metrico generico.

\begin{definition}
	Sia $\CL(X)$ l'insieme dei sottoinsiemi chiusi e limitati non vuoti di $X$.
\end{definition}

\begin{definition}
	Dato $A\in \CL(X)$, definiamo $d_A: X\to [0,+\infty)$ tale che $d_A(x)=\inf_{a\in A}d(a,x)$ per ogni $x\in X$.
\end{definition}

\begin{lemma}\label{DistanzaChiusoAppartenenza}
	Dati $x\in X$ e $A \in \CL(X)$, $d_A(x)=0$ se e solo se $x\in A$.
\end{lemma}
\begin{proof}
	Se $x\in A$ vale facilmente che $d_A(x)=\inf_{a\in A}d(a,x)=d(x,x)=0$. 

	Se invece $d_A(x)=0$, dato che $\inf_{a\in A}d(a,x)=d_A(x)=0$, esiste una successione $(a_n)$ a valori in $A$ tale che $\lim_{n\to\infty}d(a_n,x)=0$, perciò $a_n\to x$. Di conseguenza $x$ appartiene ad $A$, perché $A$ è chiuso. 
\end{proof}


\begin{definition}\label{DistanzaAsimmetrica}
	Dato $A\in\CL(X)$, sia $\delta_A:\CL(X)\to [0,+\infty)$ tale che $\delta_A(B)=\sup_{b\in B} d_A(b)$.
\end{definition}
\begin{remark}
	Si ha effettivamente che $\delta_A(B)$ ha valori in $[0,+\infty)$ perché $A\cup B$ è limitato.
\end{remark}

\begin{lemma}\label{ApprossimazioneDistanzaAsimmetrica}
	Dati $A,B\in \CL(X)$, per ogni $\varepsilon>0$ e per ogni $a\in A$ esiste $b\in B$ tale che $d(a,b)\le \delta_B(A)+\varepsilon$.
\end{lemma}
\begin{proof}
	Poichè $d_B(a)=\inf_{b\in B}d(a,b)$, dato $\varepsilon$ esiste $b\in B$ tale che $d(a,b)\le d_B(a)+\varepsilon$, per la definizione di estremo inferiore. Quindi in particolare
	\begin{equation*}
		d(a,b)\le d_B(a)+\varepsilon\le \sup_{a\in A}d_B(a)+\varepsilon=\delta_B(A)+\varepsilon
	\end{equation*}
	che è quello che volevamo dimostrare.
\end{proof}

\begin{lemma}\label{EquivalenzaAllargamento}
	Dati $A,B\in \CL(X)$ vale:
	\begin{equation*}
		\delta_A(B)=\inf\{r\in\mathbb{R}:\ B\subseteq U_r(A)\}
	\end{equation*}
	dove $U_r(A)=\{x\in X :\ d_A(x)<r \}$.
\end{lemma}
\begin{proof}
	Sia $r_0=\inf\{r\in\mathbb{R}:\ B\subseteq U_r(A)\}$.
	
	Sappiamo che se $r$ è tale che $B\subseteq U_r(A)$, allora per ogni $b\in B$ vale $d_A(b)<r$, da cui
	\begin{equation}\label{delta'>delta}
		r\ge \sup_{b\in B} d_A(b)=\delta_A(B) \Longrightarrow r_0=\inf\{r:B\subseteq U_r(A) \}\ge \delta_A(B)
	\end{equation}
	
	D'altra parte se $r>\delta_A(B)$, per ogni $b\in B$ vale che $r>d_A(b)$, quindi $B\subseteq U_r(A)$ e perciò $r\ge r_0$. Passando all'estremo inferiore per $r>\delta_A(B)$ otteniamo quindi
	\begin{equation}\label{delta>delta'}
		\delta_A(B)=\inf\{r:r>\delta_A(B)\}\ge r_0
	\end{equation}
	
	Unendo le \cref{delta'>delta,delta>delta'} otteniamo quindi $\delta_A(B)=r_0$, che è quello che volevamo dimostrare.
\end{proof}




\begin{definition}[Distanza di Hausdorff]\label{HausdorffDefinizione}
	Definiamo infine $\delta:\CL(X)\times \CL(X) \to [0,+\infty)$ come $\delta(A,B)=\max\{ \delta_A(B),\delta_B(A) \}$.
\end{definition}

\begin{theorem}
	La funzione definita nella \cref{HausdorffDefinizione} è una distanza chiamata distanza di Hausdorff.
\end{theorem}

\begin{proof}
	Innanzitutto $\delta({}\cdot{},{}\cdot{})$ è veramente una funzione a valori in $[0,+\infty)$, perché lo erano $\delta_A(B),\delta_B(A)$.

	Dimostriamo ora che rispetta le proprietà di una distanza:
	\begin{itemize}
		\item (simmetria) $\delta(A,B)=\delta(B,A)$
		
		$\delta(A,B)=\max\{ \delta_A(B),\delta_B(A) \}=\delta(B,A)$.
		\item (non degenerazione) $\delta(A,B)=0 \iff A=B$
		
		Se $A=B$ vale banalmente $\delta(A,B)=0$; se invece esiste $a\in A$ tale che $a\not\in B$ abbiamo che $\delta(A,B)\ge \delta_B(A)\ge \delta_B(a)>0$, dove l'ultima disuguaglianza è vera per il \cref{DistanzaChiusoAppartenenza}.
		\item (disuguaglianza triangolare) $\delta(A,C)\le \delta(A,B)+\delta(B,C)$

		Senza perdita di generalità possiamo assumere che $\delta(A,C)=\delta_A(C)$. 

		Per ogni $c\in C$ e per ogni $\varepsilon>0$ per il \cref{ApprossimazioneDistanzaAsimmetrica} esiste $b\in B$ tale che $d(b,c)\le \delta_B(C)+\varepsilon$. A sua volta fissato tale $b$ esiste $a\in A$ tale che $d(a,b)\le \delta_A(B)+\varepsilon$. Ho quindi che
		\begin{equation*}
			d_A(c)\le d(a,c)\le d(a,b)+d(b,c)\le \delta_A(B)+\delta_B(C)+2\varepsilon\le \delta(A,B)+\delta(B,C)+2\varepsilon
		\end{equation*}
		da cui passando all'estremo superiore su $c$ otteniamo
		\begin{equation*}
			\delta(A,C)=\delta_A(C)=\sup_{c\in C}d_A(c)\le \delta(A,B)+\delta(B,C)+2\varepsilon
		\end{equation*}
		Facendo infine il limite per $\varepsilon\to 0$, abbiamo proprio quello che volevamo dimostrare
		\begin{equation*}
			\delta(A,C)\le \lim_{\varepsilon\to 0}(\delta(A,B)+\delta(B,C)+2\varepsilon)=\delta(A,B)+\delta(B,C)
		\end{equation*}
	\end{itemize}
\end{proof}

\begin{example}
	Sia $X=\mathbb{R}^2$ dotato della distanza euclidea. Consideriamo i due chiusi e limitati $A=\{(x,y)\in\mathbb{R}^2:|x|,|y|\le 1)\}$ e $B=\{(x,y)\in\mathbb{R}^2:(x-1)^2+(y-1)^2\le 1\}$.
	\begin{figure}[h]
	\begin{center}
		IMMAGINE\\
		IMMAGINE\\
		IMMAGINE\\
		IMMAGINE
% 		\includegraphics[width=0.65\textwidth]{EsempioHausdorff.png}
	\end{center}
	\end{figure}
	
	Abbiamo che $d_B((-1,-1))=2\sqrt{2}-1$, ma è facile verificare che per ogni altro punto $(x,y)\in A$ vale che $d_B((x,y))\le d_B((-1,-1))$, quindi $\delta_B(A)=2\sqrt{2}-1$.
	
	Inoltre per ogni punto $(x,y)\in B$ vale che $d_A((x,y))\le d((x,y),(1,1))\le 1$ e $d_A((1,2))=1$, quindi $\delta_A(B)=1$.
	
	Vale quindi che $\delta(A,B)=\max\{ \delta_A(B),\delta_B(A) \}=2\sqrt{2}-1$.
\end{example}


\begin{lemma}\label{IsometriaCanonica}
	Esiste un'isometria canonica $\varphi: X\to \CL(X)$, tale che $\varphi(X)$ è un chiuso in $\CL(X)$.
\end{lemma}
\begin{proof}
	Definiamo l'isometria $\varphi: X\to\CL(X)$ tale che $\varphi(x)=\{x\}$ per ogni $x\in X$. 
	
	Innanzitutto $\varphi({}\cdot{})$ è un'isometria, perché segue facilmente dalle definizioni che $\delta(\{ x \}, \{ x' \})=d(x,x')$.
	
	Dimostriamo ora che $\varphi(X)$ è un chiuso. Sia $(\{ x_n \})$ una successione convergente a $K\in \CL(X)$, allora dato $x\in K$, vale che
	\begin{equation*}
		d(x,x_n)\le \sup_{k\in K} d(k,x_n)=d_K(x_n)=\delta_K(\{x_n\})\le \delta(K,\{x_n \})
	\end{equation*}
	Quindi, dato che $\delta(K,\{x_n \})$ va a 0, anche $d(x,x_n)$ va a 0, cioè $(x_n)$ converge in $X$ a $x$. 
	
	Ma allora vale $\lim_{n\to\infty} \delta(\{x\}, \{ x_n \})=\lim_{n\to\infty} d(x,x_n)=0$, quindi per l'unicità del limite $K=\{x\}\in\varphi(X)$.
\end{proof}









