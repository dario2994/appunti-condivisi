\section{Trasmissione della completezza}



In questa sezione dimostrerò che se $\left (X,d\right )$ è completo allora lo è anche $\left (\CL(X),\delta\right  ) $.


\begin{lemma} \label {inglobati}
Sia $\left (X,d\right )$ uno spazio metrico completo e $\left (K_n \right )$ una successione di Cauchy di chiusi e limitati in $\left (X,d\right )$ tali che $K_{n+1}\subseteq K_n$ $ \forall n \in \mathbb{N}$. Allora la successione ammette limite in $\left (\CL (X),\delta\right  ) $ uguale a $ K=\bigcap_{n \in \mathbb{N}}K_n $.
\end{lemma}

\begin{proof}
Dimostro innanzitutto che l'intersezione dei chiusi e limitati non è vuota, costruendo una successione di Cauchy $\left (x_n\right ) \subseteq X$ tale che $x_n \in K_n$ $ \forall n \in \mathbb{N}$. 

Essendo la successione $\left(K_n\right)$ di Cauchy, posso definire la successione $(\varepsilon_n)$ in modo che valgano:
\begin{gather*}
	\forall m\geq n:\ \delta \left ( K_n, K_m \right )\leq \varepsilon_n \\
	\lim_{n \to \infty}\varepsilon_n =0
\end{gather*}
infatti basta prendere $n_k$ tale che $\delta \left ( K_m, K_l \right )\leq \frac{1}{2^k}$ $\forall m,l\geq n_k$ e ad ogni $n$ associo $\varepsilon_n=\frac{1}{2^j}$ in modo che $j=\max { \left \{ i:n_i \leq n \right \} }$. Per i primi termini della successione che voglio costruire faccio rispettare solo l'appartenenza sopra enunciata. Invece, a partire da $n=n_1$, una volta definiti tutti i termini fino a $x_n$ so che la palla di raggio $2\varepsilon_n$ centrata in $x_n$ interseca tutti i chiusi e limitati successivi, e dunque posso costruire induttivamente una successione di Cauchy. Dato che $\left (X,d\right )$ è completo, esiste il limite $x$ della successione. Per ogni $n$ la successione è contenuta definitivamente in $K_n$, ergo per chiusura il limite appartiene a $ \bigcap_{n \in \mathbb{N}}K_n $, che di conseguenza non è vuota (e chiaramente chiusa e limitata a sua volta).

Dato che per ogni $n$ $K_n\supseteq K$ allora ovviamente $\delta\left  (K_n,K\right )=\delta_{K}\left (K_n\right )$. Inoltre $\delta_{K}\left (K_n\right )$ è decrescente in $n$, quindi esiste il limite $r$ della distanza di Haurdorff:
\begin{equation*}
r=\lim_{n \to \infty}\delta_{K}\left (K_n\right )>0.
\end{equation*}
Considero la sottosuccessione $K_{n(k)}$, dove $n(k)$ è tale che $\delta \left ( K_m, K_l \right )\leq \frac{r}{2^{k+2}}$ $\forall m,l\geq n(k)$, e costruisco una successione $\left ( x_{n(k)} \right )$ tale che $d_{K} \left (x_{n(1)} \right )\geq r$, $ x_{n(k)} \in K_{n(k)}$ e che sia di Cauchy, ovvero in modo che $d \left ( x_{n(l)}, x_{n(m)} \right )\leq \frac{r}{2^{k+1}}$ da $n(k)$ in poi (lo faccio come sopra). Dato che la serie
\begin{equation*}
\sum_{i=2}^{\infty} \frac{r}{2^i}
\end{equation*}
converge a $\frac{r}{2}$, allora la distanza del limite $x$ da $K$ è maggiore o uguale a $\frac{r}{2}$. Ma come prima il limite appartiene all'intersezione, dunque si ha un assurdo.
\end{proof}

\begin{lemma} \label{UnioniDiCauchy}
Sia $(K_n)$ una successione di Cauchy di chiusi e limitati in $\left (\CL(X),\delta\right  ) $. 
Allora $\forall k \in \mathbb{N}$ vale che
\begin{equation*}
A_k:=\overline{\bigcup_{n\geq k} K_n}
\end{equation*}
è chiuso e limitato, ed inoltre gli $A_k$ formano a loro volta una successione di Cauchy.
\end{lemma}

\begin{proof}
Per definizione $A_k$ è chiuso per ogni $k$. 

Dato $\varepsilon$, esiste $n_0$ tale che $\delta(K_{n_0}, K_n)< \varepsilon$ per ogni $n\geq n_0$. 
Inoltre siano $x_0\in X,r>0$, che esistono per la limitatezza di $K_{n_0}$, tali che $K_{n_0}\subseteq B_r(x_0)$. 
Allora, con una facile applicazione della disuguaglianza triangolare ottengo:
\begin{equation*}
	\forall n\ge n_0:\ K_n\subseteq B_{r+\varepsilon}(x_0) \Rightarrow A_{n_0}\subseteq \overline{B_{r+\varepsilon}(x_0)}
\end{equation*}
Quindi $A_{n_0}$ è limitato, ma allora lo è anche $A_1$ visto che lo posso esprimere come $\bigcup_{n<n_0}K_n \bigcup A_{n_0}$ e unione finita di limitati è limitata. Da questo discende che $A_k$ è limitato per ogni $k$ visto che $A_k\subseteq A_1$.

Unendo quanto detto ottengo che gli $A_k$ sono tutti chiusi e limitati.

Se $\delta(K_n, K_m)\leq \varepsilon$ $\forall n,m \geq n_0$  allora vale:
\begin{equation*}
	\forall n,m\ge n_0:\ \delta_{A_n}(A_m)\le \delta_{K_n}\left(\overline{\bigcup_{i\ge m}K_i} \right)
	=\sup_{i\ge m} \delta_{K_n}(K_i) \le \varepsilon
\end{equation*}
E questo dimostra che $(A_k)$ è una successione di Cauchy.
\end{proof}

\begin{theorem} \label{CompletezzaCL}
	Se $(X,d)$ è uno spazio metrico completo allora anche $(\CL(X),\delta)$ lo è.
\end{theorem}
\begin{proof}
	Sia $(K_n)$ una successione di Cauchy in $(\CL(X),\delta)$.
	Per \cref{UnioniDiCauchy} gli insiemi $A_n$ (definiti come nell'enunciato del lemma) formano una successione di Cauchy.
	Inoltre vale ovviamente $A_{n+1}\subseteq A_n$ quindi posso applicare \cref{inglobati} e ottenere che la successione $(A_n)$ converge ad $A\in\CL(X)$.
	
	Vale però che $K_n\subseteq A_n$ e quindi $\delta_{A_n}(K_n)=0$.
	
	D'altra parte, fissato $\varepsilon > 0$, esiste $n_0$ tale che per ogni $n,m\ge n_0$ vale $\delta(K_n,K_m)\le \varepsilon$, e perciò ricavo:
	\begin{equation*}
		\forall n\ge n_0:\ \delta_{K_n}(A_n)=\delta_{K_n}\left(\overline{\bigcup_{i\ge n} K_i}\right)
		=\sup_{i\ge n}\delta_{K_n}(K_i)\le \varepsilon
	\end{equation*}
	
	Perciò unendo le ultime due affermazioni ottengo $\delta(K_n,A_n)\to 0$, e questo implica (come fatto generale riguardo le successioni in un metrico) che $\lim_{n\to\infty}K_n=\lim_{n\to\infty}A_n=A$ e quindi ho dimostrato che la generica successione di Cauchy $K_n$ converge, ottenendo la tesi.
\end{proof}

\begin{remark}\label{CompletezzaInverso}
Se $\left (\CL(X),\delta \right ) $ è completo lo è anche $(X,d)$.
\end{remark}
\begin{proof}
Per quanto dimostrato in \cref{IsometriaCanonica} $\varphi(X)$ è un chiuso di $\CL(X)$, che però è per ipotesi un completo, perciò $\varphi(X)$ è un completo. Ma ancora sfruttando quanto detto in \cref{IsometriaCanonica} so che $(\varphi(X),\delta)$ è isometrico a $(X,d)$ e quindi ne ricavo che anche $(X,d)$ è un completo.
\end{proof}
