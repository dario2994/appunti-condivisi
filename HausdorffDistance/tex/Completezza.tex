\section{Trasmissione della completezza}



In questa sezione dimostrerò che se $(X,d)$ è completo allora lo è anche $(\CL(X),\delta) $.

\begin{lemma} \label {inglobati}
Sia $\left (X,d\right )$ uno spazio metrico completo e $\left (K_n \right )$ una successione di Cauchy di chiusi e limitati in $\left (X,d\right )$ tali che $K_{n+1}\subseteq K_n$ $ \forall n \in \mathbb{N}$. Allora la successione ammette limite in $\left (\CL (X),\delta\right  ) $ uguale a $ K=\bigcap_{n \in \mathbb{N}}K_n $.
\end{lemma}
\begin{proof}
	È ovvio che $K$ è a sua volta un chiuso e limitato.
	
	Assumendo che, fissato $\varepsilon>0$, per $n$ sufficientemente grande, per ogni $\bar{x}\in K_n$ esista un elemento $x\in K$ tale che $d(\bar{x},x)\le \varepsilon$, ricavo in particolare $d_K(\bar{x})\le \varepsilon$ e perciò: 
	\begin{equation*}
		\delta(K,K_n)=\max\left\{\delta_{K_n}(K),\delta_K(K_n)\right\}=\max\left\{0,{}\sup_{\bar{x}\in K_n} d_K(\bar x)\right\}\le \varepsilon
	\end{equation*}
	perciò ho proprio $K_n\to K$.
	
	Non resta ora che dimostrare l'assunzione fatta. Sia $N$ tale che per ogni $n,m\ge N$ vale $\delta(K_n,K_m) < \frac{\varepsilon}2$, tale $N$ esiste poichè $(K_n)$ è di Cauchy. Fisso $\bar{n}\ge N$ e $\bar x\in K_{\bar n}$.
	
	Definisco la successione $(n_i)$ di modo che valgano le seguenti (anche questa esiste perchè $(K_n)$ è di Cauchy):
	\begin{equation*}
	\begin{cases}
		n_1=\bar n\\
		\forall i\ge 1:\ n_i<n_{i+1} \\
		\forall n\ge n_i:\ \delta(K_n,K_{n_i}) < \frac{\varepsilon}{2^i}
	\end{cases}\end{equation*}
	e definisco anche la successione $(x_n)_{n\ge \bar n}$ in modo che:
	\begin{equation*}\begin{cases}
		x_{\bar n}=\bar x\\
		x_n\in K_n\\
		n_i<n\le n_{i+1}\Longrightarrow d(x_{n_i},x_n)\le \frac{\varepsilon}{2^i}
	\end{cases}\end{equation*}
	Riesco sempre a trovare $x_n$ come richiesto poichè la condizione $\delta(K_n,K_{n_i}) < \frac{\varepsilon}{2^i}$, sfruttando \cref{ApprossimazioneDistanzaAsimmetrica}, me ne assicura l'esistenza.
	
	Allora ora vale la seguente stima:
	\begin{align*}
		n_i<n\le n_{i+1}\Longrightarrow d(\bar x, x_n)&\le d(\bar x, x_2)+d(x_2,x_3)+\cdots+d(x_{n_{i-1}},x_{n_i})+
		d(x_{n_i},x_n)\\
		&\le\frac{\varepsilon}2+\frac{\varepsilon}4+\cdots+\frac{\varepsilon}{2^{i-1}}
		+\frac{\varepsilon}{2^i}<\varepsilon
	\end{align*}
	e con un ragionamento del tutto analogo si ricava che la successione è definitivamente contenuta nella palla $B_{\frac{\varepsilon}{2^{i-1}}}(x_{n_i})$ e perciò è di Cauchy. 
	Ma $X$ è completo per ipotesi, quindi $x_n\to x\in X$ e $x\in K_n$ per ogni $n$ poichè definitivamente $x_i\in K_i \subseteq K_n$ e $K_n$ è chiuso. Perciò $x\in K$.
	
	Inoltre, poichè $d(\bar x,x_n)\le \varepsilon$, risulta che $d(\bar x, x)\le \varepsilon$ e per quanto appena detto implica che ho trovato, come cercavo, un punto in $K$ che dista meno di $\varepsilon$ da $\bar x$.
	
\end{proof}

\begin{lemma} \label{UnioniDiCauchy}
Sia $(K_n)$ una successione di Cauchy di chiusi e limitati in $\left (\CL(X),\delta\right  ) $. 
Allora $\forall k \in \mathbb{N}$ vale che
\begin{equation*}
A_k:=\overline{\bigcup_{n\geq k} K_n}
\end{equation*}
è chiuso e limitato, ed inoltre gli $A_k$ formano a loro volta una successione di Cauchy.
\end{lemma}

\begin{proof}
Per definizione $A_k$ è chiuso per ogni $k$. 

Dato $\varepsilon$, esiste $n_0$ tale che $\delta(K_{n_0}, K_n)< \varepsilon$ per ogni $n\geq n_0$. 
Inoltre siano $x_0\in X,r>0$, che esistono per la limitatezza di $K_{n_0}$, tali che $K_{n_0}\subseteq B_r(x_0)$. 
Allora, con una facile applicazione della disuguaglianza triangolare ottengo:
\begin{equation*}
	\forall n\ge n_0:\ K_n\subseteq B_{r+\varepsilon}(x_0) \Rightarrow A_{n_0}\subseteq \overline{B_{r+\varepsilon}(x_0)}
\end{equation*}
Quindi $A_{n_0}$ è limitato, ma allora lo è anche $A_1$ visto che lo posso esprimere come $\bigcup_{n<n_0}K_n \bigcup A_{n_0}$ e unione finita di limitati è limitata. Da questo discende che $A_k$ è limitato per ogni $k$ visto che $A_k\subseteq A_1$.

Unendo quanto detto ottengo che gli $A_k$ sono tutti chiusi e limitati.

Se $\delta(K_n, K_m)\leq \varepsilon$ $\forall n,m \geq n_0$  allora vale:
\begin{equation*}
	\forall n,m\ge n_0:\ \delta_{A_n}(A_m)\le \delta_{K_n}\left(\overline{\bigcup_{i\ge m}K_i} \right)
	=\sup_{i\ge m} \delta_{K_n}(K_i) \le \varepsilon
\end{equation*}
E questo dimostra che $(A_k)$ è una successione di Cauchy.
\end{proof}

\begin{theorem} \label{CompletezzaCL}
	Se $(X,d)$ è uno spazio metrico completo allora anche $(\CL(X),\delta)$ lo è.
\end{theorem}
\begin{proof}
	Sia $(K_n)$ una successione di Cauchy in $(\CL(X),\delta)$.
	Per \cref{UnioniDiCauchy} gli insiemi $A_n$ (definiti come nell'enunciato del lemma) formano una successione di Cauchy.
	Inoltre vale ovviamente $A_{n+1}\subseteq A_n$ quindi posso applicare \cref{inglobati} e ottenere che la successione $(A_n)$ converge ad $A\in\CL(X)$.
	
	Vale però che $K_n\subseteq A_n$ e quindi $\delta_{A_n}(K_n)=0$.
	
	D'altra parte, fissato $\varepsilon > 0$, esiste $n_0$ tale che per ogni $n,m\ge n_0$ vale $\delta(K_n,K_m)\le \varepsilon$, e perciò ricavo:
	\begin{equation*}
		\forall n\ge n_0:\ \delta_{K_n}(A_n)=\delta_{K_n}\left(\overline{\bigcup_{i\ge n} K_i}\right)
		=\sup_{i\ge n}\delta_{K_n}(K_i)\le \varepsilon
	\end{equation*}
	
	Perciò unendo le ultime due affermazioni ottengo $\delta(K_n,A_n)\to 0$, e questo implica (come fatto generale riguardo le successioni in un metrico) che $\lim_{n\to\infty}K_n=\lim_{n\to\infty}A_n=A$ e quindi ho dimostrato che la generica successione di Cauchy $K_n$ converge, ottenendo la tesi.
\end{proof}

\begin{remark}\label{CompletezzaInverso}
Se $\left (\CL(X),\delta \right ) $ è completo lo è anche $(X,d)$.
\end{remark}
\begin{proof}
Per quanto dimostrato in \cref{IsometriaCanonica} $\varphi(X)$ è un chiuso di $\CL(X)$, che però è per ipotesi un completo, perciò $\varphi(X)$ è un completo. Ma ancora sfruttando quanto detto in \cref{IsometriaCanonica} so che $(\varphi(X),\delta)$ è isometrico a $(X,d)$ e quindi ne ricavo che anche $(X,d)$ è un completo.
\end{proof}
