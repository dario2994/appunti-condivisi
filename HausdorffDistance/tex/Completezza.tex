\section{Trasmissione della completezza}



In questa sezione dimostrerò che se $\left (X,d\right )$ è completo allora lo è anche $\left (\CL(X),\delta\right  ) $.


\begin{lemma} \label {inglobati}
Sia $\left (X,d\right )$ uno spazio metrico completo e $\left (K_n \right )$ una successione di Cauchy di chiusi e limitati in $\left (X,d\right )$ tali che $K_{n+1}\subseteq K_n$ $ \forall n \in \mathbb{N}$. Allora la successione ammette limite in $\left (\CL (X),\delta\right  ) $ uguale a $ K=\bigcap_{n \in \mathbb{N}}K_n $.
\end{lemma}

\begin{proof}
Dimostro innanzitutto che l'intersezione dei chiusi e limitati è non è vuota, costruendo una successione di Cauchy $\left (x_n\right ) \subseteq X$ tale che $x_n \in K_n$ $ \forall n \in \mathbb{N}$. Essendo la successione dei chiusi e limitati di Cauchy, per ogni $n$ esiste $\varepsilon_n$ tale che $\delta \left ( K_n, K_m \right )\leq \varepsilon_n$ $\forall m\geq n$ e
\begin{equation*}
\lim_{n \to \infty}\varepsilon_n =0;
\end{equation*}
infatti basta prendere $n_k$ tale che $\delta \left ( K_m, K_l \right )\leq \frac{1}{2^k}$ $\forall m,l\geq n_k$ e ad ogni $n$ associo $\varepsilon_n=\frac{1}{2^j}$ in modo che $j=\max { \left \{ i:n_i \leq n \right \} }$. Per i primi termini della successione che voglio costruire faccio rispettare solo l'appartenenza sopra enunciata. Invece, a partire da $n=n_1$, una volta definiti tutti i termini fino a $x_n$ so che la palla di raggio $2\varepsilon_n$ centrata in $x_n$ interseca tutti i chiusi e limitati successivi, e dunque posso costruire induttivamente una successione di Cauchy. Dato che $\left (X,d\right )$ è completo, esiste il limite $x$ della successione. Per ogni $n$ la successione è contenuta definitivamente in $K_n$, ergo per chiusura il limite appartiene a $ \bigcap_{n \in \mathbb{N}}K_n $, che di conseguenza non è vuota (e chiaramente chiusa e limitata a sua volta).

Dato che per ogni $n$ $K_n\supseteq K$ allora ovviamente $\delta\left  (K_n,K\right )=\delta_{K}\left (K_n\right )$. Inoltre $\delta_{K}\left (K_n\right )$ è decrescente in $n$, quindi esiste il limite $r$ della distanza di Haurdorff:
\begin{equation*}
r=\lim_{n \to \infty}\delta_{K}\left (K_n\right )>0.
\end{equation*}
Considero la sottosuccessione $K_{n(k)}$, dove $n(k)$ è tale che $\delta \left ( K_m, K_l \right )\leq \frac{r}{2^{k+2}}$ $\forall m,l\geq n(k)$, e costruisco una successione $\left ( x_{n(k)} \right )$ tale che $d_{K} \left (x_{n(1)} \right )\geq r$, $ x_{n(k)} \in K_{n(k)}$ e che sia di Cauchy, ovvero in modo che $d \left ( x_{n(l)}, x_{n(m)} \right )\leq \frac{r}{2^{k+1}}$ da $n(k)$ in poi (lo faccio come sopra). Dato che la serie
\begin{equation*}
\sum_{i=2}^{\infty} \frac{r}{2^i}
\end{equation*}
converge a $\frac{r}{2}$, allora la distanza del limite $x$ da $K$ è maggiore o uguale a $\frac{r}{2}$. Ma come prima il limite appartiene all'intersezione, dunque si ha un assurdo.
\end{proof}

\begin{theorem}
Sia $(K_n)$ una successione di Cauchy di chiusi e limitati in $\left (\CL(X),\delta\right  ) $, dove $\left (X, d\right )$ è uno spazio metrico completo. Allora $\forall k \in \mathbb{N}$ vale che
\begin{equation*}
B_k:=\overline{\bigcup_{n\geq k} K_n}
\end{equation*}
è chiuso e limitato, ed inoltre i $B_k$ formano a loro volta una successione di Cauchy.
\end{theorem}

\begin{proof}
 Per definizione $B_k$ è chiuso per ogni $k$. Inoltre, dato $\varepsilon$, esiste $n_0$ tale che $\delta\left (K_{n_0}, K_n\right )< \varepsilon$ per ogni $n\geq n_0$. Quindi se $K_{n_0}$ è contenuto in una palla di raggio $R$, $\bigcup_{n\geq n_0} K_n$ è contenuta nella stessa palla di raggio $R+\varepsilon$. Inoltre aumentando ulteriormente il raggio posso sicuramente ricoprire anche i chiusi e limitati precedenti, che sono in numero finito. Infine raddoppiando il raggio finale sono sicuro di non perdermi la chiusura.

Inoltre se $\delta\left  (K_{n_0}, K_n\right )\leq \varepsilon$ $\forall n \geq n_0$  allora l'$\varepsilon$-intorno di $K_{n_0}$ comprende l'unione dei successivi, il che, considerando che le eventuali chiusure non influiscono (se non rendendo non strette le disuguaglianze), implica che anche i $B_k$ sono di Cauchy.
\end{proof}

Dato che per ogni $k\in \mathbb{N}$ $B_{k+1}\subseteq B_k$ per il teorema appena dimostrato e per il \cref{inglobati} esiste $B=\lim_{k \to \infty} B_k$.

Adesso dimostro che la successione di chiusi e limitati ha limite, mostrando che esso non è altro che $B$.

\begin{theorem}
Sia $\left (K_n\right )$ una successione di Cauchy in $\left (\CL(X),\delta \right ) $, dove $\left (X, d\right )$ è uno spazio metrico completo. Allora $\lim_{n \to \infty}K_n=B$, ove
\begin{equation*}
B=\lim_{k \to \infty} \left(\overline{\bigcup_{n\geq k} K_n} \right).
\end{equation*}
\end{theorem}

\begin{proof}
Dato $\varepsilon$, esiste $n_0$ tale che $\delta\left  (K_{n_0}, K_n\right )< \varepsilon$ $\forall n\geq n_0$, il che implica $\delta _{K_{n_0}}\left  (K_n\right )<\varepsilon$ $\forall n\geq n_0$. Ma allora vale anche che $\delta _{K_{n_0}}\left  (\cup_{n\geq n_0} K_n\right )<\varepsilon$, e anche che $\delta _{K_{n_0}}\left  (\overline {\cup_{n\geq n_0} K_n} \right ) \leq \varepsilon$. Dunque, dato che per tutti gli $n$ $K_n \subseteq B_n$,  definitivamente $\delta\left  (K_n, B_n\right ) \leq \varepsilon$. Ma allora accade anche che $\delta\left  (K_n, B\right ) \to 0$.
\end{proof}

Dunque in uno spazio metrico completo $\left (X,d\right )$ vale effettivamente che una successione di Cauchy di chiusi e limitati converge ad un chiuso e limitato, ovvero che $\left (\CL(X),\delta \right ) $ è a sua volta completo.


Si può dimostrare anche il viceversa:
\begin{theorem}
Sia  $\left (X,d\right )$ uno spazio metrico e $\left (\CL(X),\delta \right ) $ lo spazio metrico dei chiusi e limitati. Allora se $\left (\CL(X),\delta \right ) $ è completo lo è anche $\left (X,d\right )$.
\end{theorem}
\begin{proof}
Sia $\left( x_n \right)$ una successione di Cauchy di $\left (X,d\right )$. Dato che $\forall y,z \in X$ vale che $d\left( y,z\right)=\delta \left(\left \{y\right \},\left \{z \right \} \right)$, allora anche $\left(\left \{ x_n\right \} \right)\subseteq \mathcal{K}(X)$ è una successione di Cauchy in  $\left (\CL(X),\delta \right ) $. Allora per ipotesi la successione dei singoletti tende ad un chiuso e limitato, che può essere soltanto a sua volta un singoletto $\left \{x \right \}$. Ma allora, dato che $\delta \left ( \left \{ x_n\right \},\left \{x \right \} \right ) \to 0$, è anche vero che $d\left (  x_n,x  \right ) \to 0$.
\end{proof}
