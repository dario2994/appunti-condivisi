\section{Trasmissione della completezza}



In questa sezione dimostrerò che se $\left (X,d\right )$ è completo allora lo è anche $\left (\mathcal{K}(X),\delta\right  ) $.


\begin{lemma} \label {pereppeppeppe}
Sia $K_n$ una successione di compatti in $\left (X,d\right )$ tali che $K_{n+1}\subseteq K_n$ $ \forall n \in \mathbb{N}$. Allora la successione ammette limite in $\left (\mathcal{K}(X),\delta\right  ) $ uguale a $ \bigcap_{n \in \mathbb{N}}K_n $.
\end{lemma}

\begin{proof}
Dimostro innanzitutto che l'intersezione dei compatti non è vuota. Considero una successione $\left (x_n\right ) \subseteq X$ tale che $x_n \in K_n$ $ \forall n \in \mathbb{N}$. In particolare  $x_n \in K_1$ $ \forall n \in \mathbb{N}$ e dunque esiste una sottosuccessione  $\left (x_n(k)\right )$ convergente a $x\in K_1$. Per ogni $n$  $x_{n(k)}\in K_n$ definitivamente, ergo $x\in K_n$ per la chiusura in $X$ di $K_n$. Di conseguenza $x \in K_n$ $ \forall n \in \mathbb{N}$, il che implica che l'intersezione dei compatti è non vuota.

Dato che per ogni $n$ $K_n\supseteq K$ allora ovviamente $\delta\left  (K_n,K\right )=\delta_{K}\left (K_n\right )$. Inoltre $\delta_{K}\left (K_n\right )$ è decrescente in $n$, quindi $ \forall n \in \mathbb{N}$ $\exists x_n \in K_n$ tale che $d_K\left (X_n\right )\geq r$ , dove 
\begin{equation*}
r=\lim_{n \to \infty}\delta_{K}\left (K_n\right )>0.
\end{equation*}

Dunque esiste una sottosuccessione $x_{n(k)}$ convergente a $x\in K_1$, ovvero, per lo stesso argomento precedente, a  $x\in K$. Ma allora 
\begin{equation*}
\lim_{n(k) \to \infty}d(x_{n(k)},x)=0,
\end{equation*}
contraddicendo il fatto che
\begin{equation*}
\lim_{n(k) \to \infty}\delta_{K}(K_{n(k)})=r>0.
\end{equation*}
\end{proof}

\begin{theorem}
Sia $(K_n)$ una successione di Cauchy in $\left (\mathcal{K}(X),\delta\right  ) $, dove $\left (X, d\right )$ è uno spazio metrico completo. Allora $\forall k \in \mathbb{N}$ vale che
\begin{equation*}
B_k:=\overline{\bigcup_{n\geq k} K_n}
\end{equation*}
è compatto.
\end{theorem}

\begin{proof}
In uno spazio metrico la compattezza per successioni è equivalente a completezza e totale limitatezza. Dunque mi basta dimostrare quest'ultima proprietà. 

Dato che lo spazio metrico è completo e $B_k$ è chiuso, allora $B_k$ è anche completo.

Ora mi resta da dimostrare che, dato $\varepsilon>0$, esiste un insieme finito di palle di raggio $\varepsilon$ che ricopre tutto $B_k$. Dato che i compatti formano una successione di Cauchy, esiste $n_0$ tale che $\delta\left (K_{n_0}, K_n\right )< \frac{\varepsilon}{4}$. Inoltre esiste un numero finito di palle di raggio $\frac{\epsilon}{4}$ che ricopre $K_{n_0}$. Ma allora se raddoppio il raggio delle palle considerate ricopro $\cup_{n\geq n_0} K_n$; infatti per ogni $x \in K_n$ c'è un elemento di $K_{n_0}$ a distanza minore di $\frac{\varepsilon}{4}$, il quale vede un centro $y$ di almeno una delle palle a distanza minore di $\frac{\varepsilon}{4}$, da cui $d(x,y)<\frac{\varepsilon}{2}$.

Inoltre mi basta un numero finito di palle di raggio $\frac{\varepsilon}{2}$ per ricoprire $\cup_{k \leq n< n_0} K_n$, visto che sono un numero finito di compatti. Ora se raddoppio il raggio di tutte le palle di sicuro mi vado a prendere anche la chiusura di $\cup_{n\geq k} K_n$, e dunque ho trovato un ricoprimento finito di palle di raggio $\varepsilon$.
\end{proof}

Dato che per ogni $k\in \mathbb{N}$ $B_{k+1}\subseteq B_k$ per il teorema appena dimostrato e per il \cref{pereppeppeppe} esiste $B=\lim_{k \to \infty} B_k$.

Adesso dimostro che la successione di compatti ha limite, mostrando che esso non è altro che $B$.

\begin{theorem}
Sia $\left (K_n\right )$ una successione di Cauchy in $\left (\mathcal{K}(X),\delta \right ) $, dove $\left (X, d\right )$ è uno spazio metrico completo. Allora $\lim_{n \to \infty}K_n=B$, ove
\begin{equation*}
B=\lim_{k \to \infty} \left(\overline{\bigcup_{n\geq k} K_n} \right).
\end{equation*}
\end{theorem}

\begin{proof}
Dato $\varepsilon$, esiste $n_0$ tale che $\delta\left  (K_{n_0}, K_n\right )< \varepsilon$ $\forall n\geq n_0$, il che implica $\delta _{K_{n_0}}\left  (K_n\right )<\varepsilon$ $\forall n\geq n_0$. Ma allora vale anche che $\delta _{K_{n_0}}\left  (\cup_{n\geq n_0} K_n\right )<\varepsilon$, e anche che $\delta _{K_{n_0}}\left  (\overline {\cup_{n\geq n_0} K_n} \right ) \leq \varepsilon$. Dunque, dato che per tutti gli $n$ $K_n \subseteq B_n$,  definitivamente $\delta\left  (K_n, B_n\right ) \leq \varepsilon$. Ma allora accade anche che $\delta\left  (K_n, B\right ) \to 0$.
\end{proof}

Dunque in uno spazio metrico completo $\left (X,d\right )$ vale effettivamente che una successione di Cauchy di compatti converge ad un compatto, ovvero che $\left (\mathcal{K}(X),\delta \right ) $ è a sua volta completo.


Si può dimostrare anche il viceversa:
\begin{theorem}
Sia  $\left (X,d\right )$ uno spazio metrico e $\left (\mathcal{K}(X),\delta \right ) $ lo spazio metrico dei compatti. Allora se $\left (\mathcal{K}(X),\delta \right ) $ è completo lo è anche $\left (X,d\right )$.
\end{theorem}
\begin{proof}
Sia $\left( x_n \right)$ una successione di Cauchy di $\left (X,d\right )$. Dato che $\forall y,z \in X$ vale che $d\left( y,z\right)=\delta \left(\left \{y\right \},\left \{z \right \} \right)$, allora anche $\left(\left \{ x_n\right \} \right)\subseteq \mathcal{K}(X)$ è una successione di Cauchy in  $\left (\mathcal{K}(X),\delta \right ) $. Allora per ipotesi la successione dei singoletti tende ad un compatto, che può essere soltanto a sua volta un singoletto $\left \{x \right \}$. Ma allora, dato che $\delta \left ( \left \{ x_n\right \},\left \{x \right \} \right ) \to 0$, è anche vero che $d\left (  x_n,x  \right ) \to 0$.
\end{proof}
