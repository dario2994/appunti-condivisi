\section{Stabilità della connessione per passaggio al limite}
In questa sezione conclusiva noteremo che nello spazio $(\CL(X),\delta)$ la proprietà di connessione può andare perduta per passaggio al limite, mentre ciò non succede se si lavora nello spazio dei compatti $(\K(X),\delta)$.

\begin{theorem}
	Sia $(K_n)$ una successione di chiusi e limitati connessi che converge a $K\in\CL(X)$. Il chiuso e limitato $K$ può non essere connesso.
\end{theorem}
\begin{proof}
	Mostriamo un controesempio all'affermazione.
	
	Considero i sottinsiemi di $\mathbb R^2$ così definiti:
	\begin{align*}
		A&=\left\{(x,y)\in\mathbb R^2:\ y=0\right\}\\
		B&=\left\{(x,y)\in\mathbb R^2:\ x\ge 0\ \wedge\ y=\frac 1x \right\}
	\end{align*}
	cioè si tratta dell'asse delle ascisse e del grafico, nel primo quadrante, della funzione inverso.
	
	Ovviamente $A,B$ sono chiusi e connessi ed altrettanto ovviamente non sono limitati e $A\cup B$ non è connesso.
	
	Per rendere $A\cup B$ connesso consideriamo la seguente successione di sottinsiemi chiusi di $\mathbb R^2$:
	\begin{equation}
		C_n=\left\{(x,y)\in\mathbb R^2:\ x=n\ \wedge\ 0\le y\le \frac 1n\right\}
	\end{equation}
	cioè un segmentino verticale di lunghezza $\frac 1n$ che connette $A$ e $B$ (li connette banalmente per archi, quindi anche per aperti).
	
	Infine definiamo $K_n=A\cup B\cup C_n$. Per quanto detto la successione $(C_n)$ è formata da sottoinsiemi chiusi e connessi di $\mathbb R^2$.
	
	Per rendere gli elementi della successione anche limitati scegliamo una metrica differente per $\mathbb R^2$, in modo che tutto lo spazio sia limitato ma la topologia non cambi. Per farlo basta scegliere il minimo tra la distanza canonica ed il valore $1$. Chiamiamo $d:\mathbb R^2\to [0,1]$ questa nuova distanza.
	
	Allora ora $(K_n)$ è una successione di chiusi e limitati connessi di $(\mathbb R^2, d)$ (visto che le proprietà di chiusura e connessione sono prettamente topologiche).
	
	Per concludere dimostriamo che la successione $(K_n)$ converge a $A\cup B$ vista come successione in $(\CL(\mathbb R^2),\delta)$, e questo chiuderebbe il controesempio visto che, come si cercava, $A\cup B$ non è un connesso.
	
	Notiamo che vale $A\cup B\in K_n$ e perciò $\delta_{K_n}(A\cup B)=0$. 
	Inoltre fissato $x\in K_n$ o questo appartiene ad $A\cup B$ e perciò $d_{A\cup B}(x)=0$ oppure appartiene a $C_n$. In quest'ultimo caso però vale, per definizione di $C_n$, $d_{A\cup B}(x)\le d\left((n,0),x\right)\le \frac 1n$. Riassumendo giungiamo a:
	\begin{gather*}
		\delta_{A\cup B}(K_n)=\sup_{x\in K_n} d_{A\cup B}(x)\le \frac 1n \\
		\Downarrow \\
		\delta(A\cup B,K_n)=\max(\delta_{K_n}\left(A\cup B),\delta_{A\cup B}(K_n)\right)\le \max\left(0,\frac1n\right)=\frac 1n
	\end{gather*}
	e quindi abbiamo ottenuto come volevamo che $K_n\to A\cup B$ visto che la distanza $\frac 1n$ tende a $0$.
\end{proof}

\begin{lemma}\label{CompattoInAperto}
	Siano $K,A$ rispettivamente un compatto ed un aperto di uno spazio metrico $(X,d)$ tali che $K\subseteq A$. Allora esiste $r>0$ tale che $U_r(K)\subseteq A$ dove $U_r$ è l'operatore definito nell'enunciato del \cref{EquivalenzaAllargamento}.
\end{lemma}
\begin{proof}
	Assumiamo per assurdo sia falsa la tesi, allora esiste una successione di coppie $(k_n,b_n)\in K\times A^c$ tali che $d(k_n,b_n)\le \frac 1n$.
	Sfruttando la compattezza di di $K$ possiamo assumere inoltre, a meno di estrarre una sottosuccessione, che $k_n$ sia una successione convergente a $k\in K$.
	
	Per una facile applicazione della disuguaglianza triangolare otteniamo:
	\begin{equation}
		d(k,b_n)\le d(k,k_n)+d(k_n,b_n)\le d(k,k_n)+\frac 1n \implies d(k,b_n)\to 0
	\end{equation}
	quindi la successione $(b_n)\in A^c$ converge a $k\in K\subseteq A$, ma questo mostra l'assurdo visto che abbiamo trovato una successione di elementi nel complementare di un aperto che converge nell'aperto.

\end{proof}



\begin{theorem}
	Sia $(K_n)$ una successione di compatti connessi nello spazio metrico $(X,d)$ che converge a $K\in\K(X)$ nella metrica di Hausdorff. Allora il compatto $K$ è connesso a sua volta.
\end{theorem}
\begin{proof}
	Assumiamo per assurdo che $K$ non sia connesso, allora esistono due aperti $A_1,A_2$ disgiunti, entrambi con intersezione non nulla con $K$, tali che $K\subseteq A\cup B$.
	
	Innanzitutto $A\cap K, B\cap K$ sono ancora dei compatti. Questo perchè presa una successione $(a_n)$ in $A\cap K$, so che esiste una sottosuccessione che converge a $k\in K$, ma questa deve anche convergere nel complementare di $B$ visto che $B$ è un aperto e gli $a_n$ non vi appartengono. Quindi $k\in K\setminus B = K\cap A$ e questo dimostra quanto voluto.
	
	Ora per il \cref{CompattoInAperto} sappiamo che esistono $r_A,r_B$ tali che $U_{r_A}(K\cap A)\subseteq A$ e $U_{r_B}(K\cap B)\subseteq B$. Chiamiamo $r=\min(r_A,r_B)$.
\end{proof}



