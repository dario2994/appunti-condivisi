\section{Stabilità della connessione per passaggio al limite}
In questa sezione conclusiva noteremo che nello spazio $(\CL(X),\delta)$ la proprietà di connessione può andare perduta per passaggio al limite, mentre ciò non succede se si lavora nello spazio dei compatti $(\K(X),\delta)$.

\begin{theorem}
	Sia $(K_n)$ una successione di chiusi e limitati connessi che converge a $K\in\CL(X)$. Il chiuso e limitato $K$ può non essere connesso.
\end{theorem}
\begin{proof}
	Mostriamo un controesempio all'affermazione.
	Considero i sottinsiemi di $\mathbb R^2$ così definiti:
	\begin{align*}
		A&=\left\{(x,y)\in\mathbb R^2:\ y=0\right\}\\
		B&=\left\{(x,y)\in\mathbb R^2:\ x\ge 0\ \wedge\ y=\frac 1x \right\}
	\end{align*}
	cioè si tratta dell'asse delle ascisse e del grafico, nel primo quadrante, della funzione inverso.
	
	Ovviamente $A,B$ sono chiusi e connessi ed altrettanto ovviamente non sono limitati e $A\cup B$ non è connesso.
	
	Per rendere $A\cup B$ connesso consideriamo la seguente successione di sottinsiemi chiusi di $\mathbb R^2$:
	\begin{equation}
		C_n=\left\{(x,y)\in\mathbb R^2:\ x=n\ \wedge\ 0\le y\le \frac 1n\right\}
	\end{equation}
	cioè un segmentino verticale di lunghezza $\frac 1n$ che connette $A$ e $B$ (li connette banalmente per archi, quindi anche per aperti).
	
	Infine definiamo $K_n=A\cup B\cup C_n$. Per quanto detto la successione $(C_n)$ è formata da sottoinsiemi chiusi e connessi di $\mathbb R^2$.
	
	Per rendere gli elementi della successione anche limitati scegliamo una metrica differente per $\mathbb R^2$, in modo che tutto lo spazio sia limitato ma la topologia non cambi. Per farlo basta scegliere il minimo tra la distanza canonica ed il valore $1$. Chiamiamo $d:\mathbb R^2\to [0,1]$ questa nuova distanza.
	
	Allora ora ovviamente $(K_n)$ è una successione di chiusi e limitati connessi di $(\mathbb R^2, d)$ (visto che le proprietà di chiusura e connessione sono prettamente topologiche).
	
	Per concludere dimostriamo che la successione $(K_n)$ converge a $A\cup B$ vista come successione in $(\CL(\mathbb R^2),\delta)$, e questo chiuderebbe il controesempio visto che, come si cercava, $A\cup B$ non è un connesso.
	
	Notiamo che vale $A\cup B\in K_n$ e perciò $\delta_{K_n}(A\cup B)=0$. 
	Inoltre fissato $x\in K_n$ o questo appartiene ad $A\cup B$ e perciò $d_{A\cup B}(x)=0$ oppure appartiene a $C_n$. In quest'ultimo caso però vale, per definizione di $C_n$, $d_{A\cup B}(x)\le d\left((n,0),x\right)\le \frac 1n$. Riassumendo perciò giungo a:
	\begin{gather*}
		\delta_{A\cup B}(K_n)=\sup_{x\in K_n} d_{A\cup B}(x)\le \frac 1n \\
		\Downarrow \\
		\delta(A\cup B,K_n)=\max(\delta_{K_n}\left(A\cup B),\delta_{A\cup B}(K_n)\right)\le \max\left(0,\frac1n\right)=\frac 1n
	\end{gather*}
	e quindi ho ottenuto come volevo che $K_n\to A\cup B$ visto che la distanza $\frac 1n$ tende a $0$.
\end{proof}

\begin{theorem}
	Sia $(K_n)$ una successione di compatti connessi nello spazio metrico $(X,d)$ che converge a $K\in\K(X)$ nella metrica di Hausdorff. Allora il compatto $K$ è connesso a sua volta.
\end{theorem}


