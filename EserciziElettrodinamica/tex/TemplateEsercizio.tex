\documentclass[../main.tex]{subfiles} 
\begin{document}

\exercise{Nome del problema} %ndp

\textex
Testo del problema

Lorem Ipsum e così via finchè la capra canta la capra suona anche.

Fine Testo del problema.

\solution
Qui ci va la soluzione.
Ecco alcuni simboli tipici dell'elettrodinamica in Latex (ovviamente basta che leggiate il codice per capire come si fa):
\begin{gather*}
	\div\vec E \\
	\lapl V \\
	-\grad V=\vec E \\
	\frac{\partial \vec E}{\partial n}\\
	\frac{\de f}{\de t}\\
	4\pi\int_{\Omega} \rho \de\vec r=\int_{\partial\Omega} \vec E\cdot\hat n\de S\\
	\rot \vec E=0 \punto
\end{gather*}
Fine della soluzione.

\solution[2]
Ecco un'altra soluzione.

Un po' di boiate a parole, poi qualche formuletta:
\begin{equation}\label{ndp:gamma}
	\Gamma(x)=\lim_{n\to\infty} \dfrac{n^xn!}{x(x+1)\cdots (x+n-1)(x+n)}
\end{equation}
Poi un'equazione senza numerino:
\begin{equation*}
	a=b \Rightarrow b=c \Rightarrow 1=0
\end{equation*}

E poi, tutto soddifatto, richiamo la \cref{ndp:gamma}, che ovviamente conclude.

\solution[3]
Qui ci andrebbe la terza soluzione.


\end{document}
