\documentclass[../main.tex]{subfiles} 
\begin{document}

\exercise{Potenziale di un'asticella carica} %pac

\textex

È data un'asta di lunghezza $2a$ posta lungo l'asse $z$ con centro nell'origine. L'asta è uniformemente carica con densità lineare $\lambda$. Determinare i primi 2 termini dello sviluppo per $r>>a$.

\solution

Consideriamo, in coordinate sferiche, lo sviluppo del potenziale in serie. Dato che il sistema è invariante per rotazioni lungo l'asse $z$, tale sviluppo sarà dato da una serie contenente i polinomi di Legendre. Inoltre, dato che le cariche sono localizzate, possiamo imporre che $V(r)\rightarrow 0$ per $r\rightarrow \infty$. Fatte queste considerazioni preliminari, abbiamo ottenuto che il potenziale si esprime
\[
	V(r,\theta)=\sum_{l=0}^{\infty} B_l\frac{1}{r^{l+1}}P_l(cos \theta)\virgola
\]
con opportuni coefficienti $B_l$.

Osserviamo che il sistema è simmetrico per riflessioni lungo l'asse $z$. In particolare $V(r,\theta)=V(r,\pi-\theta)$. Questo ci dice che in particolare sono nulli tutti i termini $B_l$ con $l$ dispari.

Passiamo quindi a calcolare i primi termini. $B_0$ è il termine di monopolo, quindi fornisce il potenziale generato dal sistema in cui tutta la carica è concentrata nell'origine. Questo ci dice che $B_0=2a\lambda$. Per quanto riguarda il secondo termine da calcolare, cioè $B_2$, calcoliamo il potenziale lungo l'asse $z$:
\[
	V(z)=\int_{-a}^a \frac{\lambda}{z-s}ds=\lambda [-ln(z-s)]_{-a}^a
\]


\end{document}
