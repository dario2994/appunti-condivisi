\documentclass[../main.tex]{subfiles} 
\begin{document}

\exercise[29 ottobre 2014]{Potenziale a simmetria sferica}

\textex

Si determini la distribuzione di carica $\rho(\vec{r})$ che induce il potenziale
\begin{displaymath}
V(\vec{r})=Q\frac{1}{r}e^{-\frac{r}{l}}.
\end{displaymath}


\solution

 Al di fuori dell'origine il potenziale è senza dubbio differenziabile, dunque si può applicare la legge di Gau\ss\ in forma
 differenziale, ricordandosi che, essendo il potenziale a simmetria sferica, vale che
 $\lapl V=\frac{1}{r}\frac{\partial^2}{\partial r^2}rV$:
 
 $$-4\pi\rho(\vec{r})=\lapl V=\frac{1}{r}\frac{\partial^2}{\partial r^2}rV=
 \frac{1}{r}\frac{\partial^2 }{\partial r^2}Qe^{-\frac{r}{l} }=\frac{Q}{l^2}\frac{1}{r}e^{- \frac{r}{l}}$$
 
 pertanto, per $\vec{r}\ne 0$,
 
  \begin{equation}\label{baba}
  \rho(\vec{r})=-\frac{Q}{4\pi l^2}\frac{1}{r}e^{- \frac{r}{l}}.
  \end{equation}
 
 Per trovare la densità di carica nell'origine, applichiamo il teorema di Gau\ss\ in forma integrale ad una sfera centrata
 nell'origine e di raggio $r$.

 Dato che $\vec{E}=-\vec{\nabla}V$ e che, essendo il potenziale in questione a simmetria sferica, vale che
 $\vec{\nabla}V(\vec{r})=\frac{\partial f}{\partial r}\hat{r}$, si ha che

  $$\vec{E}=-\vec{\nabla}V=-Q \frac{-\frac{r}{l}e^{-\frac{r}{l}}\frac{1}{r}-e^{-\frac{r}{l}}}{r^2}\hat{r}=
  Q\frac{e^{-\frac{r}{l}}}{r^2}\left(1+\frac{r}{l}\right)\hat{r}$$
  
 Essendo il campo perpendicolare alla superficie della sfera il flusso del campo attraverso la sfera vale $4\pi r^2 E$,
 e dunque
 
  $$4\pi r^2Q\frac{e^{-\frac{r}{l}}}{r^2}\left(1+\frac{r}{l}\right)=4\pi\, Q_{int}(r)$$
 
 ove $Q_{int}(r)$ rappresenta la carica contenuta nella sfera di raggio $r$ centrata nell'origine. Da ciò si ottiene che
 
 $$Q_{int}(r)=Q\, e^{-\frac{r}{l}}\left(1+\frac{r}{l}\right).$$
 
 Giacché per $r\to 0$ tale espressione mostra che $Q_{int}(r)\to Q$, nell'origine è posta una carica puntiforme $Q$.
 
 Di conseguenza la densità di carica è data da
 
 $$\rho(\vec{r})=Q\cdot\delta(\vec{r})-\frac{Q}{4\pi l^2}\frac{1}{r}e^{- \frac{r}{l}}.$$
 
 Si noti che se nella soluzione del problema si fosse erroneamente applicata la legge di Gau\ss\ in forma differenziale anche
 nell'origine, ci si sarebbe dovuti accorgere dell'errore, imperciocché l'equazione \ref{baba} non può essere soluzione del
 problema, in quanto fisicamente priva di senso: infatti tale distribuzione di carica è ovunque negativa, mentre il potenziale
 è ovunque positivo.

\end{document}