\documentclass[../main.tex]{subfiles} 
\begin{document}

\exercise[10 novembre 2014]{Sfera con potenziale opposto in due emisferi}

\textex

L'emisfero con $z>0$ di una sfera di raggio $a$ centrata nell'origine è a potenziale $V_0$, l'altro è a potenziale $-V_0$.
Si trovino i primi due termini non nulli dello sviluppo in serie di potenze di $r$ del potenziale all'interno e all'esterno
della sfera.

\solution

Poiché il sistema è a simmetria azimutale, si possono calcolare i coefficienti dello sviluppo del potenziale tramite i
polinomî di Legendre.
Bisogna distinguere l'interno e l'esterno della sfera.
All'esterno della sfera i termini $A_l$ dell'equazione \ref{Legendre} sono nulli in quanto il potenziale all'infinito tende a
zero.
Dunque per $r>a$ si ha che

\begin{equation}\label{sd:sviluppo}
V(r,\theta)=\sum_{l=0}^{\infty}B_l\frac{1}{r^{l+1}}P_l(\cos\theta).
\end{equation}

Per trovare i coefficienti $B_l$ imponiamo la condizione al contorno del potenziale sulla sfera.
Notiamo che $V(a,\theta)=V_0\sgn(\cos\theta)$, e pertanto cerchiamo lo sviluppo in polinomî di
Legendre di tale funzione:

\begin{equation}\label{sd:segno}
V_0\sgn(\cos\theta)=\sum_{l=0}^{\infty}B'_lP_l(\cos\theta).
\end{equation}

Poiché $V_0\sgn(\cos\theta)$ è una funzione dispari, per $l$ pari si ha che $B'_l=0$.
Per trovare gli altri coefficienti sfruttiamo il fatto che i polinomî di Legendre sono ortogonali, e che

$$\int_{-1}^1P_l^2(\cos\theta)\de\cos\theta=\frac{2}{2l+1}.$$

Si ha dunque che

$$B'_l=\frac{\int_{-1}^1P_l(\cos\theta)V_0\sgn(\cos\theta)\de \cos\theta}{\int_{-1}^1P_l^2(\cos\theta)\de\cos\theta}=
V_0\frac{2l+1}{2}\left[\int_{-1}^0P_l(x)\sgn(x)\de x+\int_{0}^1P_l(x)\sgn(x)\de x\right]=$$
$$=V_0(2l+1)\int_{0}^1P_l(x)\de x$$

(dove l'ultimo passaggio è giustificato dal fatto che, se $l$ è dispari, anche $P_l$ è dispari).

Usando tale formula ed eguagliando le equazioni \ref{sd:sviluppo} e \ref{sd:segno} si ottiene che per $r>a$ vale che

\begin{equation*}
 V(r,\theta)=V_0\sum_{\substack{l=1\\ l\ dispari}}^{\infty}\left(\frac{a}{r}\right)^{l+1}P_l(\cos\theta)(2l+1)
 \int_{0}^1P_l(x)\de x
\end{equation*}

e dunque con qualche calcolo si può rispondere alla richiesta del problema:

$$V(r,\theta)\approx\frac{3}{2}V_0\frac{a^2}{r^2}P_1(\cos\theta)-\frac{7}{8}V_0\frac{a^4}{r^4}P_1(\cos\theta)=
\frac{3}{2}V_0\frac{a^2}{r^2}\cos\theta-\frac{7}{16}V_0\frac{a^4}{r^4}(5\cos^3\theta-3\cos\theta).$$

Analoghi conti permettono di trovare il potenziale all'interno della sfera (notando che in questo caso sono i coefficienti
$B_l$ ad annullarsi):

\begin{equation*}
 V(r,\theta)=V_0\sum_{\substack{l=1\\ l\ dispari}}^{\infty}\left(\frac{r}{a}\right)^lP_l(\cos\theta)(2l+1)\int_{0}^1P_l(x)
 \de x
\end{equation*}

dando luogo all'approssimazione

$$V(r,\theta)\approx\frac{3}{2}V_0\frac{r}{a}\cos\theta-\frac{7}{16}V_0\frac{r^3}{a^3}(5\cos^3\theta-3\cos\theta).$$

\end{document}