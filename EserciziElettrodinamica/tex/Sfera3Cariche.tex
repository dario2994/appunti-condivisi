\documentclass[../main.tex]{subfiles} 
\begin{document}

\exercise{Sfera con tre cariche} %stc

\textex
Si ha una sfera conduttrice di raggio $b$ centrata nell'origine, messa a terra. Al suo interno si trovano una carica $2q$ nel centro della sfera e due cariche $-q$ sull'asse $z$ a distanza $a < b$ dall'origine. Trovare il potenziale all'interno della sfera per distanze dal centro maggiori di $a$, nel limite $a \rightarrow 0$ con $q\cdot a^2 = Q$ costante.

\solution
%Qui ci va la soluzione.
%Ecco alcuni simboli tipici dell'elettrodinamica in \LaTeX (ovviamente basta che leggiate il codice per capire come si fa):
%\begin{gather*}
%	\div\vec E \\
%	\lapl V \\
%	-\grad V=\vec E \\
%	\frac{\partial \vec E}{\partial n}\\
%	\frac{\de f}{\de t}\\
%	4\pi\int_{\Omega} \rho \de\vec r=\int_{\partial\Omega} \vec E\cdot\hat n\de S\\
%	\rot \vec E=0 \punto
%\end{gather*}

Comincio a trovare il potenziale che avrei all'interno della sfera se questa non ci fosse: il potenziale è semplicemente quello dato dalle tre cariche, cioè
$$ V(\vec r) = -\frac{2q}{r} + \frac{q}{|\vec r - a\hat z } + frac{q}{|\vec r + a\hat z } $$.
Per la formula \cref{rR} bla bla lo scriverò dopo

\end{document}
