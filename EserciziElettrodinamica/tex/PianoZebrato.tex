\documentclass[../main.tex]{subfiles} 
\begin{document}

\exercise[29 ottobre 2014]{Piano zebrato} %pz

\textex

Si determini il potenziale dovuto al piano $xy$ a strisce di larghezza $a/2$ parallelle all'asse $y$ cariche alternatamente con densità superficiale $\sigma_0$  e $0$.\newline inoltre
Si determini inoltre, per $\abs{z} \gg 1$ il termine dominante dipendente da $x$.

\solution
Considero la situazione come sovrapposizione di un piano con densità superficiale uniforme $\sigma_0/2$ e di un piano a strisce con densità $\pm \sigma_0/2$.
Come semplice applicazione del teorema di Gauss (\cref{GaussIntegrale}) il campo elettrico del piano uniformemente carico è

\begin{equation*}
  \vec E (\vec r)= \pi \sigma_0 \hat z 
\end{equation*}
e quindi il potenziale, posto a $0$ sul piano, vale

\begin{equation}
  \label{pz:potenzialeuniforme}
  V(\vec r)= \pi \sigma_0 z
\end{equation}
Per il problema del piano a strisce cerco un potenziale del tipo

\begin{equation*}
  V(x, y, z)=X(x)Y(y)Z(z)
\end{equation*}
quindi noto che, data l'invarianza per traslazione lungo $y$, posso supporre $Y(y)=1$.
Inoltre $X$ deve essere invariante per traslazioni di $a$ (una striscia positiva e una negativa) lungo $x$.
$Z$ invece deve essere invariante per simmetria rispetto al piano $xy$ ($z \to -z$), quindi posso limitarmi a considerare il caso $z > 0$.

Se ricavassi il potenziale integrando su tuttto il piano il potenziale dovuto ad un punto, otterrei che il termine di monopolo \`e nullo, perch\'e la carica complessiva del piano \`e nulla, %TODO: \`e vero?
e il termine di dipolo a grandi distanze \`e proporzionale a $r^{-3}$, quindi integrato su tutto il piano darebbe un termine proporzionale a $r^{-1}$, ossia il potenziale va a $0$ all'infinito.
\end{document}