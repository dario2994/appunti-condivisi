\documentclass[../main.tex]{subfiles} 
\begin{document}

\exercise[29 ottobre 2014]{Piano zebrato} %pz

\textex

Si determini il potenziale dovuto al piano $xy$ a strisce di larghezza $a/2$ parallelle all'asse $y$ cariche alternatamente con densità superficiale $\sigma_0$  e $0$.\newline inoltre
Si determini inoltre, per $\abs{z} \gg 1$ il termine dominante dipendente da $x$.

\solution
Considero la situazione come sovrapposizione di un piano con densità superficiale uniforme $\sigma_0/2$ e di un piano a strisce con densità $\pm \sigma_0/2$.
Come semplice applicazione del teorema di Gauss (\cref{GaussIntegrale}) il campo elettrico del piano uniformemente carico è

\begin{equation*}
  \vec E (\vec r)= \pi \sigma_0 \hat z 
\end{equation*}
e quindi il potenziale, posto a $0$ sul piano, vale

\begin{equation}
  \label{pz:potenzialeuniforme}
  V(\vec r)= \pi \sigma_0 z
\end{equation}
Per il problema del piano a strisce cerco un potenziale del tipo

\begin{equation*}
  V(x, y, z)=X(x)Y(y)Z(z)
\end{equation*}
quindi noto che, data l'invarianza per traslazione lungo $y$, posso supporre $Y(y)=1$.
Inoltre $X$ deve essere invariante per traslazioni di $a$ (una striscia positiva e una negativa) lungo $x$.
$Z$ invece deve essere invariante per simmetria rispetto al piano $xy$ ($z \to -z$), quindi posso limitarmi a considerare il caso $z > 0$.
Dall'\cref{EquazioniSeparazione}, con $\displaystyle\frac{\de^2Y}{\de y^2}=0$, ottengo

\begin{equation}
  \left\{
    \begin{aligned}
      \frac{1}{X(x)}\frac{\de^2X(x)}{\de x^2}=-b^2\\
      \frac{1}{Z(z)}\frac{\de^2Z(z)}{\de z^2}=c^2\\
      c^2=b^2
    \end{aligned}
  \right.
\end{equation}
che ha come soluzione 

\begin{equation}
  \left\{
    \begin{aligned}
      X(x)=e^{bix} \\
      Z(z)=e^c \\
      c^2=b^2 
    \end{aligned}
  \right.
\end{equation}
dalla condizione che $X$ sia periodica di periodo $a$ si ottiene che

\begin{equation}
  b=\frac{2k \pi}{a}
\end{equation}
Se l'origine \`e messa in modo tale che . Allora posso scivere:

\begin{equation}
  \left\{
    \begin{aligned}
      X(x)=e^{\frac{2 k \pi i}{a}x} \\
      Z(z)=e^{\pm \frac{2 k \pi}{a}z}
    \end{aligned}
  \right.
\end{equation}
con $k\in \mathbb Z$. Quindi

\begin{equation}
  V(x,y,z)=\sum_{k=-\infty}^\infty A_k e^{\frac{2 k \pi i}{a} x \pm \frac{2 k \pi}{a} z }
\end{equation}

Per trovare gli $A_k$ calcolo il potenziale in qualche punto particolare, per esempio sull'asse $z$, avendo messo l'oigine in modo tale che l'asse $y$ sia una separazione tra due striscie. Allora 

\end{document}