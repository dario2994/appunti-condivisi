\documentclass[../main.tex]{subfiles} 
\begin{document}

\exercise[29 ottobre 2014]{Piano zebrato} %pz

\textex

Si determini il potenziale dovuto al piano $xy$ a strisce di larghezza $a/2$ parallelle all'asse $y$ cariche alternatamente con densità superficiale $\sigma_0$  e $0$.\newline inoltre
Si determini inoltre, per $\abs{z} \gg 1$ il termine dominante dipendente da $x$.

\solution
Considero la situazione come sovrapposizione di un piano con densità superficiale uniforme $\sigma_0/2$ e di un piano a strisce con densità $\pm \sigma_0/2$.
Come semplice applicazione del teorema di Gauss (\cref{GaussIntegrale}) il campo elettrico del piano uniformemente carico è

\begin{equation*}
  \vec E (\vec r)= \pi \sigma_0 \hat z 
\end{equation*}
e quindi il potenziale, posto a $0$ sul piano, vale

\begin{equation}
  V(\vec r)= \pi \sigma_0 z
\end{equation}


\end{document}