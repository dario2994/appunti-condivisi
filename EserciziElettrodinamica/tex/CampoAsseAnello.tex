\documentclass[../main.tex]{subfiles} 
\begin{document}

\exercise{Campo elettrico sull'asse di un anello} %caa

\textex
Un anello centrato nell'origine e giacente sul piano $xy$ ha raggio $a$ e densità lineare di carica uniforme $\lambda$. Determinare il campo elettrico sull'asse $z$.

Determinare anche il campo elettrico sull'asse $z$ di un disco pieno (e densità superficiale uniforme $\sigma$).

\solution

Una qualsiasi simmetria rotazionale di asse $z$ ci permette di dire che $\vec{E}$ calcolato nei punti dell'asse $z$ deve essere un vettore lungo $z$.
Dobbiamo quindi determinare solamente $E_z(z)$.

Per fare ciò calcoliamo il potenziale $V(z)$ solo lungo l'asse, infatti ciò è sufficiente per risolvere il problema grazie alla definizione di potenziale elettrico:
\[
	E_z(z)=-\frac{dV(z)}{dz}\punto
\]

Dato che le cariche sono localizzate, siamo autorizzati a scrivere
\begin{equation}\label{caa:potenziale}
	V(\vec{r})=\int d\vec{r'}^3 \frac{\rho(\vec{r'})}{|\vec{r}-\vec{r'}|}
\end{equation}
che, per $\vec{r}=z\hat{z}$, cioè sull'asse $z$, dato che $|\vec{r}-\vec{r'}|=\sqrt{z^2+a^2}$, diventa
\[
	V(z)=\lambda 2\pi a \frac{1}{\sqrt{z^2+a^2}}\virgola
\]
da cui, derivando lungo $z$,
\[
	E_z(z)=\frac{2\pi a\lambda z}{(z^2+a^2)^{\frac{3}{2}}}\punto
\]

Passando al problema del disco pieno, dato che le cariche sono ancora localizzate, possiamo ancora usare l'\cref{caa:potenziale}. Per calcolare l'integrale sul disco, sviluppiamo in coordinate polari:
\[
	V(z)=\int_0^a \int_0^{2\pi} \frac{\sigma r}{\sqrt{z^2+r^2}}\de \theta \de r = \int_0^a \frac{2\pi\sigma r}{\sqrt{z^2+r^2}}\de r = 2\pi\sigma (\sqrt{z^2+a^2}-z)\punto
\]
da cui, ancora una volta, abbiamo il campo elettrico:
\[
	E_z(z)=2\pi\sigma \left(1-\frac{z}{\sqrt{z^2+a^2}}\right)\punto
\]



\end{document}