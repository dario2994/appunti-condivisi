\documentclass[../main.tex]{subfiles} 
\begin{document}
\section{Formulario}
\setcounter{equation}{0}
\renewcommand{\theequation}{F.\arabic{equation}}

Se qualcuno vuole scrivere un po' di formulario questo è il posto giusto.

\subsection{Legge di Coulomb, campo elettrico e potenziale}\label{Preliminari}
La carica elettrica in unit\`a cgs \`e definita dalla legge di Coulomb, che esprime la forza tra due cariche:
\begin{equation}
	\label{Coulomb}
	F_{ab}=\frac{q_aq_b}{\abs{\vec r_a -\vec r_b}^2}
\end{equation}
Il campo elettrico \`e definito a partire da una ``carica di prova''
\begin{equation}
	\label{CampoElettrico}
	\vec E_{a}(\vec r)=\frac{\vec F_{ab}}{q_b}=\frac{q_a}{\abs{\vec r}^2}\hat r
\end{equation}
Il potenziale elettrico \`e un campo scalare $V$ tale che
\begin{equation}
	\label{Potenziale}
	\vec E(\vec r) = - \vec \nabla \cdot V(\vec r) = - \div (V(\vec r))
\end{equation}

\subsection{Legge di Gauss}\label{Gauss}
La legge di Gauss pu\`o essere espressa in due forme.\newline
Forma integrale: lega la quantit\`a di carica contenuta all'interno di una superficie chiusa S e il flusso uscente da S
\begin{equation}
	\label{GaussIntegrale}
	\Phi(S)= \oint_S \vec E d\vec S=-4\pi Q_{interna}
\end{equation}
Forma differenziale: lega campo elettrico e densit\`a di carica in un punto 
\begin{equation}
	\label{GaussDifferenziale}
	\div \vec E=-4\pi\rho
\end{equation}
Come diretta conseguenza si ha anche l'equazione di Poisson:
\begin{equation}
	\label{Poisson}
	\nabla^2V= -4\pi \rho
\end{equation}

\subsection{Energia elettrostatica}\label{EnergiaElettrostatica}
Distribuzione discreta di cariche puntiformi:
\begin{equation}
	\label{EEDiscreta}
	U= \frac{1}{2} \sum_{i\neq j} \frac{q_iq_j}{\abs{\vec r_i -\vec r_j}}
\end{equation}


\end{document}
