\documentclass[../main.tex]{subfiles} 
\begin{document}
\section{Formulario}
\setcounter{equation}{0}
\renewcommand{\theequation}{F.\arabic{equation}}

Se qualcuno vuole scrivere un po' di formulario questo è il posto giusto.
\subsection{Legge di Gauss}\label{Gauss}
La legge di Gauss pu\`o essere espressa in due forme.\newline
Forma integrale: lega la quantit\`a di carica contenuta all'interno di una superficie chiusa S e il flusso uscente da S
\begin{equation}
	\label{GaussIntegrale}
	\Phi(S)= \oint_S \vec E d\vec S=-4\pi Q_{interna}
\end{equation}
Forma differenziale: lega campo elettrico e densit\`a di carica in un punto 
\begin{equation}
	\label{GaussDifferenziale}
	\div \vec E=-4\pi\rho
\end{equation}

Come diretta conseguenza si ha anche l'equazione di Poisson:
\begin{equation}
	\label{Poisson}
	\nabla^2V= -4\pi \rho
\end{equation}



\end{document}
