\documentclass[../main.tex]{subfiles} 
\begin{document}
\section{Formulario}
\setcounter{equation}{0}
\renewcommand{\theequation}{F.\arabic{equation}}

Se qualcuno vuole scrivere un po' di formulario questo è il posto giusto.

\subsection{Legge di Coulomb, campo elettrico e potenziale}\label{Preliminari}
La carica elettrica in unit\`a cgs \`e definita dalla legge di Coulomb, che esprime la forza tra due cariche:
\begin{equation}
  \label{Coulomb}
  F_{ab}=\frac{q_aq_b}{\abs{\vec r_a -\vec r_b}^2}
\end{equation}
Il campo elettrico \`e definito a partire da una ``carica di prova''; si verifica sperimentalmente che il campo non dipende dalla carica utilizzata, purch\'e abbastanza piccola:
\begin{equation}
  \label{CampoElettrico}
  \vec E(\vec r)=\lim_{q \to 0} \frac{\vec F_q(\vec r)}{q}
\end{equation}
Il potenziale elettrico \`e un campo scalare $V$ tale che
\begin{equation}
  \label{Potenziale}
  \vec E(\vec r) = - \vec \nabla V(\vec r)
\end{equation}

\subsection{Matematica utile}\label{Matematica}
\begin{theorem}[Teorema della media]
  \label{ThMedia}
  Se $f:S^2 \to \mathbb{R}$ \`e una funzione armonica la media di $f$ sulla sfera \`e uguale al valore al centro, cio\`e
  \begin{equation}
    \int_{S^2} f(x) \de x = f(0,0)
  \end{equation}
\end{theorem}

\begin{theorem}[Teorema della divergenza]
  \label{ThDivergenza}
  \begin{equation}
    \int_V \de^3 r \div(\vec A) = \oint_S \vec A \cdot \hat n \de a
  \end{equation}
\end{theorem}

\begin{definition}[Delta di Dirac]
La fuzione delta di Dirac \`e in realt\`a una pseudofunzione, tale che
  \begin{subequations}
    \label{DeltaDirac}
    \begin{align}
      \delta(x-a)=
      \begin{cases}
	\infty,	& \text{se $x=a$,} \\
	0, 	& \text{se $x \neq a$.}
      \end{cases}\\
      \int \delta (x-a) \de x =1
    \end{align}
  \end{subequations}
\end{definition}

\subsection{Legge di Gauss}\label{Gauss}
La legge di Gauss pu\`o essere espressa in due forme.\newline
Forma integrale: lega la quantit\`a di carica contenuta all'interno di una superficie chiusa S e il flusso uscente da S
\begin{equation}
  \label{GaussIntegrale}
  \Phi(S)= \oint_S \vec E \hat n \de \vec a=-4\pi Q_{interna}
\end{equation}
Forma differenziale: lega campo elettrico e densit\`a di carica in un punto 
\begin{equation}
  \label{GaussDifferenziale}
  \div \vec E=-4\pi\rho
\end{equation}
Come diretta conseguenza della definizione di potenziale (\cref{Potenziale}) e dell'\cref{GaussIntegrale} si ha anche l'equazione di Poisson:
\begin{equation}
  \label{Poisson}
  \lapl V= -4\pi \rho
\end{equation}

\subsection{Teorema di Green}\label{Green}
\begin{theorem}[Prima identit\`a di Green]
  \label{Green1}
  \begin{equation}
  \int_V (f \lapl g + \grad f \cdot \grad g) \de^3 r = \oint_S f \frac{\partial g}{\partial n} \de a
  \end{equation}
\end{theorem}

\begin{theorem}[Seconda identit\`a di Green o Teorema di Green]
  \label{Green2}
  \begin{equation}
  \int_V \de^3 r (f \lapl g - g \lapl f) = \oint_S \left( f \frac{\partial g}{\partial n} - g \frac{\partial f}{\partial n} \right) \de a
  \end{equation}
\end{theorem}

\subsection{Energia elettrostatica}\label{EnergiaElettrostatica}
Distribuzione discreta di cariche puntiformi:
\begin{equation}
  \label{EEDiscreta}
  U= \frac{1}{2} \sum_{i\neq j} \frac{q_iq_j}{\abs{\vec r_i -\vec r_j}}
\end{equation}
Distribuzione continua di carica:
\begin{equation}
  \label{EEContinua}
  U= \frac{1}{8\pi} \int \de^3 r \abs{\vec E}^2
\end{equation}
Le due formule \cref{EEDiscreta} e \cref{EEContinua} \emph{non} sono equivalenti; in particolare indicando una carica puntiforme $q$ in posizione $\vec {r'}$ come una distribuzione di carica $\rho(\vec r)=\delta(\vec r - \vec {r'})$ ottengo in genere risultati diversi. Questo accade perch\'e l'autoenergia di una carica puntiforme \`e infinita, come mostra l'\cref{EEContinua} applicata ad una sola carica.

\subsection{Conduttori}\label{Conduttori}
\begin{definition}[Conduttore]
  Un conduttore in elettrostatica \`e ``un mezzo in cui all'equilibrio il campo elettrco \`e nullo'' 
\end{definition}
La discontinuit\`a del campo elettrico a cavallo di un piano carico con densit\`a superficiale $\sigma$ vale
\begin{equation}
  \label{DiscontinuitaSigma}
  \Delta E = 4 \pi \sigma
\end{equation}


\end{document}
