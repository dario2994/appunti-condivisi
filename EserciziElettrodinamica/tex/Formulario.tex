\documentclass[../main.tex]{subfiles} 
\begin{document}
\section{Formulario}
\setcounter{equation}{0}
\renewcommand{\theequation}{F.\arabic{equation}}

Se qualcuno vuole scrivere un po' di formulario questo è il posto giusto.

\subsection{Legge di Coulomb, campo elettrico e potenziale}\label{Preliminari}
La carica elettrica in unità cgs è definita dalla legge di Coulomb, che esprime la forza tra due cariche:
\begin{equation}
  \label{Coulomb}
  F_{ab}=\frac{q_aq_b}{\abs{\vec r_a -\vec r_b}^2}
\end{equation}
Il campo elettrico è definito a partire da una ``carica di prova''; si verifica sperimentalmente che il campo non dipende dalla carica utilizzata, purché abbastanza piccola:
\begin{equation}
  \label{CampoElettrico}
  \vec E(\vec r)=\lim_{q \to 0} \frac{\vec F_q(\vec r)}{q}
\end{equation}
Il potenziale elettrico è un campo scalare $V$ tale che
\begin{equation}
  \label{Potenziale}
  \vec E(\vec r) = - \vec \nabla V(\vec r)
\end{equation}
Le due leggi fondamentali dell'elettrostatica sono, in forma differenziale:
\begin{subequations}
  \label{DifferenzialiElettrostatica}
  \begin{align}
    \vec \nabla \cdot \vec E = 4 \pi \rho\\
    \rot \vec E = 0
  \end{align}
\end{subequations}


\subsection{Matematica utile}\label{Matematica}
\begin{theorem}[Teorema della media]
  \label{ThMedia}
  Se $f:S^2 \to \mathbb{R}$ \`e una funzione armonica la media di $f$ sulla sfera è uguale al valore al centro, cioè
  \begin{equation}
    \int_{S^2} f(x) \de x = f(0,0)
  \end{equation}
\end{theorem}

\begin{theorem}[Teorema della divergenza]
  \label{ThDivergenza}
  \begin{equation}
    \int_V \de^3 r \div(\vec A) = \oint_S \vec A \cdot \hat n \de a
  \end{equation}
\end{theorem}

\begin{definition}[Delta di Dirac]
La fuzione delta di Dirac è in realtà una pseudofunzione, tale che
  \begin{subequations}
    \label{DeltaDirac}
    \begin{align}
      \delta(x-a)=
      \begin{cases}
	\infty,	& \text{se $x=a$,} \\
	0, 	& \text{se $x \neq a$.}
      \end{cases}\\
      \int \delta (x-a) \de x =1
    \end{align}
  \end{subequations}
\end{definition}

\subsection{Legge di Gauss}\label{Gauss}
La legge di Gauss può essere espressa in due forme.\newline
Forma integrale: lega la quantità di carica contenuta all'interno di una superficie chiusa S e il flusso uscente da S
\begin{equation}
  \label{GaussIntegrale}
  \Phi(S)= \oint_S \vec E \hat n \de \vec a=-4\pi Q_{interna}
\end{equation}
Forma differenziale: lega campo elettrico e densità di carica in un punto 
\begin{equation}
  \label{GaussDifferenziale}
  \div \vec E=-4\pi\rho
\end{equation}
Come diretta conseguenza della definizione di potenziale (\cref{Potenziale}) e dell'\cref{GaussIntegrale} si ha anche l'equazione di Poisson:
\begin{equation}
  \label{Poisson}
  \lapl V= -4\pi \rho
\end{equation}
E se $\rho=0$ si ottiene l'equazione di Laplace:
\begin{equation}
  \label{Laplace}
  \lapl V= 0
\end{equation}


\subsection{Teorema di Green}\label{Green}
\begin{theorem}[Prima identità di Green]
  \label{Green1}
  \begin{equation}
  \int_V (f \lapl g + \grad f \cdot \grad g) \de^3 r = \oint_S f \frac{\partial g}{\partial n} \de a
  \end{equation}
\end{theorem}

\begin{theorem}[Seconda identità di Green o Teorema di Green]
  \label{Green2}
  \begin{equation}
  \int_V \de^3 r (f \lapl g - g \lapl f) = \oint_S \left( f \frac{\partial g}{\partial n} - g \frac{\partial f}{\partial n} \right) \de a
  \end{equation}
\end{theorem}

\subsection{Energia elettrostatica}\label{EnergiaElettrostatica}
Distribuzione discreta di cariche puntiformi:
\begin{equation}
  \label{EEDiscreta}
  U= \frac{1}{2} \sum_{i\neq j} \frac{q_iq_j}{\abs{\vec r_i -\vec r_j}}
\end{equation}
Distribuzione continua di carica:
\begin{equation}
  \label{EEContinua}
  U= \frac{1}{8\pi} \int \de^3 r \abs{\vec E}^2
\end{equation}
Le due formule \cref{EEDiscreta} e \cref{EEContinua} \emph{non} sono equivalenti; in particolare indicando una carica puntiforme $q$ in posizione $\vec {r'}$ come una distribuzione di carica $\rho(\vec r)=\delta(\vec r - \vec {r'})$ ottengo in genere risultati diversi. Questo accade perch\'e l'autoenergia di una carica puntiforme \`e infinita, come mostra l'\cref{EEContinua} applicata ad una sola carica.

\subsection{Conduttori}\label{Conduttori}
\begin{definition}[Conduttore]
  Un conduttore in elettrostatica è ``un mezzo in cui all'equilibrio il campo elettrio è nullo'' 
\end{definition}
La discontinuità del campo elettrico a cavallo di un piano carico con densità superficiale $\sigma$ vale
\begin{equation}
  \label{DiscontinuitaSigma}
  \Delta E = 4 \pi \sigma
\end{equation}

\subsection{Separazione delle variabili in coordinate cartesiane}\label{Separazione}
Se si cerca una soluzione del potenziale $V(x, y, z)=X(x)Y(y)Z(z)$
Dall'equazione di Laplace (\cref{Laplace}) in coordinate cartesiane, dividendo per V si ottiene:
\begin{equation}
  \frac{1}{X(x)}\frac{\de^2X(x)}{\de x^2}+\frac{1}{Y(y)}\frac{\de^2Y(y)}{\de y^2}+\frac{1}{Z(z)}\frac{\de^2Z(z)}{\de z^2}=0
\end{equation}
da cui il sistema (in cui eventualmente $a, b, c \in \mathbb{C}$)
\begin{equation}
  \left\{
    \begin{aligned}
      \frac{1}{X(x)}\frac{\de^2X(x)}{\de x^2}=a^2\\
      \frac{1}{Y(y)}\frac{\de^2Y(y)}{\de y^2}=b^2\\
      \frac{1}{Z(z)}\frac{\de^2Z(z)}{\de z^2}=c^2\\
      a^2+b^2+c^2=0
    \end{aligned}
  \right.
\end{equation}
e quindi
\begin{equation}
  V(x, y, z)=e^{\pm ax}e^{\pm by}e^{\pm cz}
\end{equation}
e combinazioni lineari dell'espressione sopra al variare di $a, b, c$



\subsection{Espansioni in serie}
\subsubsection{Armoniche sferiche}
Una funzione arbitraria $g(r, \theta, \phi)$ si può sviluppare in armoniche sferiche come
\begin{equation}
  \label{Armoniche}
  g(r, \theta, \phi)= \sum_{l=0}^\infty \sum_{m=-l}^l (A_{lm}r^l+B_{lm}r^{-l-1}) Y_{lm}(\theta,\phi)
\end{equation}
\subsubsection{Siluppo in multipoli}
\begin{equation}
  \label{Multipoli}
  V(r, \theta, \phi)=\sum_{l=0}^\infty \sum_{m=-l}^l \frac{4 \pi}{2l+1}q_{lm} \frac{Y_{lm}(\theta, \phi)}{r^{l+1}}
\end{equation}




\end{document}
