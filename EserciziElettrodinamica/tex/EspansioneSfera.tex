\documentclass[../main.tex]{subfiles} 
\begin{document}

\exercise[18 Novembre 2014]{Espansione di una sfera carica} %esc

\textex
All'interno di una sfera di raggio $a$ sono uniformemente distribuite $N \gg 1$ cariche tutte uguali, ognuna di massa $m$ e carica totale $Q$.
Istantaneamente, al tempo $t=0$, le cariche diventano libere di muoversi.

Studiare l'evoluzione del sistema.

\solution
Trascuriamo la granularità della carica, poiché il numero enorme di particelle ci consente di farlo.

Inizialmente il sistema gode di simmetria sferica e di conseguenza tale simmetria rimarrà anche in seguito.

Consideriamo due particelle che inizialmente distano dal centro $r_1<r_2$. Assumiamo, più o meno in maniera immotivata, se non a posteriori, che le particelle non invertano mai la propria posizione, cioè che la prima rimanga sempre la più vicina all'origine.

A questo punto consideriamo una particella che inizialmente dista $r_0$ dal centro, e sia $r(t)$ la sua distanza dal centro al tempo $t$.
Per la simmetria sferica e per quanto detto in precedenza sull'ordine che non si inverte, al tempo $t$ applicando Gauss sulla sfera centrata nell'origine di raggio $r(t)$ otteniamo che il campo elettrico presente nel punto in cui si trova la particella considerata è radiale e con modulo $Q\frac{r_0^3}{a^3}\frac 1{r^2(t)}$; perciò ricordando il secondo principio della dinamica otteniamo:
\begin{equation*}
	\frac QN Q\frac{r_0^3}{a^3}\frac 1{r^2} =m \ddot{r} \iff 
	\frac {Q^2}{mNa^3}\left(\frac{r}{r_0}\right)^{-2} =\frac{\ddot r}{r_0}
\end{equation*}
e quindi chiamando $y=\frac r{r_0}$ e $k^2=\frac {2Q^2}{mNa^3}$, otteniamo la differenziale
\begin{equation}\label{esc:differenziale}
	\frac {k^2}{2y^2}=\ddot y
\end{equation}
con condizioni iniziali $y(0)=1$ e $\dot y(0)=0$.

È importante notare che risolvere l'\cref{esc:differenziale} permette di trovare la posizione di una qualunque carica ad ogni istante, infatti è sufficiente moltiplicare $y(t)$ per la posizione iniziale per ottenere la posizione al tempo $t$.
Quindi, scelte due cariche all'inizio, il rapporto tra le loro distanze dal centro non cambia e cioè il sistema si evolve semplicemente tramite un'omotetia e nient'altro. In particolare allora al tempo $t$ il sistema sarà ancora formato da $N$ particelle uniformemente distrubuite all'interno di una sfera, ma tale sfera avrà raggio $ay(t)$.

Per risolvere la differenziale, troviamone innanzitutto un integrale primo. 
Notando l'analogia con la legge di Coulomb, è facile trovare l'integrale primo $\dot y^2+\frac{k^2}y$. Per verificare che sia un integrale primo è sufficiente il calcolo esplicito
\begin{equation*}
 	\frac{\de}{\de t}\left(\dot y^2+\frac{k^2}y\right)=2\dot y\left(\ddot y-\frac{k^2}{2y^2}\right)=0
\end{equation*}
e da questo ricaviamo allora
\begin{equation}\label{esc:Integrale}
	\dot y^2+\frac{k^2}y=k^2 \iff \dot y=k\sqrt{1-\frac 1y}
\end{equation}

Studiando qualitativamente l'\cref{esc:differenziale} è facile accorgersi che $\ddot y$ è sempre positiva e di conseguenza $\dot y$ è crescente e quindi $y$ tende ad infinito. 
Allora l'\cref{esc:Integrale} ci assicura che, per tempi molto grandi, vale $\dot y(t)\approx k$ e perciò la velocità con cui si allarga il raggio della sfera asintoticamente è $ak$.

Infine l'\cref{esc:Integrale} può essere ``risolta'' attraverso i seguenti passaggi (tipici in dinamica classica):
\begin{equation*}
	\dot y=k\sqrt{1-\frac 1y} \iff \frac{\dot y}{\sqrt{1-\frac 1y}}=k \iff
	\int_0^T \frac{\dot y}{\sqrt{1-\frac 1y}} \de t=kT \iff \int_0^{y(T)} \left(1-\frac 1y\right)^{-\frac12} \de y=kT
\end{equation*}
ma poiché vale
\begin{equation*}
	\int \left(1-\frac 1y\right)^{-\frac12}=\sqrt{y(y-1)}+\log\left(\sqrt{y}+\sqrt{y-1}\right)
\end{equation*}
riusciamo a concludere, ricordando $y(0)=1$, che
\begin{equation*}
	\sqrt{y(y-1)}+\log\left(\sqrt{y}+\sqrt{y-1}\right)=kt
\end{equation*}


\end{document}
