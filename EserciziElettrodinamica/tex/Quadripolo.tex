\documentclass[../main.tex]{subfiles} 
\begin{document}

\exercise{Quattro cariche alternate ai vertici di un quadrato} %qca

\textex
Nel piano $xy$ sono date 4 cariche puntiformi poste nei punti $\vec{r_x}=(a,0,0)$, $-\vec{r_x}$, $\vec{r_y}=(0,a,0)$, $-\vec{r_y}$; le prime 2 hanno carica $+q$, le ultime 2 hanno carica $-q$. Determinare il potenziale in tutto lo spazio al secondo ordine nello sviluppo in armoniche sferiche.

\solution

Consideriamo lo sviluppo in multipoli dato dalla localizzazione delle cariche
\begin{equation}\label{qca:multipoli}
	V(r,\theta,\phi)=\sum_{l=0}^{\infty} \sum_{m=-l}^l B_{l,m}\frac{1}{r^{l+1}}Y_{l,m}(\theta,\phi)
\end{equation}
dove si ha che
\begin{equation}\label{qca:coefficienti}
	B_{l,m}=\frac{4\pi}{2l+1}\int \de'r^3 \rho(\vec{r'})(r')^l Y_{l,m}^*(\theta',\phi')\punto
\end{equation}

Notiamo ora che il sistema ruotato di $\pi/2$ attorno all'asse $z$ è equivalente al sistema con $q\rightarrow -q$. Se imponiamo questa invarianza otteniamo, dall'\cref{qca:coefficienti} che 
\[
	B_{l,m}=\frac{4\pi}{2l+1}\int \de'r^3 -\rho(\vec{r'})(r')^l Y_{l,m}^*(\theta',\phi'+\frac{\pi}{2})\virgola
\]
in particolare, se per una certa coppia $l,m$ non si ha che $Y_{l,m}(\theta,\phi+\frac{\pi}{2})=-Y_{l,m}(\theta,\phi)$, allora $B_{l,m}=0$.

Sappiamo anche che tutte le armoniche sferiche si scrivono come
\[
	Y_{l,m}=e^{im\phi}P_{l,m}(cos\theta)
\]
per opportuni polinomi $P_{l,m}$. In particolare, la dipendenza da $\phi$ è esponenziale.

Se vogliamo, dunque, che cambi segno con l'aumentare di $\pi/2$, vogliamo che
\[
	e^{im(\phi+\frac{pi}{2}}=-e^{im\phi}\virgola
\]
cioè $m\equiv 2$ modulo 4. Che ci dice che l'unico termine non irrilevante al secondo ordine è $B_{2,\pm 2}$.

Calcoliamo, dunque, questi 2 termini. Sappiamo che
\[
	\rho(\vec{r})=q(\delta(\vec{r}-\vec{r_x})+\delta(\vec{r}+\vec{r_x})-\delta(\vec{r}-\vec{r_y})-\delta(\vec{r}+\vec{r_y}))\virgola
\]
dunque, utilizzando ancora l'\cref{qca:coefficienti} abbiamo che
\begin{align*}
	B_{2,2}&=\frac{4\pi}{5}\int \de'r^3 \rho(\vec{r'})(r')^2 Y_{2,2}^*(\theta',\phi')\\
		&=\frac{4\pi}{5} q\left( (r')^2 {Y_{2,2}^*(\theta',\phi')_\arrowvert}_{r_x}+(r')^2 {Y_{2,2}^*(\theta',\phi')_\arrowvert}_{-r_x}-(r')^2 {Y_{2,2}^*(\theta',\phi')_\arrowvert}_{r_y}-(r')^2 {Y_{2,2}^*(\theta',\phi')_\arrowvert}_{-r_y}\right)\\
		&=\frac{4\pi}{5} qa^2\left( Y_{2,2}^*\left(\frac{\pi}{2},0\right)+Y_{2,2}^*\left(\frac{\pi}{2},\pi\right)-Y_{2,2}^*\left(\frac{\pi}{2},\frac{\pi}{2}\right)-Y_{2,2}^*\left(\frac{\pi}{2},\frac{3\pi}{2}\right)\right)\\
		&=\frac{4\pi}{5} qa^2 4 Y_{2,2}^*\left(\frac{\pi}{2},0\right)\\
		&=\frac{4\pi}{5} qa^2 4 \frac{1}{4} \sqrt{\frac{15}{2\pi}} \cdot 1 = \sqrt{\frac{24\pi}{5}} qa^2\punto
\end{align*}
Sappiamo, infine, che $B_{2,-2}=(-1)^2 B_{2,2}^*=B_{2,2}$.

Quindi il potenziale si approssima con
\[
	V(r,\theta,\phi)\approx \sqrt{\frac{24\pi}{5}} \frac{qa^2}{r^{3}}\left( Y_{2,2}(\theta,\phi)+Y_{2,-2}(\theta,\phi)\right)\punto
\]


\end{document}
