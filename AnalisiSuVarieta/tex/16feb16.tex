\chapter{16 febbraio 2016}

\begin{remark}
	I campi tensoriali possono essere definiti anche nel modo seguente.
	$\chi^*(M)$ può essere definito come una mappa $\Lin_{C^\infty(M)} (\chi(M), C^\infty(M))$ \footnote{Si intende la linerità $C^\infty(M)$, cioè dato $\alpha\in\chi^*(M)$ vale che $\alpha(p)(f(p)X(p)) = f(p) \alpha(p) X(p)$ per ogni $X\in\chi(M)$ e $f\in C^\infty(M)$}.
	
	Allo stesso modo $\Tau_s^r(M)$ può essere definito come
	\begin{equation*}
	\Lin_{C^\infty(M)}^{r+s}(\underbrace{\chi^*(M),\ldots,\chi^*(M)}_{\text{$r$ copie}},\underbrace{\chi(M),\ldots,\chi(M)}_{\text{$s$ copie}},C^\infty(M))\punto
	\end{equation*}
\end{remark}

\begin{definition} \index{algebra dei tensori}
	Sia $\Tau(M)$ la somma diretta dei $\Tau_s^r(M)$ per $r\ge 0$ ed $s\ge 0$. Risulta che $\Tau(M)$ è uno spazio vettoriale con prodotto $\otimes$ ed è detto \emph{algebra dei tensori} su $M$.
\end{definition}

Se $\varphi:M\to N$ è un diffeomorfismo, allora $\varphi_*: \Tau(M) \to \Tau(N)$ è un isomorfismo di algebre.

\section{Derivata di Lie sui tensori}

Ci sono due possibili approcci:
\begin{enumerate}
	\item approccio dinamico, che usa i flussi ed è quello che abbiamo usato per definire la derivata di Lie sui campi vettoriali;
	\item approccio algebrico, che è invece quello che seguiremo noi e che utilizza la proprietà di Leibnitz delle derivate.
\end{enumerate}

\begin{example}
	Vediamo come definire per esempio la derivata di Lie per 1-forme.
	Se $Y\in\chi(M)$ e $\alpha\in\chi^*(M)$, richiediamo che $\Lie_X(\alpha \cdot Y) = (\Lie_X \alpha)\cdot Y + \alpha \cdot (\Lie_X Y)$, dove sappiamo cosa vuol dire la derivata di Lie su funzioni ($\alpha\cdot Y$) e su campi ($Y$), perciò la ricaviamo anche per le forme: $(\Lie_X\alpha)Y = \Lie_X(\alpha\cdot Y) - \alpha(\Lie_XY)$.
\end{example}

\begin{definition} \index{operatore!differenziale}
	Un \emph{operatore differenziale} su $\Tau(M)$ è una famiglia di mappe $\mathcal D_s^r$ da $\Tau_s^r(M)$ in sé tale che
	\begin{description}
	 \item [$\mathcal D_1$)] $\mathcal D$ è una derivazione tensoriale, cioè commuta con le contrazioni.
	 Vogliamo che $\mathcal D$ sia $\R$-lineare e che, dati $t\in\Tau_s^r(M)$, $\alpha_1,\ldots,\alpha_r\in\chi^*(M)$, $X_1,\ldots,X_s\in\chi(M)$, valga
	 \begin{multline*}
	 	\mathcal D (t(\alpha_1,\ldots,\alpha_r,X_1,\ldots,X_s)) = (\mathcal D t) (\alpha_1,\ldots,\alpha_r,X_1,\ldots,X_s) +\\
	 	+\sum_{j=1}^r t(\alpha_1,\ldots,\mathcal D \alpha_j,\ldots, \alpha_r, X_1,\ldots, X_s) +
	 	\sum_{k=1}^s t(\alpha_1,\ldots, \alpha_r, X_1,\ldots, \mathcal D X_k,\ldots, X_s)
	 \end{multline*}

	\item [$\mathcal D_2$)] (Supponiamo che $\mathcal D$ sia definito su $\Tau(U)$ in $\Tau(U)$ per ogni $U\subseteq M$ aperto) $\mathcal D$ è locale (naturale rispetto alle restrizioni).
	Se $U\subseteq V \subseteq M$ sono aperti e $t\in\Tau_s^r(M)$, allora $(\mathcal D t)\restrict U = \mathcal D (t\restrict U)$, ovvero che il seguente diagramma commuti:
	$
	\begin{CD} 
	 \Tau_s^r(V) @>i_U>> \Tau_s^r(U) \\
	 @V\mathcal{D}VV  @V\mathcal{D}VV \\
	 \Tau_s^r(V) @>i_U>> \Tau_s^r(U) 
	\end{CD} $

	\end{description}
\end{definition}

\begin{theorem} \label{teo:EsistenzaOperatoriDifferenziali}
	Supponiamo che per ogni $U\subseteq M$ aperto esistano mappe $\mathcal E_u : C^\infty(U) \to C^\infty(U)$ e $\mathcal F_u : \chi(U) \to \chi(U)$ che siano derivazioni (naturali rispetto alle restrizioni), cioè
	\begin{enumerate}
		\item $\mathcal E_U(f\otimes g) = (\mathcal E_U f) \otimes g + f \otimes (\mathcal E_U g)$; \label{eod:LeibnitzPerE}
		\item se $f\in C^\infty(M)$, allora $\mathcal E_U(f\restrict U) = (\mathcal E_M f)\restrict U$; \label{eod:LocalitaPerE}
		\item $\mathcal F_U(f\otimes X) = (\mathcal E_U f) \otimes X + f\otimes \mathcal F_U X$ per $f\in C^\infty(U)$ e $X\in\chi(U)$ \label{eod:LeibnitzPerF}
		\item per $X$ campo vettoriale su $M$, vale $\mathcal F_U(X\restrict U) = (\mathcal F_M X)\restrict U$ \label{eod:LocalitaPerF}
	\end{enumerate}
	Allora esiste unico un operatore differenziale $\mathcal D$ che coincide con $\mathcal E_U$ su $C^\infty(U)$ e con $\mathcal F_U$ su $\chi(U)$ per ogni $U$ aperto di $M$.

\end{theorem}

\begin{proof}
	Verifichiamo l'esistenza su $\chi^*(M)$. In $U$, usiamo la proprietà $(\mathcal D\alpha)\cdot X = \mathcal D(\alpha\cdot X) - \alpha\cdot(\mathcal D X) = \mathcal E_U(\alpha\cdot X) - \alpha \mathcal F_UX$ per ogni $X\in\chi(U)$ e per ogni $\alpha \in \chi^*(U)$.
	
	Vogliamo verificare che $\mathcal D\alpha$ sia $C^\infty(M)$-lineare su $\chi(U)$.
	Vale che
	\begin{align*}
		(\mathcal D\alpha) \cdot (fX) &= \mathcal D (\alpha(fX)) -\alpha(\mathcal D(fX)) = \mathcal E_U(\alpha(fX)) - \alpha\mathcal F_U(fX)=\\
		&= \mathcal E_U(f(\alpha X)) - \alpha [(\mathcal E_Uf)\otimes X + f \otimes \mathcal F_UX] = \\
		&= (\mathcal E_U f)\otimes (\alpha X) + f \otimes \mathcal E_U (\alpha X) - \alpha (\mathcal E_U f) \otimes X - \alpha f \otimes F_U X = \\
		&= f((\mathcal D\alpha) X)
	\end{align*}
	[dove abbiamo utilizzato le proprietà \ref{eod:LeibnitzPerE} e \ref{eod:LeibnitzPerF}]
	
	Perciò $\mathcal D\alpha \in \chi^*(U)$ per ogni $\alpha\in\chi^*(U)$.

Estendiamo ora le definizione ai tensori, richiedendo la proprietà $\mathcal D_1$ (vedere sopra).

Vogliamo quindi che
\begin{align*}
	(\mathcal D_U t) (\alpha_1,\ldots,\alpha_r,X_1,\ldots,X_s) =& \mathcal E_U(t(\alpha_1,\ldots,\alpha_r,X_1,\ldots,X_s)) -\\
	&-\sum_{j=1}^rt(\alpha_1,\ldots, \mathcal D\alpha_j,\ldots,\alpha_r,X_1,\ldots,X_s) -\\
	&- \sum_{k=1}^s t(\alpha_1,\ldots,\alpha_r,X_1,\ldots,\mathcal F_UX_k,\ldots,X_s)\punto
\end{align*}
Usando ancora \ref{eod:LeibnitzPerE} e \ref{eod:LeibnitzPerF} si vede come prima che $\mathcal D_U$ su $\Tau_s^r(U)$ è $C^\infty(M)$-multilineare.

Per concludere, se $V\subseteq U$ aperto, da \ref{eod:LocalitaPerE} e \ref{eod:LocalitaPerF} $\mathcal D_V(t\restrict U) = (\mathcal D_U t)\restrict V$, per $t\in\Tau(U)$.
Quindi definiamo $\mathcal D$ su $\Tau_s^r(M)$ come $(\mathcal D t) (m) = (\mathcal D_U t) (m)$ per un qualunque $U$ aperto che contiene $m$ e vale $\mathcal D_2$.
\end{proof}

\begin{corollary}
	Valgono
	\begin{enumerate}
		\item $\mathcal D(t_1\otimes t_2) = (\mathcal D t_1) \otimes t_2 + t_1\otimes (\mathcal D t_2)$
		\item $\mathcal D\delta = 0$ ($\delta$ = Kronecker)
	\end{enumerate}
\end{corollary}
\begin{proof}
	1 segue da $\mathcal D_1$.
	
	Siano $X\in\chi(U)$ e $\alpha\in\chi^*(U)$, allora: 
	\begin{align*}
	(\mathcal D\delta)(\alpha, X) = \mathcal D(\delta(\alpha,X)) - \delta(\mathcal D\alpha,X) - \delta(\alpha,\mathcal D X) = \mathcal D (\alpha X) - (\mathcal D\alpha)X - \alpha(\mathcal DX) = 0\punto %TODO: sistemare
	\end{align*}
\end{proof}

Trattiamo ora il caso particolare della derivata di Lie.
Dati $X\in\chi(M)$ e $U\subseteq M$ aperto, definiamo $\mathcal E_U$ ed $\mathcal F_U$ come $\Lie_X\restrict U$.

Per Leibnitz le ipotesi del teorema sono soddisfatte.

\begin{definition}
	Se $X\in\chi(M)$, definiamo $\Lie_X$ come l'unico operatore come sopra su $\Tau(M)$ (ancora detto \emph{derivata di Lie}), tale che $\Lie_X$ coincide con le derivate di Lie rispetto a $X$ su $C^\infty(M)$ e $\chi(M)$.
\end{definition}

\begin{proposition}
	Sia $\varphi: M \to N$ un diffeomorfismo e sia $X\in\chi(M)$. Allora $\Lie_X$ è naturale rispetto al push-forward, cioè $\Lie_{\varphi_*X} (\varphi_*t) = \varphi_*(\Lie_Xt)$ dato $t\in\Tau_s^r(M)$.
\end{proposition}
\begin{proof}
	Dato $U\subseteq M$ aperto, definiamo $\mathcal D: \Tau_s^r(U) \to \Tau_s^r(U)$ tale che $\mathcal Dt = \varphi^*\Lie_{\varphi_*X\restrict U}(\varphi_*t)$.
	La derivata di Lie (su funzioni e campi vettoriali) è naturale rispetto ai push-forward, quindi $\mathcal D$ definita come sopra è una derivazione su funzioni $C^\infty(U)$ e su $\chi(U)$.
	Mostriamo che $\mathcal D$ è un operatore differenziale:
	\begin{description}
	 \item [$\mathcal D_1$]

	Dati $X_1,\ldots,X_s\in\chi(U)$ e $\alpha_1,\ldots,\alpha_r\in\chi^*(U)$, allora
	\begin{equation*}
	\varphi_*(t(\alpha_1,\ldots,\alpha_r,X_1,\ldots,X_s) )= (\varphi_*t)(\varphi_*\alpha_1,\ldots,\varphi_*\alpha_r,\varphi_*X_1,\ldots,\varphi_*X_s)\punto
	\end{equation*}
	
	  $  \mathcal D(t(\alpha_1,\ldots,\alpha_r,X_1,\ldots,X_s)) = \varphi^*\Lie_{\varphi_*X}(\varphi_*(t(\alpha_1,\ldots,\alpha_r,X_1,\ldots,X_s))) = $
	  $  \varphi^*\Lie_{\varphi_*X}((\varphi_*t)(\varphi_*\alpha_1,\ldots,\varphi_*\alpha_r,\varphi_*X_1,\ldots,\varphi_*X_s))) = $ \\
	  $  \varphi^*[(\Lie_{\varphi_*X}\varphi_*t)(\varphi_*\alpha_1,\ldots,\varphi_*\alpha_r,\varphi_*X_1,\ldots,\varphi_*X_s) + $ \\
	  $  +\sum_{j=1}^r{(\varphi_*t)(\varphi_*\alpha_1,\ldots,\Lie_{\varphi_*X}\varphi_*\alpha_j,\ldots,\varphi_*\alpha_r,\varphi_*X_1,\ldots,\varphi_*X_s)} +$ \\
	  $  +\sum_{k=1}^s{(\varphi_*t)(\varphi_*\alpha_1,\ldots,\varphi_*\alpha_r,\varphi_*X_1,\ldots,\Lie_{\varphi_*X}\varphi_*X_k,\ldots,\varphi_*X_s)}] $
	
	Ma sapendo che $\varphi^*=(\varphi^{-1})_*$ l'ultima espressione diventa:
	
	  $ (\mathcal D t)(\alpha_1,\ldots,\alpha_r,X_1,\ldots,X_s)+\sum_{j=1}^r{t(\alpha_1,\ldots,\mathcal D \alpha_j, \ldots \alpha_r,X_1,\ldots,X_s)} +$\\
	  $ +\sum_{k=1}^s{t(\alpha_1,\ldots, \alpha_r,X_1,\ldots,\mathcal D X_k, \ldots,X_s)}$
	 
	 ed ho quindi la tesi.
	
	\item [$\mathcal D_2$]
		Sia $t\in\Tau_s^r(M)$, allora
		\begin{equation*}
			(\mathcal Dt)\restrict U = [(\varphi_*)^{-1} \Lie_{\varphi_*X}\varphi_*t]\restrict U 
			= (\varphi_*)^{-1}[\Lie_{\varphi_*X}\varphi_*t]\restrict U 
			= (\varphi_*)^{-1} \Lie_{\varphi_*X\restrict U}\varphi_*t \restrict U = \mathcal D(t\restrict U) \punto
		\end{equation*}
	\end{description}
\end{proof}

Vediamo ora le formule in coordinate.
Dato $t\in\Tau_s^r(M)$, vogliamo le componenti di $\Lie_Xt$.
Sia $\varphi: U \to \R^n$ una carta e sia $X\in\chi(M)$. Siano $X'$ e $t'$ le coordinate di $\varphi_*X$ e $\varphi_*t$.
Se $\Diff$ è il differenziale in $\R^n$ (e $Y\in\chi(M)$), allora
$(\Lie_Xf)' (x) = \Diff f'(x) X'(x)$ e $(\Lie_XY)'(x) = \Diff Y'(x)X'(x) - \Diff X'(x)Y'(x)$.

Quindi $(\Lie_X f)' = X' \DerParz{f'}{x^i}$ e $(\Lie_XY)' = X^j\DerParz{Y^i}{x^j} - Y^j\DerParz{X^i}{x^j}$.

Per l'ultima proposizione abbiamo che $\varphi_*(\Lie_X\alpha) = \Lie_{\varphi_*X}(\varphi_*\alpha) = \Lie_{X'}\alpha'$ per $\alpha\in\chi^*(M)$, dove $\alpha' : \varphi(U)\to\R^n$.

Dato $v\in\R^n$ fissato allora $\Lie_{X'}(\alpha'v) = (\Lie_{X'}\alpha')v + \alpha'\Lie_{X'}v$,

$\Diff (\alpha'v)X' = (\Lie_{X'} \alpha') v - \alpha'(\Diff X' \cdot v) \Rightarrow (\Lie_{X'}\alpha') v = (\Diff \alpha' \cdot X') v + \alpha' (\Diff X' \cdot v)$ 

Allora $(\Lie_X\alpha)_iv^i = \DerParz{\alpha^i}{x^j}X^jv^i + \alpha_j\DerParz{X^i}{x^j}v^i$
quindi $(\Lie_X\alpha)_i = X^j\DerParz{\alpha_i}{x^j} + \alpha_j\DerParz{X^j}{x^i}$



































































