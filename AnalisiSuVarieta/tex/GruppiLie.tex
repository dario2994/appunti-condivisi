\chapter{Gruppi di Lie}

\section{Gruppi e algebre di Lie}

\begin{definition} \index{gruppo!topologico}
	Un \emph{gruppo topologico} è un insieme con struttura sia di gruppo che di spazio topologico tale che le operazioni di prodotto ed inverso siano continue.
\end{definition}

\begin{definition} \index{gruppo!di Lie}
	Un \emph{gruppo di Lie} è un gruppo che ha una struttura di varietà differenziabile (di classe $C^\infty$) tale che il prodotto e l'inverso sono $C^\infty$.
\end{definition}

\begin{example}
	Sono gruppi di Lie $(\R^n,+)$, $(S^1\subseteq \C, \cdot)$, $(\C^*,\cdot)$, il toro $(T^n, +)$, $\mathrm{GL}(n,\R)$.
\end{example}

\begin{proposition}
	Sia $G$ un gruppo topologico, allora la componente connessa $K$ contenente $e=\id_G$ è un sottogruppo normale e chiuso di $G$. Se $G$ è di Lie, allora $K$ è un sottogruppo di Lie aperto.
\end{proposition}
\begin{proof}
	Sia $a\in K$, allora $a^{-1}K$ è connesso poiché $b\mapsto a^{-1}b$ è un omeomorfismo.
	Inoltre se $e\in a^{-1}K$, allora $a^{-1}K\subseteq K$. Questo è vero per ogni $a\in K$, quindi $K^{-1}K\subseteq K$ e perciò $K$ è un sottogruppo.
	
	Sia ora $b\in G$, allora $bKb^{-1}$ è connesso ed $e\in bKb^{-1}$, quindi $bKb^{-1}\subseteq K$ per ogni $b\in G$ e perciò $K$ è normale. Inoltre $K$ è chiuso essendo una componente connessa.
	
	Se $G$ è di Lie, allora è localmente connesso e quindi $K$ è aperto. Perciò $K$ è una sottovarietà e un sottogruppo e di conseguenza ha una struttura di gruppo di Lie.
\end{proof}


\begin{definition} \index{gruppo!di Lie!moltiplicazione}
	Dati $G$ gruppo di Lie ed $a\in G$, definiamo le \emph{moltiplicazioni} sinistra e destra $L_a,R_a:G\to G$ per $a$ come $L_a(b) = ab$ ed $R_a(b) = ba$.
\end{definition}

Le due funzioni $L_a,R_a$ sono diffeomorfismi con inverse $L_{a^{-1}}$ ed $R_{a^{-1}}$.
In particolare $(L_a)_* : T_bG \to T_{ab} G$ e $(R_a)_* : T_bG \to T_{ba} G$ sono isomorfismi per ogni $b\in G$.


\begin{definition} \index{campo!vettoriale!invariante a sinistra}
	Un campo $X\in\chi(G)$ è detto \emph{invariante a sinistra} se $(L_a)_*X=X$ per ogni $a\in G$.
\end{definition}

\begin{remark}
	Per avere l'invarianza a sinistra basta verificare che $(L_a)_*(X(e)) = X(a)$ per ogni $a\in G$.
\end{remark}

Dato $X_e \in T_eG$, esiste $X\in\chi(G)$ invariante a sinistra definito dalla formula precedente tale che $X(e) = X_e$.


\begin{proposition}
	Sia $X$ un campo vettoriale (senza richieste di regolarità) invariante a sinistra, allora $X$ è di classe $C^\infty$.
\end{proposition}
\begin{proof}
	Basta verificare la regolarità in un intorno dell'identità. Sia $(U,\varphi)$ carta di $G$ tale che $e\in U$ e chiamiamo $x$ le sue funzioni coordinate. Scegliamo $V\subseteq U$ che $V$ contiene $e$ e tale che $a b, a^{-1}b \in U$ per ogni $a,b\in V$.
	Dato $a\in V$, abbiamo 
	\begin{equation*}
	X^i(a) = X(x^i)(a) = (L_a)_* X_e (x^i) = X_e (x^i\circ L_a) \punto
	\end{equation*}
	Poiché $(a,b) \mapsto ab$ è $C^\infty$, allora $x^i(ab) = x^i( L_a(b) ) = f^i(x^1(a),\ldots, x^n(a), x^1(b), \ldots, x^n(b) )$ con $f^i$ funzioni di classe $C^\infty$.
	
	Ponendo $X_e = \sum_{j=1}^n c^j \DerParz{}{x^j}\restrict e$, abbiamo
	\begin{equation*}
	X^i (a) = X_e (x^i \circ L_a) = \sum_{j=1}^n c^j \DerParz{(x^i\circ L_a)}{x^j} \restrict e = 
	\sum_{j=1}^n c^j \DerParz{f^i}{x^{n+j}}(x(a), x(e))\virgola
	\end{equation*}
	quindi $X^i$ è $C^\infty$ e di conseguenza anche $X$.
\end{proof}

\begin{corollary}
	Un gruppo di Lie $G$ ha sempre fibrato tangente banale e quindi è orientabile.
\end{corollary}
\begin{proof}
	Sia $X_{1,e},\ldots,X_{n,e}$ base di $T_eG$ e siano $\seqb Xn,$ estensioni degli elementi della base a campi vettoriali invarianti a sinistra. Allora $\seqb Xn,$ sono ovunque linearmente indipendenti e possiamo definire $f:TG \to G\times \R^n$ tale che $f(\sum c^iX_i(a)) = (a,\seqa cn,)$, che risulta un diffeomorfismo.
\end{proof}

Dalle proprietà del push-forward $(L_a)_* \comm XY = \comm {(L_a)_*X}{(L_a)_*Y}$. In particolare, se $X,Y$ sono invarianti a sinistra, allora anche $\comm XY$ è invariante a sinistra.

Dati $X,Y \in T_eG$, definiamo con $\tilde X,\tilde Y$ le loro estensioni invarianti a sinistra.
Definiamo quindi un'operazione $\comm\cdot\cdot$ su $T_eG$ tramite $\comm XY \coloneqq \comm{\tilde X}{\tilde Y}(e)$. %TODO: le parentesi intorno ai tilde escono enormi

\begin{definition} \index{algebra!di Lie}
	Il tangente $T_eG$ con questa operazione è detto \emph{algebra di Lie} di $G$ ed è indicato $\Lie(G)$ o $\mathfrak g$.
\end{definition}

L'algebra di Lie di un gruppo di Lie è finito dimensionale e soddisfa $\comm XX = 0$ e l'identità di Jacobi $\comm {\comm XY} Z + \comm {\comm YZ} X + \comm {\comm ZX} Y = 0$.


\begin{definition} \index{algebra!di Lie!commutativa}
	Un'algebra di Lie è detta \emph{commutativa} se $\comm XY = 0$ per ogni $X,Y \in T_eG$.
\end{definition}

\begin{remark}
	L'algebra di Lie di $(\R^n,+)$ è commutativa.
\end{remark}


Sia $H$ un sottogruppo di Lie del gruppo di Lie $G$ e sia $i: H \hookrightarrow G$ l'inclusione. Allora $T_eH$ si identifica con un sottospazio di $T_eG$. Vediamo inoltre che $T_eH$ risulta una sottoalgebra $\mathfrak h$ di $\mathfrak g$, cioè $T_eH$ è chiuso rispetto a $\comm\cdot\cdot$.


Ogni $X \in T_eH$ si estende a $\tilde X \in \chi(H)$ invariante a sinistra in $H$ e a $\hat X \in \chi(G)$ invariante a sinistra in $G$.
Sia $a\in H$ e siano $L_a:H\to H$, $\hat L_a:G \to G$ le trasformazioni a sinistra, allora $\hat L_a \circ i = i \circ L_a$. Quindi
\begin{equation*}
i_* \tilde X(a) = i_* (L_a)_* X = (\hat L_a)_* (i_* X) = \hat X(a)\virgola
\end{equation*}
perciò $i_* \tilde X = \hat X$ in $H$.
Dunque, se $Y \in T_eH$, allora $\comm {\hat X}{\hat Y}(e) = i_*\left( \comm{\tilde X}{\tilde Y}\right)(e)$, da cui quanto cercato.

Vediamo ora che vale anche un viceversa, nella forma del seguente teorema.
\begin{theorem}
	Siano $G$ un gruppo di Lie e $\mathfrak h$ una sottoalgebra di $\mathfrak g$. Allora esiste unico un sottogruppo di Lie $H$ di $G$ connesso tale che $\Lie (H) = \mathfrak h$.
\end{theorem}
\begin{proof} %TODO: manca la dimostrazione dell'unicità di H?
	Dato $a \in G$, sia $\Delta_a$ il sottospazio di $T_aG$ generato dagli $\tilde X(a)$ con $X\in \mathfrak h$.
	Poiché $\mathfrak h$ è una sottoalgebra, $\Delta$ è una distribuzione integrabile per la \cref{prop:CondizioneIntegrabilitaDistribuzioniDaGeneratori}.
	Sia $H$ una varietà integrale massimale contenente $e$, la cui esistenza ci è garantita dal \cref{thm:Frobenius}. Se $b \in G$, allora $(L_b)_* (\Delta_a) = \Delta_{ba}$. Quindi $(L_b)_*$ lascia invariante la distribuzione. Perciò $L_b$ permuta le varietà integrali massimali di $\Delta$.
	
	Se $b\in H$, $L_{b^{-1}}$ porta $H$ nella varietà integrale massimale contenente $L_{b^{-1}} (b) =e$; di conseguenza $L_{b^{-1}} H = H$ e perciò $H$ è un sottogruppo di $G$.
	Manca ora verificare che $(a,b) \mapsto ab$ e $a\mapsto a^{-1}$ sono operazioni $C^\infty$.
	Però lo spazio tangente di $H$ è $\Delta$, che è di classe $C^\infty$ poiché generata da campi vettoriali invarianti a sinistra e quindi $C^\infty$.
	Come prima si dimostra la regolarità delle funzioni coordinate di prodotti.
\end{proof}

\begin{remark}
	Una sottovarietà di un gruppo di Lie che è anche un sottogruppo è un sottogruppo di Lie.
\end{remark}

\section{Omomorfismi fra gruppi di Lie}

Siano $G,H$ gruppi di Lie e $\phi : G\to H$ un omomorfismo di classe $C^\infty$.
Allora è definito $\phi_* : T_eG \to T_eH$.

Se $X \in T_eG$ definiamo $\tilde{\tilde X}$ come l'estensione in $\chi(H)$ invariante a sinistra di $\phi_*X $ che vale $\phi_*X$ in $e_H$.
Allora, poiché per ogni $a \in G$ vale $\phi \circ L_a = L_{\phi(a)} \circ \phi$, abbiamo 
\begin{equation*}
\phi_* \tilde X(a) = \phi_* (L_a)_*X = (L_{\phi(a)})_* \phi_*X = \tilde{\tilde X} (\phi(a))
\end{equation*}
con $\tilde X\in\chi(G)$ invariante a sinistra tale che $\tilde X(e) = X$.

Quindi $\tilde{\tilde X} \phi = \phi_*\tilde X$ e perciò $(\phi_*)_e\coloneqq \phi_*\restrict{T_eG}:\mathfrak g \to \mathfrak h$ è un omomorfismo di algebra di Lie, cioè una mappa lineare che soddisfa $(\phi_*)_e \comm XY _{\mathfrak g} = \comm{ (\phi_*)_eX}{ (\phi_*)_eY }_{\mathfrak h}$.

\begin{theorem}\label{thm:OmomorfismiAlgebreLie}
	Siano $G,H$ gruppi di Lie e $\Phi: \mathfrak g \to \mathfrak h$ un omomorfismo di algebre di Lie. Allora esiste $U$ intorno di $e$ in $G$ e una mappa $\phi: U \to H$ di classe $C^\infty$ tale che $\phi(ab) = \phi(a) \phi(b)$ per ogni $a,b \in U$ con $ab \in U$ e tale che $(\phi_*)_e X = \Phi X$ per ogni $X\in\mathfrak g$.
	Inoltre, se esistono $\phi,\psi : G \to H$ omomorfismi $C^\infty$ tali che $(\phi_*)_e = (\psi_*)_e = \Phi$ e se $G$ è connesso, allora $\phi = \psi$.
\end{theorem}
\begin{proof}
	Sia $\mathfrak f \subseteq \mathfrak g \times \mathfrak h$ l'insieme $\mathfrak f = \{ (X,\Phi(X)) \suchthat X \in \mathfrak g \}$.
	Dato che $\Phi$ è un omomorfismo di algebre di Lie, allora $\mathfrak f$ è una sottoalgebra di $\mathfrak g \times \mathfrak h = \Lie(G\times H)$.
	Quindi esiste $K$ sottogruppo di Lie di $G\times H$ tale che $\Lie(K) = \mathfrak f$.
	Sia $\pi_1:G\times H \to G$ la proiezione sul primo fattore e $\omega = \pi_1\restrict K$, allora $\omega: K \to G$ è un omomorfismo.
	Dato $X\in \mathfrak g$, abbiamo $\omega_*(X,\Phi(X)) = X$, quindi $\omega_*:T_{(e,e)}K\to T_e G$ è un isomorfismo.
	Perciò esiste $V$ intorno di $(e,e) \in K$ tale che $\omega$ mappa $V$ diffeomorficamente su $U$ intorno di $e\in G$.
	
	Se $\pi_2 : G\times H \to H$ è la proiezione su $H$, definiamo $\phi \coloneqq \pi_2 \circ \omega^{-1}$ su $U$. Quindi $\phi$ realizza la prima condizione. Per la seconda, sia $X\in\mathfrak g$, abbiamo $\omega_*(X,\Phi(X)) = X$, allora $\phi_*X = (\pi_2)_*(X,\Phi(X)) =\Phi(X)$.
	
	Dati $\phi,\psi:G\to H$ omomorfismi, definiamo $\theta:G \to G\times H$ iniettivo come $\theta(a) = (a,\psi(a))$. L'immagine $G'$ di $\theta$ è un sottogruppo di Lie di $G\times H$ e dato $X \in \mathfrak g$ vale $\theta_*(X) = (X,\Phi(X))$, allora $\Lie(G') = \mathfrak f$, perciò $G'=K$ e di conseguenza $\psi(a) = \phi(a)$ per ogni $a \in G$. 
\end{proof}



\begin{corollary} \label{cor:AlgebreIsomorfeLocIsomorfi}
	Se due gruppi di Lie hanno algebre di Lie isomorfe, allora sono localmente isomorfi (in un intorno dell'identità).
\end{corollary}
\begin{proof}
	Dato $\Phi: \mathfrak g \to \mathfrak h$ isomorfismo, sia $\phi$ la mappa data dal \cref{thm:OmomorfismiAlgebreLie}
	Questa soddisfa $(\phi_*)_e = \Phi$ isomorfismo, quindi $\phi$ è un diffeomorfismo in un intorno di $e \in G$.
\end{proof}



\begin{corollary}
	Sia $G$ un gruppo di Lie connesso con algebra di Lie commutativa, allora $G$ è abeliano.
\end{corollary}
\begin{proof}
	Per il \cref{cor:AlgebreIsomorfeLocIsomorfi}, $G$ è localmente isomorfo ad $\R^n$ e quindi è commutativo in un intorno di $e \in G$. Si vede che, dalla connessione, ogni intorno $U$ di $e$ genera $G$ tramite le operazioni di gruppo; perciò anche $G$ è abeliano.
\end{proof}


\begin{definition} \index{sottogruppo ad un parametro}
	Sia $G$ un gruppo di Lie, allora un omomorfismo $\phi : \R \to G$ di classe $C^\infty$ è detto un \emph{sottogruppo ad un parametro} di $G$.
\end{definition}

% Abbiamo visto che, dato $X \in T_eG \setminus \{0\}$, esiste unico un sottogruppo ad un parametro $\phi(t)$ tale che $\frac{\de\phi}{\de t} = \tilde X(\phi(t))$. 

\begin{corollary}
	Per ogni $X \in T_eG$, esiste un unico sottogruppo ad un parametro $\phi: \R \to G$ tale che $\frac{\de\phi}{\de t}\restrict{t=0} = X$.
\end{corollary}
\begin{proof}
	Sia $f:G\to \R$ e se $\phi: \R \to G$ è un omomorfismo $C^\infty$ tale che $\frac{\de\phi}{\de t}\restrict{t=0} = X$, abbiamo che
	\begin{align*}
		\frac{\de\phi}{\de t}(f) &= \lim_{h\to\infty} \frac{f(\phi(t+h))-f(\phi(t))}{h} = \lim_{h\to\infty} \frac{f(\phi(t)\phi(h))-f(\phi(t))}{h} =\\
		&=\frac{\de}{\de s} (f\circ L_{\phi(t)} \circ \phi(s)) \restrict{s=0} = (L_{\phi(t)})_* \frac{\de\phi}{\de s}\restrict{s=0} (f) = (( L_{\phi(t)} )_* X) (f) = \tilde X(\phi(t)) (f) \punto
	\end{align*}
	Quindi se esiste un tale omomorfismo $\phi$, deve essere una curva integrale di $\tilde X$ (estensione di $X$ a $G$), abbiamo quindi l'unicità.
	
	Viceversa, sia $\phi: \R \to G$ una curva integrale di $\tilde X$ e consideriamo, fissato $s$, la mappa $t\mapsto \phi(s)\cdot \phi(t)$.
	Questa è una curva integrale di $\tilde X$ che passa per $\phi(s)$ al tempo 0.
	Però la cosa vale anche per $\phi(\cdot + s)$, perciò $\phi(t+s) = \phi(s) \phi(t)$ per unicità.
	Di conseguenza $\phi$ è il sottogruppo ad un parametro cercato.
\end{proof}

\begin{remark}
	Le curve integrali esistono localmente, ma usando le proprietà di gruppo  si può mostrare l'esistenza globale.
\end{remark}


\begin{definition} \index{mappa!esponenziale}
	Siano $G$ un gruppo di Lie, $X\in T_eG$ e $\phi$ definita come sopra (cioè $\frac{\de\phi}{\de t}\restrict{t=0} =X$). Definiamo la \emph{mappa esponenziale} $\exp: \mathfrak g \to G$ tramite $\exp(X) \coloneqq \phi(1)$.
\end{definition}

\begin{remark}
Questa mappa soddisfa $\exp((t_1+t_2)X) = \exp(t_1X) \exp(t_2X)$ e $\exp(-tX) = \exp(tX)^{-1}$.
\end{remark}

\begin{proposition} \label{prop:ProprietaExp}
	La mappa $\exp: \mathfrak g \to G$ è di classe $C^\infty$ e 0 è un punto regolare, cioè esiste $U$ intorno di 0 in $T_eG$ tale che $\exp\restrict U$ è un diffeomorfismo su un intorno di $e\in G$.
	
	Inoltre, se $\psi : G \to H$ è un omomorfismo $C^\infty$, allora $\exp\circ \psi_* = \psi \circ \exp$.
\end{proposition}
\begin{proof}
	Dati $X \in T_eG$ e $a \in G$, consideriamo $T_{(X,a)}(T_eG\times G)$ che si identifica con $T_eG \times T_a G$.
	Definiamo un campo vettoriale $Y \in \chi(T_eG \times G)$ tramite $Y(X,a) = 0 \oplus \tilde X(a)$.
	Allora $Y$ genera un flusso $\alpha: \R \times (T_eG \times G) \to T_eG \times G$ di classe $C^\infty$. Si ha che $\exp X$ è la proiezione su $G$ di $\alpha(1,0\oplus X)$, quindi $\exp$ è di classe $C^\infty$.
	
	Dato $v\in T_0(T_eG)$, questo si identifica con un vettore in $T_eG$ e la curva $c(t) = tv$ (in $T_eG$) ha vettore tangente $v$ in 0.
	Quindi $(\exp_*)_0 (v) = \frac{\de \exp(c(t))}{\de t}\restrict {t=0} = \frac{\de}{\de t} \exp(tv)\restrict {t=0}$, perciò $(\exp_*)_0$ è l'identità e di conseguenza $\exp$ è un diffeomorfismo in un intorno di $0\in T_eG$.
	
	Sia $\psi: G \to H$ un omomorfismo e sia $\phi: \R \to G$ omomorfismo tale che $\frac{\de \phi}{\de t}\restrict{t=0} = X \in \mathfrak g$.
	Allora $\psi \circ \phi$ è un omomorfismo e 
	\begin{equation*}
		\frac{\de(\psi\circ \phi)}{\de t}\restrict{t=0} = \psi_*\left(\frac{\de \phi}{\de t}\restrict{t=0}\right) = \psi_*X\punto
	\end{equation*}
	Quindi $\exp(\psi_*X) = (\psi\circ\phi)(1) = \psi(\phi(1)) = \psi(\exp(X))$.
\end{proof}


\begin{corollary}
	Ogni omomorfismo iniettivo $\phi : G \to H$ di classe $C^\infty$ è un'immersione fra varietà. In particolare l'immagine è un sottogruppo di Lie di $H$.
\end{corollary}
\begin{proof}
	Per assurdo supponiamo che $\phi_*\tilde X(p) = 0$ per qualche $X \in \mathfrak g$, allora $\phi_*X = 0$. Di conseguenza, per la \cref{prop:ProprietaExp}, abbiamo $\phi(\exp(X)) = \exp(\phi_*X) = \exp(0) = e$, che contraddice l'iniettività.
\end{proof}


\begin{corollary}
	Ogni omomorfismo continuo tra gruppi di Lie è di classe $C^\infty$.
\end{corollary}
%senza dimostrazione



\section{Forme invarianti}

\begin{definition}
	Una forma $\omega$ su $G$ gruppo di Lie è detta \emph{invariante a sinistra} se soddisfa $(L_a)^*\omega = \omega$ per ogni $a \in G$, cioè $\omega(b) = (L_a)^*(\omega(ab))$.
\end{definition}

Forme di questo tipo sono determinate dal loro valore nell'identità.

Se $\seqa\omega n,$ sono 1-forme invarianti a sinistra e tali che $\omega^1(e),\ldots,\omega^n(e)$ generano $T_e^*G$, allora ogni $k$-forma $\omega$ invariante a sinistra si scrive come $\sum_{\seqb ik<} a_{\seqb ik{}}\ \omega^{i_1}\wedge \ldots \wedge\omega^{i_k}$ con coefficienti $a_{\seqb ik{}}$ costanti.

Sia $\omega$ invariante a sinistra, allora $(L_a)^*\de\omega = \de ((L_a)^*\omega) = \de \omega$. Quindi $\de\omega$ è anch'essa invariante a sinistra. 
Inoltre se $\omega$ è una 1-forma invariante a sinistra e $\tilde X,\tilde Y \in \chi(G)$ sono campi vettoriali invarianti a sinistra, allora le funzioni $\omega(\tilde X), \omega(\tilde Y)$ sono invarianti a sinistra e quindi costanti (perché $G$ è connesso).
Sfruttando quanto appena detto e la \cref{prop:DeForma}, abbiamo dunque che $\de \omega (\tilde X, \tilde Y) = \tilde X \omega (\tilde Y) - \tilde Y \omega(\tilde X) - \omega(\comm {\tilde X}{\tilde Y}) = - \omega (\comm{\tilde X}{\tilde Y})$, otteniamo quindi $\de \omega (e) (X,Y) = - \omega(e) (\comm XY)$.

Siano $\omega^1(e), \ldots, \omega^n(e)$ come sopra e consideriamo una base duale $\seqb Xn,$ di $T_eG$. Esistono delle costanti $c_{ij}^k$ tali che $\comm{X_i}{X_j} = \sum_{k=1}^n c_{ij}^k X_k$. Questo implica anche che $\comm{\tilde X_i}{\tilde X_j} = \sum_{k=1}^n c_{ij}^k \tilde X_k$.

\begin{definition}
	I coefficienti $c_{ij}^k$ sono detti \emph{costanti di struttura} di $G$ rispetto alla base $\seqb Xn,$.
\end{definition}

Grazie all'antisimmetria della parentesi di Lie e Jacobi dati dalla \cref{prop:ProprietaParentesiLie}, le costanti di struttura rispettano:
\begin{enumerate}
	\item $c_{ij}^k = -c_{ji}^k$;
	\item $\sum_{\alpha=1}^n c_{ij}^\alpha c_{\alpha k}^l + c_{jk}^\alpha c_{\alpha i}^l + c_{ki}^\alpha c_{\alpha j}^l = 0$.
\end{enumerate}
Inoltre, per quanto detto prima, abbiamo $\de \omega^k = -\sum_{i<j} c_{ij}^k \omega^i \wedge \omega^j = -\frac 12 \sum_{i,j} c_{ij}^k \omega^i \wedge \omega^j$.


\begin{theorem}
	Sia $G$ gruppo di Lie con base di 1-forme invarianti a sinistra $\seqa \omega n,$ e costanti di struttura $c_{ij}^k$. Sia $M^n$ una varietà differenziabile con 1-forme $\seqa \theta n,$ linearmente indipendenti tali che $\de \theta^k = - \sum_{i<j} c_{ij}^k \theta^i\wedge \theta^j$.
	Allora per ogni $p\in M$ esiste $U$ intorno di $p$ ed $f: U \to G$ diffeomorfismo tale che $\theta^i = f^*\omega^i$. 
\end{theorem}
\begin{proof}
	Siano $\pi_i : M \times G\to M,G$ le proiezioni sui fattori e siano $\bar\theta^k = \pi_1^*\theta^k$ e $\bar \omega^k = \pi_2^* \omega^k$.
	Allora
	\begin{equation} \label{eq:starstar}
		\de (\bar \theta^k - \bar \omega^k) = -\sum_{i<j} c_{ij}^k ([\bar \theta^i \wedge \bar \theta^j] - [\bar \omega^i \wedge \bar \omega^j]) = - \sum_{i<j} c_{ij}^k [\bar \theta^i \wedge (\bar\theta^j - \bar \omega^j) + (\bar\theta^i - \bar \omega^i)\wedge \bar \omega^j] \punto
	\end{equation}
	
	Consideriamo la distribuzione in $M\times G$ (che si dimostra avere dimensione $n$) generata da tutti i campi vettoriali che sono annullati da ogni $\bar \theta^k - \bar \omega^k$. Vediamo che l'\cref{eq:starstar}, ci dà l'integrabilità di questa distribuzione.
	
	Infatti, siano $X,Y$ campi nella distribuzione, allora
		\begin{align*}
			(\bar\theta^k - \bar\omega^k) (\comm{X}{ Y}) &= -\de (\bar\theta^k - \bar\omega^k)(X, Y) + X((\bar\theta^k - \bar\omega^k)(Y)) - Y((\bar\theta^k - \bar\omega^k) (X)) =\\
			&= - \de (\bar\theta^k - \bar\omega^k) (X,Y) = 0\virgola
		\end{align*}
		dove l'ultima uguaglianza è data proprio dall'\cref{eq:starstar}. Questo ci dice proprio che se $X,Y$ stanno nella distribuzione, anche $\comm XY$ appartiene a tale distribuzione.	
	Quindi, per il \cref{thm:Frobenius}, dato $a \in G$ troviamo una varietà integrale $\Gamma$ che passa per $(p,a)$.
	
	Le forme $\seqa {\bar \theta}n, , \seqa {\bar\omega}n,$ sono linearmente indipendenti in $M\times G$, quindi sia $\seqa{\bar\theta}n,$ che $\seqa {\bar\omega}n,$ sono linearmente indipendenti su $\Gamma$.
	Di conseguenza $\pi_1: \Gamma \to M$ e $\pi_2: \Gamma \to G$ sono diffeomorfismi locali e perciò $\Gamma$ è il grafico di un diffeomorfismo $f$ da $U$ in un intorno di $a$ in $G$.
	Sia $\tilde f: U \to M\times G$ tale che $\tilde f(q) = (q,f(q)) \subseteq \Gamma$.
	Visto che $(\bar\theta^k - \bar\omega^k)\restrict \Gamma = 0$, allora $0 = \tilde f^*(\bar\theta^k - \bar\omega^k) = \tilde f^* \pi_1^*\theta^k - \tilde f^* \pi_2^*\omega^k = (\pi_1\circ \tilde f)^* \theta^k - (\pi_2 \circ \tilde f)^* \omega^k = \theta^k - f^* \omega^k$.
\end{proof}


\begin{exercise}
	Mostrare che il fibrato tangente $TG$ di $G$ gruppo di Lie ammette struttura di gruppo di Lie.
\end{exercise}
\begin{exercise}
	Siano $X,Y \in T_eG$ tali che $\comm XY = 0$. Mostrare che
	\begin{enumerate}
		\item $\exp(sX) \exp (tY) = \exp(tY) \exp(sX)$;
		\item $\exp(X+Y) = \exp X \exp Y$.
	\end{enumerate}
\end{exercise}



