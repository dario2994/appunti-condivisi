\chapter{Teorema di Stokes}

%Prossime lezioni: 17 marzo, 31 marzo
%Esame: 20 maggio e settimana successiva 



\section{Orientazione}

\begin{definition} \index{varietà!orientabile} \index{forma!di volume}
	Una varietà $n$-dimensionale connessa $M$ è detta \emph{orientabile} se esiste $\mu\in\Omega^n(M)$ tale che $\mu(p)\not=0$ per ogni $p\in M$. Una tale forma è detta una \emph{forma di volume} su $M$.
\end{definition}

\begin{proposition}
	Sia $M$ una $n$-varieta connessa. Allora $M$ è orientabile se e solo se esiste un atlante $(U_i,\varphi_i)$ con $\varphi_i:U_i\to V_i\subseteq \R^n$ tale che il determinante jacobiano delle mappe di transizione sia positivo.
\end{proposition}

\begin{definition} \index{varietà!orientata}
	Sia $M$ varietà orientabile. Due forme di volume $\mu_1,\mu_2$ sono dette equivalenti se esiste $f\in C^\infty(M)$, $f>0$ tale che $\mu_1= f \mu_2$.
	
	Una \emph{varietà orientata} è una varietà $M$ con un'orientazione $[\mu]$, dove per orientazione si intende una classe di equivalenza di forme di volume.
\end{definition}

\begin{exercise}
Mostrare che $S^n$ è orientabile per ogni $n\in\N$.
\end{exercise}
\begin{exercise}
	\begin{enumerate}
		\item Sia $\sigma:M\to M$ tale che $\sigma^2=\id$. Supponiamo che esista $\pi:M\to N$ tale che $\de \pi$ sia invertibile puntualmente e $\pi^{-1}(n) = \{m,\sigma(m)\}$. Sia $\Omega_\pm(M) = \{ \alpha\in\Omega^n(M) \suchthat \sigma^*\alpha = \pm\alpha \}$. Mostrare che $\pi^* : \Omega^n(N) \to \Omega_+(M)$ è un isomorfismo.
		
		\item Mostrare che $\R \mathbb P^n$ è orientabile per $n$ dispari e non orientabile per $n$ pari.
		
		\emph{Suggerimento:} Prendere $M=S^n$, $N=\R\mathbb P^n$, $\sigma(x)=-x$ nel punto precedente. Mostrare che $\sigma^*\omega_{S^n} = (-1)^{n+1} \omega_{S^n}$.
	\end{enumerate}
\end{exercise}
\begin{exercise}
	\emph{(Nastro di M\"{o}bius)} Il nastro di M\"{o}bius $\mathbb M$ si ottiene quozientando $\R^2$ tramite la relazione di equivalenza data da $(x,y) \sim (x+k,(-1)^ky)$ per ogni $k\in\Z$. \index{nastro di M\"{o}bius} 
	In particolare $\mathbb M$ risulta una varietà connessa non compatta. Definiamo $\sigma:\R^2\to\R^2$ tale che $\sigma(x,y) = (x+1,-y)$. Sia $\pi$ la proiezione da $\R^2$ in $\mathbb M$, allora $\pi\circ\sigma=\pi$.
	Se $\nu\in\Omega^2(\mathbb M)$, sia $f\in C^\infty(\R^2)$ tale che $f\omega_{\R^2} = \pi^*\nu$.
	Mostrare che $f(x+1,-y) = -f(x,y)$ e che $f$ si deve annullare in qualche punto.
\end{exercise}

\begin{proposition} [Criterio di orientabilità]
	Sia $N$ una varietà $n$-dimensionale orientabile e sia $M$ una sottovarietà $k$-dimesionale con fibrato normale banale, cioè esistono $V_i:M\to TN$ per $i=1,\ldots,n-k$ tali che $\spanrm\{ T_pM,V_1(p), \ldots, V_{n-k}(p) \} = T_pN$ per ogni $p\in M$.
	Allora $M$ è orientabile.
\end{proposition}

\begin{corollary}
	Sia $M$ orientabile e sia $f \in C^\infty(M)$. Sia $c\in\R$ valore regolare per $f$ (cioè $\grad f (p)\not=0$ per ogni $f(p)=c$). Allora $f^{-1}(c)$ è orientabile. 
\end{corollary}

\begin{remark}
	Se $M$ non è orientabile, esiste un doppio rivestimento $\tilde M$ di $M$ orientabile.
	
	Sia $\tilde M \coloneqq \{ (m,[\mu_m]) \suchthat m\in M, \text{$[\mu_m]$ orientazione di $T_mM$ } \}$. Definiamo delle carte su $\tilde M$ come segue. Sia $[\omega]$ un'orientazione di $\R^n$ e sia $A\in\Lin(\R^n,\R^n)$ tale che $A(e_1) = -e_1$ e $A(e_i) = e_i$ per $i=2,\ldots,n$, dove $\seqb en,$ è la base standard di $\R^n$. Allora ovviamente $A$ cambia l'orientazione di $\R^n$.
	Se $\varphi:U\subseteq M \to V\subseteq \R^n$ è una carta di $M$, siano $\tilde U^\pm = \{ (u,[\mu_u]) \suchthat u \in U, [\varphi_*(\mu_u)] = \pm[\omega] \}$ e poniamo $\tilde\varphi^+(u,[\mu_u]) = \varphi(u)$, $\tilde\varphi^-(u,[\mu_u]) = (A\circ\varphi)(u)$.
	Allora $(\tilde U^\pm,\tilde\varphi^\pm)$ sono carte di $\tilde M$ che la rendono una varietà orientabile.
	
	Inoltre $\pi:\tilde M\to M$ tale che $\pi(m, [\mu_m]) = m$ è un doppio rivestimento di $M$ con $\pi^{-1}(m) = \{ (m,[\mu_m] ), (m,-[\mu_m]) \}$.
\end{remark}


\section{Integrazione su varietà}

\begin{definition}
	Sia $U$ un aperto di $\R^n$ e $\omega\in\Omega^n(U)$ a supporto compatto. Se $\omega(x) = \frac 1{n!}\ \omega_{\seqb in{}} \de x^{i_1}\wedge \ldots \wedge \de x^{i_n} = \omega_{1\ldots n} \seqa{\de x}n\wedge$, definiamo
	\begin{equation*}
		\int_U \omega \coloneqq \int_{\R^n} \omega_{1 \ldots n} \seqa {\de x}n{} \punto 
	\end{equation*}
\end{definition}

\begin{proposition} \label{prop:CambioVariabileRn}
	Siano $U,V\in\R^n$ aperti e sia $f:U\to V$ diffeomorfismo. Supponiamo che $f$ preservi l'orientazione. Se $\omega \in \Omega^n(V)$ ha supporto compatto, allora $f^*\omega \in \Omega^n(U)$ ha supporto compatto e 
	\begin{equation*}
		\int_V \omega = \int_U f^*\omega \punto
	\end{equation*}
\end{proposition}
\begin{proof}
	Applichiamo la \cref{prop:DeterminanteCoordinate} con $e_i= \DerParz{}{x^i}$, allora $A_i^j = \DerParz{f^j}{x^i}$ è la matrice jacobiana $Jf$ di $f$. Quindi
	\begin{align*}
		\int_U f^*\omega &= \int_U \det (Jf) \cdot (\omega\circ f) = \int_U \det (Jf)(x)\cdot \omega_{1\ldots n}(f(x)) \seqa {\de x}n{} =\\
		&= \int_V \omega_{1\ldots n}(y) \seqa {\de y}n{} = \int_V \omega \virgola
	\end{align*}
	dove abbiamo sostituito $y=f(x)$ e abbiamo sfruttato che $f$ preserva l'orientazione per il cambio di variabile.
\end{proof}

\begin{definition}
	Sia $M$ una varietà orientata con orientazione $[\Omega]$. Supponiamo che $\omega \in \Omega^n(M)$ abbia supporto compatto contenuto in $U$, dove $(U,\varphi)$ è una carta di $M$ orientata positivamente.
	Definiamo
	\begin{equation*}
		\int_{(\varphi)} \omega \coloneqq \int_{\varphi(U)} \varphi_*(\omega \restrict U) \punto
	\end{equation*}
\end{definition}

\begin{proposition}
	Supponiamo che $\omega\in \Omega^n(M)$ abbia supporto compatto $C\subseteq U\cap V$, con $(U,\varphi)$, $(V,\psi)$ carte orientate positivamente in $M$ varietà orientata.
	Allora $\int_{(\varphi)} \omega = \int_{(\psi)} \omega$.
\end{proposition}
\begin{proof}
	Per la \cref{prop:CambioVariabileRn}, abbiamo
	\begin{equation*}
		\int_{(\varphi)} \omega = \int_{\varphi(U)} \varphi_*(\omega\restrict U) = \int_{\varphi(U\cap V)} \varphi_*(\omega\restrict {U\cap V}) =\int_{\psi(U\cap V)} (\psi\circ \varphi^{-1})_* \varphi_*(\omega \restrict {U\cap V}) = \int_{(\psi)} \omega \punto
	\end{equation*}
\end{proof}

\begin{corollary}
	Se $\omega$ ha supporto compatto in $U$, dominio di una carta $(U,\varphi)$, è ben definito
	\begin{equation*}
		\int_U \omega \coloneqq \int_{(\varphi)} \omega \punto
	\end{equation*}
\end{corollary}


\begin{definition} \index{integrale} \label{def:IntegraleForma}
	Sia $\omega \in \Omega^n(M)$ a supporto compatto. Sia $\mathcal A$ un atlante di $M$ orientabile composto da carte con orientazione positiva. Sia $(U_\alpha, \varphi_\alpha, \chi_\alpha)$ una partizione dell'unità subordinata ad $\mathcal A$. Sia $\omega_\alpha \coloneqq \chi_\alpha\omega$, allora $\omega_\alpha$ ha supporto compatto nella carta $U_\alpha\in\mathcal A$.
	Definiamo allora l'\emph{integrale} della forma $\omega$ su $M$ come
	\begin{equation*}
		\int_M \omega \coloneqq \sum_\alpha \int_{U_\alpha} \omega_\alpha \punto
	\end{equation*}
\end{definition}

\begin{proposition}
	Valgono i due seguenti fatti riferiti alla \cref{def:IntegraleForma}.
	\begin{enumerate}
		\item La sommatoria contiene solo un numero finito di termini.
		\item La definizione è indipendente dalla scelta dell'atlante (orientato positivamente) e della partizione dell'unità.
	\end{enumerate}
\end{proposition}
\begin{proof}
	\begin{enumerate}
		\item Per ogni $p\in M$, esiste $U$ intorno tale che solo un numero finito di $\chi_\alpha$ sono non nulle. Inoltre, per compattezza, possiamo ricoprire il supporto di $\omega$ con un numero finito di tali intorni.
		
		\item Sia $(V_\beta, \psi_\beta, \tilde \chi_\beta)$ un'altra partizione dell'unità subordinata a $\mathcal B$ atlante orientato positivamente. Allora le funzioni $(\chi_\alpha \tilde\chi_\beta)_{\alpha\beta}$ soddisfano $\chi_\alpha \tilde\chi_\beta(p) = 0$ tranne che per un numero finito di indici. Inoltre $\sum_{\alpha,\beta} \chi_\alpha \tilde\chi_\beta(p) = 1$ per  ogni $p$.
		Abbiamo
		\begin{align*}
			\int_\alpha \omega &= \sum_\alpha \int_{U_\alpha}\chi_\alpha \omega = \sum_{\alpha,\beta} \int_{U_\alpha \cap U_\beta} \tilde \chi_\beta \chi_\alpha \omega =\sum_{\alpha,\beta} \int_{U_\alpha\cap U_\beta}  \chi_\alpha \tilde \chi_\beta \omega =\\
			&=\sum_\beta \int_{U_\beta} \tilde\chi_\beta \omega = \int_\beta \omega \virgola
		\end{align*}
		dove abbiamo usato $\sum \tilde \chi_\beta = 1$ e $\sum \chi_\alpha = 1$.
	\end{enumerate}
\end{proof}

\begin{exercise}
	Siano $M,N$ orientate ed $f:M\to N$ un diffeomorfismo che preserva l'orientazione. Mostrare che, se $\omega\in\Omega^n(N)$ ha supporto compatto, allora $\int_N \omega = \int_M f^*\omega$.
\end{exercise}

\begin{exercise}
	(Fubini) Siano $M^m,N^n$ varietà orientate e si orienti il prodotto $M\times N$ con l'orientazione prodotto (tramite il prodotto wedge). Siano $p_M,p_N : M\times N \to M,N$ le proiezioni dal prodotto ai fattori.
	Siano $\alpha \in \Omega^m(M)$ e $\beta \in \Omega^n(N)$ e definiamo
	$\alpha \times \beta \coloneqq (p_M^*\alpha) \wedge (p_N^*\beta)$, che è a supporto compatto.
	Mostrare che $\int_{M\times N} \alpha \times \beta = (\int_M\alpha )( \int_N \beta)$.
\end{exercise}


\begin{remark}
	Data una scelta di orientazione $[\mu]$ su una $m$-varietà $M$, diciamo che una base $\seqb vm,$ di $T_xM$ è orientata positivamente se $\mu(\seqb vm,)\ge 0$.
\end{remark}


\section{Varietà con bordo}

\begin{definition} \label{def:MappeSemispazi}
	Siano $E_1,E_2\subseteq \R^n$ semispazi chiusi. Siano $U,V$ aperti di $E_1,E_2$. Una mappa $f:U\to V$ è regolare se per ogni $x\in U$ esiste $U_1$ intorno di $x$ in $\R^n$ e $V_1$ intorno di $f(x)$ in $\R^n$ ed esiste $f_1:U_1\to V_1$ regolare tale che $f\restrict{U\cap U_1} = f_1\restrict{U\cap U_1}$.
	In questo caso definiamo il differenziale di $f$ in $x$ come $\Diff f(x) \coloneqq \Diff f_1(x)$.
	
	La mappa $f$ è detta diffeomorfismo se esiste $g:V\to U$ regolare che è l'inversa di $f$.
\end{definition}

\begin{remark}
	\begin{enumerate}
		\item $\Diff f$ non dipende dalla scelta dell'estensione $f_1$. Questo è ovvio se $x$ non sta sul bordo, mentre al bordo si ragiona per approssimazione.
		
		\item Siano $U$ aperto in $E_1$, $V$ aperto in $E_2$, $f:U \to V$ diffeomorfismo. Allora le restrizioni $\Int f:\Int U \to \Int V$ e $\partial f: \partial U \to \partial V$ sono diffeomorfismi.
	\end{enumerate}
\end{remark}

\begin{definition} \index{varietà!con bordo}
	Una \emph{varietà con bordo} è un insieme $M$ munito di un atlante di carte di bordo, cioè coppie $(U,\varphi)$ tali che $U\subseteq M$ aperto e $\varphi(U)\subseteq E$, con $E$ semispazio chiuso di $\R^n$. Si richiede che valgano la proprietà di ricoprimento e la regolarità delle mappe di transizione (nel senso della \cref{def:MappeSemispazi}).
\end{definition}

\begin{definition} \index{parte interna} \index{bordo}
	Data $M$ varietà con bordo, definiamo la \emph{parte interna} $\Int M = \bigcup_{i} \varphi_i^{-1}(\Int (\varphi_i(U_i)))$ e il \emph{bordo} $\partial M = \bigcup_i \varphi_i^{-1}(\partial (\varphi_i(U_i)))$ di $M$.
\end{definition}

Estendiamo ora alle varietà con bordo alcune definizioni date per varietà non con bordo, per definire infine l'orientazione.

Pensiamo al tangente come un sottospazio $n$-dimensionale anche per i punti sul bordo.
Una forma di volume è una $n$-forma che non si annulla in nessun punto. Diciamo che $(U,\varphi)$ carta di bordo è orientata positivamente se $T\varphi(u)$ preserva l'orientazione per ogni $u\in U$.

\begin{remark}
	È stato conveniente poter scegliere semispazi arbitrari e non solo $\R_n^+ = \{x_n\ge 0\}$. Per esempio è comodo per orientare $M = \cc 01$.
\end{remark}

Vediamo quindi ora come possiamo orientare il bordo.
\begin{definition}
	Sia $M$ una $m$-varietà orientata con bordo. Siano $x\in \partial M$ e $\varphi:U \to E$ una carta orientata positivamente, con $x\in U$ ed $E$ semispazio chiuso. Una base $\seqb v{m-1},$ di $T_x(\partial M)$ è detta orientata positivamente se $\{ (T_x\varphi)^{-1} (n), \seqb v{m-1}, \}$ è orientata positivamente in $M$ per ogni vettore $n\in\R^m$ che punta all'esterno di $E$ in $\varphi(x)$.
\end{definition}

\begin{remark}
	In $\R^m$ ogni semispazio chiuso si scrive come $E = \{x \suchthat \Lambda(x) \ge 0\}$ per qualche funzionale lineare $\Lambda$ su $\R^m$. Una scelta canonica di $n$ è $n = -\grad \Lambda$.
\end{remark}



\section{Teorema di Stokes}

\begin{theorem} [Stokes] \label{thm:Stokes} \index{teorema!di Stokes}
	Sia $M$ una varietà $n$-dimensionale orientata. Inoltre sia $\alpha \in \Omega^{n-1}(M)$ una forma a supporto compatto e sia $i:\partial M \to M$ l'inclusione del bordo in $M$. Allora
	\begin{equation*}
		\int_{\partial M} i^*\alpha = \int_M \de \alpha \punto
	\end{equation*}
	Il membro di sinistra si denota anche con $\int_{\partial M} \alpha$ e, se $\partial M = \emptyset$, intendiamo che tale integrale sia 0.
\end{theorem}
\begin{proof}
	Per linearità in $\alpha$ e per l'esistenza di partizioni dell'unità possiamo supporre che $\alpha$ abbia supporto nel dominio $U$ di un'unica carta.
	Allora in coordinate
	\begin{equation*}
		\alpha = \sum_{i=1}^n (-1)^{i-1} \alpha^i \de x^1 \wedge \ldots \wedge \hat{\de x^i} \wedge \ldots \wedge \de x^n \punto
	\end{equation*}
	Di conseguenza
	\begin{equation*}
		\de \alpha = \sum_{i=1}^n \DerParz{\alpha^i}{x^i} \seqa {\de x}{n}{\wedge} \virgola
	\end{equation*}
	e quindi
	\begin{equation*}
		\int_U \de \alpha = \sum_{i=1}^n \int_{\varphi(U)} \DerParz{\alpha^i}{x^i} \seqa {\de x}{n}{} \virgola
	\end{equation*}
	che è ben definito se $M$ è orientata.
	
	Distinguiamo ora in casi a seconda che $U$ intersechi o no il bordo di $M$:
	\begin{enumerate}
		\item ($\partial U = \emptyset$) In questo caso l'$i$-esimo termine della sommatoria è
		\begin{equation*}
			\int_{\R^n} \DerParz{\alpha^i}{x^i} \seqa {\de x}{n}{} = \int_{\R^{n-1}} \left(\int_\R \DerParz{\alpha^i}{x^i} \de x^i \right) \de x^1  \ldots \hat{\de x^i} \ldots \de x^n = 0 \virgola
		\end{equation*}
		perché $\alpha^i$ ha supporto compatto. Quindi in questo caso abbiamo quello che volevamo, perché ovviamente $\int_{\partial U} i^*\alpha = 0$.
		
		\item ($\partial U \not = \emptyset$) Possiamo supporre $E=\R^n_+$, allora l'integrale diventa
		\begin{equation*}
			\sum_{i=1}^n \int_{\R^n_+} \DerParz{\alpha^i}{x^i} \seqa {\de x}{n}{} \punto
		\end{equation*}
		Se $i<n$ ragioniamo come prima, altrimenti se $i=n$ l'integrale diventa
		\begin{equation*}
			- \int_{\R^{n-1}} \alpha^n \seqa {\de x}{n-1}{} \punto
		\end{equation*}
		D'altra parte abbiamo
		\begin{equation*}
			\int_{\partial U}\alpha = \int_{\partial \R^n_+} \alpha = \int_{\partial \R^n_+} (-1)^{n-1} \alpha^n(\seqa x{n-1}, ,0) \seqa {\de x}{n-1}{} \punto
		\end{equation*}
		L'orientazione di $\partial M$ non è quella standard, infatti la normale esterna è $-e_n = -(0,\ldots,0,1)$. Perciò, l'orientazione del bordo ha il segno di $\{-e_n, \seqb e{n-1},\}$ che è  $(-1)^n$, quindi
		\begin{equation*}
			\int_{\partial U}\alpha = (-1)^{2n-1} \int_{\R^{n-1}} \alpha^n \seqa{\de x}{n-1}{} = - \int_{\R^{n-1}} \alpha^n \seqa {\de x}{n-1}{} = \int_U\de \alpha \punto
		\end{equation*}


	\end{enumerate}


\end{proof}


\begin{definition} \index{divergenza}
	Sia $\mu$ una forma di volume su $M$ orientata e sia $X\in \chi(M)$. La \emph{divergenza} $\div X \in C^\infty(M)$ di $X$ è definita da $\Lie_X\mu = (\div X)\mu$.
\end{definition}

La divergenza si interpreta come espansione o contrazione del volume.

\begin{theorem} [Gauss]\label{thm:Gauss} \index{teorema!di Gauss}
	Sia $M$ orientata e con bordo. Sia $X\in\chi(M)$ a supporto compatto e sia $\mu$ una forma di volume su $M$. Allora
	\begin{equation*}
		\int_M (\div X) \mu = \int_{\partial M} i_X \mu \punto
	\end{equation*}
\end{theorem}
\begin{proof}
	Sfruttando il punto \ref{ppi:DeForma} del \cref{thm:ProprietaProdottoInterno} e il fatto che $\de \mu =0$ poiché è una $(n+1)$-forma, abbiamo che
	\begin{equation*}
		(\div X) \mu = \Lie_X \mu = \de (i_X \mu) + i_X \de \mu = \de (i_X \mu)
	\end{equation*}
	e di conseguenza la conclusione segue dal \cref{thm:Stokes}.
\end{proof}


Se $M$ ammette una metrica $g$ esiste un'unica normale unitaria (rispetto a $g$) che punta esternamente a $\partial M$ e che chiamiamo $n_{\partial M}$. Inoltre $g$ determina univocamente forme di volume $\mu_M,\mu_{\partial M}$ su $M,\partial M$.

\begin{corollary}
	Se $X\in\chi(M)$, allora
	\begin{equation*}
		\int_M (\div X) \mu_M = \int_{\partial M} g(X,n_{\partial M}) \mu_{\partial M} \punto
	\end{equation*}
\end{corollary}
\begin{proof}
	Localmente, scegliamo una carta $(U,\varphi)$ tale che $n_{\partial M} = - \DerParz{}{x^m}$. Perciò, se $\seqb v{m-1},$ è una base orientata di $T_x\partial M$, vale
	\begin{equation*}
	\mu_{\partial M}(x) (\seqb v{m-1},) = \mu_M(x) (-\DerParz{}{x^m},\seqb v{m-1},)\punto
	\end{equation*}
	Segue allora che
	\begin{align*}
		(i_X\mu_M)(x)(\seqb v{m-1},) &= \mu_M (X^i(x)v_i + X^m\DerParz{}{x^m},\seqb v{m-1},) =\\
		&= -X^m(x) \mu_{\partial M}(x) (\seqb v{m-1},) \punto
	\end{align*}
	Inoltre $X^m = -g(X,n_{\partial M})$, quindi basta applicare il \cref{thm:Gauss}.
\end{proof}


\begin{exercise}
	Sia $M$ una varietà compatta, orientabile e senza bordo, sia $\mu$ forma di volume e sia infine $X\in\chi(M)$ con $\div X=0$. Mostrare che per ogni $f,g \in C^\infty(M)$ vale
	\begin{equation*}
		\int_M gX(f)\mu = -\int_M fX(g) \mu \punto
	\end{equation*}
\end{exercise}
\begin{exercise}
	Sia $M$ una varietà orientabile $(n+1)$-dimensionale, sia $f:\partial M \to N^{n+1}$ una mappa regolare e sia $\omega \in \Omega^n(N)$ con $\de \omega =0$. Mostrare che se $f$ si estende ad $M$, allora $\int_{\partial M} f^*\omega =0$.
\end{exercise}


\section{Formula di coarea}

Sia $(M,g)$ una varietà orientabile $(m+1)$-dimensionale. Sia $f:M \to \R$ una funzione regolare senza punti critici. Allora $M_t = f^{-1} (t)$ è una sottovarietà regolare $m$-dimensionale orientabile, che supponiamo compatta per ogni $t$.
Inoltre per ogni $t$ chiamiamo $\mu_t$ la forma di volume su $M_t$ indotta da $g$, $\mu$ la forma di volume su $M$ sempre indotta da $g$ e poniamo $n = \frac{\grad f}{\abs{\grad f}_g}$.

\begin{theorem} [Formula di coarea] \label{thm:FormulaCoarea} \index{formula!di coarea}
Sia $F:M \to \R$ a supporto compatto, allora con le notazione precedenti vale
\begin{equation*}
	\int_M F \mu = \int_\R \left( \int_{M_t} \frac F{\abs{\grad f}_g} \mu_t \right) \de t \punto
\end{equation*}
\end{theorem}

\begin{proof} %TODO: magari sistemare meglio sia la dimostrazione che l'applicazione dopo
	Fissiamo $t_0\in\R$ e $p\in M_{t_0}$. Visto che $\de f(p) \not = 0$, per il teorema della funzione implicita esistono $\seqa xm,$ tali che $(f, \seqa xm,)$ sono coordinate di $M$ in un intorno di $p$.
	Quindi esiste $\rho$ funzione regolare positiva tale che $\mu = \rho \de f \wedge \seqa {\de x}m\wedge$. Allora
	\begin{multline*}
		\mu_{t_0} = i_n \mu \restrict{M_{t_0}} = \rho i_n (\de f \wedge \seqa {\de x}m\wedge)\restrict{M_{t_0}} =\\
		= \rho i_n(\de f) \wedge \seqa {\de x}m\wedge \restrict{M_{t_0}} - \rho \de f \wedge i_n (\seqa {\de x}m\wedge)\restrict{M_{t_0}} =\\
		=\rho i_n(\de f) \wedge \seqa {\de x}m\wedge \restrict{M_{t_0}} =\rho \abs{\grad f}_g \seqa {\de x}m\wedge \restrict{M_{t_0}} \virgola
	\end{multline*}
	dove abbiamo utilizzato che $\de f\restrict {M_{t_0}} = 0$.
	A questo punto integrando abbiamo la conclusione.
\end{proof}

Consideriamo come prima applicazione del \cref{thm:FormulaCoarea} il calcolo dei volumi delle sfere.

Consideriamo l'immersione $S^n \hookrightarrow \R^{n+1} = \{(t,\seqa xn,)\}$ e siano $p_\pm = S^n\cap\{t=\pm 1\} \in S^n$ i due poli. Sia $\pi:S^n \to \R$ la proiezione su $t$, allora $\de \pi \not=0$ se $t\not=\pm 1$.

Abbiamo $\pi^{-1}(t) \cong S^{n-1}_{r(t)}$ con $r(t) = (1-t^2)^{\frac 12}$, dove al pedice indichiamo il raggio della sfera.
Se ora $\theta$ è l'angolo fra l'asse $t$ e $T_pS_n$, allora $\cos\theta = \abs{p-p'} = (1-t^2)^{\frac 12}$, dove $p'$ la proiezione su $\{t\}\times \{0\}$; perciò $\abs{\grad \pi(p)} = (1-t^2)^{\frac 12}$.

Per il \cref{thm:FormulaCoarea} vale
\begin{equation*}
	\sigma_n = 2\int_0^1 \frac 1{\abs{\grad \pi}} \int_{M_t} \mu_t = 2\sigma_{n-1} \int_0^1 (1-t^2)^{\frac{n-2}{2}} \de t = \sigma_{n-1} \int_0^1 (1-s)^{\frac{n-2}{2}} s^{-\frac 12} \de s \virgola
\end{equation*}
da cui 
\begin{equation*}
	\sigma_n = \frac{2\Gamma(\frac 12)^{n+1}}{\Gamma(\frac{n+1}2)} \punto
\end{equation*}

Altre applicazioni includono:
\begin{enumerate}
	\item disuguaglianze ottimali di Sobolev, tipo $\int \abs u^{\frac n{n-1}} \le C_n \int \abs{\grad u}$;
	
	\item riarrangiamento sferico. Sia $u:\R^n\to \R$, sia $A_t = \{ u> t \}$, $u_*$ radiale descrescente tale che $\abs{A_t^*} = \abs{A_t}$ con $A_t^* = \{u^*>t\}$. Questa procedura serve a regolarizzare e conserva la normale $L^p$
	\begin{equation*}
		\int u^p = \int (u^*)^p \punto
	\end{equation*}
	Con la formula di coarea si può dimostrare
	\begin{equation*}
		\int \abs{\grad u^*}^2 \le \int \abs{\grad u}^2 \punto
	\end{equation*}
	
	%TODO: scrivere meglio
\end{enumerate}







