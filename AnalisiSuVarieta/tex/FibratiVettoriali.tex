\chapter{Fibrati vettoriali}

\section{Definizione}

L'idea è associare (in modo ragionevole) ad ogni punto di $M$ varietà uno spazio vettoriale opportuno.

\begin{definition} \index{fibrato!vettoriale}
	Un \emph{fibrato vettoriale} di rango $k\in\N$ consiste di uno spazio totale $E$, una base $M$ ($E$ ed $M$ varietà) ed una proiezione $\pi:E\to M$ (mappa regolare) tale che per ogni $x\in M$ la sua fibra $E_x\coloneqq \pi^{-1}(x)$ ha una struttura di spazio vettoriale.
	
	Si richiede inoltre una ``banalità locale'', cioè che localmente $E$ sia un prodotto: per ogni $x\in M$ esiste $U$ intorno di $x$ e un diffeomorfismo $\tilde\varphi:\pi^{-1}(U)\to U\times \R^k$ tale che per ogni $y\in U$ vale $\tilde\varphi_y\coloneqq \tilde\varphi\restrict{E_y}:E_y\to \{y\}\times \R^k$ è un isomorfismo.
	La coppia $(U,\tilde\varphi)$ è detta \emph{carta di fibrato}.
\end{definition}

\begin{remark}
	Un fibrato vettoriale è localmente uno spazio prodotto, ma in generale questa proprietà non è vera globalmente. Se è vera globalmente il fibrato è detto \emph{banale}.
\end{remark}

\begin{example} [Nastro di Moebius]
	Il nastro di Moebius di può vedere come un fibrato vettoriale su $S^1$, che però non è banale.
\end{example}

\section{Funzioni di transizione}

Sia $(E,\pi, M)$ un fibrato vettoriale di rango $k$, sia $(U_\alpha)_{\alpha\in A}$ un ricoprimento tramite aperti che banalizza localmente il fibrato e siano $\tilde\varphi_\alpha:\pi^{-1}(U_\alpha)\to U_\alpha\times \R^k$ le banalizzazioni locali.

\begin{definition} \index{mappa!di transizione}
	Se $U_\alpha\cap U_\beta\ne \emptyset$ definiamo la \emph{mappa di transizione} $\tilde\varphi_{\beta\alpha}:U_\alpha \cap U_\beta \to \mathrm{GL}(\R^k,\R^k)$ tramite le formule $\tilde\varphi_\beta \circ \tilde\varphi_\alpha^{-1} (x,v) = (x, \tilde\varphi_{\beta\alpha}(x)v)$ con $x\in U_\alpha \cap U_\beta$, $v\in\R^k$. Oppure analogamente $\tilde\varphi_{\beta\alpha}(x) = (\tilde\varphi_\beta\restrict{E_x})(\tilde\varphi_\alpha\restrict{E_x})^{-1}$.
\end{definition}
	
\begin{proposition}
	Valgono le seguenti proprietà della funzione di transizione
	\begin{itemize}
	 \item $\tilde\varphi_{\alpha\alpha}(x)=\id_{\R^k}$ per ogni $x\in U_\alpha$;
	 \item $\tilde\varphi_{\alpha\beta}(x)\tilde\varphi_{\beta\alpha}(x) = \id_{\R^k}$ per ogni $x\in U_\alpha\cap U_\beta$;
	 \item $\tilde\varphi_{\alpha\gamma}(x) \tilde\varphi_{\gamma\beta}(x) \tilde\varphi_{\beta\alpha}(x) = \id_{\R^k}$ per ogni $x\in U_\alpha\cap U_\beta \cap U_\gamma$.
	\end{itemize}
\end{proposition}

\begin{remark}
	È possibile ricostruire un fibrato vettoriale tramite relazioni di equivalenza in base alle funzioni di transizione.\footnote{Si può trovare nel libro di Husemoller.}
	
	In particolare si ottiene che $E = \bigsqcup_{\alpha\in A} U_\alpha \times \R^k / \sim$ dove $(x,v)\sim (y,w)$ se e solo se $x=y$ e $w = \tilde\varphi_{\beta\alpha}v$.
\end{remark}

\begin{example} %TODO: attenzione: potrei aver aggiunto dei typo
	(Fibrato tangente di una varietà) Dati $(U_1,\varphi_1)$ carta di $M$ con coordinate $(\seqa xn,)$ e $p\in U$, un vettore $v$ di $T_pM$ si esprime come $v = \sum_{i=1}^n V^i_1\DerParz{}{x^i}(\varphi_1(p))$. Sia $(U_2, \varphi_2)$ una nuova carta, con $p\in U_2$ e coordinate $(y^1,\dots, y^n)$.
	 Se $V^j_2$ sono le componenti di $v$ nella carta $(U_2,\varphi_2)$, abbiamo (per la controvarianza) che
	 \begin{equation*}
	 	V^i_2 = \sum_{j=1}^n \DerParz{y^i}{x^j}(\varphi_1(p))V^j_1\punto
	 \end{equation*}
	 La funzione di transizione da $\varphi_1$ a $\varphi_2$ è quindi
	 \begin{equation*}
	 \tilde\varphi_{21}(p)(V^1,\dots,V^n) = \left(\sum_{j=1}^n \DerParz{y^i}{x^j}(\varphi_1(p))V^j\right)_{j=1,\dots,n}\punto
	 \end{equation*}
	 Se $(U_3, \varphi_3)$ è una terza carta, con $p\in U_3$ e con coordinate $(z^1,\dots,z^n)$, allora
	 \begin{equation*}
	 	\DerParz{z^i}{x^j}(\varphi_1(p)) = \sum_{l=1}^n \DerParz{z^i}{y^l}(\varphi_2(p)) \DerParz{y^l}{x^j}(\varphi_1(p))  
	 \end{equation*}
	e quindi $\tilde\varphi_{31}=\tilde\varphi_{32}\circ \tilde\varphi_{21}$.
	
	Si può fare la stessa cosa per distribuzioni $k$-dimensionali di $M$ varietà.
	
\end{example}

\begin{example}
	Consideriamo la sfera $S^2$ come unione di $U_1 = S^2\setminus \{S\}$ e $U_2 = S^2\setminus \{N\}$, con $S$ ed $N$ polo sud e nord come già visti. Usando le coordinate stereografiche, su $U_1\cap U_2\cong \R^2\setminus\{0\}$ definiamo in coordinate polari $\tilde\varphi_{21}(r,\theta) = r^k \left(\begin{matrix} \cos(k\theta)& -\sin(k\theta)\\ \sin(k\theta)& \cos(k\theta) \end{matrix}\right)$ con $k\in\N$.
	
	Questa costruzione definisce un fibrato di rango 2 su $S^2$.
\end{example}


\section{Mappe fra fibrati}

\begin{definition} \index{fibrato!vettoriale!mappa locale} \index{fibrato!vettoriale!isomorfismo locale}
	Siano $U\times\R^k$ e $U'\times\R^l$ banalizzazioni locali di fibrati vettoriali $E$ ed $E'$. Una mappa $f:U\times\R^k\to U'\times\R^l$ è detta una \emph{mappa locale} di fibrati vettoriali se ha la forma $f(p,v) = (f_1(p), f_2(p)(v) )$, dove $f_1:U\to U'$ e $f_2:U \to \Lin(\R^k,\R^l)$ sono mappe regolari.
	
	Inoltre $f$ è detta un \emph{isomorfismo locale} di fibrati vettoriali se $f_2(p)\in \mathrm{GL}(\R^k,\R^l)$.
\end{definition}


\begin{definition} \index{fibrato!vettoriale!mappa} \index{fibrato!vettoriale!isomorfismo}
	Siano $E,E'$ due fibrati vettoriali. Una mappa $f:E \to E'$ è detta una \emph{mappa} (rispettivamente un \emph{isomorfismo locale}) di fibrati vettoriali se, per ogni $v \in E$ e per ogni banalizzazione locale $(W',\tilde\psi)$ di $E'$ tale che $\pi'(f(v))\in W'$, esiste una banalizzazione locale $(W,\tilde\varphi)$ con $\pi(f(W))\subseteq \pi'(W')$ tale che il rappresentante locale $f_{\tilde\psi\tilde\varphi}\coloneqq \tilde\psi\circ f \circ \tilde\varphi^{-1}$ è una mappa locale di fibrati vettoriali (rispettivamente un isomorfismo locale).
	
	Se la mappa $f$ è biettiva, $f$ è detta un \emph{isomorfismo} di fibrati vettoriali.
\end{definition}


\begin{definition} \index{sezione!locale} \index{sezione!globale}
	Sia $\pi:E\to B$ un fibrato vettoriale. Una \emph{sezione locale} di $\pi$ è una mappa (regolare) $\xi:U \to E$ con $U$ aperto in $B$ tale che $\pi(\xi(b)) = b$ per ogni $b\in U$.
	
	Se $U = B$, la mappa $\xi$ è detta \emph{sezione globale}.
\end{definition}

\begin{remark}
	Le sezioni (diciamo $C^r$) formano uno spazio vettoriale. Usando carte di fibrato è possibile sommare nelle seconde componenti delle banalizzazioni locali.
	
	In particolare la sezione nulla (che esiste sempre) associa in ogni carta di fibrato il punto $b\in B$ al punto $(b,0)$ ed è in corrispondenza naturale con $B$.
	
	La caratteristica di Eulero è un indicatore dell'esistenza di sezioni globali non nulle ovunque.
\end{remark}

\begin{proposition} \label{prop:ProprietaMappeFibratiVettoriali}
	Sia $f:E \to E'$ una mappa di fibrati. Allora valgono le seguenti proprietà:
	\begin{enumerate}
	 \item $f$ preserva la sezione nulla (cioè, in modo improprio, $f(B)\subseteq B'$); \label{pmfv:SezioneNulla}
	 \item $f$ induce univocamente una mappa $f_B:B \to B'$ tale che il seguente diagramma commuta:  \label{pmfv:InduceMappaBasi}
	 $\begin{CD}
	 	E	@>f>>	E' \\
	 	@VV\pi V	@VV\pi' V \\
	 	B	@> f_B >> B'
	 \end{CD}\ \ $ %TODO: qui ho messo due spazi a mano che non sono granché belli
	 (cioè $\pi'\circ f = f_B \circ \pi$);
	 \item Una mappa $g: E \to E'$ è una mappa di fibrato se e solo se esiste $g_B:B \to B'$ tale che $\pi'\circ g = g_B\circ \pi$ e $g$ ristretta ad ogni fibra è continua e lineare.
	\end{enumerate}
\end{proposition}
\begin{proof}
	Sono tutte facili verifiche, eventualmente passando in carta.
\end{proof}




