\chapter{3 marzo 2016}

\begin{theorem} \label{thm:PullBackDerivataEsterna}
	Sia $f:M\to N$ regolare. Allora, per ogni $\omega \in \Omega^k(M)$ e $\psi \in \Omega^l(M)$, $f^*$ soddisfa:
	\begin{enumerate}
		\item $f^*(\omega \wedge \psi) = f^*(\omega) \wedge f^*(\psi)$; \label{pbde:Wedge}
		\item $f^*(\de \omega) = \de (f^*\omega)$. \label{pbde:DerivataEsterna}
	\end{enumerate}
\end{theorem}
\begin{proof}
	La \ref{pbde:Wedge} è già vista, per esempio come conseguenza del punto \ref{pbf:Composizione} della \cref{prop:PullBackForme}. Quindi dimostriamo il punto \ref{pbde:DerivataEsterna}.
	
	Sia $(U,\varphi)$ carta di $M$ e $(V,\rho)$ carta di $N$ tali che $f(U)\subseteq V$. Sia $\omega = \omega_{\seqb ik{}} \de x^{i_1}\wedge\ldots\wedge \de x^{i_k}$, allora
	\begin{equation*}
		\de \omega = \DerParz{\omega_{\seqb ik{}}}{x_i} \de x^i \wedge \de x^{i_1} \wedge \ldots \wedge \de x^{i_k} \puntovirgola
	\end{equation*}
	inoltre per la \ref{pbde:Wedge} abbiamo
	\begin{equation*}
		f^*\omega = f^*(\omega_{\seqb ik{}}) f^*(\de x^{i_1}) \wedge \ldots \wedge f^* (\de x^{i_k})\punto
	\end{equation*}
	Se $\psi \in C^\infty(N)$, $\de (f^*\psi) = f^* (\de \psi)$. Quindi usando nuovamente la \ref{pbde:Wedge} e $\de\circ\de =0$, abbiamo
	\begin{equation*}
		\de (f^*\omega) = f^*(\de \omega_{\seqb ik{}})\wedge f^*\de x^{i_1} \wedge \ldots \wedge f^*\de x^{i_k} = f^*(\de\omega)\punto
	\end{equation*}
\end{proof}

\begin{corollary}
	Se $f$ è un diffeomorfismo, allora $f_*(\de\omega) = \de (f_* \omega)$.
\end{corollary}

\begin{corollary} \label{cor:CommutazioneLieDerivataEsterna}
	Se $X\in\chi(M)$, $\omega \in \Omega^k(M)$, allora $\Lie_X\omega \in \Omega^k(M)$ e $\de (\Lie_X\omega) = \Lie_X \de \omega$.
\end{corollary}

\begin{proof}
Per definizione $\Lie_X\omega(p) = \frac{\de}{\de t} (F_t^*\omega)(p)\restrict{t=0}$, dove $F_t$ è il flusso generato da $X$. %TODO: rivedere un attimo questa cosa
Perciò, visto che $F_t^*\omega\in \Omega^k(M)$, otteniamo che $\Lie_X\omega \in \Omega^k(M)$.

Inoltre $\de$ commuta con il pull-back, allora $F_t^* (\de\omega) = \de (F_t^* \omega)$ passando al limite nei rapporti incrementali otteniamo proprio $\de \Lie_X = \Lie_X \de$.
\end{proof}


\section{Prodotto interno}

Ricalcando la \cref{def:ProdottoInternoTensore}, estendiamo il concetto di prodotto interno alle forme.
\begin{definition}
	Sia $M$ una varietà, $X\in\chi(M)$ e $\omega\in\Omega^{k+1}(M)$. Definiamo il \emph{prodotto interno} o \emph{contrazione} di $X$ e $\omega$ come $i_X\omega \in \Tau_k^0(M)$ tale che $i_X\omega(\seqb Xk,) = \omega(X,\seqb Xk,)$. %Chiamiamo $i_X\omega$ \emph{prodotto interno} o \emph{contrazione} di $X$ e $\omega$.
\end{definition}

\begin{remark}
	Se $\omega\in C^\infty(M)$, allora $i_X\omega = 0$.
\end{remark}

\begin{theorem} \label{thm:ProprietaProdottoInterno}
	Dato $X\in\chi(M)$, $i_X$ manda $\Omega^{k+1}(M)$ in $\Omega^k(M)$. Inoltre, se $\alpha\in\Omega^k(M)$, $\beta\in\Omega^l(M)$, $f\in C^\infty(M)$, allora
	\begin{enumerate}
		\item $i_X$ è $\R$-lineare e $i_X(\alpha\wedge\beta) = (i_X\alpha)\wedge \beta + (-1)^k \alpha \wedge (i_X\beta)$; \label{ppi:Wedge}
		\item $i_{fX}\alpha = f\ i_X\alpha$; \label{ppi:Cinfinito}
		\item $i_X\de f = \Lie_X f$; \label{ppi:DeCinfinito}
		\item $\Lie_X(\alpha\wedge\beta) = (\Lie_X \alpha) \wedge \beta + \alpha \wedge (\Lie_X \beta)$; \label{ppi:LieWedge}
		\item $\Lie_X \alpha = i_X \de\alpha + \de( i_X\alpha)$ (\emph{formula di Cartan}); \label{ppi:DeForma}\index{formula di Cartan}
		\item $\Lie_{fX}\alpha = f\Lie_X\alpha + \de f \wedge i_X\alpha$. \label{ppi:LiePerCinfinito}
	\end{enumerate}
\end{theorem}
\begin{proof}
	Il fatto che $i_X$ mandi $\Omega^{k+1}(M)$ in $\Omega^k(M)$ e che $i_X$ sia $\R$-lineare sono ovvi.
	
	%1) $i_X(\alpha \wedge \beta) (X_2,X_3,\ldots,X_{k+l}) = (\alpha\wedge\beta) (X, X_2,X_3,\ldots,X_{k+l})$.
	%D'altra parte
	%\begin{equation*}
	%(i_X\alpha\wedge \beta) + (-1)^k \alpha \wedge (i_X\beta) = \frac{(k+l-1)!}{(k-1)!\ l!}\ \antisimm (i_X\alpha \otimes \beta) + %(-1)^k \frac{(k+l-1)!}{k!\ (l-1)!} \ \antisimm (\alpha \otimes i_X\beta) \punto
	%\end{equation*}
	%Scriviamo il secondo termine come somma di permutazioni $\sigma \in S_{k+l-1}$. Scriviamo $\sigma = \tau \circ \sigma_0$ con $\tau %\in S_{k+l-1}$ e $\sigma_0(2,3,\ldots, k+1, 1, k+2, \ldots, k+l) = (1,2,3,\ldots, k+l)$.
	
	%\begin{align*}
	%	\antisimm (\alpha \otimes i_X\beta) (v_2,\ldots,v_{k+l}) &= \frac 1{(k+l-1)!} \sum_{\sigma\in S_{k-1}}\sgn(\sigma) (\alpha \otimes i_X\beta) (v_{\sigma(2)},\ldots,v_{\sigma(k+l)})=\\
	%	&= \frac 1{(k+l-1)!} \sum_{\sigma\in S_{k-1}}\sgn(\sigma) \alpha(v_{\sigma(2)},\ldots,v_{\sigma(k+1)})\beta(X,v_{\sigma(k+2)},\ldots,v_{\sigma(k+l)})=
	%\end{align*}
	%e poi boh... continua la prossima volta.
	
	%Rivediamola bene:
	
% 	Per quanto riguarda la seconda parte della \ref{ppi:Wedge} abbiamo che
% 	\begin{equation*}
% 		\alpha\wedge\beta (\seqb e{k+l},) = \sum_{\sigma \in (k,l)-mescolamenti} \sgn(\sigma) \alpha(e_{\sigma(1)}\ldots e_{\sigma(k)}) \beta(e_{\sigma(k+1)}\ldots e_{\sigma(k+l)})
% 	\end{equation*}
	
	Introduciamo innanzitutto una notazione. Dati due multi-indici $I$ e $J$ indichiamo con
	\begin{equation*}
		\delta_J^I = \begin{cases}
		             	1 & \text{se } J \text{ è una permutazione pari di }I\\
		             	-1 & \text{se } J \text{ è una permutazione dispari di }I\\
		             	0 & \text{altrimenti}
		             \end{cases}
	\end{equation*}
	Allora, con questa notazione, vale
	\begin{equation*}
		(\alpha\wedge\beta)(v_I) = \sum_{\vec J, \vec K} \delta_I^{JK} \alpha(v_J)\beta(v_K)
	\end{equation*}
	con $\vec J,\vec K$ sottoinsiemi rispettivamente di $k$ ed $l$ indici ordinati. Allora
 	\begin{align*} %TODO: questa parte fa un po' schifo
 		i_{v_1}&(\alpha\wedge\beta)(v_2,\ldots, v_{k+l}) = (\alpha\wedge\beta) (\seqb v{k+l},) =
 		\sum_{\vec I,\vec J} \delta_{1\ldots k+l}^{IJ} \alpha(v_I)\beta(v_J) =\\
 		&= \sum_{\vec I,\vec J,1\in I} \delta_{1 \ldots k+l}^{IJ} \alpha(v_I)\beta(v_J) + \sum_{\vec I,\vec J,1\in J} \delta_{1\ldots k+l}^{IJ} \alpha(v_I)\beta(v_J) =\\
 		&= \sum_{1<i_2<\ldots < i_k,\vec J} \delta_{1 \ldots k+l}^{1i_2 \ldots i_kJ}\alpha(v_1,v_{i_2},\ldots,v_{i_k}) \beta(v_J) +\\
 		& +\sum_{\vec I, 1<j_2<\ldots < j_l} \delta_{1 \ldots k+l}^{I1j_2 \ldots j_l}\alpha(v_I) \beta(v_1,v_{j_2},\ldots,v_{j_l}) =\\
 		& = \sum_{1<i_2<\ldots<i_k} \sum_{\vec J\setminus \{1\}} \delta_{2\ldots k+l}^{i_2\ldots i_kJ} (i_{v_1}\alpha)(v_{i_2},\ldots,v_{i_k}) \beta (v_J) +\\
 		& +(-1)^k \sum_{\vec I\setminus \{1\}} \sum_{1<j_2<\ldots<j_l} \delta_{2\ldots k+l}^{Ij_2\ldots j_l}\ \alpha(v_I) (i_{v_1}\beta)(v_{j_2}, \ldots, v_{j_l}) =\\
 		& = (i_{v_1}\alpha \wedge \beta)(v_2, \ldots, v_{k+l}) + (-1)^k (\alpha \wedge i_{v_1}\beta)(v_2, \ldots, v_{k+l}) \punto
 	\end{align*}
 	
 	La \ref{ppi:Cinfinito} segue dal fatto che $\alpha$ è $C^\infty(M)$-multilineare.	
	La \ref{ppi:DeCinfinito} segue dalla definizione di $\de f$, mentre la \ref{ppi:LieWedge} segue dal fatto che la derivata di Lie è una derivazione tensoriale e commuta con la mappa alternante.
	
	Vediamo quindi la \ref{ppi:DeForma} tramite un'induzione in $k$. Il risultato è vero per $k=0$ per la \ref{ppi:DeCinfinito}, supponiamo ora che valga per le $k$ forme e dimostriamolo per le $k+1$ forme. Ogni $k+1$ forma si scrive come $\sum \de f_i \wedge \omega_i$ dove $f_i \in C^\infty(M)$ e $\omega_i\in\Omega^k(M)$; infatti data $\omega$ una $(k+1)$-forma, abbiamo
	\begin{equation*}
	\omega = \omega_{\seqb i{k+1}{}} \de x^{i_1} \wedge \ldots \wedge \de x^{i_{k+1}} = \de x^{i_1} \wedge (\omega_{\seqb i{k+1}{}} \de x^{i_2} \wedge \ldots \wedge \de x^{i_{k+1}}) \punto
	\end{equation*}
	Utilizzando quindi la \ref{ppi:DeCinfinito} e l'ipotesi induttiva, abbiamo
	\begin{align*}
	i_X & \de (\de f  \wedge  \omega) + \de i_X(\de f \wedge \omega) = -i_X(\de f\wedge \de \omega) + \de (i_X(\de f) \wedge \omega - \de f \wedge i_X\omega) =\\
	&= -i_X(\de f)\wedge \de \omega + \de f\wedge i_X (\de \omega) + \de (i_X (\de f)) \wedge \omega + i_X (\de f) \wedge \de \omega + \de f \wedge \de (i_X \omega) =\\
	&= \de f(i_X(\de\omega) + \de (i_X\omega)) + \de(i_X(\de f)) \wedge \omega=\de f \wedge \Lie_X\omega + \de (\Lie_Xf) \wedge \omega \punto
	\end{align*}
	Sfruttando infine che per il punto \ref{ppi:LieWedge} abbiamo $\Lie_X(\de f\wedge\omega) = (\Lie_X\de f)\wedge \omega + \de f \wedge (\Lie_X\omega)$ e che per il \cref{cor:CommutazioneLieDerivataEsterna} vale $\de(\Lie_Xf) = \Lie_X \de f$, otteniamo il risultato.
	
	Vediamo infine che il punto \ref{ppi:LiePerCinfinito} segue dalla \ref{ppi:DeForma}, poiché
	\begin{align*}
	\Lie_{fX}\alpha &= i_{fX} \de \alpha + \de (i_{fX} \alpha) = f i_X \de \alpha + \de (fi_X\alpha) =\\
	&=f i_X\de \alpha + \de f \wedge i_X\alpha + f \de i_X\alpha = f\Lie_X \alpha + \de f \wedge i_X\alpha \punto
	\end{align*}
\end{proof}

\begin{proposition}
	Sia $f:M\to N$ e siano $\omega \in \Omega^k(N)$ e $Y \in \chi(M)$. Allora $i_Y$ è naturale rispetto al push-forward, cioè $i_Yf^*\omega = f^* i_{f_*Y}\omega$.
\end{proposition}
\begin{proof}
	Chiamiamo $X = f_*Y \in \chi(N)$ e siano $p\in M$, $q=f(p)\in N$ e $\seqb v{k-1}, \in T_pM$. Allora
	\begin{align*}
		i_Y(f^*\omega)(p)(\seqb v{k-1},) &= f^* \omega (p) (Y(p), \seqb v{k-1},) = \omega (f_*(Y(p)), f_*v_1,\ldots, f_*v_{k-1}) = \\
		&= \omega(q)((X\circ f)(p),f_*v_1,\ldots, f_*v_{k-1}) =\\
		&=(i_X\omega)(q) (f_*v_1,\ldots, f_*v_{k-1})  =(f^*i_X\omega) (p) (\seqb v{k-1},) \punto 
	\end{align*}
\end{proof}

\begin{proposition}
	Siano $X_0,X_1,\ldots,X_k\in\chi(M)$ e $\omega \in \Omega^k(M)$. Allora
	\begin{enumerate}
		\item $(\Lie_{X_0} \omega) (\seqb Xk,) = \Lie_{X_0}(\omega(\seqb Xk,)) - \sum_{i=1}^k \omega(X_1,\ldots, \Lie_{X_0}X_i, \ldots, X_k)$;
		\item $\de \omega (X_0,X_1,\ldots,X_k) = \sum_{l=0}^k (-1)^l\Lie_{X_l}(\omega(X_0,\ldots, \hat X_l, \ldots, X_k)) + \\
		+\sum_{0\le i<j\le k}(-1)^{i+j} \omega (\Lie_{X_i}(X_j),X_0,\ldots,\hat X_i,\ldots, \hat X_j,\ldots, X_k)$
	\end{enumerate}
\end{proposition}
\begin{proof} %TODO: ancora da sistemare
	La 1 è già vista. Per la 2 usiamo l'induzione su $k$.
	Per $k=0$ $\de \omega (X_0) = \Lie_{X_0} \omega$.
	
	Supponiamo valga per $k-1$ e sia $\omega \in \Omega^k(M)$. Per la formula di Cartan
	\begin{align*}
		\de \omega (X_0,\seqb Xk,) &= (i_{X_0}\de \omega)(\seqb Xk,) = \Lie_{X_0}\omega(\seqb Xk,) - (\de i_{X_0} \omega) (\seqb Xk,) =\\
		&= \Lie_{X_0}(\omega(\seqb Xk,)) - \sum_{l=1}^k \omega (X_1,\ldots,\Lie_{X_0}X_l,\ldots, X_k)+ \\
		&-(\de i_{X_0}\omega) (\seqb Xk,)
	\end{align*}
	ma visto che $i_{X_0}\omega \in \Omega^{k-1}(M)$
	\begin{align*}
		(\de i_{X_0}\omega) (\seqb Xk,) =& \sum_{l=1}^{k} (-1)^{l-1} \Lie_{X_l}(i_{X_0}\omega (X_1,\ldots, \hat X_l, \ldots, X_k)) +\\
		&+\sum_{1\le i< j \le k} (-1)^{i+j-2} (i_{X_0}\omega)(\Lie_{X_i}(X_j),X_1,\ldots, \hat X_i, \ldots, \hat X_j,\ldots, X_k) = \\
		=& \sum_{l=1}^{k} (-1)^{l-1} \Lie_{X_l}(\omega (X_0,X_1,\ldots, \hat X_l, \ldots, X_k)) +\\
		&+\sum_{1\le i< j \le k} (-1)^{i+j-2} \omega(X_0,\Lie_{X_i}(X_j),X_1,\ldots, \hat X_i, \ldots, \hat X_j,\ldots, X_k)
	\end{align*}
	Quindi
	\begin{align*}
		\de \omega (X_0,\seqb Xk,) =& \Lie_{X_0}(\omega(\seqb Xk,)) - \sum_{l=1}^{k} (-1)^{l-1} \Lie_{X_l}(\omega(X_0,X_1, \ldots, \hat X_l,\ldots, X_k)) -\\
		&-\sum_{l=1}^k\omega(X_1,\ldots, \Lie_{X_0}X_l,\ldots,X_k)-\\
		&-\sum_{1\le i<j \le k} (-1)^{i+j-2} \omega (X_0,\Lie_{X_i}(X_j),X_1,\ldots,\hat X_i,\ldots,\hat X_j,\ldots,X_k)=\\
		=&\sum_{l=0}^{k} (-1)^l \Lie_{X_l}(\omega(X_0, \ldots, \hat X_l,\ldots, X_k)) +\\
		&+\sum_{l=1}^k(-1)^l\omega(\Lie_{X_0}X_l,\ldots X_1,\ldots,\hat X_l,\ldots ,X_k)+\\
		&+\sum_{1\le i<j \le k} (-1)^{i+j} \omega (\Lie_{X_i}(X_j),X_0,X_1,\ldots,\hat X_i,\ldots,\hat X_j,\ldots,X_k)
	\end{align*}

	
\end{proof}

\begin{corollary}
	Siano $X,Y\in\chi(M)$, allora $\comm{\Lie_X}{i_Y} = i_{\comm XY}$ e $\comm{\Lie_X}{\Lie_Y} = \Lie_{\comm XY}$.
	Quindi in particolare $i_X\Lie_X = \Lie_X i_X$.
\end{corollary}
\begin{proof}
	Sia $\omega \in \Omega^k(U)$ e $\seqb X{k-1}, \in \chi(U)$, allora
	\begin{align*}
		(i_Y\Lie_X\omega)(\seqb X{k-1},) =& (\Lie_X\omega)(Y,\seqb X{k-1},) =\\
		=&\Lie_X(\omega(Y,\seqb X{k-1},)) -\\
		&-\sum_{l=1}^{k-1}\omega(Y,X_1,\ldots,\comm{X}{X_l},\ldots X_{k-1}) -\\
		&-\omega (\comm XY, \seqb X{k-1},)=\\
		=& \Lie_X(i_Y\omega(\seqb X{k-1},)) -\sum_{l=1}^{k-1} i_Y\omega(X_1,\ldots, \comm X{X_l}, \ldots, X_{k-1}) -\\
		&-i_{\comm XY} \omega (\seqb X{k-1},) =\\
		=& (\Lie_X i_Y\omega)(\seqb X{k-1},) - (i_{\comm XY}\omega)(\seqb X{k-1},) \punto
	\end{align*}

	Per la seconda si usa la prima e la formula di Cartan.
\end{proof}




































