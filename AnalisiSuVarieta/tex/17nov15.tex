\chapter{Lezione 17 novembre 2015}

\section{Campi vettoriali e flussi}


$M$ varietà, $X$ campo vettoriale di classe $C^r$, $p\in M$.
\begin{definition}
	Un flusso locale di $X$ in $p$ è una tripla $(U_0, a, F)$ tale che
	\begin{enumerate}
		\item $U_0$ aperto, $p\in U_0$, $a \in \R^+ \cup \{+\infty\}$
		\item $F:U_0\times I_a \to M$ di classe $C^r$
		\item per ogni $u\in U_0$, $c_u=F(u,\cdot):I_a\to M$ curva integrale di $X$, $c_u(0)=u$
		\item $F_t:U_0\to M$ $F_t(u)=F(u,t)$, $F_t(U_0)$ aperto, $F_t$ diffeo $C^r$
	\end{enumerate}
\end{definition}

\begin{proposition}
	Curve integrali di $X$ coincidono sull'intersezione dei loro domini.
\end{proposition}

\begin{proposition}[Esistenza e unicità di flussi locali]
	Sia $X$ campo vettoriale di classe $C^r$, allora per ogni $p\in M$ esiste un flusso locale di $X$ in $p$. Inoltre se $(U_0,a,F)$, $(U_0',a',F')$ sono flussi locali devono coincidere su $(U_0\cap U_0')\times (I_a\cap I_{a'})$.
\end{proposition}
\begin{proof}
	\begin{description}
	\item [unicità]
	$u\in U_0\cap U_0'$, sia $I=I_a\cap I_{a'}$. Allora $F_{{u}\times I}=F'_{{u}\times I}$.
	
	\item [esistenza]
	Sia $(U,\varphi)$ carta di $M$, $u\in M$. Consideriamo il rappresentate locale $X_\varphi$ di $X$ ($X_\varphi(\varphi(q))=T\varphi(x)(q)$), che genera un flusso locale $(U_\varphi,a_\varphi,F_\varphi)$ per $X_\varphi$.
	Supponiamo $U_\varphi\subseteq \varphi(U)$ e $F_\varphi(U_\varphi\times I_{a\varphi})\subseteq \varphi(U)$, $\tilde U=\varphi^{-1}(U_\varphi)$.
	Poniamo $F:\tilde U\times I_{a\varphi}\to M$ tale che $F(u,t)=\varphi^{-1}(F_\varphi(\varphi(u),t))$.
	Per continuità esiste $b\in(0,a_\varphi)$, $V\subseteq \tilde U$, $p\in V$ tale che $F(V\times I_b)\subseteq \tilde U$.
	
	$F$ verifica 1,2,3. Per avere 4 notiamo che $F_t$ ha inversa $F_{-t}$ di classe $C^r$.
	\end{description}
\end{proof}

\begin{exercise}
	\begin{itemize}
		\item Sia $M$ varietà $C^k$, $X\in\chi^k(M)$. Sia $p\in M$ tale che $X(p)\not=0$. Dimostrare che esiste $(U,\varphi)$ carta tale che $X\restrict_U=\frac{\partial}{\partial x'}\restrict_U$.
		\item Sia $F:M\times\R\to M$ regolare tale che $F_{t+s}=F_t\circ F_s$ e $F_0=F(\cdot,0)=Id_M$. Mostrare che esiste unico $X$ campo vettoriale tale che $F_t$ coincide con il flusso indotto da $X$.
	\end{itemize}
\end{exercise}

\begin{definition}
	Sia $M$ una varietà e $X$ un campo vettoriale. Sia $\mathcal D_X\subseteq M\times \R$ l'insieme degli $(x,t)$ tale che esiste $c:I\to M$ curva integrale, $c(0)=x$.
	
	$X$ è detto completo se $\mathcal D_X=M\times \R$ e completo per tempi positivi (negativi) se $\mathcal D_X\supseteq M\times\R^+$ ($\R^-$).
	
	Chiamiamo $(T(x)^-,T(x)^+)$ l'intervallo massimale di una curva che passa per $x$ al tempo 0. 
\end{definition}

\begin{example}
\begin{itemize}
	\item $M=\R^2$. $X=(1,0)$ è completo $(x,y)\to (x+t,y)$.
	\item $M=\{(x,y):x>0\}$. $X=(1,0)$ è completo per tempi positivi
	\item $M=\R$. $X(x)=1+x^2$, allora $c(t)=\arctan t$ è una curva integrale, con $c(0)=0$, allora $T^\pm(0)=\pm \frac \pi 2$.
\end{itemize}
\end{example}

\begin{proposition}
	Sia $M$ un varietà, $X\in \chi^r(M)$, $r\ge 1$. Allora:
	\begin{enumerate}
	 \item $M\times\{0\}\subseteq \mathcal D_X$;
	 \item $\mathcal D_X$ aperto;
	 \item esiste unico $F_X:\mathcal D_X\to M$ di classe $C^r$ tale che $t\mapsto F_X(p,t)$ è una curva integrale che passa per $p$ a tempo 0;
	 \item per $(p,t),(p,t+s)\in\mathcal D_X$ vale che $F_X(p,t+s)=F_X(F_X(p,t),s)$.
	\end{enumerate}
\end{proposition}

\begin{proof}
	1 e 2 seguono da esistenza e unicità delle curve integrali. L'esistenza di $F_X$ si ottiene incollando curve integrali e 4 segue dall'unicità globale.
	
	La regolarità $C^r$ globale, segue da quella locale ricoprendo una data traiettoria con un numero finito di piccoli intorni.
\end{proof}

\begin{definition}
	Chiamiamo $t\mapsto F_X(p,t)$, con $(p,t)\in\mathcal D_X$, la curva integrale massimale passante per $p$ a tempo 0.
	
	Se $X$ è completo, $F_X$ è detto il flusso generato da $X$. In questo caso abbiamo una famiglia a un parametro di diffeomorfismo.
\end{definition}

\begin{proposition}
	Supponiamo che $X$ sia a supporto compatto in $M$. Allora $X$ è completo.
\end{proposition}

\begin{proof}
	Se $p\not \in\supp (X)$, allora $T^{\pm}(p)=\pm\infty$ con $F_X(p,t)=p$.
	
	Se $p\in\supp(X)$, supponiamo per esempio $T^+(p)<\infty$. Sia $t_n$ che converge crescendo a $T^+(p)$, allora per compatteza esiste $t_{n_k}$ tale che $F_X(p,t_{n_k})$ converge a $\bar p\in M$.
	Ma $\mathcal D_X$ è aperto, quindi contiene un intorno di $(\bar p,0)$.
	Esiste $\tau >0$ (indipendente da $k$ sufficientemente grande) tale che il flusso che passa per $c(t_{n_k})$ a tempo 0 è definito almeno per un tempo $\tau$.
	Allora potevamo estendere $c(t)$ fino a $t_{n_k}+\tau$, il che è assurdo per $k$ abbastanza grande.
\end{proof}


\begin{corollary}
	Se $M$ è compatta, allora $X$ è completo.
\end{corollary}


\begin{exercise}
	Sia $X$ campo vettoriale di classe $C^r(\R)$ e sia $f:\R^n\to \R$ di classe $C^1$ e propria. Supponiamo che esistano $K,L>0$ tali che $\abs{X(f)(p)}\le K\abs{f(p)}+L$ per ogni $p\in\R^n$ \footnote{Osservazione: $X(f)=\Diff_X(f)$}. Dimostrare che $X$ è completo.
\end{exercise}


\section{Derivata di Lie}
Sia $f$ una funzione regolare su $\R^n$ e $p,v\in\R^n$. Allora $\Diff_vf(p)=\lim_{h\to 0} \frac{f(p+hv)-f(p)}h$ è la derivata direzionale in $p$ lungo $v$.

Si può ritrovare considerando $\gamma(t)=p+tv$ e calcolando $\frac \de{\de t} f(\gamma(t))\restrict_{t=0}$.

Ora vogliamo studiare il caso generale di varietà.

Sia $p\in M$ e $f$ una funzione regolare definita in un intorno di $p$. Per definizione esiste $c$ curva $C^1$ tale che $[c]_p=v$.

\begin{definition}
	Definiamo la derivata direzionale di $f$ in $p$ lungo $v$ come $(v(f))(p)=\frac \de {\de t} f(c(t))\restrict_{t=0}$.
\end{definition}

Se $(U,\varphi)$ è una carta e $(\varphi\circ c)(t)=(c^1(t),\ldots,c^n(t))$, allora $\frac \de{\de t} (f\circ c)(t)=\Diff_{[(c^1)'(0),\ldots(c^n)'(0)]}(f\circ\varphi^{-1})(\varphi(p))$. Quindi non dipende dalla scelta di $c\in[c]_p$.

Proviamo ad usare invece le ``traslazioni''.

Una curva integrale di un campo vettoriale $X$ ha velocità $X(p)$ in $p$. Scegliamo la curva integrale come $c(t)$, allora la formula precedente diventa:
\begin{equation*}
	(X(p))(f)(p)=\lim_{h\to 0} \frac 1h [f(F_X(p,h))-f(p)]
\end{equation*}
e si chiama derivata di Lie di $f$ rispetto ad $X$.

Questo approccio consente di estendere la definizione ai campi vettoriali, e non solo...

Sia $F_X^h(p)=F_X(p,h)$ la sezione del flusso locale generato da $X$. Allora $F=F_X^h$ è di classe $C^r$ ed esiste la mappa tangente $F_*(v)=TF(v)\in T_{F(q)}M$, per $q\in M$, $v\in T_qM$.

Sia $p\in M$ e $Y$ campo vettoriale definito in un intorno di $p$. Per $h\in\R^+$ abbastanza piccolo, abbiamo due vettori tangenti in $p$: $Y(p)$ e $(TF_X^h)(Y(F_X^{-h}(p)))$.

\begin{definition}
	Definiamo la derivata di Lie in $p$ di $Y$ rispetto a $X$ come \begin{equation*}(\Lie_X(Y))(p)=\lim_{h\to 0}\frac{[Y(p)-((F_X^h)_*Y)(p)]}h \punto
	\end{equation*}
\end{definition}

\begin{proposition}
	\begin{enumerate}
		\item $\Lie_XY$ è lineare in $Y$ ($X,Y\in\chi(M)$)
		\item $\Lie_X(fY)=(Xf)Y+f\Lie_XY$ ($X,Y\in\chi(M)$, $f\in C^\infty(M)$).
	\end{enumerate}
\end{proposition}

\begin{proof}
	\begin{enumerate}
	 \item Ovvia
	 \item 
	 \begin{align*}
	 (\Lie_X(fY))(p)=\lim_{h\to 0} \frac{[f(p)Y(p)-(F_X^h)_*(fY)(F_X^{-h}(p))]}h\\
	 =\lim_{h\to 0}\frac 1h [f(p)Y(p)-f(F_X^{-h}) ((F_X^h)_*(Y(F_X^{-h}(p)))]\\
	 =\lim_{h\to 0}\frac 1h \{f(p)[Y(p)-((F_X^h)_*(Y(F_X^{-h}(p)))]+....\}\\
	 =\lim_{h\to 0} (\Lie_XY)(p)+(X(f))(p)Y(p)
	 \end{align*}
	 \end{enumerate}

\end{proof}




