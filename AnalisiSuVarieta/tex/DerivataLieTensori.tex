\chapter{Derivata di Lie sui tensori}

\section{Operatore differenziale sull'algebra dei tensori}

Ci sono due possibili approcci:
\begin{enumerate}
	\item approccio dinamico, che usa i flussi ed è quello che abbiamo usato per definire la derivata di Lie sui campi vettoriali;
	\item approccio algebrico, che è invece quello che seguiremo noi e che utilizza la proprietà di Leibniz delle derivate.
\end{enumerate}

\begin{example}
	Vediamo come definire per esempio la derivata di Lie per 1-forme.
	Se $Y\in\chi(M)$ e $\alpha\in\chi^*(M)$, richiediamo che $\Lie_X(\alpha ( Y)) = (\Lie_X \alpha)(Y) + \alpha(\Lie_X Y)$, dove sappiamo cosa vuol dire la derivata di Lie su funzioni (in questo caso $\alpha(Y)$) e su campi (in questo caso $Y$), perciò la ricaviamo anche per le forme: $(\Lie_X\alpha)Y = \Lie_X(\alpha(Y)) - \alpha(\Lie_XY)$.
\end{example}

\begin{definition} \index{operatore!differenziale} \label{def:OperatoreDifferenziale}
	Un \emph{operatore differenziale} sull'algebra dei tensori $\Tau(M)$ è una famiglia di mappe $\mathcal D_s^r$ da $\Tau_s^r(M)$ in sé tale che:
	\begin{enumerate}
	 \item $\mathcal D$ sia una derivazione tensoriale, cioè commuta con le contrazioni. \label{od:Derivazione} %TODO: scrivere meglio che commuta con le contrazioni
	 
	 Vogliamo quindi che $\mathcal D$ sia $\R$-lineare e che, dati $t\in\Tau_s^r(M)$, $\alpha_1,\ldots,\alpha_r\in\chi^*(M)$ e $X_1,\ldots,X_s\in\chi(M)$, valga
	 \begin{align*}
	 	\mathcal D (t(\alpha_1,\ldots,\alpha_r,X_1,\ldots,X_s)) =\ & (\mathcal D t) (\alpha_1,\ldots,\alpha_r,X_1,\ldots,X_s) +\\
	 	&+\sum_{j=1}^r t(\alpha_1,\ldots,\mathcal D \alpha_j,\ldots, \alpha_r, X_1,\ldots, X_s) +\\
	 	&+\sum_{k=1}^s t(\alpha_1,\ldots, \alpha_r, X_1,\ldots, \mathcal D X_k,\ldots, X_s) \punto
	 \end{align*}

	\item $\mathcal D$ è locale (naturale rispetto alle restrizioni), dove supponiamo che $\mathcal D$ sia definito su $\Tau(U)$ in $\Tau(U)$ per ogni $U\subseteq M$ aperto. \label{od:Restrizioni} %[$\mathcal D_2$)]
	
	Chiediamo cioè che, se $U\subseteq V \subseteq M$ sono aperti e $t\in\Tau_s^r(M)$, allora $(\mathcal D t)\restrict U = \mathcal D (t\restrict U)$, ovvero che il seguente diagramma commuti
	\begin{equation*}
	\begin{CD} 
	 \Tau_s^r(V) @>i_U>> \Tau_s^r(U) \\
	 @V\mathcal{D}VV  @V\mathcal{D}VV \\
	 \Tau_s^r(V) @>i_U>> \Tau_s^r(U) 
	\end{CD}
	\end{equation*}

	\end{enumerate}
\end{definition}

\begin{theorem} \label{thm:EsistenzaOperatoriDifferenziali}
	Supponiamo che per ogni $U\subseteq M$ aperto esistano mappe $\mathcal E_U : C^\infty(U) \to C^\infty(U)$ e $\mathcal F_U : \chi(U) \to \chi(U)$ che siano $\R$-lineari e derivazioni (naturali rispetto alle restrizioni), cioè
	\begin{enumerate}
		\item $\mathcal E_U(f\otimes g) = (\mathcal E_U f) \otimes g + f \otimes (\mathcal E_U g)$ per $f,g\in C^\infty(U)$; \label{eod:LeibnitzPerE}
		\item $\mathcal F_U(f\otimes X) = (\mathcal E_U f) \otimes X + f\otimes \mathcal F_U X$ per $f\in C^\infty(U)$ e $X\in\chi(U)$; \label{eod:LeibnitzPerF}
		\item se $f\in C^\infty(M)$, allora $\mathcal E_U(f\restrict U) = (\mathcal E_M f)\restrict U$; \label{eod:LocalitaPerE}
		\item per $X\in\chi(M)$, vale $\mathcal F_U(X\restrict U) = (\mathcal F_M X)\restrict U$. \label{eod:LocalitaPerF}
	\end{enumerate}
	Allora esiste unico un operatore differenziale $\mathcal D$ che coincide con $\mathcal E_U$ su $C^\infty(U)$ e con $\mathcal F_U$ su $\chi(U)$ per ogni $U$ aperto di $M$.

\end{theorem}

\begin{proof}
	Innanzitutto definiamo $\mathcal D$ su $\chi^*(U)$ come 
	\begin{equation*}
	(\mathcal D\alpha) (X) \coloneqq \mathcal E_U(\alpha(X)) - \alpha( \mathcal F_UX)
	\end{equation*}
	per ogni $X\in\chi(U)$ e per ogni $\alpha \in \chi^*(U)$, ricordandoci che vogliamo che valga $(\mathcal D\alpha)\cdot X = \mathcal D(\alpha\cdot X) - \alpha\cdot(\mathcal D X)$.
	
	Verifichiamo che $\mathcal D\alpha$ sia $C^\infty(M)$-lineare su $\chi(U)$ e che quindi sia effettivamente un elemento di $\chi^*(U)$.
	Abbiamo infatti che
	\begin{align*}
		(\mathcal D\alpha)(fX) &= \mathcal E_U(\alpha(fX)) - \alpha(\mathcal F_U(fX))= \mathcal E_U(f\alpha (X)) - \alpha [(\mathcal E_Uf)\otimes X + f \otimes \mathcal F_UX] = \\
		&= (\mathcal E_U f)\otimes \alpha (X) + f \otimes \mathcal E_U (\alpha (X)) - \alpha [(\mathcal E_U f) \otimes X] - \alpha [f \otimes \mathcal F_U X] = \\
		&= f [\mathcal E_U (\alpha (X))- \alpha( \mathcal F_UX)] = f(\mathcal D\alpha) (X) \virgola
	\end{align*}
	dove abbiamo utilizzato le proprietà \ref{eod:LeibnitzPerE} e \ref{eod:LeibnitzPerF}.
	
	Abbiamo perciò definito $\mathcal D\alpha \in \chi^*(U)$ per ogni $\alpha\in\chi^*(U)$. Possiamo quindi estendere la definizione in $U$ ai tensori, utilizzando la proprietà \ref{od:Derivazione}, tramite 
	\begin{align*}
		(\mathcal D_U t) (\alpha_1,\ldots,\alpha_r,X_1,\ldots,X_s) \coloneqq\ & \mathcal E_U(t(\alpha_1,\ldots,\alpha_r,X_1,\ldots,X_s)) -\\
		&-\sum_{j=1}^rt(\alpha_1,\ldots, \mathcal D\alpha_j,\ldots,\alpha_r,X_1,\ldots,X_s) -\\
		&- \sum_{k=1}^s t(\alpha_1,\ldots,\alpha_r,X_1,\ldots,\mathcal F_UX_k,\ldots,X_s)\virgola
	\end{align*}
	per ogni $t\in\Tau_s^r(M)$, $\alpha_1,\ldots,\alpha_r\in\chi^*(M)$ e $X_1,\ldots,X_s\in\chi(M)$.
	Usando ancora le proprietà \ref{eod:LeibnitzPerE} e \ref{eod:LeibnitzPerF} si vede come prima che $\mathcal D_U$ su $\Tau_s^r(U)$ è $C^\infty(M)$-multilineare.

Per concludere, se $V\subseteq U$ aperto, per le proprietà \ref{eod:LocalitaPerE} e \ref{eod:LocalitaPerF} abbiamo $\mathcal D_V(t\restrict U) = (\mathcal D_U t)\restrict V$, per ogni $t\in\Tau(U)$.
Quindi definiamo $\mathcal D$ su $\Tau_s^r(M)$ come $(\mathcal D t) (m) = (\mathcal D_U t) (m)$ per un qualunque $U$ aperto che contiene $m$.
Per l'unicità di $\mathcal D_U$, si ha anche l'unicità di $\mathcal D$ e inoltre vale anche la proprietà \ref{od:Restrizioni} richiesta.
\end{proof}

\begin{corollary} \label{cor:OperatoreDifferenzialeProdottoTensoreEDelta}
	Dato $\mathcal D$ operatore differenziale sull'algebra dei tensori, valgono
	\begin{enumerate}
		\item $\mathcal D(t_1\otimes t_2) = (\mathcal D t_1) \otimes t_2 + t_1\otimes (\mathcal D t_2)$, con $t_1 \in \Tau^{r_1}_{s_1}(M)$ e $t_2 \in \Tau^{r_2}_{s_2}(M)$; \label{odptd:ProdottoTensore}
		\item $\mathcal D\delta = 0$, dove $\delta$ è la delta di Kronecker. \label{odptd:Delta}
	\end{enumerate}
\end{corollary}
\begin{proof}
	Il punto \ref{odptd:ProdottoTensore} segue dalla proprietà \ref{od:Derivazione} richiesta all'operatore differenziale. 
	
	Verifichiamo quindi il punto \ref{odptd:Delta}.	
	Siano $X\in\chi(U)$ e $\alpha\in\chi^*(U)$; allora, ancora per le proprietà di operatore differenziale, abbiamo
	\begin{align*}
	(\mathcal D\delta)(\alpha, X) = \mathcal D(\delta(\alpha,X)) - \delta(\mathcal D\alpha,X) - \delta(\alpha,\mathcal D X) = \mathcal D (\alpha X) - (\mathcal D\alpha)X - \alpha(\mathcal DX) = 0\virgola
	\end{align*}
	da cui $\mathcal D\delta = 0$ per l'arbitrarietà di $X$ ed $\alpha$.
\end{proof}

\section{Estensione della derivata di Lie} %TODO: aggiungere ``ai campi tensoriali''

Trattiamo ora il caso particolare della derivata di Lie.
Dati $X\in\chi(M)$ e $U\subseteq M$ aperto, definiamo $\mathcal E_U$ ed $\mathcal F_U$ come $\Lie_X\restrict U$.
Allora, per Leibniz, le ipotesi del \cref{thm:EsistenzaOperatoriDifferenziali} sono soddisfatte e possiamo dare quindi la seguente definizione.

\begin{definition} \index{derivata!di Lie!su tensori}
	Se $X\in\chi(M)$, definiamo $\Lie_X$ come l'unico operatore differenziale su $\Tau(M)$ (ancora detto \emph{derivata di Lie}), tale che $\Lie_X$ coincide con le derivate di Lie rispetto a $X$ su $C^\infty(M)$ e $\chi(M)$.
\end{definition}

\begin{proposition}
	Sia $\varphi: M \to N$ un diffeomorfismo e sia $X\in\chi(M)$. Allora $\Lie_X$ è naturale rispetto al push-forward, cioè $\Lie_{\varphi_*X} (\varphi_*t) = \varphi_*(\Lie_Xt)$ per ogni $t\in\Tau_s^r(M)$.
\end{proposition}
\begin{proof}
	Dato $U\subseteq M$ aperto, definiamo $\mathcal D: \Tau_s^r(U) \to \Tau_s^r(U)$ tale che $\mathcal Dt = \varphi^*\Lie_{\varphi_*X\restrict U}(\varphi_*t)$.
	La derivata di Lie (su funzioni e campi vettoriali) è naturale rispetto ai push-forward per il \cref{lemma:DerivataLieNaturalePushForwardFunzioni} e la \cref{prop:DerivataLieNaturalePushForwardCampiVett}, quindi $\mathcal D$ definita come sopra è una derivazione su funzioni $C^\infty(U)$ e su $\chi(U)$ che coincide con $\Lie_Xt$ su di essi.
	Mostriamo che $\mathcal D$ è un operatore differenziale, verificandone le due proprietà:
	\begin{description}
	 \item [\ref{od:Derivazione}]
	Dati $X_1,\ldots,X_s\in\chi(U)$ e $\alpha_1,\ldots,\alpha_r\in\chi^*(U)$, ricordiamo che
	\begin{equation*}
	\varphi_*(t(\alpha_1,\ldots,\alpha_r,X_1,\ldots,X_s) )= (\varphi_*t)(\varphi_*\alpha_1,\ldots,\varphi_*\alpha_r,\varphi_*X_1,\ldots,\varphi_*X_s)\virgola
	\end{equation*}
	perciò abbiamo
	\begin{align*}
		\mathcal D(t(\alpha_1,\ldots,\alpha_r&,X_1,\ldots,X_s)) = \varphi^*\Lie_{\varphi_*X}(\varphi_*(t(\alpha_1,\ldots,\alpha_r,X_1,\ldots,X_s))) =\\
		=& \varphi^*\Lie_{\varphi_*X}((\varphi_*t)(\varphi_*\alpha_1,\ldots,\varphi_*\alpha_r,\varphi_*X_1,\ldots,\varphi_*X_s))) = \\
		=& \varphi^*[(\Lie_{\varphi_*X}\varphi_*t)(\varphi_*\alpha_1,\ldots,\varphi_*\alpha_r,\varphi_*X_1,\ldots,\varphi_*X_s) +\\
		&+\sum_{j=1}^r{(\varphi_*t)(\varphi_*\alpha_1,\ldots,\Lie_{\varphi_*X}\varphi_*\alpha_j,\ldots,\varphi_*\alpha_r,\varphi_*X_1,\ldots,\varphi_*X_s)} +\\
		&+\sum_{k=1}^s{(\varphi_*t)(\varphi_*\alpha_1,\ldots,\varphi_*\alpha_r,\varphi_*X_1,\ldots,\Lie_{\varphi_*X}\varphi_*X_k,\ldots,\varphi_*X_s)}] \punto
	\end{align*}
	Sapendo che $\varphi^*=(\varphi^{-1})_*$, dall'ultima espressione ottengo
	\begin{align*}
		\mathcal D(t(\alpha_1,\ldots,\alpha_r,X_1,\ldots,X_s)) =& (\mathcal D t)(\alpha_1,\ldots,\alpha_r,X_1,\ldots,X_s)+\\
		&+\sum_{j=1}^r{t(\alpha_1,\ldots,\mathcal D \alpha_j, \ldots \alpha_r,X_1,\ldots,X_s)} +\\
		&+\sum_{k=1}^s{t(\alpha_1,\ldots, \alpha_r,X_1,\ldots,\mathcal D X_k, \ldots,X_s)} \virgola
	\end{align*}
	che è proprio la proprietà richiesta.
	
	\item [\ref{od:Restrizioni}]
		Sia $t\in\Tau_s^r(M)$, allora
		\begin{equation*}
			(\mathcal Dt)\restrict U = [(\varphi_*)^{-1} \Lie_{\varphi_*X}\varphi_*t]\restrict U 
			= (\varphi_*)^{-1}[\Lie_{\varphi_*X}\varphi_*t]\restrict U 
			= (\varphi_*)^{-1} \Lie_{\varphi_*X\restrict U}\varphi_*t \restrict U = \mathcal D(t\restrict U) \punto
		\end{equation*}
	\end{description}
	
	Allora, per l'unicità data dal \cref{thm:EsistenzaOperatoriDifferenziali}, abbiamo che gli operatori differenziali $\varphi^*\Lie_{\varphi_*X\restrict U}(\varphi_*t)$ e $\Lie_Xt$ coincidono e otteniamo perciò quanto cercato.
\end{proof}

Vediamo ora le formule in coordinate; cioè, dato $t\in\Tau_s^r(M)$ e $X\in\chi(M)$, vogliamo le componenti di $\Lie_Xt$.
Sia $\varphi: U \to \R^n$ una carta con coordinate $\varphi=(\seqa xn,)$, rispetto alle quali $X= X^i\DerParz{}{x^i}$ e $t = t^{\seqb ir,}_{\seqb js,} \DerParz{}{x^{i_1}} \otimes \ldots \otimes \DerParz{}{x^{i_r}} \otimes \de x^{j_1} \otimes \ldots \otimes \de x^{j_s}$.

Ricordiamo innanzitutto che per $f \in C^\infty(M)$ vale
\begin{equation*}
\Lie_X f = X^i \DerParz{f}{x^i} \virgola
\end{equation*}
mentre per $Y\in \chi(M)$ abbiamo
\begin{equation*}
\Lie_XY = \left( X^j\DerParz{Y^i}{x^j} - Y^j \DerParz{X^i}{x^j} \right) \DerParz{}{x^i}\punto
\end{equation*}
Calcoliamo quindi la formula in coordinate di $\Lie_X\alpha$ con $\alpha = \alpha_i \de x^i \in \chi^*(M)$, per poi ricavare la formula per qualsiasi tensore sfruttando la definizione. Sfruttando la definizione di $\Lie_X\alpha$ e quanto detto fino ad ora, abbiamo
\begin{align*}
	(\Lie_X\alpha)_i &= \Lie_X\alpha \left(\DerParz{}{x^i}\right) = \Lie_X \left(\alpha\left(\DerParz{}{x^i}\right)\right) - \alpha\left(\Lie_X\DerParz{}{x^i}\right) = \Lie_X(\alpha_i) - \alpha\left( - \DerParz{X^j}{x^i} \DerParz{}{x^j} \right) =  \\
	& = X^j \DerParz{\alpha_i}{x^j} + \DerParz{X^j}{x^i} \alpha\left( \DerParz{}{x^j} \right) = X^j \DerParz{\alpha_i}{x^j} + \alpha_j\DerParz{X^j}{x^i} \virgola
\end{align*}
perciò
\begin{equation*}
	\Lie_X\alpha = \left( X^j \DerParz{\alpha_i}{x^j} + \alpha_j\DerParz{X^j}{x^i} \right) \de x^i \punto
\end{equation*}

Passiamo quindi al calcolo in coordinate di $\Lie_Xt$. Vale che
\begin{align*}
	(\Lie_Xt)^{\seqb ir,}_{\seqb js,} =& \Lie_Xt \left(\de x^{i_1},\ldots, \de x^{i_r}, \DerParz{}{x^{j_1}}, \ldots ,\DerParz{}{x^{j_s}}\right) = \\
	=& \Lie_X \left(t\left(\de x^{i_1},\ldots, \de x^{i_r}, \DerParz{}{x^{j_1}}, \ldots ,\DerParz{}{x^{j_s}} \right) \right) -  \\
	&- \sum_{h=1}^r t\left(\de x^{i_1},\ldots, \Lie_X \de x^{i_h},\ldots, \de x^{i_r}, \DerParz{}{x^{j_1}}, \ldots ,\DerParz{}{x^{j_s}} \right) -\\
	&- \sum_{h=1}^s t\left(\de x^{i_1},\ldots, \de x^{i_r}, \DerParz{}{x^{j_1}}, \ldots, \Lie_X\DerParz{}{x^{j_h}},\ldots ,\DerParz{}{x^{j_s}} \right) = \\
	= & \Lie_X t^{\seqb ir,}_{\seqb js,}- \sum_{h=1}^r t\left(\de x^{i_1},\ldots, \DerParz{X^{i_h}}{x^l} \de x^l,\ldots, \de x^{i_r}, \DerParz{}{x^{j_1}}, \ldots ,\DerParz{}{x^{j_s}} \right) -\\
	& - \sum_{h=1}^s t\left(\de x^{i_1},\ldots, \de x^{i_r}, \DerParz{}{x^{j_1}}, \ldots, - \DerParz{X^m}{x^{j_h}} \DerParz{}{x^m},\ldots ,\DerParz{}{x^{j_s}} \right) = \\
	= & X^k \DerParz{t^{\seqb ir,}_{\seqb js,}}{x^k} - \sum_{h=1}^r\DerParz{X^{i_h}}{x^l} t^{i_1\ldots i_{h-1} l i_{h+1},\ldots, i_r}_{\seqb js,} + \sum_{h=1}^s \DerParz{X^m}{x^{j_h}} t^{\seqb ir,} _{j_1 \ldots j_{h-1} m j_{h+1} \ldots j_s} \virgola
\end{align*}
che è la formula cercata.


%%%%%%%%%%%%%%%%%%%% parte sostituita dalla cosa sopra

% Sia $\varphi: U \to \R^n$ una carta e siano $X'$ e $t'$ le coordinate di $\varphi_*X$ e $\varphi_*t$.
% Se $\Diff$ è il differenziale in $\R^n$ (e $Y\in\chi(M)$), ricordiamo che per $f\in C^\infty(M)$ e $Y\in \chi(M)$ abbiamo
% $(\Lie_Xf)' (x) = \Diff f'(x) X'(x)$ e $(\Lie_XY)'(x) = \Diff Y'(x)X'(x) - \Diff X'(x)Y'(x)$, cioè quindi
% $\Lie_X f = X^i \DerParz{f}{x^i}$ e $(\Lie_XY)^i = X^j\DerParz{Y^i}{x^j} - Y^j\DerParz{X^i}{x^j}$.
% 
% Per l'ultima proposizione abbiamo che $\varphi_*(\Lie_X\alpha) = \Lie_{\varphi_*X}(\varphi_*\alpha) = \Lie_{X'}\alpha'$ per $\alpha\in\chi^*(M)$, dove $\alpha' : \varphi(U)\to\R^n$.
% 
% Dato $v\in\R^n$ fissato allora $\Lie_{X'}(\alpha'v) = (\Lie_{X'}\alpha')v + \alpha'\Lie_{X'}v$,
% 
% $\Diff (\alpha'v)X' = (\Lie_{X'} \alpha') v - \alpha'(\Diff X' \cdot v) \Rightarrow (\Lie_{X'}\alpha') v = (\Diff \alpha' \cdot X') v + \alpha' (\Diff X' \cdot v)$ 
% 
% Allora $(\Lie_X\alpha)_iv^i = \DerParz{\alpha^i}{x^j}X^jv^i + \alpha_j\DerParz{X^i}{x^j}v^i$
% quindi $(\Lie_X\alpha)_i = X^j\DerParz{\alpha_i}{x^j} + \alpha_j\DerParz{X^j}{x^i}$
% 
% 
% 
% %%%%%%%%% aggiunta parte del 18 febbraio
% 
% Formule in coordinate per le derivate di Lie di tensori.
% 
% $\varphi:U\subseteq M \to \R^n$ carta locale e $X'$, $t'$ coordinate di $\varphi_*X$ e $\varphi_*t$.
% 
% $Y\in\chi(M)$ e $\Diff$ differenziale in $\R^n$.
% 
% $(\Lie_Xf)'(x) = \Diff f'\cdot X'(x)$, $\Lie_X f = X'\DerParz{}{x'}$, per $f\in C^\infty(U)$.
% 
% $(\Lie_XY)' = \Diff Y'(x) X'(x) -\Diff X'(x) Y'(x)$
% 
% Da Leibnitz, se $\alpha\in\chi^*(M)$, $v$ vettore costante in $\R^n$:
% \begin{equation*}
% 	(\Lie_{X'}\alpha')v = (\Diff \alpha'\cdot X') v + \alpha' (\Diff X' v)
% \end{equation*}
% perciò
% \begin{equation*}
% 	(\Lie_X\alpha)_i = X^j \DerParz{\alpha_i}{x^j} + \alpha_j \DerParz{X^j}{x^i}
% \end{equation*}
% %Fino a qui fatto l'altra volta
% 
% Dato $t\in\Tau_s^r(M)$, allora $t':\varphi(U)\to T_s^r(\R^n)$.
% Vogliamo trovare le componenti di $\Lie_Xt$.
% 
% Siano $\alpha^1,\ldots,\alpha^r$ elementi costanti del duale di $\R^n$ e $\seq{v}{s}{,}$ vettori costanti di $\R^n$. Allora
% \begin{align*}
% 	\Lie_{X'}[t'(\alpha^1,\ldots,\alpha^r,\seq{v}{s}{,})] =& (\Lie_{X'}t')(\alpha^1,\ldots,\alpha^r,\seq{v}{s}{,}) +\\
% 	&+\sum_{i=1}^r t'(\alpha^1,\ldots, \Lie_{X'}\alpha^i,\ldots, \alpha^r, \seq{v}{s}{,}) +\\
% 	&+\sum_{j=1}^s t'(\alpha^1,\ldots,\alpha^r,v_1,\ldots,\Lie_{X'}v_j,\ldots,v_s) \punto
% \end{align*}
% 
% Dalle formule locali (le entrate sono costanti)
% \begin{align*}
% 	(\Diff t'X') (\seqa{\alpha}{r}{,}, \seqb{v}{s}{,}) =& (\Lie_{X'}t')(\seqa{\alpha}{r}{,}, \seqb{v}{s}{,})+\\ 
% 	&+ \sum_{i=1}^r t'(\alpha^1,\ldots, \alpha^i\cdot\Diff X',\ldots, \alpha^r, \seq{v}{s}{,}) +\\
% 	&+\sum_{j=1}^s t'(\alpha^1,\ldots,\alpha^r,v_1,\ldots,-\Diff X'\cdot v_j,\ldots,v_s)
% \end{align*}
% \begin{equation*}
% 	(\Lie_Xt)_{\seqb{j}{s}{}}^{\seqb{i}{r}{}} = X^k \DerParz{}{x^k} t_{\seqb{j}{s}{}}^{\seqb{i}{r}{}} - \DerParz{X^{i_1}}{x^l} t_{\seqb{j}{s}{}}^{li_2\ldots i_r} + \DerParz{X^m}{x^{j_1}} t^{\seqb{i}{r}{}}_{mj_2\ldots j_s} \virgola
% \end{equation*}
% dove sommo su tutti gli indici alti per la prima e bassi per la seconda.


%%%%%%%%%%%%%%%%%%%%%%%%%%% fine della parte sostituita



\begin{example}
	\begin{enumerate} %TODO: da controllare e magari migliorare
		\item Calcoliamo in $\R^2$ $\Lie_Xt$, con $t=x\DerParz{}{y}\otimes \de x \otimes \de y + y\DerParz{}{y} \otimes \de y \otimes \de y$ e $X = \DerParz{}{x} + x \DerParz{}{y}$.
		Per linearità in $X$, $\Lie_Xt = \Lie_{\DerParz{}{x}}t + \Lie_{x\DerParz{}{y}}t$ e abbiamo che
		\begin{equation*}
			\Lie_{\DerParz{}{x}} t = \Lie_{\DerParz{}{x}} \left[x \DerParz{}{y}\otimes \de x \otimes \de y\right] + \Lie_{\DerParz{}{x}} \left[y \DerParz{}{y}\otimes \de y \otimes \de y\right]
		\end{equation*}
		Chiamiamo ora $\tilde t = x\DerParz{}{y}\otimes \de x \otimes \de y$ e $\hat t = y\DerParz{}{y} \otimes \de y \otimes \de y$. Visto che $\tilde t_{12}^2 = x$ e le altre componenti sono 0, abbiamo
		\begin{equation*}
			\left(\Lie_{\DerParz{}{x}}\tilde t\right)_{12}^2 = \DerParz{}{x} \tilde t^2_{12}=1
		\end{equation*}
		e le altre componenti sono nulle. 
		Dato che $\hat t_{22}^2$ è l'unica componente non nulla di $\hat t$, abbiamo che facendo il conto
		\begin{equation*}
			\left(\Lie_{\DerParz{}{x}}\hat t\right)_{22}^2 = \DerParz{}{x}\hat t^2_{22} =  0
		\end{equation*}
		e per le altri componenti è 0. Di conseguenza
		\begin{equation*}
			\Lie_{\DerParz{}{x}} t = \DerParz{}{y} \otimes \de x \otimes \de y \punto
		\end{equation*}
		Analogamente, sfruttando la formula in coordinate come appena fatto, abbiamo
		\begin{align*}
			\Lie_{x\DerParz{}{y}} t &= \Lie_{x\DerParz{}{y}}\left[x \DerParz{}{y}\otimes \de x \otimes \de y\right] + \Lie_{x\DerParz{}{y}} \left[y \DerParz{}{y}\otimes \de y \otimes \de y\right] =\\
			&=x \DerParz{}{y} \otimes \de x \otimes \de x + x \DerParz{}{y} \otimes \de y \otimes \de y + y \DerParz{}{y} \otimes \de x \otimes \de y + y \DerParz{}{y} \otimes \de y \otimes \de x \punto
		\end{align*}
		E unendo le due espressioni abbiamo quindi
		\begin{equation*}
			\Lie_X t = x \DerParz{}{y} \otimes \de x \otimes \de x + x \DerParz{}{y} \otimes \de y \otimes \de y + (y+1) \DerParz{}{y} \otimes \de x \otimes \de y + y \DerParz{}{y} \otimes \de y \otimes \de x \punto
		\end{equation*}

% 		\begin{align*}
% 			\Lie_{x\DerParz{}{y}} \tilde t &= \left(\Lie_{x\DerParz{}{y}} \tilde t\right)_{j_1j_2}^i \DerParz{}{x^i}\otimes\de x^{j_1}\otimes \de x^{j_2} - \tilde t_{2j_2}^{i_1}\DerParz{}{x^{i_1}} \otimes \de y\otimes \de x^{j_2}
% 			+ \tilde t_{j_12}^{i_1} \DerParz{}{x^{i_1}} \otimes x^{j_1}\otimes \de y\\
% 			&= \DerParz{}{y}\otimes \de x \otimes \de y \punto
% 		\end{align*}

		
		\item (Campi di Killing) Sia $M$ una varietà con metrica $g$. Una campo vettoriale $X\in\chi(M)$ è detto \emph{di Killing} se $\Lie_Xg \equiv 0$. Il significato geometrico è che una metrica dà una lunghezza della curva e di conseguenza una distanza sulla varietà (prendendo l'inf delle lunghezze delle curve che collegano due punti) e un campo di Killing genera un flusso di isometrie sulla varietà. %TODO: scrivere sensatamente
		
		Vediamo che equazioni deve rispettare $X$. Siano $X = X^i\DerParz{}{x^i}$ e $g = g_{ij}\de x^i \otimes \de x^j$ in coordinate (dove $g_{ij} = g_{ji}$).
		Allora
		\begin{align*}
		\Lie_Xg &= \Lie_x (g_{ij}\de x^i \otimes \de x^j) =\\
		&=(\Lie_Xg_{ij})\otimes \de x^i\otimes x^j + g_{ij} \otimes (\Lie_X\de x^i) \otimes \de x^j + g_{ij}\otimes \de x^i \otimes (\Lie_X\de x^j) =\\
		&= X^k \DerParz{g_{ij}}{x^k}\otimes \de x^i \otimes \de x^j + g_{ij}\DerParz{X^i}{x^k} \otimes \de x^k \otimes \de x^j + g_{ij} \DerParz{X^j}{x^k} \otimes \de x^i \otimes \de x^k =\\
		&= \left[X^k \DerParz{g_{ij}}{x^k} + g_{kj} \DerParz{X^k}{x^i} + g_{ik} \DerParz{X^k}{x^j} \right] \de x^i \otimes \de x^j \virgola
		\end{align*}
		dove nell'ultima uguaglianza abbiamo semplicemente cambiato nome agli indici ripetuti. Le equazioni tra parentesi quadre sono dette \emph{equazioni di Killing}. Si può vedere che $\Lie_Xg=0$ se e solo se $\Lie_X$ commuta con le operazioni $\sharp$ e $\flat$ (di alzamento e abbassamento di indici).
	\end{enumerate}
\end{example}

\begin{exercise}
	Sia $g$ una metrica su $M$ e sia $g^{\sharp}$ il tensore ottenuto alzando un indice. Sia $X$ un campo vettoriale. Calcolare in coordinate $(\Lie_X g^\sharp)^\flat - \Lie_X g$.
	
	[Sarà legato alle equazioni di Killing.]
\end{exercise}

%TODO: spostare e dare un senso a quest'ultima parte
Vediamo ora la seguente osservazione sull'approccio dinamico.
\begin{remark}
	Sia $\varphi_s$ il flusso generato da $X\in\chi(M)$. Allora, se $t\in\Tau_s^r(M)$, vale $\Lie_Xt = \frac{\de}{\de s} (\varphi_s^* t)$ \footnote{Ricordiamo che $\varphi_s$ è un diffeomorfismo e quindi è legale fare pull-back di tensori di ogni tipo.}.
\end{remark}