\chapter{17 marzo 2016}

\begin{remark}
	Data una scelta di orientazione $[\mu]$ su una $m$-varietà $M$, diciamo che una base $\seqb vm,$ di $T_xM$ è orientata positivamente se $\mu(\seqb vm,)\ge 0$. %TODO: rivedere
\end{remark}


\section{Varietà con bordo}

\begin{definition}
	Siano $E_1,E_2\subseteq \R^n$ semispazi chiusi. Siano $U,V$ aperti di $E_1,E_2$. Una mappa $f:U\to V$ è regolare se per ogni $x\in U$ esiste $U_1$ intorno di $x$ in $\R^n$ e $V_1$ intorno di $f(x)$ in $\R^n$ ed esiste $f_1:U_1\to V_1$ tale che $f\restrict{U\cap U_1} = f_1\restrict{U\cap U_1}$.
	In questo caso definiamo il differenziale di $f$ in $x$ come $\Diff f(x) \coloneqq \Diff f_1(x)$.
	
	La mappa $f$ è detta diffeomorfismo se esiste $g:V\to U$ regolare che è l'inversa di $f$.
\end{definition}

\begin{remark}
	\begin{enumerate}
		\item $\Diff f$ non dipende dalla scelta dell'estensione $f_1$. Questo è ovvio se $x$ non sta sul bordo, mentre al bordo si ragiona per approssimazione.
		
		\item Siano $U$ aperto in $E_1$, $V$ aperto in $E_2$, $f:U \to V$ diffeomorfismo. Allora le restrizioni $\Int f:\Int U \to \Int V$ e $\partial f: \partial U \to \partial V$ sono diffeomorfismi.
	\end{enumerate}
\end{remark}

\begin{definition}
	Una \emph{varietà con bordo} è un insieme $M$ munito di un atlante di carte di bordo, cioè coppie $U,\varphi)$ con $u\subseteq M$ aperto, $\varphi(U)\subseteq E$, con $E$ semispazio chiuso di $\R^n$. Si richiede che valgano la proprietà di ricoprimento e la regolarità delle mappe di transizione (nel senso della definizione sopra). %TODO: aggiungere cref
\end{definition}

\begin{definition}
	Data $M$ varietà con bordo, definiamo $\Int M = \bigcup_{i} \varphi_i^{-1}(\Int (\varphi_i(U_i)))$ e $\partial M = \bigcup_i \varphi_i^{-1}(\partial (\varphi_i(U_i)))$.
\end{definition}

Estendiamo ora a varietà con bordo alcune definizioni date per varietà non con bordo, per definire infine l'orientazione.

Pensiamo al tangente come un sottospazio $n$-dimensionale anche per i punti sul bordo.
Una forma di volume è una $n$-forma che non si annulla in nessun punto. Diciamo che $(U,\varphi)$ carta di bordo è orientata positivamente se $T_u\varphi$ preserva l'orientazione per ogni $u\in U$.

\begin{remark}
	È stato conveniente poter scegliere semispazi arbitrari e non solo $\R_n^+ = \{x_n\ge 0\}$. Per esempio è comodo per orientare $M = \cc 01$.
\end{remark}

Vediamo ora come possiamo orientare il bordo.
\begin{definition}
	Sia $M$ una $m$-varietà orientata con bordo. Siano $x\in \partial M$ e $\varphi:U \to E$ (con $x\in U$ ed $E$ semispazio chiuso) una carta orientata positivamente. Una base $\seqb v{m-1},$ di $T_x(\partial M)$ è detta orientata positivamente se $\{ (T_x\varphi)^{-1} (n), \seqb v{m-1}, \}$ è orientata positivamente in $M$ per ogni vettore $n$ (in $\R^m$) che punta all'esterno di $E$ in $\varphi(x)$.
\end{definition}

\begin{remark}
	In $\R^m$ ogni semispazio chiuso si scrive come $E = \{x \suchthat \Lambda(x) \ge 0\}$ per qualche funzionale lineare $\Lambda$ su $\R^m$. Una scelta canonica di $n$ è $n = -\grad \Lambda$.
\end{remark}



\section{Teorema di Stokes}

\begin{theorem} [Stokes]
	Sia $M$ una varietà $n$-dimensionale paracompatta e orientata. Sia $\alpha \in \Omega^{n-1}(M)$ una forma a supporto compatto. Sia $i:\partial M \to M$ l'inclusione. Allora
	\begin{equation*}
		\int_{\partial M} i^*\alpha = \int_M \de \alpha \punto
	\end{equation*}
	Il membro di sinistra si denota anche con $\int_{\partial M} \alpha$ e, se $\partial M = \emptyset$, intendiamo che tale integrale sia 0.
\end{theorem}
\begin{proof}
	Per linearità in $\alpha$ e per l'esistenza di partizioni dell'unità possiamo supporre che $\alpha$ abbia supporto nel dominio $U$ di un'unica carta.
	In coordinate
	\begin{equation*}
		\alpha = \sum_{i=1}^n (-1)^{i-1} \alpha^i \de x^1 \wedge \ldots \wedge \hat{\de x^i} \wedge \ldots \wedge \de x^n \punto
	\end{equation*}
	Allora
	\begin{equation*}
		\de \alpha = \sum_{i=1}^n \DerParz{\alpha^i}{x^i} \seqa {\de x}{n}{\wedge} \virgola
	\end{equation*}
	e quindi
	\begin{equation*}
		\int_U \de \alpha = \sum_{i=1}^n \int_{\varphi(U)} \DerParz{\alpha^i}{x^i} \seqa {\de x}{n}{} \virgola
	\end{equation*}
	che è ben definito se $M$ è orientata.
	
	Distinguiamo ora in casi
	\begin{enumerate}
		\item ($\partial U = \emptyset$) In questo caso l'$i$-esimo termine della sommatoria è
		\begin{equation*}
			\int_{\R^n} \DerParz{\alpha^i}{x^i} \seqa {\de x}{n}{} = \int_{\R^{n-1}} \left(\int_\R \DerParz{\alpha^i}{x^i} \de x^i \right) \de x^1  \ldots \hat{\de x^i} \ldots \de x^n = 0 \virgola
		\end{equation*}
		perché $\alpha^i$ ha supporto compatto.
		
		\item ($\partial U \not = \emptyset$) Possiamo supporre $E=\R^n_+$, allora l'integrale diventa
		\begin{equation*}
			\sum_{i=1}^n \int_{\R^n_+} \DerParz{\alpha^i}{x^i} \seqa {\de x}{n}{} \punto
		\end{equation*}
		Se $i<n$ ragioniamo come prima, altrimenti se $i=n$ l'integrale diventa
		\begin{equation*}
			- \int_{\R^{n-1}} \alpha^n \seqa {\de x}{n-1}{}
		\end{equation*}
		Ma
		\begin{equation*}
			\int_{\partial U}\alpha = \int_{\partial \R^n_+} \alpha = \int_{\partial \R^n_+} (-1)^{n-1} \alpha^n(\seqa x{n-1}, ,0) \seqa {\de x}{n-1}{} \punto
		\end{equation*}
		L'orientazione di $\partial M$ non è quella standard, la normale esterna è $-e_n = -(0,\ldots,0,1)$. L'orientazione del bordo ha il segno di $\{-e_n, \seqb e{n-1},\}$ che è  $(-1)^n$, quindi
		\begin{equation*}
			\int_{\partial U}\alpha = (-1)^{2n-1} \int_{\R^{n-1}} \alpha^n \seqa{\de x}{n-1}{} \punto
		\end{equation*}

		%TODO: ho messo i wedge dove non andavano

	\end{enumerate}


\end{proof}


\begin{definition}
	Sia $\mu$ una forma di volume su $M$ orientata. Sia $X\in \chi(M)$. La funzione $\div X \in C^\infty(M)$ è definita da $\Lie_X\mu = (\div X)\mu$.
\end{definition}

La divergenza si interpreta come espansione o contrazione del volume.

\begin{theorem} [Gauss]
	Sia $M$ orientata, paracompatta e con bordo. Sia $X\in\chi(M)$ a supporto compatto e sia $\mu$ forma di volume su $M$. Allora
	\begin{equation*}
		\int_M (\div X) \mu = \int_{\partial M} i_X \mu \punto
	\end{equation*}
\end{theorem}
\begin{proof}
	Abbiamo che
	\begin{equation*}
		(\div X) \mu = \Lie_X \mu = \de i_X \mu + i_X \de \mu = \de i_X \mu
	\end{equation*}
	e di conseguenza la conclusione segue da Stokes.
\end{proof}


Se $M$ ammette una metrica $g$ esiste un'unica normale unitaria (rispetto a $g$) che punta esternamente a $\partial M$ e che chiamiamo $n_{\partial M}$. Inoltre $g$ determina univocamente forme di volume $\mu_M,\mu_{\partial M}$ su $M,\partial M$.

\begin{corollary}
	Se $X\in\chi(M)$, allora
	\begin{equation*}
		\int_M (\div X) \mu_M = \int_{\partial M} g(X,n_{\partial M}) \mu_{\partial M} \punto
	\end{equation*}
\end{corollary}
\begin{proof}
	Localmente, scegliamo una carta $(U,\varphi)$ tale che $n_{\partial M} = - \DerParz{}{x^m}$. Se $\seqb v{n-1},$ è una base orientata di $T_x\partial M$, vale $\mu_{\partial M}(x) (\seqb v{m-1},) = \mu_m(x) (-\DerParz{}{x^m},\seqb v{m-1},)$
	
	\begin{equation*}
		(i_X\mu_M)(x)(\seqb v{m-1},) = \mu_M (X^i(x)v_i + X^n\DerParz{}{x^n},\seqb v{n-1},) = X^n(x) \mu_{\partial M}(x) (\seqb v{n-1},) \punto
	\end{equation*}
	Inoltre $X^n = -g(X,n_{\partial M})$, quindi basta applicare il teorema di Gauss. %TODO: aggiungere cref

\end{proof}

%TODO: controllare che la normale n e la dimensione n di M non si chiamino uguale

\begin{exercise}
	\begin{enumerate}
		\item $M$ compatta, orientabile e senza bordo, $\mu$ forma di volume e $X\in\chi(M)$ con $\div X=0$. Mostrare che per ogni $f,g \in C^\infty(M)$ vale
		\begin{equation*}
			\int_M g(Xf)\mu = -\int_M f(Xg) \mu \punto
		\end{equation*}

		\item $M$ orientabile, $(n+1)$-dimensionale, $f:\partial M \to N^{n+1}$ mappa regolare, $\omega \in \Omega^n(N)$ con $\de \omega =0$. Mostrare che se $f$ si estende ad $M$, allora $\int_{\partial M} f^*\omega =0$.
	\end{enumerate}
\end{exercise}


\section{Formula di coarea}

Sia $(M,g)$ varietà orientabile $(m+1)$-dimensionale. Sia $f:M \to \R$ una funzione regolare senza punti critici. Allora $M_t = f^{-1} (t)$ è una sottovarietà regolare $m$-dimensionale orientabile, che supponiamo compatta per ogni $t$.
Per ogni $t$ chiamiamo $\mu_t$ la forma di volume su $M_t$ indotta da $g$ e poniamo $n = \frac{\grad f}{\abs{\grad f}_g}$.

Sia $F:M \to \R$ a supporto compatto, allora
\begin{equation*}
	\int_M F \mu = \int_\R \left( \int_{M_t} \frac F{\abs{\grad f}_g} \mu_t \right) \de t \punto
\end{equation*}

\begin{proof}
	Fissiamo $t_0$ e $p\in M_{t_0}$. Visto che $\de f(p) \not = 0$, per il teorema della funzione implicita esistono $\seqa xm,$ tali che $(f, \seqa xm,)$ sono coordinate in un intorno di $p$.
	Quindi esiste $\rho$ funzione regolare positiva tale che $\mu = \rho \de f \wedge \seqa {\de x}m\wedge$. Allora
	\begin{equation*}
		\mu_{t_0} \restrict{M_{t_0}} = i_n \mu = \rho i_n (\de f \wedge \seqa {\de x}m\wedge)\restrict{M_{t_0}} = \rho \abs{\grad f}_g \seqa {\de x}m\wedge \restrict{M_{t_0}}
	\end{equation*}
	e integrando abbiamo la conclusione.
\end{proof}

$S^n \hookrightarrow \R^{n+1} = \{(t,\seqa xn,)\}$, siano $p_\pm = S^n\cap\{t=\pm 1\} \in S^n$ i due poli. Sia $\pi:S^n \to \R$ la proiezione su $t$, allora $\de \pi \not=0$ se $t\not=\pm 1$.

Abbiamo $\pi^{-1}(t) = S^{n-1}_{r(t)}$ con $r(t) = (1-t^2)^{\frac 12}$.
Definisco $\cos\theta = \abs{p-p'} = (1-t^2)^{\frac 12}$ con $p'$ la proiezione su $\{t\}\times \{0\}$. Allora $\abs{\de \pi(p)} = (1-t^2)^{\frac 12}$. %TODO: è giusto?

Per la formula di coarea
\begin{equation*}
	\sigma_n = 2\int_0^1 \frac 1{\abs{\grad \pi}} \int_{M_t} \mu_t = 2\sigma_{n-1} \int_0^1 (1-t^2)^{\frac{n-2}{2}} \de t = \sigma_{n-1} \int_0^1 (1-s)^{\frac{n-2}{2}} s^{-\frac 12} \de s \virgola
\end{equation*}
da cui 
\begin{equation*}
	\sigma_n = \frac{2\Gamma(\frac 12)^{n+1}}{\Gamma(\frac{n+1}2)} \punto
\end{equation*}

Altre applicazioni includono
\begin{enumerate}
	\item disuguaglianze ottimali di Sobolev, tipo $\int \abs u^{\frac n{n-1}} \le C_n \int \abs{\grad u}$;
	
	\item riarrangiamento sferico. Sia $u:\R^n\to \R$, sia $A_t = \{ u> t \}$, $u_*$ radiale descrescente tale che $\abs{A_t^*} = \abs{A_t}$ con $A_t^* = \{u^*>t\}$. Questa procedura serve a regolarizzare e conserva la normale $L^p$
	\begin{equation*}
		\int u^p = \int (u^*)^p \punto
	\end{equation*}
	Con la formula di coarea si può dimostrare
	\begin{equation*}
		\int \abs{\grad u^*}^2 \le \int \abs{\grad u}^2 \punto
	\end{equation*}
	
	%TODO: scrivere meglio
\end{enumerate}






































