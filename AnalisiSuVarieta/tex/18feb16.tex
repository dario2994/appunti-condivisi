\chapter{Forme differenziali} %TODO: sarà un capitolo

\begin{definition} \index{forma!differenziale}
	Le \emph{forme differenziali} sono campi tensoriali di tipo $\Tau_k^0(M)$ totalmente antisimmetrici.
\end{definition}

Hanno applicazioni per esempio in geometria perché possono dare informazioni sulla topologia delle varietà.

\section{Algebre esterne}

\begin{definition} \index{mappa!antisimmetrica} \index{forma!esterna}
Dati $V,W$ spazi vettoriali (finito dimensionali), le \emph{mappe antisimmetriche} in $T^0_k(V,W)$ sono quelle per cui vale $t(\seqb{e}{k}{,}) = \sgn(\sigma) t(e_{\sigma(1)},\ldots, e_{\sigma(k)})$ per ogni $\seqb{e}{k}{,} \in V$ e per ogni $\sigma\in S_k$.
Si indicano con $\Lambda^k(V,W)$ (per brevità indicheremo anche $\Lambda^k(V,\R) = \Lambda^k(V)$) e vengono dette \emph{$k$-forme esterne}.

\end{definition}

Dato $t\in T^0_k(V,W)$ è possibile ottenere un elemento di $\Lambda^k(V,W)$ tramite antisimmetrizzazione.
Se per esempio $k=2$, dato $t\in T^0_2(V,W)$, possiamo definire il suo antisimmetrizzato come $\underline{A}t(e_1,e_2) = \frac 12 [t(e_1,e_2) - t(e_2,e_1)]$.

Vediamo quindi la definizione generale.

\begin{definition} \index{mappa!alternante}
	La \emph{mappa alternante} $\antisimm:T^0_k(V,W) \to \Lambda^k(V,W)$ è definita da
	\begin{equation*}
		\antisimm t(\seqb{e}{k}{,}) \coloneqq \frac 1 {k!} \sum_{\sigma \in S_k} (\sgn \sigma) t(e_{\sigma(1)},\ldots, e_{\sigma(k)}) \virgola
	\end{equation*}
	per ogni $\seqb{e}{k}{,} \in V$.
\end{definition}

\begin{proposition}
	La mappa $\antisimm$ è lineare e suriettiva su $\Lambda^k(V,W)$. Inoltre è l'identità sulle $k$-forme, cioè $\antisimm\restrict{\Lambda^k(V,W)} = \id\restrict{\Lambda^k(V,W)}$, ed è idempotente, cioè $\antisimm \circ \antisimm = \antisimm$.
\end{proposition}
\begin{proof}
	Mostriamo che $\antisimm$ è l'identità sulle $k$-forme. Sia $t\in\Lambda^k(V,W)$, allora
	\begin{align*}
		\antisimm t(\seqb{e}{k}{,}) &= \frac 1 {k!} \sum_{\sigma \in S_k} \sgn (\sigma) t(e_{\sigma(1)},\ldots, e_{\sigma(k)}) =\\
		& = \frac 1 {k!} \sum_{\sigma \in S_k} (\sgn (\sigma))^2 t(\seqb{e}{k}{,}) = t (\seqb{e}{k}{,}) \virgola
	\end{align*}
	dove abbiamo usato che $t\in\Lambda^k(V,W)$ e che $\mathrm{card} S_k = k!$.
\end{proof}

\begin{remark}
	Visto che $\antisimm^2=\antisimm$, abbiamo $\norm\antisimm \le \norm\antisimm^2$ e quindi $\norm\antisimm \ge 1 $.
	D'altronde, le permutazioni preservano la norma, perciò $\norm{\antisimm t} \le \norm{t}$ e di conseguenza $\norm \antisimm \le 1$.
	Unendo le due disuguaglianze otteniamo proprio $\norm\antisimm = 1$.
\end{remark}

Parliamo di forme esterne e parleremo di algebre esterne perché si può fare un prodotto esterno, detto wedge, che definiamo di seguito.

\begin{definition} \index{prodotto!wedge}
	Se $\alpha \in T^0_k(V)$ e $\beta \in T^0_l(V)$. Il \emph{prodotto wedge} $\alpha \wedge \beta \in \Lambda^{k+l}(V)$ è definito da
	\begin{equation*}
		\alpha\wedge \beta \coloneqq \frac {(k+l)!}{k!\ l!}\ \antisimm(\alpha\otimes \beta)\punto
	\end{equation*}
\end{definition}

\begin{example}
	Se $\alpha,\beta\in\Lambda^1(V) \cong T^0_1(V)$, allora $(\alpha\wedge \beta)(e_1,e_2) = \alpha(e_1)\beta(e_2) - \alpha(e_2)\beta(e_1)$.
\end{example}

\begin{exercise} \index{mescolamento}
	Un \emph{mescolamento} di tipo $(k,l)$ è una permutazione $\sigma \in S_{k+l}$ tale che $\sigma(1) < \ldots < \sigma(k)$ e $\sigma(k+1) <\ldots < \sigma(k+l)$.
	Indichiamo $\mathrm{mesc} (k,l)$ l'insieme dei mescolamenti di tipo $(k,l)$.
	
	Mostrare che, se $\alpha\in\Lambda^k(V)$ e $\beta\in\Lambda^l(V)$, allora
	\begin{equation*}
	(\alpha\wedge\beta)(\seqb{e}{k+l}{,}) = \sum_{\sigma \in \mathrm{mesc}(k,l)} \alpha(e_{\sigma(1)},\ldots,e_{\sigma(k)}) \beta(e_{\sigma(k+1)},\ldots,e_{\sigma(k+l)})\punto
	\end{equation*}
\end{exercise}

\begin{proposition} \label{prop:ProprietaWedge}
	Se $\alpha \in T^0_k(V)$, $\beta \in T^0_l(V)$ e $\gamma \in T^0_m(V)$, si ha
	\begin{enumerate}
		\item $\alpha\wedge\beta = (\antisimm \alpha) \wedge \beta = \alpha\wedge (\antisimm \beta)$; \label{pw:Antisimmetrica}
		\item $\wedge$ è bilineare; \label{pw:Bilineare}
		\item $\alpha\wedge \beta = (-1)^{kl} \beta \wedge \alpha$; \label{pw:Anticommutativa}
		\item $\alpha\wedge(\beta\wedge \gamma) = (\alpha\wedge\beta)\wedge\gamma = \frac{(k+l+m)!}{k!\ l!\ m!}\ \antisimm(\alpha\otimes\beta\otimes\gamma)$. \label{pw:Associativa}
	\end{enumerate}
\end{proposition}
\begin{proof}
\begin{description}
 \item [\ref{pw:Antisimmetrica}]
	Siano $\sigma \in S_k$ e $t\in T^0_k(V)$ e definiamo $\sigma t(\seqb{e}{k}{,}) \coloneqq t(e_{\sigma(1)}, \ldots, e_{\sigma(k)})$. Allora, con questa notazione, $\antisimm t =\frac 1{k!} \sum_{\sigma\in S_k} (\sgn\sigma) \sigma t$.
	Abbiamo inoltre che 
	\begin{align*}
		\antisimm (\sigma t)(\seqb ek,) &= \frac 1{k!} \sum_{\rho \in S_k} \sgn (\rho) t(e_{\rho\circ\sigma (1)},\ldots,e_{\rho\circ\sigma(k)}) =\\
		&= \frac 1{k!} \sum_{\tau\in S_k} \sgn(\tau)\sgn(\sigma) t(e_{\tau(1)}, \ldots, e_{\tau(k)}) = \sgn(\sigma) (\antisimm t)(\seqb{e}{k}{,})\virgola
	\end{align*}
	cioè $\antisimm (\sigma t) = \sgn (\sigma) \antisimm t$.%, da cui 
% 	\begin{equation*}
% 		\antisimm t =\sgn (\sigma) \antisimm (\sigma t)= \frac 1{k!} \sum_{\sigma\in S_k} (\sgn\sigma) \sigma t \punto
% 	\end{equation*}

	Ora, grazie alla linearità di $\antisimm$, abbiamo
	\begin{equation*}
		\antisimm(\antisimm \alpha \otimes \beta) = \antisimm\left(\frac 1{k!} \sum_{\tau\in S_k} \sgn(\tau) (\tau\alpha \otimes \beta)\right) =
		\frac 1{k!} \sum_{\tau\in S_k}\sgn(\tau) \antisimm(\tau\alpha\otimes \beta)\punto
	\end{equation*}
	Sia quindi $\tau'\in S_{k+l}$ tale che
	\begin{equation*}
		\tau'(1,\ldots,k,k+1,\ldots,k+l) = (\tau(1),\ldots,\tau(k),k+1,\ldots,k+l) \virgola
	\end{equation*}
	per cui in particolare $\sgn(\tau')=\sgn(\tau)$.
	Allora otteniamo 
	\begin{align*}
		\antisimm(\antisimm \alpha \otimes \beta) &= \frac 1{k!} \sum_{\tau\in S_k} \sgn(\tau')\antisimm\tau'(\alpha\otimes \beta)= \frac 1{k!} \sum_{\tau\in S_k} (\sgn\tau')(\sgn\tau')\antisimm(\alpha\otimes \beta)=\\
		&=\frac 1{k!} \sum_{\tau\in S_k} \antisimm(\alpha\otimes \beta)\virgola
	\end{align*}
	dove abbiamo utilizzato la relazione ricavata prima.
	Perciò $\antisimm(\antisimm\alpha\otimes \beta) = \antisimm(\alpha\otimes\beta)$ e quindi $(\antisimm\alpha)\wedge\beta = \alpha\wedge\beta$.
	
	\item [\ref{pw:Bilineare}] Ovvia.

	\item [\ref{pw:Anticommutativa}]
	Sia $\sigma_0\in S_{k+l}$ data da $\sigma_0(1,\ldots,k+l) = (k+1,\ldots,k+l,1,\ldots,k)$, allora $(\alpha\otimes \beta)(\seqb{e}{k+l}{,}) = (\beta\otimes \alpha)(e_{\sigma_0(1)},\ldots,e_{\sigma_0(k+l)})$.
	Utilizzando la formula ottenuta nel punto \ref{pw:Antisimmetrica}, abbiamo $\antisimm(\alpha\otimes\beta) = \sgn(\sigma_0) \antisimm(\beta\otimes\alpha)=(-1)^{kl} \beta \wedge \alpha$.

	\item [\ref{pw:Associativa}]
	Applicando le definizioni e l'associatività del prodotto tensore, otteniamo
	\begin{align*}
	\alpha\wedge(\beta\wedge\gamma) &= \frac{(k+l+m)!}{k!\ (l+m)!}\ \antisimm(\alpha\otimes(\beta\wedge\gamma)) =\\
	&=\frac{(k+l+m)!}{k!\ (l+m)!}\cdot \frac{(l+m)!}{l!\ m!}\ \antisimm(\alpha\otimes\antisimm(\beta\otimes\gamma))=\\
	&=\frac{(k+l+m)!}{k!\ l!\ m!}\ \antisimm(\alpha\otimes\beta\otimes\gamma)
	\end{align*}
	e analogamente anche l'altra equazione.

	\end{description}
\end{proof}


\begin{remark}
	Dati $\seqa{\alpha}{k}{,}\in\Lambda^1(V)$, abbiamo
	\begin{equation*}
		(\seqa{\alpha}{k}{\wedge})(\seqb{e}{k}{,}) = \sum_{\sigma\in S_k} \sgn(\sigma) \alpha^1(e_{\sigma(1)}) \ldots \alpha^k(e_{\sigma(k)}) = \det[\alpha^i(e_j)] \punto
	\end{equation*}
\end{remark}
\begin{remark}
	Dal punto \ref{pw:Associativa} della \cref{prop:ProprietaWedge}, dati $\gamma^1\in \Lambda^{d_1}(V),\ldots, \gamma^1\in \Lambda^{d_k}(V)$, otteniamo $\seqa{\gamma}{k}{\wedge} = \frac{(\seqb{d}{k}{+})!}{d_1!\ldots d_k!}\antisimm (\seqa{\gamma}{k}{\otimes})$.
	
	In particolare, applicando questa formula a 1-forme, si ha $\seqb{\alpha}{k}{\wedge} = k!\ \antisimm(\seqb{\alpha}{k}{\otimes})$.
	Perciò, se $\seqb en,$ è una base di $V$ e se $\seqa en,$ è la sua base duale, allora $(\seqa ek\wedge)(\seqb en,) = 1$.

\end{remark}

Vediamo ora la scrittura del prodotto wedge in componenti.
Sia $\seqb en,$ una base di $V$, allora le componenti $t_{\seqb ik{}}$ di $t\in T^0_k(V)$ sono date da $t(e_{i_1},\ldots,e_{i_k})$.
Abbiamo quindi $(\antisimm t)_{\seqb ik{}} = \frac 1{k!} \sum_{\sigma\in S_k} \sgn(\sigma)\ t_{\sigma(i_1)\ldots\sigma(i_k)}$, cioè $\antisimm$ simmetrizza le componenti.
% Se $t\in\Lambda^k(V)$, c'è antisimmetria e $(\antisimm t) = \frac 1{k!} \sum_{\sigma\in S_k} \sgn(\sigma) t_{\sigma(i_1)\ldots\sigma(i_k)}$. $\antisimm$ antisimmetrizza le componenti.

Perciò, se $\alpha\in\Lambda^k(V)$ e $\beta\in\Lambda^l(V)$, abbiamo
\begin{equation*}
	(\alpha\wedge\beta)_{\seqb{i}{k+l}{}} = \sum_{\sigma\in \mathrm{mesc}(k,l)} \sgn(\sigma)\ \alpha_{\sigma(i_1)\ldots\sigma(i_k)}\ \beta_{\sigma(i_{k+1})\ldots\sigma(i_{k+l})} \punto
\end{equation*}

\begin{definition}
	Dato $V$ spazio vettoriale, $\Lambda(V)=\bigoplus_k\Lambda^k(V)$ è detta \emph{algebra esterna} di $V$.
\end{definition}

\begin{proposition}
	Se $V$ ha dimensione $n$, $\Lambda^k(V)=0$ per $k>n$ e ha dimensione $\binom nk$ per $k=1,\ldots,n$. Quindi $\Lambda(V)$ ha dimensione $2^n$.
	
	Se $\seqb en,$ è una base di $V$, l'insieme $\{ e^{i_1}\wedge\ldots\wedge e^{i_k} \suchthat 1\le i_1< \ldots < i_k\le n\}$ è una base di $\Lambda^k(V)$.
\end{proposition}

\begin{corollary}
	Sia $\theta\in\Lambda^1(V)$ e $\alpha\in\Lambda^k(V)$. Allora $\theta\wedge\alpha=0$ se e solo se esiste $\beta\in\Lambda^{k-1}(V)$ tale che $\alpha = \theta \wedge \beta$.
\end{corollary}
\begin{proof}
	Si usa il fatto che, se $\theta\in\Lambda^1(V)$, allora $\theta\wedge\theta=0$ \footnote{Questo in generale è falso per $k$ pari: sia per esempio $\omega=e^1\wedge e^2+e^3\wedge e^4$ in $\R^4$, allora $\omega\wedge\omega = 2e^1\wedge e^2 \wedge e^3\wedge e^4 \not=0$} (per la 3).
	Quindi se $\alpha=\theta\wedge\beta$, per la 4 $\theta\wedge\alpha=0$.
	
	Viceversa, sia $\theta\wedge\alpha=0$. Sia $\seqb en,$ base di $V$ tale che $e^n=\theta$. Se $\alpha=\sum_{i_1<\ldots<i_k} \alpha_{\seqb ik{}} e^{i_1}\wedge\ldots\wedge e^{i_k}$, la condizione $\theta\wedge\alpha=0$ implica che i coefficienti in cui $e^n$ non compare sono nulli. Quindi $\alpha$ si fattorizza con $e^n$ e una $k-1$ forma $\beta$.
\end{proof}

\begin{example}
	\begin{enumerate}
		\item Sia $V = \R^2$, $\{e_1,e_2\}$ la base canonica e $\{e^1,e^2\}$ la base duale. Ogni $\omega\in\Lambda^1(\R^2)$ si scrive come $\omega = \omega_1e^1+\omega_2e^2$ e ogni $\omega\in\Lambda^2(\R^2)$ si scrive come $\omega = \omega_{12}\ e^1\wedge e^2$. 
		
		\item In $\R^3$, $\Lambda^1$ e $\Lambda^2$ hanno la stessa dimensione e dunque sono isomorfi. Data $\{e_i\}_{i=1,2,3}$ base canonica e $\{e^i\}$ base duale un isomorfismo è il seguente: $e^1\mapsto e^2\wedge e^3$ e cicliche.
		Questo è detto \emph{operatore Hodge star} che indichiamo con $*$. L'isomorfismo standard di $\R^3$ con $\Lambda^1(\R^3)$ è l'operatore $\flat$: $^\flat(e_i) = e^i$.
		
		Allora $*\circ \flat:\R^3\to \Lambda^2(\R)$ soddisfa
		\begin{equation*}
			(*\circ \flat) (e\times f) = {^\flat e} \wedge {^\flat f}\punto 
		\end{equation*}
		Dove il $\times$ indica il prodotto tensore usale.
		
		Se $\alpha = \alpha_ie^i$ e $\beta = \beta_j e^j$, allora
		\begin{equation*}
			\alpha\wedge\beta = (\alpha_2\beta_3 - \beta_2\alpha_3)e^2\wedge e^3 + (\alpha_3\beta_1 - \alpha_1\beta_3)e^3\wedge e^1 + (\alpha_1\beta_2 - \alpha_2\beta_1) e^1\wedge e^2 \punto
		\end{equation*}
	\end{enumerate}
\end{example}

\begin{exercise}
	Siano $\seqb vk, \in V$ linearmente dipendenti. Mostrare che $\alpha(\seqb vk,) = 0$ qualsiasi $\alpha\in\Lambda^k(V)$. 
\end{exercise}


