\chapter{1 marzo 2016}

Un prodotto interno (simmetrico e definito positivo) in $V$ ne induce uno anche sui $\Lambda^k(V)$, vediamo come.

Siano $\alpha = \alpha_{\seqb ik{}} e^{i_1}\wedge\ldots\wedge e^{i_k}$ e $\beta = \beta_{\seqb ik{}} e^{i_1}\wedge\ldots\wedge e^{i_k}$ elementi di $\Lambda^k(V)$ e sia $\beta^{\seqb ik{}} = g^{i_1j_1}\ldots g^{i_kj_k} \beta_{\seqb jk{}}$.

Definisco
\begin{equation*}
	g^{(k)} (\alpha,\beta) \coloneqq \sum_{\seqb ik<} \alpha_{\seqb ik{}} \beta^{\seqb ik{}} 
	= \frac 1{k!} \sum_{\seqb ik{}} \alpha_{\seqb ik{}} \beta^{\seqb ik{}} \punto
\end{equation*}

Vediamo che $g^{(k)}$ è indipendente dalla base scelta.
Se $\seqb fn,$ è un'altra base, siano $\alpha = \alpha'_{\seqb ak{}} f^{a_1}\wedge\ldots\wedge f^{a_k}$ e $\beta = \beta'_{\seqb ak{}} f^{a_1}\wedge\ldots\wedge f^{a_k}$. Se $e_i =  A^a_i f_a$ e $B=A^{-1}$, allora
\begin{align*}
	\alpha'_{\seqb ak{}} \beta'^{\seqb ak{}} &= \alpha_{\seqb ik{}} B^{i_1}_{a_1}\ldots B^{i_k}_{a_k} A^{a_1}_{j_1}\ldots A^{a_k}_{j_k} \beta^{\seqb jk{}}= \\
	&= \alpha_{\seqb ik{}} \delta^{i_1}_{j_1}\ldots \delta^{i_k}_{j_k}\beta^{\seqb jk{}} =\alpha_{\seqb ik{}} \beta^{\seqb ik{}} \virgola
\end{align*}
da cui la buona definizione di $g^{(k)}$. Dati $\alpha,\beta \in \Lambda^k(V)$ indicheremo anche $\Scal\alpha\beta=g^{(k)}(\alpha,\beta)$.

\begin{proposition}
	Sia $g$ un prodotto interno in $V$ (simmetrico e definito positivo). Allora $g$ induce il prodotto interno $g^{(k)}$ su $\Lambda^k(V)$. Inoltre, se $\seqb en,$ è ortonormale rispetto a $g$, allora la base $\{e^{i_1}\wedge\ldots\wedge e^{i_k} \suchthat \seqb ik< \}$ è ortonormale rispetto a $g^{(k)}$.
\end{proposition}

\section{Operatore di Hodge star}

\begin{proposition} \index{operatore! di Hodge star}
	Sia $V$ uno spazio $n$-dimensionale orientato e sia $g\in T^0_2(V)$ un tensore simmetrico e definito positivo.
	Sia $\mu=\mu(g)$ l'elemento di volume corrispondente.
	Allora esiste un unico isomorfismo $*:\Lambda^k(V) \to \Lambda^{n-k}(V)$, detto \emph{operatore di Hodge star}, che soddisfa
	\begin{equation} \label{eq:star}
	 \alpha \wedge (* \beta)= \Scal \alpha\beta \mu
	\end{equation}
	per ogni $\alpha,\beta \in\Lambda^k(V)$.
	
	Inoltre, se $\{\seqb en, \}$ è una base di $V$ orientata positivamente e ortornormale con base duale $e^1, \dots ,e^n$, allora
	\begin{equation}\label{eq:triangolo}
	*(e^{\sigma(1)}\wedge\ldots\wedge e^{\sigma(k)}) = \sgn(\sigma) (e^{\sigma(k+1)}\wedge\ldots\wedge e^{\sigma(n)})
	\end{equation}
	per ogni $\sigma$ mescolamento di tipo $(k,n-k)$.
\end{proposition}
\begin{proof} %TODO: capire e sistemare
	\begin{description}
		\item [unicità:] Supponiamo che $*$ soddisfi l'\cref{eq:star}. Sia $\beta = e^{\sigma(1)} \wedge\ldots\wedge e^{\sigma(k)}$ e sia $\alpha$ uno dei vettori ortonormali (in $\Lambda^k(V)$) del tipo $e^{i_1}\wedge\ldots\wedge e^{i_k}$ con $\seqb ik<$.
		
		Dall'\cref{eq:star} $\alpha \wedge * \beta=0$ a meno che $(i_1, \ldots,i_k)=(\sigma(1),\ldots,\sigma(k))$
		quindi esiste $a\in\R$ tale che $*\beta = a e^{\sigma(k+1)} \wedge\ldots\wedge e^{\sigma(n)}$, per cui $\beta\wedge * \beta = a\ \sgn(\sigma) \mu$ e la proposizione precedente implica $\Scal \beta\beta=1$.
		
		Quindi deve valere $a = \sgn(\sigma)$ e perciò $*$ è unico.
		
		\item [esistenza:] Usiamo l'\cref{eq:triangolo} ($e^{\sigma(1)}\wedge\ldots\wedge e^{\sigma(n)}$ è ortonormale). Quindi l'\cref{eq:star} si può verificare usando questa base ($*$ è un isomorfismo e manda basi ortonormali in basi ortonormali).
	\end{description}
\end{proof}

\begin{proposition} \label{prop:ProprietaHodge}
	Siano $V,g,\mu$ come sopra. Allora per ogni $\alpha,\beta\in\Lambda^k(V)$ l'operatore $*$ soddisfa le seguenti proprietà
	\begin{enumerate}
		\item $\alpha\wedge *\beta = \beta \wedge * \alpha = \Scal \alpha\beta \mu$; \label{ph:commutare}
		\item $*1 = \mu$ e $*\mu=1$; \label{ph:Identita}
		\item $**\alpha = (-1)^{k(n-k)} \alpha$; \label{ph:StarStar}
		\item $\Scal \alpha\beta = \Scal {*\alpha}{*\beta}$. \label{ph:ScalareDegliHodge}
	\end{enumerate}
\end{proposition}
\begin{proof}
	La \ref{ph:commutare} segue dall'\cref{eq:star} e dal fatto che $g^{(k)}(\cdot,\cdot)=\Scal {\cdot}{\cdot}$ è simmetrico.
	Mentre la \ref{ph:Identita} segue dall'\cref{eq:triangolo} ponendo $\sigma = \id$.
	
	Vediamo ora il punto \ref{ph:StarStar}. Sia $\alpha = e^{\sigma(1)}\wedge \ldots \wedge e^{\sigma(k)}$, allora per l'\cref{eq:triangolo} applicata alla permutazione $(\sigma(k+1),\ldots,\sigma(n),\sigma(1),\ldots,\sigma(k))$ vale 
	\begin{equation*}
		*(e^{\sigma(k+1)} \wedge\ldots\wedge e^{\sigma(n)}) = (-1)^{k(n-k)}\sgn(\sigma) \ e^{\sigma(1)} \wedge\ldots\wedge e^{\sigma(k)}
	\end{equation*}
% 	$*(e^{\sigma(k+1)} \wedge\ldots\wedge e^{\sigma(n)}) = b e^{\sigma(1)} \wedge\ldots\wedge e^{\sigma(k)}$. Per trovare $b$ usiamo l'\cref{eq:star} con $\alpha = \beta = e^{\sigma(k+1)} \wedge\ldots\wedge e^{\sigma(n)}$, da cui $b e^{\sigma(k+1)} \wedge\ldots\wedge e^{\sigma(n)} \wedge e^{\sigma(1)} \wedge\ldots\wedge e^{\sigma(k)} = \mu$ e quindi $b = (-1)^{k(n-k)}\sgn(\sigma)$.
% 	
	Perciò, riapplicando l'\cref{eq:triangolo}, abbiamo
	\begin{align*}
	**(e^{\sigma(1)} \wedge\ldots\wedge e^{\sigma(k)}) &= \sgn(\sigma) * (e^{\sigma(k+1)} \wedge\ldots\wedge e^{\sigma(n)}) =\\
	&=\sgn(\sigma)^2 (-1)^{k(n-k)} e^{\sigma(1)} \wedge\ldots\wedge e^{\sigma(k)} =\\
	&=(-1)^{k(n-k)} e^{\sigma(1)} \wedge\ldots\wedge e^{\sigma(k)} \virgola
	\end{align*}
	da cui abbiamo quanto cercato poiché gli elementi del tipo $e^{\sigma(1)}\wedge \ldots \wedge e^{\sigma(k)}$ generano $\Lambda^k(V)$.
	
	La \ref{ph:ScalareDegliHodge} si può verificare usando una base ortonormale per $\Lambda^k(V)$, oppure usare la \ref{ph:commutare} e la \ref{ph:StarStar} per trovare
	\begin{equation*}
		\Scal{*\alpha}{*\beta} \mu = (*\alpha) \wedge (**\beta) = (-1)^{k(n-k)} (*\alpha)\wedge\beta = \beta\wedge (*\alpha) = \Scal \alpha\beta \mu\punto
	\end{equation*}
\end{proof}


\section{Forme differenziali}

Estendiamo l'algebra esterna al fibrato tangente di una varietà.
Se $\varphi: U \times V \to U' \times V'$ è un isomorfismo locale di fibrati, allora $\varphi_* : U\times \Lambda^k(V) \to U' \times \Lambda^k(V')$ è anch'esso un isomorfismo locale di fibrati.

\begin{definition}
	Sia $\pi : E\to B$ un fibrato vettoriale. Per ogni $A\subseteq B$ poniamo $\Lambda^k(E)\restrict{A} = \bigcup_{b\in A} \Lambda^k(E_b)$ e $\Lambda^k(E) = \Lambda^k(E) \restrict B$.
	Chiamiamo l'ovvia proiezione $\Lambda^k(\pi) : \Lambda^k(E) \to B$.
\end{definition}

Quando $E = TM$ per $M$ varietà, poniamo $\Lambda^k(M) = \Lambda^k(TM)$. Inoltre poniamo $\Omega^1(M) = \Tau^0_1(M) = \chi^*(M)$ e in generale $\Omega^k(M) = \Gamma^\infty(\Lambda^k(M))$ le sezioni $C^\infty$ su $\Lambda^k(M)$.

Estendiamo ora il prodotto wedge alle forme definite su una varietà.
\begin{proposition}
	Se $\alpha \in \Omega^k(M)$ e $\beta \in \Omega^l(M)$, sia $\alpha \wedge \beta : M \to \Lambda^{k+l}(M)$ definita da $(\alpha\wedge\beta)(m) = \alpha(m) \wedge \beta(m)$. Allora $\alpha\wedge\beta \in \Omega^{k+l}(M)$ e questo operatore è bilineare e associativo.
\end{proposition}
\begin{proof}
	Bilinearità e associatività seguono dalle proprietà del prodotto wedge $\wedge$ in $\Lambda(T_mM)$. La regolarità si mostra lavorando in carta.
\end{proof}

\begin{definition} \index{forma!differenziale} \index{algebra!delle forme differenziali}
	Elementi di $\Omega^k(M)$ sono detti \emph{$k$-forme} e $\Omega(M) = \oplus_{k=0}^n \Omega^k(M)$ è detta \emph{algebra delle forme differenziali esterne}.
\end{definition}

\begin{remark}
Vediamo la scrittura di una forma in coordinate locali. Se $t\in\Tau^r_s(M)$, abbiamo visto che $t = t^{\seqb ir{}}_{\seqb js{}} \DerParz{}{x^{i_1}} \otimes \ldots \otimes \DerParz{}{x^{i_r}} \otimes \de x^{j_1} \otimes \ldots \otimes \de x^{j_s}$.
Data $\omega \in \Omega^k(M)$, allora $\omega = \omega_{\seqb ik{}} \de x^{i_1} \wedge \ldots \wedge \de x^{i_k}$ con $\seqb ik<$ e $\omega_{\seqb ik{}} = \omega\left(\DerParz{}{x^{i_1}}, \ldots, \DerParz{}{x^{i_k}}\right)$.
\end{remark}

\begin{remark}
	Se $f:M\to N$ è regolare, è naturale considerare il pull-back $f^*: \Omega^k(N) \to \Omega^k(M)$.
\end{remark}

\section{Derivata esterna}

Studiamo ora una mappa $\de:\Omega(M) \to \Omega(M)$ che ha proprietà particolarmente interessanti.

\begin{theorem} \index{derivata!esterna} \label{thm:CostruzioneDerivataEsterna}
	sia $M$ una varietà $n$-dimensionale. Allora per ogni $k=0,\ldots,n$ e per ogni $U\subseteq M$ aperto esiste unico $\de = \de^k : \Omega^k(U) \to \Omega^{k+1}(U)$ (detta \emph{derivata esterna}) tale che
	\begin{enumerate}
		\item $\de$ è una $\wedge$-antiderivazione, cioè $\de$ è $\R$-lineare e per ogni $\alpha\in\Omega^k(U)$ e $\beta\in\Omega^l(U)$ si ha $\de(\alpha\wedge\beta) = \de\alpha \wedge \beta +(-1)^k \alpha\wedge \de \beta$; \label{cde:Antiderivazione}
		\item se $f\in C^\infty(M)$, allora $\de f$ coincide con il differenziale di $f$; \label{cde:Cinfinito}
		\item $\de^2 = \de\circ \de = 0$; \label{cde:DDUgualeAZero}
		\item $\de$ è naturale rispetto alle restrizioni, cioè se $U\subseteq V \subseteq M$ con $U,V$ aperti e $\alpha\in\Omega^k(V)$, allora $\de(\alpha\restrict U) = (\de\alpha)\restrict U$ (quindi $\de$ è locale). \label{cde:Locale}
	\end{enumerate}
\end{theorem}
\begin{proof}
	\begin{description}
		\item [unicità:] Sia $(U,\varphi)$ una carta di $M$ e consideriamo una $k$-forma $\alpha\in\Omega^k(U)$ con $\alpha = \alpha_{\seqb ik{}} \de x^{i_1} \wedge \ldots \wedge \de x^{i_k}$.
		Se $k=0$ e $f=x^i$, allora $\de f = \de(x^i) = \de x^i$.
		
		Dalla \ref{cde:DDUgualeAZero} abbiamo $\de(\de x^i) = 0$, quindi per la \ref{cde:Antiderivazione} otteniamo $\de(\de x^{i_1} \wedge \ldots \wedge \de x^{i_k}) = 0$.
		
		Perciò $\de\alpha = (\de \alpha_{\seqb ik{}}) \wedge \de x^{i_1} \wedge \ldots\wedge \de x^{i_k} = \DerParz{\alpha_{\seqb ik{}}} {x^i} \de x^i \wedge \de x^{i_1} \wedge\ldots \wedge \de x^{i_k}$, con $i=1,\ldots,n$ e $\seqb ik<$.
		
		Questa formula caratterizza $\de$ su tutti gli aperti di un atlante di $M$, quindi per la \ref{cde:Locale} su tutti gli aperti di $M$.
		
		\item [esistenza:] Definiamo $\de$ come dalla formula sopra e verifichiamo tutte le proprietà. Innanzitutto $\de$ è ovviamente $\R$-lineare.
		
		Siano ora $\alpha\in\Omega^k(U)$ e $\beta\in\Omega^l(U)$ con $\alpha = \alpha_{\seqb ik{}} \de x^{i_1} \wedge \ldots \wedge \de x^{i_k}$ e $\beta= \beta_{\seqb jl{}} \de x^{j_1} \wedge\ldots\wedge \de x^{j_l}$. Allora
		\begin{align*}
			\de& (\alpha \wedge \beta) =\\ 
			&= \de (\alpha_{\seqb ik{}} \beta_{\seqb jl{}} \de x^{i_1} \wedge\ldots\wedge \de x^{i_k}\wedge \de x^{j_1} \wedge\ldots\wedge \de x^{j_l}) =\\
			&= \left(\DerParz{\alpha_{\seqb ik{}}}{x^i} \beta_{\seqb jl{}} + \alpha_{\seqb ik{}} \DerParz{\beta_{\seqb jl{}}}{x^i} \right) \de x^i\wedge\de x^{i_1} \wedge\ldots\wedge \de x^{i_k}\wedge \de x^{j_1} \wedge\ldots\wedge \de x^{j_l} = \\
			&=\de \alpha \wedge \beta + (-1)^k \alpha \wedge \de \beta \punto
		\end{align*}
		
		Per la \ref{cde:DDUgualeAZero}, utilizzando il lemma di Schwartz, abbiamo
		\begin{equation*}
		\de(\de\alpha) = \frac{\partial^2 \alpha_{\seqb ik{}}}{\partial x^i \partial x^j} \de x^i\wedge\de x^j \wedge\de x^{i_1} \wedge\ldots\wedge \de x^{i_k} =0 \punto
		\end{equation*}
		
		Per verificare \ref{cde:Locale} mostriamo che la definizione non dipende dalle carte. Siano $(U,\varphi)$, $(U',\varphi')$ carte tali che $U\cap U' \not=\emptyset$. Per unicità locale $\de = \de'$ in $U\cap U'$ e quindi abbiamo anche l'unicità globale.
	\end{description}
\end{proof}

\begin{corollary}
	Sia $\omega \in \Omega^k(U)$ con $U\subseteq \R^n$. Allora
	\begin{equation*}
		\de \omega (x) (v_0,\ldots,v_k) = \sum_{i=0}^k (-1)^i\ \Diff \omega (x) v_i\ (v_0,\ldots,\hat v_i, \ldots, v_k)\virgola
	\end{equation*}
	dove $\Diff \omega$ è il differenziale classico.
\end{corollary}
\begin{proof}
	Sia $\omega(x) = \omega_{\seqb ik{}} (x) \de x^{i_1}\wedge\ldots\wedge \de x^{i_k}$, allora
	\begin{equation*}
		\Diff \omega(x) v_i = \DerParz{\omega_{\seqb ik{}}}{x^j}\ v_i^j\ \de x^{i_1}\wedge\ldots\wedge \de x^{i_k} \virgola
	\end{equation*}
	quindi il membro destro dell'enunciato è
	\begin{equation*}
		\sum_{\eta\in S_k}(-1)^i \DerParz{\omega_{\seqb ik{}}}{x^j}\ v_i^j\ \sgn(\eta)\ v_{\eta(1)}^{i_1} \ldots v_{\eta(k)}^{i_k} \punto
	\end{equation*}
	D'altra parte
	\begin{align*}
		\de \omega (v_0,\ldots,v_k) &= \DerParz{\omega_{\seqb ik{}}}{x^j} \de x^j \wedge \de x^{i_1}\wedge\ldots\wedge \de x^{i_k} (v_0,\ldots,v_k)= \\ 
		&=\sum_{\sigma \in S_{k+1}}\DerParz{\omega_{\seqb ik{}}}{x^j}\ \sgn(\sigma)\ v_{\sigma(0)}^{j} v_{\sigma(1)}^{i_1} \ldots v_{\sigma(k)}^{i_k} \punto
	\end{align*}
	A questo punto però è facile verificare che le due espressioni trovate coincidono, passando da una permutazione $\eta \in S_k$ ad una permutazione $\sigma \in S_{k+1}$ mandando lo 0 in $i$.
\end{proof}

\begin{example}
	\begin{enumerate}
	 \item In $\R^2$, sia $\alpha = f(x,y) \de x + g(x,y) \de y$, allora $\de \alpha= \de f\wedge \de x + \de g \wedge \de y = \left(\DerParz{g}{x} - \DerParz fy\right) \de x\wedge \de y $.
	 \item In $\R^3$, data $f$ funzione regolare, abbiamo $\de f = \DerParz fx \de x + \DerParz fy \de y + \DerParz fz \de z$ e $\grad f = (\de f)^\sharp$.
	\end{enumerate}
\end{example}

\begin{exercise} %TODO: verificare che l'esercizio abbia senso
	In $\R^3$, sia $\underline F=(F_1,F_2,F_3)$, allora $\underline F^\flat = F_1 \de x + F_2 \de y + F_3 \de z$ e
	$*(\underline F^\flat) = F_3\de x \wedge \de y - F_2 \de x \wedge \de z + F_1 \de y \wedge \de z$.
	Mostrare che $\de \underline{F}^\flat = *(\mathrm{rot} \underline{F})^\flat$.	
	Verificare che $\div \underline F = *\de(* \underline F^\flat)$ e $\de^2 = 0$ e quindi che $\div (\mathrm{rot} \underline F)=0$.
	
	Allo stesso modo, se $f:\R^3\to\R$ è regolare $\de\de f=0$ equivale a $\mathrm{rot}(\grad f)=0$.
\end{exercise}
























































































