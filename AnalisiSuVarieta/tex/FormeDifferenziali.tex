\chapter{Forme differenziali}

In questo capitolo tratteremo le forme differenziali su varietà, che sono campi tensoriali del tipo $(0,k)$ totalmente antisimmetrici.
% \begin{definition} \index{forma!differenziale}
% 	Le \emph{forme differenziali} sono campi tensoriali di tipo $\Tau_k^0(M)$ totalmente antisimmetrici.
% \end{definition}
Hanno applicazioni per esempio in geometria perché possono dare informazioni sulla topologia delle varietà.

\section{Forme esterne su spazi vettoriali}

\begin{definition} \index{mappa!antisimmetrica} \index{forma!esterna}
Dati $V,W$ spazi vettoriali (finito dimensionali), le \emph{mappe antisimmetriche} in $T^0_k(V,W)$ sono quelle per cui vale $t(\seqb{e}{k}{,}) = \sgn(\sigma) t(e_{\sigma(1)},\ldots, e_{\sigma(k)})$ per ogni $\seqb{e}{k}{,} \in V$ e per ogni $\sigma\in S_k$.
Si indicano con $\Lambda^k(V,W)$ (per brevità indicheremo anche $\Lambda^k(V,\R) = \Lambda^k(V)$) e vengono dette \emph{$k$-forme esterne}.

\end{definition}

Dato $t\in T^0_k(V,W)$ è possibile ottenere un elemento di $\Lambda^k(V,W)$ tramite antisimmetrizzazione.
Se per esempio $k=2$, dato $t\in T^0_2(V,W)$, possiamo definire il suo antisimmetrizzato come $\underline{A}t(e_1,e_2) = \frac 12 [t(e_1,e_2) - t(e_2,e_1)]$.

Vediamo quindi la definizione generale.

\begin{definition} \index{mappa!alternante}
	La \emph{mappa alternante} $\antisimm:T^0_k(V,W) \to \Lambda^k(V,W)$ è definita da
	\begin{equation*}
		\antisimm t(\seqb{e}{k}{,}) \coloneqq \frac 1 {k!} \sum_{\sigma \in S_k} (\sgn \sigma) t(e_{\sigma(1)},\ldots, e_{\sigma(k)}) \virgola
	\end{equation*}
	per ogni $\seqb{e}{k}{,} \in V$.
\end{definition}

\begin{proposition}
	La mappa $\antisimm$ è lineare e suriettiva su $\Lambda^k(V,W)$. Inoltre è l'identità sulle $k$-forme, cioè $\antisimm\restrict{\Lambda^k(V,W)} = \id\restrict{\Lambda^k(V,W)}$, ed è idempotente, cioè $\antisimm \circ \antisimm = \antisimm$.
\end{proposition}
\begin{proof}
	Mostriamo che $\antisimm$ è l'identità sulle $k$-forme. Sia $t\in\Lambda^k(V,W)$, allora
	\begin{align*}
		\antisimm t(\seqb{e}{k}{,}) &= \frac 1 {k!} \sum_{\sigma \in S_k} \sgn (\sigma) t(e_{\sigma(1)},\ldots, e_{\sigma(k)}) =\\
		& = \frac 1 {k!} \sum_{\sigma \in S_k} (\sgn (\sigma))^2 t(\seqb{e}{k}{,}) = t (\seqb{e}{k}{,}) \virgola
	\end{align*}
	dove abbiamo usato che $t\in\Lambda^k(V,W)$ e che $\mathrm{card} S_k = k!$.
\end{proof}

\begin{remark}
	Visto che $\antisimm^2=\antisimm$, abbiamo $\norm\antisimm \le \norm\antisimm^2$ e quindi $\norm\antisimm \ge 1 $.
	D'altronde, le permutazioni preservano la norma, perciò $\norm{\antisimm t} \le \norm{t}$ e di conseguenza $\norm \antisimm \le 1$.
	Unendo le due disuguaglianze otteniamo proprio $\norm\antisimm = 1$.
\end{remark}

Parliamo di forme esterne e parleremo di algebre esterne perché si può fare un prodotto esterno, detto wedge, che definiamo di seguito.

\begin{definition} \index{prodotto!wedge}
	Se $\alpha \in T^0_k(V)$ e $\beta \in T^0_l(V)$. Il \emph{prodotto wedge} $\alpha \wedge \beta \in \Lambda^{k+l}(V)$ è definito da
	\begin{equation*}
		\alpha\wedge \beta \coloneqq \frac {(k+l)!}{k!\ l!}\ \antisimm(\alpha\otimes \beta)\punto
	\end{equation*}
\end{definition}

\begin{example}
	Se $\alpha,\beta\in\Lambda^1(V) \cong T^0_1(V)$, allora $(\alpha\wedge \beta)(e_1,e_2) = \alpha(e_1)\beta(e_2) - \alpha(e_2)\beta(e_1)$.
\end{example}

\begin{exercise} \index{mescolamento}
	Un \emph{mescolamento} di tipo $(k,l)$ è una permutazione $\sigma \in S_{k+l}$ tale che $\sigma(1) < \ldots < \sigma(k)$ e $\sigma(k+1) <\ldots < \sigma(k+l)$.
	Indichiamo $\mathrm{mesc} (k,l)$ l'insieme dei mescolamenti di tipo $(k,l)$.
	Mostrare che, se $\alpha\in\Lambda^k(V)$ e $\beta\in\Lambda^l(V)$, allora
	\begin{equation*}
	(\alpha\wedge\beta)(\seqb{e}{k+l}{,}) = \sum_{\sigma \in \mathrm{mesc}(k,l)} \sgn(\sigma) \alpha(e_{\sigma(1)},\ldots,e_{\sigma(k)}) \beta(e_{\sigma(k+1)},\ldots,e_{\sigma(k+l)})\punto
	\end{equation*}
\end{exercise}

\begin{proposition} \label{prop:ProprietaWedge}
	Se $\alpha \in T^0_k(V)$, $\beta \in T^0_l(V)$ e $\gamma \in T^0_m(V)$, si ha
	\begin{enumerate}
		\item $\alpha\wedge\beta = (\antisimm \alpha) \wedge \beta = \alpha\wedge (\antisimm \beta)$; \label{pw:Antisimmetrica}
		\item $\wedge$ è bilineare; \label{pw:Bilineare}
		\item $\alpha\wedge \beta = (-1)^{kl} \beta \wedge \alpha$; \label{pw:Anticommutativa}
		\item $\alpha\wedge(\beta\wedge \gamma) = (\alpha\wedge\beta)\wedge\gamma = \frac{(k+l+m)!}{k!\ l!\ m!}\ \antisimm(\alpha\otimes\beta\otimes\gamma)$. \label{pw:Associativa}
	\end{enumerate}
\end{proposition}
\begin{proof}
\begin{description}
 \item [\ref{pw:Antisimmetrica}]
	Dati $\sigma \in S_k$ e $t\in T^0_k(V)$, definiamo $\sigma t(\seqb{e}{k}{,}) \coloneqq t(e_{\sigma(1)}, \ldots, e_{\sigma(k)})$. Allora, con questa notazione, $\antisimm t =\frac 1{k!} \sum_{\sigma\in S_k} (\sgn\sigma) \sigma t$.
	Abbiamo inoltre che 
	\begin{align*}
		\antisimm (\sigma t)(\seqb ek,) &= \frac 1{k!} \sum_{\rho \in S_k} \sgn (\rho) t(e_{\rho\circ\sigma (1)},\ldots,e_{\rho\circ\sigma(k)}) =\\
		&= \frac 1{k!} \sum_{\tau\in S_k} \sgn(\tau)\sgn(\sigma) t(e_{\tau(1)}, \ldots, e_{\tau(k)}) = \sgn(\sigma) (\antisimm t)(\seqb{e}{k}{,})\virgola
	\end{align*}
	cioè $\antisimm (\sigma t) = \sgn (\sigma) \antisimm t$.%, da cui 
% 	\begin{equation*}
% 		\antisimm t =\sgn (\sigma) \antisimm (\sigma t)= \frac 1{k!} \sum_{\sigma\in S_k} (\sgn\sigma) \sigma t \punto
% 	\end{equation*}

	Ora, grazie alla linearità di $\antisimm$, abbiamo
	\begin{equation*}
		\antisimm(\antisimm \alpha \otimes \beta) = \antisimm\left(\frac 1{k!} \sum_{\tau\in S_k} \sgn(\tau) (\tau\alpha \otimes \beta)\right) =
		\frac 1{k!} \sum_{\tau\in S_k}\sgn(\tau) \antisimm(\tau\alpha\otimes \beta)\punto
	\end{equation*}
	Sia quindi $\tau'\in S_{k+l}$ tale che
	\begin{equation*}
		\tau'(1,\ldots,k,k+1,\ldots,k+l) = (\tau(1),\ldots,\tau(k),k+1,\ldots,k+l) \virgola
	\end{equation*}
	per cui in particolare $\sgn(\tau')=\sgn(\tau)$.
	Allora otteniamo 
	\begin{align*}
		\antisimm(\antisimm \alpha \otimes \beta) &= \frac 1{k!} \sum_{\tau\in S_k} \sgn(\tau')\antisimm\tau'(\alpha\otimes \beta)= \frac 1{k!} \sum_{\tau\in S_k} (\sgn\tau')(\sgn\tau')\antisimm(\alpha\otimes \beta)=\\
		&=\frac 1{k!} \sum_{\tau\in S_k} \antisimm(\alpha\otimes \beta)\virgola
	\end{align*}
	dove abbiamo utilizzato la relazione ricavata prima.
	Perciò $\antisimm(\antisimm\alpha\otimes \beta) = \antisimm(\alpha\otimes\beta)$ e quindi $(\antisimm\alpha)\wedge\beta = \alpha\wedge\beta$.
	
	\item [\ref{pw:Bilineare}] Ovvia.

	\item [\ref{pw:Anticommutativa}]
	Sia $\sigma_0\in S_{k+l}$ data da $\sigma_0(1,\ldots,k+l) = (k+1,\ldots,k+l,1,\ldots,k)$, allora $(\alpha\otimes \beta)(\seqb{e}{k+l}{,}) = (\beta\otimes \alpha)(e_{\sigma_0(1)},\ldots,e_{\sigma_0(k+l)})$.
	Utilizzando la formula ottenuta nel punto \ref{pw:Antisimmetrica}, abbiamo $\antisimm(\alpha\otimes\beta) = \sgn(\sigma_0) \antisimm(\beta\otimes\alpha)=(-1)^{kl} \beta \wedge \alpha$.

	\item [\ref{pw:Associativa}]
	Applicando le definizioni e l'associatività del prodotto tensore, otteniamo
	\begin{align*}
	\alpha\wedge(\beta\wedge\gamma) &= \frac{(k+l+m)!}{k!\ (l+m)!}\ \antisimm(\alpha\otimes(\beta\wedge\gamma)) =\\
	&=\frac{(k+l+m)!}{k!\ (l+m)!}\cdot \frac{(l+m)!}{l!\ m!}\ \antisimm(\alpha\otimes\antisimm(\beta\otimes\gamma))=\\
	&=\frac{(k+l+m)!}{k!\ l!\ m!}\ \antisimm(\alpha\otimes\beta\otimes\gamma)
	\end{align*}
	e analogamente anche l'altra equazione.

	\end{description}
\end{proof}


\begin{remark}
	Dati $\seqa{\alpha}{k}{,}\in\Lambda^1(V)$, abbiamo
	\begin{equation*}
		(\seqa{\alpha}{k}{\wedge})(\seqb{e}{k}{,}) = \sum_{\sigma\in S_k} \sgn(\sigma) \alpha^1(e_{\sigma(1)}) \ldots \alpha^k(e_{\sigma(k)}) = \det[\alpha^i(e_j)] \punto
	\end{equation*}
\end{remark}
\begin{remark}
	Dal punto \ref{pw:Associativa} della \cref{prop:ProprietaWedge}, dati $\gamma^1\in \Lambda^{d_1}(V),\ldots, \gamma^k\in \Lambda^{d_k}(V)$, otteniamo $\seqa{\gamma}{k}{\wedge} = \frac{(\seqb{d}{k}{+})!}{d_1!\ldots d_k!}\antisimm (\seqa{\gamma}{k}{\otimes})$.
	
	In particolare, applicando questa formula a 1-forme, si ha $\seqb{\alpha}{k}{\wedge} = k!\ \antisimm(\seqb{\alpha}{k}{\otimes})$.
	Perciò, se $\seqb en,$ è una base di $V$ e se $\seqa en,$ è la sua base duale, allora $(\seqa ek\wedge)(\seqb en,) = 1$.

\end{remark}

Vediamo ora la scrittura del prodotto wedge in componenti.
Sia $\seqb en,$ una base di $V$, allora le componenti $t_{\seqb ik{}}$ di $t\in T^0_k(V)$ sono date da $t(e_{i_1},\ldots,e_{i_k})$.
Abbiamo quindi $(\antisimm t)_{\seqb ik{}} = \frac 1{k!} \sum_{\sigma\in S_k} \sgn(\sigma)\ t_{\sigma(i_1)\ldots\sigma(i_k)}$, cioè $\antisimm$ simmetrizza le componenti.
% Se $t\in\Lambda^k(V)$, c'è antisimmetria e $(\antisimm t) = \frac 1{k!} \sum_{\sigma\in S_k} \sgn(\sigma) t_{\sigma(i_1)\ldots\sigma(i_k)}$. $\antisimm$ antisimmetrizza le componenti.

Perciò, se $\alpha\in\Lambda^k(V)$ e $\beta\in\Lambda^l(V)$, abbiamo
\begin{equation*}
	(\alpha\wedge\beta)_{\seqb{i}{k+l}{}} = \sum_{\sigma\in \mathrm{mesc}(k,l)} \sgn(\sigma)\ \alpha_{\sigma(i_1)\ldots\sigma(i_k)}\ \beta_{\sigma(i_{k+1})\ldots\sigma(i_{k+l})} \punto
\end{equation*}

\begin{definition} \index{algebra!esterna}
	Dato $V$ spazio vettoriale, lo spazio $\Lambda(V)=\bigoplus_k\Lambda^k(V)$ è detto \emph{algebra esterna} di $V$.
\end{definition}

\begin{proposition}
	Se $V$ è uno spazio vettoriale di dimensione $n$, $\Lambda^k(V)$ è 0 per $k>n$ e ha dimensione $\binom nk$ per $k=1,\ldots,n$. Quindi $\Lambda(V)$ ha dimensione $2^n$.
	
	In particolare, se $\seqb en,$ è una base di $V$, l'insieme $\{ e^{i_1}\wedge\ldots\wedge e^{i_k} \suchthat 1\le i_1< \ldots < i_k\le n\}$ è una base di $\Lambda^k(V)$.
\end{proposition}

\begin{corollary}
	Sia $\theta\in\Lambda^1(V)$ e $\alpha\in\Lambda^k(V)$. Allora $\theta\wedge\alpha=0$ se e solo se esiste $\beta\in\Lambda^{k-1}(V)$ tale che $\alpha = \theta \wedge \beta$.
\end{corollary}
\begin{proof}
	Se $\theta\in\Lambda^1(V)$, allora $\theta\wedge\theta=0$ \footnote{Questo in generale è falso per $k$ pari: sia per esempio $\omega=e^1\wedge e^2+e^3\wedge e^4$ in $\R^4$, allora $\omega\wedge\omega = 2e^1\wedge e^2 \wedge e^3\wedge e^4 \not=0$} per il punto \ref{pw:Anticommutativa} della \cref{prop:ProprietaWedge}.
	Quindi se $\alpha=\theta\wedge\beta$, abbiamo $\theta\wedge\alpha=0$ per il punto \ref{pw:Associativa} della stessa proposizione.
	
	Viceversa, supponiamo che $\theta\wedge\alpha=0$. Sia $\seqb en,$ base di $V$ tale che $e^n=\theta$. Allora, se $\alpha=\sum_{i_1<\ldots<i_k} \alpha_{\seqb ik{}} e^{i_1}\wedge\ldots\wedge e^{i_k}$, la condizione $\theta\wedge\alpha=0$ implica che i coefficienti in cui $e^n$ non compare sono nulli. Quindi $\alpha$ si fattorizza come $e^n$ per una $(k-1)$-forma $\beta$.
\end{proof}

\begin{example}
	\begin{enumerate}
		\item Sia $V = \R^2$, con $\{e_1,e_2\}$ la base canonica e $\{e^1,e^2\}$ la base duale. Ogni $\omega\in\Lambda^1(\R^2)$ si scrive come $\omega = \omega_1e^1+\omega_2e^2$ e ogni $\omega\in\Lambda^2(\R^2)$ si scrive come $\omega = \omega_{12}\ e^1\wedge e^2$. 
		
		\item In $\R^3$, gli spazi $\Lambda^1$ e $\Lambda^2$ hanno la stessa dimensione e dunque sono isomorfi. Data $\{e_1,e_2,e_3\}$ base canonica e $\{e^1,e^2,e^3\}$ base duale un isomorfismo è il seguente: $e^1\mapsto e^2\wedge e^3$ e cicliche.
		Questo è detto \emph{operatore Hodge star} che indichiamo con $*$ e approfondiremo più avanti. L'isomorfismo standard di $\R^3$ con $\Lambda^1(\R^3)$ è l'operatore $\flat$ tale che $(e_i)^\flat = e^i$.
		
		Allora $*\circ \flat:\R^3\to \Lambda^2(\R)$ soddisfa
		\begin{equation*}
			(*\circ \flat) (e\times f) = e^\flat \wedge f^\flat \virgola
		\end{equation*}
		per ogni $e,f \in \R^3$.
		
		Questo segue dal fatto che, se $\alpha = \alpha_ie^i$ e $\beta = \beta_j e^j$, allora
		\begin{equation*}
			\alpha\wedge\beta = (\alpha_2\beta_3 - \beta_2\alpha_3)\ e^2\wedge e^3 + (\alpha_3\beta_1 - \alpha_1\beta_3)\ e^3\wedge e^1 + (\alpha_1\beta_2 - \alpha_2\beta_1)\ e^1\wedge e^2 \punto
		\end{equation*}
	\end{enumerate}
\end{example}

\begin{exercise}
	Siano $\seqb vk, \in V$ linearmente dipendenti. Mostrare che per ogni $\alpha\in\Lambda^k(V)$ vale $\alpha(\seqb vk,) = 0$. 
\end{exercise}


\section{Determinanti e volumi}

Il determinante di una matrice è multilineare e antisimmetrico su righe e colonne di una matrice.
Se $\seqb xn, \in\R^n$ hanno componenti $x_i^j$, allora il determinante di $(x_i^j)$ è il volume del parallelepipedo generato da $\seqb xn,$.
Possiamo indicare
\begin{equation*}
	\det[\seqb xn,] = \sum_{\sigma\in S_n} \sgn(\sigma) x_{\sigma(1)}^1\ldots x_{\sigma(n)}^n \punto
\end{equation*}
Se $\varphi:\R^n\to\R^n$ è lineare $\det\varphi = \frac{\text{volume del parallelepipedo generato da $\varphi(e_1),\ldots,\varphi(e_n)$}}{\text{volume del cubo unitario}}$.

Rivediamo ora le proprietà del pull-back nel caso particolare di forme differenziali o più in generale di tensori solo covarianti.
\begin{proposition} \label{prop:PullBackForme}
	Sia $\varphi\in\Lin(V,W)$, allora $\varphi^* : T^0_k(W) \to T^0_k(V)$ è lineare e valgono 
	\begin{enumerate}
		\item $\varphi^*(\Lambda^k(W)) \subseteq \Lambda^k(V)$;
		\item se $\psi\in\Lin(W,Z)$, $(\psi\circ\varphi)^* = \varphi^* \circ \psi^*$; \label{pbf:Composizione}
		\item $(\id_V)^* = \id_{T^0_k(V)}$;
		\item se $\varphi\in \mathrm{GL}(V,W)$, allora $\varphi^*\in \mathrm{GL}(T^0_k(W), T^0_k(V))$ e $(\varphi^*)^{-1} = (\varphi^{-1})^* = \varphi_*$.
		Se anche $\psi\in\mathrm{GL}(W,Z)$, allora $(\psi\circ\varphi)_* = \psi_*\circ \varphi_*$;
		\item se $\alpha\in\Lambda^k(W)$, $\beta\in\Lambda^l(W)$, allora $\varphi^*(\alpha\wedge\beta) = (\varphi^*\alpha) \wedge (\varphi^*\beta)$.
	\end{enumerate}
\end{proposition}

Ricordiamo che $\dim(\Lambda^n(V)) = 1$, dove $\dim(V)=n$, per cui ha senso la seguente definizione.
\begin{definition} \index{determinante}
	Se $V$ ha dimensione $n$ e $\varphi\in\Lin(V,V)$, il \emph{determinante} $\det(\varphi)$ di $\varphi$ è definito come l'unica costante tale che $\varphi^*\omega = \det(\varphi) \omega$, per qualsiasi $\omega\in\Lambda^n(V)$.
\end{definition}

\begin{proposition} \label{prop:DeterminanteCoordinate}
	Sia $\seqb en,$ una base di $V$ con base duale $\seqa en,$. Sia $\varphi \in \Lin(V,V)$ tale che $\varphi(e_i) = A_i^je_j$.
	Allora $\det(\varphi) = \det A$.
\end{proposition}
\begin{proof}	
	Dalla dimostrazione del punto \ref{pw:Associativa} della \cref{prop:ProprietaWedge}, abbiamo
	\begin{align*}
		\det(\varphi) &=\varphi^*(\seqa en\wedge) (\seqb en,) = (\seqa en\wedge)(\varphi(e_1),\ldots, \varphi(e_n)) =\\
		&=n!\ \antisimm (\seqa en\otimes)(\varphi(e_1),\ldots,\varphi(e_n)) =\\
		&= \sum_{\sigma \in S_n} \sgn(\sigma) e^1(\varphi(e_{\sigma(1)})) \ldots e^n(\varphi(e_{\sigma(n)})) = \sum_{\sigma \in S_n} \sgn(\sigma) A^1_{\sigma(1)} \ldots A^n_{\sigma(n)}\\
		&=\det A
		\virgola
	\end{align*}
	che è quanto cercato.
\end{proof}

\begin{proposition}
	Siano $\varphi,\psi \in\Lin(V,V)$. Allora
	\begin{enumerate}
		\item $\det(\varphi\circ\psi) = \det \varphi \cdot \det \psi$;
		\item $\det (\id_V) = 1$;
		\item $\varphi$ è un isomorfismo se e solo se $\det\varphi\not=0$ e in tal caso $\det(\varphi^{-1}) = (\det \varphi)^{-1}$.
	\end{enumerate}
\end{proposition}
\begin{proof}
	L'unica dimostrazione non ovvia è quella del fatto che se $\det\varphi\not=0$, allora $\varphi$ è un isomorfismo.
% 	L'unica non ovvia è la freccia $\Leftarrow$ della 3.
	Basta dimostrare che se $\varphi$ non è un isomorfismo, allora $\det\varphi = 0$.
	
	Se $\varphi$ non è un isomorfismo, esiste $e=e_1\in\ker\varphi$. Completiamo $e_1$ ad una base $\{\seqb en,\}$ di $V$. Data $\omega\in\Lambda^n(V)$, allora
	\begin{equation*}
		(\varphi^*\omega)(\seqb en,) = \omega(\varphi(e_1),\ldots,\varphi(e_n)) = \omega(0,\varphi(e_2),\ldots,\varphi(e_n)) = 0 \virgola
	\end{equation*}
	perché $\omega$ è multilineare. Perciò $\det\varphi = 0$, come voluto.
\end{proof}

\begin{definition} \index{elementi di volume}\index{orientazione} \index{spazio orientato}
	Gli elementi di $\Lambda^n(V)$ sono detti \emph{elementi di volume}. Se $\omega_1,\omega_2$ sono elementi di volume diciamo che $\omega_1\sim \omega_2$ se esiste $c>0$ tale che $\omega_1 = c\omega_2$. Una classe di equivalenza di elementi di volume è detta un'\emph{orientazione} su $V$.
	
	Uno \emph{spazio orientato} $(V, [\omega])$ è uno spazio vettoriale con un'orientazione $[\omega]$. La classe $[-\omega]$ è detta \emph{orientazione inversa}.
	
	Una base $\{\seqb en,\}$ di $(V,[\omega])$ orientato è detta essere orientata positivamente se $\omega(\seqb en,)>0$. Ovviamente tale definizione è indipendente dalla scelta di un elemento di $[\omega]$.
\end{definition}

\begin{proposition} \index{metrica!base ortonormale} \label{prop:EsistenzaBaseOrtonormale}
	Sia $g\in T^0_2(V)$ simmetrico e definito positivo. Allora esiste una base $\{\seqb en,\}$ di $V$ con base duale $\{\seqa en,\}$ tale che $g = \sum_{i=1}^n e^i\otimes e^i$. Questa base è detta \emph{ortonormale} rispetto a $g$.
\end{proposition}
\begin{proof}
	La dimostrazione è analoga al caso di $\R^n$ utilizzando un processo di ortogonalizzazione di Gram-Schmidt.
	
	Sia $e_1$ tale che $g(e_1,e_1) = 1$ e siano $V_1 = \spanrm \{e_1\}$ e $V_2 = \{e \suchthat g(e,e_1) = 0\}$.
	Visto che $g$ è definita positiva, $V_1\cap V_2= \{0\}$. Dato $z\in V$, allora $z = g(e_1,z)e_1 + (z-g(e_1,z)e_1)$, dove il secondo termine appartiene a $V_2$. Quindi $V = V_1\oplus V_2$.
	Scegliamo quindi $e_2\in V_2$ tale che $g(e_2,e_2)=1$ e così via fino a completare la base.
\end{proof}


\begin{proposition} %\index{$g$-volume}
	Sia $V$ uno spazio vettoriale $n$ dimensionale e sia $g\in T^0_2(V)$ simmetrico, definito positivo. Allora, se $[\omega]$ è un'orientazione in $V$, esiste un unico elemento di volume $\mu(g) \in [\omega]$, detto \emph{$g$-volume}, tale che $\mu(g)(\seqb en,) = 1$ per ogni base $\seqb en,$ di $V$ ortonormale rispetto a $g$ e orientata.
	Se poi $\seqa en,$ è la base duale, allora $\mu(g) = \seqa en\wedge$.
	
	In generale, se $\seqb fn,$ è una base orientata positivamente e se $f^i$ è la base duale, allora
	\begin{equation*}
		\mu(g) = \abs{ \det[g(f_i,f_j)] }^{\frac 12}\ \seqa fn\wedge \punto
	\end{equation*}
\end{proposition}
\begin{proof}
	Sia $\{\seqb en,\}$ una base ortonormale rispetto a $g$, che ci è garantita dalla \cref{prop:EsistenzaBaseOrtonormale}. Allora $g(f_i,f_j) = \sum_{p=1}^n e^p\otimes e^p (f_i,f_j)$.
	
	Se $\varphi \in \mathrm{GL}(V,V)$ è tale che $f_i = \varphi(e_i) = A_i^je_j$, allora
	\begin{equation*}
	g(f_i,f_j) = \sum_{p=1}^n e^p\otimes e^p (A_i^k e_k, A_j^le_l) = \sum_{p=1}^n \delta_k^p\ \delta_l^p\ A_i^k\ A_j^l = \sum_{p=1}^nA_i^pA_j^p = (A \cdot A^T)_i^j\puntovirgola
	\end{equation*}
	perciò, sfruttando la \cref{prop:DeterminanteCoordinate}, abbiamo $\det[g(f_i,f_j)] = (\det A)^2 = (\det\varphi)^2$.
	
	Se ora $\{\seqb en,\}$ è orientata positivamente e $g$-ortonormale, allora per la richiesta di multilinearità esiste un'unica $\mu(g)=\mu$ tale che $\mu(\seqb en,) = 1$. Mostriamo che $\mu(g)$ così definita rispetta le ipotesi richieste.
	
	Se $\seqb fn,$ è un'altra base orientata positivamente e $g$-ortonormale, sia $\varphi$ come sopra. Allora per quanto detto prima $\abs{\det\varphi} = 1$. Inoltre per la positività della base e la definizione di $\varphi$ abbiamo $0<\mu(\seqb fn,) = (\varphi^*\mu)(\seqb en,) = \det\varphi$, quindi $\det\varphi=1$ e in particolare $\mu(\seqb fn,)=1$, come voluto.
	
	La seconda affermazione è implicata dalla terza e per quest'ultima si usa la formula $\mu(\seqb fn,) = \det\varphi = \abs{ \det[g(f_i,f_j)] }^\frac 12$, con le notazioni date all'inizio.
\end{proof}

Un prodotto interno (simmetrico e definito positivo) in $V$ ne induce uno anche sui $\Lambda^k(V)$, vediamo come.

Siano $\alpha = \alpha_{\seqb ik{}} e^{i_1}\wedge\ldots\wedge e^{i_k}$ e $\beta = \beta_{\seqb ik{}} e^{i_1}\wedge\ldots\wedge e^{i_k}$ elementi di $\Lambda^k(V)$ e sia $\beta^{\seqb ik{}} = g^{i_1j_1}\ldots g^{i_kj_k} \beta_{\seqb jk{}}$.

Definisco
\begin{equation*}
	g^{(k)} (\alpha,\beta) \coloneqq \sum_{\seqb ik<} \alpha_{\seqb ik{}} \beta^{\seqb ik{}} 
	= \frac 1{k!} \sum_{\seqb ik{}} \alpha_{\seqb ik{}} \beta^{\seqb ik{}} \punto
\end{equation*}

Vediamo che $g^{(k)}$ è indipendente dalla base scelta.
Se $\seqb fn,$ è un'altra base, siano $\alpha = \alpha'_{\seqb ak{}} f^{a_1}\wedge\ldots\wedge f^{a_k}$ e $\beta = \beta'_{\seqb ak{}} f^{a_1}\wedge\ldots\wedge f^{a_k}$. Se $e_i =  A^a_i f_a$ e $B=A^{-1}$, allora
\begin{align*}
	\alpha'_{\seqb ak{}} \beta'^{\seqb ak{}} &= \alpha_{\seqb ik{}} B^{i_1}_{a_1}\ldots B^{i_k}_{a_k} A^{a_1}_{j_1}\ldots A^{a_k}_{j_k} \beta^{\seqb jk{}}= \\
	&= \alpha_{\seqb ik{}} \delta^{i_1}_{j_1}\ldots \delta^{i_k}_{j_k}\beta^{\seqb jk{}} =\alpha_{\seqb ik{}} \beta^{\seqb ik{}} \virgola
\end{align*}
da cui la buona definizione di $g^{(k)}$. Dati $\alpha,\beta \in \Lambda^k(V)$ indicheremo anche $\Scal\alpha\beta=g^{(k)}(\alpha,\beta)$.

\begin{proposition} \label{prop:BaseOrtonormaleMetricaLambdaK}
	Sia $g$ un prodotto interno in $V$ (simmetrico e definito positivo). Allora $g$ induce il prodotto interno $g^{(k)}$ su $\Lambda^k(V)$. Inoltre, se $\seqb en,$ è ortonormale rispetto a $g$, allora la base $\{e^{i_1}\wedge\ldots\wedge e^{i_k} \suchthat \seqb ik< \}$ è ortonormale rispetto a $g^{(k)}$.
\end{proposition}

\section{Operatore di Hodge star}

\begin{proposition} \index{operatore! di Hodge star}
	Sia $V$ uno spazio $n$-dimensionale orientato e sia $g\in T^0_2(V)$ un tensore simmetrico e definito positivo.
	Sia $\mu=\mu(g)$ l'elemento di volume corrispondente.
	Allora esiste un unico isomorfismo $*:\Lambda^k(V) \to \Lambda^{n-k}(V)$, detto \emph{operatore di Hodge star}, che soddisfa
	\begin{equation} \label{eq:star}
	 \alpha \wedge (* \beta)= \Scal \alpha\beta \mu
	\end{equation}
	per ogni $\alpha,\beta \in\Lambda^k(V)$.
	
	Inoltre, se $\{\seqb en, \}$ è una base di $V$ orientata positivamente e ortonormale con base duale $e^1, \dots ,e^n$, allora
	\begin{equation}\label{eq:triangolo}
	*(e^{\sigma(1)}\wedge\ldots\wedge e^{\sigma(k)}) = \sgn(\sigma) (e^{\sigma(k+1)}\wedge\ldots\wedge e^{\sigma(n)})
	\end{equation}
	per ogni $\sigma$ mescolamento di tipo $(k,n-k)$.
\end{proposition}
\begin{proof}
	\begin{description}
		\item [unicità:] Supponiamo che $*$ soddisfi l'\cref{eq:star}. Sia $\beta = e^{\sigma(1)} \wedge\ldots\wedge e^{\sigma(k)}$ e sia $\alpha$ uno dei vettori in $\Lambda^k(V)$ ortonormali del tipo $e^{i_1}\wedge\ldots\wedge e^{i_k}$ con $\seqb ik<$.
		
		Dall'\cref{eq:star} $\alpha \wedge * \beta=0$ a meno che $(i_1, \ldots,i_k)=(\sigma(1),\ldots,\sigma(k))$
		quindi esiste $a\in\R$ tale che $*\beta = a e^{\sigma(k+1)} \wedge\ldots\wedge e^{\sigma(n)}$.
		Di conseguenza abbiamo $\beta\wedge * \beta = a\ \sgn(\sigma) \mu$, ma la \cref{prop:BaseOrtonormaleMetricaLambdaK} implica $\Scal \beta\beta=1$, quindi deve valere $a = \sgn(\sigma)$ e perciò $*$ è unico.
		
		\item [esistenza:] Definiamo l'operatore sulla base ortonormale $\{e^{i_1}\wedge\ldots\wedge e^{i_k} \suchthat \seqb ik <\}$, utilizzando l'\cref{eq:triangolo}. Possiamo verificare poi l'\cref{eq:star} usando questa base. In particolare $*$ è un isomorfismo e manda basi ortonormali in basi ortonormali.
	\end{description}
\end{proof}

\begin{proposition} \label{prop:ProprietaHodge}
	Siano $V,g,\mu$ come sopra. Allora per ogni $\alpha,\beta\in\Lambda^k(V)$ l'operatore $*$ soddisfa le seguenti proprietà
	\begin{enumerate}
		\item $\alpha\wedge *\beta = \beta \wedge * \alpha = \Scal \alpha\beta \mu$; \label{ph:commutare}
		\item $*1 = \mu$ e $*\mu=1$; \label{ph:Identita}
		\item $**\alpha = (-1)^{k(n-k)} \alpha$; \label{ph:StarStar}
		\item $\Scal \alpha\beta = \Scal {*\alpha}{*\beta}$. \label{ph:ScalareDegliHodge}
	\end{enumerate}
\end{proposition}
\begin{proof}
	La \ref{ph:commutare} segue dall'\cref{eq:star} e dal fatto che $g^{(k)}(\cdot,\cdot)=\Scal {\cdot}{\cdot}$ è simmetrico.
	Mentre la \ref{ph:Identita} segue dall'\cref{eq:triangolo} ponendo $\sigma = \id$.
	
	Vediamo ora il punto \ref{ph:StarStar}. Sia $\alpha = e^{\sigma(1)}\wedge \ldots \wedge e^{\sigma(k)}$, allora per l'\cref{eq:triangolo} applicata alla permutazione $(\sigma(k+1),\ldots,\sigma(n),\sigma(1),\ldots,\sigma(k))$ vale 
	\begin{equation*}
		*(e^{\sigma(k+1)} \wedge\ldots\wedge e^{\sigma(n)}) = (-1)^{k(n-k)}\sgn(\sigma) \ e^{\sigma(1)} \wedge\ldots\wedge e^{\sigma(k)}
	\end{equation*}
% 	$*(e^{\sigma(k+1)} \wedge\ldots\wedge e^{\sigma(n)}) = b e^{\sigma(1)} \wedge\ldots\wedge e^{\sigma(k)}$. Per trovare $b$ usiamo l'\cref{eq:star} con $\alpha = \beta = e^{\sigma(k+1)} \wedge\ldots\wedge e^{\sigma(n)}$, da cui $b e^{\sigma(k+1)} \wedge\ldots\wedge e^{\sigma(n)} \wedge e^{\sigma(1)} \wedge\ldots\wedge e^{\sigma(k)} = \mu$ e quindi $b = (-1)^{k(n-k)}\sgn(\sigma)$.
% 	
	Perciò, riapplicando l'\cref{eq:triangolo}, abbiamo
	\begin{align*}
	**(e^{\sigma(1)} \wedge\ldots\wedge e^{\sigma(k)}) &= \sgn(\sigma) * (e^{\sigma(k+1)} \wedge\ldots\wedge e^{\sigma(n)}) =\\
	&=\sgn(\sigma)^2 (-1)^{k(n-k)} e^{\sigma(1)} \wedge\ldots\wedge e^{\sigma(k)} =\\
	&=(-1)^{k(n-k)} e^{\sigma(1)} \wedge\ldots\wedge e^{\sigma(k)} \virgola
	\end{align*}
	da cui abbiamo quanto cercato poiché gli elementi del tipo $e^{\sigma(1)}\wedge \ldots \wedge e^{\sigma(k)}$ generano $\Lambda^k(V)$.
	
	La \ref{ph:ScalareDegliHodge} si può verificare usando una base ortonormale per $\Lambda^k(V)$, oppure usare la \ref{ph:commutare} e la \ref{ph:StarStar} per trovare
	\begin{equation*}
		\Scal{*\alpha}{*\beta} \mu = (*\alpha) \wedge (**\beta) = (-1)^{k(n-k)} (*\alpha)\wedge\beta = \beta\wedge (*\alpha) = \Scal \alpha\beta \mu\punto
	\end{equation*}
\end{proof}


\section{Forme differenziali su varietà}

Estendiamo l'algebra esterna al fibrato tangente di una varietà.
Se $\varphi: U \times V \to U' \times V'$ è un isomorfismo locale di fibrati, allora $\varphi_* : U\times \Lambda^k(V) \to U' \times \Lambda^k(V')$ è anch'esso un isomorfismo locale di fibrati.

\begin{definition}
	Sia $\pi : E\to B$ un fibrato vettoriale. Per ogni $A\subseteq B$ poniamo $\Lambda^k(E)\restrict{A} = \bigsqcup_{b\in A} \Lambda^k(E_b)$ e $\Lambda^k(E) = \Lambda^k(E) \restrict B$.
	Chiamiamo l'ovvia proiezione $\Lambda^k(\pi) : \Lambda^k(E) \to B$.
\end{definition}

Quando $E = TM$ per $M$ varietà, poniamo $\Lambda^k(M) = \Lambda^k(TM)$. Inoltre poniamo $\Omega^1(M) = \Tau^0_1(M) = \chi^*(M)$ e in generale $\Omega^k(M) = \Gamma^\infty(\Lambda^k(M))$ le sezioni $C^\infty$ su $\Lambda^k(M)$.

Estendiamo ora il prodotto wedge alle forme definite su una varietà.
\begin{proposition}
	Se $\alpha \in \Omega^k(M)$ e $\beta \in \Omega^l(M)$, sia $\alpha \wedge \beta : M \to \Lambda^{k+l}(M)$ definita da $(\alpha\wedge\beta)(m) = \alpha(m) \wedge \beta(m)$. Allora $\alpha\wedge\beta \in \Omega^{k+l}(M)$ e questo operatore è bilineare e associativo.
\end{proposition}
\begin{proof}
	Bilinearità e associatività seguono dalle proprietà del prodotto wedge $\wedge$ in $\Lambda(T_mM)$. La regolarità si mostra lavorando in carta.
\end{proof}

\begin{definition} \index{forma!differenziale} \index{algebra!delle forme differenziali}
	Elementi di $\Omega^k(M)$ sono detti \emph{$k$-forme} e $\Omega(M) = \oplus_{k=0}^n \Omega^k(M)$ è detta \emph{algebra delle forme differenziali esterne}.
\end{definition}

\begin{remark}
Vediamo la scrittura di una forma in coordinate locali. Se $t\in\Tau^r_s(M)$, abbiamo visto che $t = t^{\seqb ir{}}_{\seqb js{}} \DerParz{}{x^{i_1}} \otimes \ldots \otimes \DerParz{}{x^{i_r}} \otimes \de x^{j_1} \otimes \ldots \otimes \de x^{j_s}$.
Data $\omega \in \Omega^k(M)$, allora $\omega = \omega_{\seqb ik{}} \de x^{i_1} \wedge \ldots \wedge \de x^{i_k}$ con $\seqb ik<$ e $\omega_{\seqb ik{}} = \omega\left(\DerParz{}{x^{i_1}}, \ldots, \DerParz{}{x^{i_k}}\right)$.
\end{remark}

\begin{remark}
	Se $f:M\to N$ è regolare, è naturale considerare il pull-back $f^*: \Omega^k(N) \to \Omega^k(M)$.
\end{remark}

\section{Derivata esterna}

Studiamo ora una mappa $\de:\Omega(M) \to \Omega(M)$ che ha proprietà particolarmente interessanti.

\begin{theorem} \index{derivata!esterna} \label{thm:CostruzioneDerivataEsterna}
	Sia $M$ una varietà $n$-dimensionale. Allora per ogni $k=0,\ldots,n$ e per ogni $U\subseteq M$ aperto esiste unico $\de = \de^k : \Omega^k(U) \to \Omega^{k+1}(U)$ (detta \emph{derivata esterna}) tale che
	\begin{enumerate}
		\item $\de$ è una $\wedge$-antiderivazione, cioè $\de$ è $\R$-lineare e per ogni $\alpha\in\Omega^k(U)$ e $\beta\in\Omega^l(U)$ si ha $\de(\alpha\wedge\beta) = \de\alpha \wedge \beta +(-1)^k \alpha\wedge \de \beta$; \label{cde:Antiderivazione}
		\item se $f\in C^\infty(M)$, allora $\de f$ coincide con il differenziale di $f$; \label{cde:Cinfinito}
		\item $\de^2 = \de\circ \de = 0$; \label{cde:DDUgualeAZero}
		\item $\de$ è naturale rispetto alle restrizioni, cioè se $U\subseteq V \subseteq M$ con $U,V$ aperti e $\alpha\in\Omega^k(V)$, allora $\de(\alpha\restrict U) = (\de\alpha)\restrict U$ (quindi $\de$ è locale). \label{cde:Locale}
	\end{enumerate}
\end{theorem}
\begin{proof}
	\begin{description}
		\item [unicità:] Sia $(U,\varphi)$ una carta di $M$ e consideriamo una $k$-forma $\alpha\in\Omega^k(U)$ con $\alpha = \alpha_{\seqb ik{}} \de x^{i_1} \wedge \ldots \wedge \de x^{i_k}$.
		Se $k=0$ e $f=x^i$, allora $\de f = \de(x^i) = \de x^i$.
		
		Dalla \ref{cde:DDUgualeAZero} abbiamo $\de(\de x^i) = 0$, quindi per la \ref{cde:Antiderivazione} otteniamo $\de(\de x^{i_1} \wedge \ldots \wedge \de x^{i_k}) = 0$.
		
		Perciò $\de\alpha = (\de \alpha_{\seqb ik{}}) \wedge \de x^{i_1} \wedge \ldots\wedge \de x^{i_k} = \DerParz{\alpha_{\seqb ik{}}} {x^i} \de x^i \wedge \de x^{i_1} \wedge\ldots \wedge \de x^{i_k}$, con $i=1,\ldots,n$ e $\seqb ik<$.
		
		Questa formula caratterizza $\de$ su tutti gli aperti di un atlante di $M$, quindi per la \ref{cde:Locale} su tutti gli aperti di $M$.
		
		\item [esistenza:] Definiamo $\de$ come dalla formula sopra e verifichiamo tutte le proprietà. Innanzitutto $\de$ è ovviamente $\R$-lineare.
		
		Siano ora $\alpha\in\Omega^k(U)$ e $\beta\in\Omega^l(U)$ con $\alpha = \alpha_{\seqb ik{}} \de x^{i_1} \wedge \ldots \wedge \de x^{i_k}$ e $\beta= \beta_{\seqb jl{}} \de x^{j_1} \wedge\ldots\wedge \de x^{j_l}$. Allora
		\begin{align*}
			\de& (\alpha \wedge \beta) =\\ 
			&= \de (\alpha_{\seqb ik{}} \beta_{\seqb jl{}} \de x^{i_1} \wedge\ldots\wedge \de x^{i_k}\wedge \de x^{j_1} \wedge\ldots\wedge \de x^{j_l}) =\\
			&= \left(\DerParz{\alpha_{\seqb ik{}}}{x^i} \beta_{\seqb jl{}} + \alpha_{\seqb ik{}} \DerParz{\beta_{\seqb jl{}}}{x^i} \right) \de x^i\wedge\de x^{i_1} \wedge\ldots\wedge \de x^{i_k}\wedge \de x^{j_1} \wedge\ldots\wedge \de x^{j_l} = \\
			&=\de \alpha \wedge \beta + (-1)^k \alpha \wedge \de \beta \punto
		\end{align*}
		
		Per la \ref{cde:DDUgualeAZero}, utilizzando il lemma di Schwarz, abbiamo
		\begin{equation*}
		\de(\de\alpha) = \frac{\partial^2 \alpha_{\seqb ik{}}}{\partial x^i \partial x^j} \de x^i\wedge\de x^j \wedge\de x^{i_1} \wedge\ldots\wedge \de x^{i_k} =0 \punto
		\end{equation*}
		
		Per verificare \ref{cde:Locale} mostriamo che la definizione non dipende dalle carte. Siano $(U,\varphi)$, $(U',\varphi')$ carte tali che $U\cap U' \not=\emptyset$. Per unicità locale $\de = \de'$ in $U\cap U'$ e quindi abbiamo anche l'unicità globale.
	\end{description}
\end{proof}

\begin{corollary}
	Sia $\omega \in \Omega^k(U)$ con $U\subseteq \R^n$. Allora
	\begin{equation*}
		\de \omega (x) (v_0,\ldots,v_k) = \sum_{i=0}^k (-1)^i\ \Diff \omega (x)\cdot v_i\ (v_0,\ldots,\hat v_i, \ldots, v_k)\virgola
	\end{equation*}
	dove $\Diff \omega$ è il differenziale classico.
\end{corollary}
\begin{proof}
	Sia $\omega(x) = \omega_{\seqb ik{}} (x) \de x^{i_1}\wedge\ldots\wedge \de x^{i_k}$, allora
	\begin{equation*}
		\Diff \omega(x)\cdot v_i = \DerParz{\omega_{\seqb ik{}}}{x^j}\ v_i^j\ \de x^{i_1}\wedge\ldots\wedge \de x^{i_k} \virgola
	\end{equation*}
	quindi il membro destro dell'enunciato è
	\begin{equation*}
		\sum_{\eta\in S_k}(-1)^i \DerParz{\omega_{\seqb ik{}}}{x^j}\ v_i^j\ \sgn(\eta)\ v_0^{\eta(i_1)}\ldots \hat v_i \ldots v_k^{\eta(i_k)} \punto
	\end{equation*}
	D'altra parte
	\begin{align*}
		\de \omega (v_0,\ldots,v_k) &= \DerParz{\omega_{\seqb ik{}}}{x^j} \de x^j \wedge \de x^{i_1}\wedge\ldots\wedge \de x^{i_k} (v_0,\ldots,v_k)= \\ 
		&=\sum_{\sigma \in S_{k+1}}\DerParz{\omega_{\seqb ik{}}}{x^j}\ \sgn(\sigma)\ v_0^{\sigma(j)} v_1^{\sigma(i_1)}\ldots v_k^{\sigma(i_k)} \punto
	\end{align*}
	A questo punto però è facile verificare che le due espressioni trovate coincidono, passando da una permutazione $\eta \in S_k$ ad una permutazione $\sigma \in S_{k+1}$ mandando lo 0 in $i$.
\end{proof}

\begin{example}
	\begin{enumerate}
	 \item In $\R^2$, sia $\alpha = f(x,y) \de x + g(x,y) \de y$, allora $\de \alpha= \de f\wedge \de x + \de g \wedge \de y = \left(\DerParz{g}{x} - \DerParz fy\right) \de x\wedge \de y $.
	 \item In $\R^3$, data $f$ funzione regolare, abbiamo $\de f = \DerParz fx \de x + \DerParz fy \de y + \DerParz fz \de z$ e $\grad f = (\de f)^\sharp$.
	\end{enumerate}
\end{example}

\begin{exercise}
	In $\R^3$, sia $\underline F=(F_1,F_2,F_3)$, allora $\underline F^\flat = F_1 \de x + F_2 \de y + F_3 \de z$ e
	$*(\underline F^\flat) = F_3\de x \wedge \de y - F_2 \de x \wedge \de z + F_1 \de y \wedge \de z$.
	Mostrare che $\de \underline{F}^\flat = *(\mathrm{rot} \underline{F})^\flat$.	
	Verificare che $\div \underline F = *\de(* \underline F^\flat)$ e $\de^2 = 0$ e quindi che $\div (\mathrm{rot} \underline F)=0$.
	
	Allo stesso modo, se $f:\R^3\to\R$ è regolare $\de\de f=0$ equivale a $\mathrm{rot}(\grad f)=0$.
\end{exercise}



\begin{theorem} \label{thm:PullBackDerivataEsterna}
	Sia $f:M\to N$ regolare. Allora, per ogni $\omega \in \Omega^k(N)$ e $\psi \in \Omega^l(N)$, $f^*$ soddisfa:
	\begin{enumerate}
		\item $f^*(\omega \wedge \psi) = f^*(\omega) \wedge f^*(\psi)$; \label{pbde:Wedge}
		\item $f^*(\de \omega) = \de (f^*\omega)$. \label{pbde:DerivataEsterna}
	\end{enumerate}
\end{theorem}
\begin{proof}
	La \ref{pbde:Wedge} è già vista, per esempio come conseguenza del punto \ref{pbf:Composizione} della \cref{prop:PullBackForme}. Quindi dimostriamo il punto \ref{pbde:DerivataEsterna}.
	
	Sia $(U,\varphi)$ carta di $M$ e $(V,\rho)$ carta di $N$ tali che $f(U)\subseteq V$. Sia $\omega = \omega_{\seqb ik{}} \de x^{i_1}\wedge\ldots\wedge \de x^{i_k}$, allora
	\begin{equation*}
		\de \omega = \DerParz{\omega_{\seqb ik{}}}{x_i} \de x^i \wedge \de x^{i_1} \wedge \ldots \wedge \de x^{i_k} \puntovirgola
	\end{equation*}
	inoltre per la \ref{pbde:Wedge} abbiamo
	\begin{equation*}
		f^*\omega = f^*(\omega_{\seqb ik{}}) f^*(\de x^{i_1}) \wedge \ldots \wedge f^* (\de x^{i_k})\punto
	\end{equation*}
	Se $\psi \in C^\infty(N)$, $\de (f^*\psi) = f^* (\de \psi)$. Quindi usando nuovamente la \ref{pbde:Wedge} e $\de\circ\de =0$, abbiamo
	\begin{equation*}
		\de (f^*\omega) = f^*(\de \omega_{\seqb ik{}})\wedge f^*\de x^{i_1} \wedge \ldots \wedge f^*\de x^{i_k} = f^*(\de\omega)\punto
	\end{equation*}
\end{proof}

\begin{corollary}
	Se $f$ è un diffeomorfismo, allora $f_*(\de\omega) = \de (f_* \omega)$.
\end{corollary}

\begin{corollary} \label{cor:CommutazioneLieDerivataEsterna}
	Se $X\in\chi(M)$, $\omega \in \Omega^k(M)$, allora $\Lie_X\omega \in \Omega^k(M)$ e $\de (\Lie_X\omega) = \Lie_X \de \omega$.
\end{corollary}

\begin{proof}
Per definizione $\Lie_X\omega(p) = \frac{\de}{\de t} ((\Fl Xt)^*\omega)(p)\restrict{t=0}$, dove $\Fl Xt$ è il flusso generato da $X$.
Perciò, visto che $(\Fl Xt)^*\omega\in \Omega^k(M)$, otteniamo che $\Lie_X\omega \in \Omega^k(M)$.

Inoltre $\de$ commuta con il pull-back, allora $(\Fl Xt)^* (\de\omega) = \de ((\Fl Xt)^* \omega)$ e passando al limite nei rapporti incrementali otteniamo proprio $\de \Lie_X = \Lie_X \de$.
\end{proof}


\section{Prodotto interno}

Ricalcando la \cref{def:ProdottoInternoTensore}, estendiamo il concetto di prodotto interno alle forme.
\begin{definition}
	Sia $M$ una varietà, $X\in\chi(M)$ e $\omega\in\Omega^{k+1}(M)$. Definiamo il \emph{prodotto interno} o \emph{contrazione} di $X$ e $\omega$ come $i_X\omega \in \Tau_k^0(M)$ tale che $i_X\omega(\seqb Xk,) = \omega(X,\seqb Xk,)$. %Chiamiamo $i_X\omega$ \emph{prodotto interno} o \emph{contrazione} di $X$ e $\omega$.
\end{definition}

\begin{remark}
	Se $\omega\in C^\infty(M)$, allora $i_X\omega = 0$.
\end{remark}

\begin{theorem} \label{thm:ProprietaProdottoInterno}
	Dato $X\in\chi(M)$, $i_X$ manda $\Omega^{k+1}(M)$ in $\Omega^k(M)$. Inoltre, se $\alpha\in\Omega^k(M)$, $\beta\in\Omega^l(M)$, $f\in C^\infty(M)$, allora
	\begin{enumerate}
		\item $i_X$ è $\R$-lineare e $i_X(\alpha\wedge\beta) = (i_X\alpha)\wedge \beta + (-1)^k \alpha \wedge (i_X\beta)$; \label{ppi:Wedge}
		\item $i_{fX}\alpha = f\ i_X\alpha$; \label{ppi:Cinfinito}
		\item $i_X\de f = \Lie_X f$; \label{ppi:DeCinfinito}
		\item $\Lie_X(\alpha\wedge\beta) = (\Lie_X \alpha) \wedge \beta + \alpha \wedge (\Lie_X \beta)$; \label{ppi:LieWedge}
		\item \emph{(formula di Cartan)} $\Lie_X \alpha = i_X \de\alpha + \de( i_X\alpha)$; \label{ppi:DeForma}\index{formula!di Cartan} %TODO: sarebbe bello linkare a lei direttamente, magari dargli la dignità di teorema
		\item $\Lie_{fX}\alpha = f\Lie_X\alpha + \de f \wedge i_X\alpha$. \label{ppi:LiePerCinfinito}
	\end{enumerate}
\end{theorem}
\begin{proof}
	Il fatto che $i_X$ mandi $\Omega^{k+1}(M)$ in $\Omega^k(M)$ e che $i_X$ sia $\R$-lineare sono ovvi.
	
	%1) $i_X(\alpha \wedge \beta) (X_2,X_3,\ldots,X_{k+l}) = (\alpha\wedge\beta) (X, X_2,X_3,\ldots,X_{k+l})$.
	%D'altra parte
	%\begin{equation*}
	%(i_X\alpha\wedge \beta) + (-1)^k \alpha \wedge (i_X\beta) = \frac{(k+l-1)!}{(k-1)!\ l!}\ \antisimm (i_X\alpha \otimes \beta) + %(-1)^k \frac{(k+l-1)!}{k!\ (l-1)!} \ \antisimm (\alpha \otimes i_X\beta) \punto
	%\end{equation*}
	%Scriviamo il secondo termine come somma di permutazioni $\sigma \in S_{k+l-1}$. Scriviamo $\sigma = \tau \circ \sigma_0$ con $\tau %\in S_{k+l-1}$ e $\sigma_0(2,3,\ldots, k+1, 1, k+2, \ldots, k+l) = (1,2,3,\ldots, k+l)$.
	
	%\begin{align*}
	%	\antisimm (\alpha \otimes i_X\beta) (v_2,\ldots,v_{k+l}) &= \frac 1{(k+l-1)!} \sum_{\sigma\in S_{k-1}}\sgn(\sigma) (\alpha \otimes i_X\beta) (v_{\sigma(2)},\ldots,v_{\sigma(k+l)})=\\
	%	&= \frac 1{(k+l-1)!} \sum_{\sigma\in S_{k-1}}\sgn(\sigma) \alpha(v_{\sigma(2)},\ldots,v_{\sigma(k+1)})\beta(X,v_{\sigma(k+2)},\ldots,v_{\sigma(k+l)})=
	%\end{align*}
	%e poi boh... continua la prossima volta.
	
	%Rivediamola bene:
	
% 	Per quanto riguarda la seconda parte della \ref{ppi:Wedge} abbiamo che
% 	\begin{equation*}
% 		\alpha\wedge\beta (\seqb e{k+l},) = \sum_{\sigma \in (k,l)-mescolamenti} \sgn(\sigma) \alpha(e_{\sigma(1)}\ldots e_{\sigma(k)}) \beta(e_{\sigma(k+1)}\ldots e_{\sigma(k+l)})
% 	\end{equation*}
	
	Introduciamo innanzitutto una notazione. Dati due multi-indici $I$ e $J$ indichiamo con
	\begin{equation*}
		\delta_J^I = \begin{cases}
		             	1 & \text{se } J \text{ è una permutazione pari di }I\\
		             	-1 & \text{se } J \text{ è una permutazione dispari di }I\\
		             	0 & \text{altrimenti}
		             \end{cases}
	\end{equation*}
	Allora, con questa notazione, vale
	\begin{equation*}
		(\alpha\wedge\beta)(v_I) = \sum_{\vec J, \vec K} \delta_I^{JK} \alpha(v_J)\beta(v_K)
	\end{equation*}
	con $\vec J,\vec K$ sottoinsiemi rispettivamente di $k$ ed $l$ indici ordinati. Allora
 	\begin{align*}
 		i_{v_1}&(\alpha\wedge\beta)(v_2,\ldots, v_{k+l}) = (\alpha\wedge\beta) (\seqb v{k+l},) =
 		\sum_{\vec I,\vec J} \delta_{1\ldots k+l}^{IJ} \alpha(v_I)\beta(v_J) =\\
 		&= \sum_{\vec I,\vec J,1\in I} \delta_{1 \ldots k+l}^{IJ} \alpha(v_I)\beta(v_J) + \sum_{\vec I,\vec J,1\in J} \delta_{1\ldots k+l}^{IJ} \alpha(v_I)\beta(v_J) =\\
 		&= \sum_{1<i_2<\ldots < i_k,\vec J} \delta_{1 \ldots k+l}^{1i_2 \ldots i_kJ}\alpha(v_1,v_{i_2},\ldots,v_{i_k}) \beta(v_J) +\\
 		& +\sum_{\vec I, 1<j_2<\ldots < j_l} \delta_{1 \ldots k+l}^{I1j_2 \ldots j_l}\alpha(v_I) \beta(v_1,v_{j_2},\ldots,v_{j_l}) =\\
 		& = \sum_{1<i_2<\ldots<i_k} \sum_{\vec J\setminus \{1\}} \delta_{2\ldots k+l}^{i_2\ldots i_kJ} (i_{v_1}\alpha)(v_{i_2},\ldots,v_{i_k}) \beta (v_J) +\\
 		& +(-1)^k \sum_{\vec I\setminus \{1\}} \sum_{1<j_2<\ldots<j_l} \delta_{2\ldots k+l}^{Ij_2\ldots j_l}\ \alpha(v_I) (i_{v_1}\beta)(v_{j_2}, \ldots, v_{j_l}) =\\
 		& = (i_{v_1}\alpha \wedge \beta)(v_2, \ldots, v_{k+l}) + (-1)^k (\alpha \wedge i_{v_1}\beta)(v_2, \ldots, v_{k+l}) \punto
 	\end{align*}
 	
 	La \ref{ppi:Cinfinito} segue dal fatto che $\alpha$ è $C^\infty(M)$-multilineare.	
	La \ref{ppi:DeCinfinito} segue dalla definizione di $\de f$, mentre la \ref{ppi:LieWedge} segue dal fatto che la derivata di Lie è una derivazione tensoriale e commuta con la mappa alternante.
	
	Vediamo quindi la \ref{ppi:DeForma} tramite un'induzione in $k$. Il risultato è vero per $k=0$ per la \ref{ppi:DeCinfinito}, supponiamo ora che valga per le $k$ forme e dimostriamolo per le $k+1$ forme. Ogni $k+1$ forma si scrive come $\sum \de f_i \wedge \omega_i$ dove $f_i \in C^\infty(M)$ e $\omega_i\in\Omega^k(M)$; infatti data $\omega$ una $(k+1)$-forma, abbiamo
	\begin{equation*}
	\omega = \omega_{\seqb i{k+1}{}} \de x^{i_1} \wedge \ldots \wedge \de x^{i_{k+1}} = \de x^{i_1} \wedge (\omega_{\seqb i{k+1}{}} \de x^{i_2} \wedge \ldots \wedge \de x^{i_{k+1}}) \punto
	\end{equation*}
	Utilizzando quindi la \ref{ppi:DeCinfinito} e l'ipotesi induttiva, abbiamo
	\begin{align*}
	i_X & \de (\de f  \wedge  \omega) + \de i_X(\de f \wedge \omega) = -i_X(\de f\wedge \de \omega) + \de (i_X(\de f) \wedge \omega - \de f \wedge i_X\omega) =\\
	&= -i_X(\de f)\wedge \de \omega + \de f\wedge i_X (\de \omega) + \de (i_X (\de f)) \wedge \omega + i_X (\de f) \wedge \de \omega + \de f \wedge \de (i_X \omega) =\\
	&= \de f(i_X(\de\omega) + \de (i_X\omega)) + \de(i_X(\de f)) \wedge \omega=\de f \wedge \Lie_X\omega + \de (\Lie_Xf) \wedge \omega \punto
	\end{align*}
	Sfruttando infine che per il punto \ref{ppi:LieWedge} abbiamo $\Lie_X(\de f\wedge\omega) = (\Lie_X\de f)\wedge \omega + \de f \wedge (\Lie_X\omega)$ e che per il \cref{cor:CommutazioneLieDerivataEsterna} vale $\de(\Lie_Xf) = \Lie_X \de f$, otteniamo il risultato.
	
	Vediamo infine che il punto \ref{ppi:LiePerCinfinito} segue dalla \ref{ppi:DeForma}, poiché
	\begin{align*}
	\Lie_{fX}\alpha &= i_{fX} \de \alpha + \de (i_{fX} \alpha) = f i_X \de \alpha + \de (fi_X\alpha) =\\
	&=f i_X\de \alpha + \de f \wedge i_X\alpha + f \de i_X\alpha = f\Lie_X \alpha + \de f \wedge i_X\alpha \punto
	\end{align*}
\end{proof}

\begin{proposition}
	Sia $f:M\to N$ e siano $\omega \in \Omega^k(N)$ e $Y \in \chi(M)$. Allora $i_Y$ è naturale rispetto al push-forward, cioè $i_Yf^*\omega = f^* i_{f_*Y}\omega$.
\end{proposition}
\begin{proof}
	Chiamiamo $X = f_*Y \in \chi(N)$ e siano $p\in M$, $q=f(p)\in N$ e $\seqb v{k-1}, \in T_pM$. Allora
	\begin{align*}
		i_Y(f^*\omega)(p)(\seqb v{k-1},) &= f^* \omega (p) (Y(p), \seqb v{k-1},) =\\
		&=\omega(q) (f_*(Y(p)), f_*v_1,\ldots, f_*v_{k-1}) = \\
		&= \omega(q)((X\circ f)(p),f_*v_1,\ldots, f_*v_{k-1}) =\\
		&=(i_X\omega)(q) (f_*v_1,\ldots, f_*v_{k-1})  =(f^*i_X\omega) (p) (\seqb v{k-1},) \punto 
	\end{align*}
\end{proof}

\begin{proposition} \label{prop:DeForma}
	Siano $X_0,X_1,\ldots,X_k\in\chi(M)$ e $\omega \in \Omega^k(M)$. Allora
	\begin{align*}
		\de \omega (X_0,X_1,\ldots,X_k) =& \sum_{l=0}^k (-1)^l\Lie_{X_l}(\omega(X_0,\ldots, \hat X_l, \ldots, X_k)) + \\
		&+\sum_{0\le i<j\le k}(-1)^{i+j} \omega (\Lie_{X_i}(X_j),X_0,\ldots,\hat X_i,\ldots, \hat X_j,\ldots, X_k) \punto
	\end{align*}

% 	\begin{enumerate}
% 		\item $(\Lie_{X_0} \omega) (\seqb Xk,) = \Lie_{X_0}(\omega(\seqb Xk,)) - \sum_{i=1}^k \omega(X_1,\ldots, \Lie_{X_0}X_i, \ldots, X_k)$;
% 		\item $\de \omega (X_0,X_1,\ldots,X_k) = \sum_{l=0}^k (-1)^l\Lie_{X_l}(\omega(X_0,\ldots, \hat X_l, \ldots, X_k)) + \\
% 		+\sum_{0\le i<j\le k}(-1)^{i+j} \omega (\Lie_{X_i}(X_j),X_0,\ldots,\hat X_i,\ldots, \hat X_j,\ldots, X_k)$
% 	\end{enumerate}
\end{proposition}
\begin{proof}
	Procediamo per l'induzione su $k$.
	Per $k=0$ abbiamo già visto nel \cref{thm:ProprietaProdottoInterno} che $\de \omega (X_0) = \Lie_{X_0} \omega$.
	
	Supponiamo quindi che la tesi valga per $k-1$ e sia $\omega \in \Omega^k(M)$. Per il punto \ref{ppi:DeForma} del \cref{thm:ProprietaProdottoInterno}, abbiamo
	\begin{align*}
		\de \omega  (X_0, &\seqb Xk,) = (i_{X_0}\de \omega)(\seqb Xk,) =\\
		=&\Lie_{X_0}\omega(\seqb Xk,) - \de (i_{X_0} \omega) (\seqb Xk,) =\\
		=& \Lie_{X_0}(\omega(\seqb Xk,)) - \sum_{l=1}^k \omega (X_1,\ldots,\Lie_{X_0}X_l,\ldots, X_k)-\de (i_{X_0}\omega) (\seqb Xk,) \virgola
	\end{align*}
	ma, visto che $i_{X_0}\omega \in \Omega^{k-1}(M)$, per ipotesi induttiva vale
	\begin{align*}
		\de (i_{X_0}\omega) (X_1,\ldots & ,X_k) =\\
		=&\sum_{l=1}^{k} (-1)^{l-1} \Lie_{X_l}(i_{X_0}\omega (X_1,\ldots, \hat X_l, \ldots, X_k)) +\\
		&+\sum_{1\le i< j \le k} (-1)^{i+j-2} (i_{X_0}\omega)(\Lie_{X_i}(X_j),X_1,\ldots, \hat X_i, \ldots, \hat X_j,\ldots, X_k) =\\
		=& \sum_{l=1}^{k} (-1)^{l-1} \Lie_{X_l}(\omega (X_0,X_1,\ldots, \hat X_l, \ldots, X_k)) +\\
		&+\sum_{1\le i< j \le k} (-1)^{i+j-2} \omega(X_0,\Lie_{X_i}(X_j),X_1,\ldots, \hat X_i, \ldots, \hat X_j,\ldots, X_k)
	\end{align*}
	Quindi sostituendo nell'equazione precedente
	\begin{align*}
		\de \omega (X_0,\seqb Xk,) =&\sum_{l=0}^{k} (-1)^l \Lie_{X_l}(\omega(X_0, \ldots, \hat X_l,\ldots, X_k)) +\\
		&+\sum_{l=1}^k(-1)^l\omega(\Lie_{X_0}X_l,\ldots X_1,\ldots,\hat X_l,\ldots ,X_k)+\\
		&+\sum_{1\le i<j \le k} (-1)^{i+j} \omega (\Lie_{X_i}(X_j),X_0,X_1,\ldots,\hat X_i,\ldots,\hat X_j,\ldots,X_k) \virgola
	\end{align*}
	da cui segue ovviamente quanto cercato.
\end{proof}

\begin{corollary}
	Siano $X,Y\in\chi(M)$, allora $\comm{\Lie_X}{i_Y} \coloneqq \Lie_X i_Y - i_Y \Lie_X = i_{\comm XY}$ e $\comm{\Lie_X}{\Lie_Y}\coloneqq \Lie_X\Lie_Y - \Lie_Y\Lie_X = \Lie_{\comm XY}$.
	Quindi in particolare $i_X\Lie_X = \Lie_X i_X$.
\end{corollary}
\begin{proof}
	Sia $\omega \in \Omega^k(U)$ e $\seqb X{k-1}, \in \chi(U)$, allora
	\begin{align*}
		(i_Y\Lie_X\omega)(\seqb X{k-1},) =& (\Lie_X\omega)(Y,\seqb X{k-1},) =\\
		=&\Lie_X(\omega(Y,\seqb X{k-1},)) -\omega (\comm XY, \seqb X{k-1},) - \\
		&-\sum_{l=1}^{k-1}\omega(Y,X_1,\ldots,\comm{X}{X_l},\ldots X_{k-1}) =\\
		=& \Lie_X(i_Y\omega(\seqb X{k-1},)) -i_{\comm XY} \omega (\seqb X{k-1},) - \\
		&-\sum_{l=1}^{k-1} i_Y\omega(X_1,\ldots, \comm X{X_l}, \ldots, X_{k-1}) =\\
		=& (\Lie_X i_Y\omega)(\seqb X{k-1},) - (i_{\comm XY}\omega)(\seqb X{k-1},) \punto
	\end{align*}
	Per la seconda equazione si usa invece la prima e il punto \ref{ppi:DeForma} del \cref{thm:ProprietaProdottoInterno}.
\end{proof}

