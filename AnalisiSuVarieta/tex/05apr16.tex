\chapter{5 aprile 2016}

\section{Omomorfismi fra gruppi di Lie}

Siano $G,H$ gruppi di Lie e $\phi : G\to H$ un omomorfismo di classe $C^\infty$.
Allora è definito $\phi_* : T_eG \to T_eH$.
Per ogni $a \in G$, $\phi \circ L_a = L_{\phi(a)} \circ \phi$.

Se $X \in T_eG$ definiamo $\tilde{\tilde X}$ come l'estensione in $\chi(H)$ invariante a sinistra di $(\phi_*)_eX $ che vale $\phi_*X$ in $e_H$.
Allora $(\phi_*)_a \tilde X(a) = (\phi_*)_a (L_a)_*X = (L_{\phi(a)})_* (\phi_*)_eX = \tilde{\tilde X} (\phi(a))$ con $\tilde X\in\chi(G)$ invariante a sinistra tale che $\tilde X(e) = X$.

Quindi $\tilde{\tilde X} = \phi_*\tilde X$ e perciò $(\phi_*)_e:\mathfrak g \to \mathfrak h$ è un omomorfismo di algebra di Lie, cioè una mappa lineare che soddisfa $(\phi_*)_e \comm XY _{\mathfrak g} = \comm{ (\phi_*)_eX}{ (\phi_*)_eY }_{\mathfrak h}$.

\begin{theorem}\label{Teo:OmomorfismiAlgebreLie}
	Siano $G,H$ gruppi di Lie e $\Phi: \mathfrak g \to \mathfrak h$ un omomorfismo di algebre di Lie. Allora esiste $U$ intorno di $e$ in $G$ e una mappa $\phi: U \to H$ di classe $C^\infty$ tale che $\phi(ab) = \phi(a) \phi(b)$ per ogni $a,b \in U$ con $ab \in U$ e tale che $(\phi_*)_e X = \Phi X$ per ogni $X\in\mathfrak g$.
	Inoltre se esistono $\phi,\psi : G \to H$ omomorfismi $C^\infty$ tali che $(\phi_*)_e = (\psi_*)_e = \Phi$ e se $G$ è connesso, allora $\phi = \psi$.
\end{theorem}
\begin{proof}
	Sia $\mathfrak f \subseteq \mathfrak g \times \mathfrak h$ l'insieme $\mathfrak f = \{ (X,\Phi(X)), X \in \mathfrak g \}$.
	Dato che $\Phi$ è un omomorfismo di algebre di Lie, allora $\mathfrak f$ è una sottoalgebra di $\mathfrak g \times \mathfrak h = \Lie(G\times H)$.
	Quindi esiste $K$ sottogruppo di Lie di $G\times H$ tale che $\Lie(K) = \mathfrak f$.
	Sia $\pi_1:G\times H \to G$ la proiezione sul primo fattore e $\omega = \pi_1\restrict K$.
	Allora $\omega: K \to G$ è un omomorfismo.
	Dato $X\in \mathfrak g$, $\omega_*(X,\Phi(X)) = X$, quindi $\omega_*:T_{(e,e)}K\to T_e G$ è un isomorfismo.
	Quindi esiste $V$ intorno di $(e,e) \in K$ tale che $\omega$ mappa $V$ diffeomorficamente su $U$ intorno di $e\in G$.
	
	Se $\pi_2 : G\times H \to H$ è la proiezione su $H$, definiamo $\phi \coloneqq \pi_2 \circ \omega^{-1}$ su $U$. Quindi $\phi$ realizza la prima condizione. Per la seconda, sia $X\in\mathfrak g$, abbiamo $\omega(X,\Phi(X)) = X$, allora $\phi_*X = (\pi_2)_*(X,\Phi(X)) =\Phi(X)$.
	
	Dati $\phi,\psi:G\to H$ omomorfismi, definiamo $\theta:G \to G\times H$ iniettivo come $\theta(a) = (a,\psi(a))$. L'immagine $G'$ di $\theta$ è un sottogruppo di Lie di $G\times H$ e dato $X \in \mathfrak g$ vale $\theta_*(X) = (X,\Phi(X))$, allora $\Lie(G') = \mathfrak f$, perciò $G'=K$ e di conseguenza $\psi(a) = \phi(a)$ per ogni $a \in G$. 
\end{proof}



\begin{corollary}
	Se due gruppi di Lie hanno algebre di Lie isomorfe, allora sono localmente isomorfi.
\end{corollary}
\begin{proof}
	Dato $\Phi: \mathfrak g \to \mathfrak h$ isomorfismo, sia $\phi$ la mappa data dal \cref{Teo:OmomorfismiAlgebreLie}
	Questa soddisfa $(\phi_*)_e = \Phi$ isomorfismo, quindi $\phi$ è un diffeomorfismo in un intorno di $e \in G$.
\end{proof}



\begin{corollary}
	Sia $G$ un gruppo di Lie connesso con algebra di Lie commutativa, allora $G$ è abeliano.
\end{corollary}
\begin{proof}
	Per il corollario precedente, $G$ è localmente isomorfo ad $\R^n$ e quindi è commutativo in un intorno di $e \in G$. Si vede che, dalla connessione, ogni intorno $U$ di $e$ genera $G$ tramite le operazioni di gruppo; perciò anche $G$ è abeliano.
\end{proof}


\begin{definition}
	Sia $G$ un gruppo di Lie. Un omomorfismo $\phi : \R \to G$ è detto un \emph{sottogruppo ad un parametro} di $G$.
\end{definition}

% Abbiamo visto che, dato $X \in T_eG \setminus \{0\}$, esiste unico un sottogruppo ad un parametro $\phi(t)$ tale che $\frac{\de\phi}{\de t} = \tilde X(\phi(t))$. 

\begin{corollary}
	Per ogni $X \in T_eG$, esiste un unico sottogruppo ad un parametro $\phi: \R \to G$ tale che $\frac{\de\phi}{\de t}\restrict{t=0} = X$.
\end{corollary}
\begin{proof}
	Sia $f:G\to \R$ e se $\phi: \R \to G$ è un omomorfismo $C^\infty$ tale che $\frac{\de\phi}{\de t}\restrict{t=0} = X$, abbiamo che
	\begin{align*}
		\frac{\de\phi}{\de t}(f) &= \lim_{h\to\infty} \frac{f(\phi(t+h))-f(\phi(t))}{h} = \lim_{h\to\infty} \frac{f(\phi(t)\phi(h))-f(\phi(t))}{h} =\\
		&=\frac{\de}{\de s}\restrict{s=0} f\circ L_{\phi(t)} \circ \phi(s) = (L_{\phi(t)})_* \frac{\de\phi}{\de s}\restrict{s=0} (f) = (( L_{\phi(t)} )_* X) f = \tilde X(\phi(t)) (f) \punto
	\end{align*}
	Quindi se esiste un tale omomorfismo $\phi$, deve essere una curva integrale di $\tilde X$ (estensione di $X$ a $G$), abbiamo quindi l'unicità.
	
	Viceversa, sia $\phi: \R \to G$ una curva integrale di $\tilde X$ e consideriamo (fissato $s$) la mappa $t\mapsto \phi(s)\cdot \phi(t)$.
	Questa è una curva integrale di $\tilde X$ che passa per $\phi(s)$ al tempo 0.
	Però la cosa vale anche per $\phi(\cdot + s)$, perciò $\phi(t+s) = \phi(s) \phi(t)$ per unicità.
\end{proof}

\begin{remark}
	Le curve integrali esistono localmente, ma usando le proprietà di gruppo  si può mostrare l'esistenza globale.
\end{remark}


\begin{definition}
	Siano $G$ un gruppo di Lie e $\phi$ definita come sopra (cioè $\frac{\de\phi}{\de t}(0) =X$). Definiamo $\exp: \mathfrak g \to G$ tramite $\exp(X) \coloneqq \phi(1)$.
\end{definition}

Questa mappa soddisfa $\exp((t_1+t_2)X) = \exp(t_1X) \exp(t_2X)$ e $\exp(-tX) = \exp(tX)^{-1}$

\begin{proposition}
	La mappa $\exp: \mathfrak g \to G$ è di classe $C^\infty$ e 0 è un punto regolare, cioè esiste $U$ intorno di 0 in $T_eG$ tale che $\exp\restrict U$ è un diffeomorfismo su un intorno di $e\in G$.
	
	Inoltre, se $\psi : G \to H$ è un omomorfismo $C^\infty$, allora $\exp\circ \psi_* = \psi \circ \exp$.
\end{proposition}
\begin{proof}
	Dati $X \in T_eG$ e $a \in G$, consideriamo $T_{(X,a)}(T_eG\times G)$ che si identifica con $T_eG \times T_a G$.
	Definiamo un campo vettoriale $Y \in \chi(T_eG \times G)$ tramite $Y_{(X,a)} = 0 \oplus \tilde X(a)$.
	Allora $Y$ genera un flusso $\alpha: \R \times (T_eG \times G) \to T_eG \times G$ di classe $C^\infty$. Si ha che $\exp X$ è la proiezione su $G$ di $\alpha(1,0\oplus X)$, quindi $\exp$ è di classe $C^\infty$.
	
	Dato $v\in T_0(T_eG)$, questo si identifica con un vettore in $T_eG$ e la curva $c(t) = tv$ (in $T_eG$) ha vettore tangente $v$ in 0.
	Quindi $(\exp_*)_0 (v) = \frac{\de \exp(c(t))}{\de t}\restrict {t=0} = \frac{\de}{\de t} \restrict {t=0} \exp(tv)$, perciò $(\exp_*)_0$ è l'identità e di conseguenza $\exp$ è un diffeomorfismo in un intorno di $0\in T_eG$.
	
	Sia $\psi: G \to H$ un omomorfismo e sia $\phi: \R \to G$ omomorfismo tale che $\frac{\de \phi}{\de t}\restrict{t=0} = X \in \mathfrak g$.
	Allora $\psi \circ \phi$ è un omomorfismo e 
	\begin{equation*}
		\frac{\de(\psi\circ \psi)}{\de t}\restrict{t=0} = \psi_*(\frac{\de \phi}{\de t}\restrict{t=0}) = \psi_*X\punto
	\end{equation*}
	Quindi $\exp(\psi_*X) = (\psi\circ\phi)(1) = \psi(\phi(1)) = \psi(\exp(X))$.
\end{proof}


\begin{corollary}
	Ogni omomorfismo iniettivo $\phi : G \to H$ è un'immersione fra varietà. In particolare l'immagine è un sottogruppo di Lie di $H$.
\end{corollary}
\begin{proof}
	Per assurdo supponiamo che $(\phi_*)_p(\tilde X(p)) = 0$ per qualche $X \in \mathfrak g$, allora $(\phi_*)_e(X) = 0$ che contraddice l'iniettività.
\end{proof}


\begin{corollary}
	Ogni omomorfismo continuo tra gruppi di Lie è di classe $C^\infty$.
\end{corollary}
%senza dimostrazione



\section{Forme invarianti}

\begin{definition}
	Una forma $\omega$ su $G$ gruppo di Lie è detta \emph{invariante a sinistra} se soddisfa $(L_a)^*\omega = \omega$ per ogni $a \in G$. ($\omega(b) = L_a^*(\omega(ab))$)
\end{definition}

Forme di questo tipo sono determinate dal loro valore nell'identità.

Se $\seqa\omega n,$ sono 1-forme invarianti a sinistra e tali che $\omega^1(e),\ldots,\omega^n(e)$ generano $T_e^*G$, allora ogni $k$-forma $\omega$ invariante a sinistra si scrive come $\sum_{\seqb ik<} a_{\seqb ik{}} \omega^{i_1}\wedge \ldots \omega^{i_k}$ con coefficienti $a_{\seqb ik{}}$ costanti.

Sia $\omega$ invariante a sinistra, allora $L_a^*\de\omega = \de L_a^*\omega = \de \omega$. Quindi $\de\omega$ è anch'essa invariante a sinistra. 
Inoltre se $\omega$ è una 1-forma invariante a sinistra e $\tilde X,\tilde Y \in \chi(G)$ sono campi vettoriali invarianti a sinistra, allora le funzioni $\omega(\tilde X), \omega(\tilde Y)$ sono invarianti a sinistra e quindi costanti (perché $G$ è connesso).
Abbiamo $\de \omega (\tilde X, \tilde Y) = \tilde X \omega (\tilde Y) - \tilde Y \omega(\tilde X) - \omega(\comm {\tilde X}{\tilde Y}) = - \omega (\comm{\tilde X}{\tilde Y})$, otteniamo quindi $\de \omega (e) (X,Y) = - \omega(e) (\comm XY)$.
Perciò $\div(e)(X,Y) = -\omega(e)(\comm XY)$. (*)

Siano $\omega^1(e), \ldots, \omega^n(e)$ come sopra e consideriamo una base duale $\seqb Xn,$ di $T_eG$. Esistono delle costanti $c_{ij}^k$ tali che $\comm{X_i}{X_j} = \sum_{k=1}^n c_{ij}^k X_k$. Questo implica anche che $\comm{\tilde X_i}{\tilde X_j} = \sum_{k=1}^n c_{ij}^k \tilde X_k$.

\begin{definition}
	I coefficienti $c_{ij}^k$ sono detti \emph{costanti di struttura} di $G$ rispetto alla base $\seqb Xn,$.
\end{definition}

Dall'antisimmetria della parentesi di Lie e da Jacobi, le costanti di struttura rispettano:
\begin{enumerate}
	\item $c_{ij}^k = -c_{ji}^k$;
	\item $\sum_{\alpha=1}^n c_{ij}^\alpha c_{\alpha k}^l + c_{jk}^\alpha c_{\alpha i}^l + c_{ki}^\alpha c_{\alpha j}^l = 0$.
\end{enumerate}

Inoltre da * abbiamo $\de \omega^k = -\sum_{i<j} c_{ij}^k \omega^i \wedge \omega^j = -\frac 12 \sum_{i,j} c_{ij}^k \omega^i \wedge \omega^j$.


\begin{theorem}
	Sia $G$ gruppo di Lie con base di 1-forme invarianti a sinistra $\seqa \omega n,$ e costanti di struttura $c_{ij}^k$. Sia $M^n$ una varietà differenziabile con forme $\seqa \theta n,$ linearmente indipendenti tali che $\de \theta^k = - \sum_{i<j} c_{ij}^k \theta^i\wedge \theta^j$.
	Allora per ogni $p\in M$ esiste $U$ intorno di $p$ ed $f: U \to G$ diffeomorfismo tale che $\theta^i = f^*\omega^i$. 
\end{theorem}
\begin{proof}
	Siano $\pi_i : M \times G\to M,G$ le proiezioni sui fattori e siano $\bar\theta^k = \pi_1^*\theta^k$ e $\bar \omega^k = \pi_2^* \omega^k$.
	Allora
	\begin{equation} \label{eq:starstar}
		\de (\bar \theta^k - \bar \omega^k) = -\sum_{i<j} c_{ij}^k ([\bar \theta^i \wedge \bar \theta^j] - [\bar \omega^i \wedge \bar \omega^j]) = - \sum_{i<j} c_{ij}^k [\bar \theta^i \wedge (\bar\theta^j - \bar \omega^j) + (\bar\theta^i - \bar \omega^i)\wedge \bar \omega^j] \virgola
	\end{equation}
	che è una condizione di integrabilità.
	
	Consideriamo la distribuzione in $M\times G$ (che si dimostra avere dimensione $n$) generata da tutti i campi vettoriali che sono annullati da ogni $\bar \theta^k - \bar \omega^k$.
	
	\begin{description}
		\item [Integrabilità:] Siano $\seqb Yn,$ campi nella distribuzione. Allora
		\begin{align*}
			(\bar\theta^k - \bar\omega^k) (\comm{Y_\alpha}{ Y_\beta}) &= -\de (\bar\theta^k - \bar\omega^k)(Y_\alpha, Y_\beta) + Y_\alpha((\bar\theta^k - \bar\omega^k)(Y_\beta)) - Y_\beta ((\bar\theta^k - \bar\omega^k) ( Y_\alpha)) =\\
			&= - \de (\bar\theta^k - \bar\omega^k) (Y_\alpha,Y_\beta) = 0\virgola
		\end{align*}
		dove l'ultima uguaglianza è data dall'\cref{eq:starstar}.
	\end{description}

	Per Frobenius, dato $a \in G$ troviamo una varietà integrale $\Gamma$ che passa per $(p,a)$.
	
	Le $2n$ forme $\seqa {\bar \theta}n, , \seqa {\bar\omega}n,$ sono linearmente indipendenti in $M\times G$, quindi sia $\seqa{\bar\theta}n,$ che $\seqa {\bar\omega}n,$ sono linearmente indipendenti su $\Gamma$.
	Quindi $\pi_1: \Gamma \to M$ e $\pi_2: \Gamma \to G$ sono diffeomorfismi locali, perciò $\Gamma$ è il grafico di un diffeomorfismo $f$ da $U$ in un intorno (in $G$) di $a$.
	Sia $\tilde f: U \to M\times G$ tale che $\tilde f(q) = (q,f(q)) \subseteq \Gamma$.
	Visto che $(\bar\theta^k - \bar\omega^k)\restrict \Gamma = 0$, allora $0 = \tilde f^*(\bar\theta^k - \bar\omega^k) = \tilde f^* \pi_1^*\theta^k - \tilde f^* \pi_2^*\omega^k = (\pi_1\circ \tilde f)^* \theta^k - (\pi_2 \circ \tilde f)^* \omega^k = \theta^k - f^* \omega^k$.
\end{proof}


\begin{exercise}
	\begin{enumerate}
		\item Mostrare che il fibrato tangente $TG$ di $G$ gruppo di Lie ammette struttura di gruppo di Lie.
		\item Siano $X,Y \in T_eG$ tali che $\comm XY = 0$. Mostrare che
		\begin{enumerate}
			\item $\exp(sX) \exp (tY) = \exp(tY) \exp(sX)$;
			\item $\exp(X+Y) = \exp X \exp Y$.
		\end{enumerate}
	\end{enumerate}

\end{exercise}



















































