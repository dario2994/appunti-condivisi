\documentclass{article}
\usepackage[italian]{babel}
\usepackage[utf8x]{inputenc}
\usepackage{amssymb}
\usepackage{mathrsfs}%Serve a poter scrivere in corsivo inglese
\usepackage{enumitem}%Serve per gli elenchi numerati
\usepackage{amsthm}%Serve per avere ambienti non numerati
\usepackage{graphicx}

\newcommand{\PHI}{\varphi}
\newcommand{\incluso}{\subseteq}
\newcommand{\composto}{\circ}
\newcommand{\inv}{^{-1}}
\newcommand{\intersez}{\cap}
\newcommand{\cors}[1]{\textit{#1}}%corsivo
\newcommand{\ci}[1]{\mathscr{#1}}%corsivo inglese
\newcommand{\g}[1]{\mathfrak{#1}}
\newcommand{\volte}{\times}
\newcommand{\perogni}{\forall}
\newcommand{\di}{\partial}
\newcommand{\vuoto}{\emptyset}

\newtheorem{definizione}{ Definizione}
\newtheorem{oss}{ Osservazione}
\newtheorem*{esempi}{ Esempi}
\newtheorem{teor}{ Teorema}
\newtheorem*{esercizi}{ Esercizi}
\newtheorem{proposizione}{ Proposizione}
\newtheorem*{lemma}{ Lemma}

\title{ Appunti del corso di calcolo diffenziale su varietà}
\author{}
\date{}


\begin{document}

 \maketitle

\section{ Varietà differenziabili}

 
 
 Le varietà sono spazi localmente simili agli spazi euclidei, ma globalmente diversi.
 
 Per motivi sia teorici che applicativi è utile sviluppare strumenti su «\cors{oggetti curvi}».
 Ad esempio si possono studiare i sistemi dinamici sulle varietà, oppure ricavare
 informazioni topologiche a partire da concetti analitici.
 
 Il modello di riferimento sono le varietà immerse in ${\bf R }^m$, ma si può dare
 una nozione intrinseca di varietà.
 
 \begin{definizione}
 Una spazio topologico $M$ è detto localmente euclideo se è
 di Hausdorff ed è localmente omeomorfo ad un aperto di ${ \bf R}^m$.
 \end{definizione}
 
 \begin{definizione}
 Se $U$ è un aperto di $M$ e $\PHI:U\to \PHI(U)\incluso {\bf R}^m$
 è un omeomorfismo, la coppia $(U, \PHI)$ è detto carta, o sistema di coordinate.
 Le componenti di $\PHI$ si dicono funzioni coordinate. Una famiglia di carte
 che ricopre $M$ è detta atlante.
 \end{definizione}
 
 Se $(U_1, \PHI_1)$ e $(U_2, \PHI_2)$ sono carte, $\PHI_2\composto\PHI_1\inv:\PHI_1(U_1\cap U_2)\to\PHI_2(U_1\cap U_2)$
 è un omeomorfismo tra aperti di ${\bf  R}^n$, detto \cors{cambio di carta} o \cors{transizione}.
 Per poter fare uso degli strumenti del calcolo differenziale, si considera un insieme
 di carte tale che i cambi di carta siano differenziabili.
 
 \begin{definizione}
 Una \cors{struttura differenziabile} di classe $\ci{C}^k$ su uno
 spazio localmente euclideo è un atlante $\g{F}$ tale che tutti i cambi di carta siano
 di classe $\ci{C}^k$, e massimale rispetto all'inclusione.
 \end{definizione}
 
 \begin{oss}
  Se $\g{F}$ non è massimale esso è contenuto in un'unica struttura differenziabile $\ci{C}^k$,
  formata da tutte le carte con transizioni di classe $\ci{C}^k$ rispetto ad $\g{F}$.
 \end{oss}
 
 \begin{definizione}
  Una varietà differenziabile $n$-dimensionale di classe $\ci{C}^k$ è una coppia $(M, \g{F})$
  tale che $M$ sia uno spazio topologico localmente euclideo a base numerabile e $\g{F}$ sia una struttura
  differenziabile su $M$.
 \end{definizione}
 
 \begin{esempi}
  \begin{enumerate}[label=\bf\Roman*)]
   \item ${ \bf R}^n$ con l'atlante formato da una sola carta (l'identità), o più in generale tutti gli aperti di ${\bf R }^n$.
   \item ${\bf S}^n$, con un atlante formato da due carte, le proiezioni stereografiche.
   \item Lo spazio proiettivo ${\bf RP}^n$ con l'atlante $(U_i,\PHI_i)_{i=0,\dots,n}$ tale che
   $U_i=\{ [x_0,\dots,x_n]\; |\; x_i\neq 0\}$ e $\PHI([x_0,\dots,x_n])=\frac{1}{x_n}(x_0,\dots,\hat{x}_i,\dots,x_n)$.
   \item La varietà graßmanniana $G_k({\bf  R}^n)$, formata dai sottospazi vettoriali $k$-dimensionali
   di ${\bf R}^n$ si può dotare di una struttura differenziabile naturale (si veda, ad
   esempio, il libro del Marsden citato in bibliografia, oppure il libro dell'Abate).
  \end{enumerate}
 \end{esempi}
 
 \begin{oss}
  Esistono varietà omeomorfe ma non diffeomorfe. Ad esempio, come scoperse il Milnor, ci sono ventotto varietà
  omeomorfe a ${\bf S}^7$ ma tra di loro non diffeomorfe, dette «sfere esotiche». Su ${\bf S}^1$, ${\bf S}^2$, ${\bf S}^3$, ${\bf S}^5$
  ed ${\bf S}^6$ questo non accade, mentre per ${\bf S}^4$ il problema è tuttora aperto.
 \end{oss}
 
 \begin{oss}
  La richiesta che le varietà siano a base numerabile ha alcune importanti conseguenze:
  \begin{enumerate}[label=\bf\Roman*)]
   \item Le varietà sono metrizzabili
   \item le varietà sono spazi topologici normali.
   \item Le varietà sono paracompatte, e dunque esistono le partizioni dell'unità.
   \item Le varietà sono $\sigma$-compatte, ossia sono un'unione numerabile di compatti.
  \end{enumerate}
 \end{oss}

 \begin{esercizi}
  \begin{enumerate}[label=\bf\Roman*)]
   \item Se $g:{\bf R}^n\to{\bf R}$ è di classe $\ci{C}^k$, $M=g\inv(\{0\})$ e $\nabla g\ne 0$
   su $M$ allora $M$ è una varietà $\ci{C}^k$ di dimensione $n-1$.
   \item $SL_n({\bf R})$ è una varietà di dimensione $n^2-1$.
   \item La funzione $t\mapsto t^3$ induce su ${\bf R}$ una struttura differenziabile diversa
   da quella ordinaria, ma le varietà ottenute sono diffeomorfe.
  \end{enumerate}
 \end{esercizi}
 
 \subsection{ Prodotti e mappe}
 
 Se $(M_1,\g{F}_1)$ e $(M_2,\g{F}_2)$ sono varietà $\ci{C}^k$ allora si può definire la
 varietà prodotto $(M_1\volte M_2,\g{F}_1\volte\g{F}_2)$, dove $\g{F}_1\volte\g{F}_2)$
 è la struttura differenziabile generata dai prodotti delle carte in $\g{F}_1$ e $\g{F}_2$.
 Ovviamente si ha che $\dim(M_1\volte M_2)=\dim(M_1)+\dim(M_2)$.
 
 In modo analogo si può definire il prodotto di più fattori.
 
 \begin{definizione}
  Se $M$ ed $N$ sono varietà $\ci{C}^k$ e $r\le k$, una funzione $f:M\to N$ si dice
  di classe $\ci{C}^r$ se $\perogni x\in M$ ci sono una carta $(U,\PHI)$ di $M$ con
  $x\in U$ ed una carta $(V,\psi)$ di $N$ con $f(U)\incluso V$ tali che $\psi\composto f\composto\PHI\inv$
  sia di classe $\ci{C}^r$.
 \end{definizione}
 
 L'ipotesi che $r\le k$ serve affiché la definizione non dipenda dalle carte scelte, come
 mostra la seguente proposizione.
 
 \begin{proposizione}
  $f:M\to N$ è $\ci{C}^r$ se e solo se tutti i suoi rappresentanti locali sono $\ci{C}^r$.
 \end{proposizione}
 
 \begin{proposizione}
   La composizione di due funzioni $\ci{C}^r$ è $\ci{C}^r$.
 \end{proposizione}
 
 \begin{definizione}
  Un diffeomorfismo $\ci{C}^r$ è una funzione $\ci{C}^r$ invertibile e con inversa $\ci{C}^r$.
 \end{definizione}
 
 \subsection{ Spazio tangente}
 
 Data una varietà $n$-dimensionale immersa in ${\bf R}^m$, per ogni suo punto esiste
 un sottospazio affine di dimensione $n$ detto «\cors{spazio tangente}». Vogliamo
 mostrare come estendere questa costruzione alle varietà astratte. Vi sono diversi
 approcci per fare ciò; noi vedremo lo spazio tangente ad un punto come «insieme delle
 velocità delle curve passanti per quel punto».
 
 \begin{definizione}
  Siano $M$ una varietà, $x\in M$, $c,\tilde{c}:(-1,1)\to M$ di classe $\ci{C}^1$ tali che
  $c(0)=\tilde{c}(0)=x$ e $(U,\PHI)$ una carta con $x\in U$. Allora diciamo che $c$ e
  $\tilde{c}$ sono tangenti in $x$ rispetto a $\PHI$ se $(\PHI\composto c)'(0)=(\PHI\composto\tilde{c})'(0)$
 \end{definizione}

 \begin{proposizione}
  La definizione precedente non dipende dalla carta.
 \end{proposizione}

 Dunque si può dare la seguente definizione.
 
 \begin{definizione}
  Due curve $c,\tilde{c}:(-1,1)\to M$ di classe $\ci{C}^1$ tali che $c(0)=\tilde{c}(0)=x$
  si dicono tangenti se le loro espressioni locali in carta sono tangenti.
 \end{definizione}

 Tutto ciò ci permette di definire il piano tangente.
 
 \begin{definizione}
  Lo spazio tangente ad un punto $x\in M$, indicato con $T_xM$, è l'insieme delle classi di equivalenza delle
  curve $c:(-1,1)\to M$ di classe $\ci{C}^1$ con $c(0)=x$ per la relazione «essere
  tangenti in x». L'unione degli spazi tangenti ad $M$ è detta spazio tangente, o
  fibrato tangente, ed è indicata con $TM$.
 \end{definizione}
 
 \begin{oss}
  Su $T_xM$ c'è un'evidente struttura lineare che lo rende uno spazio vettoriale con la
  stessa dimensione di $M$.
 \end{oss}

 Ora vediamo due definizioni alternative.
 
 \begin{paragraph}
  {\bf  Prima definizione alternativa: controvarianza} Per definire il tangente si possono
  usare le proprietà di trasformazione dei vettori.
  
  Infatti, se $(U,\PHI)$ e $(\tilde{U},\tilde{\PHI})$ sono due carte, e chiamiamo
  $(y^1,\dots,y^n)=(\PHI\composto c)(t)$ e $(\tilde{y}^1,\dots,\tilde{y}^n)=(\tilde{\PHI}\composto c)(t)$,
  allora il vettore tangente associato a $c'(0)$ si scrive nei due sistemi di coordinate
  come $(V^1,\dots,V^n)=(\frac{dy^1}{dt}(0),\dots,\frac{dy^n}{dt}(0))$ e
  $(\tilde{V}^1,\dots,\tilde{V}^n)=(\frac{d\tilde{y}^1}{dt}(0),\dots,\frac{d\tilde{y}^n}{dt}(0))$.
  Usando il cambio di carta si trova che la funzione $y\mapsto\tilde{y}$ è un diffeomorfismo
  locale, e la relazione tra le due coordinate è
  $$\tilde{V}^i=\frac{d\tilde{y}^i}{dt}(0)=\sum_{j=1}^n\frac{\di\tilde{y}^i}{\di y^j}(\PHI(x))
  \frac{dy^j}{dt}(0)=\sum_{j=1}^n\frac{\di\tilde{y}^i}{\di y^j}(\PHI(x))V^j$$
  Questo tipo di trasformazione è detto «\cors{controvarianza}», e può essere usato per
  definire lo spazio tangente.
  
  Con questa notazione, dato un sistema di coordinate $(U,\PHI)$, un vettore $v\in T_x M$
  (con $x\in U$) si scrive così:
  $$v=\sum_{i=1}^nv^i\frac{\di}{\di y^i}(\PHI(x))$$
  dove $\frac{\di}{\di y^i}(\PHI(x))=[\PHI(x)+te_i]_{\PHI(x)}$, essendo $e_i$ l'$i$-esimo
  vettore della base canonica.
 \end{paragraph}

 \begin{paragraph}
  {\bf  Seconda definizione alternativa: le derivazioni} Data $f:{bf R }^n\to{\bf R}$
  e $v\in{\bf R}^n$ sia $D_vf$ la derivata direzionale lungo $v$. Allora $D_v$ è lineare
  in $f$, ed inoltre vale la regola di Leibniz $D_v(fg)=fD_vg+gD_vf$. Si definiscono
  «\cors{derivazioni}» gli operatori lineari $A:\ci{C}^r(U)\in{\bf R}$ che godono della
  regola di Leibniz. Si può dimostrare che in questo modo si ottiene una definizione
  equivalente di spazio tangente.
  \end{paragraph}
 
 \subsection{ Mappe tangenti}
 
 Una funzione regolare agisce sui vettori tangenti alla varietà, permettendo di estendere
 il concetto di differenziale alle varietà.
 
 \begin{lemma}
  Se $c$ e $\tilde{c}$ sono curve tangenti in $x\in M$ e $f:M\to N$ è di classe $\ci{C}^1$
  allora $f\composto c$ e $f\composto\tilde{c}$ sono curve tangenti in $f(x)$.
 \end{lemma}

 \begin{proof}
  Basta passare in carta.
 \end{proof}

 \begin{definizione}
  Data $f:M\to N$ di classe $\ci{C}^1$ si definisce mappa tangente, o differenziale,
  di f, la mappa $Tf:TM\to TN$ (indicata anche con $df$) tale che $Tf([c]_x)=[f\composto c]_{f(x)}$
  (la definizione è ben posta per il lemma precedente).
  La restrizione di $Tf$ allo spazio tangente ad un punto x è indicata con $T_xf$.
 \end{definizione}

 \begin{proposizione}
  $T_xf:T_xM\to T_xN$ è lineare.
 \end{proposizione}
 
 \begin{teor}
  \begin{enumerate}[label=\bf\Roman*)]
   \item Se $f$ e $g$ sono $\ci{C}^r$ e la loro composizione ha senso allora $g\composto f$
   è $\ci{C}^r$ e $T(g\composto f)=Tg\composto Tf$.
   \item $T(id_M)=id_{TM}$.
   \item Se $f:M\to N$ è un diffeomorfismo allora $Tf$ è invertibile e $(Tf)\inv=T(f\inv)$.
  \end{enumerate}
 \end{teor}

 \begin{proof}
  Il primo punto si dimostra passando in carta, il secondo è banale, il terzo è
  conseguenza immediata dei due precedenti.
 \end{proof}

 Ora vogliamo dotare il fibrato tangente di una struttura differenziabile.
 
 Data una carta $(U,\PHI)$ definiamo $T\PHI:TU\to T(\PHI(U))$ così:
 $T\PHI([c]_x)=[\PHI\composto c]_{\PHI(x)}$. Si verifica facilmente che $T\PHI$ è biunivoca.
 
 \begin{proposizione}
  Se $k\ge 1$ e $(M,\g{F})$ è una varietà $\ci{C}^{k+1}$ allora $T\g{F}=\{(Tu,T\PHI)\;|\;(U,\PHI)\in\g{F}\}$
  è un atlante $\ci{C}^k$ su $TM$, detto atlante naturale.
 \end{proposizione}

 \begin{proof}
  Dato che $\g{F}$ ricopre $M$, $T\g{F}$ ricopre $TM$. Inoltre si verifica che $T\PHI_i\composto(T\PHI_j)\inv=T(\PHI_i\composto\PHI_j)$,
  che è un diffeomorfismo $\ci{C}^k$, nonché che con la topologia indotta da $T\g{F}$
  $TM$ è di Hausdorff e a base numerabile.
 \end{proof}

 \begin{proposizione}
  Se $f:M\to N$ è un diffeomorfismo $\ci{C}^{r+1}$, $Tf:TM\to TN$ è un diffeomorfismo $\ci{C}^r$.
 \end{proposizione}

 Ora definiamo alcuni tipi particolari di funzioni tra varietà.
 
 \begin{definizione}
  Sia $f:M\to N$ di classe $\ci{C}^1$.
  \begin{enumerate}[label=\bf\Roman*)]
   \item Se il diffenziale di $f$ è iniettivo in ogni punto, $f$ è detta immersione.
   \item Se $f$ è un'immersione iniettiva, $(M,f)$ è detta sottovarietà.
   \item Se $f$ è un'immersione ed un omeomorfismo con l'immagine, è detta embedding.
  \end{enumerate}
 \end{definizione}

 \begin{proposizione}
  Se $f:M\to N$ è di classe $\ci{C}^1$ e $T_xf$ è invertibile allora $f$ è localmente invertibile in $x$.
 \end{proposizione}

 \begin{proof}
  Basta passare in carta ed applicare il teorema della funzione implicita.
 \end{proof}

 \begin{proposizione}
  Se $f:M\to N$ è di classe $\ci{C}^1$ e $T_xf$ è iniettivo allora $f$ è localmente iniettiva.
 \end{proposizione}

 \begin{proof}
  Passando in carta si può supporre $M={\bf R}^m$, $N={\bf R}^n$ e $x=0$. Allora, dato
  che $Tf=Df$, si ha che
  $$f(p)-f(q)=\int_0^1Df(q+s(p-q))[p-q]ds=(Df(0)+o(1))[p-q]$$
  per $p,q\to 0$. Quindi $f(p)\ne f(q)$ per $p,q$ piccoli.
 \end{proof}

 \begin{esercizi}
  \begin{enumerate}[label=\bf\Roman*)]
   \item Si dimostri che $U(n)$ (il gruppo delle trasformazioni unitarie di ${\bf C}^n$)
   è una sottovarietà non compatta di $\ci{L}({\bf C}^n,{\bf C}^n)$.
   \item Si dimostri che ${\bf RP}^1$ è una sottovarietà di ${\bf RP}^2$.
   \item Si dimostri che $\g{P}=\{ Q\in O(3)\;|\;\det Q=1,\; Q=Q^T\}\backslash\{I\}$
   è una sottovarietà compatta di $O(3)$. Si descrivano inoltre gli elementi di $\g{P}$ in
   termini geometrici.
  \end{enumerate}
 \end{esercizi}

 \subsection{ Campi vettoriali}
 
 \begin{definizione}
  Sia $M$ una varietà $\ci{C}^k$. Un campo vettoriale su $M$ è una funzione $X:M\to TM$
  tale che $X(p)\in T_pM$ per ogni $p\in M$. Un campo vettoriale è detto di classe $\ci{C}^r$
  (con $r\le k$) se le sue componenti sono di classe $\ci{C}^r$.
 \end{definizione}

 Se $(U,\PHI)$ è una carta e $(x^1,\dots,x^n)$ sono coordinate allora localmente
 $$X=\sum_{i=1}^nX^i(x)\frac{\di}{\di x^i}$$
 dove $(X_1,\dots,X_n)=T\PHI(X(\PHI\inv(X)))$.
 
 Chiamiamo $\g{X}^r(M)$ l'insieme dei campi vettoriali $\ci{C}^r$ su $M$ (con $r$
 eventualmente infinito).
 
 \begin{definizione}
  Una curva integrale di un campo vettoriale $X$ è una curva diffenziabile $c$ tale
  che $c'(t)=X(c(t))$ per ogni $t$.
 \end{definizione}

 Se $(U,\PHI)$ è una carta allora le coordinate $c^i(t)$ di $c$ soddisfano le equazioni
 diffenziali $\frac{dc^i}{dt}=X^i(c^1(t),\dots,c^n(t))$ per $i=1,\dots,n$. Questo ha un
 sistema autonomo, anche se in generale si possono considerare anche campi vettoriali
 dipendenti dal tempo.
 
 Ora richiamiamo dei risultati sulle equazioni diffenziali ordinarie che daremo per noti.
 
 \begin{teor}[ Cauchy-Lipschitz]
  Siano $U\incluso{\bf R}^n$ aperto e $X:U\to{\bf R}^n$ di classe $\ci{C}^1$. Allora
  $\perogni x_0\in U$ esiste $I\incluso{\bf R}$ aperto con $0\in I$ e $c:I\to{\bf R}^n$
  tali che $c(0)=x_0$, $c'(t)=X(c(t))$. Inoltre se esistono $c_1,c_2:I_1,I_2\to{\bf R}^n$
  come sopra, allora esse coincidono su $I_1\intersez I_1$.
 \end{teor}

 \begin{teor}[ differenziabilità del flusso]
  Se, nelle stesse ipotesi del teorema precedente, $X$ è di classe $\ci{C}^k$, allora
  esistono un intorno aperto $U_0\incluso U$, un numero $a>0$ ed una funzione
  $\ci{F}:U_0\volte (-a,a)\to{\bf R}^n$ di classe $\ci{C}^k$ tali che $\perogni u\in U_0$ la curva $c_u(t)=\ci{F}(u,t)$
  sia una curva integrale di $X$ tale che $c_u(0)=u$.
 \end{teor}

 \begin{definizione}
  Sia $M$ una varietà e $X\in\g{X}^r(M)$. Allora un flusso locale di $X$ in $p\in M$ è
  una tripla $(U_0,a,\ci{F})$ tale che:
  \begin{enumerate}[label=\bf\Roman*)]
   \item $p\in U_0\incluso M$, $U_0$ è aperto e $a>0$;
   \item $\ci{F}:U_0\volte I\to M$ è di classe $\ci{C}^r$ (dove $I=(-a,a)$;
   \item $\perogni u\in U_0$ si ha che $c_u:I\to M$ è una curva integrale di $X$
   con dato iniziale $u$;
   \item se, per $t\in I$ chiamiamo $\ci{F}_t(u)=\ci{F}(u,t)$ allora $\ci{F}_t(U_0)$ è aperto e
   $\ci{F}_t$ è un diffeomorfismo $\ci{C}^r$ con l'immagine.
  \end{enumerate}
 \end{definizione}

 Prima di dimostrare l'esistenza di un flusso locale, mostriamone le proprietà di
 unicità e di semigruppo.
 
 \begin{proposizione}
  Due curve integrali di un campo vettoriale con la stessa condizione iniziale
  coincidono sull'intersezione dei loro dominî.
 \end{proposizione}

 \begin{proof}
  Osserviamo che non si può applicare direttamente il teorema di Cauchy-Lipschitz
  perché la curva potrebbe non appartenere ad una sola carta.
  
  Sia $\g{K}=\{ t\in I\;|\; c_1(t)=c_2(t)\}\incluso I$, dove $I$ è l'intersezione dei
  dominî delle due curve (ed è dunque un intervallo di ${\bf R}$). Allora $\g{K}$ è chiuso perché $M$ è di
  Hausdorff, aperto perché $\perogni t\in\g{K}$ si può prendere una carta che lo contiene
  ed applicare in carta il teorema di Cauchy-Lipschitz, e non vuoto perché per ipotesi
  $0\in\g{K}$. Dunque, dato che $I$ è connesso, $\g{K}=I$.
 \end{proof}

 \begin{proposizione}
  Se la tripla $(U_0,a,\ci{F})$ soddisfa le ipotesi {\bf I}, {\bf II} e {\bf III} nella
  definizione di curva integrale, e se $t,t,s+t\in I$, allora $\ci{F}_{s+t}=
  \ci{F}_s\composto \ci{F}_t=\ci{F}_t\composto\ci{F}_s$, dove le composizioni sono definite.
  Inoltre $\ci{F}_0$ è l'identità, e se $U_t=\ci{F}_t(U_0)$ e $U_t\intersez U_0\ne\vuoto$, si ha
  che $\ci{F}_t|_{U_{-t}\intersez U_0}:U_{-t}\intersez U_0\to U_0\intersez U_t$ è un
  diffeomorfismo con inverso $\ci{F}_{-t}|_{U_0\intersez U_t}$
 \end{proposizione}

 \begin{proof}
  $\ci{F}_{s+t}(u)=c_u(s+t)$ e $\ci{F}_t(\ci{F}_s(u)=\ci{F}_t(c_u(s))$ sono entrambe
  curve integrali che passano per $c_u(s)$ per $t=0$, dunque coincidono per la proposizione
  precedente. Perciò $\ci{F}_{s+t}=\ci{F}_t\composto\ci{F}_s$ laddove la composizione
  abbia senso. Da ciò si deduce facilmente il resto.
 \end{proof}

\end{document}