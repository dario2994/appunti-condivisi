\chapter{Calcolo tensoriale} %TODO: unire a tutti i tensori

\section{Definizione e operazioni su spazi vettoriali}
Dati $V_1,\ldots,V_k,W$ spazi vettoriali, denotiamo con $\Lin^k(V_1,\ldots,V_k,W)$ lo spazio delle mappe $k$-multilineari da $V_1\times\ldots\times V_k$ in $W$. In tutta la trattazione considereremo spazi vettoriali di dimensione finita.
Denotiamo inoltre con $V^* = \Lin(V,\R)$ lo spazio duale di $V$.
Se $V$ è $n$-dimensionale ed $e_1,\ldots,e_n$ è una sua base, esiste una base $e^1,\ldots,e^n$ di $V^*$ tale che $\Scal{e^i}{e_j} = \delta^i_j$ \footnote{Dati $\alpha \in V^*$ e $v\in V$, indichiamo con $\Scal{\alpha}{v}$ l'elemento $\alpha$ applicato a $v$.}.
In particolare, dati $v\in V$ e $\alpha \in V^*$, abbiamo che $v= \sum_{i=1}^n \Scal{e^i}{v} e_i$ e $\alpha= \sum_{i=1}^n \Scal{\alpha}{e_i} e^i$.

\begin{definition} \index{tensore} \index{tensore!covariante} \index{tensore!controvariante}
	Dato $V$ spazio vettoriale definiamo 
	\begin{equation*}
	T_s^r(V) \coloneqq \Lin^{r+s} (\underbrace{V^*,\ldots,V^*}_{\text{$r$ copie}},\underbrace{V,\ldots,V}_{\text{$s$ copie}},\R) \punto
	\end{equation*}
	Gli elementi di $T_s^r(V)$ sono detti \emph{tensori} su $V$, \emph{controvarianti} di ordine $r$ e \emph{covarianti} di ordine $s$ (oppure tensori di tipo $(r,s)$).
\end{definition}

\begin{remark}
	Analogamente possiamo definire $T_s^r(V,W) = \Lin^{r+s} (V^*,\ldots,V^*,V,\ldots,V,W)$.
\end{remark}


\begin{definition} \index{prodotto!tensore}
	Dati $t_1\in T_{s_1}^{r_1}(V)$ e $t_2\in T_{s_2}^{r_2}(V)$, il \emph{prodotto tensore} di $t_1$ e $t_2$ è il tensore $t_1\otimes t_2 \in T_{s_1+s_2}^{r_1+r_2}(V)$ definito come 
	\begin{multline*}
		t_1\otimes t_2(\beta^1,\ldots,\beta^{r_1},\gamma^1,\ldots,\gamma^{r_2}, f_1,\ldots,f_{s_1},g_1,\ldots,g_{s_2}) \coloneqq\\
		t_1(\beta^1,\ldots,\beta^{r_1},f_1,\ldots,f_{s_1})t_2(\gamma^1,\ldots,\gamma^{r_2},g_1,\ldots,g_{s_2}) \virgola
	\end{multline*}
	con $\beta^j,\gamma^j\in V^*$ e $f_i,g_i\in V$.
\end{definition}

\begin{example}
	Vediamo alcuni casi interessanti:
	\begin{enumerate}
		\item $T_0^1(V) = \Lin(V^*,\R) \cong V$, infatti un elemento $v\in V$ agisce su $V^*$ tramite l'applicazione $\alpha \mapsto \Scal\alpha v$. In questo senso scriveremo $v \in T_0^1$ e in generale quando parleremo di elementi di $V$ come tensori sottoinderemo questa corrispondenza.
		\item $T_1^0(V) \cong V^*$. Analogamente al caso precedente, nel seguito daremo per scontata questa corrispondenza e diremo $\alpha \in T_1^0$ per un elemento $\alpha \in V^*$.
		\item $T_2^0(V) \cong \Lin(V,V^*)$, tramite l'isomorfismo dato dalla relazione $t(v,w) = \Scal{f(v)}{w}$, con $t\in T_2^0(V)$ ed $f\in\Lin(V,V^*)$.
		\item $T_1^1(V) \cong \Lin(V,V)$, tramite la relazione $t(\alpha,v) = \Scal{\alpha}{\tilde f (v)}$, con $t\in T_1^1(V)$ e $\tilde f\in \Lin(V,V)$.
	\end{enumerate}
\end{example}

\begin{proposition}
	Sia $V$ uno spazio vettoriale $n$-dimensionale con base $e_1,\ldots,e_n$. Sia $e^1,\ldots,e^n$ la base duale di $V^*$. Allora elementi del tipo $e_{i_1}\otimes \ldots \otimes e_{i_r}\otimes e^{j_1}\otimes \ldots \otimes e^{j_s}$ formano una base di $T_s^r(V)$.
\end{proposition}
\begin{proof}
	Dobbiamo dimostrare che elementi come sopra sono linearmente indipendenti e generano linearmente $T_s^r(V)$.
	
	Supponiamo che non valga la lineare indipendenza, allora esiste una loro combinazione lineare a coefficienti non nulli che si annulla:
	\begin{equation*}
		\sum t_{j_1\ldots j_s}^{i_1\ldots i_r}\ e_{i_1}\otimes \ldots \otimes e_{i_r}\otimes e^{j_1}\otimes \ldots \otimes e^{j_s} = 0\punto
	\end{equation*}
	Applichiamo questo tensore a $(e^{k_1},\ldots,e^{k_r},e_{l_1},\ldots,e_{l_s})$, allora otteniamo $t_{l_1\ldots l_s}^{k_1 \ldots k_r} = 0$. Questo per ogni $(k_1,\ldots, k_r)$ e $(l_1,\ldots,l_s)$, che è assurdo.
	
	Dato ora $t\in T_s^r(V)$, possiamo scrivere
	\begin{equation*}
		t = \sum t(e^{i_1},\ldots, e^{i_r},e_{j_1},\ldots, e_{j_s})\ e_{i_1}\otimes \ldots \otimes e_{i_r}\otimes e^{j_1}\otimes \ldots \otimes e^{j_s}\virgola
	\end{equation*}	
	da cui abbiamo anche che gli elementi cercati generano $T_s^r(V)$.
\end{proof}

\begin{definition} \index{tensore!componenti}
	I coefficienti $t_{j_1\ldots j_s}^{i_1\ldots i_r}$ sono detti \emph{componenti} di $t$ rispetto alla base $e_1,\ldots,e_n$ di $V$.
\end{definition}

\begin{example} \index{tensore!simmetrico}
\begin{enumerate}
	\item Se $t\in T_2^0(V)$, le sue componenti sono $t_{ij} = t(e_i,e_j)$, che formano una matrice $n\times n$. A questa matrice associamo la forma bilineare ovvia.
	Nel caso di $\R^2$, se $t_{ij} = \left(\begin{matrix}
	                                  A & B \\
	                                  C & D 
	                                 \end{matrix}\right)$
	, associamo quindi la forma 
	\begin{equation*}
		t(x,y) = Ax_1y_1+Bx_1y_2+Cx_2y_1+Dx_2y_2\punto
	\end{equation*}
	Il tensore $t$ è detto \emph{simmetrico} se $t(e_i,e_j) =t(e_j,e_i)$. In questo caso anche la matrice associata è simmetrica. Questo genera la forma quadratica $Q(e) = t(e,e)$, dalla polarizzata $t(e_i,e_j) = \frac 14 [Q(e_i+e_j)-Q(e_i-e_j)]$.
	
	\item In generale $t\in T_0^r(V)$ è detto \emph{simmetrico} se $t(\alpha^1,\ldots,\alpha^r) = t(\alpha^{\sigma(1)},\ldots,\alpha^{\sigma(r)})$ per ogni permutazione $\sigma$ e per tutti gli $\alpha^1,\ldots,\alpha^r \in V^*$.
	Inoltre possiamo ancora associare al tensore $t$ il polinomio $P(\alpha) = t(\alpha,\ldots,\alpha)$, omogeneo di grado $r$.
	
	Vale una definizione analoga per i tensori covarianti simmetrici.
	\item Data $\sigma$ permutazione di $\{1,\ldots,k\}$, lo spazio $\Lin^k(V_1,\ldots,V_k,W)$ è isomorfo allo spazio $\Lin^k(V_{\sigma(1)},\ldots,V_{\sigma(k)},W)$.
	Tale isomorfismo si vede come 
	\begin{equation*}
		A \in \Lin^k(V_1,\ldots,V_k,W) \mapsto A'\in \Lin^k(V_{\sigma(1)},\ldots,V_{\sigma(k)},W)
	\end{equation*}
	tale che $A'(e_{\sigma(1)},\ldots,e_{\sigma(k)}) = A(e_1,\ldots,e_k)$.
\end{enumerate}
\end{example}

\begin{definition} \index{prodotto!interno} \label{def:ProdottoInternoTensore}
	Il \emph{prodotto interno} di un tensore con un vettore $v\in V$  è una mappa $i_v:T_s^r(V)\to T_{s-1}^r(V)$ tale che
	\begin{equation*}
		i_vt(\beta^1,\ldots,\beta^r,v_1,\ldots,v_{s-1}) \coloneqq t(\beta^1,\ldots,\beta^r,v,v_1,\ldots,v_{s-1})\punto
	\end{equation*}
	Analogamente possiamo definire il \emph{prodotto interno} con un covettore $\beta \in V^*$ come una mappa $i^\beta:T_s^r(V)\to T_{s}^{r-1}(V)$ tale che
	\begin{equation*}
		i^\beta t(\beta^1,\ldots,\beta^{r-1},v_1,\ldots,v_s) \coloneqq t(\beta,\beta^1,\ldots,\beta^{r-1},v_1,\ldots,v_s)\punto
	\end{equation*}
\end{definition}

\begin{remark}
	Le mappe $i_v:T_s^r(V)\to T_{s-1}^r(V)$ e $i^\beta:T_s^r(V)\to T_s^{r-1}(V)$ sono lineari, così come le mappe $v\mapsto i_v$ e $\beta\mapsto i^\beta$.
\end{remark}

Vediamo ora i prodotti interni in componenti. Per linearità, ci basta esplicitarli per i tensori della base.
\begin{equation*}
	i_{e_k} (e_{i_1}\otimes \ldots \otimes e_{i_r}\otimes e^{j_1}\otimes \ldots\otimes e^{j_s}) = \delta_k^{j_1}\ e_{i_1}\otimes \ldots \otimes e_{i_r}\otimes e^{j_2}\otimes \ldots\otimes e^{j_s}\virgola
\end{equation*}
\begin{equation*}
	i^{e^k} (e_{i_1}\otimes \ldots \otimes e_{i_r}\otimes e^{j_1}\otimes \ldots\otimes e^{j_s}) = \delta_{i_1}^k\ e_{i_2}\otimes \ldots \otimes e_{i_r}\otimes e^{j_1}\otimes \ldots\otimes e^{j_s}\punto
\end{equation*}

\begin{definition} \index{contrazione}
	La \emph{contrazione} del $k$-esimo indice controvariante con l'$l$-esimo indice covariante (contrazione di tipo $(k,l)$) è la mappa lineare $C_l^k:T_s^r(V)\to T_{s-1}^{r-1}(V)$ definita da 
	\begin{multline*}
		C_l^k(t_{j_1\ldots j_s}^{i_1\ldots i_r}\ e_{i_1}\otimes \ldots \otimes e_{i_r}\otimes e^{j_1}\otimes \ldots\otimes e^{j_s}) \coloneqq\\
		=t_{j_1\ldots j_{l-1}pj_{l+1}\ldots j_s}^{i_1\ldots i_{k-1}pi_{k+1}\ldots i_r}\ e_{i_1}\otimes \ldots \otimes\hat e_{i_k}\otimes \ldots\otimes e_{i_r}\otimes e^{j_1}\otimes \ldots\otimes \hat e^{j_l}\otimes \ldots \otimes e^{j_s} \virgola
	\end{multline*}
	dove per convenzione quando si scrivono indici alti e bassi ripetuti si intende che si somma da $1$ ad $n$ (notazione di Einstein). In questo particolare caso si sta quindi sommando su $p$.
\end{definition}

\begin{remark}
	La contrazione $(k,l)$ è indipendente dalla base $e_1,\ldots,e_n$ di $V$. %TODO: perché? mettere la dimostrazione?
\end{remark}

\begin{definition} \index{delta di Kronecker} %TODO: eventualmente spostare sopra.
	La \emph{delta di Kronecker} è il tensore $\delta \in T_1^1(V)$ definito come
	\begin{equation*}
		\delta(\alpha,e) \coloneqq \Scal{\alpha}{e}\punto
	\end{equation*}
\end{definition}

\begin{remark}
	La $\delta$ corrisponde all'identità su $V$ quando consideriamo $T_1^1(V) \cong \Lin(V,V)$.
	
	Inoltre, fissando una base, risulta che $\delta = \sum_{i,j}\delta_j^i\ e_i\otimes e^j$, dove $\delta_j^i$ sono i classici simboli di Kronecker.
\end{remark}

Supponiamo ora che $V$ sia dotato di un prodotto interno $\ll \cdot, \cdot \gg$ simmetrico e definito positivo. Siano $e_1,\ldots,e_n$ una base di $V$ e $e^1,\ldots,e^n$ la base del duale $V^*$.
Sia inoltre $g_{ij} \coloneqq \ll e_i,e_j\gg$ la matrice dei coefficienti del prodotto interno.

Allora abbiamo due isomorfismi (musicali) che sono:
\begin{enumerate}
	\item $\flat:V\to V^*$ (\emph{bemolle}) tale che $x\mapsto \ll x,\cdot \gg$; \index{operatore!bemolle $\flat$}
	\item $\sharp:V^*\to V$ (\emph{diesis}) che è il suo inverso. \index{operatore!diesis $\sharp$}
\end{enumerate}

La matrice dei coefficienti di $\flat$ è $g_{ij}$ stessa, poiché $(x^\flat)_i = g_{ij}x^j$.
Invece la matrice dei coefficienti di $\sharp$ è $g^{ij}$, cioè la matrice inversa di $g_{ij}$; infatti $(\alpha^\sharp)^i = g^{ij}\alpha_j$.

L'operatore $\flat$ è detto \emph{operatore di abbassamento di indice}, mentre $\sharp$ \emph{operatore di innalzamento}. Questo perché permettono di passare da tensori del tipo $(r,s)$ a tensori del tipo $(r-1,s+1)$ e $(r+1,s-1)$ rispettivamente. \index{operatore! di abbassamento} \index{operatore! di innalzamento}

\begin{example}
	Se $t$ è di tipo $(0,2)$, possiamo definire $t'\in T_1^1(V)$ tramite $t'(\alpha, e) = t(\alpha^\sharp, e)$ e avremo che $(t')_j^i = g^{ik}t_{kj}$.
\end{example}


\begin{example}
	Vediamo ora vari esempi riguardanti le varie operazioni definite.
	\begin{enumerate}
		\item Dati $t\in T_1^2(V)$ e $x=x^ie_i$, risulta che il prodotto interno di $t$ con $x$ è 
		\begin{align*}
		i_xt &= x^p\ i_{e_p}(t_j^{kl}\ e_k\otimes e_l\otimes e^j) = x^p\ t_j^{kl}\ i_{e_p} (e_k\otimes e_l\otimes e^j) =\\
		&= x^p\ t_j^{kl}\ \delta_p^j\ e_k\otimes e_l = x^j\ t_j^{kl}\ e_k\otimes e_l \punto
		\end{align*}
		Mentre il prodotto interno di $t$ con $\alpha = \alpha_pe^p$ è
		\begin{align*}
		i^\alpha t = \alpha_p\ t_j^{kl}\ i^{e^p}(e_k\otimes e_l\otimes e^j) = \alpha_k\ t_j^{kl}\ e_l\otimes e^j \punto
		\end{align*}
		
		\item Se $t\in T_3^2(V)$, facendo una contrazione $(2,1)$, ottengo
		\begin{equation*}
			C_1^2 (t_{klm}^{ij}\ e_i\otimes e_j\otimes e^k\otimes e^l\otimes e^m ) = t_{klm}^{ij}\ \delta_j^k\ e_i\otimes e^l\otimes e^m = t_{klm}^{ik}\ e_i\otimes e^l\otimes e^m \punto
		\end{equation*}
		
		\item Se $t\in T_1^1(V)$ la sua \emph{traccia} è definita da $C_1^1(t) = t_i^i$. \index{traccia}

		\item Siano $g_{ij}$ i coefficienti di $\ll \cdot, \cdot \gg$, con inversa $g^{ij}$. Alzando e abbassando gli indici otteniamo $g^{jk}g_{kl} = \delta_l^j$ e $g_{jk}g^{kl} = \delta_j^l$.
		
		Tramite un prodotto, si può definire anche la traccia di $t\in T_0^2(V)$ dalla traccia del tensore $(1,1)$ associato. Se $t=t^{ij}\ e_i\otimes e_j$, possiamo perciò definire $\tr (t) = \tr (g_{ij}t^{jk}) = g_{ik}t^{ik}$.
	\end{enumerate}
\end{example}

\begin{exercise}
	Calcolare il prodotto interno di $t=e_1\otimes e_2\otimes e^2+3e_2\otimes e_2\otimes e^1$ con $e = -e_1+2e_2$ e $\alpha = 2e^1+e^2$, dove in questo caso $V=\R^2$.
\end{exercise}
\begin{exercise}
	Sia $\dim(V) = n$ e $\dim(V) = m$. Dimostrare che $T_s^r(V,W)$ ha dimensione $(mn)^{r+s}$.
\end{exercise}

\section{Push-forward e pull-back di trasformazioni lineari}

Definiamo innanzitutto il duale di una trasformazione lineare.
\begin{definition} \index{duale} \index{trasporto}
	Siano $V,W$ spazi vettoriali e $\varphi \in \Lin(V,W)$. Il \emph{duale}, o \emph{trasporto}, di $\varphi$ è $\varphi^* \in \Lin(W^*,V^*)$ definita come 
	\begin{equation*}
		\Scal{\varphi^*(\beta)}{e}_V = \Scal{\beta}{\varphi(e)}_W \virgola %TODO: cambiare notazione per non usare \Scal in questo caso?
	\end{equation*}
	per ogni $\beta \in W^*$ ed $e\in V$.
\end{definition}

Vediamo ora le relazioni fra i coefficienti di $\varphi$ e $\varphi^*$.
Siano $\{e_1,\ldots, e_n\}$ e $\{f_1,\ldots, f_m\}$ basi di $V^n$ e $W^m$. Supponiamo $\varphi(e_i) = A_i^af_a$ e diciamo che $A_i^a$ \footnote{Supponiamo che il pedice sia la colonna e l'apice la riga.} è la matrice associata a $\varphi$.
Dato $v = v^ie_i\in V$, vale che $\varphi(v)^a = A_i^a v^i$.
La matrice moltiplica a sinistra, quindi
\begin{equation*}
	\Scal{\varphi^*(f^a)}{e_i}_V = \Scal{f^a}{\varphi(e_i)}_W = \Scal{f^a}{A_i^b f_b} = A_i^b\delta_b^a = A_i^a\punto
\end{equation*}
Perciò vale che $\varphi^*(f^a) = A_i^a e^i$.

Se $\beta = \beta_a f^a \in W^*$, allora $\varphi^*(\beta) = \beta_a \varphi^*(f^a) = \beta_aA_i^ae^i$.
Di conseguenza $(\varphi^*(\beta))_i = \beta_aA_i^a$, cioè la matrice moltiplica a destra.


Definiamo di seguito push-forward e pull-back di isomorfismi fra spazi vettoriali, vedendo poi che tale definizione si può estendere a tutte le trasformazioni nel caso di tensori solo controvarianati o solo covarianti rispettivamente.

\begin{definition} \index{isomorfismo!push-forward}
	Se $\varphi \in \Lin(V,W)$ è un isomorfismo, il \emph{push-forward} $\varphi_* = T_s^r(\varphi)$ è la mappa in $\Lin(T_s^r(V), T_s^r(W))$ definita da
	\begin{equation*}
		\varphi_*(t) (\beta^1,\ldots,\beta^r,f_1,\ldots,f_s) = t(\varphi^*(\beta^1),\ldots,\varphi^*(\beta^r), \varphi^{-1}(f_1),\ldots, \varphi^{-1}(f_s)) \virgola
	\end{equation*}
	con $\beta^i\in W^*$ e $f_j\in W$.
\end{definition}

\begin{remark}
Si verifica che $\varphi_*$ è continua.
\end{remark}

\begin{remark}
	Nel caso di tensori $(1,0)$, cioè $\varphi_*=T_0^1(\varphi)$, si ha $T_0^1(V)\cong V$ e $T_0^1(W)\cong W$, allora $T_0^1(\varphi) \in \Lin(V,W)$ ed è proprio $\varphi$.
\end{remark}

\begin{proposition} \label{prop:ProprietaPushForwardIsomorfismi}
	Siano $\varphi\in \Lin(V,W)$, $\psi\in \Lin(W,Z)$ isomorfismi. Allora
	\begin{enumerate}
	 \item $(\psi\circ\varphi)_* = \psi_*\circ \varphi_*$; \label{ppfi:Distributiva}
	 \item se $i:V\to V$ è l'identità, allora $i_*:T_s^r(V) \to T_s^r(V)$ è l'identità; \label{ppfi:Identita}
	 \item $\varphi_*:T_s^r(V) \to T_s^r(W)$ è un isomorfismo e $(\varphi_*)^{-1} = (\varphi^{-1})_*$; \label{ppfi:Isomorfismo}
	 \item se $t_1\in T_{s_1}^{r_1}(V)$, $t_2\in T_{s_2}^{r_2}(V)$, allora $\varphi_*(t_1\otimes t_2) = \varphi_*(t_1)\otimes \varphi_*(t_2)$. \label{ppfi:ProdottoTensore}
	\end{enumerate}
\end{proposition}
\begin{proof}
	Dimostriamo innanzitutto il punto \ref{ppfi:Distributiva}. Abbiamo infatti che
	\begin{multline*}
		\psi_*(\varphi_*(t)) (\gamma^1,\ldots,\gamma^r,g_1,\ldots,g_s) = \varphi_*(t) (\psi^*(\gamma^1),\ldots,\psi^*(\gamma^r), \psi^{-1}(g_1),\ldots, \psi^{-1}(g_s)) =\\
		= t(\varphi^*\circ \psi^*(\gamma^1),\ldots, \varphi^*\circ \psi^*(\gamma^r), \varphi^{-1}\circ \psi^{-1}(g_1), \ldots, \varphi^{-1}\circ \psi^{-1}(g_s) ) =\\
		= t((\psi\circ\varphi)^*(\gamma^1),\ldots, (\psi\circ\varphi)^*(\gamma^r), (\psi\circ\varphi)^{-1}(g_1), \ldots, (\psi\circ\varphi)^{-1}(g_s) ) = \\
		= (\psi\circ\varphi)_*\ t (\gamma^1,\ldots,\gamma^r,g_1,\ldots,g_s)\virgola
	\end{multline*}
	con $\gamma^i\in Z^*$ e $g_j\in Z$.
	Poi la \ref{ppfi:Identita} è ovvia, la \ref{ppfi:Distributiva} implica la \ref{ppfi:Isomorfismo} e la \ref{ppfi:ProdottoTensore} segue dalla definizione di prodotto tensore.
\end{proof}

\begin{definition} \index{isomorfismo!pull-back}
	Dato $\varphi \in \Lin(V,W)$ isomorfismo, $(\varphi^{-1})_* \in \Lin(T_s^r(W), T_s^r(V))$ è detto \emph{pull-back} di $\varphi$ ed è indicata con $\varphi^*$.
\end{definition}

\begin{proposition}
	Sia $\varphi\in \Lin(V,W)$ un isomorfismo e siano $e_1,\ldots,e_n,f_1,\ldots,f_n$ basi di $V$ e $W$.
	Sia $A_i^a$ la matrice dei coefficienti di $\varphi$ rispetto alle basi (cioè $\varphi(e_i) = A_i^af_a$).
	Sia $B_a^i$ la matrice dei coefficienti di $\varphi^{-1}$.
	Allora $[B_a^i]$ è l'inversa di $[A_i^a]$. Inoltre, siano $t\in T_s^r(V)$, $q\in T_s^r(W)$ tensori con componenti $t_{j_1 \ldots j_s}^{i_1 \ldots i_r}$ e $q_{b_1 \ldots b_s}^{a_1 \ldots a_r}$, allora
	\begin{equation*} 
		(\varphi_*t)_{b_1 \ldots b_s}^{a_1 \ldots a_r} = A_{i_1}^{a_1}\ldots A_{i_r}^{a_r}\ t_{j_1\ldots j_s}^{i_1 \ldots i_r}\ B_{b_1}^{j_1}\ldots B_{b_s}^{j_s} \virgola
	\end{equation*}
	\begin{equation*}
		(\varphi^*q)_{j_1 \ldots j_s}^{i_1 \ldots i_r} =B_{a_1}^{i_1}\ldots B_{a_r}^{i_r}\ q_{b_1 \ldots b_s}^{a_1 \ldots a_r}\ A_{j_1}^{b_1}\ldots A_{j_s}^{b_s} \punto
	\end{equation*}
\end{proposition}

\begin{proof}
	Facilmente vale che
	\begin{equation*}
		e_i = \varphi^{-1}(\varphi(e_i)) = \varphi^{-1}(A_i^af_a) = A_i^a\varphi^{-1}(f_a) = A_i^a B_a^j e_j \virgola
	\end{equation*}
	perciò $A_i^a B_a^j = \delta_i^j$. E analogamente si vede che $A_i^bB_a^i = \delta_a^b$. 
	Vediamo ora che
	\begin{align*}
		(\varphi_*t)_{b_1 \ldots b_s}^{a_1 \ldots a_r} &= (\varphi_*t)(f^{a_1},\ldots, f^{a_r},f_{b_1},\ldots,f_{b_s}) =\\
		&= t(\varphi^*(f^{a_1}),\ldots, \varphi^*(f^{a_r}),\varphi^{-1}(f_{b_1}),\ldots,\varphi^{-1}(f_{b_s})) =\\
		&=t (A_{i_1}^{a_1} e^{i_1}, \ldots, A_{i_r}^{a_r} e^{i_r},B_{b_1}^{j_1}e_{j_1}, \ldots, B_{b_s}^{j_s}e_{j_s})
	\end{align*}
	E per linearità otteniamo l'enunciato, mentre l'altra formula si ricava nello stesso modo.
\end{proof}

A questo punto, come preannunciato, definiamo push-forward e pull-back per tutte le trasformazioni nel caso di tensori solo controvarianti o solo covarianti.

\begin{definition} \index{tensore!covariante!pull-back} \index{tensore!controvariante!push-forward}
	Sia $\varphi\in \Lin(V,W)$ (non necessariamente invertibile). Definiamo il suo \emph{pull-back} $\varphi^*\in \Lin(T_s^0(W),T_s^0(V))$ tramite $\varphi^*t(e_1,\ldots,e_s) = t(\varphi(e_1),\ldots,\varphi(e_s))$ con $t\in T_s^0(W)$.
	
	Analogamente si definisce il \emph{push-forward} per tensori che siano solo controvarianti.
\end{definition}

Vediamo ora l'analogo della \cref{prop:ProprietaPushForwardIsomorfismi}, dove però il pull-back è definito per tensori solo covarianti. %TODO: tranne che la proposizione sopra lo faceva con il push-forward e così è un po' un casino

\begin{proposition}
	Siano $\varphi\in \Lin(V,W)$ e $\psi\in \Lin(W,Z)$, allora
	\begin{enumerate}
		\item $(\psi\circ\varphi)^*=\varphi^*\circ\psi^*$;
		\item $i^*:T_s^0(V) \to T_s^0(V)$ è l'identità;
		\item se $\varphi$ è un isomorfismo, lo è anche $\varphi^*$ e $(\varphi^*)^{-1} = (\varphi^{-1})^*$;
		\item se $t_1\in T_{s_1}^0(V)$, $t_2\in T_{s_2}^0(V)$, allora $\varphi^*(t_1\otimes t_2) = \varphi^*(t_1)\otimes \varphi^*(t_2)$.
	\end{enumerate}
\end{proposition}


