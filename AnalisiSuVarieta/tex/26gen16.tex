\chapter{26 gennaio 2016}

\section{Tensori}
Dati $V_1,\dots,V_k,W$ spazi vettoriali, denotiamo con $L^k(V_1,\dots,V_k,W)$ lo spazio delle mappe $k$-multilineari da $V_1\times\dots\times V_k$ in $W$.

Denotiamo inoltre $V^* = L(V,\R)$ lo spazio duale di $V$.
Se $V$ è $n$-dimensionale ed $e_1,\dots,e_n$ è una base di $V$, esiste una base $e^1,\dots,e^n$ di $V^*$ tale che $<e^i,e_j> = \delta_{ij}$ \footnote{Dati $\alpha \in V^*$ e $v\in V$, indichiamo con $<\alpha, v>$ l'elemento $\alpha$ applicato a $v$.}.

In particolare, dati $v\in V$ e $\alpha \in V^*$, abbiamo che $v= \sum_{i=1}^n <e^i,v> e_i$ e $\alpha= \sum_{i=1}^n <\alpha,e_i> e^i$.

\begin{definition}
	Dato $V$ spazio vettoriale definiamo 
	\begin{equation*}
	T_s^r(V) = L^{r+s} (\underbrace{V^*,\dots,V^*}_{\text{$r$ copie}},\underbrace{V,\dots,V}_{\text{$s$ copie}},\R) \punto
	\end{equation*}
	
	Gli elementi di $T_s^r(V)$ sono detti \emph{tensori} su $V$, \emph{controvarianti} di ordine $r$ e \emph{covarianti} di ordine $s$ (tensori di tipo $(r,s)$).
\end{definition}

\begin{remark}
	Allo stesso modo possiamo definire $T_s^r(V,W) = L^{r+s} (V^*,\dots,V^*,V,\dots,V,W)$.
\end{remark}


\begin{definition}
	Dati $t_1\in T_{s_1}^{r_1}(V)$ e $t_2\in T_{s_2}^{r_2}(V)$, il \emph{prodotto tensore} di $t_1$ e $t_2$ è il tensore $t_1\otimes t_2 \in T_{s_1+s_2}^{r_1+r_2}(V)$ definito come 
	\begin{multline*}
		t_1\otimes t_2(\beta^1,\dots,\beta^{r_1},\gamma^1,\dots,\gamma^{r_2}, f_1,\dots,f_{s_1},g_1,\dots,g_{s_2}) =\\
		=t_1(\beta^1,\dots,\beta^{r_1},f_1,\dots,f_{s_1})t_2(\gamma^1,\dots,\gamma^{r_2},g_1,\dots,g_{s_2}) \virgola
	\end{multline*}
	con $\beta^j,\gamma^j\in V^*$ e $f_i,g_i\in V$.
\end{definition}

\begin{example}
	Vediamo alcuni casi interessanti:
	\begin{enumerate}
		\item $T_0^1(V) = L(V^*,\R) \cong V$;
		\item $T_1^0(V^*) \cong V^*$;
		\item $T_2^0(V) \cong L(V,V^*)$ tramite $t(v,w) = <f(v),w>$, con $t\in T_2^0(V)$;
		\item $T_1^1(V) \cong L(V,V)$ tramite $t(\alpha,v) = <\alpha, \tilde f (v)>$, con $\tilde f\in L(V,V)$.
	\end{enumerate}
\end{example}

\begin{proposition}
	Sia $V$ uno spazio vettoriale $n$-dimensionale con base $e_1,\dots,e_n$. Sia $e^1,\dots,e^n$ la base duale di $V^*$. Allora elementi del tipo $e_{i_1}\otimes \dots \otimes e_{i_r}\otimes e^{j_1}\otimes \dots \otimes e^{j_s}$ formano una base di $T_s^r(V)$
\end{proposition}
\begin{proof}
	Dobbiamo dimostrare che elementi come sopra sono linearmente indipendenti e generano linearmente $T_s^r(V)$.
	
	Supponiamo che non valga la lineare indipendenza, allora esiste una loro combinazione lineare a coefficienti non nulli che si annulla:
	\begin{equation*}
		\sum t_{j_1\dots j_s}^{i_1\dots i_r} e_{i_1}\otimes \dots \otimes e_{i_r}\otimes e^{j_1}\otimes \dots \otimes e^{j_s} = 0\punto
	\end{equation*}
	Applichiamo questo tensore a $(e^{k_1},\dots,e^{k_r},e_{l_1},\dots,e_{l_s})$, allora otteniamo $t_{l_1\dots l_s}^{k_1 \dots k_r} = 0$ e quindi un assurdo.
	
	Dato ora $t\in T_s^r(V)$, possiamo scrivere $t = \sum t(e^{i_1},\dots, e^{i_r},e_{j_1},\dots, e_{j_s})e_{i_1}\otimes \dots \otimes e_{i_r}\otimes e^{j_1}\otimes \dots \otimes e^{j_s}$, da cui abbiamo anche la ``generazione''.
\end{proof}

\begin{definition}
	I coefficienti $t_{j_1\dots j_s}^{i_1\dots i_r}$ sono detti \emph{componenti} di $t$ rispetto alla base $e_1,\dots,e_n$.
\end{definition}

\begin{example}
\begin{enumerate}
	\item Se $t\in T_2^0(V)$, le componenti sono $t_{ij} = t(e_i,e_j)$ (matrice $n\times n$). A questa matrice associamo la forma bilineare 
	Nel caso di $\R^2$, se $t_{ij} = \left(\begin{matrix}
	                                  A & B \\
	                                  C & D 
	                                 \end{matrix}\right)$
	, associamo la forma $t(x,y) = Ax_1y_1+Bx_1y_2+Cx_2y_1+Dx_2y_2$.
	$t$ è detto \emph{simmetrico} se $t(e_i,e_j) =t(e_j,e_i)$. In questo caso anche la matrice associata è simmetrica. Questo genera la forma quadratica $Q(e) = t(e,e)$, dalla polarizzata $t(e_i,e_j) = \frac 14 [Q(e_i+e_j)-Q(e_i-e_j)]$.
	\item In generale $t\in T_0^r(V)$ è detto \emph{simmetrico} se $t(\alpha^1,\dots,\alpha^r) = t(\alpha^{\sigma(1)},\dots,\alpha^{\sigma(r)})$ per ogni permutazione $\sigma$ e per tutti gli $\alpha^1,\dots,\alpha^r \in V^*$.
	
	A $t$ possiamo ancora associare il polinomio $P(\alpha) = t(\alpha,\dots,\alpha)$, omogeneo di grado $r$.
	\item Data $\sigma$ permutazione dei $\{1,\dots,k\}$, lo spazio $L^k(V_1,\dots,V_k,W)$ è isomorfo allo spazio $L^k(V_{\sigma(1)},\dots,V_{\sigma(k)},W)$.
	Tale isomorfismo si vede come $A \in L^k(V_1,\dots,V_k,W) \mapsto A'\in L^k(V_{\sigma(1)},\dots,V_{\sigma(k)},W)$ tale che $A'(e_{\sigma(1)},\dots,e_{\sigma(k)}) = A(e_1,\dots,e_k)$.
\end{enumerate}
\end{example}

\begin{definition} \index{prodotto!interno}
	Definiamo il \emph{prodotto interno} di un tensore con un vettore $v\in V$, che è una mappa $T_s^r(V)\to T_{s-1}^r(V)$ tale che
	\begin{equation*}
		i_vt(\beta^1,\dots,\beta^r,v_1,\dots,v_{s-1}) = t(\beta^1,\dots,\beta^r,v,v_1,\dots,v_{s-1})\punto
	\end{equation*}
	Analogamente possiamo definire il prodotto interno con un vettore $\beta \in V^*$ come una mappa $T_s^r(V)\to T_{s}^{r-1}(V)$ tale che
	\begin{equation*}
		i^\beta t(\beta^1,\dots,\beta^{r-1},v_1,\dots,v_s) = t(\beta,\beta^1,\dots,\beta^r,v_1,\dots,v_{s-1})\punto
	\end{equation*}
\end{definition}

\begin{remark}
	Le mappe $it:T_s^r(V)\to T_{s-1}^r(V)$ e $i^\beta:T_s^r(V)\to T_s^{r-1}(V)$ sono lineari, così come $v\mapsto i_v$ e $\beta\mapsto i^\beta$.
\end{remark}

In componenti (per linearità basta capirlo sui tensori della base)
\begin{equation*}
	i_{e_k} (e_{i_1}\otimes \dots \otimes e_{i_r}\otimes e^{j_1}\otimes \dots\otimes e^{j_s}) = \delta_k^{j_1} e_{i_1}\otimes \dots \otimes e_{i_r}\otimes e^{j_2}\otimes \dots\otimes e^{j_s}\virgola
\end{equation*}
\begin{equation*}
	i^{e^k} (e_{i_1}\otimes \dots \otimes e_{i_r}\otimes e^{j_1}\otimes \dots\otimes e^{j_s}) = \delta_{i_1}^k e_{i_2}\otimes \dots \otimes e_{i_r}\otimes e^{j_1}\otimes \dots\otimes e^{j_s}\punto
\end{equation*}

\begin{definition} \index{contrazione}
	La \emph{contrazione} del $k$-esimo indice controvariante con l'$l$-esimo indice covariante (contrazione di tipo $(k,l)$) è la mappa lineare $C_l^k:T_s^r(V)\to T_{s-1}^{r-1}(V)$ definita da \footnote{ Per convenzione quando si scrivono indici alti e bassi ripetuti  si intende che si somma da $1$ ad $n$ (notazione di Einstein).}
	\begin{multline*}
		C_l^k(t_{j_1\dots j_s}^{i_1\dots i_r} e_{i_1}\otimes \dots \otimes e_{i_r}\otimes e^{j_1}\otimes \dots\otimes e^{j_s}) =\\
		=t_{j_1\dots j_{l-1}Pj_{l+1}\dots j_s}^{i_1\dots i_{k-1}Pi_{k+1}\dots i_r}e_{i_1}\otimes \dots \otimes\hat e_{i_k}\otimes \dots\otimes e_{i_r}\otimes e^{j_1}\otimes \dots\otimes \hat e^{j_l}\otimes \dots \otimes e^{j_s}\punto
	\end{multline*}
\end{definition}

\begin{remark}
	La contrazione $(k,l)$ è indipendente dalla base $e_1,\dots,e_n$ di $V$.
\end{remark}

\begin{definition}
	La delta di Kronecker è il tensore $\delta \in T_1^1(V)$ definito come $\delta(\alpha,e) = <\alpha,e>$.
\end{definition}

\begin{remark}
	La $\delta$ corrisponde all'identità su $V$ quando consideriamo $T_1^1(V) \cong L(V,V)$.
	
	Fissando una base $\delta = \sum_{i,j}\delta_j^i e_i\otimes e^j$, con $\delta_j^i$ i classici simboli di Kronecker.
\end{remark}

Supponiamo che $V$ sia dotato di un prodotto interno $\ll \cdot, \cdot \gg$ (simmetrico, definito positivo). Siano $e_1,\dots,e_n$ una base di $V$ e $e^1,\dots,e^n$ la base del duale $V^*$.
Sia $g_{ij} = \ll e_i,e_j\gg$ la matrice dei coefficienti del prodotto interno.

Allora abbiamo due isomorfismi (musicali) $b:V\to V^*$ (bemolle) tale che $x\mapsto \ll x,\cdot \gg$ e l'inverso $d:V^*\to V$ (diesis).

La matrice dei coefficienti di $b$ è $g_{ij}$ stessa ($(x^b)_i = g_{ij}x^j$).
Invece la matrice dei coefficienti di $d$ è $g^{ij}$, cioè la matrice inversa di $g_{ij}$ ($(\alpha^d)^i = g^{ij}\alpha_j$).

$b$ è detto operatore di abbassamento di indice, mentre $d$ operatore di innalzamento. Questo perché permettono di passare da tensori del tipo $(r,s)$ a tensori del tipo $(r-1,s+1)$ e $(r+1,s-1)$ rispettivamente.

\begin{example}
	Se $t$ è di tipo $(0,2)$, possiamo definire $t'\in T_1^1(V)$ tramite $t'(\alpha, e) = t(\alpha^d, e)$ e avremo che $(t')_j^i = g^{ik}t_{kj}$.
\end{example}

