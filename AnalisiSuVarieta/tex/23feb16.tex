\chapter{23 febbraio 2016}






\section{Determinanti e volumi}

Il determinante di una matrice è multilineare e antisimmetrico su righe e colonne di una matrice.
Se $\seqb xn, \in\R^n$ hanno componenti $x_i^j$, allora il determinante di $(x_i^j)$ è il volume del parallelepipedo generato da $\seqb xn,$.
Possiamo indicare
\begin{equation*}
	\det[\seqb xn,] = \sum_{\sigma\in S_n} \sgn(\sigma) x_{\sigma(1)}^1\ldots x_{\sigma(n)}^n \punto
\end{equation*}
Se $\varphi:\R^n\to\R^n$ è lineare $\det\varphi = \frac{\text{volume del parallelepipedo generato da $\varphi(e_1),\ldots,\varphi(e_n)$}}{\text{volume del cubo unitario}}$.

\begin{proposition}
	Sia $\varphi\in\Lin(V,W)$, allora $\varphi^* : T^0_k(W) \to T^0_k(V)$ è lineare e valgono 
	\begin{enumerate}
		\item $\varphi^*(\Lambda^k(W)) \subseteq \Lambda^k(V)$;
		\item se $\psi\in\Lin(W,Z)$, $(\psi\circ\varphi)^* = \varphi^* \circ \psi^*$;
		\item $(\id_V)^* = \id_{T^0_k(V)}$;
		\item se $\varphi\in \mathrm{GL}(V,W)$, allora $\varphi^*\in \mathrm{GL}(T^0_k(W), T^0_k(V))$ e $(\varphi^*)^{-1} = (\varphi^{-1})^* = \varphi_*$. %TODO: GL era chiamato inv
		Se anche $\psi\in\mathrm{GL}(W,Z)$, allora $(\psi\circ\varphi)_* = \psi_*\circ \varphi_*$;
		\item se $\alpha\in\Lambda^k(V)$, $\beta\in\Lambda^l(W)$, allora $\varphi^*(\alpha\wedge\beta) = (\varphi^*\alpha) \wedge (\varphi^*\beta)$.
	\end{enumerate}
\end{proposition}

Ricordiamo che $\dim(\Lambda^n(V)) = 1$, dove $\dim(V)=n$.
\begin{definition}
	Se $V$ ha dimensione $n$ e $\varphi\in\Lin(V,V)$, $\det(\varphi)$ è definito come l'unica costante tale che $\varphi^*\omega = \det(\varphi) \omega$, qualsiasi $\omega\in\Lambda^n(V)$.
\end{definition}

\begin{proposition}
	Sia $\seqb en,$ una base di $B$ con base duale $\seqa en,$. Sia $\varphi(e_i) = A_i^je_j$.
	Dalla dimostrazione dell'associatività di $\Lambda$
	\begin{align*}
		\varphi^*(\seqa en\wedge) (\seqb en,) &= (\seqa en\wedge)(\varphi(e_1),\ldots, \varphi(e_n)) =\\
		&=n!\ \antisimm (\seqa en\otimes)(\varphi(e_1),\ldots,\varphi(e_n)) = \det A
		\punto
	\end{align*}
	(multilineare e normalizzata a 1 sulla mappa identica)
\end{proposition}

\begin{proposition}
	Siano $\varphi,\psi \in\Lin(V,V)$. Allora
	\begin{enumerate}
		\item $\det(\varphi\circ\psi) = \det \varphi \cdot \det \psi$;
		\item $\det \id = 1$;
		\item $\varphi$ è un isomorfismo se e solo se $\det\varphi\not=0$. In tal caso $\det\varphi^{-1} = (\det \varphi)^{-1}$.
	\end{enumerate}
\end{proposition}
\begin{proof}
	L'unica non ovvia è la freccia $\Leftarrow$ della 3.
	
	Basta dimostrare che se $\varphi$ non è un isomorfismo, allora $\det\varphi = 0$.
	Se $\varphi$ non è un isomorfismo, esiste $e=e_1\in\ker\varphi$. Completo $e_1$ a una base $\{\seqb en,\}$ di $V$. Data $\omega\in\Lambda^n(V)$, allora
	\begin{equation*}
		(\varphi^*\omega)(\seqb en,) = \omega(\varphi(e_1),\ldots,\varphi(e_n)) = \omega(0,\varphi(e_2),\ldots,\varphi(e_n)) = 0
	\end{equation*}
	perché $\omega$ è multilineare. Perciò $\det\varphi = 0$.
\end{proof}

\begin{definition} \index{orientazione}
	Gli elementi di $\Lambda^n(V)$ sono detti \emph{elementi di volume}. Se $\omega_1,\omega_2$ sono elementi di volume diciamo che $\omega_1\sim \omega_2$ se esiste $c>0$ tale che $\omega_1 = c\omega_2$. Una classe di equivalenza di elementi di volume è detta un'\emph{orientazione} su $V$.
	
	Uno \emph{spazio orientato} $(V, [\omega])$ è uno spazio vettoriale con un'orientazione $[\omega]$. La classe $[-\omega]$ è detta \emph{orientazione inversa}.
	
	Una base $\{\seqb en,\}$ di $(V,[\omega])$ orientato è detta essere orientata positivamente se $\omega(\seqb en,)>0$ \footnote{Ovviamente tale definizione è indipendente dalla scelta di un elemento di $[\omega]$.}.
\end{definition}

\begin{proposition}
	Sia $g\in T^0_2(V)$ simmetrico e definito positivo. Allora esiste una base $\{\seqb en,\}$ di $V$ (con base duale $\{\seqa en,\}$) tale che $g = \sum_{i=1}^n e^i\otimes e^i$. Questa base è detta \emph{ortogonale} rispetto a $g$.
\end{proposition}
\begin{proof}
	La dimostrazione è analoga al caso di $\R^n$ utilizzando un processo di ortogonalizzazione di Gram-Schmidt.
	
	Dalla polarizzata $g(e,f) = \frac 14 [g(e+f,e+f) + g(e-f,e-f)]$.
	Sia $e_1$ tale che $g(e_1,e_1) = 1$. Sia $V_1 = \spanrm \{e_1\}$ e sia $V_2 = \{e \suchthat g(e,e_1) = 0\}$.
	Visto che $g$ è definita positiva, $V_1\cap V_2= \{0\}$. Dato $z\in V$, allora $z = g(e_1,z)e_1 + (z-g(e_1,z)e_1)$, dove il secondo termine appartiene a $V_2$. Quindi $V = V_1\oplus V_2$.
	Scegliamo quindi $e_2\in V_2$ tale che $g(e_2,e_2)=1$ e così via fino a completare la base.
\end{proof}


\begin{proposition}
	Sia $V$ uno spazio vettoriale $n$ dimensionale e sia $g\in T^0_2(V)$ simmetrico, definito positivo. Allora, se $[\omega]$ è un'orientazione in $V$, esiste un unico elemento di volume $\mu(g) \in [\omega]$ (detto $g$-volume) tale che $\mu(g)(\seqb en,) = 1$ per ogni base ortonormale e orientata $\seqb en,$ di $V$.
	Se poi $\seqa en,$ è la base duale, allora $\mu(g) = \seqa en\wedge$.
	
	In generale, se $\seqb fn,$ è una base orientata positivamente e se $f^i$ è la base duale, allora
	\begin{equation*}
		\mu(g) = \abs{ \det[g(f_i,f_j)] }^{\frac 12}\ \seqa fn\wedge \punto
	\end{equation*}
\end{proposition}
\begin{proof}
	Per la proposizione precedente $g(f_i,f_j) = \sum_p e^p\otimes e^p (f_i,f_j)$. Esiste quindi $\varphi \in \mathrm{inv}(V,V)$ tale che $f_i = \varphi(e_i) = A_i^j(e_j)$.
	Perciò $g(f_i,f_j) = \sum_p e^p\otimes e^p (A_i^k e_k, A_i^le_l) = \sum_p \delta_k^p \delta_l^p A_i^k A_j^l = \sum_pA_i^pA_j^p$.
	Quindi $\det[g(f_i,f_j)] = (\det\varphi)^2\det[g(e_i,e_j)] = (\det\varphi)^2$.
	Se $\{\seqb en,\}$ è orientata positivamente e $g$-ortogonale $\mu(\seqb en,) = 1$ determina $\mu$ univocamente.
	
	Se $\seqb fn,$ è un'altra base orientata positivamente e $g$-ortogonale e sia $\varphi$ come sopra. Allora $\abs{\det\varphi} = 1$. Ma $0<\mu(\seqb fn,) = (\varphi^*\mu)(\seqb en,)\det\varphi$, quindi $\det\varphi=1$.
	
	La 3 implica la 2.
	
	Per la formula si usa $\mu(\seqb fn,) = \det\varphi = \abs{ \det[g(f_i,f_j)] }^\frac 12$.
\end{proof}































































