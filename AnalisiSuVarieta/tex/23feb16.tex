\chapter{23 febbraio 2016}






\section{Determinanti e volumi}

Il determinante di una matrice è multilineare e antisimmetrico su righe e colonne di una matrice.
Se $\seqb xn, \in\R^n$ hanno componenti $x_i^j$, allora il determinante di $(x_i^j)$ è il volume del parallelepipedo generato da $\seqb xn,$.
Possiamo indicare
\begin{equation*}
	\det[\seqb xn,] = \sum_{\sigma\in S_n} \sgn(\sigma) x_{\sigma(1)}^1\ldots x_{\sigma(n)}^n \punto
\end{equation*}
Se $\varphi:\R^n\to\R^n$ è lineare $\det\varphi = \frac{\text{volume del parallelepipedo generato da $\varphi(e_1),\ldots,\varphi(e_n)$}}{\text{volume del cubo unitario}}$.

Rivediamo ora le proprietà del pull-back nel caso particolare di forme differenziali o più in generale di tensori solo covarianti.
\begin{proposition}
	Sia $\varphi\in\Lin(V,W)$, allora $\varphi^* : T^0_k(W) \to T^0_k(V)$ è lineare e valgono 
	\begin{enumerate}
		\item $\varphi^*(\Lambda^k(W)) \subseteq \Lambda^k(V)$;
		\item se $\psi\in\Lin(W,Z)$, $(\psi\circ\varphi)^* = \varphi^* \circ \psi^*$;
		\item $(\id_V)^* = \id_{T^0_k(V)}$;
		\item se $\varphi\in \mathrm{GL}(V,W)$, allora $\varphi^*\in \mathrm{GL}(T^0_k(W), T^0_k(V))$ e $(\varphi^*)^{-1} = (\varphi^{-1})^* = \varphi_*$. %TODO: GL era chiamato inv
		Se anche $\psi\in\mathrm{GL}(W,Z)$, allora $(\psi\circ\varphi)_* = \psi_*\circ \varphi_*$;
		\item se $\alpha\in\Lambda^k(V)$, $\beta\in\Lambda^l(W)$, allora $\varphi^*(\alpha\wedge\beta) = (\varphi^*\alpha) \wedge (\varphi^*\beta)$.
	\end{enumerate}
\end{proposition}

Ricordiamo che $\dim(\Lambda^n(V)) = 1$, dove $\dim(V)=n$, per cui ha senso la seguente definizione.
\begin{definition} \index{determinante}
	Se $V$ ha dimensione $n$ e $\varphi\in\Lin(V,V)$, il \emph{determinante} $\det(\varphi)$ di $\varphi$ è definito come l'unica costante tale che $\varphi^*\omega = \det(\varphi) \omega$, per qualsiasi $\omega\in\Lambda^n(V)$.
\end{definition}

\begin{proposition}
	Sia $\seqb en,$ una base di $B$ con base duale $\seqa en,$. Sia $\varphi \in \Lin(V,V)$ tale che $\varphi(e_i) = A_i^je_j$.
	Allora
	\begin{equation*}
		\varphi^*(\seqa en\wedge) (\seqb en,) = \det A \punto %TODO: è questa la conclusione, giusto?
	\end{equation*}
\end{proposition}
\begin{proof}	
	Dalla dimostrazione del punto \ref{pw:Associativa} della \cref{prop:ProprietaWedge}, abbiamo
	\begin{align*}
		\varphi^*(\seqa en\wedge) (\seqb en,) &= (\seqa en\wedge)(\varphi(e_1),\ldots, \varphi(e_n)) =\\
		&=n!\ \antisimm (\seqa en\otimes)(\varphi(e_1),\ldots,\varphi(e_n)) = \det A
		\punto
	\end{align*}
	%(multilineare e normalizzata a 1 sulla mappa identica) %TODO: che vuol dire questa frase?
\end{proof}

\begin{proposition}
	Siano $\varphi,\psi \in\Lin(V,V)$. Allora
	\begin{enumerate}
		\item $\det(\varphi\circ\psi) = \det \varphi \cdot \det \psi$;
		\item $\det (\id_V) = 1$;
		\item $\varphi$ è un isomorfismo se e solo se $\det\varphi\not=0$ e in tal caso $\det(\varphi^{-1}) = (\det \varphi)^{-1}$.
	\end{enumerate}
\end{proposition}
\begin{proof}
	L'unica dimostrazione non ovvia è quella del fatto che se $\det\varphi\not=0$, allora $\varphi$ è un isomorfismo.
% 	L'unica non ovvia è la freccia $\Leftarrow$ della 3.
	Basta dimostrare che se $\varphi$ non è un isomorfismo, allora $\det\varphi = 0$.
	
	Se $\varphi$ non è un isomorfismo, esiste $e=e_1\in\ker\varphi$. Completiamo $e_1$ ad una base $\{\seqb en,\}$ di $V$. Data $\omega\in\Lambda^n(V)$, allora
	\begin{equation*}
		(\varphi^*\omega)(\seqb en,) = \omega(\varphi(e_1),\ldots,\varphi(e_n)) = \omega(0,\varphi(e_2),\ldots,\varphi(e_n)) = 0 \virgola
	\end{equation*}
	perché $\omega$ è multilineare. Perciò $\det\varphi = 0$, come voluto.
\end{proof}

\begin{definition} \index{elementi di volume}\index{orientazione} \index{spazio orientato}
	Gli elementi di $\Lambda^n(V)$ sono detti \emph{elementi di volume}. Se $\omega_1,\omega_2$ sono elementi di volume diciamo che $\omega_1\sim \omega_2$ se esiste $c>0$ tale che $\omega_1 = c\omega_2$. Una classe di equivalenza di elementi di volume è detta un'\emph{orientazione} su $V$.
	
	Uno \emph{spazio orientato} $(V, [\omega])$ è uno spazio vettoriale con un'orientazione $[\omega]$. La classe $[-\omega]$ è detta \emph{orientazione inversa}.
	
	Una base $\{\seqb en,\}$ di $(V,[\omega])$ orientato è detta essere orientata positivamente se $\omega(\seqb en,)>0$. Ovviamente tale definizione è indipendente dalla scelta di un elemento di $[\omega]$.
\end{definition}

\begin{proposition} \index{base!ortonormale} \label{prop:EsistenzaBaseOrtonormale}
	Sia $g\in T^0_2(V)$ simmetrico e definito positivo. Allora esiste una base $\{\seqb en,\}$ di $V$ con base duale $\{\seqa en,\}$ tale che $g = \sum_{i=1}^n e^i\otimes e^i$. Questa base è detta \emph{ortonormale} rispetto a $g$.
\end{proposition}
\begin{proof} %TODO: sistemare
	La dimostrazione è analoga al caso di $\R^n$ utilizzando un processo di ortogonalizzazione di Gram-Schmidt.
	
	Dalla polarizzata $g(e,f) = \frac 14 [g(e+f,e+f) + g(e-f,e-f)]$.
	Sia $e_1$ tale che $g(e_1,e_1) = 1$. Sia $V_1 = \spanrm \{e_1\}$ e sia $V_2 = \{e \suchthat g(e,e_1) = 0\}$.
	Visto che $g$ è definita positiva, $V_1\cap V_2= \{0\}$. Dato $z\in V$, allora $z = g(e_1,z)e_1 + (z-g(e_1,z)e_1)$, dove il secondo termine appartiene a $V_2$. Quindi $V = V_1\oplus V_2$.
	Scegliamo quindi $e_2\in V_2$ tale che $g(e_2,e_2)=1$ e così via fino a completare la base.
\end{proof}


\begin{proposition} \index{$g$-volume}
	Sia $V$ uno spazio vettoriale $n$ dimensionale e sia $g\in T^0_2(V)$ simmetrico, definito positivo. Allora, se $[\omega]$ è un'orientazione in $V$, esiste un unico elemento di volume $\mu(g) \in [\omega]$, detto \emph{$g$-volume}, tale che $\mu(g)(\seqb en,) = 1$ per ogni base $\seqb en,$ di $V$ ortonormale rispetto a $g$ e orientata.
	Se poi $\seqa en,$ è la base duale, allora $\mu(g) = \seqa en\wedge$.
	
	In generale, se $\seqb fn,$ è una base orientata positivamente e se $f^i$ è la base duale, allora
	\begin{equation*}
		\mu(g) = \abs{ \det[g(f_i,f_j)] }^{\frac 12}\ \seqa fn\wedge \punto
	\end{equation*}
\end{proposition}
\begin{proof}
	Sia $\{\seqb en,\}$ una base ortonormale rispetto a $g$, che ci è garantita dalla \cref{prop:EsistenzaBaseOrtonormale}. Allora $g(f_i,f_j) = \sum_{p=1}^n e^p\otimes e^p (f_i,f_j)$.
	
	Se $\varphi \in \mathrm{GL}(V,V)$ è tale che $f_i = \varphi(e_i) = A_i^j(e_j)$, allora
	\begin{equation*}
	g(f_i,f_j) = \sum_{p=1}^n e^p\otimes e^p (A_i^k e_k, A_j^le_l) = \sum_{p=1}^n \delta_k^p\ \delta_l^p\ A_i^k\ A_j^l = \sum_{p=1}^nA_i^pA_j^p \punto
	\end{equation*}
	Perciò abbiamo $\det[g(f_i,f_j)] = (\det\varphi)^2\det[g(e_i,e_j)] = (\det\varphi)^2$.
	
	Se ora $\{\seqb en,\}$ è orientata positivamente e $g$-ortogonale, allora per la richiesta di multilinearità esiste un'unica $\mu(g)=\mu$ tale che $\mu(\seqb en,) = 1$. Mostriamo che $\mu(g)$ così definita rispetta le ipotesi richieste.
	
	Se $\seqb fn,$ è un'altra base orientata positivamente e $g$-ortogonale, sia $\varphi$ come sopra. Allora per quanto detto prima $\abs{\det\varphi} = 1$. Inoltre per la positività della base e la definizione di $\varphi$ abbiamo $0<\mu(\seqb fn,) = (\varphi^*\mu)(\seqb en,) = \det\varphi$, quindi $\det\varphi=1$ e in particolare $\mu(\seqb fn,)=1$, come voluto.
	
	La seconda affermazione è implicata dalla terza e per quest'ultima si usa la formula $\mu(\seqb fn,) = \det\varphi = \abs{ \det[g(f_i,f_j)] }^\frac 12$, con le notazioni date all'inizio.
\end{proof}































































