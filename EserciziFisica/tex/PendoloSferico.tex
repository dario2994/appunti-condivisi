\documentclass[../main.tex]{subfiles} 
\begin{document}

\exercise[24/10/2014]{Il pendolo sferico}%ps

\textex

Vogliamo studiare il moto di un punto materiale vincolato sull'estremità di un asta di lunghezza $l$ libera di muoversi nello spazio purché l'altra estremità resti incernierata ad un punto $O$.
In particolare, dato un sistema di coordinate polari tridimensionali centrato nel punto $O$ vogliamo determinare:
\begin{enumerate}
\item L'equazione del moto nella coordinata $\theta$, rappresentante la distanza angolare dal punto di minimo.
\item Quali condizioni iniziali sono necessarie affinché il moto del pendolo sia conico, ossia la coordinata $\theta$ risulti costante nel tempo.
\item La frequenza con cui l'angolo $\theta$ oscilla intorno alla posizione d'equilibrio.
\end{enumerate}

\solution
Le uniche forze agenti sul punto materiale sono il suo peso $m\vec{g}$ e la reazione vincolare dell'asta $-N\hat{r}$ (parallela alla direzione dell'asta perch\'e questa \`e da considerarsi priva di massa).
La somma $F$ di queste forze, proiettata sui tre versori ortogonali $\hat{r}, \hat{\theta}, \hat{\varphi}$, sar\`a:
\begin{itemize}
 \item $\vec{F}\cdot\hat{r}=m\vec{g}\cdot\hat{r}-N\hat{r}\cdot\hat{r}=mg\cos\theta -N$.
 \item $\vec{F}\cdot\hat{\theta}=m\vec{g}\cdot\hat{\theta}-N\hat{r}\cdot\hat{\theta}=-mg\sin\theta$.
 \item $\vec{F}\cdot\hat{\varphi}=m\vec{g}\cdot\hat{\varphi}-N\hat{r}\cdot\hat{\varphi}=0$.
\end{itemize}

Per la \cref{AccCooSferiche}, il valore dell'accelerazione di un punto materiale in termini delle sue coordinate è:
$$ \vec{a}=\ddot{\vec{r}}=(\ddot{r}-r\dot{\theta}^2-r\sin^2{\theta}\dot{\varphi}^2)\hat{r}+(2\dot{r}\dot{\theta}+r\ddot{\theta}-r\sin\theta\cos\theta\dot{\varphi}^2)\hat{\theta}+(2\dot{r}\sin\theta\dot{\varphi}+2r\dot{\theta}\dot{\varphi}\cos\theta+r\ddot{\varphi}\sin\theta)\hat{\varphi} $$

Inoltre la coordinata $r$ del nostro punto materiale \'e costantemente uguale a $l$, sicch\'e le sue derivate temporali saranno nulle, dunque la relazione precedente si esprime

$$\vec{a}=(-l\dot{\theta}^2-l\sin^2{\theta}\dot{\varphi}^2)\hat{r}+(l\ddot{\theta}-l\sin\theta\cos\theta\dot{\varphi}^2)\hat{\theta}+(2l\dot{\theta}\dot{\varphi}\cos\theta+l\ddot{\varphi}\sin\theta)\hat{\varphi}$$


Confrontando questo risultato con le tre componenti della forza $F$ si ottengono le tre equazioni:
\begin{align}
 mg\cos\theta -N & =-m(l\dot{\theta}^2+l\sin^2\theta\dot{\varphi}^2) \label{ps:force1} \\
 -mg\sin\theta & =m(l\ddot{\theta}-l\sin\theta\cos\theta\dot{\varphi}^2) \label{ps:force2} \\
 0 & =2l\dot{\theta}\dot{\varphi}\cos\theta+l\ddot{\varphi}\sin\theta \label{ps:force3}
\end{align}

Si noti che la \cref{ps:force3} \`e equivalente alla conservazione del momento angolare lungo l'asse $z$ parallelo alla direzione della gravit\`a. Infatti 
\begin{equation*}
 \vec{L}=m\vec{r}\wedge\vec{v}=ml\hat{r}\wedge(l\dot{\theta}\hat{\theta}+l\dot{\varphi}\sin\theta\hat{\varphi})=ml^2\dot{\theta}\hat{r}\wedge\hat{\theta}+ml^2\dot{\varphi}\sin\theta\hat{r}\wedge\hat{\varphi}
\end{equation*}
ed ora, essendo $\hat{r}\wedge\hat{\theta}=-\hat{\varphi}\perp\hat{z}$ e $\hat{r}\wedge\hat{\varphi}=\hat{\theta}$, si ha che
$\vec{L}\cdot\hat{z}=ml^2\dot{\varphi}\sin^2\theta$, pertanto 
\begin{equation*}
 \frac{(d\vec{L}\cdot\hat{z})}{dt}=2ml^2\dot{\theta}\dot{\varphi}\sin\theta\cos\theta+ml^2\ddot{\varphi}\sin^2\theta
\end{equation*}
e uguagliando a 0 questa espressione si ottiene la \cref{ps:force3}.

Dalla conservazione del momento angolare, infatti, otteniamo la relazione $\displaystyle \dot{\varphi}=\frac{L_z}{ml^2\sin^2\theta}$ che, una volta sostituita nella \cref{ps:force2} ci restituisce un'equazione differenziale nella variabile $\theta$:
\begin{equation}\label{ps:Differenziale}
g\sin\theta+l\ddot{\theta}-\frac{L_z^2\cos\theta}{m^2l^3\sin^3\theta}=0
\end{equation}

Affinch\`e il moto sia conico le derivate temporali dell'angolo $\theta$ saranno entrambe nulle e dunque il suo valore sar\`a una costante $\theta_0$ che risolve l'equazione seguente (ottenuta dalla \cref{ps:Differenziale}).
\begin{equation}\label{ps:MotoConico}
m^2l^3g\sin^4\theta_0=L_z^2\cos\theta_0
\end{equation}

Studiamo ora le piccole oscillazioni.

Poniamo $\theta=\theta_0+\alpha$ nella \cref{ps:Differenziale}, otteniamo, dunque
$$ g\sin(\theta_0+\alpha)+l\ddot{\alpha}-\frac{L_z^2\cos(\theta_0+\alpha)}{m^2l^3\sin^3(\theta_0+\alpha)}=0 $$
che sviluppata secondo Taylor fino al primo ordine diviene
\begin{equation}\label{ps:Approssimazione}
g\sin(\theta_0)+\alpha g\cos\theta_0+l\ddot{\alpha}-\frac{L_z^2\cos(\theta_0)}{m^2l^3\sin^3(\theta_0)}-\alpha\frac{L_z^2(1-2\cos^2\theta_0)}{m^2l^3\sin^4(\theta_0)}=0
\end{equation}
Per la \cref{ps:MotoConico} la \cref{ps:Approssimazione} viene semplificata come
\begin{equation}
\ddot{\alpha}=\alpha\left({\frac{L_z^2(1-2\cos^2\theta_0)}{m^2l^4\sin^4(\theta_0)}-\frac{g\cos\theta_0}{l}}\right)
\end{equation}
da cui si ricava facilmente che il pendolo oscilla armonicamente con frequenza $$\sqrt{\frac{L_z^2(1-2\cos^2\theta_0)}{m^2l^4\sin^4(\theta_0)}-\frac{g\cos\theta_0}{l}}$$ "attorno alla traiettoria conica".

\solution[2]
Si giunger\`a alla \cref{ps:Differenziale} attraverso la conservazione dell'energia e quella del momento angolare lungo l'asse $z$ senza passare dalle considerazioni cinematiche svolte in precedenza.

La velocit\`a del punto materiale \`e $\vec{v}=\dot{\vec{r}}=l\dot{\theta}\hat{\theta}+l\dot{\varphi}\sin\theta\hat{\varphi}$, per cui la sua energia cinetica \`e $K=l^2\dot{\theta}^2+l^2\dot{\varphi}^2\sin^2\theta$.

La sua energia potenziale \`e $U=-mgl(1-\cos\theta)$, quindi l'energia meccanica del punto \`e
\begin{equation*}
 E=l^2\dot{\theta}^2+l^2\dot{\varphi}^2\sin^2\theta-mgl(1-\cos\theta)
\end{equation*}
ci\`o vuol dire che esiste una costante $c$ per cui vale
\begin{equation}\label{ps:Energie}
c=l^2\dot{\theta}^2+l^2\dot{\varphi}^2\sin^2\theta+mgl\cos\theta
\end{equation}
Essendo $\displaystyle \dot{\varphi}=\frac{L_z}{ml^2\sin^2\theta}$, si ottiene l'equazione desiderata direttamente dalla \cref{ps:Energie}. 



\end{document}
