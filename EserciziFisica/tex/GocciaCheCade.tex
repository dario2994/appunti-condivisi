\documentclass[../main.tex]{subfiles} 
\begin{document}

\exercise[6/02/2014]{Goccia che Cade} %gcc

\textex
Trovare le equazioni del moto di uno goccia d'acqua che precipita attraverso una nuvola, data $\lambda$ la densità di vapor acqueo della nuvola e $\rho$
la densità della goccia.

\solution

La goccia cadendo aumenta la propria massa, si tratta quindi di un problema di masse variabili, la cui equazione generale è \cref{MassaVariabile}.
Chiaramente in questo caso $v-v_{R}=0$ in quanto posso supporre le particelle d'acqua ferme rispetto al sistema del laboratorio e
l'unica forza esterna che interviene è la gravità che posso supporre positiva orientando l'asse delle $y$ verso il basso.
L'equazione che devo risolvere è quindi:
\begin{equation}\label{gcc:1}
\de mv+m\de v=mg\de t 
\end{equation}
Ovvero dividendo tutto per $\de t$:
\begin{equation}\label{gcc:2}
 \dot{m}\dot{y}+m\ddot{y}=mg
\end{equation}
Bisogna ora trovare la relazione tra la velocità e la massa per far questo noto che l'aumento di massa in funzione del tempo può essere espresso in due modi distinti,
nel primo caso considero l'aumento di volume in funzione del raggio, cioè $\dot{m}=\rho 4 \pi r^2 \dot{r}$, oppure l'aumento di massa posso anche vederlo
come il numero di particelle incontrate, ovvero il volume spazzato nell'unità di tempo moltiplicato per la densità delle nuvole, quindi:
$\dot{m}=\lambda \pi r^2 \dot{y}$. 

Sostituendo $\dot{m}$ ottengo che $\dot{y}=4\rho \dot{r}/{\lambda}$ e derivando rispetto al tempo $\ddot{y}=4\rho \ddot{r}/\lambda$. La \cref{gcc:2} diventa quindi
\begin{equation}\label{gcc:3}
 4\rho \pi r^2 \dot{r} (\frac{4 \rho \dot{r}}{\lambda})+\frac{4m\rho \ddot{r}}{\lambda}=mg
\end{equation}
Dividendo per la massa e ricordando che essa è data dal rapporto tra volume e densità si ottiene:
\begin{equation}\label{gcc:4}
 \frac{12\rho}{\lambda}\frac{\dot{r}^2}{r}+\frac{4\rho}{\lambda}\ddot{r}=g
\end{equation}
La cui soluzione è chiaramente nella forma $r=At^2$ dove svolgendo i conti si scopre che $A=\frac{g \lambda}{56 \rho}$.

Infine supponendo che all'istante iniziale la velocità sia nulla e ponendo $y(0)=0$ si ottiene $a=g/7$ e derivando $y=\frac{9}{14}t^2$.


\end{document}
