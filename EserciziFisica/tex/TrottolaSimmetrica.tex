\documentclass[../main.tex]{subfiles} 
\begin{document}

\exercise[17/02/2014]{Moto libero di una trottola simmetrica} %mlts

\textex
Una trottola simmetrica è un corpo rigido per cui valga $I_x=I_y$ dove $\hat x, \hat y, \hat z$ sono gli assi principali del corpo.
Studiare il moto libero (cioè in assenza di forze esterne) di una trottola simmetrica.
\solution
Lavoro nel centro di massa del corpo (è chiaro che il moto traslatorio di questo non è interessante ai fini del problema).

Chiamo $\overrightarrow M$ il momento angolare del corpo e $\vec{\omega}$ l'asse di rotazione istantaneo. Inoltre sfrutterò la notazione di Einstein per le sommatorie, e tutti gli indici varieranno su $x,y,z$.

Per la seconda equazione cardinale della dinamica, poichè sul corpo non agiscono forze esterne, ho che vale:
\begin{equation}\label{mlts:EqCardinale}
	0=\dot{\overrightarrow M}
\end{equation}

Inoltre per la definizione di tensore di inerzia, rispetto agli assi principali, ho anche che vale:
\begin{equation*}
	\overrightarrow M = I_i\omega_i \hat\iota
\end{equation*}
e derivando quest'ultima rispetto al tempo, sfruttando \cref{OmegaSferiche} arrivo a:
\begin{equation*}
	\dot{\overrightarrow M}= I_i\left(\omega'_i\hat\iota + \omega_i\omega\times \hat\iota\right)
\end{equation*}
ora applico \cref{mlts:EqCardinale} e arrivo a
\begin{equation}\label{mlts:Fondamentale}
	0=I_i\left(\omega'_i\hat\iota + \omega_i\omega\times \hat\iota\right) =
	\sum_{cyc} \left( I_x\omega'_x+\omega_y\omega_z(I_z-I_y) \right)\hat x
\end{equation}
dove nell'ultima uguaglianza ho semplicemente fatto i conti e raccolto per coordinata.

Guardando la \cref{mlts:Fondamentale} rispetto alla coordinata $\hat z$ ottengo:
\begin{equation*} 
	0=I_z\omega'_z+\omega_x\omega_y(I_y-I_x)=I_z\omega'_z\Rightarrow \omega'_z=0
\end{equation*}
quindi ottengo $\omega_z$ costante.

Chiamo $\alpha=\frac{I_z}{I_x}=\frac{I_z}{I_y}$ e $k=\omega_z(\alpha-1)$. Per quanto appena detto $k$ è una costante.

Guardo la \cref{mlts:Fondamentale} rispetto alla coordinata $\hat x$ e ottengo:
\begin{equation}\label{mlts:x}
	0=I_x\omega'_x+\omega_y\omega_z(I_z-I_x)\Rightarrow 0=\omega'_x+k\omega_y
\end{equation}
e analogamente, solo guardando la coordinata $\hat y$ ottengo:
\begin{equation}\label{mlts:y}
	0=I_y\omega'_y+\omega_z\omega_x(I_x-I_z)\Rightarrow 0=\omega'_y-k\omega_x
\end{equation}

Ora sommo la \cref{mlts:x} alla \cref{mlts:y} moltiplicata per $i$ ottenendo:
\begin{equation*}
	0=\omega'_x+k\omega_y+i(\omega'_y-k\omega_x)=(\omega_x+i\omega_y)'-ik(\omega_x+i\omega_y)
\end{equation*}
e chiamando $z=\omega_x+i\omega_y$ arrivo ad una differenziale di variabile complessa facile da risolvere:
\begin{equation}\label{mlts:complessa}
	z'=ikz \Rightarrow z=A\cdot e^{ikt+\varphi}
\end{equation}
dove $A,\phi$ sono costanti che dipendono dai dati iniziali.



\end{document}
