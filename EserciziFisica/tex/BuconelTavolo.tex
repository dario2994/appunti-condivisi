\documentclass[../main.tex]{subfiles}

\begin{document}


\exercise[21/11/2013]{Palline legate da un filo} %bt

\textex
Due palline sono legate da un filo lungo $l$ (inestensibile e di massa trascurabile). La prima (massa  $m_1$) \`e appoggiata su un tavolo (liscio) con un buco, la seconda (massa  $m_2$) \`e sotto il buco (attacata al filo). Determinare cosa succede (posizione nel tempo delle due palline o equazioni del moto).

\solution
\begin{enumerate}
  \item Sistema di riferimento: inerziale, origine nel buco, coordinate cilindriche, ossia polari ($r,\theta $) per la pallina sul tavolo, l'asse $z$ ($h=r-l$) per l'altra. La gravit\`a \`e verso il basso in direzione $z$.
  \item Forze esterne: gravit\`a (entrambe le palline), reazione del piano (verticale, verso l'alto, solo la prima pallina, bilancia la gravit\`a). Sono tutte conservative.
  \item Forze interne: tensione del filo; verticale (verso l'alto) per la seconda pallina, radiale (verso il centro) per la prima.
  \item Cosa si conserva:
  \begin{itemize}
    \item somma delle energie meccaniche delle due palline;
    \item momento angolare della prima pallina.
  \end{itemize}
  \item Condizioni iniziali: $r,\dot{\theta }$.
  \item Relazioni varie: la velocit\`a della seconda pallina ($\dot h=v_2$) e la velocit\`a radiale della prima ($\dot r=v_{1,r}$) sono uguali:
  \begin{equation}
    \label{bt:rdot}
    \dot r=v_{1,r}=v_2
  \end{equation}
\end{enumerate}

Pongo  $\frac{m_2}{m_1+m_2}=\mu $. Quindi elimino la tensione (grazie a  $F=ma$ ):
\begin{equation}
  \left\{ \begin{matrix}
    m_1\ddot r=-T \\
    m_2\ddot r=T-m_2g
  \end{matrix} \right .
\end{equation}
che mi da
\begin{equation}
  m_2\ddot r=-m_1\ddot r-m_2g
\end{equation}
ossia
\begin{equation}
  \ddot r=-g\frac{m_2}{m_1+m_2}=-\mu g
\end{equation}

Conservazione del momento angolare:
$L=m_1\dot \theta r^2$ \`e costante. \ Esplicito $ \dot \theta $
\begin{equation}
  \label{bt:torque}
  \dot \theta =\frac L{m_1r^2}
\end{equation}

Conservazione \ dell'energia:
\begin{align}
  E&=m_2 g h +      \frac{m_1}{2} v_1^2 +\frac{m_2}{2} v_2^2= \\
  &=m_2 g (r-l)+   \frac{m_2+m_1}{2} \dot r^2+ \frac{m_1}{2} (r\dot\theta)^2
\end{align}
e, usando la \cref{bt:torque}
\begin{equation}
  E=m_2 g (r-l)+   \frac{m_2+m_1}{2} \dot r^2+ \frac{L^2}{2 m_1 r^2}
\end{equation}
o, poich\'e l'energia
potenziale \`e definita a meno di una costante

\begin{equation}
  \label{bt:energy}
  E=m_2gr+\frac{m_2+m_1} 2\dot r^2+\frac{L^2}{2m_1r^2}
\end{equation}
Quindi

\begin{equation}
  \label{bt:denergy}
  \frac{dE}{dt}=m_2 g \dot r+(m_1+m_2) \dot r \ddot r - \dot r \frac{L^2}{m_1r^3}=0
\end{equation}

Esamino a parte il caso  $\dot r=0$ (moto circolare). In generale si ha
\begin{equation}
  \ddot{\vec r}=\hat r(\ddot r- r \dot \theta^2)+ \hat \theta ( 2 \dot r \dot \theta +r \ddot \theta)
\end{equation}
Semplificando  $\dot r=0$ (e $\ddot r=0$ ), e usando ancora $F=ma$ si ottiene 

\begin{equation}
  \ddot{ \vec r }= -r \dot \theta ^2 \hat r + r \ddot \theta \hat \theta =\mu g \hat r
\end{equation}
perch\'e la forza sulla massa 1 \`e radiale; quindi, eguagliando le componenti $\ddot \theta =0$ e

\begin{equation}
  \dot{\theta }=\sqrt{\frac{\mu g} r}
\end{equation}

Quest'ultima equazione esprime la relazione tra le condizioni iniziali affinch\'e il moto sia circolare. Quindi, poich\'e 
$r$ \`e costante, posso integrare nel tempo, ottenendo 
\begin{equation}
\theta(t)=\sqrt{\frac{\mu g}{r}}t
\end{equation}

Caso  $\dot r \neq 0$ :
Dalla \cref{bt:denergy} si ottiene
\begin{equation}
\dot r\ddot r=\dot r\frac{L^2}{m_1(m_1+m_2)r^3}-\mu g\dot r
\end{equation}
e, integrando nel tempo
\begin{equation}
2\dot r^2=-\frac{L^2}{2m_1(m_1+m_2)r^2}-\mu gr+2c
\end{equation}
che \`e un'equazione differenziale a variabili separabili:

\begin{equation}
dt=\frac{dr}{\sqrt{-\frac{L^2}{4m_1(m_1+m_2)r^2}-\frac{\mu gr} 2+c}}
\end{equation}
Questa equzione fornisce la posizione $r$ della particella in funzione del tempo. $c$ \`e una costante che si determina (dopo aver risolto l'equazione) a partire dalle condizioni iniziali.

\end{document}
