\documentclass[../main.tex]{subfiles} 
\begin{document}

\exercise{Campo di forze con resistenza} %cfr
\textex 
Un punto materiale di massa $m$ si muove lungo una retta soggetto ad una forza costante $F$ e ad una resistenza $F_r=-kv^2$, dove $v$ è la velocità del punto materiale.
Calcolare la distanza $x(t)$ a partire da un istante in cui sia $x$ che $v$ sono nulli (può essere utile il seguente cambio di variabile $z=e^{\frac{2\sqrt{Fk}}{m}}t$). 
Come si può calcolare la distanza $x(v)$ percorsa in funzione di $v$ senza mai passare attraverso il tempo?


\solution
La massa $m$ si muove lungo una retta, quindi nel corso di tutta la soluzione considererò solo le intensità dei vettori, poichè saranno tutti diretti lungo la retta del moto.

Innanzitutto ho che la massa $m$ è soggetta alla forza $F+F_r=F-kv^2$, da cui:
\begin{gather*}
	m a=F-kv^2 \\
	\iff m \frac{\de v}{\de t}= F-kv^2 \\
	\iff \frac{m}{F-kv^2}\de v= \de t \\ 
	\Longrightarrow \int_{v(0)}^{v(t)} \frac{m}{F-kv^2}\de v = t
\end{gather*}
da cui, sostituendo $V=\sqrt{\frac{F}{k}}$ e utilizzando che $v(0)=0$, ottengo
\begin{gather*}
	t=\frac{m}{k}\int_{0}^{v(t)} \frac{1}{V^2-v^2}\de v \\
	\iff t=\frac{m}{2Vk}\int_{0}^{v(t)} \left(\frac{1}{V-v}+\frac{1}{V+v}\right)\de v \\
	\Longrightarrow t=\frac{m}{2Vk}\ln{\left(\frac{V-v(t)}{V+v(t)}\right)}
\end{gather*}
dove non devo aggiungere nessun modulo all'argomento del logaritmo perchè ho sempre che $v(t)\le \sqrt{\frac{F}{k}}=V$.
Quindi dall'ultima equazione, chiamando $\tau=\frac{m}{2\sqrt{Fk}}$, ottengo:
\begin{gather*}
	v(t)=V\cdot\frac{1-e^{\frac{t}{\tau}}}{1+e^{\frac{t}{\tau}}}\\
	\Longrightarrow x(t)=\int_{0}^t V\cdot\frac{1-e^{\frac{u}{\tau}}}{1+e^{\frac{u}{\tau}}}\de u
\end{gather*}
da cui, sostituendo $z=e^{\frac{u}{\tau}}$ e utilizzando che $\de u=\frac{\tau}{z}\de z$, ho:
\begin{gather*}
	x(t)=V\tau\int_1^{e^{\frac{t}{\tau}}} \frac{1-z}{z(z+1)}\de z \\
	\iff x(t)=\frac{m}{2k} \int_1^{e^{\frac{t}{\tau}}}\left( \frac{1}{z}- \frac{2}{z+1}\right)\de z \\
	\Longrightarrow x(t)=\frac{m}{2k}\left( \ln(e^{\frac{t}{\tau}}) -2\ln(e^{\frac{t}{\tau}}+1) +2\ln 2\right) \\
	\iff x(t)= Vt - \frac{m}{k} \ln \left( \frac{e^{\frac{t}{\tau}}+1}{2}\right)
\end{gather*}

Voglio calcolare ora $x(v)$ senza mai passare attraverso il tempo. Ho che:
\begin{gather*}
	m a=F-kv^2 \\
	\iff m \frac{\de v}{\de t}= F-kv^2 \\
	\iff m \frac{\de v}{\de x} v= F-kv^2 \\
	\Longrightarrow x(v)= \int _0^v \frac{mu}{F-ku^2}\de u \\
	\iff x(v)= -\frac{m}{2k}\int_0^v \frac{-2u}{V^2-u^2} \de u \\
	\Longrightarrow x(v)= -\frac{m}{2k} \left( \ln(V^2-v^2)-\ln V^2 \right) \\
	\Longleftrightarrow x(v)= -\frac{m}{2k}\ln\left (1-\frac{v^2}{V^2}\right)
\end{gather*}




\end{document}