\documentclass[../main.tex]{subfiles} 
\begin{document}

\exercise[17/10/2013]{Asta Vincolata Rotante} %avr

\textex
Un'asta vincolata in un punto si muove su un piano con velocità angolare $\omega$ costante. 
\begin{enumerate}
	\item Studiare il moto di un corpo di massa $m$ che si muove liberamente lungo l'asta.
	\item Studiare lo stesso sistema se l'asta è disposta in verticale e quindi interviene la forza di gravità.
\end{enumerate}

\solution

1) Studiamo il problema in coordinate cartesiane prendendo come origine il vincolo dell'asta, e chiamando $\theta$ l'angolo tra l'asta e l'asse polare.

Per \cref{AccCooPolari}, so che in coordinate polari l'accelerazione vale: 
\begin{equation*}\label{avr:1}
	\overrightarrow{a}=\dot{\overrightarrow{v}} =(\ddot{r}-r\dot{\theta}^2)\hat{r}+(2\dot{r}\dot{\theta}+r\ddot{\theta})\hat{\theta}
\end{equation*}

In questo problema in particolare ho che poiché l'asta si muove di velocità angolare costante lungo  $\hat{\theta}$ allora
 $\dot{\theta}=\omega$ e $\ddot{\theta}=0$. Inoltre, poiché per ipotesi possiamo supporre che non siano presenti forze d'attrito e che la reazione vincolare sia 
 perpendicolare all'asta (vincolo liscio) abbiamo che la forza lungo $\hat{r}$ deve essere nulla, quindi:
 \begin{equation*}\label{avr:2}
  \ddot{r}=\omega^2r
 \end{equation*}
La soluzione di questa differenziale è un'esponenziale con due gradi di libertà, che risulta facilmente essere:
\begin{equation}\label{avr:3}
 r=Ae^{\omega t}+Be^{-\omega t}
\end{equation}
dove A e B sono univocamente determinate dalle condizioni iniziali ovvero da $r_0$ e $v_0$, intendendo con $r_0$ la distanza della mia massa dall'origine
e con $v_0$ la sua velocità iniziale lungo $\hat{r}$.

\`{E} sufficiente infatti imporre in \cref{avr:3} che al tempo $t=0$ valga $r=r_0$ e $\dot{r}=v_0$, ovvero $A+B=r_0$ e $\omega(A-B)=v_0$, per ottenere $A={(\omega r_0 +v_0)}/{2\omega}$ e $B={(\omega r_0 -v_0)}/{2\omega}$.

Per concludere quindi l'equazione del moto della mia massa $m$ sarà:
\begin{equation}
 \overrightarrow{r}=\left(\frac{(\omega r_0 +v_0)}{2\omega}e^{\omega t}+\frac{(\omega r_0 -v_0)}{2\omega}e^{-\omega t}\right)\hat{r}
\end{equation}

2) Come prima mi riduco a studiare la forza lungo $\hat{r}$ che però stavolta non sarà nulla, ma sarà invece la componente lungo $\hat r$ della forza di gravità, quindi:
\begin{equation}\label{avr:4}
  \ddot{r}-\omega^2r=-g \sin{\omega t}
 \end{equation}
Si tratta di un'equazione differenziale non omogenea, quindi le sue soluzioni saranno date da una sua soluzione particolare più tutte le soluzioni dell'omogenea associata,
che sono già state trovate nel punto precedente. Quindi poiché una soluzione particolare è ad esempio quella con $r$ costante ed uguale a 
$(g \sin{\omega t})/2\omega^2$ allora la soluzione generale sarà data da:
\begin{equation}\label{avr:5}
 r=Ae^{\omega t}+Be^{-\omega t}+\frac{g \sin{\omega t}}{2\omega^2}
\end{equation}

\end{document}
