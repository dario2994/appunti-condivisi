\documentclass[../main.tex]{subfiles} 
\begin{document}

\exercise{Asta Vincolata Rotante} %ndp

\textex
Un'asta vincolata in un punto si muove con velocit\'{a} $\omega$ costante. Studiare il moto di un corpo di massa $m$ che si muove liberamente lungo l'asta. \\
2) Studiare lo stesso sistema se l'asta \'{e} disposta in verticale e quindi interviene la forza di gravit\'{a}.

\solution
Studiamo il problema in coordinate cartesiane prendendo come origine il vincolo dell'asta, e chiamando $\theta$ l'angolo tra l'asta e l'asse polare. \\
So che in coordinate polari l'accelerazione vale: 
\begin{equation}\label{AV1}
	\overrightarrow{a}=\dot{\overrightarrow{v}} =(\ddot{r}-r\dot{\theta}^2)\hat{r}+(2\dot{r}\dot{\theta}+r\ddot{\theta})\hat{\theta}
\end{equation}

In questo problema in particolare ho che poich\'{e} l'asta si muove di velocit\'{a} angolare costante lungo  $\hat{\theta}$ allora
 $\dot{\theta}=\omega$ e $\ddot{\theta}=0$. Inoltre, poich\'{e} per ipotesi possiamo supporre che non siano presenti forze d'attrito e che la reazione vincolare sia 
 perpendicolare all'asta (vincolo liscio) abbiamo che la forza lungo $\hat{r}$ deve essere nulla, quindi:
 \begin{equation}\label{AV2}
  \ddot{r}=\omega^2r
 \end{equation}
La soluzione della differenziale \'{e} chiaramente un'esponenziale, ovvero:
\begin{equation}\label{AV3}
 r=Ae^{\omega t}+Be^{-\omega t}
\end{equation}
Dove A e B sono univocamente determinate dalle condizioni iniziali ovvero da $r_0$ e $v_0$, intendendo con $r_0$ la distanza della mia massa dall'origine
e con $v_0$ la sua velocit\'{a} iniziale lungo $\hat{r}$. \\
\'{E} sufficiente infatti risolvere le due equazioni al tempo $t=0$ con r e con $\dot{r}$ ovvero $A+B=r_0$ e $\omega(A-b)=v_0$ per ottenere $A={(\omega r_0 +
v_0)}/{2\omega}$ e $B={(\omega r_0 -v_0)}/{2\omega}$.\\
Per concludere quindi l'equazione del moto della mia massa m sar\'{a}:
\begin{equation}
 \overrightarrow{r}=(\frac{(\omega r_0 +v_0)}{2\omega}e^{\omega t}+\frac{(\omega r_0 -v_0)}{2\omega}e^{-\omega t})\hat{r}
\end{equation}




Un po' di boiate a parole, poi qualche formuletta:
\begin{equation}\label{ndp:gamma}
	\Gamma(x)=\lim_{n\to\infty} \dfrac{n^xn!}{x(x+1)\cdots (x+n-1)(x+n)}
\end{equation}
Poi un'equazione senza numerino:
\begin{equation*}
	a=b \Rightarrow b=c \Rightarrow 1=0
\end{equation*}

E poi, tutto soddifatto, richiamo la \cref{ndp:gamma}.



\end{document}