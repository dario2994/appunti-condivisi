\documentclass[../main.tex]{subfiles}


\begin{document}

\exercise[11/11/2013]{Pallina su un filo che si arrotola su un perno} %pa

\textex
Una massa \`e attaccata all'estremit\`a di un filo di lunghezza $l_0$ che si avvolge intorno ad un perno di raggio $R$ senza attrito, con velocit\`a iniziale $v$. Determinare il moto della massa.

\solution
\begin{enumerate}
\item Sistema di riferimento: inerziale, origine nel punto di contatto, coordinate: lunghezza $l$ del filo e spostamento angolare $\theta $ .
\item Forze: tensione del filo.
\item Condizioni iniziali: $R,\dot \theta$ oppure $R,v$.
\item Relazioni varie: si conserva l'energia cinetica perch\'e non viene svolto lavoro sulla massa, dunque la velocit\`a $v$ \`e costante.
\end{enumerate}

Se inizialmente $v$ \`e perpendicolare al filo e questo \`e teso, queste condizioni saranno valide anche in seguito: il filo non pu\`o svolgere lavoro (perch\'e non ci sono attriti e non c'\`e apporto di energia dall'esterno) e la tensione \`e esercitata lungo la direzione del filo. Se la velocit\`a non fosse perpendicolare, ci sarebbe una componente della forza parallela alla velocit\`a, che svolgerebbe lavoro; infine ci deve essere una tensione del filo, perch\'e inizialmente era teso e quindi \`e necessaria una forza centripeta.

Sotto queste condizioni posso in ogni istante considerare il moto come un rotazione attorno al punto di contatto. Quindi pongo $\theta(0)=0$ e ottengo
\begin{equation}
\label{pa:speed}
v=l \dot \theta
\end{equation}
dove $v$ \`e costante, e
\begin{equation}
\label{pa:lung}
l=l_0 - R \theta
\end{equation}
che, se derivo entrambi i membri diventa:

\begin{equation}
\label{pa:dlung}
\dot l= -R \dot \theta
\end{equation}
Posso allora impostare un'equazione differenziale sostituendo nella \cref{pa:dlung} $\dot \theta$ dalla \cref{pa:speed} :

\begin{equation}
\label{pa:diff}
\dot l = -R \frac{v}{l}
\end{equation}

\begin{equation}
\label{pa:sol}
l \dot l = -R v
\end{equation}
Ora posso integrare i due membri, ottenendo

\begin{equation}
\label{pa:lq}
\frac{l^2}{2}= -Rvt + \frac{c}{2}
\end{equation}
dove $c$ \`e la costante di integrazione, che si ricava ponendo $l=l_0$ a $t=0$.

\begin{equation}
\label{pa:t0}
l_0^2 = c
\end{equation}
e quindi risolvendo in $l$ la \cref{pa:lq}

\begin{equation}
\label{pa:l}
l= \sqrt{l_0^2 - 2Rvt}
\end{equation}
che in effetti soddisfa l'equazione differenziale; quindi per la \cref{pa:lung}

\begin{equation}
\theta = \frac{l_0 - l}{R}
\end{equation}

Se vogliamo la posizione $\vec r$ rispetto al centro del perno, questa si ricava, in coordinate polari, da $l$, dato che $r^2=R^2+l^2$:

\begin{equation}
\vec r = \hat r \sqrt{R^2 +l_0^2 -2Rvt}
\end{equation}
\end{document}
