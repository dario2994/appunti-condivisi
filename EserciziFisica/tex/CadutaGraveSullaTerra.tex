\documentclass[../main.tex]{subfiles} 
\begin{document}

\exercise{Caduta di un grave sulla terra} %cgt

\textex
Trascurando l'attrito dell'aria, trovare lo spostamento orizzontale (nel sistema non inerziale della terra che gira) che subisce un
grave in caduta libera da un'altezza $h$.


\solution
Pongo i 3 assi cartesiani $\hat{x}, \hat{y}, \hat{z}$ ponendo $\hat{x}$ sul meridiano del luogo, $\hat{y}$ sul parallelo del luogo
e $\hat{z}$ rivolto verso il centro della terra.

Siano inoltre $\theta$ l'angolo che si forma tra piano equatoriale e congiungente con il luogo, $\vec{\omega}$
l'asse di rotazione terrestre ed $R$ il raggio della terra.

Inizialmente il grave si trova in $(0,0,0)$ con velocità $(0,0,0)$.

Calcolo le forze agenti sul corpo, sfruttando implicitamente \cref{AccNonInerziale}.

L'accelerazione di gravità vale $g\hat{z}$ e l'accelerazione di trascinamento (che equivale
alla forza centrifuga agente nel luogo) risulta essere 
$w^2R\cos\theta\left(-\cos\theta\hat{z}+\sin\theta\hat{x}\right)$. (dove ho assunto trascurabile lo spostamento relativo del corpo
rispetto al raggio terrestre: $h\ll R$)

L'effetto complessivo di queste due accelerazioni (costanti) porta a definire la $\vec{g_e}$ cioè l'accelerazione di gravità
del luogo:
\begin{equation}\label{cgt:geff}
	\vec{g_e}=g\hat{z}-w^2R\cos\theta\left(-\cos\theta\hat{z}+\sin\theta\hat{x}\right)
\end{equation}
e porta anche a considerare l'angolo $\theta_0$ cioè l'angolo che forma $g_e$ con il piano equatoriale.

Ora calcolo la forza centrifuga, assumendo di poter trascurare, per calcolare lo spostamento relativo del grave,
la forza centrifuga e di Coriolis. Risulta allora che la norma della forza centrifuga vale
\begin{equation}\label{cgt:centrifuga}
	|\vec{a_c}|=|\vec{\omega}\times(\vec{\omega}\times \vec r) | \approx \omega^2\frac 12|g_e|t^2\cos \theta_0
\end{equation}
Ho calcolato solo la norma, perchè mostrerò che è del tutto trascurabile rispetto alle altre forze.

E infine per calcolare l'accelerazione di Coriolis assumo, con ottima approssimazione, che la velocità 
del grave sia esattamente come se non fossero presenti le forze di Coriolis e Centrifuga. Di conseguenza ottengo
$\vec{v}=\vec{g_e}t$ e ciò implica che l'accelerazione di Coriolis risulta:
\begin{equation}\label{cgt:Coriolis}
	\vec {a_{cor}}=2\vec \omega \times \vec{g_e} t
\end{equation}
Da quest'ultima ricavo anche subito una buona approssimazione per la forza di Coriolis:
\begin{equation}\label{cgt:CoriolisApprox}
	|\vec {a_{cor}}| =2\omega|g_e| t\cos \theta_0 
\end{equation}

Ora sfruttando \cref{cgt:centrifuga,cgt:CoriolisApprox} ottengo che l'accelerazione centrifuga è trascurabile rispetto
a quella di Coriolis:
\begin{equation*}
	|a_c|\ll|a_cor| \iff \omega^2\frac 12|g_e|t^2\cos \theta_0\ll 2\omega|g_e| t\cos \theta_0 \iff
	\omega t\ll 4
\end{equation*}
Ma l'ultima disuguaglianza è verissima, poichè $\omega=\frac{2\pi}{3600\cdot 24}\approx 10^{-4}$

\end{document}
