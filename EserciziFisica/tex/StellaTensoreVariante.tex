\documentclass[../main.tex]{subfiles} 
\begin{document}

\exercise[20/03/2014]{Stella con tensore d'inerzia variante} %stv
\textex
Una stella ha i momenti d'inerzia lungo gli assi principali che variano nel tempo seconda le leggi
\begin{align*}
	I_x=I_y & =I\left(1-\frac \varepsilon 2\cos\Omega t\right)\\
	I_z & = I\left(1+\varepsilon \cos\Omega t\right)
\end{align*}
Si determini la velocità angolare $\vec\omega$ della stella al prim'ordine in $\varepsilon\ll 1$, conoscendo il valore iniziale $\vec\omega_0$.

\solution
Non sono presenti forze esterne che agiscono sul sistema, quindi per la seconda equazione cardinale ottengo
\begin{equation}\label{stv:eqCardinale}
	\frac{\de \vec L}{\de t}=0
\end{equation}

So inoltre che per \cref{ten:MomentoAngolare}, vale
\begin{gather*}
	\vec L = (I_x\omega_x,I_y\omega_y,I_z\omega_z)\\
	\Longrightarrow \frac{\de \vec L}{\de t}=\frac{\partial L}{\partial t}+\vec\omega\times \vec L = (I_x\dot\omega_x+\dot I_x\omega_x,I_y\dot \omega_y+\dot I_y\omega_y,I_z\dot\omega_z+\dot I_z\omega_z)+(\omega_x,\omega_y,\omega_z)\times (I_x\omega_x,I_y\omega_y,I_z\omega_z)
\end{gather*}

Svolgendo i conti, la \cref{stv:eqCardinale} diventa quindi:
\begin{equation*}
\begin{cases}
	I_x\dot\omega_x+\dot I_x\omega_x+\omega_y\omega_z(I_z-I_y)=0\\
	I_y\dot\omega_y+\dot I_y\omega_y+\omega_x\omega_z(I_x-I_z)=0\\
	I_z\dot\omega_z+\dot I_z\omega_z+\omega_x\omega_y(I_y-I_x)=0
\end{cases}
\end{equation*}
da cui, chiamando $I_x=I_y=\bar I$, ottengo:
\begin{equation}\label{stv:sistemaEq}
\begin{cases}
	\bar I\dot\omega_x+\dot {\bar I}\omega_x+\omega_y\omega_z(I_z-\bar I)=0\\
	\bar I\dot\omega_y+\dot {\bar I} \omega_y+\omega_x\omega_z(\bar I-I_z)=0\\
	I_z\dot\omega_z+\dot I_z\omega_z=0
\end{cases}
\end{equation}

Risolvo innanzitutto la terza di queste equazioni. Deve valere:
\begin{equation*}
	I_z\dot\omega_z+\dot I_z\omega_z=0 \iff I(1+\varepsilon \cos\Omega t)\dot\omega_z=I\Omega\varepsilon\sin\Omega t \omega_z \iff 
	\frac{\de \omega_z}{\omega_z}=\frac{\Omega\varepsilon\sin\Omega t}{1+\varepsilon \cos\Omega t}\de t
\end{equation*}
che nell'approssimazione al prim'ordine per $\varepsilon \ll 1$, diventa:
\begin{gather*}
	\frac{\de \omega_z}{\omega_z}=\Omega\varepsilon\sin\Omega t\de t \Longrightarrow 
	\int_{\omega_z(0)}^{\omega_z(t)}\frac{\de \omega_z}{\omega_z}=\int_0^t \Omega\varepsilon\sin\Omega s\de s\\
	\Longrightarrow \omega_z(t)=\omega_z(0)e^{\varepsilon(1-\cos\Omega t)}\simeq \omega_z(0)[1+\varepsilon(1-\cos\Omega t)]
\end{gather*}

Sostituendo quindi nella prima equazione (la seconda è analoga) di \cref{stv:sistemaEq} i valori di $\bar I$ e $I_z$ e il valore di $\omega_z$ appena trovato e approssimando al prim'ordine in $\varepsilon$, ottengo:
\begin{gather*}
	\bar I\dot\omega_x+\dot {\bar I}\omega_x+\omega_y\omega_z(I_z-\bar I)=0 \\
	\Longrightarrow (1-\frac \varepsilon 2\cos\Omega t) \dot\omega_x+\frac{\varepsilon}2 \Omega \sin\Omega t \omega_x + \frac 32 \varepsilon\cos\Omega t \omega_z(0)[1+\varepsilon(1-\cos\Omega t)] \omega_y=0\\
	\Longrightarrow \dot\omega_x+\frac{\varepsilon}2 \Omega \sin\Omega t \omega_x + \frac 32 \varepsilon\cos\Omega t \omega_z(0)\omega_y=0\\
\end{gather*}
Quindi mi sono ricondotta al sistema di due equazioni
\begin{equation*}
	\begin{cases}
		\dot\omega_x+\frac{\varepsilon}2 \Omega \sin\Omega t \omega_x + \frac 32 \varepsilon\cos\Omega t \omega_z(0)\omega_y=0\\
		\dot\omega_y+\frac{\varepsilon}2 \Omega \sin\Omega t \omega_y - \frac 32 \varepsilon\cos\Omega t \omega_z(0)\omega_x=0
	\end{cases}
\end{equation*}

Chiamo ora $z=\omega_x+i \omega_y$, sommando alla prima equazione del sistema $i$ volte la seconda ottengo:
\begin{gather*}
	\dot z+\frac{\varepsilon}2 \Omega \sin\Omega t z - \frac 32 \varepsilon\cos\Omega t \omega_z(0)z=0\\
	\begin{split}
	\Longrightarrow z&=(\omega_x(0)+i\omega_y(0))\exp\left[\frac\varepsilon 2 \left(3i\omega_z(0)\frac{\sin\Omega t}{\Omega}+\cos\Omega t -1\right)\right]\\
	&=(\omega_x(0)+i\omega_y(0))\left[1+\frac\varepsilon 2 \left(3i\omega_z(0)\frac{\sin\Omega t}{\Omega}+\cos\Omega t -1\right)\right]
	\end{split}\\
	\Longrightarrow 
	\begin{cases}
		\omega_x(t)=\omega_x(0)\left( 1+\frac \varepsilon 2 \cos\Omega t -\frac\varepsilon 2 \right) - \frac 32 \varepsilon \omega_y(0)\omega_z(0)\frac{\sin \Omega t}{\Omega}\\
		\omega_y(t)=\omega_y(0)\left( 1+\frac \varepsilon 2 \cos\Omega t -\frac\varepsilon 2 \right) + \frac 32 \varepsilon \omega_x(0)\omega_z(0)\frac{\sin \Omega t}{\Omega}
	\end{cases}
\end{gather*}

Quindi, poichè avevo già ricavato $\omega_z(t)$, ho trovato quindi il valore di $\vec\omega$ rispetto al tempo, che è quello che volevo.

\solution[2]

Si vuole adesso spostare l'attenzione sulle componenti $(L_x, \ L_y, \ L_z)$ del vettore momento angolare, mettendo in luce cosa accade alla stella  per tempi molto lunghi (che le approssimazioni della precedente soluzione, esatte solo per tempi brevi, non consentono di osservare). \\
Imponendo come prima che la derivata del momento angolare sia nulla, si ha:
\begin{equation*}
\begin{cases}
	\displaystyle \dot L_x+(\bar I^{-1}-I_z^{-1}) L_yL_z=0\\
           \displaystyle \dot L_y+(\bar I^{-1}-I_z^{-1}) L_xL_z=0\\
           \displaystyle \dot L_z=0\\
\end{cases}
\end{equation*}
Dalla terza equazione deduciamo che $L_z$ è costante. \\
Sia $\Gamma=L_x+iL_y$ .\\
 Manipolando le prime due equazioni del sistema si trova: 
\begin{gather*}
	\displaystyle \frac{\dot\Gamma}\Gamma=i(\bar I^{-1}-I_z^{-1})L_z=
           \frac{3}{2}  \varepsilon i\frac{L_z}{I}\cos \Omega t\left (1-\frac \varepsilon 2 \cos\Omega t\right)^{-1} \left( 1+\varepsilon      \cos\Omega t\right)^{-1}\approx \\ \approx
           \frac{3}{2}  \varepsilon i\frac{L_z}{I}\cos \Omega t\left (1-\frac \varepsilon 2 \cos\Omega t\right) \\
            \Longrightarrow \ln\left(\frac{\Gamma}\Gamma_0\right)= \frac{3}{2}  \varepsilon i\frac{L_z}{I}\left (\frac{\sin \Omega t}{\Omega}- \frac{\varepsilon}{8} \frac{\sin 2 \Omega t}{\Omega}-\frac{\varepsilon t}{4}\right)
\end{gather*}
Siamo liberi di supporre che nell'istante iniziale $L_y(0)=0$. Quindi:
\begin{equation*}
\begin{cases}
		\displaystyle L_x(t)=\Gamma_0 \cos \left[\frac{3\varepsilon }{2} \frac{L_z}{I}\left (\frac{\sin \Omega t}{\Omega}- \frac{\varepsilon}{8} \frac{\sin 2 \Omega t}{\Omega}-\frac{\varepsilon t}{4}\right) \right] \\
                     \displaystyle L_y(t)=\Gamma_0 \sin \left[\frac{3\varepsilon }{2} \frac{L_z}{I}\left (\frac{\sin \Omega t}{\Omega}- \frac{\varepsilon}{8} \frac{\sin 2 \Omega t}{\Omega}-\frac{\varepsilon t}{4}\right) \right] \\
	\end{cases}
\end{equation*}
Per tempi dell'ordine di $\Omega^{-1}$ si ha: 
\begin{equation*}
\begin{cases}
		\displaystyle L_x(t)\approx \Gamma_0 \\
                     \displaystyle L_y(t)\approx \Gamma_0  \left(\frac{3\varepsilon }{2} \frac{L_z}{I\Omega}  \ {\sin \Omega t} \right) \\
	\end{cases}
\end{equation*}
che equivale esattamente alla prima soluzione, passando alle velocità angolari e approssimando al prim'ordine.  \\ Invece, per $t\gg (\varepsilon\Omega)^{-1}$, si trova:
\begin{equation*}
\begin{cases}
		\displaystyle L_x(t)\approx\Gamma_0 \cos \left(\frac{3\varepsilon^2 }{8} \frac{L_z}{I}\  t\right) \\
                     \displaystyle L_y(t)\approx-\Gamma_0 \sin \left(\frac{3\varepsilon^2 }{8} \frac{L_z}{I} \ t\right) \\
	\end{cases}
\end{equation*}
Si nota in particolare che, per tempi molto lunghi, il sistema acquista simmetria.




\end{document}
