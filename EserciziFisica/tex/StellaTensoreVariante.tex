\documentclass[../main.tex]{subfiles} 
\begin{document}

\exercise[20/03/2014]{Stella con tensore d'inerzia variante} %stv
\textex
Una stella ha i momenti d'inerzia lungo gli assi principali che variano nel tempo seconda le leggi
\begin{align*}
	I_x=I_y & =I\left(1-\frac \varepsilon 2\cos\Omega t\right)\\
	I_z & = I\left(1+\varepsilon \cos\Omega t\right)
\end{align*}
Si determini la velocità angolare $\vec\omega$ della stella al prim'ordine in $\varepsilon\ll 1$, conoscendo il valore iniziale $\vec\omega_0$.

\solution
Non sono presenti forze esterne che agiscono sul sistema, quindi per la seconda equazione cardinale ottengo
\begin{equation}\label{stv:eqCardinale}
	\frac{\de \vec L}{\de t}=0
\end{equation}

So inoltre che per \cref{ten:MomentoAngolare}, vale
\begin{gather*}
	\vec L = (I_x\omega_x,I_y\omega_y,I_z\omega_z)\\
	\Longrightarrow \frac{\de \vec L}{\de t}=\frac{\delta L}{\delta t}+\vec\omega\times \vec L = (I_x\dot\omega_x+\dot I_x\omega_x,I_y\dot \omega_y+\dot I_y\omega_y,I_z\dot\omega_z+\dot I_z\omega_z)+(\omega_x,\omega_y,\omega_z)\times (I_x\omega_x,I_y\omega_y,I_z\omega_z)
\end{gather*}

Svolgendo i conti, la \cref{stv:eqCardinale} diventa quindi:
\begin{equation*}
\begin{cases}
	I_x\dot\omega_x+\dot I_x\omega_x+\omega_y\omega_z(I_z-I_y)=0\\
	I_y\dot\omega_y+\dot I_y\omega_y+\omega_x\omega_z(I_x-I_z)=0\\
	I_z\dot\omega_z+\dot I_z\omega_z+\omega_x\omega_y(I_y-I_x)=0
\end{cases}
\end{equation*}
da cui, chiamando $I_x=I_y=\bar I$, ottengo:
\begin{equation}\label{stv:sistemaEq}
\begin{cases}
	\bar I\dot\omega_x+\dot {\bar I}\omega_x+\omega_y\omega_z(I_z-\bar I)=0\\
	\bar I\dot\omega_y+\dot {\bar I} \omega_y+\omega_x\omega_z(\bar I-I_z)=0\\
	I_z\dot\omega_z+\dot I_z\omega_z=0
\end{cases}
\end{equation}

Risolvo innanzitutto la terza di queste equazioni. Deve valere:
\begin{equation*}
	I_z\dot\omega_z+\dot I_z\omega_z=0 \iff I(1+\varepsilon \cos\Omega t)\dot\omega_z=I\Omega\varepsilon\sin\Omega t \omega_z \iff 
	\frac{\de \omega_z}{\omega_z}=\frac{\Omega\varepsilon\sin\Omega t}{1+\varepsilon \cos\Omega t}\de t
\end{equation*}
che nell'approssimazione al prim'ordine per $\varepsilon \ll 1$, diventa:
\begin{gather*}
	\frac{\de \omega_z}{\omega_z}=\Omega\varepsilon\sin\Omega t\de t \Longrightarrow 
	\int_{\omega_z(0)}^{\omega_z(t)}\frac{\de \omega_z}{\omega_z}=\int_0^t \Omega\varepsilon\sin\Omega s\de s\\
	\Longrightarrow \omega_z(t)=\omega_z(0)e^{\varepsilon(1-\cos\Omega t)}\simeq \omega_z(0)[1+\varepsilon(1-\cos\Omega t)]
\end{gather*}

Sostituendo quindi nella prima equazione (la seconda è analoga) di \cref{stv:sistemaEq} i valori di $\bar I$ e $I_z$ e il valore di $\omega_z$ appena trovato e approssimando al prim'ordine in $\varepsilon$, ottengo:
\begin{gather*}
	\bar I\dot\omega_x+\dot {\bar I}\omega_x+\omega_y\omega_z(I_z-\bar I)=0 \\
	\Longrightarrow (1-\frac \varepsilon 2\cos\Omega t) \dot\omega_x+\frac{\varepsilon}2 \Omega \sin\Omega t \omega_x + \frac 32 \varepsilon\cos\Omega t \omega_z(0)[1+\varepsilon(1-\cos\Omega t)] \omega_y=0\\
	\Longrightarrow \dot\omega_x+\frac{\varepsilon}2 \Omega \sin\Omega t \omega_x + \frac 32 \varepsilon\cos\Omega t \omega_z(0)\omega_y=0\\
\end{gather*}
Quindi mi sono ricondotta al sistema di due equazioni
\begin{equation*}
	\begin{cases}
		\dot\omega_x+\frac{\varepsilon}2 \Omega \sin\Omega t \omega_x + \frac 32 \varepsilon\cos\Omega t \omega_z(0)\omega_y=0\\
		\dot\omega_y+\frac{\varepsilon}2 \Omega \sin\Omega t \omega_y - \frac 32 \varepsilon\cos\Omega t \omega_z(0)\omega_x=0
	\end{cases}
\end{equation*}

Chiamo ora $z=\omega_x+i \omega_y$, sommando alla prima equazione del sistema $i$ volte la seconda ottengo:
\begin{gather*}
	\dot z+\frac{\varepsilon}2 \Omega \sin\Omega t z - \frac 32 \varepsilon\cos\Omega t \omega_z(0)z=0\\
	\Longrightarrow z=(\omega_x(0)+i(\omega_y(0)))e^{\frac\varepsilon 2 (3i\omega_z(0)\frac{\sin\Omega t}{\Omega}+\cos\Omega t -1)}
\end{gather*}







\end{document}