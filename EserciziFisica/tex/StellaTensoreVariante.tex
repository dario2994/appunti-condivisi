\documentclass[../main.tex]{subfiles} 
\begin{document}

\exercise[20/03/2014]{Stella con tensore d'inerzia variante} %stv
\textex
Una stella ha i momenti d'inerzia lungo gli assi principali che variano nel tempo seconda le leggi
\begin{align*}
	I_x=I_y & =I\left(1-\frac \varepsilon 2\cos\Omega t\right)\\
	I_z & = I\left(1+\varepsilon \cos\Omega t\right)
\end{align*}
Si determini la velocità angolare $\vec\omega$ della stella al prim'ordine in $\varepsilon\ll 1$, conoscendo il valore iniziale $\vec\omega_0$.

\solution
Non sono presenti forze esterne che agiscono sul sistema, quindi per la seconda equazione cardinale ottengo
\begin{equation}\label{stv:eqCardinale}
	\frac{\de \vec L}{\de t}=0
\end{equation}

So inoltre che per \cref{ten:MomentoAngolare}, vale
\begin{gather*}
	\vec L = (I_x\omega_x,I_y\omega_y,I_z\omega_z)\\
	\Longrightarrow \frac{\de \vec L}{\de t}=\frac{\delta L}{\delta t}+\vec\omega\times \vec L = (I_x\dot\omega_x+\dot I_x\omega_x,I_y\dot \omega_y+\dot I_y\omega_y,I_z\dot\omega_z+\dot I_z\omega_z)+(\omega_x,\omega_y,\omega_z)\times (I_x\omega_x,I_y\omega_y,I_z\omega_z)
\end{gather*}

Svolgendo i conti, la \cref{stv:eqCardinale} diventa quindi:
\begin{equation*}
\begin{cases}
	I_x\dot\omega_x+\dot I_x\omega_x+\omega_y\omega_z(I_z-I_y)=0\\
	I_y\dot\omega_y+\dot I_y\omega_y+\omega_x\omega_z(I_x-I_z)=0\\
	I_z\dot\omega_z+\dot I_z\omega_z+\omega_x\omega_y(I_y-I_x)=0
\end{cases}
\end{equation*}
da cui, chiamando $I_x=I_y=\bar I$, ottengo:
\begin{equation*}
\begin{cases}
	\bar I\dot\omega_x+\dot {\bar I}\omega_x+\omega_y\omega_z(I_z-\bar I)=0\\
	\bar I\dot\omega_y+\dot {\bar I} \omega_y+\omega_x\omega_z(\bar I-I_z)=0\\
	I_z\dot\omega_z+\dot I_z\omega_z=0
\end{cases}
\end{equation*}




\end{document}