\documentclass[../main.tex]{subfiles} 
\begin{document}

\exercise[16/01/2014]{Molla con pendolo} %mcp
\textex
Una massa $m_1$ è appesa al soffitto tramite una molla di costante elastica $k$ ed è vincolata a muoversi
solo sulla verticale. A sua volta un pendolo di
lunghezza $l$, alla cui estremità si trova una massa $m_2$, è appeso alla massa $m_1$.
Su tutto il sistema agisce la forza di gravità. Calcolare la frequenza delle piccole oscillazioni della molla e
del pendolo.

\solution
Mi pongo nel sistema di riferimento cartesiano che ha asse $y$ coincidente con la verticale della molla e diretto 
verso l'alto e asse $x$ parallelo al soffitto e passante per il ``punto di equilibrio`` della molla, cioè il punto
più basso che raggiungerebbe se non ci fosse appesa la massa $m_1$.

Le posizioni delle due masse in questo sistema di riferimento sono rispettivamente:
\begin{gather*}
	\vec{r_1}=(0,y)\\
	\vec{r_2}=(l\sin\theta,y-l\cos\theta)
\end{gather*}
dove $\theta$ è l'angolo che il pendolo forma con la verticale.

Derivando due volte queste due espressioni ottengo le componenti delle accelerazioni delle due masse:
\begin{gather*}
	\vec{a_1}=(0,\ddot{y})\\
	\vec{a_2}=(l\cos\theta \ddot\theta-l\sin\theta\dot\theta^2, \ddot{y}+l\sin\theta\ddot\theta+l\cos\theta\dot\theta^2)
\end{gather*}

Sulla massa $m_1$, nella direzione dell'asse $y$ (l'unica in cui si può muovere), agisce la forza peso $m_2\vec{g}$,
la forza della molla $-ky$ e la componente verticale della tensione del filo del pendolo.
Sulla massa $m_2$ invece, agiscono solo la forza di gravità $m_1\vec{g}$ e la tensione del filo.
Posso scrivere quindi $\vec{F}=m\vec{a}$ per le due masse. 

Per la massa $m_1$ lungo la verticale ho:
\begin{equation*}
	m_1\ddot{y}=-ky-T\cos\theta-m_1g
\end{equation*}
Per la massa $m_2$, rispettivamente lungo $x$ e lungo $y$, ho invece:
\begin{gather*}
	m_2(l\cos\theta \ddot\theta-l\sin\theta\dot\theta^2)=-T\sin\theta \\
	m_2(\ddot{y}+l\sin\theta\ddot\theta+l\cos\theta\dot\theta^2)=T\cos\theta-m_2g
\end{gather*}

Quindi mi sono ricondotta a risolvere il sistema
% \begin{equation*}
\begin{align}
	m_1\ddot{y} &=-ky-T\cos\theta-m_1g \label{mcp:eq1} \\
	m_2(l\cos\theta \ddot\theta-l\sin\theta\dot\theta^2) &=-T\sin\theta \label{mcp:eq2} \\
	m_2(\ddot{y}+l\sin\theta\ddot\theta+l\cos\theta\dot\theta^2) &=T\cos\theta-m_2g \label{mcp:eq3}
\end{align}
% \end{equation*}

Innanzitutto osservo che, in regime di piccole oscillazioni, nella \cref{mcp:eq2} $l\sin\theta\dot\theta^2$, 
che è al terz'ordine,
è trascurabile rispetto a $l\cos\theta \ddot\theta$, che invece è al prim'ordine.
Nella \cref{mcp:eq3}, invece, $l\sin\theta\ddot\theta+l\cos\theta\dot\theta^2$ è trascurabile rispetto
a $\ddot{y}$ in quanto ha un ordine maggiore (sto implicitamente assumendo che avvengano dei moti armonici).

Riassumendo le semplificazioni e approssimando al prim'ordine $\sin\theta\approx\theta$ e $\cos\theta\approx 1$ ottengo:
\begin{align}
	m_1\ddot{y} &=-ky-T-m_1g \label{mcp:e1} \\
	m_2l\ddot\theta &=-T\theta \label{mcp:e2} \\
	m_2\ddot{y} &=T-m_2g \label{mcp:e3}
\end{align}

Ora noto che nelle \cref{mcp:e1,mcp:e3} non compare $\theta$, quindi risolvo in $y$ ottenendo:
\begin{equation*}
	\ddot y=-\frac k{m_1+m_2}\left(y-\frac{m_1+m_2}k g\right)
\end{equation*}

che è proprio lo stesso moto che assumerebbe la molla se il pendolo fosse fermo. Quindi la molla oscilla di
moto armonico intorno alla posizione di equilibrio $\frac{m_1+m_2}k g$ con frequenza $\sqrt{\frac k{m_1+m_2}}$.

Per quanto riguarda $\theta$, al prim'ordine ricavo da \cref{mcp:e3} che $T\approx m_2g$ e sostituendo questo in
\cref{mcp:e2} ottengo:
\begin{equation*}
	\ddot \theta=-\frac gl \theta
\end{equation*}
che è l'equazione classica del pendolo.

Quindi in conclusione il moto del pendolo e della molla avvengono in maniera disaccoppiata, con frequenze diverse e
entrambi avvengono come se l'altro non ci fosse.







\end{document}