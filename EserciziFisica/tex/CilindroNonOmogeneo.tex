\documentclass[../main.tex]{subfiles} 
\begin{document}

\exercise[Compitino 7/04/2014]{Cilindro non omogeneo che rotola} %cno
\textex
Un cilindro di massa $M$ e raggio $R$ rotola senza strisciare su un piano orizzontale. La distribuzione di massa del cilindro non rispetta la simmetria cilindrica ma è invariante per traslazioni parallele all'asse del cilindro: il baricentro si trova a una distanza $l_B$ dal centro ed il momento di inerzia rispetto al baricentro lungo l'asse parallelo all'asse del cilindro è $I_B$.
\begin{itemize}
	\item Mostrare che l'asse in questione è un asse principale.
	\item Calcolare il periodo delle piccole oscillazioni del cilindro intorno alla posizione di riposo.
\end{itemize}

\solution
\begin{itemize}
	\item Considero il piano perpendicolare all'asse che taglia a metà il cilindro. Per l'ipotesi di omogeneità per traslazioni parallele all'asse il cilindro è simmetrico rispetto all'asse considerato. Esiste di certo un asse principale che non appartiene al piano, allora di certo anche il suo simmetrico è un asse principale con lo stesso momento di inerzia. Di conseguenza tutto lo spazio da loro generato è composto di assi principali. Se tale spazio ha dimensione 1 allora è facile accorgersi che l'asse principale deve essere proprio l'asse del cilindro, altrimenti l'asse del cilindro appartiene allo spazio generato e perciò è comunque un asse principale.
	\item Sia $\vec B$ il baricentro e $\vec O$ il centro del cilindro. Sia $\theta$ l'angolo tra $\overrightarrow{OB}$ e la verticale.
	
	Pongo un sistema di assi cartesiani in modo che quando $\theta=0$ valgano $\vec O=(0,R)$ e $\vec B=(0,R-l_B)$.
	
	Per la condizione di puro rotolamento si ha $\vec O=(-\theta R, R)$, mentre per facili considerazioni cinematiche si ha $\overrightarrow{OB}=(l_B\sin\theta,-l_B\cos\theta)$.
	
	Unendo queste due si scrivono facilmente posizione e velocità di $\vec B$:
	\begin{align}
		\vec B &=\vec O+\overrightarrow{OB}=(-\theta R+l_B\sin\theta , R-l_B\cos\theta) \label{cno:posizione}\\
		\dot{\vec B} &= (-\dot\theta R+l_B\dot\theta\cos\theta, l_B\dot\theta\sin\theta) \label{cno:velocita}
	\end{align}
	
	Sfruttando \cref{cno:posizione} è facile scrivere l'energia potenziale del sistema:
	\begin{equation}\label{cno:potenziale}
		U=Mg(R-l_B\cos\theta) \Longrightarrow U=\frac12 Mgl_B\theta^2
	\end{equation}
	dove l'ultima uguaglianza vale, a meno di costanti, approssimando al second'ordine. Da notare che prima di approssimare si nota subito che l'unica posizione di equilibrio stabile (di minimo del potenziale) è $\theta=0$ e perciò tutte le approssimazioni saranno fatte intorno a $\theta=0$.
	
	Ed ora sfruttando \cref{cno:velocita} calcolo l'energia cinetica al second'ordine, scomponendo in cinetica del baricentro e rotazionale:
	\begin{equation}\label{cno:cinetica}
		T=\frac 12 I_B \dot\theta^2 +\frac 12 M\dot{\vec B}^2\approx 
		\frac 12 I_B \dot\theta^2 +\frac 12 M\left(-\dot\theta R+l_B\dot\theta\right)^2=
		\frac 12\left(I_B+M(R-l_B)^2\right)\dot\theta^2
	\end{equation}
	
	Perciò ora, sfruttando \cref{cno:potenziale,cno:cinetica} posso scrivere la conservazione dell'energia e derivare, ottenendo:
	\begin{equation*}
		0=\frac{\de}{\de t}(U+T) \implies
		0=\dot\theta\left(Mgl_B\theta+\left(I_B+M(R-l_B)^2\right)\ddot\theta\right)\implies
		\ddot\theta=-\frac{Mgl_B}{I_B+M(R-l_B)^2}\theta
	\end{equation*}
	ma quest'ultima è l'equazione classica del moto armonico, da cui deduco che il periodo è:
	\begin{equation*}
		T=\frac{2\pi\sqrt{I_B+M(R-l_B)^2}}{\sqrt{Mgl_B}}
	\end{equation*}

\end{itemize}

\end{document}
