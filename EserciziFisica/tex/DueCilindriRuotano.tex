\documentclass[../main.tex]{subfiles} 
\begin{document}

\exercise[10/03/2014]{Due cilindri ruotano l'uno nell'altro} %dcr
\textex
È fissato un cilindro cavo di massa $m$ e raggio $2a$ con un perno sulla sua circonferenza, intorno al quale può ruotare.
All'interno di questo cilindro ne è presente un altro, anch'esso cavo, di massa $m$ e raggio $a$, che si muove di puro rotolamento sul bordo del cilindro grande.

Studiare le piccole oscillazioni del sistema intorno alla posizione d'equilibrio sotto l'effetto della gravità.
\solution
Sia $\vec P$ il punto fisso corrispondente al perno che fissa il cilindro grande, sia $\vec Q$ il centro del primo cilindro e $\vec R$ il centro del secondo. Chiamo $\varphi$ l'angolo tra $\overrightarrow{PQ}$ e la verticale, e $\theta$ l'angolo tra $\overrightarrow{QR}$ e la verticale.

Pongo un sistema di assi cartesiani in modo che l'origine sia $\vec P$ e l'asse $\hat y$ punti verso il basso.

Nel sistema di coordinate descritto risultano valere:
\begin{equation}\begin{split}\label{dcr:Posizione}
	\vec Q&=2a(\sin\varphi,\cos\varphi) \\
	\vec R&=\vec Q + a(\sin\theta,\cos\theta)
\end{split}\end{equation}
e derivando ottengo anche:
\begin{equation}\begin{split}\label{dcr:Velocita}
	\dot{\vec Q}&=2a\dot{\varphi}(\cos\varphi,-\sin\varphi) \\
	\dot{\vec R}&=\dot{\vec Q} + a\dot{\theta}(\cos\theta,-\sin\theta)
\end{split}\end{equation}

Voglio calcolare l'energia cinetica e potenziale del sistema.

Per quanto riguarda l'energia potenziale (che pongo nulla nella posizione d'equilibrio) è molto facile ricavarla sfruttando \cref{dcr:Posizione}:
\begin{equation}\label{dcr:Potenziale}
	U=mg(2a)+mg(3a)-mgQ_y-mgR_y=mga\left(4(1-\cos\varphi)+(1-\cos\theta)\right)
	\approx \frac{mga}2\left(4\varphi^2+\theta^2\right)
\end{equation}
dove nell'ultima identità ho approssimato al second'ordine.

Per quanto riguarda l'energia cinetica del cilindro grande, è facile decomporla in cinetica di $\vec Q$ più rotazionale e poi calcolarla usando \cref{dcr:Velocita}:
\begin{equation}\label{dcr:Cinetica1}
	T_1=\frac 12 m {\dot{\vec Q}}^2+\frac 12I{\dot\varphi}^2=4ma^2\dot\varphi^2
\end{equation}

Anche per calcolare l'energia cinetica del cilindro piccolo la decompongo in cinetica più rotazionale, ma risulta più difficile capire la velocità angolare del cilindro.

È arduo spiegare a parole come trovare l'angolo di cui ruota il cilindro piccolo in funzione di $\varphi,\theta$ ma provo ora a descrivere il metodo.
È chiaro che il termine $\varphi$ è solo addittivo (sia sulla rotazione del cilindro piccolo, sia sull'angolo $\theta$, quindi lo tengo costante, mentre noto che se l'angolo $\theta$ passa da $0$ a $\theta_0$ il cilindro risulta essere ruotato di $-\theta_0$ (questo va fatto guardando un po' gli angoli della figura).
Riaggiungendo la variazione di $\varphi$ ottengo che l'angolo di rotazione del cilindro (partendo a contare da quando è tutto nella posizione di equilibrio) risulta essere $2\varphi-\theta$.

Perciò ora ho gli strumenti per calcolare l'energia cinetica del cilindro piccolo (sfrutto ancora \cref{dcr:Velocita}):
\begin{equation}\begin{split}\label{dcr:Cinetica2}
	T_2=\frac 12 m {\dot{\vec R}}^2+\frac 12 I \dot{(2\varphi-\theta)}^2=&
	\frac12m\left(4a^2{\dot\varphi}^2+a^2{\dot\theta}^2+4a^2\dot\theta\dot\varphi(\cos\varphi\cos\theta+\sin\varphi\sin\theta)\right)
	+\frac12ma^2\left(2\dot\varphi-\dot\theta\right)^2 \\
	=& \frac 12ma^2\left(8{\dot\varphi}^2+2{\dot\theta}^2
	+4\dot\varphi\dot\theta(\cos\varphi\cos\theta+\sin\varphi\sin\theta-1)\right) \\
	\approx & \frac12ma^2\left(8{\dot\varphi}^2+2{\dot\theta}^2\right)
\end{split}\end{equation}
dove nell'ultimo passaggio ho approssimato al second'ordine.

Unendo \cref{dcr:Potenziale,dcr:Cinetica1,dcr:Cinetica2} ottengo che al second'ordine risulta:
\begin{equation}\label{dcr:Energia}
	E=\frac12ma^2\left(16{\dot\varphi}^2+2{\dot\theta}^2\right)+\frac{mga}2\left(4\varphi^2+\theta^2\right)
	\Rightarrow \frac{E}{ma}=
	\frac12a\left(16{\dot\varphi}^2+2{\dot\theta}^2\right)
	+\frac12g\left(4\varphi^2+\theta^2\right)
\end{equation}

Ora applicando la teoria delle oscillazioni a più gradi di libertà \cref{opgl} posso trovare le piccole oscillazioni del sistema (nelle coordinate $\begin{psmallmatrix}\varphi \\ \theta\end{psmallmatrix}$).
Noto che in questo caso le matrici da studiare $K,M$ sono già diagonali:
\begin{align*}
	M=a\begin{pmatrix}
		16 & 0 \\
		0  & 2
	\end{pmatrix}
	& &
	K=g\begin{pmatrix}
		4 & 0 \\
		0 & 1
	\end{pmatrix}
\end{align*}
e perciò ottengo che, detta $\omega$ la frequenza dei moti normali, vale:
\begin{equation*}
	0=\det \left ( K-\omega^2M \right )=(4g-16a\omega^2)(g-2a\omega^2) \Rightarrow \omega^2=\frac g{4a},\frac g{2a}
\end{equation*}
e l'autovalore relativo a $\omega^2=\frac g{4a}$ è $\begin{psmallmatrix}1 \\ 0\end{psmallmatrix}$ mentre quello relativo a $\omega^2=\frac g{2a}$ è $\begin{psmallmatrix}0 \\ 1\end{psmallmatrix}$. Perciò i moti armonici sono completamente disaccoppiati tra $\varphi$ e $\theta$.

In conclusione ottengo che $\varphi$ varia di moto armonico con $\omega_\varphi=\sqrt{\frac g{4a}}$ e con ampiezza dipendente dai dati iniziali, analogamente $\theta$ varia di moto armonico con $\omega_\theta=\sqrt{\frac g{2a}}$ e con ampiezza dipendente dai dati iniziali.


\end{document}