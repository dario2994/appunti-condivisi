\documentclass[../main.tex]{subfiles} 
\begin{document}

\exercise[31/10/2013]{Circonferenza con Molla} %ccm

\textex
Una pallina di massa $m$ si può muovere senza attrito su una guida circolare di raggio $R$ disposta verticalmente, ed è collegata a una molla (lunghezza a riposo $l$, costante elastica $k$) che ha l'altro estremo attaccato al punto più alto della circonferenza. Quali sono le condizioni di equilibrio (stabile o instabile)? Calcolare la pulsazione delle piccole oscillazioni attorno alla posizione di equilibrio stabile. % Si capisce o è necessario un disegno?

\solution
%Forze che agiscono
Le forze che agiscono sulla pallina sono:
\begin{itemize}
\item Forza di gravità: è conservativa
\item Reazione vincolare: è perpendicolare alla velocità perchè non c'è attrito, quindi non compie lavoro
\item Forza di richiamo della molla: è conservativa
\end{itemize}
Di conseguenza, l'energia meccanica totale della pallina si conserva.

%Conservazione dell'energia
Sia $\theta$ l'angolo che il raggio compreso tra il diametro verticale e il raggio che collega il centro della circonferenza con la pallina (per chiarire, quando la pallina è nel punto più basso ho $\theta = 0$, quando sta sul diametro orizzontale ho $\theta = \frac{\pi}{2}$). L'energia cinetica della pallina è $T=\frac{1}{2}mR^2\dot\theta^2$, mentre l'energia potenziale gravitazionale è data dalla somma dell'energia potenziale gravitazionale e dell'energia potenziale elastica: $U_{gr} = -mgR\cos\theta$ (considero $U_{gr}=0$ quando la pallina sta sul diametro orizzontale della circonferenza) e $U_{el} = \frac{1}{2}k\left ( 2R\cos\frac{\theta}{2} - l\right )^2$. Dunque
\begin{equation}\label{ccm:ConsEnergia}
	E = \frac{1}{2}mR^2\dot\theta^2 - mgR\cos\theta + \frac{1}{2}k\left ( 2R\cos\frac{\theta}{2} - l\right ) ^2
\end{equation}

%Derivo e trovo i punti di equilibrio
I punti di equilibrio si hanno quando l'energia potenziale è minima (equilibrio stabile) o massima (equilibrio instabile). Per trovarli studio la derivata dell'energia potenziale in funzione di $\theta$. Ho che $U(\theta) =  - mgR\cos\theta + \frac{1}{2}k\left ( 2R\cos\frac{\theta}{2} - l\right ) ^2$ dunque
\begin{equation*}
	\frac{dU}{d\theta} = gmR\sin\theta + k\left ( 2R\cos\frac{\theta}{2} - l\right )\left (-R\sin\frac{\theta}{2}\right ) = 2R\sin\frac{\theta}{2}\cos\frac{\theta}{2} \left (mg - Rk \right ) + Rkl\sin\frac{\theta}{2}
\end{equation*}
Gli zeri della funzione (cioè i massimi/minimi dell'energia potenziale) sono dati da $$\sin\frac{\theta_1}{2} = 0,\  \cos\frac{\theta_2}{2}=\frac{kl}{2\left ( Rk - mg\right )}$$
Osservo in particolare che, poichè $\theta$ è compreso tra $0$ e $\frac{\pi}{2}$, la seconda soluzione ha senso solo se $0 \le \frac{kl}{2\left ( Rk - mg\right )} \le 1$ cioè solo quando $l\le2\left ( R - \frac{gm}{k}\right )$.
Per capire se l'equilibrio è stabile o no, e per trovare la frequenza delle piccole oscillazioni attorno alla posizione di equilibrio, derivo la \cref{ccm:ConsEnergia} (rispetto al tempo) e sviluppo in un intorno di $\theta_1$ e poi $\theta_2$. Poiché l'energia è costante, la sua derivata è nulla, quindi 
\begin{gather}
	\frac{dE}{dt}=2 \frac{1}{2}mR^2\dot\theta\ddot\theta-mgR\left (-\dot\theta\sin\theta\right ) + \frac{1}{2}k\cdot 2\left (2R\cos\frac{\theta}{2}-l\right ) \left ( -\frac{1}{2}\cdot 2R\sin\frac{\theta}{2}\right )\dot\theta = 0 \notag \\
	\Longrightarrow mR^2\dot\theta\ddot\theta+mgR\dot\theta\sin\theta-2kR^2\dot\theta\sin\frac{\theta}{2}\cos\frac{\theta}{2}+klR\dot\theta\sin\frac{\theta}{2}=0 \notag\\
	\Longrightarrow mR\ddot\theta = \left ( kR - mg \right )\sin\theta - kl\sin\frac{\theta}{2} \label{ccm:EqMotoGiusta}
\end{gather}

che, sviluppata al primo ordine in un intorno di $\theta_0$, diventa:
\begin{equation}
	mR\ddot\theta = \left ( kR - mg \right ) \left ( \sin\theta_0 + \cos\theta_0\left ( \theta - \theta_0 \right ) \right ) - kl \left ( \sin\frac{\theta_0}{2} + \frac{1}{2}\cos\frac{\theta_0}{2}\left ( \theta - \theta_0 \right ) \right )
\end{equation}

Ponendo $C(\theta_0) = \left ( kR - mg \right ) \left ( \sin\theta_0 - \theta_0\cos\theta_0\right ) - kl\left (\sin\frac{\theta_0}{2} - \frac{1}{2}\theta_0\cos\frac{\theta_0}{2}\right )$ l'equazione del moto può essere riscritta nella forma
\begin{equation}\label{ccm:EqApprossimata}
	mR\ddot\theta = \left ( \cos\theta_0\left (kR-mg\right ) -\frac{1}{2}kl\cos\frac{\theta_0}{2}\right ) \theta + C(\theta_0)
\end{equation}


% theta = 0
Per $\theta_0 = \theta_1 = 0$ ottengo (sostituendo nella \cref{ccm:EqApprossimata})
\begin{equation*}
	mR\ddot\theta = \left ( -mg + kR -\frac{1}{2}kl\right )\theta + C(0)
\end{equation*}
Dunque l'equilibrio è stabile se il coefficiente di $\theta$ è negativo, cioè se $l> 2\left ( R - \frac{mg}{k} \right )$ e in questo caso la frequenza angolare $\omega_1$ è $$\omega_1 = \sqrt{\frac{2gm-2kR+kl}{2mR}}$$

%theta = schifo
Per $\theta_0=\theta_2$ la \cref{ccm:EqApprossimata} diventa:
\begin{equation*}\begin{split}
	mR\ddot\theta = \left (  \left ( kR-mg\right ) \left ( 2\cos^2\frac{\theta_2}{2}-1\right ) - \frac{kl}{2}\cos\frac{\theta_2}{2}\right ) \theta  + C(\theta_2) =\\
	= \frac {k^2 l^2 - 2\left (Rk-mg\right ) ^2-\frac{1}{2}k^2 l^2}{2\left ( Rk - mg\right )}\theta + C(\theta_2) = \frac{k^2 l^2 - 4 \left ( Rk - mg\right )^2}{4\left (Rk-mg\right )} \theta + C(\theta_2)
\end{split}\end{equation*}
Se è verificata la condizione  $l<2\left ( R - \frac{gm}{k}\right )$ (condizione necessaria per l'esistenza di $\theta_2$), allora il coefficiente di $\theta$ è negativo, dunque si tratta di equilibrio stabile e la frequenza angolare $\omega_2$ è 
$$ \omega_2 = \sqrt {\frac{4\left ( Rk-mg\right )^2-k^2 l^2 }{4mR\left (Rk-mg\right )}} $$

%Ultimo caso --> theta = 0 ma con condizione ulteriore


\end{document}
