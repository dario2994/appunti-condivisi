\documentclass[../main.tex]{subfiles} 
\begin{document}

\exercise{Forze conservative e centrali} %fcc

\textex
Data una forza nella forma:
$$
\left\{
\begin{array}{rl}
F_x & = ax+by \\
F_y  & = cx+dy
\end{array}
\right.
$$
Dire quando essa è conservativa e quando centrale.


\solution

Una forza è conservativa se $\nabla \times F=0$ (vedere Morin pag. 150). In questo caso quindi pochè $F_z$ è nullo:
\begin{equation}\label{fcc:1}
 \frac{\partial F_y}{\partial x}- \frac{\partial F_x}{\partial y}=0
\end{equation}
cioè $b=c$.

Affinchè $F$ invece sia centrale è necessario che, oltre ad essere conservativa, essa dipenda solo da $r=\sqrt{x^2+y^2}$. Che è equivalente a richiedere
che il potenziale dipenda solo da $r$. Calcolo quindi il potenziale $U(x,y)$, sapendo che $F_x=-\frac{\partial U(x,y)}{ \partial x}$ e che
$F_y=-\frac{\partial U(x,y)}{ \partial y}$ quindi:
\begin{equation}\label{fcc:2}
 -U(x,y)=\int{F_x \de x}=\int{F_y \de y}=\frac{a}{2}x^2+bxy+g(y)=bxy+\frac{d}{2}y^2+f(x)=\frac{a}{2}x^2+bxy+\frac{d}{2}y^2
\end{equation}
Ora sostituisco $x=\pm \sqrt{r^2-y^2}$ ed ottengo: $-U=\frac{a}{2}(r^2-y^2) \pm b \sqrt{r^2-y^2}y+\frac{d}{2}y^2$ poichè voglio che la $y$ sparisca
 devo avere che $b=0$ e $a=d$ ottenendo così che $U(r)=\frac{a}{2}r^2$


\end{document}
