\documentclass[../main.tex]{subfiles} 
\begin{document}

\exercise[13/02/2014]{Rotazione del tensore d'inerzia} %rti
\label{ex:rti}

\textex
Come si trasforma il tensore d'inerzia dopo una rotazione del sistema di riferimento?

\solution
Mi pongo in un sistema di riferimento $A$ rispetto al quale posso applicare la \cref{ten:MomentoAngolare}, cioè un sistema che si muove solidalmente rispetto al corpo rigido considerato e il cui centro rispetta la \cref{ten:PointCondition}. Chiamo $A'$ il suo ruotato, che rispetta ancora tutte le ipotesi per applicare la \cref{ten:MomentoAngolare}. Vale quindi:
\begin{equation} \label{rti:eqMomentoAngolarePrima}
	L_i=I_{ij}\omega_j
\end{equation}
dove $L_i$ è il momento d'inerzia, $\omega_j$ è la velocità angola del corpo e $I_{ij}$ è il tensore d'inerzia definito dall'equazione \cref{ten:definition}, tutti calcolati nel sistema di riferimento $A$. Quindi definiti $L_i'$, $\omega_j'$ e $I_{ij}'$ rispettivamente il momento angolare, la velocità angolare e il tensore d'inerzia rispetto ad $A'$, vale analogamente che:
\begin{equation}\label{rti:eqMomentoAngolareDopo}
	L_i'=I_{ij}'\omega_j'
\end{equation}

Inoltre chiamata $R$ la matrice ortogonale che rappresenta la rotazione considerata, abbiamo che un qualsiasi vettore $v_i$ rispetto ad $A$, dopo la rotazione vale $v_i'=R_{ij}v_j$.

$L_i$ e $\omega_j$ sono vettori, quindi per quanto appena detto nel sistema di riferimento $A'$ valgono
\begin{equation}\label{rti:vettoriDopo}
	L_i'=R_{ij}L_j, \quad \omega_j'=R_{jk}\omega_k
\end{equation}

Sostituendo ora la \cref{rti:vettoriDopo} e la \cref{rti:eqMomentoAngolarePrima} nella \cref{rti:eqMomentoAngolareDopo}, ottengo:
\begin{gather*}
	L_i'=I_{ij}'\omega_j' \Longrightarrow R_{ij}L_j=I_{ij}'R_{jk}\omega_k \Longrightarrow R_{ij}I_{jk}\omega_k=I_{ij}'R_{jk}\omega_k \\
	\Longrightarrow RI\omega=I'R\omega
\end{gather*}
Poichè quest'ultima equazione deve valere per ogni velocità angolare $\omega$ (il tensore d'inerzia dipende solo dal corpo e dal sistema di riferimento), ottengo:
\begin{gather*}
	RI=I'R \Longrightarrow I'=RIR^{T}
\end{gather*}
che è proprio quello che volevo ottenere.

\end{document}