\documentclass[../main.tex]{subfiles} 
\begin{document}

\exercise{Masse che interagiscono con una forza strana} %mis
\textex
Due masse $m_1$ ed $m_2$ interagiscono fra loro con una forza $\vec F$, che dipende dalla distanza $\vec r$ fra le due masse e che è definita come segue:
\begin{equation*}
\begin{cases}
	\vec F=-k\vec r, &\mbox{se } |\vec r|<r_*\\
	\vec F=0, & \mbox{se } |\vec r|\ge r_*\\
\end{cases}
\end{equation*}

Inizialmente la massa $m_1$ si muove con velocità $v_0$ e parametro d'impatto $b$ rispetto alla massa $m_2$.
Trovare la minima distanza di avvicinamento nei casi:
\begin{enumerate}
	\item $m_2\gg m_1$
	\item $m_1=m_2$
\end{enumerate}

\solution
Innanzitutto ho che se $b\ge r_*$ la minima distanza di avvicinamento è banalmente $b$, poichè non c'è interazione fra le due masse.

Considero quindi ora solo il caso in cui $b<r_*$. 
Per la \cref{ForzaMassaRidotta} ho che vale:
\begin{equation*}
	\vec F=\mu \vec r
\end{equation*}
da cui, per la \cref{Cinetica2Corpi}, ottengo:
\begin{equation}\label{mis:energia}
	E=\frac12(m_1+m_2)\dot{\overrightarrow{r_Q}}^2+\frac12\mu\dot{\vec{r}}^2+U(r)
\end{equation}
dove $U(r)=-\int_{r_*}^r F(x) dx$, cioè:
\begin{equation*}
	\begin{cases}
		U(r)=-\frac12k(r_*^2-r^2) &\mbox{per } r<r_*\\
		U(r)=0 &\mbox{per } r\ge r_*
	\end{cases}
\end{equation*}

Sul sistema non agiscono forze esterne, quindi la velocità del centro di massa è costante.
Si ha quindi che $E'=E-\frac12(m_1+m_2)\dot{\overrightarrow{r_Q}}^2$ è una costante del moto e in particolare, utilizzando la \cref{mis:energia}, vale:
\begin{equation*}
	E'=\frac12\mu\dot{\vec{r}}^2+U(r)=\frac12\mu\dot r^2+\frac12\mu r^2\dot \theta^2+U(r)
\end{equation*}

So però che il momento angolare rispetto al centro di massa è una costante del moto e vale 
\begin{equation*}
	\vec L=\mu \vec r \times \dot {\vec r}=\mu r^2\dot\theta \hat z
\end{equation*}




\end{document}