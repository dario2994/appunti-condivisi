\documentclass[../main.tex]{subfiles} 
\begin{document}

\exercise{Nome del problema} %ndp

\textex
Testo del problema

Lorem Ipsum e così via finchè la capra canta la capra suona anche.

Fine Testo del problema.

\solution
Soluzione del problema.

Un po' di boiate a parole, poi qualche formuletta:
\begin{equation}\label{ndp:gamma}
	\Gamma(x)=\lim_{n\to\infty} \dfrac{n^xn!}{x(x+1)\cdots (x+n-1)(x+n)}
\end{equation}
Poi un'equazione senza numerino:
\begin{equation*}
	a=b \Rightarrow b=c \Rightarrow 1=0
\end{equation*}

E poi, tutto soddifatto, richiamo la \cref{ndp:gamma}.

Fine Soluzione del problema.

\solution[2]
Questa è la seconda soluzione del problema...
\begin{equation}
	1+2+\dots+n=\frac{n(n+1)}2
\end{equation}

\solution[3]
E per esagerare questa è la terza.

\end{document}
