\documentclass[../main.tex]{subfiles}
\begin{document}

\exercise[31/10/2013]{Vettore di Lenz} %vl

\textex
Dimostrare che in un campo di forze centrale di tipo Coulombiano, quindi con $\vec{F}=-\dfrac{\alpha}{r^2}\hat r$, il vettore:
\begin{equation} \label{vl:Definizione}
	\vec{A}=-\alpha \hat r + \vec v\times \vec L
\end{equation}
è una costante del moto e calcolare $|\vec A|,\ \vec{A} \cdot \vec{r}$.


\solution
Per mostrare che il vettore di Lenz è una costante del moto dimostro che la sua derivata rispetto al tempo è nulla.

Sfrutto il fatto che $\vec L$ è a sua volta una costante del moto:
\begin{equation}\label{vl:Derivata}
	0=\frac{\de \vec A}{\de t}=\frac{\de \left(-\alpha\hat r + \vec v\times \vec L\right) }{\de t} \iff \alpha\dot{\hat r} = \vec a\times \vec L
\end{equation}

Ma usando l'espressione della forza in un campo Coulombiano, la definizione del momento angolare, le proprietà del prodotto vettore 
e la formula della velocità in polari \cref{VelCooPolari} vale la seguente catene di identità:
\begin{equation}\label{vl:Uguaglianze}
	\vec a\times \vec L=\left(\dfrac{\alpha}{mr^2}\right)\times\left(m\vec r\times \vec v\right)=\frac{\alpha}{r} \left(\hat r\times \left(\hat r\times \vec v\right)\right)
        =\frac{\alpha}{r} \left(\hat r\left(\hat r \cdot \vec v\right)-\vec v\left(\hat r\cdot \hat r\right)\right)
	=\frac{\alpha}{r} \left(\hat r\dot{r}-\vec v\right)=\frac{\alpha}{r}\left(r\dot{\hat r}\right)=\alpha \dot{\hat r}
\end{equation}
Sfruttando le \cref{vl:Definizione,vl:Uguaglianze} ottengo che il vettore $\vec A$ è una costante del moto.

Ora scrivo alcune identità vettoriali per poi usarle nel calcolo di $\vec A\cdot \vec r$.

Vale la seguente formula per il momento angolare:
\begin{equation*}
	\vec L=m\vec r\times \vec v=mr\left(\hat r\times (r\dot{ \hat r })+\hat r\times(\dot r \hat r)\right)
	= mr^2\hat r\times\dot{\hat r}
\end{equation*}
e applicando questa, sfruttando le proprietà del prodotto vettore e scalare ottengo:
\begin{equation}\label{vl:VperL}
	\vec v\times \vec L=(\dot r \hat r \times \vec L)+r\left(\dot{\hat r}\times \vec L\right)
	=(\dot r \hat r \times \vec L)+mr^3\left(\hat r(\dot{\hat r}\cdot \dot{\hat r})-\dot{\hat r}(\dot{\hat r}\cdot \hat r) \right)
	=(\dot r \hat r \times \vec L)+mr^3 \hat r \dot{\hat r}^2
\end{equation}

Inoltre, sfruttando la formula per la derivata dei versori \cref{OmegaPolari} e la definizione di $L$ come scalare ottengo:
\begin{equation*}
	\dot{\hat r}^2=\dot{\theta}^2=\frac{L^2}{m^2r^4}
\end{equation*}
e sostituendo nella \cref{vl:VperL} arrivo a:
\begin{equation}\label{vl:fineConto}
	\vec v\times \vec L=(\dot r \hat r \times \vec L)+\frac{L^2}{mr}\hat r
\end{equation}

Finalmente calcolo $\vec A\cdot \vec r$ applicando \cref{vl:VperL}:
\begin{equation}\label{vl:ScalareL}
	\vec A\cdot \vec r=-\alpha\hat r\cdot \vec r+(\vec v\times L)\cdot \vec r=
	-\alpha+\left((\dot r \hat r \times \vec L)+\frac{L^2}{mr}\hat r\right)\cdot \vec r=
	-\alpha r+\frac{L^2}{m}
\end{equation}
Alternativamente posso arrivare a questo risultato usando direttamente le proprietà del prodotto vettoriale:
\begin{equation*}
	\vec A\cdot \vec r=-\alpha\hat r\cdot \vec r+(\vec v\times L)\cdot \vec r=
	-\alpha r+(\vec r\times \vec v)\cdot \vec L=-\alpha r + \left(\frac{\vec L}{m}\right)\cdot \vec L=
	-\alpha r+\frac{L^2}m
\end{equation*}

Ora calcolo $|\vec A|$ sfruttando $\vec v\perp \vec L$ :
\begin{equation}\label{vl:Norma}
	\vec{A} ^{\,2}=\alpha^2-2\alpha\hat r\cdot\left(\vec v\times \vec L\right)+(\vec v\times \vec L)^2=
	\alpha^2-\frac{2\alpha L^2}{mr}+v^2L^2=\alpha^2+\frac{2L^2}m\left(-\frac{\alpha}{r}+\frac12mv^2\right)=\alpha^2+\frac{2L^2E}m
\end{equation}

Ed ora unendo i risultati \cref{vl:ScalareL,vl:Norma} ottengo la formula polare delle orbite Kepleriane:
\begin{equation}\label{vl:OrbiteKepleriane}
	|\vec A|r\cos \theta=-\alpha r+\frac{L^2}m\Rightarrow r=\frac{\frac{L^2}m}{\alpha +|\vec A|\cos \theta}=
	\frac{\frac{L^2}{\alpha m}}{1+\left(\sqrt{1+\frac{2L^2E}{\alpha^2m}}\right)\cos \theta}
\end{equation}


\end{document}
















