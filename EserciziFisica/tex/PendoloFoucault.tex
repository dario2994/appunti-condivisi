\documentclass[../main.tex]{subfiles} 
\begin{document}

\exercise[16/12/2013]{Il pendolo di Foucault}%pf

\textex

Vogliamo studiare il moto di un pendolo matematico di lunghezza $l$ posto sulla Terra ad una latitudine $\lambda$.
In particolare ci interessa l'approssimazione al prim'ordine in $\theta << 1 $ dove $\theta$ è l'angolo forma il pendol
con la verticale rispetto il filo a piombo
\footnote{consideriamo implicitamente che l'asse $z$ non sia diretto ortogonalmente al terreno,
ma lungo la direzione del filo a piombo. La differenza è legata agli effetti centrifughi della rotazione terrestre}.

\solution
Scegliamo come sistema di riferimento cartesiano $(O, x,y,z)$ dove $O$ è il punto di equilibrio del pendolo, $z$ è uscente
dalla terra diretto come il filo a piombo, $y$ che punta a nord.

Avendo scelto di approssimare il moto al prim'ordine, la coordinata $z$ del pendolo è costante. La soluzione userà, pertanto,
solo le coordinate $x,y$.

Richiamando la \cref{ForzaNonInerziale} relativa ai sistemi di riferimento non inerziali abbiamo che 
\[
  m\vec{a_r}=\vec{F}-m\vec{a_{tr}}-m\left[2\vec{\omega}\times\vec{v_r}+\dot{\vec{\omega}}\times\vec{r}+
  \vec{\omega}\times(\vec{\omega}\times\vec{r})\right]
\]

Il termine $\dot{\vec{\omega}}\times\vec{r}$ è nullo perché $\dot{\vec{\omega}}=0$.
Trovandoci nel sistema in cui $z$ è diretto lungo il filo a piombo, i termini
$m\vec{g}-m\overrightarrow{a_{tr}}-\vec{\omega}\times(\vec{\omega}\times\vec{r})$ li possiamo scrivere come
$m\tilde{g}\hat{g} = m\tilde{g}\hat{z}$ dove $\tilde{g}$ tiene conto degli effetti
dovuti alle forze apparenti del sistema non inerziale. Tuttavia, è un facile esercizio vedere che $g-\tilde{g}<< g$, è dunque
possibile approssimare $\tilde{g}$ a $g$. In questa soluzione cercheremo di tenere comunque $\tilde{g}$. Otteniamo dunque

\begin{equation}
 \label{pf:Fma}
  \vec{a_r}=\tilde{g}\hat{z}+\vec{T}-2\vec{\omega}\times\vec{v_r} 
\end{equation}


Cerchiamo ora di ridurre \cref{pf:Fma} ad un sistema nelle variabili $x,y$, ovvero le coordinate del pendolo nel nostro sistema di riferimento.
Al prim'ordine in $\theta$ consideriamo la tensione $\vec{T}$ del pendolo costante. La proiezione dei primi 2 termini in $\hat{x}$ e $\hat{y}$
vale pertanto rispettivamente $-x\omega^2_0$ e $-y\omega^2_0$, dove $\omega^2_0 = \frac{g}{l}$.

Quanto all'altro termine, abbiamo che $\vec{\omega}=\omega (\sin \lambda \hat{y} + \cos \lambda \hat{z})$; quindi
$\vec{\omega}\times\vec{v_r} = \omega (-\cos \lambda \dot{y}\hat{x} + \cos \lambda \dot{x}\hat{y} - \sin \lambda \dot{x} \hat{z})$.

Dopo aver trascurato il termine lungo $\hat{z}$ essendo trascurabile rispetto $\tilde{g}$, il sistema che ne risulta è

\begin{align*}
	\ddot{x} &=  -x\omega^2_0 +2\omega\cos \lambda \dot{y}\\
	\ddot{y} &=  -y\omega^2_0 -2\omega\cos \lambda \dot{x}.
\end{align*}

Detto $\xi = x + iy$, dalla somma dell'equazione di $\ddot{x}$ con $i$ quella di $\ddot{y}$ si ottiene una sola equazione in $\xi$.
Presa la soluzione, la parte reale darà la funzione di $x$, la parte immaginaria la funzione di $y$.
L'equazione in $\xi$ è

\begin{equation}
 \label{pf:xieq}
  \ddot{\xi} = -\xi \omega^2_0 - i 2\omega\cos \lambda \dot{\xi}.
\end{equation}

Il polinomio associato è $x^2 + 2i\omega\cos\lambda x + \omega^2_0$, le sue radici sono
$ x_{\pm} = i\left(\omega\cos\lambda \pm \sqrt{\omega^2\cos^2 \lambda + \omega^2_0} \right) = \Omega_{\pm}$,
quindi la soluzione generica, per opportuni $A,B$ complessi, sarà del tipo

\[
  \xi = Ae^{i\Omega_+ t} + Be^{i\Omega_-t}.
\]

e la soluzione in $x,y$, per opportune costanti $a,b,\alpha,\beta$ associate ad $A,B$,

\begin{align*}
	x &=  a\cos(\Omega_+ t + \alpha) + b\cos(\Omega_- t + \beta)\\
	y &=  a\sin(\Omega_+ t + \alpha) + b\sin(\Omega_- t + \beta).
\end{align*}

Ora che abbiamo la soluzione generale, consideriamo quella particolare in cui il pendolo parte da $(0,0)$ con una velocità iniziale diretta
lungo $y$. Si vede facilmente che le costanti arbitrarie che soddisfano sono $\alpha=\beta=0$, $a=d$, $b=d$
per un certo $d$ in funzione della velocità iniziale.
Ora, sfruttando una piccola identità trigonometrica possiamo riscrivere le soluzioni in $x,y$ come

\begin{align*}
	x &=  2d\cos(\omega\cos\lambda t)\cos\left(\sqrt{\omega^2\cos^2 \lambda + \omega^2_0} t\right)\\
	y &=  2d\sin(\omega\cos\lambda t)\cos\left(\sqrt{\omega^2\cos^2 \lambda + \omega^2_0} t\right).
\end{align*}

Effettivamente il moto si capisce meglio se lo si descrive in coordinate polari:

\begin{align*}
	r &=  2d\cos\left(\sqrt{\omega^2\cos^2 \lambda + \omega^2_0} t\right)\\
	\theta &=  \omega\cos\lambda t,
\end{align*}

cioè un pendolo che oscilla con pulsazione $\sqrt{\omega^2\cos^2 \lambda + \frac{g}{l}}$ su un piano verticale
il quale ruota su se stesso con velocità angolare $\omega\cos\lambda$.

\end{document}
