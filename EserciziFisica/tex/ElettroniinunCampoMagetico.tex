\documentclass[../main.tex]{subfiles} 
\begin{document}

  \newcommand\ecmvprod{\times}

  \exercise[11/07/14]{Elettroni in un campo magnetico} %ecm
  \textex
  Un atomo, schematizzabile come un sistema di N elettroni sottoposti ad una forza centrale ed interagenti fra loro, \`e soggetto ad un campo magnetico esterno uniforme e costante $\vec B$, che agisce sull'elettrone $i$-esimo con una forza $\vec F_i = \frac{e}{c} \cdot \left(\vec v_i \ecmvprod \vec B\right)$, dove $e$ ed $m$ sono la carica elettrica e la massa dell'elettrone, e $c$ \`e la velocit\`a della luce.

  \begin{itemize}
    \item[(a)] Mostrare che i moti possibili degli elettroni in presenza del campo magnetico, purch\'e sufficientemente piccolo, sono gli stessi di quelli in assenza del campo, sovrapposti ad una rotazione dell'intero sistema con velocit\`a angolare opportuna. \newline
    Qual \`e tale velocit\`a angolare?
    \item[(b)]Qual \`e la condizione che deve soddisfare il campo magnetico?
  \end{itemize}

  \solution
  La forza sull'elettrone $i$-esimo vale:
  
  \begin{equation}
    \label{ecm:forzai}
    \vec F_i= \sum_{j \neq i} \vec f_{ji}(\vec r_i-\vec r_j)+\vec \alpha(\vec r_i)+\frac{e}{c} \cdot \left(\dot {\vec {r_i}} \ecmvprod \vec B\right)
  \end{equation}
  dove $\vec f_{ij}$ \`e la forza di interazione tra gli elettroni e $\vec \alpha$ \`e la forza centrale
  
  
  
  

\end{document}
