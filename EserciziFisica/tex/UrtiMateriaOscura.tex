\documentclass[../main.tex]{subfiles} 
\begin{document}

\exercise[30/01/2014]{Urti con materia oscura} %uma

\textex
Si consideri una galassia sferica. Si supponga che la grande maggioranza delle stelle (massa totale $M$) sia contenuta entro un raggio $R$, con simmetria sferica.\\
a) Calcolare, in questa approssimazione, la velocità di una stella che ruota su una circonferenza attorno alla galassia, posta a una distanza $r > R$ dal centro della galassia.\\
b) Ciò che in realtà si verifica è che la velocità dipende da $r$ secondo la formula
$$ v(r) = \sqrt{\frac{k}{1+\frac{r}{r_0}}}$$
Ci sono allora due possibilità: o la gravità a grandi distanze non si comporta secondo la legge che conosciamo, oppure non era corretto supporre che la massa al di fuori della sfera di raggio $R$ fosse trascurabile.
Assumendo che si verifichi questa seconda ipotesi, calcolare il potenziale della forza e la distribuzione di massa in funzione del raggio $r$.\\
c) Si pensa che questa massa invisibile (materia oscura) sia costituita da particelle, dette neutralini, di massa circa 100 volte maggiore di quella del protone: $m_1 = 100 m_p$.
Inoltre i neutralini hanno, rispetto alla Terra, una velocità $\overrightarrow{v_1}$ di circa $130$ km/s.
Si vuole studiare, sulla Terra, l'urto tra un neutralino e un atomo campione di massa $m_2$. Determinare $m_2$ affinchè dopo l'urto l'energia dell'atomo usato sia massima (si suppone che l'urto sia elastico).

\solution
a) Devo semplicemente usare la formula $\overrightarrow F = m\overrightarrow a$, sapendo che sia la forza sia l'accelerazione (centripeta) sono in direzione radiale. Dunque
\begin{equation}
 m\frac{v(r)^2}{r} = \frac{GMm}{r^2} \Rightarrow v(r) = \sqrt{\frac{MG}{r}}
\end{equation}
b) So che $m\overrightarrow a = \overrightarrow F = -\frac{\de U}{\de r}$. Sostituendo il valore di $v$ trovo che
\begin{equation}
  \frac{mk}{r\left ( 1+\frac{r}{r_0}\right )} = -\frac{\de U}{\de r} \rightarrow 
\frac{mkr_0}{r(r+r_0)} = -\frac{\de U}{\de r} \rightarrow mk\left ( \frac{1}{r} - \frac{1}{r+r_0}\right ) \de r = -\de U
\end{equation}
Calcolo dunque il potenziale, ponendolo nullo quando $r$ è infinito:
\begin{equation}
 \int_{\infty}^{r} mk\left ( -\frac{1}{r} + \frac{1}{r+r_0}\right ) \de r = \int_{\infty}^{U(r)} \de U \Rightarrow
 U(r) = mk \ln \left ( 1 + \frac{r_0}{r}\right )
\end{equation}
Calcolo infine la massa in funzione di $r$: 
\begin{equation}
 \frac{GM(r)m}{r^2} = F(r) = -\frac{\de U}{\de r} = \frac{mkr_0}{r(r+r_0)} \Rightarrow 
 M(r) = \frac{krr_0}{G(r+r_0)}
\end{equation}
c) Per l'equzione degli urti elastici \cref{UrtiElastici}, ho che la velocità dell'atomo di prova dopo l'urto sarà:
\begin{equation*}
 \overrightarrow {{v_2}'} = \frac{m_1}{M}v\hat{v_f}+\overrightarrow {V_Q}
\end{equation*}
A parità di massa $m_2$, vorrei che ${v_2}'$ sia il più grande possibile, e ciò avviene se i vettori $\hat{v_f}$ e $\overrightarrow {V_Q}$ sono nella stessa direzione, cioè se l'angolo di deviazione è nullo.
Trovo le velocità $\overrightarrow {V_Q} = \frac{m_1\overrightarrow {v_1}}{m_1+m_2}$, $\overrightarrow v = \overrightarrow {v_2}-\overrightarrow {v_1} = -\overrightarrow {v_1}$. Calcolo dunque l'energia cinetica dell'atomo di prova dopo l'urto:
$$ E = \frac{1}{2}m_2\left ( {v_2}'\right )^2 = \frac{1}{2} m_2 \left ( \frac{m_1}{m_2+m_1}v_1+\frac{m_1v_1}{m_1+m_2}\right )^2 = \frac{2{m_1}^2m_2{v_1}^2}{\left (m_1+m_2\right )^2} $$
Se pongo $\frac{m_2}{m_1} = x$ l'equazione precedente si riscrive come:
$$ E = 2m_1{v_1}^2\frac{x}{\left ( x+1\right )^2} $$ e questa ha il suo massimo quando $x = 1$ (si vede facilmente derivando), dunque affinchè l'energia cinetica dell'atomo usato sia massima dopo l'urto, la sua massa deve essere uguale a quella di un neutralino.

\end{document}
