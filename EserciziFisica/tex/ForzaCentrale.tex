\documentclass[../main.tex]{subfiles} 
\begin{document}

\exercise[24/10/2013]{Date le orbite spiraliformi trovare la forza centrale} %foc

\textex
In un piano, dato un sistema di riferimento polare di centro $O$ con versori $\hat{r},\hat{\theta}$, una particella di massa $m$ si muove lungo una traiettoria descritta da
\begin{equation}
	\label{foc:traiettoria}
	\theta=\frac{k}{r^\alpha},
\end{equation}
dove $k, \alpha$ sono costanti fissate. Sapendo che la particella \`e soggetta ad una forza centrale $\vec{F}=f(r)\hat{r}$, determinare $f(r)$.

\solution

Ricordiamo che, per \cref{AccCooPolari}, l'accelerazione in coordinate polari \`e data da
\begin{equation*}
	\ddot{\vec{r}}=(\ddot{r}-r\dot{\theta}^2)\hat{r}+(2\dot{r}\dot{\theta}+r\ddot{\theta})\hat{\theta}
\end{equation*}
che unita con la seconda legge della dinamica porta a
\begin{equation}
\begin{cases}
	f(r)=m(\ddot{r}-r\dot{\theta}^2) \label{foc:rcomp}\\
	0=m(2\dot{r}\dot{\theta}+r\ddot{\theta})
\end{cases}
\end{equation}

Dalla seconda equazione ricavo che deve valere
\begin{equation}\label{foc:momangolare}
	0=mr(2\dot{r}\dot{\theta}+r\ddot{\theta})=2mr\dot{r}\dot{\theta}+mr^2\ddot{\theta}=\frac{d(mr^2\dot{\theta})}{dt} \Longrightarrow mr^2\dot{\theta}=L
\end{equation}
con $L$ costante. In effetti quest'ultima considerazione si poteva omettere osservando che quella descritta è proprio la conservazione del momento angolare, che ha modulo $L=mr^2\dot{\theta}$.

Ora, sfruttando \cref{foc:momangolare,foc:traiettoria}, abbiamo
\begin{equation*}
	\frac{L}{mr^2}=\dot\theta=\frac{d\left(\frac{k}{r^{\alpha}}\right)}{dt}=\frac{-k\alpha}{r^{\alpha+1}}\dot{r}
\end{equation*}
da cui
\begin{equation*}
	\dot{r}=-\frac{L}{mk\alpha}r^{\alpha-1}
\end{equation*}
e
\begin{equation}
	\label{foc:r2punti}
	\ddot{r}=-\frac{L(\alpha-1)}{mk\alpha}r^{\alpha-2}\dot{r}=\frac{L^2(\alpha-1)}{m^2k^2\alpha^2}r^{2\alpha-3}
\end{equation}

Infine, nella prima equazione di \cref{foc:rcomp} sostituiamo $\ddot{r}$ dato dalla \cref{foc:r2punti} e $\dot{\theta}$ dato dalla \cref{foc:momangolare}, e otteniamo:
\begin{equation*}
	f(r)=m\left(\frac{L^2(\alpha-1)}{m^2k^2\alpha^2}r^{2\alpha-3}-\frac{L^2}{m^2r^3}\right)
	=\frac{L^2}{mr^3}\left(\frac{\alpha-1}{k^2\alpha^2}r^{2\alpha}-1\right)
\end{equation*}
che era proprio quello che cercavamo.

\end{document}
