\documentclass[../main.tex]{subfiles} 
\begin{document}

\exercise[24/10/2014]{Date le orbite spiraliformi trovare la forza centrale} %foc

\textex
In un piano, dato un sistema di riferimento polare di centro $O$ con versori $\hat{r},\hat{\theta}$, una particella di massa $m$ si muove lungo una traiettoria descritta da
\begin{equation}
	\label{foc:traiettoria}
	\theta=\frac{k}{r^\alpha},
\end{equation}
dove $k, \alpha$ sono costanti fissate. Sapendo che la particella \`e soggetta ad una forza centrale $\vec{F}=f(r)\hat{r}$, determinare $f(r)$.

\solution

Ricordiamo che l'accelerazione in coordinate polari \`e data da
\[
	\ddot{\vec{r}}=(\ddot{r}-r\dot{\theta}^2)\hat{r}+(2\dot{r}\dot{\theta}+r\ddot{\theta})\hat{\theta},
\]

che unita con la seconda legge della dinamica porta a
\begin{align}
	f(r)&=m(\ddot{r}-r\dot{\theta}^2), \label{foc:rcomp}\\
	0&=m(2\dot{r}\dot{\theta}+r\ddot{\theta}).
\end{align}

La seconda \`e equivalente a
\begin{equation}
	\label{foc:momangolare}	
	r^2\dot{\theta}=c
\end{equation}
con $c$ costante, infatti, derivandola  rispetto il tempo si ottiene
\[
	0=\frac{d(r^2\dot{\theta})}{dt}=2r\dot{r}\dot{\theta}+r^2\ddot{\theta}=r(2\dot{r}\dot{\theta}+r\ddot{\theta}),
\]
ma $r\neq 0$ dalla \cref{foc:traiettoria}.

Ora, confrontando $\dot{\theta}$ nella \cref{foc:momangolare} e nella \cref{foc:traiettoria}	abbiamo
\[
	\frac{c}{r^2}=\frac{d\left(\frac{k}{r^{\alpha}}\right)}{dt}=\frac{-k\alpha}{r^{\alpha+1}}\dot{r}
\]
da cui
\[
	\dot{r}=-\frac{c}{k\alpha}r^{\alpha-1}
\]
e
\begin{equation}
	\label{foc:r2punti}
	\ddot{r}=-\frac{c(\alpha-1)}{k\alpha}r^{\alpha-2}\dot{r}=\frac{c^2(\alpha-1)}{k^2\alpha^2}r^{2\alpha-3}.
\end{equation}

Infine nella \cref{foc:rcomp} sostituiamo $\ddot{r}$ dato dalla \cref{foc:r2punti} e $\dot{\theta}$ dato dalla \cref{foc:momangolare}:

\[
	f(r)=m\left(\frac{c^2(\alpha-1)}{k^2\alpha^2}r^{2\alpha-3}-\frac{c^2}{r^3}\right)
	=\frac{mc^2}{r^3}\left(\frac{\alpha-1}{k^2\alpha^2}r^{2\alpha}-1\right).
\]

\end{document}
