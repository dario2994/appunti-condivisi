\documentclass[../main.tex]{subfiles} 
\begin{document}

\exercise[6/03/2014]{Pendolo a cui sono appesi due pendoli} %pap
\label{ex:pap}

\textex
Ad un pendolo matematico di massa $10m$ e lunghezza $2a$ sono appesi altri due pendoli matematici di massa $3m$ e lunghezza $a$ (che si muovono senza scontrarsi fra loro). Trovare le equazione del moto di tale sistema intorno alla posizione di equilibrio.

\solution
Mi pongo nel sistema cartesiano con asse $\hat x$ orizzontale e asse $\hat y$ rivolto verso l'alto. Chiamo $\theta,\theta_1$ e $\theta_2$ rispettivamente gli angoli che i tre pendoli formano con la verticale.

L'idea della soluzione sarà trovare le due matrici $M,K\in M(3,\mathbb{R})$ tali che
\begin{equation*}
	\frac 12 \dot z^TM\dot z+\frac 12 z^TKz=E
\end{equation*}
dove $z=(\theta,\theta_1,\theta_2)$ ed $E$ è l'energia del sistema, che si conserva. Tale equazione infatti è equivalente a 
\begin{equation*}
	M\ddot z=-Kz
\end{equation*}
che è in realtà un sistema di equazioni lineari omogenee, che so quindi risolvere.

Calcolo perciò l'energia del sistema in funzione dei dati del problema e di $\theta,\theta_1$ e $\theta_2$.
Innanzitutto calcolo le posizioni $(x,y), (x_1,y_1), (x_2,y_2)$ dei tre pendoli:
\begin{equation*}
\begin{cases}
	(x,y)=2a(\sin\theta,-\cos\theta)\\
	(x_1,y_1)=(x,y)+a(\sin\theta_1,-\cos\theta_1)\\
	(x_2,y_2)=(x,y)+a(\sin\theta_2,-\cos\theta_2)
\end{cases}
\end{equation*}
da cui
\begin{equation*}
\begin{cases}
	(\dot x,\dot y)=2a\dot \theta(\cos\theta,\sin\theta)\\
	(\dot x_1,\dot y_1)=(\dot x,\dot y)+a\dot \theta_1(\cos\theta_1,\sin\theta_1)\\
	(\dot x_2,\dot y_2)=(\dot x,\dot y)+a\dot \theta_2(\cos\theta_2,\sin\theta_2)
\end{cases}
\end{equation*}

Da queste ultime equazioni ottengo facilmente che l'energia cinetica del sistema è
\begin{equation*}
	E_{cin}=\frac 12 16m (4a^2\dot\theta^2)+\frac 12 3m a^2(\dot\theta_1^2+\dot\theta_2) +\frac 12 3m [4a^2\dot\theta\dot\theta_1(\cos\theta\cos\theta_1+\sin\theta\sin\theta_1)+4a^2\dot\theta\dot\theta_2(\cos\theta\cos\theta_2+\sin\theta\sin\theta_2)]
\end{equation*}
e sviluppando al second'ordine ottengo quindi
\begin{equation}\label{pap:EnergiaCinetica}
\begin{split}
	E_{cin} & =\frac 12 16m (4a^2\dot\theta^2)+\frac 12 3m a^2(\dot\theta_1^2+\dot\theta_2) +\frac 12 3m (4a^2\dot\theta\dot\theta_1+4a^2\dot\theta\dot\theta_2)\\
	& = 32ma^2\dot\theta^2+\frac 32 ma^2(\dot\theta_1^2+\dot\theta_2^2)+6ma^2\dot\theta(\dot\theta_1+\dot\theta_2)
\end{split}
\end{equation}

L'energia potenziale, sempre sviluppando al second'ordine e poi non considerando le costanti (poichè l'energia è definita a meno di costanti), vale invece
\begin{equation}\label{pap:EnergiaPotenziale}
	E_{pot}=-32mga\cos\theta-3mga(\cos\theta_1+\cos\theta_2)\Longrightarrow E_{pot}=16mga\theta^2+\frac 32 mga(\theta_1^2+\theta_2^2)
\end{equation}

Dall'equazione \cref{pap:EnergiaCinetica} ottengo quindi facilmente
\begin{equation*}
	M=ma^2
	\begin{pmatrix}
		64 & 6 & 6 \\
		6  & 3 & 0 \\
		6  & 0 & 3
	\end{pmatrix}
\end{equation*}
e analogamente dall'equazione \cref{pap:EnergiaPotenziale} ottengo invece
\begin{equation*}
	K=mga
	\begin{pmatrix}
		32 & 0 & 0 \\
		0  & 3 & 0 \\
		0  & 0 & 3
	\end{pmatrix}
\end{equation*}

Mi sono quindi ricondotta a risolvere l'equazione
\begin{equation*}
	M\ddot z=-K z
\end{equation*}

Cerco innanzitutto le soluzioni della forma $z_\alpha(t)=A_\alpha e^{-i\omega_\alpha t}$. In particolare deve valere
\begin{equation*}
	\omega_\alpha^2 M A_\alpha=KA_\alpha \iff (K-\omega_\alpha^2M)A_\alpha=0
\end{equation*}

Perchè esista una soluzione $A_\alpha\neq 0$ devo avere, chiamando $\omega_0^2=\frac ga$, che
\begin{equation*}
	\det (K-\omega_\alpha^2M) =0 \iff \omega_\alpha^2=\omega_0^2,\ 2\omega_0^2,\ \frac 25 \omega_0^2
\end{equation*}








\end{document}