\documentclass[../main.tex]{subfiles} 
\begin{document}

\exercise{Piccole perturbazioni attorno agli assi principali} %pap

\textex
 Studiare le piccole perturbazioni di una rotazione di un corpo rigido attorno ad uno degli assi principali.

\solution
 Siamo nel caso in cui non ci sono forze esterne, quindi per la seconda equzione cardinale si ha
 \begin{equation}
	\label{pap:cardinale}
	\vec{L} = 0,
 \end{equation}
 
 consideriamo quindi, come sistema di riferimento, quello degli assi principali. Per ipotesi abbiamo che
 \begin{equation}
	\vec{\omega} = \omega \hat{x} + \epsilon y(t)\hat y + \epsilon z(t) \hat z
 \end{equation}
 con $\epsilon<< 1$, cioè $\vec{\omega}$ subisce una lieve deviazione rispetto la direzione lungo l'asse principale.
 
 Una considerazione legittima potrebbe essere quella che manca una componente di deviazione lungo $\hat x$, ma questa trascurabile al primo ordine e,
 addirittura, imponendo le equazioni che andiamo a studiare ora, otterremo che questa deviazione risulta costante nel tempo (sempre al primo ordine).
 
 Veniamo ora ad imporre l'equazione sul momento angolare (la seconda equazione cardinale della meccanica), tenendo conto che non ci sono forze 
 esterne e che il polo rispetto cui calcoliamo il momento angolare è fisso abbiamo che $\dot{\vec{L}}=0$, cioè
 (consideriamo solo le componenti in $\hat y$ e $\hat z$)
 
 \begin{align}
	I_y \epsilon\dot y + (I_x-I_z)\omega \epsilon z = 0 \\
	I_z \epsilon\dot z + (I_y-I_x)\omega \epsilon y = 0
 \end{align}
 
 che è un sistema nella forma
 
 \[
	\dot v = A v
 \]
 dove $v=(y,z)$ e $A=\left(\begin{smallmatrix} 0 & a \\ b & 0 \end{smallmatrix} \right)$ e $a = \frac{I_z-I_x}{I_y}\omega$,
 $b=\frac{I_x-I_y}{I_z}\omega$. Per trovare le soluzioni ci basta fare la matrice esponenziale o, equivalentemente, 
 cercare autovalori e farne l'esponenziale.
 In questo caso, gli autovalori sono $\pm \sqrt{ab}$ e quindi le soluzioni sono un'opportuna combinazione lineare di 
 $e^{\sqrt{ab}}$ e di $e^{-\sqrt{ab}}$.
 
 Tuttavia, cercavamo le piccole oscillazioni, quindi avevamo intenzione di studiare solo le soluzioni limitate, ma queste si hanno solo se 
 non ci sono esponenziali di numeri con perte reale positiva. Quindi ciò accade quando $ab$ è nonpositivo, così che otteniamo il classico moto armonico,
 la cui pulsazione risulta essere $\sqrt{|ab|}$. La condizione ora è $ab\leq 0$ cioè
 
 \begin{equation}
	\frac{I_z-I_x}{I_y}\frac{I_x-I_y}{I_z}\omega^2 \leq 0
 \end{equation}

 che è verificata quando $I_x$ non è intermedio a $I_y$ e $I_z$, cioè, quando è il più grande o il più piccolo dei 3 momenti di inerzia.
 In caso contrario il moto non è stabile attorno all'asse di inerzia.
 

\end{document}
