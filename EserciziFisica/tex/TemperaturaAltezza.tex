\documentclass[../main.tex]{subfiles} 
\begin{document}

\exercise[31/03/2014]{Temperatura nella colonna d'aria} %tca

\textex

Assumendo che una colonna d'aria sia in equilibrio termico e che l'aria sia un isolante termico,
determinare l'andamento della temperatura in funzione dell'altezza.

\solution

Consideriamo le seguenti relazioni tra pressione, altezza e temperatura:
\begin{equation}
	\label{tca:stevino}
	\de p = -\rho g \de h
\end{equation}

\begin{equation}
	\label{tca:densita}
	\rho = \frac{M}{V} = \frac{Mp}{nRT}
\end{equation}

dove abbiamo usato la legge dei gas perfetti (\cref{term:perfetti}), inoltre dato che i vari strati di aria sono isolati tra loro, non avviene scambio
di calore, quindi vale

\begin{equation}
	\label{tca:adiabatica}
	pV^{\gamma} = \text{costante} \Rightarrow (1-\gamma)\de p+p\gamma\frac{\de T}{T} = 0
\end{equation}

da cui, confrontando i $\de p$ nelle \cref{tca:stevino,tca:adiabatica} e usando anche la \cref{tca:densita}  otteniamo

\[
	\frac{Mpg}{nRT}\de h = \frac{p\gamma}{(1-\gamma)T}\de T
\]

da cui la relazione lineare:
\begin{equation}
	\label{tca:andamento}
	\Delta T = \frac{1-\gamma}{\gamma}\frac{Mg}{nR}\Delta h \approx -10 \frac {\text{C}^{\circ}}{\text{km}} \Delta h
\end{equation}




\end{document}
