\documentclass[../main.tex]{subfiles} 
\begin{document}

\exercise[06/02/2014]{Carrello che spara acqua} %csa

\textex
Un carrello di massa $m$ contiene inizialmente una quantità $m$ di acqua.
Su un lato del carrello c'è un foro da cui viene fatta uscire acqua inclinata di un angolo $\alpha$ rispetto all'orizzontale, con una velocità $\overrightarrow {v_0}$ rispetto al carrello fissata, con $\frac{\de m}{\de t}=\lambda$.
Si chiede di calcolare il moto del carrello.


\solution
Determino la massa del carrello + acqua contenuta in funzione del tempo:
\begin{equation*}\label{csa:MassaTempo}
m(t) = 2m-\gamma t
\end{equation*}

Chiaramente ciò che mi interessa è la proiezione delle forze lungo l'asse $x$ (l'asse orizzontale lungo quale viene sparata l'acqua). Riscrivo e risolvo l'equazione \cref{MassaVariabile},  considerando che in questo caso ho $\overrightarrow{F_{est}}=0$,  $v_{rel} = -v_0\cos\alpha$ e posso considerare solo quantità scalari (con segno) perchè sono in una dimensione.
L'equazione diventa allora:
\begin{equation*}
m\cdot \de v + \de m \cdot v = \de m\left ( v - v_{rel}\right )
\end{equation*}
Dividendo entrambi i membri per $\de t$ ho che 
\begin{equation*}
\dot m v + m\dot v = \dot m v - \dot m v_{rel}\Rightarrow \dot v = - \frac{\dot m}{m}v_{rel} \Rightarrow 
\frac{\de v}{\de t} = \frac{\gamma}{2m-\gamma t} \Rightarrow \int_{v_{iniz}}^{v} \de v = \int_{0}^{t}\frac{\gamma}{2m-\gamma t}v_{rel}\de t
\end{equation*}
da cui ottengo la velocità in funzione del tempo
\begin{equation*}
v = v_{iniz} -v_{rel}\ln \left ( 1-\frac{\gamma t}{2m}\right )
\end{equation*}

\solution[2]
Sapendo che la massa espulsa per unità di tempo è $\gamma$ e ponendoci nella convenzione in cui $\de m$ è negativo, ricaviamo facilmente che:
\begin{equation*}
	\frac{\de m}{\de t}=-\gamma \Longrightarrow m(t)=2m-\gamma t
\end{equation*}
per ogni $0\le t \le \frac{m}{\gamma}$. Dopo tempo $t=\frac{m}{\gamma}$ il carrello ha espulso tutta l'acqua che aveva al suo interno e si muoverà quindi di moto rettilineo uniforme.

L'unica componente del moto che mi interessa (l'unica che ha il carrello) è quella lungo il piano orizzontale. Mi pongo quindi nel sistema di riferimento che ha asse $x$ coincidente con l'asse orizzontale, origine nel punto da cui parte il carrello e verso uguale al verso del moto del carrello.
In particolare lungo tale asse la risultante delle forze esterne è 0, quindi per l'equazione \cref{MassaVariabile}, vale:
\begin{equation*}
	0=\frac{\de}{\de t} (m\vec v)-\frac{\de m}{\de t}(\vec v+\vec{v}_{rel})
\end{equation*}
che corrisponde alla conservazione della quantità di moto lungo l'asse orizzontale.

Ottengo quindi:
\begin{equation*}
	\frac{\de}{\de t} (m\vec v)=\frac{\de m}{\de t}(\vec v+\vec{v_{rel}}) \Longrightarrow m \frac{\de \vec v}{\de t}=\frac{\de m}{\de t}\vec{v}_{rel}\Longrightarrow m\de \vec v=\vec{v}_{rel}\de m
\end{equation*}
Poichè i vettori considerati sono tutti lungo l'asse $x$ posso considerarne solo l'intensità. Ho quindi $\vec v=v\hat x$, mentre $\vec v_{rel}=-(v_0\cos\alpha)\hat x=-v_r\hat x$, da cui:
\begin{gather*}
	m\de v=-v_r\de m \Longrightarrow \int_0^{v(t)}\de v=-v_r\int_{m(0)}^{m(t)}\frac{\de m}{m}\\
	\Longrightarrow v(t)=-v_r \ln\left( \frac{m(t)}{m(0)} \right)=-v_r\ln\left( 1-\frac{\gamma t}{2m} \right) \\
	\Longrightarrow x(t)=\int_0^{t} v(u) \de u=-v_r \int_0^{t} \ln\left( 1-\frac{\gamma u}{2m} \right) \de u\\
							=c_r \frac{2m}{\gamma} \left[ u (\ln u-1) 	\right]_1^{1-\frac{\gamma t}{2m}}
\end{gather*}
E sostituendo $w= 1-\frac{\gamma u}{2m}$ ottengo infine:
\begin{gather*}
	x(t)=\frac{2m v_r}{\gamma}\int_{0}^{1-\frac{\gamma t}{2m}} \ln w \de w=\frac{2mv_r}{\gamma} \left[ w (\ln w-1) 	\right]_1^{1-\frac{\gamma t}{2m}}
\end{gather*}





\end{document}
