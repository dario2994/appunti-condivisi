\documentclass[../main.tex]{subfiles} 
\begin{document}

\exercise[06/02/2014]{Carrello che spara acqua} %csa

\textex
Un carrello di massa $m$ contiene inizialmente una quantità $m$ di acqua.
Su un lato del carrello c'è un foro da cui viene fatta uscire acqua inclinata di un angolo $\alpha$ rispetto all'orizzontale, con una velocità $\overrightarrow {v_0}$ rispetto al carrello fissata, con $\frac{dm}{dt}=\lambda$.
Si chiede di calcolare il moto del carrello.


\solution
Determino la massa del carrello + acqua contenuta in funzione del tempo:
\begin{equation}\label{csa:MassaTempo}
m(t) = 2m-\gamma t
\end{equation}

Chiaramente ciò che mi interessa è la proiezione delle forze lungo l'asse $x$ (l'asse orizzontale lungo quale viene sparata l'acqua). Riscrivo e risolvo l'equazione \cref{MassaVariabile},  considerando che in questo caso ho $\overrightarrow{F_{est}}=0$,  $v_{rel} = -v_0\cos\alpha$ e posso considerare solo quantità scalari (con segno) perchè sono in una dimensione.
L'equazione diventa allora:
\begin{equation}
m\cdot dv + dm \cdot v = dm\left ( v - v_{rel}\right )
\end{equation}
Dividendo entrambi i membri per $dt$ ho che 
\begin{equation}
\dot m v + m\dot v = \dot m v - \dot m v_{rel}\Rightarrow \dot v = - \frac{\dot m}{m}v_{rel} \Rightarrow 
\frac{dv}{dt} = \frac{\gamma}{2m-\gamma t} \Rightarrow \int_{v_{iniz}}^{v} dv = \int_{0}^{t}\frac{\gamma}{2m-\gamma t}v_{rel}dt
\end{equation}
da cui ottengo la velocità in funzione del tempo
\begin{equation}
v = v_{iniz} -v_{rel}\ln \left ( 1-\frac{\gamma t}{2m}\right )
\end{equation}




\end{document}
