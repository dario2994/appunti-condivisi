\documentclass[../main.tex]{subfiles} 
\begin{document}

\exercise{Orbite che precedono} %ocp

\textex

Fissato un sistema di riferimento nel piano con centro $O$ è data una forza centrale con potenziale
\begin{equation}
  \label{ocp:potenziale}
  U(r)=-\frac{\alpha}{r}+\frac{\varepsilon}{r^2},
\end{equation}
con $\varepsilon$ piccolo e positivo, ossia una piccola perturbazione ad un campo gravitazionale.

Dire come cambia l'orbita ellittica, ossia studiare $r(\theta)$ nel caso di un moto limitato.

\solution

Scriviamo l'equzione dell'energia

\begin{equation}
  \label{ocp:energia}
  E=K+U=\frac{1}{2}m{\dot{\overrightarrow r}^2}-\frac{\alpha}{r}+\frac{\varepsilon}{r^2};
\end{equation}

ora la velocità in coordinate polari si scrive

\[
 \dot{\overrightarrow{r}}=\dot{r}\hat{r}+r\dot{\theta}\hat{\theta}
\]
che porta a
\begin{equation}
 \label{ocp:vel}
 {\dot{\overrightarrow{r}}}^2={\dot{r}}^2+r^2{\dot{\theta}}^2;
\end{equation}

la conservazione del momento angolare dà

\begin{equation}
 \label{ocp:mom}
 L=mr^2\dot{\theta}.
\end{equation}

Inoltre vogliamo esprimere $r(\theta)$, quindi usiamo la regola della derivata di funzioni composte:

\begin{equation}
  \label{ocp:rtheta}
 \dot{r}=\frac{dr}{dt}=\frac{dr}{d\theta}\frac{d\theta}{dt}=\frac{dr}{d\theta}\dot{\theta}.
\end{equation}

Sostituiamo quindi nella conservazione dell'energia \cref{ocp:energia} $\dot{\overrightarrow{r}}$ della \cref{ocp:vel},
$\dot{\theta}$ della \cref{ocp:mom}, $\dot{r}$ della \cref{ocp:rtheta}:

\begin{align*}
 E&=\frac{1}{2}m({\dot{r}}^2+r^2{\dot{\theta}}^2)-\frac{\alpha}{r}+\frac{\varepsilon}{r^2} \\
 &=\frac{1}{2}m\left(\frac{dr}{d\theta}\dot{\theta}\right)^2+\frac{1}{2}mr^2\left(\frac{L}{mr^2}\right)^2-\frac{\alpha}{r}+\frac{\varepsilon}{r^2} \\
 &=\frac{1}{2}m\left(\frac{dr}{d\theta}\right)^2\left(\frac{L}{mr^2}\right)^2+\frac{L^2}{2mr^2}-\frac{\alpha}{r}+\frac{\varepsilon}{r^2} \\
 \frac{2mE}{L^2}&=\left(\frac{dr}{d\theta}\right)^2\frac{1}{r^4}+\frac{1}{r^2}\left(1+\frac{2m\varepsilon}{L^2}\right)-\frac{2m\alpha}{L^2r}
\end{align*}

Sostituiamo ora $u=\frac{1}{r}$, si ha $\frac{du}{dr}=-\frac{1}{r^2}$ e

\[
 \frac{2mE}{L^2}=\left(\frac{du}{d\theta}\right)^2+u^2\left(1+\frac{2m\varepsilon}{L^2}\right)-u\frac{2m\alpha}{L^2}
\]

ora deriviamo rispetto a $\theta$

\begin{align*}
 0&=2\frac{du}{d\theta}\frac{d^2u}{d\theta^2}+2u\frac{du}{d\theta}\left(1+\frac{2m\varepsilon}{L^2}\right)-\frac{du}{d\theta}\frac{2m\alpha}{L^2}\\
 0&=\frac{du}{d\theta}\left(\frac{d^2u}{d\theta^2}+u\left(1+\frac{2m\varepsilon}{L^2}\right)-\frac{m\alpha}{L^2}\right).
\end{align*}

Quindi abbiamo 2 possibilità:

\[
 \frac{du}{d\theta}=0 \Rightarrow u = C \Rightarrow r = C^{-1}
\]

cioè una circonferenza (raggio costante attorno a $O$), oppure
\begin{equation}
  \label{ocp:diff}
 \frac{d^2u}{d\theta^2}+u\left(1+\frac{2m\varepsilon}{L^2}\right)-\frac{m\alpha}{L^2}=0
\end{equation}
la quale, posto $\omega^2=\left(1+\frac{2m\varepsilon}{L^2}\right)$, ha una soluzione della forma

\[
 u = A\cos(\omega\theta+\phi) + C
\]

dove $A$ e $\phi$ sono arbitrarie (che però dipendono da $E$ e da $L$) e $C$ è da determinare in modo che soddisfi la \cref{ocp:diff}:

\begin{align}
  &-A\omega^2\cos(\omega\theta+\phi)+(A\cos(\omega\theta+\phi) + C)\omega^2-\frac{m\alpha}{L^2}=0 \\
  &C\omega^2-\frac{m\alpha}{L^2}=0 \\
  &C=\frac{m\alpha}{L^2\omega^2}
\end{align}

Risostituendo $r=\frac{1}{u}$, la traiettoria ottenuta risulta:

\[
 r=\frac{1}{A\cos(\omega\theta+\phi) + \frac{m\alpha}{L^2\omega^2}}
\]

Sappiamo che in assenza di perturbazione ($\varepsilon=0$, ossia $\omega=1$), se la traiettoria è limitata allora è un ellisse.
In questo caso, come si vede, la traiettoria rimane molto simile: l'unica rilevante differenza è la presenza di $\omega$ nell'argomento del coseno.
Poiché $\omega>1$ risulta che il periodo $T=\frac{2\pi}{\omega}$ è più piccolo, cioè si osserva la precessione dell'orbita ellittica.

\end{document}
