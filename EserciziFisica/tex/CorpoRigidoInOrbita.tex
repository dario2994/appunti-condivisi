\documentclass[../main.tex]{subfiles} 
\begin{document}

\exercise[Lezione 17/02/2014]{Corpo rigido in orbita circolare intorno alla Terra} %crc
\label{ex:crc}

\textex
Un satellita ruota intorno alla Terra su un'orbita circolare di raggio $r_Q$. Calcolare il momento delle forze rispetto al baricentro, noto $I$ il tensore d'inerzia del satellite. 

Inizialmente il satellite ha i tre assi principali rivolti uno radialmente, uno tangenzialmente e l'ultimo uscente dal piano dell'orbita. Dimostrare che l'orientamento degli assi rimane fisso nel sistema che ruota solidalmente al satellite. Calcolare inoltre il periodo delle piccole oscillazioni sul piano dell'orbita.

\solution
Nel corso della soluzione utilizzerò la notazione di Einstein come definita in \cref{ten}.

Calcolo innanzitutto il momento delle forze rispetto al baricentro, dove considero come sistema fisso il sistema centrato nel centro della Terra.
\begin{equation}\label{crc:MomentoForze1}
\begin{split}
	\vec N_Q	&= \sum_\alpha\vec r_{Q\alpha} \times \vec F_\alpha=- \sum_\alpha \vec r_{Q\alpha} \times \frac {G_N M m_\alpha }{r_\alpha^3}\vec r_\alpha \\
			&= -G_NM \sum_\alpha \frac {m_\alpha}{r_\alpha^3}\vec r_{Q\alpha} \times (\vec r_Q+\vec r_{Q\alpha})=-G_NM \sum_\alpha \frac {m_\alpha}{r_\alpha^3}(\vec r_{Q\alpha} \times \vec r_Q)
\end{split}
\end{equation}

Sapendo che $\vec r_\alpha=\vec r_Q+\vec r_{Q\alpha}$, ricavo per il teorema di Carnot e utilizzando che $r_{Q\alpha}\ll r_Q$:
\begin{equation*}
	r_\alpha^2=r_Q^2+r_{Q\alpha}^2+2\vec r_Q\cdot \vec r_{Q\alpha}\simeq r_Q^2 \left( 1+2\frac{\vec r_Q\cdot \vec r_{Q\alpha}}{r_Q^2} \right)
	\Longrightarrow  r_\alpha \simeq r_Q \left( 1+\frac{\vec r_Q\cdot \vec r_{Q\alpha}}{r_Q^2} \right) \Longrightarrow  \frac{1}{r_\alpha^3} \simeq \frac{1}{r_Q^3} \left( 1-3\frac{\vec r_Q\cdot \vec r_{Q\alpha}}{r_Q^2} \right)
\end{equation*}
Sostituendo ora quest'ultima relazione nella \cref{crc:MomentoForze1} e utilizzando che $\sum_\alpha m_\alpha(\vec r_{Q\alpha} \times \vec r_Q)=0$, ottengo:
\begin{gather*}
	\vec N_Q = -\frac{G_NM}{r_Q^3} \sum_\alpha m_\alpha\left( 1-3\frac{\vec r_Q\cdot \vec r_{Q\alpha}}{r_Q^2} \right) (\vec r_{Q\alpha} \times \vec r_Q) =  \frac{3G_N M}{r_Q^5} \sum_\alpha m_\alpha(\vec r_Q\cdot \vec r_{Q\alpha})( \vec r_{Q\alpha}\times \vec r_Q)\\
	\Longrightarrow N_Q^i =\frac{3G_N M}{r_Q^5} \sum_\alpha m_\alpha\varepsilon_{ijk}r_{Q\alpha}^jr_Q^k r_Q^lr_{Q\alpha}^l=\frac{3G_N M}{r_Q^5} \varepsilon_{ijk} r_Q^k r_Q^l\sum_\alpha m_\alpha (r_{Q\alpha}^jr_{Q\alpha}^l-\delta_{jl} r_{Q\alpha}^2)
\end{gather*}
dove $i,j,k,l$ all'apice indicano le componenti dei vettori su cui sono applicati, relative al sistema di assi fissi rispetto al corpo, e l'ultimo passaggio è vero perchè $\varepsilon_{ijk} r_Q^k r_Q^l \delta_{jl}=\varepsilon_{ijk} r_Q^k r_Q^j=\vec r_Q \times \vec r_Q =0$.

Quindi ricordando la definizione di tensore d'inerzia data dalla \cref{ten:definizione}, ottengo:
\begin{equation}\label{crc:MomentoForze}
	N_Q^i=\frac{3G_N M}{r_Q^5} \varepsilon_{ijk} I_{jl} r_Q^k r_Q^l
\end{equation}
che ponendosi nel sistema degli assi principali diventa:
\begin{equation}\label{crc:MomentoForzeAssiPrincipali}
	N_Q^i=\frac{3G_N M}{r_Q^5} \varepsilon_{ijk} I_{j} r_Q^k r_Q^j
\end{equation}

Mi pongo ora nel caso in cui inizialmente gli assi principali $\hat x, \hat y, \hat z$ siano rivolti rispettivamente tangenzialmente, radialmente e in direzione uscente dal piano dell'orbita (come nel testo).
In questo caso ho che $\vec r_Q=(0,r_Q,0)$ e sostituendo tale valore di $\vec r_Q$ nella \cref{crc:MomentoForzeAssiPrincipali}, ho facilmente che $\vec N=0$. Ciò dimostra quindi che l'orientamento degli assi rimane fisso rispetto al sistema di riferimento che si muove solidalmente al satellite.

Voglio calcolare ora il periodo delle piccole oscillazioni che si compiono sul piano dell'orbita (l'asse $\hat z$ rimane quindi sempre fisso rispetto al sistema che si muove con il satellite).

Sia $\theta$ l'angolo che l'asse $\hat y$ forma con la direzione radiale. In questo caso, considerando $\theta$ ``piccolo'', la direzione del baricentro rispetto agli assi principali è espressa da $\vec r_Q=r_Q(\sin\theta,\cos\theta,0)\simeq r_Q(\theta,1,0)$.

Sostituendo quindi nella \cref{crc:MomentoForzeAssiPrincipali}, ho che vale:
\begin{equation}\label{crc:SistemaMomentoForze}
\begin{cases}
	N_Q^x=N_Q^y=0\\
	N_Q^z=\frac{3G_N M}{r_Q^5} (\varepsilon_{zxy} I_{x}r_Q^2\theta +\varepsilon_{zyx} I_{y}r_Q^2\theta)=\frac{3G_N M}{r_Q^3}(I_x-I_y)\theta
\end{cases}
\end{equation}

Poichè il centro del sistema di riferimento è il baricentro, è rispettata la condizione \cref{ten:CondizioneSulCentro}, e quindi per la \cref{ten:MomentoAngolare} vale:
\begin{equation}\label{crc:MomentoAngolareZ}
	L_Q^z=I_z\omega_z=I_z\dot\theta
\end{equation}

Per la seconda equazione cardinale dei sistemi (sempre perchè il centro del sistema è il baricentro) vale:
\begin{equation*}
	\vec N_Q=\frac{\de \vec L_Q}{\de t} \Longrightarrow N_Q^z=\frac{\de L_Q^z}{\de t}+(\vec \omega \times \vec L_Q)_z
\end{equation*}
ma $\vec \omega$ è diretto lungo l'asse $\hat z$, quindi la componente $z$ di $\vec \omega \times \vec L_Q$ è sicuramente 0. Perciò sostituendo ad $L_Q^z$ la \cref{crc:MomentoAngolareZ} ottengo:
\begin{equation*}
	N_Q^z=\frac{\de L_Q^z}{\de t}=\frac{\de I_z\dot\theta}{\de t}=I_z\ddot\theta
\end{equation*}
Ponendo infine in quest'ultima equazione la \cref{crc:SistemaMomentoForze} per $N_Q^z$, ho che:
\begin{equation*}
	I_z\ddot\theta=\frac{3G_N M}{r_Q^3}(I_x-I_y)\theta \Longrightarrow \ddot\theta=\frac{3G_N M}{r_Q^3}\cdot\frac{I_x-I_y}{I_z}\theta
\end{equation*}

Da quest'ultima equazione ricaviamo che l'equilibrio è stabile se $I_x-I_y<0$ e in tal caso la pulsazione delle piccole oscillazioni è
\begin{equation*}
	\omega=\sqrt{\frac{3G_N M}{r_Q^3}\cdot\frac{I_y-I_x}{I_z}}
\end{equation*}






\end{document}