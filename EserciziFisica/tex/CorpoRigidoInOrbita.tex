\documentclass[../main.tex]{subfiles} 
\begin{document}

\exercise[Lezione 13/02/2014]{Corpo rigido in orbita circolare intorno alla Terra} %crc
\label{ex:crc}

\textex
Un satellita ruota intorno alla Terra su un'orbita circolare di raggio $r_Q$. Calcolare il momento delle forze rispetto al baricentro, noto $I$ il tensore d'inerzia del satellite. 

Inizialmente il satellite ha i tre assi principali rivolti uno radialmente, uno tangenzialmente e l'ultimo uscente dal piano dell'orbita. Dimostrare che l'orientamento degli assi rimane fisso nel sistema che ruota solidalmente al satellite. Calcolare inoltre il periodo delle piccole oscillazioni sul piano dell'orbita.

\solution
Calcolo innanzitutto il momento dele forze rispetto al baricentro, dove considero come sistema fisso il sistema centrato nel centro della Terra.
\begin{equation}\label{crc:MomentoForze1}
\begin{split}
	\vec N_Q	& =\sum_\alpha m_\alpha \vec r_{Q\alpha} \times \vec F_\alpha=-\sum_\alpha m_\alpha \vec r_{Q\alpha} \times \frac {G_N M}{r_\alpha^3}\vec r_\alpha \\
				&= -G_NM\sum_\alpha \frac {m_\alpha}{r_\alpha^3} \vec r_{Q\alpha} \times (\vec r_Q+\vec r_{Q\alpha})=-G_NM\sum_\alpha \frac {m_\alpha}{r_\alpha^3} \vec r_{Q\alpha} \times \vec r_Q
\end{split}
\end{equation}

Sapendo che $\vec r_\alpha=\vec r_Q+\vec r_{Q\alpha}$, ricavo per il teorema di Carnot e utilizzando che $r_{Q\alpha}\ll r_Q$:
\begin{equation*}
	r_\alpha^2=r_Q^2+r_{Q\alpha}^2+2\vec r_Q\cdot \vec r_{Q\alpha}\simeq r_Q^2 \left( 1+2\frac{\vec r_Q\cdot \vec r_{Q\alpha}}{r_Q^2} \right)
	\Longrightarrow  r_\alpha \simeq r_Q \left( 1+\frac{\vec r_Q\cdot \vec r_{Q\alpha}}{r_Q^2} \right) \Longrightarrow  \frac{1}{r_\alpha^3} \simeq \frac{1}{r_Q^3} \left( 1-3\frac{\vec r_Q\cdot \vec r_{Q\alpha}}{r_Q^2} \right)
\end{equation*}
Sostituendo ora quest'ultima relazione nella \cref{crc:MomentoForze1}, ottengo:
\begin{equation}
\begin{split}
	\vec N_Q & = -\frac{G_NM}{r_Q^3}\sum_\alpha m_\alpha   \left( 1-3\frac{\vec r_Q\cdot \vec r_{Q\alpha}}{r_Q^2} \right) \vec r_{Q\alpha} \times \vec r_Q = \frac{G_NM}{r_Q^3}\sum_\alpha 3 m_\alpha   \frac{\vec r_Q\cdot \vec r_{Q\alpha}}{r_Q^2} \vec r_{Q\alpha}\times \vec r_Q\\
		&= \frac{3G_NM}{r_Q^5}\sum_\alpha m_\alpha   (\vec r_Q\cdot \vec r_{Q\alpha})( \vec r_{Q\alpha}\times \vec r_Q)
\end{split}
\end{equation}




\end{document}