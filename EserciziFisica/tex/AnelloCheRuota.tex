\documentclass[../main.tex]{subfiles} 
\begin{document}

\exercise{Anello che ruota} %acr
\textex
Un anello ruota attorno ad un suo diametro (parallelo alla direzione della forza di gravità) ad una velocità
angolare $\omega$ costante nel tempo. Su di esso vi è un punto materiale di massa $m$, vincolato a muoversi lungo l'anello e su cui agisce
la forza di gravità.

Determinare le posizioni di equilibrio stabile ed instabile in funzione di $\omega$ del punto materiale e studiare la
frequenza delle piccoli oscillazioni attorno ai punti di equilibrio stabile.
\solution

\solution[2]

\end{document}