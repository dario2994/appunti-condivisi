\documentclass[../main.tex]{subfiles} 
\begin{document}
\section{Formulario}
\setcounter{equation}{0}
\renewcommand{\theequation}{F.\arabic{equation}}
\subsection{Coordinate polari} 
In coordinate polari si ha che per ogni versore $\hat{\iota}$ vale $\dot{\hat{\iota}}=\vec{\omega}\times\hat{\iota}$,
dove
\begin{equation}\label{OmegaPolari}
	\vec{\omega}=\dot{\theta}\hat{z}
\end{equation}
Derivando l'equazione $\vec{r}=r\hat{r}$ rispetto al tempo, ottengo quindi
\begin{equation}\label{VelCooPolari}
	\vec{v}=\dot{\vec{r}}=\dot{r}\hat{r}+r\dot{\theta}\hat{\theta}
\end{equation}
Derivando una seconda volta ottengo invece:
\begin{equation}\label{AccCooPolari}
	\vec{a}=\dot{\vec{v}} =(\ddot{r}-r\dot{\theta}^2)\hat{r}+(2\dot{r}\dot{\theta}+r\ddot{\theta})\hat{\theta}
\end{equation}

\subsection{Coordinate sferiche}
Ponendosi in un sistema di coordinate sferiche, vale $\dot{\hat{\iota}}=\vec{\omega}\times\hat{\iota}$
per ogni versore $\hat{\iota}$, dove
\begin{equation}\label{OmegaSferiche}
	\vec{\omega}=\dot{\theta}\hat{\varphi}+\dot{\varphi}\hat{z}
\end{equation}
Quindi derivando l'equazione $\vec{r}=\dot{r}\hat{r}$, ottengo
\begin{equation}\label{VelCooSferiche}
	\vec{v} =\dot{\vec{r}}=\dot{r}\hat{r}+r\dot{\theta}\hat{\theta}+r\hat{\varphi}\sin{\theta}\hat{\varphi}
\end{equation}
E derivando nuovamente rispetto al tempo ho che:
\begin{equation}\label{AccCooSferiche}
\begin{split}
	\vec{a}=	\dot{\vec{v}}=	&\left(\ddot{r}-r\dot{\theta}^2-r\dot{\varphi}^2\sin^2\theta \right)\hat{r}+\\
													&	+\left( r\ddot{\theta}+2\dot{r}\dot{\theta}-r\dot{\varphi}^2\sin\theta\cos\theta \right)\hat{\theta}+\\
													&	+\left( 2\dot{r}\dot{\varphi}\sin\theta+2r\dot{\varphi}\dot{\theta}\cos\theta+r\ddot{\varphi}\sin\theta \right)\hat{\varphi}
\end{split}
\end{equation}

\subsection{Moto in campo centrale}
Consideriamo un campo di forze centrali descritto da $\vec{F}=f(r)\hat{r}$. 
Innanzitutto abbiamo che tale campo è un campo conservativo, in quanto ammette un potenziale della forma:
\begin{equation*}
	U(r)=-\int_r^\infty{f(r) dr}
\end{equation*}
Inoltre in un campo di forze centrali si conserva il momento angolare, poichè il momento delle forze è uguale a 0,
e in particolare vale, grazie alle \cref{VelCooPolari}:
\begin{equation*}
	\vec{L}=m\vec{r}\times\vec{v}=mr^2\dot{\theta}\hat{z}
\end{equation*}

Il caso di un corpo in un campo di forze centrale si può sempre ricondurre ad un modo unidimensionale in $r$, secondo la formula:
\begin{equation}\label{CentraleUnidimensionale}
	E=\frac 12m\dot{r}^2+U_{eff}(r)
\end{equation}
dove il potenziale efficace $U_{eff}$ è definito come:
\begin{equation}\label{PotenzialeEfficace}
	U_{eff}(r)=U(r)+\frac{L^2}{2mr^2}
\end{equation}

Mi concentro in particolare sul campo gravitazionale generato da un corpo di massa $M$, che è proprio un campo 
centrale caratterizzato dalla forza:
\begin{equation*}
	\vec{F}=-\frac{GMm}{r^2}\hat{r}
\end{equation*}
e che genera quindi il potenziale:
\begin{equation*}
	U(r)=-\frac{GMm}{r}
\end{equation*}

Si dimostra che l'orbita che descrive un corpo in tale campo centrale è ellittica se l'energia di tale corpo è
minore di 0, parabolica se la sua energia è uguale a 0 e iperbolica se invece ha energia maggiore di 0.
In particolare se l'orbita è ellittica di semiasse maggiore $a$, il corpo ha energia:
\begin{equation} \label{EnergiaTotaleOrbita}
	E=-\frac{GMm}{2a}
\end{equation}
e il periodo dell'orbita è:
\begin{equation} \label{PeriodoOrbita}
	T=\frac{2\pi}{\sqrt{GM}}a^{\frac{3}{2}}
\end{equation}

Vale inoltre il teorema del viriale, che afferma che in un moto limitato si ha:
\begin{equation} \label{TeoremaDelViriale}
	<K>=-\frac{1}{2}<U>
\end{equation}
dove $<K>$ è l'energia cinetica media del corpo in moto e $<U>$ è la sua energia potenziale media.

Si trova che la velocità a cui si muove un corpo in orbita circolare, a distanza $r$ dal centro, è:
\begin{equation} \label{VelocitaMotoCircolare}
	v^2=\frac{GM}{r}
\end{equation}

mentre invece la velocità di fuga (ottenibile usando \cref{VelocitaMotoCircolare,TeoremaDelViriale}) è (sempre a distanza $r$):
\begin{equation} \label{VelocitaDiFuga}
	v^2=\frac{2GM}{r}
\end{equation}




\subsection{Sistemi di riferimento non inerziali}
Sia $K$ un sistema di riferimento non inerziale di centro $O'$, rispetto ad un sistema di riferimento
inerziale $K^I$ di centro $O$. Poichè sabbiamo che l'equazione di Newton $\vec{F}=m\vec{a}$
vale solo in sistemi di riferimento inerziali, vogliamo scoprire cosa si può dire del rapporto fra la forza
che agisce su un corpo e la sua accelerazione rispetto al sistema $K$.

Calcoliamo innanzitutto la relazione fra le accelerazioni nei due sistemi di riferimento. Poichè vale la
relazione $\overrightarrow{OP}=\overrightarrow{OO'}+\overrightarrow{O'P}$ per un qualsiasi punto $P$, derivando 
rispetto al tempo e chiamando $\overrightarrow{O'P}=\vec{r}$, ottengo:
\begin{equation}\label{VelNonInerziale}
\begin{split}
	\vec{v}=\dot{\overrightarrow{OP}}	& =\dot{\overrightarrow{OO'}}+\dot{\vec{r}}\\
													& =\overrightarrow{v_{tr}}+\frac{\delta \vec r}{\delta t}+\vec{\omega}\times\vec{r}\\
													& =\overrightarrow{v_{tr}}+\vec{v_r}+\vec{\omega}\times\vec{r}
\end{split}
\end{equation}
dove $\overrightarrow{v_{tr}}$ (velocità di trascinamento) è la velocità di $O'$ rispetto ad $O$ e $\overrightarrow{v_r}$ è la velocità relativa
a $K$ di $P$.

Derivando nuovamente ottengo invece:
\begin{equation}\label{AccNonInerziale}
\begin{split}
	\vec{a}=\vec{v}	& =\dot{\overrightarrow{v_{tr}}}+\dot{\vec{v_r}}+\left(\dot{\vec{\omega}}\times\vec{r}+\vec{\omega}\times\dot{\vec{r}}\right)\\
											& =\overrightarrow{a_{tr}}+\left(\vec{a_r}+\vec{\omega}\times\vec{v_r}\right)+\dot{\vec{\omega}}\times\vec{r}+\left[\vec{\omega}\times\vec{v_r}+\vec{\omega}\times(\vec{\omega}\times\vec{r})\right]\\
											& =\overrightarrow{a_{tr}}+\vec{a_r}+2\vec{\omega}\times\vec{v_r}+\dot{\vec{\omega}}\times\vec{r}+\vec{\omega}\times(\vec{\omega}\times\vec{r})\\
\end{split}
\end{equation}
dove analogamente a prima $\overrightarrow{a_{tr}}$ (accelerazione di trascinamento) è l'accelerazione di $O'$ rispetto a $O$ e $\overrightarrow{a_r}$ è l'accelerazione relativa a $K$ di $P$.

Quindi dall'equazione di Newton ottengo:
\begin{equation*}
\begin{split}
	\vec{F}=m\vec{a}	& =m\left[\overrightarrow{a_{tr}}+\vec{a_r}+2\vec{\omega}\times\vec{v_r}+\dot{\vec{\omega}}\times\vec{r}+\vec{\omega}\times(\vec{\omega}\times\vec{r})\right]\\
											& =m\vec{a_r}+\underline{m\overrightarrow{a_{tr}}}+\underline{2m\vec{\omega}\times\vec{v_r}}+\underline{m\dot{\vec{\omega}}\times\vec{r}}+\underline{m\vec{\omega}\times(\vec{\omega}\times\vec{r})}
\end{split}
\end{equation*}
I termini sottolineati sono forze apparenti; in particolare $2m\vec{\omega}\times\vec{v_r}$
è chiamata forza di Coriolis, mentre $m\vec{\omega}\times(\vec{\omega}\times\vec{r})$
è la forza centrifuga.

In un sistema non inerziale si può quindi riscrivere l'equazione di Newton come:
\begin{equation}\label{ForzaNonInerziale}
	m\vec{a_r}=\vec{F_r}=\vec{F}-m\overrightarrow{a_{tr}}-m\left[2\vec{\omega}\times\vec{v_r}+\dot{\vec{\omega}}\times\vec{r}+\vec{\omega}\times(\vec{\omega}\times\vec{r})\right]
\end{equation}

\subsection{Problema dei 2 corpi}
Sono dati 2 corpi di masse $m_1,m_2$ che interagiscono attraverso una forza che dipende 
solo dalla distanza $\vec F(\vec r)$.
Detta $\mu=\frac{m_1m_2}{m_1+m_2}$ la massa ridotta, vale l'equazione:
\begin{equation}\label{ForzaMassaRidotta}
	\vec F(\vec r)=\mu \ddot {\vec{r}}
\end{equation}
Detta $\overrightarrow{ r_Q }$ la posizione del centro di massa del sistema, vale la seguente espressione per l'energia cinetica:
\begin{equation}\label{Cinetica2Corpi}
	K=\frac12(m_1+m_2)\dot{\overrightarrow{r_Q}}^2+\frac12\mu\dot{\vec{r}}^2
\end{equation}
e questa è la formula per il momento angolare:
\begin{equation}\label{Momento2Corpi}
	\vec L=(m_1+m_2)\overrightarrow{r_Q}\times \dot{\overrightarrow{r_Q}}+\mu \vec r\times\dot{\vec r}
\end{equation}

Quindi nel complesso un sistema con 2 corpi si può disaccoppiare completamente nel moto del centro di massa e nel
moto relativo tra le 2 masse.

\subsection{Urti}
I problemi di urto in generale riguardano due particelle di massa $m_1$ e $m_2$ e velocità iniziali $\overrightarrow {v_1}$ e $\overrightarrow {v_2}$ e sono caratterizzati da:
\begin{itemize}
 \item interazione di corto raggio
 \item terzo principio di Newton $\rightarrow$ la quantità di moto totale si conserva
 \item conservazione dell'energia meccanica (oppure supponiamo di conoscere la differenza $\Delta E$ di energia)
\end{itemize}

Vogliamo trovare $\overrightarrow{{v_1}'}$ e $\overrightarrow{{v_2}'}$, le velocità dopo l'urto.
Scomponiamo il problema nel moto del centro di massa e nel moto relativo: poniamo $\overrightarrow v = \overrightarrow {v_2} - \overrightarrow {v_1}$,
$M = m_1+m_2$, $\mu = \frac{m_1m_2}{M}$, $\overrightarrow {V_Q} = \frac{m_1\overrightarrow {v_1}+m_2\overrightarrow{v_2}}{M}$,
$\overrightarrow P = M\overrightarrow {V_Q}$, $\overrightarrow p = \mu \overrightarrow v$ e abbiamo che
\begin{equation}
 E = \frac{P^2}{2M} + \frac{p^2}{2\mu}
\end{equation}
Poichè $\Delta E = 0$ e $\overrightarrow P$ è costante, anche $p$ è costante, e l'unica cosa che può cambiare dopo l'urto è la sua direzione.
Dunque per risolvere completamente il problema è necessario anche conoscere l'angolo $\theta$ formato da $\overrightarrow p$ con $\overrightarrow {p'}$.
Sia $\hat{v_f}$ il versore diretto lungo $\overrightarrow {p'}$. Le velocità della massa $m_1$ prima e dopo l'urto, calcolate rispetto al centro di massa, sono
\begin{equation}
 \overrightarrow{v_{1Q}} = -\frac{m_2}{M}v\hat v \qquad \qquad
 \overrightarrow{{v_{1Q}}'} = -\frac{m_2}{M}v\hat{v_f}
\end{equation}
Nel sistema di riferimento del laboratorio, dunque, le velocità delle particelle dopo l'urto saranno
\begin{equation}\label{UrtiElastici}
 \overrightarrow {{v_1}'} = -\frac{m_2}{M}v\hat{v_f}+\overrightarrow {V_Q} \qquad \qquad
 \overrightarrow {{v_2}'} = \frac{m_1}{M}v\hat{v_f}+\overrightarrow {V_Q}
\end{equation}
Osservo infine che, se $|\overrightarrow {V_Q}| > |\overrightarrow{v_{1Q}}|$ ci sono dei limiti sull'angolo di deflessione della prima particella;
in particolare il massimo angolo di deflessione è $\theta_{max} = \arctan \frac{v_{1Q}}{V_Q}$.





\subsection{Problemi con massa variabile}
Si tratta di problemi in cui la massa cambia nel tempo, ad esempio un razzo che espelle carburante oppure una goccia di pioggia che cade nel vapore e man mano aumenta la sua dimensione.
Per risolverli, considero che al tempo $t$ avrò:
massa = $m$, velocità = $\vec v$
e al tempo $t+\de t$ avrò una massa $m+\de m$ che si muove a velocità $\vec v + \de \vec v$ e una massa $-\de m$ che si muove a velocità $\vec v + \vec v_{rel}$, dove $\vec v_{rel}$ è la velocità della massa espulsa/acquisita rispetto al corpo che sto considerando.
Poichè so che $\de \vec P = \vec F_{est} \de t$, ho che
$$ \left ( m + \de m \right ) \cdot \left ( \vec v + \de \vec v\right ) -\de m\left ( \vec v + \vec v_{rel} \right ) - m\vec v = \vec F_{est}\de t$$
da cui ottengo l'equazione generale per risolvere un problema con massa variabile:
\begin{equation}\label{MassaVariabile}
	m \de \vec v + \de m \vec v = \de m \left ( \vec v + \vec v_{rel} \right ) + \vec F_{est} \de t
\end{equation}
o, equivalentemente,
\begin{equation}
	\frac{\de }{\de t} m\vec v = \frac{\de m}{\de t}\left ( \vec v + \vec v_{rel} \right ) + \vec F_{est}
\end{equation}



\subsection{Tensore d'inerzia} \label{ten}
Userò la notazione di Einstein per le sommatorie contratte.
Sia $\vec O$ un punto fisso rispetto al corpo rigido e $\hat x,\hat y,\hat z$ degli assi centrati in $\vec O$ fissi rispetto al corpo rigido.
Inoltre $\alpha$ sarà un indice che si muoverà sui punti del corpo rigido e $m_\alpha,r_\alpha$ sono massa e posizione rispetto al punto O del punto $\alpha$. Invece gli indici $i,j$ varieranno su $x,y,z$.

Il tensore d'inerzia di un corpo rigido è un tensore a due indici definito come segue:
\begin{equation} \label{ten:definizione}
	I_{ij}=\sum_\alpha m_\alpha(\delta_{ij}\vec r_{\alpha}^2 -r_{\alpha i}r_{\alpha j})
\end{equation}

Fissato ora un riferimento inerziale, sia $\vec v_O$ la velocità di $\vec O$ in tale sistema. 
Trovo la relazione fra il momento angolare calcolato rispetto ad $\vec O$ nel sistema di riferimento inerziale e il tensore d'inerzia di un corpo sotto l'ipotesi che valga
\begin{equation} \label{ten:CondizioneSulCentro}
	\vec r_Q\times \vec v_O=0
\end{equation}
dove $\vec r_Q$ è il vettore che congiunge $\vec O,\vec Q$ dove $\vec Q$ è il baricentro del corpo.

Utilizzando la definizione di momento angolare e la \cref{ten:CondizioneSulCentro}, ottengo:
\begin{equation*}
	\begin{split}
	\vec L	& = m_\alpha \vec r_{\alpha}\times \left(\dot{\vec r}_{\alpha}+\vec v_O\right)=
	m_\alpha \vec r_\alpha\times\dot{\vec r}_{\alpha}+ m_\alpha\vec r_\alpha\times\vec v_O =
	m_\alpha \vec r_\alpha\times\left(\vec\omega\times\vec r_\alpha\right)+m\vec r_Q\times \vec v_O =
	m_\alpha\left(\vec\omega\vec r^2_\alpha-\vec r_\alpha\left(\vec\omega\cdot\vec r_\alpha\right)\right)
	\end{split}
\end{equation*}
dove ho usato ho usato una nota identità coi prodotti vettoriali nell'ultima identità.
Ora considero la componente $i$-esima dell'identità vettoriale trovata e ottengo:
\begin{equation*}
	L_i=m_\alpha\left(\delta_{ij}\omega_j\vec r^2_\alpha-r_{\alpha i}\omega_j r_{\alpha j}\right)=
	\omega_j m_\alpha\left(\delta_{ij}\vec r^2_\alpha-r_{\alpha i} r_{\alpha j}\right)
\end{equation*}
da cui, sfruttando la \cref{ten:definizione}, ottengo proprio la relazione cercata:
\begin{equation}\label{ten:MomentoAngolare}
	L_i=I_{ij}\omega_j
\end{equation}

Ora pongo $O$ nel centro di massa del corpo rigido. Ne ricavo che vale anche che l'energia cinetica rispetto al centro di massa si può calcolare come:
\begin{equation} \label{ten:EnergiaCinetica}
	K=\frac 12 I_{ij}\omega_i\omega_j
\end{equation}

Infine è importante notare che, per il teorema spettrale, esistono 3 versori ortogonali, detti assi principali del corpo, tali che calcolando $I$ rispetto a loro ottengo che $I_{ij}=0$ se $i\not=j$.

\subsection{Oscillazioni a più gradi di libertà}\label{opgl}
Dato un sistema a più gradi di libertà considero il vettore di coordinate generalizzate:
\begin{equation*}
	\vec x=(x_1,\dots,x_n)
\end{equation*}
e assumo di riuscire a scrivere l'energia come:
\begin{equation} \label{opgl:EnergiaBrutta}
	E=T\left(\dot{\vec x}\right)+U\left(\vec x\right)
\end{equation}
dove $T,U$ sono ovviamente rispettivamente energia cinetica e potenziale.

Sia $\vec x_0$ un punto di minimo del potenziale.

Per ipotesi di regolarità (tutte da decidere...) si riesce a ricavare che, considerando per comodità $\vec y=\vec x-\vec x_0$, sia $T$ che $U$ sono (in approssimazione al second'ordine centrato in $\vec x_0$) delle forme quadratiche. 
Allora, semplicemente scrivendo quanto appena detto in forma matriciale, esistono delle matrici, che chiamerò $K,M$ per analogia col caso unidimensionale, tali che valgono:
\begin{align*}
	T &= \frac 12\dot{\vec{y}}^T M \dot{\vec{y}} \\
	U &= \frac 12\vec{y}^T K \vec y
\end{align*}
dove inoltre scelgo $K,M$ simmetriche (posso sceglierle simmetriche). Inoltre ancora per strani motivi $K,M$ risultano rispettivamente semidefinita e definita positiva.

Sostituendo quest'ultime nella formula dell'energia \cref{opgl:EnergiaBrutta} ricavo:
\begin{equation}\label{opgl:Energia}
	E=\frac 12\dot{\vec{y}}^T M \dot{\vec{y}} + \frac 12\vec{y}^T K \vec y
\end{equation}

Da questa, sempre con qualche passaggio pindarico misterioso, si arriva alla differenziale lineare omogenea:
\begin{equation}\label{opgl:Fondamentale}
	M\ddot{\vec y}=-K\vec y
\end{equation}
e ora, basta mostrare $2n$ soluzioni indipendenti e dai teoremi di esistenza ed unicità discenderà che lo spazio delle soluzioni è formato da combinazioni lineari di queste soluzioni.

Cerco soluzioni della forma $\vec y=\vec A e^{i\omega t}$, e sostituendo ottengo che deve valere:
\begin{equation*}
	(K-\omega^2M)\vec A=0 \iff M^{-1}K\vec A=\omega^2\vec A
\end{equation*}
quindi devo avere $\vec A$ autovettore relativo all'autovalore $\omega^2$ di $M^{-1}K$, ma quest'ultima matrice risulta essere diagonalizzabile e con autovalori positivi poichè le altre due sono simmetriche e positive. Di conseguenza ho la certezza di trovare un numero di autovettori $\vec A$ sufficiente.

Per trovare gli autovalori $\omega^2$ e gli autovettori mi basta imporre le seguenti due condizioni:
\begin{gather}\label{opgl:SistemaSoluzioni}
	\det(K-\omega^2M)=0 \\
	\vec A\in \ker(K-\omega^2M)
\end{gather}
e per ogni coppia $\omega^2,\vec A$ che si trova, è soluzione sia quella in cui $\omega>0$ sia quella per cui $\omega<0$.

Siano quindi $\omega_i, \vec A_i$ le $n$ soluzioni con $\omega_i>0$.
La soluzione generale della differenziale risulterà essere (proiettando le soluzioni sull'asse reale):
\begin{equation}\label{opgl:Soluzioni}
	\vec y = \sum_{i=1}^n \vec A_i\left( \lambda^+_i\cos\omega_i t + \lambda^-_i\sin \omega_i t \right)
\end{equation}

\subsection{Lagrangiana}\label{lag}
Si consideri un sistema fisico, dipendente dalle coordinate generalizzate $q$, per cui sia possibile scrivere l'energia cinetica $K(q,\dot q)$ e l'energia potenziale $V(q)$.
Si chiama Lagrangiana del sistema la quantità:
\begin{equation}\label{lag:Definizione}
	\mathcal L(q,\dot q)=K(q,\dot q)-V(q)
\end{equation}
cioè si assume implicitamente che il potenziale non dipende dalla velocità (cioè l'energia si mantiene) e che nulla dipende dal tempo. Queste assunzioni sono credibili quando non ci sono forze esterne agenti in modo periodico e le forze sono conservative.

Consideriamo innanzitutto un sistema inerziale di coordinate cartesiane $x=(x_1,x_2,\dots,x_n)$, nel quale la Lagrangiana è quindi della forma:
\begin{equation*}
	\mathcal L(x,\dot x)=K(x,\dot x)-V(x)=\frac 12 m \sum_i\dot x_i^2-V(x)
\end{equation*}
In questo caso posso applicare l'equazione di Newton e ho quindi che 
\begin{equation*}
	m_i\ddot x_i+\frac{\delta V}{\delta x_i}=0 \Longrightarrow \frac{\de}{\de t} m\dot x_i +\frac{\delta V}{\delta x_i}=0 \Longrightarrow \frac{\de}{\de t} \frac{\delta \frac 12 m \dot x_i^2}{\delta \dot x_i} +\frac{\delta V}{\delta x_i}=0 \Longrightarrow \frac\de{\de t} \frac{\delta \mathcal L}{\delta \dot x_i} - \frac{\delta \mathcal L}{\delta x_i} =0
\end{equation*}

Voglio ora dimostrare che quest'ultima equazione trovata vale indipendentemente dal sistema di coordinate scelto. Cioè che, dette $q_i$ le singole componenti di un qualunque sistema di coordinate generalizzate $q$, vale analogamente al caso inerziale:
\begin{equation}\label{Lag:EqEuleroLagrange}
	\frac\de{\de t} \frac{\delta \mathcal L}{\delta \dot q_i} = \frac{\delta \mathcal L}{\delta q_i}
\end{equation}
dove queste ultime si dicono equazioni di Eulero-Lagrange.

Supponiamo quindi di scrivere le componenti di $x$ in funzione delle $m$ componenti della coordinata generalizzata $q$, cioè $x_i=x_i(q_1,q_2,\dots,q_m)$. Utilizzando nuovamente l'equazione di Newton per il sistema inerziale $x$, ottengo che:
\begin{equation*}
	m \ddot x_i=-\frac{\delta V}{\delta x_i} \Longrightarrow \sum_i m \ddot x_i \frac{\delta x_i}{\delta q_j}=-\sum_i\frac{\delta V}{\delta x_i} \frac{\delta x_i}{\delta q_j}
\end{equation*}
Calcolo ora separatamente i due termini dell'ultima equazione. 

NON FINITO.

È importante notare che le Lagrangiane ci danno le equazioni del moto \emph{indipendentemente} dalle coordinate usate (ammessa una certa regolarità di queste) e questo le rende più versatili di $\vec F=m\vec a$ che invece si poteva applicare solo nei casi di coordinate cartesiane.

Inoltre dalle equazioni di Eulero-Lagrange si possono sempre ricavare le leggi di conservazione del sistema. Ad esempio il solo fatto che la Lagrangiana non dipende esplicitamente dal tempo implica la conservazione dell'energia.

Se il potenziale non dipende dalla coordinata $q_i$ è ovvio ricavare da \cref{Lag:EqEuleroLagrange} che la quantità $\frac{\delta \mathcal L}{\delta \dot q_i}$ si conserva (tale quantità è chiamata quantità di moto generalizzata).

In generale vale il Teorema di Noether, che afferma che per ogni simmetria della Lagrangiana si ha una relativa quantità conservata. Per simmetria si intende che se le coordinate vengono cambiate di una quantità piccola, la Lagrangiana rimane costante al prim'ordine in questo cambiamento.

Appunto col formalismo Lagrangiano si riescono a dimostrare agevolmente i risultati enunciati nella sezione precedente \cref{opgl}, poichè discendono quasi banalmente dalle equazioni di Eulero-Lagrange.
\end{document}

