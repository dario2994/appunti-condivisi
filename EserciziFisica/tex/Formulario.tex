\documentclass[../main.tex]{subfiles} 
\begin{document}
\section{Formulario}
\setcounter{equation}{0}
\renewcommand{\theequation}{F.\arabic{equation}}
\subsection{Coordinate polari} 
In coordinate polari si ha che per ogni versore $\hat{\iota}$ vale $\dot{\hat{\iota}}=\vec{\omega}\times\hat{\iota}$,
dove
\begin{equation}\label{OmegaPolari}
	\vec{\omega}=\dot{\theta}\hat{z}
\end{equation}
Derivando l'equazione $\vec{r}=r\hat{r}$ rispetto al tempo, ottengo quindi
\begin{equation}\label{VelCooPolari}
	\vec{v}=\dot{\vec{r}}=\dot{r}\hat{r}+r\dot{\theta}\hat{\theta}
\end{equation}
Derivando una seconda volta ottengo invece:
\begin{equation}\label{AccCooPolari}
	\vec{a}=\dot{\vec{v}} =(\ddot{r}-r\dot{\theta}^2)\hat{r}+(2\dot{r}\dot{\theta}+r\ddot{\theta})\hat{\theta}
\end{equation}

\subsection{Coordinate sferiche}
Ponendosi in un sistema di coordinate sferiche, vale $\dot{\hat{\iota}}=\vec{\omega}\times\hat{\iota}$
per ogni versore $\hat{\iota}$, dove
\begin{equation}\label{OmegaSferiche}
	\vec{\omega}=\dot{\theta}\hat{\varphi}+\dot{\varphi}\hat{z}
\end{equation}
Quindi derivando l'equazione $\vec{r}=\dot{r}\hat{r}$, ottengo
\begin{equation}\label{VelCooSferiche}
	\vec{v} =\dot{\vec{r}}=\dot{r}\hat{r}+r\dot{\theta}\hat{\theta}+r\hat{\varphi}\sin{\theta}\hat{\varphi}
\end{equation}
E derivando nuovamente rispetto al tempo ho che:
\begin{equation}\label{AccCooSferiche}
\begin{split}
	\vec{a}=	\dot{\vec{v}}=	&\left(\ddot{r}-r\dot{\theta}^2-r\dot{\varphi}^2\sin^2\theta \right)\hat{r}+\\
													&	+\left( r\ddot{\theta}+2\dot{r}\dot{\theta}-r\dot{\varphi}^2\sin\theta\cos\theta \right)\hat{\theta}+\\
													&	+\left( 2\dot{r}\dot{\varphi}\sin\theta+2r\dot{\varphi}\dot{\theta}\cos\theta+r\ddot{\varphi}\sin\theta \right)\hat{\varphi}
\end{split}
\end{equation}

\subsection{Moto in campo centrale}
Consideriamo un campo di forze centrali descritto da $\vec{F}=f(r)\hat{r}$. 
Innanzitutto abbiamo che tale campo è un campo conservativo, in quanto ammette un potenziale della forma:
\begin{equation*}
	U(r)=-\int_r^\infty{f(r) dr}
\end{equation*}
Inoltre in un campo di forze centrali si conserva il momento angolare, poichè il momento delle forze è uguale a 0,
e in particolare vale, grazie alle \cref{VelCooPolari}:
\begin{equation*}
	\vec{L}=m\vec{r}\times\vec{v}=mr^2\dot{\theta}\hat{z}
\end{equation*}

Il caso di un corpo in un campo di forze centrale si può sempre ricondurre ad un modo unidimensionale in $r$, secondo la formula:
\begin{equation}\label{CentraleUnidimensionale}
	E=\frac 12m\dot{r}^2+U_{eff}(r)
\end{equation}
dove il potenziale efficace $U_{eff}$ è definito come:
\begin{equation}\label{PotenzialeEfficace}
	U_{eff}(r)=U(r)+\frac{L^2}{2mr^2}
\end{equation}

Mi concentro in particolare sul campo gravitazionale generato da un corpo di massa $M$, che è proprio un campo 
centrale caratterizzato dalla forza:
\begin{equation*}
	\vec{F}=-\frac{GMm}{r^2}\hat{r}
\end{equation*}
e che genera quindi il potenziale:
\begin{equation*}
	U(r)=-\frac{GMm}{r}
\end{equation*}

Si dimostra che l'orbita che descrive un corpo in tale campo centrale è ellittica se l'energia di tale corpo è
minore di 0, parabolica se la sua energia è uguale a 0 e iperbolica se invece ha energia maggiore di 0.
In particolare se l'orbita è ellittica di semiasse maggiore $a$, il corpo ha energia:
\begin{equation} \label{EnergiaTotaleOrbita}
	E=-\frac{GMm}{2a}
\end{equation}
e il periodo dell'orbita è:
\begin{equation} \label{PeriodoOrbita}
	T=\frac{2\pi}{\sqrt{GM}}a^{\frac{3}{2}}
\end{equation}

Vale inoltre il teorema del viriale, che afferma che in un moto limitato si ha:
\begin{equation} \label{TeoremaDelViriale}
	<K>=-\frac{1}{2}<U>
\end{equation}
dove $<K>$ è l'energia cinetica media del corpo in moto e $<U>$ è la sua energia potenziale media.

Si trova che la velocità a cui si muove un corpo in orbita circolare, a distanza $r$ dal centro, è:
\begin{equation} \label{VelocitaMotoCircolare}
	v^2=\frac{GM}{r}
\end{equation}

mentre invece la velocità di fuga (ottenibile usando \cref{VelocitaMotoCircolare,TeoremaDelViriale}) è (sempre a distanza $r$):
\begin{equation} \label{VelocitaDiFuga}
	v^2=\frac{2GM}{r}
\end{equation}




\subsection{Sistemi di riferimento non inerziali}
Sia $K$ un sistema di riferimento non inerziale di centro $O'$, rispetto ad un sistema di riferimento
inerziale $K^I$ di centro $O$. Poichè sabbiamo che l'equazione di Newton $\vec{F}=m\vec{a}$
vale solo in sistemi di riferimento inerziali, vogliamo scoprire cosa si può dire del rapporto fra la forza
che agisce su un corpo e la sua accelerazione rispetto al sistema $K$.

Calcoliamo innanzitutto la relazione fra le accelerazioni nei due sistemi di riferimento. Poichè vale la
relazione $\overrightarrow{OP}=\overrightarrow{OO'}+\overrightarrow{O'P}$ per un qualsiasi punto $P$, derivando 
rispetto al tempo e chiamando $\overrightarrow{O'P}=\vec{r}$, ottengo:
\begin{equation}\label{VelNonInerziale}
\begin{split}
	\vec{v}=\dot{\overrightarrow{OP}}	& =\dot{\overrightarrow{OO'}}+\dot{\vec{r}}\\
													& =\overrightarrow{v_{tr}}+\dot{r}\hat{r}+\vec{\omega}\times\vec{r}\\
													& =\overrightarrow{v_{tr}}+\vec{v_r}+\vec{\omega}\times\vec{r}
\end{split}
\end{equation}
dove $v_{tr}$ (velocità di trascinamento) è la velocità di $O'$ rispetto ad $O$ e $v_r$ è la velocità relativa
a $K$ di $P$.

Derivando nuovamente ottengo invece:
\begin{equation}\label{AccNonInerziale}
\begin{split}
	\vec{a}=\vec{v}	& =\dot{\overrightarrow{v_{tr}}}+\dot{\vec{v_r}}+\left(\dot{\vec{\omega}}\times\vec{r}+\vec{\omega}\times\dot{\vec{r}}\right)\\
											& =\overrightarrow{a_{tr}}+\left(\vec{a_r}+\vec{\omega}\times\vec{v_r}\right)+\dot{\vec{\omega}}\times\vec{r}+\left[\vec{\omega}\times\vec{v_r}+\vec{\omega}\times(\vec{\omega}\times\vec{r})\right]\\
											& =\overrightarrow{a_{tr}}+\vec{a_r}+2\vec{\omega}\times\vec{v_r}+\dot{\vec{\omega}}\times\vec{r}+\vec{\omega}\times(\vec{\omega}\times\vec{r})\\
\end{split}
\end{equation}
dove analogamente a prima $a_{tr}$ (accelerazione di trascinamento) è l'accelerazione di $O'$ rispetto a $O$ e $a_r$ e 
$v_r$ sono rispettivamente l'accelerazione e la velocità relative a $K$ di $P$.

Quindi dall'equazione di Newton ottengo:
\begin{equation*}
\begin{split}
	\vec{F}=m\vec{a}	& =m\left[\overrightarrow{a_{tr}}+\vec{a_r}+2\vec{\omega}\times\vec{v_r}+\dot{\vec{\omega}}\times\vec{r}+\vec{\omega}\times(\vec{\omega}\times\vec{r})\right]\\
											& =m\vec{a_r}+\underline{m\overrightarrow{a_{tr}}}+\underline{2m\vec{\omega}\times\vec{v_r}}+\underline{m\dot{\vec{\omega}}\times\vec{r}}+\underline{m\vec{\omega}\times(\vec{\omega}\times\vec{r})}
\end{split}
\end{equation*}
I termini sottolineati sono forze apparenti; in particolare $2m\vec{\omega}\times\vec{v_r}$
è chiamata forza di Coriolis, mentre $m\vec{\omega}\times(\vec{\omega}\times\vec{r})$
è la forza centrifuga.

In un sistema non inerziale si può quindi riscrivere l'equazione di Newton come:
\begin{equation}\label{ForzaNonInerziale}
	m\vec{a_r}=\vec{F_r}=\vec{F}-m\overrightarrow{a_{tr}}-m\left[2\vec{\omega}\times\vec{v_r}+\dot{\vec{\omega}}\times\vec{r}+\vec{\omega}\times(\vec{\omega}\times\vec{r})\right]
\end{equation}

\subsection{Problema dei 2 corpi}
Sono dati 2 corpi di masse $m_1,m_2$ che interagiscono attraverso una forza che dipende 
solo dalla distanza $\vec F(\vec r)$.
Detta $\mu=\frac{m_1m_2}{m_1+m_2}$ la massa ridotta, vale l'equazione:
\begin{equation}\label{ForzaMassaRidotta}
	\vec F(\vec r)=\mu \ddot {\vec{r}}
\end{equation}
Detta $\overrightarrow{ r_Q }$ la posizione del centro di massa del sistema, vale la seguente espressione per l'energia cinetica:
\begin{equation}\label{Cinetica2Corpi}
	K=\frac12(m_1+m_2)\dot{\overrightarrow{r_Q}}^2+\frac12\mu\dot{\vec{r}}^2
\end{equation}
e questa è la formula per il momento angolare:
\begin{equation}\label{Momento2Corpi}
	\vec L=(m_1+m_2)\overrightarrow{r_Q}\times \dot{\overrightarrow{r_Q}}+\mu \vec r\times\dot{\vec r}
\end{equation}

Quindi nel complesso un sistema con 2 corpi si può disaccoppiare completamente nel moto del centro di massa e nel
moto relativo tra le 2 masse.

\subsection{Urti}
I problemi di urto in generale riguardano due particelle di massa $m_1$ e $m_2$ e velocità iniziali $\overrightarrow {v_1}$ e $\overrightarrow {v_2}$ e sono caratterizzati da:
\begin{itemize}
 \item interazione di corto raggio
 \item terzo principio di Newton $\rightarrow$ la quantità di moto totale si conserva
 \item conservazione (oppure no) dell'energia meccanica (ora supponiamo che si conservi)
\end{itemize}

Vogliamo trovare $\overrightarrow{{v_1}'}$ e $\overrightarrow{{v_2}'}$, le velocità dopo l'urto.
Scomponiamo il problema nel moto del centro di massa e nel moto relativo: poniamo $\overrightarrow v = \overrightarrow {v_2} - \overrightarrow {v_1}$,
$M = m_1+m_2$, $\mu = \frac{m_1+m_2}{M}$, $\overrightarrow {V_Q} = \frac{m_1\overrightarrow {v_1}+m_2\overrightarrow{v_2}}{M}$,
$\overrightarrow P = M\overrightarrow {V_Q}$, $\overrightarrow p = \mu \overrightarrow v$ e abbiamo che
\begin{equation}
 E = \frac{P^2}{2M} + \frac{p^2}{2\mu}
\end{equation}
Poichè $\Delta E = 0$ e $\overrightarrow P$ è costante, anche $p$ è costante, e l'unica cosa che può cambiare dopo l'urto è la sua direzione.
Dunque per risolvere completamente il problema è necessario anche conoscere l'angolo $\theta$ formato da $\overrightarrow p$ con $\overrightarrow {p'}$.
Sia $\hat{v_f}$ il versore diretto lungo $\overrightarrow {p'}$. Le velocità della massa $m_1$ prima e dopo l'urto, calcolate rispetto al centro di massa, sono
\begin{equation}
 \overrightarrow{v_{1Q}} = -\frac{m_2}{M}v\hat v \qquad \qquad
 \overrightarrow{{v_{1Q}}'} = -\frac{m_2}{M}v\hat{v_f}
\end{equation}
Nel sistema di riferimento del laboratorio, dunque, le velocità delle particelle dopo l'urto saranno
\begin{equation}\label{UrtiElastici}
 \overrightarrow {{v_1}'} = -\frac{m_2}{M}v\hat{v_f}+\overrightarrow {V_Q} \qquad \qquad
 \overrightarrow {{v_2}'} = \frac{m_1}{M}v\hat{v_f}+\overrightarrow {V_Q}
\end{equation}
Osservo infine che, se $|\overrightarrow {V_Q}| > |\overrightarrow{v_{1Q}}|$ ci sono dei limiti sull'angolo di deflessione della prima particella;
in particolare il massimo angolo di deflessione è $\theta_{max} = \arctan \frac{v_{1Q}}{V_Q}$.





\subsection{Problemi con massa variabile}
Si tratta di problemi in cui la massa cambia nel tempo, ad esempio un razzo che espelle carburante oppure una goccia di pioggia che cade nel vapore e man mano aumenta la sua dimensione.
Per risolverli, considero che al tempo $t$ avrò:
massa = $m$, velocità = $\vec v$
e al tempo $t+\de t$ avrò una massa $m+\de m$ che si muove a velocità $\vec v + \de \vec v$ e una massa $-\de m$ che si muove a velocità $\vec v + \vec v_{rel}$, dove $\vec v_{rel}$ è la velocità della massa espulsa/acquisita rispetto al corpo che sto considerando.
Poichè so che $\de \vec P = \vec F_{est} \de t$, ho che
$$ \left ( \left ( m + \de m \right ) \cdot \left ( \vec v + \de \vec v\right ) -\de m\left ( \vec v + \vec v_{rel} \right ) \right ) - m\vec v = \vec F_{est}\de t$$
da cui ottengo l'equazione generale per risolvere un problema con massa variabile:
\begin{equation}\label{MassaVariabile}
	m \de \vec v + \de m \vec v = \de m \left ( \vec v + \vec v_{rel} \right ) + \vec F_{est} \de t
\end{equation}
o, equivalentemente,
\begin{equation}
	\frac{\de }{\de t} m\vec v = \frac{\de m}{\de t}\left ( \vec v + \vec v_{rel} \right ) + \vec F_{est}
\end{equation}


\end{document}

