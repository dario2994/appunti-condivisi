\documentclass[../main.tex]{subfiles} 
\begin{document}
\section{Formulario}
\setcounter{equation}{0}
\renewcommand{\theequation}{F.\arabic{equation}}
\subsection{Coordinate polari} 
In coordinate polari si ha che per ogni versore $\hat{\iota}$ vale $\dot{\hat{\iota}}=\overrightarrow{\omega}\times\hat{\iota}$,
dove
\begin{equation}\label{OmegaPolari}
	\overrightarrow{w}=\dot{\theta}\hat{z}
\end{equation}
Derivando l'equazione $\overrightarrow{r}=r\hat{r}$ rispetto al tempo, ottengo quindi
\begin{equation}\label{VelCooPolari}
	\overrightarrow{v}=\dot{\overrightarrow{r}}=\dot{r}\hat{r}+r\dot{\theta}\hat{\theta}
\end{equation}
Derivando una seconda volta ottengo invece:
\begin{equation}\label{AccCooPolari}
	\overrightarrow{a}=\dot{\overrightarrow{v}} =(\ddot{r}-r\dot{\theta}^2)\hat{r}+(2\dot{r}\dot{\theta}+r\ddot{\theta})\hat{\theta}
\end{equation}

\subsection{Coordinate sferiche}
Ponendosi in un sistema di coordinate sferiche, vale $\dot{\hat{\iota}}=\overrightarrow{\omega}\times\hat{\iota}$
per ogni versore $\hat{\iota}$, dove
\begin{equation}\label{OmegaSferiche}
	\overrightarrow{\omega}=\dot{\theta}\hat{\varphi}+\dot{\varphi}\hat{z}
\end{equation}
Quindi derivando l'equazione $\overrightarrow{r}=\dot{r}\hat{r}$, ottengo
\begin{equation}\label{VelCooSferiche}
	\overrightarrow{v} =\dot{\overrightarrow{r}}=\dot{r}\hat{r}+r\dot{\theta}\hat{\theta}+r\hat{\varphi}\sin{\theta}\hat{\varphi}
\end{equation}
E derivando nuovamente rispetto al tempo ho che:
\begin{equation}\label{AccCooSferiche}
\begin{split}
	\overrightarrow{a}=	\dot{\overrightarrow{v}}=	&\left(\ddot{r}-r\dot{\theta}^2-r\dot{\varphi}^2\sin^2\theta \right)\hat{r}+\\
													&	+\left( r\ddot{\theta}+2\dot{r}\dot{\theta}-r\dot{\varphi}^2\sin\theta\cos\theta \right)\hat{\theta}+\\
													&	+\left( 2\dot{r}\dot{\varphi}\sin\theta+2r\dot{\varphi}\dot{\theta}\cos\theta+r\ddot{\varphi}\sin\theta \right)\hat{\varphi}
\end{split}
\end{equation}

\subsection{Moto in campo centrale}
Consideriamo un campo di forze centrali descritto da $\overrightarrow{F}=f(r)\hat{r}$. 
Innanzitutto abbiamo che tale campo è un campo conservativo, in quanto ammette un potenziale della forma:
\begin{equation*}
	U(r)=-\int_r^\infty{f(r) dr}
\end{equation*}
Inoltre in un campo di forze centrali si conserva il momento angolare, poichè il momento delle forze è uguale a 0,
e in particolare vale, grazie alle \cref{VelCooPolari}:
\begin{equation*}
	\overrightarrow{L}=m\overrightarrow{r}\times\overrightarrow{v}=mr^2\dot{\theta}\hat{z}
\end{equation*}

Mi concentro in particolare sul campo gravitazionale generato da un corpo di massa $M$, che è proprio un campo 
centrale caratterizzato dalla forza:
\begin{equation*}
	\overrightarrow{F}=-\frac{GMm}{r^2}\hat{r}
\end{equation*}
e che genera quindi il potenziale:
\begin{equation*}
	U(r)=-\frac{GMm}{r}
\end{equation*}

Si dimostra che l'orbita che descrive un corpo in tale campo centrale è ellittica se l'energia di tale corpo è
minore di 0, parabolica se la sua energia è uguale a 0 e iperbolica se invece ha energia maggiore di 0.
In particolare se l'orbita è ellittica di semiasse maggiore $a$, il corpo ha energia:
\begin{equation} \label{EnergiaTotaleOrbita}
	E=-\frac{GMm}{2a}
\end{equation}
e il periodo dell'orbita è:
\begin{equation} \label{PeriodoOrbita}
	T=\frac{2\pi}{\sqrt{GM}}a^{\frac{3}{2}}
\end{equation}

Vale inoltre il teorema del viriale, che afferma che in un moto limitato si ha:
\begin{equation} \label{TeoremaDelViriale}
	<K>=-\frac{1}{2}<U>
\end{equation}
dove $<K>$ è l'energia cinetica media del corpo in moto e $<U>$ è la sua energia potenziale media.

Si trova che la velocità a cui si muove un corpo in orbita circolare, a distanza $r$ dal centro, è:
\begin{equation} \label{VelocitaMotoCircolare}
	v^2=\frac{GM}{r}
\end{equation}

mentre invece la velocità di fuga (ottenibile usando \cref{VelocitaMotoCircolare,TeoremaDelViriale}) è (sempre a distanza $r$):
\begin{equation} \label{VelocitaDiFuga}
	v^2=\frac{2GM}{r}
\end{equation}




\subsection{Sistemi di riferimento non inerziali}
Sia $K$ un sistema di riferimento non inerziale di centro $O'$, rispetto ad un sistema di riferimento
inerziale $K^I$ di centro $O$. Poichè sabbiamo che l'equazione di Newton $\overrightarrow{F}=m\overrightarrow{a}$
vale solo in sistemi di riferimento inerziali, vogliamo scoprire cosa si può dire del rapporto fra la forza
che agisce su un corpo e la sua accelerazione rispetto al sistema $K$.

Calcoliamo innanzitutto la relazione fra le accelerazioni nei due sistemi di riferimento. Poichè vale la
relazione $\overrightarrow{OP}=\overrightarrow{OO'}+\overrightarrow{O'P}$ per un qualsiasi punto $P$, derivando 
rispetto al tempo e chiamando $\overrightarrow{O'P}=\overrightarrow{r}$, ottengo:
\begin{equation}\label{VelNonInerziale}
\begin{split}
	\overrightarrow{v}=\dot{\overrightarrow{OP}}	& =\dot{\overrightarrow{OO'}}+\dot{\overrightarrow{r}}\\
													& =\overrightarrow{v_{tr}}+\dot{r}\hat{r}+\overrightarrow{\omega}\times\overrightarrow{r}\\
													& =\overrightarrow{v_{tr}}+\overrightarrow{v_r}+\overrightarrow{\omega}\times\overrightarrow{r}
\end{split}
\end{equation}
dove $v_{tr}$ (velocità di trascinamento) è la velocità di $O'$ rispetto ad $O$ e $v_r$ è la velocità relativa
a $K$ di $P$.

Derivando nuovamente ottengo invece:
\begin{equation}\label{AccNonInerziale}
\begin{split}
	\overrightarrow{a}=\overrightarrow{v}	& =\dot{\overrightarrow{v_{tr}}}+\dot{\overrightarrow{v_r}}+\left(\dot{\overrightarrow{\omega}}\times\overrightarrow{r}+\overrightarrow{\omega}\times\dot{\overrightarrow{r}}\right)\\
											& =\overrightarrow{a_{tr}}+\left(\overrightarrow{a_r}+\overrightarrow{\omega}\times\overrightarrow{v_r}\right)+\dot{\overrightarrow{\omega}}\times\overrightarrow{r}+\left[\overrightarrow{\omega}\times\overrightarrow{v_r}+\overrightarrow{\omega}\times(\overrightarrow{\omega}\times\overrightarrow{r})\right]\\
											& =\overrightarrow{a_{tr}}+\overrightarrow{a_r}+2\overrightarrow{\omega}\times\overrightarrow{v_r}+\dot{\overrightarrow{\omega}}\times\overrightarrow{r}+\overrightarrow{\omega}\times(\overrightarrow{\omega}\times\overrightarrow{r})\\
\end{split}
\end{equation}
dove analogamente a prima $a_{tr}$ (accelerazione di trascinamento) è l'accelerazione di $O'$ rispetto a $O$ e $a_r$ e 
$v_r$ sono rispettivamente l'accelerazione e la velocità relative a $K$ di $P$.

Quindi dall'equazione di Newton ottengo:
\begin{equation*}
\begin{split}
	\overrightarrow{F}=m\overrightarrow{a}	& =m\left[\overrightarrow{a_{tr}}+\overrightarrow{a_r}+2\overrightarrow{\omega}\times\overrightarrow{v_r}+\dot{\overrightarrow{\omega}}\times\overrightarrow{r}+\overrightarrow{\omega}\times(\overrightarrow{\omega}\times\overrightarrow{r})\right]\\
											& =m\overrightarrow{a_r}+\underline{m\overrightarrow{a_{tr}}}+\underline{2m\overrightarrow{\omega}\times\overrightarrow{v_r}}+\underline{m\dot{\overrightarrow{\omega}}\times\overrightarrow{r}}+\underline{m\overrightarrow{\omega}\times(\overrightarrow{\omega}\times\overrightarrow{r})}
\end{split}
\end{equation*}
I termini sottolineati sono forze apparenti; in particolare $2m\overrightarrow{\omega}\times\overrightarrow{v_r}$
è chiamata forza di Coriolis, mentre $m\overrightarrow{\omega}\times(\overrightarrow{\omega}\times\overrightarrow{r})$
è la forza centrifuga.

In un sistema non inerziale si può quindi riscrivere l'equazione di Newton come:
\begin{equation}\label{ForzaNonInerziale}
	m\overrightarrow{a_r}=\overrightarrow{F_r}=\overrightarrow{F}-m\overrightarrow{a_{tr}}-m\left[2\overrightarrow{\omega}\times\overrightarrow{v_r}+\dot{\overrightarrow{\omega}}\times\overrightarrow{r}+\overrightarrow{\omega}\times(\overrightarrow{\omega}\times\overrightarrow{r})\right]
\end{equation}





\end{document}
