\documentclass[a4paper,12pt]{article}
\usepackage{stilebase}

\title{Appunti sulla funzione Gamma}
\author{Federico Glaudo \and Giada Franz}

\begin{document}

\maketitle
\clearpage


\begin{abstract}
	Queste sono delle dispense che raccolgono i primi risultati che si ottengono
	studiando la funzione Gamma e le funzioni a lei collegate.

	Le dimostrazioni sono volutamente tutte elementari: basta la teoria che si
	ha dopo un corso di analisi I per comprenderle tutte.

	Questo documento lo abbiamo scritto sia per allenarci a scrivere in latex, sia
	perchè è impossibile trovare una guida alla Gamma che sia allo stesso tempo 
	elementare, completa e in italiano.
	
	Ovunque in questo documento tratteremo la Gamma ponendo come dominio i reali positivi, ma intendiamo chiarire
	che questa funzione può essere studiata come funzione dal piano complesso in se stesso, solo che non abbiamo i mezzi per farlo.

	Speriamo di esservi d'aiuto.
\end{abstract}
\clearpage

\tableofcontents
\clearpage

\section{Approssimazione del fattoriale}
Questa sezione è interamente dedicata ad un risultato che riteniamo preliminare allo studio della funzione Gamma:
l'approssimazione di Stirling.

Questo risultato, oltre ad essere usato successivamente per ottenere vari risultati sulla Gamma, può avvicinare il lettore
all'idea che la Gamma nasce come estensione ``esatta'' del fattoriale a tutti i numeri reali positivi.

\begin{lemma}[Prodotto di Wallis]\label{f:WallisProduct}
	Vale la seguente identità:
	\begin{equation*}
		\frac{\pi}{2}=\prod_{n=1}^\infty\frac{(2n)^2}{(2n-1)(2n+1)}
	\end{equation*}
\end{lemma}
\begin{proof}
	Definisco
	\begin{equation*}
		I_n=\int_0^\pi \sin^n{x} dx
	\end{equation*}
	Integrando per parti ottengo che
	\begin{equation*}
		I_n=\frac{n-1}{n}\cdot I_{n-2}
	\end{equation*}
	In particolare ho che $I_0=\pi$ e $I_1=2$, da cui:
	\begin{gather*}
		I_{2n}=\frac{(2n-1)!!}{(2n)!!}\cdot\pi\\
		I_{2n+1}=\frac{(2n)!!}{(2n+1)!!}\cdot 2
	\end{gather*}
	Dato che $\sin^{n+1}x\le \sin^n x\le \sin^{n-1} x$ per ogni $x\in\mathbb{R}$, ho che $I_{n+1}\le I_n\le I_{n-1}$,
	da cui dividendo per $I_{n+1}$:
	\begin{gather*}
		1\le \frac{I_n}{I_{n+1}}\le \frac{I_{n-1}}{I_{n+1}}=1+\frac{1}{n}\\
		\Longrightarrow \lim_{n\to\infty}\frac{I_n}{I_{n+1}}=1\\
		\Longrightarrow \lim_{n\to\infty}\frac{(2n-1)!!}{(2n)!!}\cdot\frac{(2n+1)!!}{(2n)!!}\cdot\frac{\pi}{2}=1\\
		\Longrightarrow \frac{\pi}{2}=\lim_{n\to\infty}\frac{\left[(2n)!!\right]^2}{(2n-1)!!(2n+1)!!}=\prod_{n=1}^\infty\frac{(2n)^2}{(2n-1)(2n+1)}
	\end{gather*}
\end{proof}


\begin{theorem}[Approssimazione di Stirling per il fattoriale]\label{f:StirlingNaturali}
	Riportiamo una dimostrazione dell'approssimazione di Stirling sui numeri naturali (che è una caso
	particolare di \cref{StirlingGamma})
	\begin{equation*}
		\lim_{n\to\infty}{\frac{n!}{(n/e)^n\sqrt{2\pi n}}}=1
	\end{equation*}
\end{theorem}
\begin{proof}
	Dimostriamo innanzitutto che esiste finito il seguente limite:
	\begin{equation}\label{f:EsisteLimite}
		\lim_{n\to\infty}{\frac{n!}{(n/e)^n\sqrt{n}}}
	\end{equation}
	cioè, passando al logaritmo, che esiste il limite:
	\begin{equation}\label{f:EsisteLimiteLogaritmo}
		\lim_{n\to\infty}{\sum_{k=1}^{n}\ln k-n\ln n+n-\frac{1}{2}\ln n}
	\end{equation}
	
	Vale la seguente identità:
	\begin{equation*}
		\sum_{k=1}^{n}\ln k=n\ln n-\sum_{k=1}^{n-1}{k\left(\ln(k+1)-\ln k \right)}=n\ln n-\sum_{k=1}^{n-1}{k\ln\left(1+\frac{1}{k}\right)}
	\end{equation*}
	da cui, sostituendo il logaritmo con il suo sviluppo di Taylor, ottengo:
	\begin{equation*}
	\begin{split}
		\sum_{k=1}^{n}\ln k	& =n\ln n-\sum_{k=1}^{n-1}{k\left(\frac{1}{k}-\frac{1}{2k^2}+\bigO\left(\frac{1}{k^3}\right) \right)}\\
							& =n\ln n-\sum_{k=1}^{n-1}\left(1-\frac{1}{2k}+\bigO\left(\frac{1}{k^2}\right)  \right)\\
							& =n\ln n-n+1+\frac{1}{2}\sum_{k=1}^{n-1}\frac{1}{k}+\sum_{k=1}^{n-1}\bigO\left(\frac{1}{k^2}\right)
	\end{split}
	\end{equation*}
	Ora sfruttando che $\displaystyle \sum_{k=1}^{n-1}\frac{1}{n}-\ln n\to \gamma$ e che $\displaystyle \sum_{k=1}^{n-1}\bigO\left(\frac{1}{k^2}\right)\to L$
	dato che converge assolutamente, ottengo:
	\begin{equation}\label{f:QuasiStirlingNaturali}
		\sum_{k=1}^{n}\ln k=n\ln n-n+\frac{1}{2}\ln n + \left(\frac{\gamma}2 + L\right)
	\end{equation}
	Che passando al limite è proprio equivalente a \cref{f:EsisteLimiteLogaritmo}.
	
	Dimostriamo ora che il valore del limite di \cref{f:EsisteLimite} è proprio $\sqrt{2\pi}$.
	Grazie a \cref{f:WallisProduct} ho che:
	\begin{equation}\label{f:PiFactorial}
	\begin{split}
		\sqrt{\pi}	& =\sqrt{2}\prod_{n=1}^\infty\frac{2n}{\sqrt{(2n-1)(2n+1)}}\\
					& =\lim_{n\to\infty}\frac{(2n)!!}{(2n-1)!!\sqrt{n}}\\
					& =\lim_{n\to\infty}\frac{\left[(2n)!!\right]^2}{(2n)!\sqrt{n}}=\lim_{n\to\infty}\frac{2^{2n}\left(n!\right)^2}{(2n)!\sqrt{n}}\\
	\end{split}
	\end{equation}
	
	Definisco ora:
	\begin{equation*}
		a_n=\frac{n!}{(n/e)^n\sqrt{n}}
	\end{equation*}
	Poichè $\lim_{n\to\infty}{a_n}$ esiste, vale facilmente che $\lim_{n\to\infty}a_n/a_{2n}=1$, quindi:
	\begin{equation*}
		\lim_{n\to\infty}{\frac{n!}{(n/e)^n\sqrt{n}}}=\lim_{n\to\infty}{a_n}=\lim_{n\to\infty}{\frac{a_n^2}{a_{2n}}}
	\end{equation*}
	Grazie alla \cref{f:PiFactorial}, ho però che:
	\begin{equation*}
	\begin{split}
		\lim_{n\to\infty}{\frac{a_n^2}{a_{2n}}}
		&=\lim_{n\to\infty}{\frac{\left(n!\right)^2}{(n/e)^{2n}n}\cdot \frac{(2n/e)^{2n}\sqrt{2n}}{(2n)!} }\\
		&=\lim_{n\to\infty}{\sqrt{2}\cdot\frac{2^{2n} \left(n!\right)^2}{(2n)!\sqrt{n}} }=\sqrt{2\pi}
	\end{split}
	\end{equation*}
	da cui la tesi.
\end{proof}

\section{Funzione Gamma}
In questa sezione definiamo la funzione Gamma e, dopo alcuni risultati introduttivi, dimostriamo il teorema di Bohr-Mollerup, che
in qualche senso mostra che la funzione Gamma è l'unico modo \emph{sensato} di estendere il fattoriale ai reali positivi.

Infine dimostreremo alcune definizioni equivalenti della funzione Gamma.

\begin{definition}[Funzione Gamma]\label{FunzioneGamma} 
	La funzione Gamma è definita da $\mathbb{R^+}$ in $\mathbb{R^+}$ come:
	\begin{equation*}
		\Gamma(x)=\int_0^{\infty}{e^{-t}t^{x-1}dt}
	\end{equation*}
\end{definition}

\begin{lemma}\label{GammaConverge}
	L'integrale mostrato in \cref{FunzioneGamma} per definire $\Gamma(x)$ converge per ogni $x>0$.
\end{lemma}
\begin{proof}
	La funzione sotto il segno di integrale è ovviamente integrabile su tutto $\mathbb{R^+}$, ed è positiva.
	
	Non resta che far vedere che l'integrale improprio esiste finito.
	
	Per quanto riguarda la convergenza in 0, è facile osservare che per ogni $t>0$ vale la disuguaglianza $e^{-t}t^{x-1}\le t^{x-1}$ 
	e perciò l'integrale in $0$ converge visto che $x>0$ per ipotesi.
	
	Riguardo la convergenza a $+\infty$, anche qui basta notare che definitivamente $e^{-t}t^{x-1}\le e^{-\frac t2}$
	e questo implica facilmente la convergenza a $+\infty$.
	
	Unendo le 2 convergenze dimostrate si ottiene proprio la convergenza dell'integrale su tutta la semiretta dei reali positivi (e quindi la 
	definizione della Gamma è coerente su tutto $\mathbb{R^+})$.
\end{proof}


\begin{remark}[Valore di $\Gamma(1)$]\label{ValoreGamma1}
	Vale in particolare che:
	\begin{equation*}
		\Gamma(1)=\int_0^{\infty}{e^{-t}dt}=\left[-e^{-t}\right]_0^{\infty}=1
	\end{equation*}
\end{remark}


\begin{lemma}\label{FunzionaleGamma}
	La funzione Gamma rispetta la seguente identità per ogni $x\in\mathbb{R^+}$:
	\begin{equation*}
		\Gamma(x+1)=x\Gamma(x)
	\end{equation*}
\end{lemma}
\begin{proof}
	Integrando per parti ottengo:
	\begin{equation*} 
		\Gamma(x+1)=\int_0^{\infty}{e^{-t}t^xdt}=\left[-e^{-t}t^x\right]_0^{\infty}+\int_0^{\infty}xe^{-t}t^{x-1}dt=x\Gamma(x)
	\end{equation*}
\end{proof}

\begin{remark} \label{RicorsioneGamma}
	Per ogni $x\in\mathbb{R^+}$ e per ogni $n\in\mathbb{N}$, sfruttando il \cref{FunzionaleGamma}, vale la seguente relazione:
	\begin{equation}
		\Gamma(x+n)=(x+n-1)!\Gamma(x)
	\end{equation}
	dove $(x+n-1)!=(x+n-1)(x+n-2)\dots(x+1)x$.
\end{remark}

\begin{remark} \label{ValoreGammaNaturali}
	Ponendo $x=0$ in \cref{RicorsioneGamma} ed utilizzando \cref{ValoreGamma1}, ottengo in particolare che per ogni $n\in\mathbb{N}$ vale:
	\begin{equation}
		\Gamma(n)=(n-1)!
	\end{equation}
\end{remark}


\begin{lemma}\label{GammaLogConvessa}
	La funzione Gamma è log-convessa.
\end{lemma}
\begin{proof}
	La funzione Gamma è log-convessa se e solo se per ogni $0<\lambda,\mu < 1$ tali che $\lambda+\mu=1$
	e per ogni $x,y \in \mathbb{R}^+$ vale:
	\begin{equation}\label{QuasiGammaLogConvessa}
	\begin{split} 
		\log \Gamma(\lambda x+\mu y )  & \le \lambda \log \Gamma(x) + \mu\log \Gamma( y )\\
		& \Updownarrow  \\
		\Gamma(\lambda x+\mu y ) & \le  \Gamma(x)^{\lambda}\Gamma( y )^{\mu}
	\end{split}\end{equation}
	Ora sostituendo la \cref{FunzioneGamma} nella \cref{QuasiGammaLogConvessa}, mi riconduco a dimostrare:
	\begin{equation*}
		\int_0^{\infty}{e^{-t}t^{\lambda x+\mu y-1}dx}=\int_0^{\infty}{(e^{-t}t^{x-1})^\lambda (e^{-t}t^{y-1})^{\mu}dt} \le 
		\left(\int_0^{\infty}{e^{-t}t^{x-1}dt}\right)^\lambda \left(\int_0^{\infty}{e^{-t}t^{y-1}}\right)^{\mu}
	\end{equation*}
	che è vera per la disuguaglianza di Hölder.
\end{proof}
 
\begin{theorem}[Teorema di Bohr-Mollerup] \label{BohrMollerup}
	Esiste un'unica funzione $f :\mathbb{R}^{+}\to\mathbb{R} $ che rispetta le tre seguenti proprietà:
	\begin{itemize}
		\item $f(1)=1$
		\item $f(x+1)=xf(x)$
		\item $f$ è log-convessa
	\end{itemize}
	In particolare tale funzione è la funzione Gamma già definita in \cref{FunzioneGamma}.
\end{theorem}
\begin{proof}
	La \cref{ValoreGamma1}, il \cref{FunzionaleGamma} e il \cref{GammaLogConvessa}, ci dicono già che la funzione
	Gamma rispetta tutte e tre le proprietà elencate. Vogliamo dimostrare che non ne esistono altre.\\
	Sia quindi $f:\mathbb{R^+}\to\mathbb{R}$ una funzione che rispetta le tre proprietà, allora analogamente a 
	\cref{RicorsioneGamma} ho che per ogni $x\in \mathbb{R^+}$ e per ogni $n\in\mathbb{N}$ vale:
	\begin{equation}\label{RicorsioneQuasiGamma}
		f(x+n)=(x+n-1)!f(x)
	\end{equation}
	e di conseguenza, dato che $f(1)=1$, per ogni $n\in\mathbb{N}$:
	\begin{equation}\label{ValoreQuasiGammaNaturali}
		f(n)=(n-1)!
	\end{equation}
	Sia ora $M(u,v)=\frac{\log(f(u))-\log(f(v))}{u-v}$ il rapporto incrementale
	di $\log(f(x))$ fra $u$ e $v$. Poichè $f$ è log-convessa, deve valere che $M(u,v)$ è crescente sia in $u$ che in $v$.
	Quindi in particolare per ogni $0<x\le 1$ vale:
	\begin{gather*}
		M(n,n-1) \le M(n,n+x) \le M(n,n+1) \\
		\Updownarrow \\
		\log(f(n))-\log(f(n-1)) \le \frac{\log(f(n+x))-\log(f(n))}{x} \le \log(f(n+1))-\log(f(n)) 
	\end{gather*}
	Da cui, utilizzando la \cref{RicorsioneQuasiGamma} e la \cref{ValoreQuasiGammaNaturali}, ottengo che:
	\begin{gather*}
		x \log(n-1) \le \log \left( \frac{f(n+x)}{f(n)} \right) \le x\log(n)\\
		\iff (n-1)^x \le \frac{f(n+x)}{f(n)} \le n^x \\
		\iff (n-1)^x \le  \frac{f(x)(n+x-1)!}{(n-1)!}  \le n^x \\
		\iff \left(1-\frac{1}{n}\right)^x \le  \frac{f(x)(n+x-1)!}{n^x(n-1)!} \le 1
	\end{gather*}
	E passando al limite:
	\begin{gather*}
		\lim_{n\to \infty} \left(1-\frac{1}{n}\right)^x \le \lim_{n\to \infty}  \frac{f(x)(n+x-1)!}{n^x(n-1)!} \le 1 \\
		\Longrightarrow \lim_{n\to \infty} \frac{f(x)(n+x-1)!}{n^x(n-1)!}  = 1 \\
		\Longrightarrow f(x) = \lim_{n\to \infty} \frac{n^x(n-1)!}{(n+x-1)!}
	\end{gather*}
	Quindi per $0<x\le 1$ vale:
	\begin{equation}\label{QuasiGaussFormulaPiccoli}
		f(x)=\lim_{n\to \infty} \frac{n^x(n-1)!}{(n+x-1)!}=\lim_{n\to \infty} \frac{n^xn!}{(n+x)!}
	\end{equation}
	
	Quindi se la funzione $f$ esiste deve rispettare la \cref{QuasiGaussFormulaPiccoli} per $0<x<1$, ma dalla \cref{RicorsioneQuasiGamma} 
	ottengo facilmente che se $f$ è fissata per $0<x\le 1$ è fissata per tutto $\mathbb{R^+}$.
	Di conseguenza se esiste una funzione $f$ che rispetta le tre proprietà, essa deve essere unica.
\end{proof}
 
\begin{corollary}[Formula di Gauss]\label{GaussFormula}
        Per ogni $x\in\mathbb{R^+}$ vale la seguente formula per la $\Gamma$:
\begin{equation}\label{GaussRealFormula}
        \Gamma(x)=\lim_{n\to \infty} \frac{n^xn!}{x(x+1)\cdots(x+n)}
\end{equation}
\end{corollary}
\begin{proof}
        Abbiamo già dimostrato che se una funzione rispetta le ipotesi del \cref{BohrMollerup}, allora vale la
        \cref{QuasiGaussFormulaPiccoli} con $0<x\le 1$, ma la funzione Gamma rispetta tali ipotesi e quindi vale \cref{GaussRealFormula}
        per $0<x\le 1$.
        Resta da estendere tale risultato ad $x>1$.
        
        Dimostro che se vale la \cref{GaussRealFormula} per $x\in\mathbb{R^+}$ allora vale anche per $x+1$, visto che ho già che vale
        su tutto l'intervallo $(0,1]$ questo implica (per facile induzione) che \cref{GaussRealFormula} vale per ogni $x$ reale positivo.
        
        Sotto l'ipotesi che la formula valga per $x$ e sfruttando \cref{FunzionaleGamma} risulta vera la seguente catena di uguaglianze:
        \begin{equation*}\begin{split}
                \Gamma(x+1) & =x\Gamma(x)=x\lim_{n\to \infty} \frac{n^xn!}{x(x+1)\cdots(x+n)} =
                \lim_{n\to \infty} \frac{n^{x+1}n!}{(x+1)\cdots(x+n)(x+n+1)}\cdot\frac{x+n+1}{n}\\
                & = \lim_{n\to \infty} \frac{n^{x+1}n!}{(x+1)\cdots(x+n)(x+n+1)}
        \end{split}\end{equation*}
        e questo è proprio quello che serviva per concludere la dimostrazione.
\end{proof}

\begin{corollary}[Formula di Weierstrass]\label{WeierstrassFormula}
	Per ogni $x\in\mathbb{R^+}$ vale la seguente formula per la $\Gamma$:
\begin{equation*}
	\Gamma(x)=\frac{e^{-\gamma x}}x\prod_{i=1}^{\infty} \frac{e^{\frac xi}}{1+\frac xi}
\end{equation*}
\end{corollary}
\begin{proof}
	Vale per ogni $x\in\mathbb{R^+},\ n\in\mathbb{N}$ la seguente identità (ottenuta solo attraverso manipolazioni algebriche):
	\begin{equation}\label{WeierIdentity}
		\frac{n^xn!}{x(x+1)\cdots (x+n)}=e^{x\left(\log{n}-\frac11-\frac12-\dots-\frac1n\right)}\frac1x\prod_{i=1}^n\frac{e^{\frac xi}}{1+\frac xi}
	\end{equation}
	
	Ricordiamo inoltre che la costante di Eulero-Mascheroni $\gamma\approx0.577$ è definita come:
	\begin{equation}\label{EuleroMascheroni}
		\gamma=\lim_{n\to\infty} \sum_{i=1}^n \frac1i -\log{n}
	\end{equation}
	
	Applicando l'operatore $\lim_{n\to\infty}$ a entrambi i membri della \cref{WeierIdentity} e sfruttando \cref{EuleroMascheroni,GaussFormula} ottengo:
	\begin{equation*}
		\Gamma(x)=\lim_{n\to \infty} \frac{n^xn!}{x(x+1)\cdots(x+n)}
		=\lim_{n\to\infty} e^{x\left(\log{n}-\frac11-\frac12-\dots-\frac1n\right)}\frac1x\prod_{i=1}^n\frac{e^{\frac xn}}{1+\frac xn}
		=\frac{e^{-\gamma x}}x\prod_{i=1}^{\infty} \frac{e^{\frac xi}}{1+\frac xi}
	\end{equation*}
	Che è proprio la formula di Weierstrass.
\end{proof}







\include{./tex/BetaFunction}
\section{Alcune formule sulla Gamma}
Proponiamo ora alcune formule rispettate dalla funzione Gamma. Queste risultano utili sia per dimostrari ulteriori proprietà della
funzione qui studiata, sia per il calcolo di molti integrali definiti.

\begin{lemma}[Prodotto infinito del seno] \label{gf:ProdottoSeno}
	Vale la seguente formula per il seno come prodotto infinito:
	\begin{equation*}
		\frac{\sin (\pi x)}{\pi x}=\prod_{k=0}^{\infty} \left(1-\frac{x^2}{ k^2}\right)
	\end{equation*}
\end{lemma}
\begin{proof}
	Sia $I_n:\mathbb{R}\to\mathbb{R}$ una funzione definita come segue
	\begin{equation} \label{gf:IntSenDefinition}
		I_n(z)=\int_0^{\frac{\pi}2}{\cos{zt}\cos^nt \de t}
	\end{equation}
	
	Innanzitutto ho che vale
	\begin{equation}\label{gf:SenRis1}
		\frac{I_0(z)}{I_0(0)}=\frac{\int_0^{\frac{\pi}2}{\cos{zt} \de t}}{\frac{\pi}2}=\frac{\sin(\frac {\pi z} 2)}{\frac {\pi z} 2}
	\end{equation}
	
	Inoltre derivando due volte per parti la \cref{gf:IntSenDefinition} ottengo
	\begin{align*}
		z^2 I_n(z) 	& =z^2 \int_0^{\frac{\pi}2}{\cos{zt}\cos^nt \de t}=z\left[ \sin{zt}\cos^nt \right]_0^{\frac{\pi}2}+nz\int_0^{\frac{\pi}2}{\sin{zt}\sin t\cos^{n-1}t \de t} \\
					& =n\left[ -\cos{zt}\sin t\cos^{n-1}t \right]_0^{\frac{\pi}2}+n\int_0^{\frac{\pi}2}{\cos{zt}(\cos^nt-(n-1)\sin^2t\cos^{n-2}t )\de t} \\
					& =n^2 \int_0^{\frac{\pi}2}{\cos{zt}\cos^nt \de t} -n(n-1) \int_0^{\frac{\pi}2}{\cos{zt}\cos^{n-2}t \de t}=n^2 I_n(z)-n(n-1) I_{n-2}(z)
	\end{align*}
	da cui 
	\begin{equation} \label{gf:SenRis2}
		(n^2-z^2)I_n(z)=n(n-1)I_{n-2}(z) \Longrightarrow \frac{I_{n-2}(z)}{I_{n-2}(0)}=\left( 1-\frac{z^2}{n^2} \right) \frac{I_n(z)}{I_n(0)}
	\end{equation}
	
	Infine, utilizzando la serie di disuguaglianze $\displaystyle 1-\cos t\le \frac{t^2}{2} \le \frac{t\tan t}{2}$, ho che
	\begin{align*}
		|I_n(0)-I_n(z)|	& \le \int_0^{\frac{\pi}2}{(1-\cos{zt})\cos^nt \de t}\le \frac{z^2}{2}\int_0^{\frac{\pi}2}{t\tan t \cos^nt \de t}\\
						& =\frac{z^2}{2}\int_0^{\frac{\pi}2}{t\sin t \cos^{n-1}t \de t}=\frac{z^2}{2n}\left[ t\cos^nt \right]_0^{\frac{\pi}2}+\frac{z^2}{2n}\int_0^{\frac{\pi}2}{\cos^nt \de t}=\frac{z^2}{2n}I_n(0)
	\end{align*}
	quindi ottengo
	\begin{equation} \label{gf:SenRis3}
		\left| 1-\frac{I_n(z)}{I_n(0)} \right|\le \frac{z^2}{2n} \Longrightarrow \frac{I_n(z)}{I_n(0)}\to 1
	\end{equation}
	
	Sfruttando la \cref{gf:SenRis1}, la \cref{gf:SenRis2} e la \cref{gf:SenRis3}, e ponendo $x=2z$, ottengo
	\begin{gather*}
		\frac{\sin(\pi x)}{\pi x}=\frac{I_0(2x)}{I_0(0)}=\frac{I_{2n}(2x)}{I_{2n}(0)}\prod_{k=1}^{n}\left( 1-\frac{x^2}{k^2} \right) \\
		\Longrightarrow \frac{\sin(\pi x)}{\pi x}=\lim_{n\to\infty} \frac{I_{2n}(2x)}{I_{2n}(0)}\prod_{k=1}^{n}\left( 1-\frac{x^2}{k^2} \right) = \prod_{k=1}^{\infty}\left( 1-\frac{x^2}{k^2} \right)
	\end{gather*}
	che è proprio la tesi.



\end{proof}



\begin{theorem}[Formula di riflessione] \label{Riflessione}
	Per $0<x<1$ reale vale:
	\begin{equation*}
		\Gamma(x)\Gamma(1-x)=\frac{\pi}{\sin(\pi x)}
	\end{equation*}
\end{theorem}
\begin{proof}
	Uso l'identità \cref{GaussFormula} per ottenere:
	\begin{equation}\begin{split} \label{MezzaRiflessione}
		\Gamma(x)\Gamma(1-x) & = \lim_{n\to\infty} \dfrac{n^xn!}{x(x+1)\cdots (x+n)} \cdot 
		\dfrac{n^{1-x}n!}{(1-x)(1-x+1)\cdots (1-x+n)}\\
		& =\lim_{n\to\infty} \frac{n}{n+1-x} \cdot \frac{1}{x} \cdot \prod_{k=1}^{n}\dfrac{k^2}{k^2-x^2} \\
		& =\left(\lim_{n\to\infty} \frac{n}{n+1-x} \right) \cdot 
		\left( \lim_{n\to\infty} x \prod_{k=1}^{n}\left(1-\frac{x^2}{k^2}\right) \right)^{-1}
	\end{split}\end{equation}

	Ora, grazie alle \cref{gf:ProdottoSeno}, ho che l'ultimo membro della \cref{MezzaRiflessione} risulta $\frac{\pi}{\sin(\pi x)}$, da cui la tesi del teorema.
\end{proof}

\begin{remark}[Valore di $\Gamma\left(\frac12\right)$]
	Ponendo $x=1$ nella formula di riflessione appena ottenuta si giunge a $\Gamma\left(\frac12\right)^2=\pi$. 
	Ricordando che la $\Gamma$ è sempre positiva sui reali positivi, ne deduco $\Gamma\left(\frac12\right)=\sqrt\pi$.
\end{remark}


\begin{theorem}\label{gf:Moltiplicazione}
	Fissato $m\in\mathbb{N}$ e $x\in\mathbb{R^+}$ vale:
	\begin{equation*}
		\Gamma(x)\Gamma\left(x+\frac 1m\right)\Gamma\left(x+\frac 2m\right)\cdots \Gamma\left(x+\frac {m-1}m\right)=
		(2\pi)^{\frac{m-1}2}\cdot m^{\frac12-mx}\cdot\Gamma(mx)
	\end{equation*}
\end{theorem}
\begin{proof}
	Sfruttando \cref{GaussFormula} e manipolando algebricamente l'espressione ottengo:
	\begin{equation}\begin{split}\label{gf:ContMolt}
		&\Gamma(x)\Gamma\left(x+\frac 1m\right)\Gamma\left(x+\frac 2m\right)\cdots \Gamma\left(x+\frac {m-1}m\right) =\\
		&=\lim_{n\to\infty} \prod_{i=0}^{m-1}\frac{n^{x+\frac im}n!}
		{\left(x+\frac im\right)\left(x+\frac im\right)\cdots \left(x+n+\frac im\right)}\\
		&=\lim_{n\to\infty}\frac{n^{mx}n^{\frac{m-1}2}(n!)^m}{\frac 1{m^{mn+1}}\cdot mx(mx+1)(mx+2)\cdots (mx+mn)}\cdot 
		\frac1 {\left(x+\frac 1m+n\right)\cdots \left(x+\frac {m-1}m+n\right)}\\
		&=\lim_{n\to\infty}\left(\frac{(mn)^{mx}mn!}{mx(mx+1)\cdots(mx+mn)}\right)m^{\frac12-mx}
		\left(\frac{n!^m}{(mn)!}\cdot\frac{m^{mn+\frac12}n^{\frac{m-1}2}}
		{\left(x+\frac 1m+n\right)\cdots \left(x+\frac {m-1}m+n\right)}\right)\\
		&=\Gamma(mx)m^{\frac12-mx} \cdot \lim_{n\to\infty} \left( \frac{n!^m}{(mn)!}\cdot 
		\frac{m^{mn+\frac12}}{n^{\frac{m-1}2}}\right)\cdot 
		\lim_{n\to\infty}\left( \left( 1+\frac{x+\frac{1}m}{n}\right)^{-1}
		\cdots \left( 1+\frac{x+\frac{m-1}m}{n}\right)^{-1}\right) \\
		&=\Gamma(mx)m^{\frac12-mx} \cdot \lim_{n\to\infty}\frac{n!^m}{(mn)!}\cdot \frac{m^{mn+\frac12}}{n^{\frac{m-1}2}}
	\end{split}\end{equation}
	
	Inoltre, applicando \cref{f:StirlingNaturali}, ho che (assumendo $m$ costante):
	\begin{equation}\label{gf:StirlingApplic}
		\frac{n!^m}{(mn)!}\sim\frac{\left(\sqrt{2\pi n}\right)^m\left(\frac ne\right)^{mn}}{\sqrt{2\pi mn}\left(\frac{mn}e\right)^{mn}}
		=\frac{(2\pi n)^{\frac{m-1}2}}{m^{mn+\frac12}}\Rightarrow 
		\lim_{n\to\infty}\frac{n!^m}{(mn)!}\cdot \frac{m^{mn+\frac12}}{n^{\frac{m-1}2}}=(2\pi)^{\frac{m-1}2}
	\end{equation}
	
	Sostituendo \cref{gf:StirlingApplic} in \cref{gf:ContMolt} ottengo la tesi del teorema.
\end{proof}

\begin{corollary}[Formula di duplicazione] \label{Duplicazione}
	Ponendo $m=2$ in \cref{gf:Moltiplicazione} ottengo che per $x>0$ reale vale:
	\begin{equation*}
		\Gamma(x)\Gamma\left(x+\frac12\right)=2^{1-2x}\sqrt{\pi}\Gamma(2x)
	\end{equation*}
\end{corollary}

\begin{remark}[Valore di $\Gamma\left(\frac12\right)$]
	Ponendo $x=\frac12$ nella formula di duplicazione appena ottenuta si giunge a 
	\begin{equation*}
		\Gamma\left(\frac12\right)\Gamma(1)=\sqrt{\pi}\Gamma(1) \to \Gamma\left(\frac12\right)=\sqrt{\pi}
	\end{equation*}
\end{remark}

\begin{remark}[Integrale di Gauss] \label{GaussIntegral}
	Sostituendo $t=s^2$ nella \cref{FunzioneGamma}, ottengo:
	\begin{equation}
		\Gamma(x)=2\int_0^{\infty}{s^{2x-1}e^{-s^2}ds}
	\end{equation}
	Da cui, ponendo $x=\frac{1}{2}$:
	\begin{equation*}
		\sqrt{\pi}=\Gamma\left(\frac{1}{2}\right)=2\int_0^{\infty}{e^{-s^2}ds}\to \int_{-\infty}^{\infty}{e^{-s^2}ds}=\sqrt{\pi}
	\end{equation*}
\end{remark}




\section{Differenziazione della gamma}\label{dg}
Questa sezione è la prima a dover usare esplicitamente ragionamenti del tipo \emph{epsilon, delta}.
Per evitare di usare questi ragionamenti si dovrebbero applicare teoremi forti riguardo la possibilità
di scambiare tra loro gli operatori di integrale, di derivata e di limite.
Tuttavia questi teoremi prescindono dal programma di analisi I e perciò abbiamo deciso di trovare strade
che li evitino, mantenendo le dimostrazioni le più elementari possibili.

Dimostreremo che la Gamma è una funzione derivabile infinite volte ed espliciteremo le sue derivate.
Inoltre studieremo alcune proprieta della Digamma (derivata logaritmica della Gamma).

\begin{lemma}\label{dg:SenzaIntegrale}
	Per ogni $t\in \mathbb{R}^+,\ x> -1$ e $h\in\mathbb{R}$ tale che $x+h>-1$ vale:
	\begin{equation*}
		|t^{x+h}-t^x-h\log t\cdot t^x| \le h^2\log^2t\cdot\max\left(t^x,t^{x+h}\right)
	\end{equation*}
\end{lemma}
\begin{proof}
	Applicando il teorema di Lagrange due volte, ottengo che esistono $\xi_1,\xi_2$ con valore assoluto minore di $|h|$ tali che:
	\begin{equation}\label{dg:LagrangeApprox}
		t^{x+h}-t^x-h\log t\cdot t^x=h\log t\cdot t^{x+\xi_1}-h\log t\cdot t^x=h\log t\left(t^{x+\xi_1}-t^x\right)=h\xi_1\log^2 t\cdot t^{x+\xi_2}
	\end{equation}
	Però $h,\xi_1$ devono avere lo stesso segno per Lagrange, e quindi $h\xi_1\le h^2$ ed inoltre vale per la convessità dell'esponenziale che $t^{x+\xi_2}\le\max\left(t^x,t^{x+h}\right)$.
	Sostituendo questi risultati in \cref{dg:LagrangeApprox} e aggiungendo un valore assoluto ottengo:
	\begin{equation*}
		|t^{x+h}-t^x-h\log t\cdot t^x|\le h^2\log^2 t\cdot \max\left(t^x,t^{x+h}\right)
	\end{equation*}
	che è la tesi.
\end{proof}

\begin{theorem}\label{dg:GammaDerivata}
	La funzione Gamma è derivabile e la derivata rispetta:
	\begin{equation*}
		\Gamma'(x)=\int_0^{\infty} \log{t}\cdot t^{x-1}e^{-t}\de t
	\end{equation*}
\end{theorem}
\begin{proof}
	Definisco $\Gamma'$ esattamente come supposto nell'enunciato del teorema e dimostro che è la derivata.

	Sfruttando la monotonia dell'operatore integrale e \cref{dg:SenzaIntegrale} ho, per $x-1,h$ come nelle ipotesi di \cref{dg:SenzaIntegrale}:
	\begin{equation}\label{dg:ConIntegrale} \begin{split}
		|\Gamma(x+h)-\Gamma(x)-h\Gamma'(x)|&\le \int_0^{\infty}e^{-t}|t^{x-1+h}-t^{x-1}-h\log t\cdot t^{x-1}| \\
		&\le h^2\int_0^{\infty} e^{-t}\log^2 t\cdot \max\left(t^{x-1},t^{x-1+h}\right)
	\end{split}\end{equation}

	Ma ora è facile notare che, indifferentemente dal valore di $h$ (purchè sufficientemente piccolo), l'integrale $\int_0^{\infty} e^{-t}\log^2 t\cdot \max\left(t^{x-1},t^{x-1+h}\right)$ esiste finito e limitato, cioè è $\bigO(1)$ in $h$, e di conseguenza dalla \cref{dg:ConIntegrale} ottengo:
	\begin{equation*}
		|\Gamma(x+h)-\Gamma(x)-h\Gamma'(x)|=\bigO(h^2)
	\end{equation*}
	e questo dimostra che la derivata di $\Gamma$ è $\Gamma'$.
\end{proof}
\begin{corollary}\label{dg:GammaDerivataN}
	La derivata $n$-esima della Gamma rispetta:
	\begin{equation*}
		\Gamma^{(n)}(x)=\int_0^{\infty} \log^{n}{t}\cdot t^{x-1}e^{-t}\de t
	\end{equation*}
\end{corollary}
\begin{proof}
	Si dimostra agevolmente per induzione su $n$. In particolare il passo induttivo si svolge
	ripetendo pedissequamente la dimostrazione di \cref{dg:GammaDerivata}, solo sostituendo ovunque $t^{x-1}e^{-x}$
	con $\log^{n-1}{t}\cdot t^{x-1}e^{-x}$.
\end{proof}
\begin{lemma}\label{dg:SommaDer}
	Siano $f_i:\mathbb{R^+}\to\mathbb R$ (con $i=1,2,\dots$) funzioni tali che:
	\begin{itemize}
		\item $\forall\ i\in\mathbb{N} :\ f_i\in C^2$
		\item $\forall x\in \mathbb{R^+}$ la sommatoria $\sum_{i=1}^{\infty}f_i(x)$ converge.
		\item La sommatoria $\sum_{i=1}^{\infty} ||f_i''||$ converge.
	\end{itemize}
	
	Allora vale la seguente identità tra derivate:
	\begin{equation*}
		\frac{\de \sum_{i=1}^{\infty}f_i(x)}{\de x}=\sum_{i=1}^{\infty}f_i'(x)
	\end{equation*}
\end{lemma}
\begin{proof}
	Applicando due volte il teorema di Lagrange ho che fissati $x,h\in\mathbb{R^+}$ e $i\in\mathbb{N}$
	esistono $0<h''\le h'\le h$ tali che valgano le identità che uso per ottenere la disuguaglianza:
	\begin{equation}\label{dg:DisFond}
		\left\lvert\frac{f_i(x+h)-f_i(x)}h - f_i'(x)\right\rvert=\left\lvert f_i'(x+h') - f_i'(x)\right\rvert
		=\left\lvert h'\cdot f''(x+h'') \right\rvert\le h||f_i''||
	\end{equation}
	
	Sia ora $C=\sum_{i=1}^{\infty} ||f_i''||$ (che esiste per ipotesi).
	
	Risulta vera, sfruttando \cref{dg:DisFond}, che:
	\begin{equation}\label{dg:DerFond}
		hC\ge \sum_{i=1}^{\infty}\left\lvert\frac{f_i(x+h)-f_i(x)}h - f_i'(x)\right\rvert \ge
		\left\lvert\sum_{i=1}^{\infty}\frac{f_i(x+h)-f_i(x)}h - f_i'(x)\right\rvert
	\end{equation}
	Dove l'ultima disuguaglianza ha senso poichè la serie converge assolutamente, quindi converge.
	
	Vale però che una somma di serie equivale alla serie della somma e perciò
	(essendo furbi, portando le cose dalla parte giusta e applicando due volte tale risultato nell'ordine giusto)
	si dimostra che:
	\begin{equation}\label{dg:IdStupida}
		\sum_{i=1}^{\infty}\frac{f_i(x+h)-f_i(x)}h - f_i'(x)=
		\frac{\sum_{i=1}^{\infty} f_i(x+h)- \sum_{i=1}^{\infty}f_i(x)}h -\sum_{i=1}^{\infty}f_i'(x)
	\end{equation}

	Unendo \cref{dg:DerFond,dg:IdStupida} ottengo, per definizione di derivata, proprio la tesi.
\end{proof}

\begin{definition}\label{dg:Digamma}
	Sia $\psi:\mathbb{R^+}\to\mathbb{R}$ la funzione Digamma, cioè la derivata logaritmica della funzione Gamma:
	\begin{equation*}
		\psi(x)=\frac{\de \log\Gamma(x)}{\de x}=\frac{\Gamma'(x)}{\Gamma(x)}
	\end{equation*}
\end{definition}

\begin{lemma}\label{dg:DigammaCresc}
	La funzione Digamma è crescente.
\end{lemma}
\begin{proof}
	La \cref{GammaLogConvessa} mi assicura che la funzione $\log\Gamma$ è convessa, ma questa è per \cref{dg:GammaDerivata}
	anche derivabile. Ma una funzione derivabile e convessa ha derivata crescente, per la definizione stessa della Digamma
	questo implica la sua crescenza.
\end{proof}

\begin{theorem}\label{dg:DigammaFunzionale}
	La funzione Digamma rispetta la seguente equazione funzionale per ogni $x>0$:
	\begin{equation*}
		\psi(x+1)=\psi(x)+\frac 1x
	\end{equation*}
\end{theorem}
\begin{proof}
	Sfruttando \cref{dg:GammaDerivata} derivo entrambi i membri di \cref{FunzionaleGamma}:
	\begin{equation}\label{dg:FunzionaleDer}
		\Gamma'(x+1)=x\Gamma'(x)+\Gamma(x)
	\end{equation}
	
	Ora unisco \cref{FunzionaleGamma,dg:FunzionaleDer,dg:Digamma} ottenendo la tesi:
	\begin{equation*}
		\psi(x+1)=\frac{\Gamma'(x+1)}{\Gamma(x+1)}=\frac{x\Gamma'(x)+\Gamma(x)}{x\Gamma(x)}=
		\frac{\Gamma'(x)}{\Gamma(x)}+\frac 1x=\psi(x)+\frac 1x
	\end{equation*}
\end{proof}

\begin{theorem}\label{dg:DigammaId}
	La funzione Digamma rispetta la seguente identità per ogni $x>0$:
	\begin{equation*}
		\psi(x)=-\gamma-\frac 1x +\sum_{i=1}^{\infty} \frac x{i(i+x)}
	\end{equation*}
\end{theorem}
\begin{proof}
	Sfruttando \cref{WeierstrassFormula} e le proprietà di base del logaritmo si ha:
	\begin{equation}\label{dg:DigammaQuasi}
		\log\Gamma(x)=-\gamma x- \log{x} +\sum_{i=1}^{\infty}\left(\frac xi -\log\left(1+\frac xi\right)\right)
	\end{equation}
	
	Definisco ora $f_i:\mathbb{R^+}\to\mathbb{R}$ come $f_i(x)=\frac xi -\log\left(1+\frac xi\right)$.
	
	Verifico che le $f_i$ rispettino tutte le ipotesi di \cref{dg:SommaDer}:
	\begin{itemize}
		\item $f_i\in C^2$ e questo è ovvio vista la definizione delle $f_i$.
		\item $\sum_{i=1}^{\infty}f_i(x)$ converge sempre, ma anche questo è ovvio per la \cref{dg:DigammaQuasi}, visto che
		il membro di sinistra esiste finito e nel membro di destra compare questa sommatoria infinita (con un numero finito
		di altri addendi).
		\item Per verificare l'ultima ipotesi derivo due volte le $f_i$:
		\begin{equation}\label{dg:Der1Fi}
			f_i'(x)=\frac 1i -\frac{\frac 1i}{1+\frac xi}=\frac{x}{i(i+x)}
		\end{equation}
		\begin{equation*}
			f_i''(x)=\frac 1{i(i+x)} - \frac{x}{i(i+x)^2}=\frac 1{i(i+x)}\left(1-\frac x{i+x}\right)\le \frac 1{i(i+x)}\le \frac 1{i^2}
		\end{equation*}
		Ma allora (notando la facile positività di $f_i''$) ho che $||f_i''||\le \frac 1{i^2}$ e questo implica facilmente 
		l'ultima ipotesi di \cref{dg:SommaDer}.
	\end{itemize}
	
	Derivo entrambi i membri di \cref{dg:DigammaQuasi} e applico \cref{dg:SommaDer,dg:Der1Fi} (di cui ho appena verificato che siano rispettate
	le ipotesi):
	\begin{equation*}
		\psi(x)=\frac{\de \log\Gamma(x)}{\de x}=-\gamma-\frac 1x + \sum_{i=1}^{\infty}\frac{x}{i(i+x)}
	\end{equation*}
	che è la tesi.
\end{proof}

\begin{corollary}\label{dg:psi(1)}
	Risulta:
	\begin{equation*}
		\psi(1)=-\gamma
	\end{equation*}
\end{corollary}
\begin{proof}
	Ponendo $x=1$ in \cref{dg:DigammaId} ottengo:
	\begin{equation*}
		\psi(1)=-\gamma-1+\sum_{i=1}^{\infty}\frac 1{i(i+1)}
	\end{equation*}
	Ma la sommatoria infinita è, per antonomasia, telescopica e risulta valere $1$. Sostistuendo il valore della sommatoria
	ottengo proprio il risultato.
\end{proof}

\begin{theorem}\label{dg:psiApprox}
	Vale la seguente stima asintotica della funzione Digamma:
	\begin{equation*}
		\psi(x)=\log{x}+\bigO\left(\frac 1x\right)
	\end{equation*}
\end{theorem}
\begin{proof}
	Dato $n\in\mathbb{N}$ applicando ripetutamente la \cref{dg:DigammaFunzionale} e sostituendo \cref{dg:psi(1)}
	ho che vale:
	\begin{equation}\label{dg:PsiIntera}
		\psi(n)=\psi(1)+\sum_{i=1}^{n-1}\frac 1i = -\gamma+\sum_{i=1}^{n}\frac 1i + \bigO\left(\frac 1n\right)
	\end{equation}
	Ora però si può applicare il fatto noto riguardo la convergenza a $\gamma$ della differenza tra logaritmo e numeri armonici, 
	così si ottiene:
	\begin{equation}\label{dg:EuMaschConv}
		\sum_{i=1}^{n}\frac 1i - \log n = \gamma +\bigO\left(\frac 1n\right)
	\end{equation}
	
	Sostituisco ora \cref{dg:EuMaschConv} in \cref{dg:PsiIntera} ottenendo:
	\begin{equation}\label{dg:PsiIntera2}
		\psi(n)=\log n +\bigO\left(\frac 1n\right)
	\end{equation}

	Fissato $x>1$ reale, sfruttando \cref{dg:DigammaCresc,dg:DigammaFunzionale} ottengo la seguente catena di
	disuguaglianze (definendo per precisione $\lceil x\rceil$ come il più piccolo intero \emph{strettamente} maggiore di $x$):
	\begin{equation}\label{dg:PsiReale}
		\psi(\partint{x})\le \psi(x) \le \psi(\lceil x\rceil)=\psi(\partint{x})+\frac 1{\partint{x}} \Rightarrow
		\psi(x)=\psi(\partint{x})+\bigO\left(\frac 1x\right)
	\end{equation}

	Applico ora \cref{dg:PsiIntera2} nella \cref{dg:PsiReale} ottenendo la tesi:
	\begin{equation*}
		\psi(x)=\psi(\partint{x})+\bigO\left(\frac 1x\right)=
		\log{\partint{x} }+\bigO\left(\frac 1{\partint{x}}\right)+\bigO\left(\frac 1x\right)=
		\log{x}+\bigO\left(\frac 1x\right)
	\end{equation*}
\end{proof}


\begin{corollary}\label{dg:RapportoAsint}
	Per $c$ in un intervallo limitato di $\mathbb{R}$, vale la seguente formula asintotica per $x\to\infty$:
	\begin{equation}
		\frac{\Gamma(x+c)}{\Gamma(x)}\sim x^c
	\end{equation}
\end{corollary}
\begin{proof}
	Applicando due volte il teorema di Lagrange e sfruttando \cref{dg:psiApprox} ne ricavo che esistono $\xi_1,\xi_2$ tali che:
	\begin{equation}\label{dg:LogaritmoAsintotico}\begin{split}
		\log\Gamma(x+c)-\log\Gamma(x)&=c\cdot\psi(x+\xi_1)=c\left(\log(x+\xi_1)+\bigO\left(\frac 1{x+\xi_1}\right)\right)\\
		&=c\left(\log x+\frac{\xi_1}{x+\xi_2}+\bigO\left(\frac 1{x+\xi}\right)\right)=c\log x+\bigO\left(\frac1{x}\right)
	\end{split}\end{equation}
	dove nell'ultimo passaggio ho sfruttato il fatto che $c$ è limitato e quindi anche $\xi_1,\xi_2$.

	Ma ora facendo l'esponenziale a entrambi i membri di \cref{dg:LogaritmoAsintotico} ottengo proprio la tesi.
\end{proof}









\section{Approssimazione della funzione Gamma}
Concludiamo queste dispense proponendo due dimostrazioni intrinsecamente distinte dell'approssimazione di Stirling per
la funzione Gamma. 

È importante notare che la prima dimostrazione è interamente autocontenuta (a parte il primo risultato, che è un fatto
noto di analisi) mentre la seconda usa pesantemente vari fatti dimostrati nelle sezioni precedenti.

\begin{lemma}\label{ga:LimiteIntegrali}
	Siano $f_n$, con $n\in\mathbb{N}$, e $g$ funzioni definite in $(0,\infty)$ e Riemann-integrabili su $[a,b]$ per ogni
	$0<a<b<\infty$. Se valgono le seguenti proprietà:
	\begin{itemize}
		\item $|f_n|\le g$ per ogni $n\in\mathbb{N}$;
		\item $f_n\to f$ uniformemente in ogni intervallo chiuso di $(0,\infty)$;
		\item $\int_0^\infty{g(x)dx}<\infty$.
	\end{itemize}
	Allora vale che:
	\begin{equation*}
		\lim_{n\to\infty}\int_0^\infty f_n(x)dx=\int_0^\infty f(x)dx
	\end{equation*}
\end{lemma}

\begin{proof}
	Dimostro innanzitutto che per ogni $0<a<b<\infty$, ho che $f$ è integrabile su $[a,b]$ e in particolare vale:
	\begin{equation*}
		\int_a^b{f(x)dx}=\lim_{n->\infty}\int_a^b{f_n(x)dx}
	\end{equation*}
	Sia $\sigma_m$ la suddivisione equispaziata dell'intervallo $[a,b]$ di nodi $x_i=a+\frac{i}{m}(b-a)$,
	allora $f$ è Riemann-integrabile su $[a,b]$ se per ogni $\varepsilon>0$ esiste $M$ tale che se $m\ge M$
	allora $|S(f,\sigma_m)-s(f,\sigma_m)|<\varepsilon$.
	Per la disuguaglianza triangolare vale però che:
	\begin{equation*}
		|S(f,\sigma_m)-s(f,\sigma_m)|\le  
		|S(f,\sigma_m)-S(f_n,\sigma_m)|+ |s(f_n,\sigma_m)-s(f,\sigma_m)|+ |S(f_n,\sigma_m)-s(f_n,\sigma_m)|
	\end{equation*}
	Dato che $f_n\to f$ uniformemente su $[a,b]$, allora per ogni $\mu>0$ esiste $N$ tale che per ogni
	$n\ge N$ e per ogni $x\in[a,b]$ vale $|f(x)-f_n(x)|<\mu$, quindi vale facilmente che per ogni $m\ge M$:
	\begin{equation} \label{ViciniInIntervallo}
	\begin{split}
		|S(f,\sigma_m)-S(f_n,\sigma_m)| & <\mu(b-a) \\
		|s(f,\sigma_m)-s(f_n,\sigma_m)| & <\mu(b-a)
	\end{split}
	\end{equation} 
	E dato che $f_n$ è Riemann-integrabile
	su $[a,b]$ per ogni $n$, allora per ogni $\mu>0$ esiste $M$ tale che per ogni $m\ge M$ vale:
	\begin{equation}
		|S(f_n,\sigma_m)-s(f_n,\sigma_m)|<\mu
	\end{equation}
	Ma allora per ogni $\mu>0$ esistono $N$ e $M$ tali che per ogni $n\ge N$ e $m\ge M$ vale:
	\begin{gather*}
		|S(f,\sigma_m)-S(f_n,\sigma_m)|+ |s(f_n,\sigma_m)-s(f,\sigma_m)|+ |S(f_n,\sigma_m)-s(f_n,\sigma_m)|<\mu(2b-2a+1) \\
		\Longrightarrow |S(f,\sigma_m)-s(f,\sigma_m)|< \mu(2b-2a+1)
	\end{gather*}
	Quindi scegliendo $\mu=\varepsilon/(2b-2a+1)$ ottengo che per ogni $m\ge M$ vale:
	\begin{equation*}
	|S(f,\sigma_m)-s(f,\sigma_m)|< \varepsilon
	\end{equation*}
	Quindi $f$ è Riemann-integrabile su $[a,b]$, e in particolare dalla \cref{ViciniInIntervallo} si ottiene facilmente
	anche che
	\begin{equation}\label{IntegraleInIntervallo}
		\int_a^b{f(x)dx}=\lim_{n->\infty}\int_a^b{f_n(x)dx}
	\end{equation}
	Da quest'ultima relazione ottengo anche che $|f|$ è Riemann-integrabile in $[a,b]$, poichè in generale se
	$h$ è una funzione Riemann-integrabile in $[a,b]$ lo è anche $|h|$.\\
	Ora, dato che $|f_n|\le g$ per ogni $n\in\mathbb{N}$, passando al limite ottengo che $|f(x)|\le g(x)$ per ogni $x\in(0,\infty)$.
	Di conseguenza, dato che per quanto già detto $|f|$ è Riemann-integrabile in $[a,b]$, $|f|$ è Riemann-integrabile
	anche in $(0,\infty)$ perchè è non negativa.\\
	Ma dato che esiste l'integrale improprio di $|f|$ su $(0,\infty)$, allora esiste anche l'integrale improprio di $f$ su 
	$(0,\infty)$, e analogamente a quanto detto prima su un intervallo, vale proprio:
	\begin{equation*}
		\int_0^\infty{f(x)dx}=\lim_{n->\infty}\int_0^\infty{f_n(x)dx}
	\end{equation*}
\end{proof}

\begin{theorem}[Approssimazione di Stirling per la funzione Gamma]\label{StirlingGamma}
	La formula di Stirling offre un'approssimazione per $\Gamma(x+1)$:
	\begin{equation*}
		\lim_{x\to\infty}\frac{\Gamma(x+1)}{(x/e)^x\sqrt{2\pi x}}=1
	\end{equation*}
\end{theorem}
\begin{proof}
	Sostituendo $t=x(1+s\sqrt{2/x})$ nella definizione della funzione Gamma \cref{FunzioneGamma} ottengo:
	\begin{equation} \label{ga:QuasiStirling}
	\begin{split}
		\Gamma(x+1) & = \int_0^\infty{e^{-t}t^{x}dt}\\
					& = \int_{-\sqrt{\frac{x}{2}}}^\infty{ e^{-x(1+s\sqrt{2/x})} x^x\left(1+s\sqrt{\frac{2}{x}}\right)^x \sqrt{2x} ds}\\
					& = e^{-x}x^x\sqrt{2x}\int_{-\sqrt{\frac{x}{2}}}^\infty{ \left[e^{-s\sqrt{2/x}} \left(1+s\sqrt{\frac{2}{x}}\right)\right]^x ds}\\
					& = e^{-x}x^x\sqrt{2x}\int_{-\sqrt{\frac{x}{2}}}^\infty{ e^{-s\sqrt{2x}+x\ln\left(1+s\sqrt{\frac{2}{x}}\right)} ds}\\
					& = e^{-x}x^x\sqrt{2x}\int_{-\sqrt{\frac{x}{2}}}^\infty{ e^{-s^2\left(\frac{\sqrt{2x}}{s}-\frac{x}{s^2}\ln\left(1+s\sqrt{\frac{2}{x}}\right)\right)} ds}
	\end{split}
	\end{equation}
	Ora definisco per comodità $h_x(s)=\frac{\sqrt{2x}}{s}-\frac{x}{s^2}\ln\left(1+s\sqrt{\frac{2}{x}}\right)$ 
	e mi concentro su quest'ultimo integrale, in particolare esso è uguale a:
	\begin{equation*}
		\int_{-\infty}^\infty{ f_x(s) ds}
	\end{equation*}
	dove 
	\begin{equation*}
		f_x(s)=\begin{cases}
				e^{-s^2h_x(s)}, & \mbox{se } -\sqrt{x/2}<s<\infty \\
							 0, & \mbox{se } s\le -\sqrt{x/2}
	\end{cases}
	\end{equation*}
	
	Dimostro innanzitutto che $f_x(s)\to e^{-s^2}$ uniformemente su $[a,b]$ per $x\to\infty$, con 
	$-\infty<a<b<\infty$. Ciò equivale a dimostrare che $h_x(s)\to 1$ uniformemente su $[a,b]$ per $x\to\infty$.
	Sostituendo a $\ln\left(1+s\sqrt{\frac{2}{x}}\right)$ il suo sviluppo di Taylor ottengo la seguente identità:
	\begin{equation*}
	\begin{split}
		h_x(s)	& =\frac{\sqrt{2x}}{s}-\frac{x}{s^2}\ln\left(1+s\sqrt{\frac{2}{x}}\right)\\
				& =\frac{\sqrt{2x}}{s}-\frac{x}{s^2}\left(s\sqrt{\frac{2}{x}}-\frac{s^2}{x}+\bigO\left(\frac{s^3}{x\sqrt{x}}\right) \right)\\
				& =1+\bigO\left( \frac{s}{\sqrt{x}} \right)
	\end{split}
	\end{equation*}
	Quindi esistono $H,K$ tali che, se $\left\lvert\frac s{\sqrt x}\right\rvert<H$ allora vale 
	$|h_x(s)-1|<K\frac{|s|}{\sqrt{x}}$ per ogni $s\in [a,b]$.
	
	Allora in particolare, ponendo $M=\max_{s\in[a,b]}\{|s|\}$, risulta che, se $\left\lvert\frac M{\sqrt x}\right\rvert<H$,
	allora $|h_x(s)-1|<K\frac{M}{\sqrt{x}}$. Ma questo equivale a dire:
	\begin{equation*}
		h_x(s)=1+\bigO\left( \frac{1}{\sqrt{x}} \right)
	\end{equation*}
	Che implica facilmente $h_x(s)\to 1$ uniformemente su $[a,b]$ per $x\to\infty$.
	
	Ora distinguo due casi:
	\begin{itemize}
		\item	Se $s<0$ ho che $0\le f_x(s) \le e^{-s^2}$, utilizzando l'uniforme convergenza di $f_x(s)$ in ogni
				intervallo e \cref{GaussIntegral}, posso applicare \cref{ga:LimiteIntegrali} alle funzioni $f_x(s)$ 
				e ottengo:
				\begin{equation}\label{ga:PartialIntegralNeg}
					\lim_{x\to\infty} \int_{-\infty}^0{f_x(s)ds}=\int_{-\infty}^0{\left(\lim_{x\to\infty}f_x(s)\right)ds}
					=\int_{-\infty}^0{e^{-s^2}ds}=\frac{\sqrt{\pi}}{2}
				\end{equation}
		\item	Se $s>0$ ho che $0\le f_x(s) \le f_1(s)$, in quanto $h_x(s)\ge h_1(s)$. Inoltre, grazie
				alla \cref{FunzioneGamma}, vale che:
				\begin{equation*}
				\begin{split}
					\int_0^{\infty}{f_1(s)ds}	&=\int_0^\infty{e^{-\sqrt{2}s}\left(1+\sqrt{2}s \right)ds}\\
												&=\frac{e}{\sqrt{2}}\int_0^\infty{e^{-t}t dt}\\
												&=\frac{e}{\sqrt{2}}\Gamma(2)<\infty
				\end{split}
				\end{equation*}
				Quindi, utilizzando la convergenza uniforme di $f_x(s)$ a $e^{-s^2}$ in ogni intervallo
				$[a,b]$, posso nuovamente applicare \cref{ga:LimiteIntegrali}, da cui ottengo:
				\begin{equation}\label{ga:PartialIntegralPos}
					\lim_{x\to\infty} \int_0^\infty{f_x(s)ds}=\int_0^\infty{\left(\lim_{x\to\infty}f_x(s)\right)ds}
					=\int_0^\infty{e^{-s^2}ds}=\frac{\sqrt{\pi}}{2}
				\end{equation}
	\end{itemize}
	
	Infine, unendo \cref{ga:PartialIntegralNeg} e \cref{ga:PartialIntegralPos}, ottengo:
	\begin{equation*}
		\lim_{x\to\infty}\int_{-\infty}^\infty{ f_x(s) ds}=\sqrt{\pi}
	\end{equation*}
	da cui passando al limite nella \cref{ga:QuasiStirling} ottengo la tesi.
\end{proof}

\begin{proof}[Seconda dimostrazione di Stirling]
	Applicando \cref{dg:RapportoAsint,f:StirlingNaturali} ottengo:
	\begin{equation}\label{ga:QuasiStirling2}\begin{split}
		\Gamma(x+1)&=\frac{\Gamma\left(\partint{x+1}+\{x\}\right)}{\Gamma\left(\partint{x+1}\right)}\cdot\Gamma\left(\partint{x+1}\right)\sim
		\left(\partint{x+1}\right)^{\{x\}}\sqrt{2\pi\partint{x}}\left(\frac{\partint{x}}e\right)^{\partint{x}} \\
		&\sim x^{\{x\}}\sqrt{2\pi x}\left(\frac{\partint{x}}e\right)^{\partint{x}}
		=\sqrt{2\pi x}\left(\frac xe\right)^x \cdot\left[\left(\frac{\partint{x}}x\right)^{\partint{x}} e^{\{x\}}\right]
	\end{split}\end{equation}

	Allo stesso tempo espandendo il logaritmo al prim'ordine ho che vale:
	\begin{equation}\label{ga:ContoLogaritmo}
		\partint{x}\log\left(1+\frac{\{x\}}{\partint{x}}\right)=\partint{x}\left[\frac{\{x\}}{\partint{x}}+\smallO\left(\frac{\{x\}}{\partint{x}}\right)\right]=\{x\}+\smallO(1) \Longrightarrow \left(1+\frac{\{x\}}{\partint{x}}\right)^{\partint{x}}\sim e^{\{x\}}
	\end{equation}
	
	Sostituendo \cref{ga:ContoLogaritmo} in \cref{ga:QuasiStirling2} ottengo proprio l'approssimazione di Stirling.
\end{proof}





\end{document}

\makeindex